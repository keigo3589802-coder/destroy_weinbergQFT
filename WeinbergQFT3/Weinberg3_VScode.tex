\documentclass[dvipdfmx]{jsarticle}
\let\headfont=\gtfamily
\usepackage[dvips]{graphicx}
\usepackage{amsmath}
\usepackage{mathrsfs} % 花文字\mathscr{M}, 筆記体\mathcal{M}, 黒板文字\mathbb{M},ドイツ文字\mathfrak{M}
\usepackage{bm} %太文字
\usepackage{amssymb}
\usepackage{latexsym}
\usepackage{braket}
\usepackage{tikz}
\usepackage{tikz-feynhand}
\usepackage{ulem}
\usepackage{bigdelim}
\usepackage{multirow}
\usepackage{tcolorbox}
\usepackage{here}
\usepackage{tensor}
\tcbuselibrary{theorems,skins}
\usetikzlibrary{decorations}
\usepackage{color}

\usetikzlibrary{intersections, calc, arrows.meta}
 \usetikzlibrary{patterns}

\newfont{\bg}{cmr9 scaled\magstep4}
\newcommand{\bigzerol}{\smash{\lower1.0ex\hbox{\bg 0}}}
\newcommand{\bigzerou}{%
   \smash{\hbox{\bg 0}}}
\newcommand{\mcO}{\mathcal{O}}
\newcommand{\VAC}{\mathrm{VAC}}
\newcommand{\Slash}[1]{{\ooalign{\hfil/\hfil\crcr$#1$}}} %ファインマンのスラッシュ記号
\renewcommand{\mc}{\mathcal}
\newcommand{\mr}[1]{\mathrm{#1}}

% \textrm{Roman デフォルト}
% \textgt{Gothic 和文ゴシック体}*専門用語に
% \textbf{Boldface 太字}*専門用語(英語)に
% \textit{Italic 斜体}
% \textsl{Slanted ローマンを傾けただけ}
% \textsf{Sans Serif サンセリフ体}
% \texttt{Typewriter タイプライタ体、等幅}
% \textsc{Small Caps 小文字が大文字に}

\setlength{\textwidth}{\fullwidth}
\setlength{\textheight}{44\baselineskip}
\addtolength{\textheight}{\topskip}
\setlength{\voffset}{-0.6in}

\allowdisplaybreaks[4]

\makeatletter
  \renewcommand{\theequation}
  {\arabic{section}.\arabic{equation}}
  \@addtoreset{equation}{section}
 \makeatother

\title{\vspace{-1cm}\Huge{WeinbergQFT3}}
\author{Laplacyan}
\date{}
\begin{document}



\maketitle
\setcounter{part}{14}
\part{非可換ゲージ理論}
\setcounter{section}{15}
\subsection{ゲージ不変性}
理論のラグランジアンが,物質場$\psi_\ell(x)$の以下の微小変換の組で不変であるとする.
\begin{align*}
\delta \psi_\ell(x)=i\epsilon^\alpha(x)\tensor{(t_\alpha)}{_\ell^m}\psi_m(x)
\end{align*}
ここで$t_\alpha$は独立な定数行列の組であり,また$\epsilon^\alpha(x)$は実の微小パラメータであり(電磁理論のゲージ変換のように)時空の位置$x^\mu$に依存できるとする.さらに,これらの対称性変換はあるリー群の微小部分であると仮定する.2.2節で示したように,これは$t_\alpha$が以下の交換関係を満たすことを意味する.
\begin{align*}
[t_\alpha,t_\beta]=i\tensor{C}{^\gamma_\alpha_\beta}t_\gamma
\end{align*}
ここで$\tensor{C}{^\gamma_{\alpha\beta}}$は実の定数の組で,群の構造定数である.交換子の反対称性から,構造定数もまた以下のように反対称であることがわかる.
\begin{align*}
&[t_\alpha,t_\beta]=i\tensor{C}{^\gamma_{\alpha\beta }}t_\gamma=-[t_\beta ,t_\alpha]=-i\tensor{C}{^\gamma_{\beta\alpha}}t_\gamma \\
\therefore \quad &\tensor{C}{^\gamma_{\alpha \beta}}=-\tensor{C}{^\gamma_{\beta\alpha}}
\end{align*}
また,ヤコビ恒等式から
\begin{align*}
0=&\Bigl[[t_\alpha,t_\beta],t_\gamma \Bigr]+\Bigl[ [t_\gamma,t_\alpha] ,t_\beta \Bigr]+\Bigl[[t_\beta, t_\gamma ],t_\alpha \Bigr] \\
=&iC^\delta_{\,\,\, \alpha\beta}[t_\delta,t_\gamma]+iC^\delta_{\,\,\, \gamma\alpha}[t_\delta,t_\beta]+iC^\delta_{\,\,\, \beta\gamma}[t_\delta,t_\alpha] \\
=&-C^\delta_{\,\,\, \alpha\beta}C^\epsilon_{\,\,\, \delta \gamma}-C^\delta_{\,\,\, \gamma\alpha}C^\epsilon_{\,\,\, \delta \beta}-C^\delta_{\,\,\, \beta\gamma}C^\epsilon_{\,\,\, \delta \alpha} \\
\therefore \quad & 0=C^\delta_{\,\,\, \alpha\beta}C^\epsilon_{\,\,\, \delta \gamma}+C^\delta_{\,\,\, \gamma\alpha}C^\epsilon_{\,\,\, \delta \beta}+C^\delta_{\,\,\, \beta\gamma}C^\epsilon_{\,\,\, \delta \alpha}
\end{align*}
も満たすことがわかる.(15.1.3)と(15.1.5)を満たす定数$C^\gamma_{\,\,\, \alpha\beta}$の組から,行列$t^A_{\,\,\,\,\alpha}$を少なくとも一組,次のように定義できる.
\begin{align*}
(t^A_{\,\,\,\, \alpha})^\beta_{\,\,\,\gamma}\equiv -i C^\beta_{\,\,\, \gamma\alpha}
\end{align*}
実際,この行列は(15.1.5)より構造定数$C^\gamma_{\,\,\, \alpha\beta}$とする交換関係(15.1.2)を満たす.
\begin{align*}
\Bigl([t^A_{\,\,\,\, \alpha},t^A_{\,\,\,\, \beta}]\Bigr)^\delta_{\,\,\, \epsilon}=&(t^A_{\,\,\,\,\alpha})^\delta_{\,\,\, \gamma}(
t^A_{\,\,\,\,\beta})^\gamma_{\,\,\, \epsilon}-(t^A_{\,\,\,\,\beta})^\delta_{\,\,\, \gamma}(
t^A_{\,\,\,\,\alpha})^\gamma_{\,\,\, \epsilon} \\
=&-C^\delta_{\,\,\, \gamma\alpha}C^\gamma_{\,\,\, \epsilon\beta}+C^\delta_{\,\,\, \gamma\beta}C^\gamma_{\,\,\, \epsilon\alpha} \\
=&-C^\gamma_{\,\,\, \epsilon\beta}C^\delta_{\,\,\, \gamma\alpha}-C^\gamma_{\,\,\, \alpha\epsilon}C^\delta_{\,\,\, \gamma\beta} \\
=&C^\gamma_{\,\,\, \beta\alpha}C^\delta_{\gamma\epsilon}=C^\gamma_{\,\,\, \alpha\beta}C^\delta_{\,\,\,\epsilon\gamma} \\
=&iC^\gamma_{\,\,\,\alpha\beta}(t^A_{\,\,\,\, \gamma})^\delta_{\,\,\, \epsilon} \\
\therefore \quad [t^A_{\,\,\,\, \alpha},t^A_{\,\,\,\, \beta}]=&iC^\gamma_{\,\,\, \alpha\beta}t^A_{\,\,\, \gamma}
\end{align*}
これは構造定数$C^\gamma_{\,\,\, \alpha\beta}$をもつリー代数の随伴表現と呼ばれる.\par
例えば,元々のヤンミルズ理論では,物質場は陽子場$\psi_p$と中性子場$\psi_n$からなる二重項
\begin{align*}
\psi=\left(
\begin{array}{cc}
\psi_p \\
\psi_n
\end{array}
\right)
\end{align*}
であり,$\alpha=1,2,3$の$t_\alpha$はアイソスピン行列
\begin{align*}
t_1=\frac{1}{2}\left(
\begin{array}{cc}
0 & 1 \\
1 & 0
\end{array}
\right) ,\quad t_2=\frac{1}{2}\left(
\begin{array}{cc}
0 & -i \\
i & 0
\end{array}
\right),\quad t_3=\frac{1}{2}\left(
\begin{array}{cc}
1 & 0 \\
0 & -1
\end{array}
\right)
\end{align*}
となっていた.これらは$C^\gamma_{\,\,\, \alpha\beta}=\epsilon_{\gamma\alpha\beta}$を構造定数とする(15.1.2)の交換関係を満たす.
\begin{align*}
[t_\alpha, t_\beta]=i\epsilon_{\gamma\alpha\beta}t_\gamma
\end{align*}
ここで$\epsilon_{\gamma\alpha\beta}$はエディントンのイプシロンだ.これは3次元回転群$SO(3)$のリー代数(2.4.18)と同じであり,行列$t_\alpha$はこのリー代数の$SU(2)$スピン$1/2$表現を与えている.随伴表現の行列(15.1.6)はここでは
\begin{align*}
t^A_1=\left[
\begin{array}{ccc}
0 & 0 & 0 \\
0 & 0 & -i \\
0 & i & 0
\end{array}
\right] , \quad t^A_2=\left[
\begin{array}{ccc}
0 & 0 & i \\
0 & 0 & 0 \\
-i & 0 & 0
\end{array}
\right], \quad t^A_3=\left[
\begin{array}{ccc}
0 & -i & 0 \\
i & 0 & 0 \\
0 & 0 & 0
\end{array}
\right]
\end{align*}
となる.これは$SO(3)$のリー代数のスピン1表現を与えている.\par
さて,ラグランジアンを(15.1.1)の変換のもとで不変にするには,どのように構成すれば良いか考えよう.もし場に微分が作用していなければ,その答えは簡単だ.$\epsilon^\alpha$を定数とした変換(15.1.1)のもとで不変な物質場の関数は$\epsilon^\alpha$を時空座標の任意の実関数$\epsilon^\alpha(x)$としても不変だ.これはラグランジアンが場の微分を含む場合には当てはまらない.なぜなら,$\epsilon^\alpha(x)$が位置に依存する関数として,物質場の微分はその場自身のようには変換せず,実際(15.1.1)の両辺を微分すると
\begin{align*}
\delta\Bigl(\partial_\mu \psi_\ell(x)\Bigr)=i\epsilon^\alpha(x)(t_\alpha)_\ell^{\,\,\, m}\Bigl(\partial_\mu \psi_m(x)\Bigr)+i\Bigl(\partial_\mu \epsilon^\alpha (x)\Bigr)(t_\alpha)_\ell^{\,\,\, m}\psi_\ell(x)
\end{align*}
が得られる.ラグランジアンを不変にするためには,場$\tensor{A}{^\alpha_\mu}$を導入して,その場の変換則が$\partial_\mu \epsilon^\alpha$を含み(15.1.8)の第二項目を丁度よく打ち消せるようにしなければならない.加えて,この場は$\alpha$の添え字を持つので,これもまた(15.1.1)のような行列の変換則で$t_\alpha$を随伴表現の行列(15.1.6)に置き換えたものに従うと期待できる.したがって,とりあえずこれらの新しい「ゲージ」場を,以下の変換則に従うような場として定義する.
\begin{align*}
\delta \tensor{A}{^\beta_\mu}=\partial_\mu \epsilon^\beta+i\epsilon^\alpha (\tensor{t}{^A_\alpha})\indices{^\beta_\gamma} \tensor{A}{^\gamma_\mu}
\end{align*}
あるいは
\begin{align*}
\delta \tensor{A}{^\beta_\mu}=\partial_\mu \epsilon^\beta+\epsilon^\alpha \tensor{C}{^\beta_{\gamma\alpha}} \tensor{A}{^\gamma_\mu}
\end{align*}
となる.これにより,次の「共変微分」が構成できる.
\begin{align*}
\Bigl(D_\mu \psi(x)\Bigr)_\ell=\partial_\mu \psi_\ell(x)-i\tensor{A}{^\beta_\mu}(x)\tensor{(t_\beta)}{_\ell^m}\psi_m(x)
\end{align*}
計画通り,(15.1.10)の第二項目のなかの$\tensor{A}{^\beta_\mu}$の変換の$\partial_\mu \epsilon^\beta$の項は,第一項目の変換から生じる$\partial_\mu \epsilon^\beta$の項を打ち消す.その結果,以下の項が残る.
\begin{align*}
\delta\Bigl(D_\mu \psi(x)\Bigr)_\ell=&\delta\Bigl(\partial_\mu \psi_\ell(x)\Bigr)-i\delta \left(\tensor{A}{^\beta_\mu}(x)\right)\tensor{(t_\beta)}{_\ell^m}\psi_m(x)-i\tensor{A}{^\beta_\mu}(x)\tensor{(t_\beta)}{_\ell^m}\delta\psi_m(x) \\
=&i\epsilon^\alpha(x)(t_\alpha)\indices{_\ell^m}\Bigl(\partial_\mu \psi_m(x)\Bigr)+i\Bigl(\partial_\mu \epsilon^\alpha (x)\Bigr)(t_\alpha)_\ell^{\,\,\, m}\psi_\ell(x) \\
&-i\Bigl(\partial_\mu \epsilon^\beta(x)\Bigr)\tensor{(t_\beta)}{_\ell^m}\psi_m(x)+\epsilon^\alpha (\tensor{t}{^A_\alpha})\indices{^\beta_\gamma} \tensor{A}{^\gamma_\mu}\tensor{(t_\beta)}{_\ell^m}\psi_m(x) \\
&+\tensor{A}{^\beta_\mu}(x)\tensor{(t_\beta)}{_\ell^m}\epsilon^\alpha(x)\tensor{(t_\alpha)}{_\ell^m}\psi_m(x) \\
=&i\epsilon^\alpha(x)(t_\alpha)\indices{_\ell^m}\Bigl(\partial_\mu \psi_m(x)\Bigr)-i\tensor{C}{^\beta_{\gamma\alpha}}\epsilon^\alpha \tensor{A}{^\gamma_\mu}\tensor{(t_\beta)}{_\ell^m}\psi_m(x) \\
&+\tensor{A}{^\beta_\mu}(x)\tensor{(t_\beta)}{_\ell^m}\epsilon^\alpha(x)\tensor{(t_\alpha)}{_m^n}\psi_n(x)
\end{align*}
(15.1.2)を用いると
\begin{align*}
\delta\Bigl(D_\mu \psi(x)\Bigr)_\ell=&i\epsilon^\alpha(x)(t_\alpha)\indices{_\ell^m}\Bigl(\partial_\mu \psi_m(x)\Bigr)-\epsilon^\alpha(x) \tensor{A}{^\gamma_\mu}(x)[t_\gamma,t_\alpha]\indices{_\ell^m}\psi_m(x) \\
&+\epsilon^\alpha(x)\tensor{A}{^\gamma_\mu}(x)\tensor{(t_\gamma)}{_\ell^m}\tensor{(t_\alpha)}{_m^n}\psi_n(x) \\
=&i\epsilon^\alpha(x)(t_\alpha)\indices{_\ell^m}\Bigl(\partial_\mu \psi_m(x)\Bigr)+\epsilon^\alpha(x)\tensor{A}{^\gamma_\mu}(x)\tensor{(t_\alpha)}{_\ell^m}\tensor{(t_\gamma)}{_m^n}\psi_n(x) \\
=&i\epsilon^\alpha(x)(t_\alpha)\indices{_\ell^m}\Bigl(\partial_\mu \psi_m(x)-i\tensor{A}{^\beta_\mu}(t_\beta)\indices{_m^n}\psi_n(x)\Bigr) \\
=&i\epsilon^\alpha(x)(t_\alpha)\indices{_\ell^m}\Bigl(D_\mu \psi(x) \Bigr)_m
\end{align*}
となるので,$D_\mu\psi$はちょうど$\psi$自身のように変換する.\par
ゲージ場の微分も心配する必要がある.$\partial_\nu \tensor{A}{^\beta_\mu}$の変換は$\partial_\nu \partial_\mu\epsilon^\beta$の項が生じるため,これを打ち消すために電磁理論の強度テンソルのように$\mu$と$\nu$について反対称化する必要がある.しかし$\partial_\nu \tensor{A}{^\beta_\mu}-\partial_\mu \tensor{A}{^\beta_\nu}$の変換には,(15.1.9)の第二項から生じる$\epsilon(x)$の一階微分に比例する項がある.そこで,「共変な回転」$\tensor{F}{^\gamma_{\nu\mu}}$を,その変換で$\epsilon(x)$の微分が全て打ち消し合うようにするには,物質場$\psi$にはたらく二つの共変微分の交換子を考えるのが最も簡単だ.
\begin{align*}
(D_\mu D_\nu \psi)_\ell=&D_\mu(\partial_\nu \psi_\ell(x)-i\tensor{A}{^\beta_\nu}(x)\tensor{(t_\beta)}{_\ell^m}\psi_m(x)) \\
=&\partial_\mu \partial_\nu \psi_\ell(x)-i(\partial_\mu\tensor{A}{^\beta_\nu})\tensor{(t_\beta)}{_\ell^m}\psi_m-i\tensor{A}{^\beta_\nu}(x)\tensor{(t_\beta)}{_\ell^m}(\partial_\mu\psi_m) \\
&-i\tensor{A}{^\beta_\mu}(x)\tensor{(t_\beta)}{_\ell^m}(\partial_\nu \psi_m)-\tensor{A}{^\gamma_\mu}(x)\tensor{(t_\gamma)}{_\ell^m}\tensor{A}{^\beta_\nu}(x)\tensor{(t_\beta)}{_m^n}\psi_n(x)) \\
=&\partial_\mu \partial_\nu \psi_\ell(x)-i\tensor{A}{^\beta_\nu}(x)\tensor{(t_\beta)}{_\ell^m}(\partial_\mu\psi_m)-i\tensor{A}{^\beta_\mu}(x)\tensor{(t_\beta)}{_\ell^m}(\partial_\nu \psi_m) \\
&-i(\partial_\mu\tensor{A}{^\beta_\nu})\tensor{(t_\beta)}{_\ell^m}\psi_m-\tensor{A}{^\gamma_\mu}\tensor{A}{^\beta_\nu}\tensor{(t_\gamma t_\beta)}{_\ell^n}\psi_n(x))
\end{align*}
より
\begin{align*}
\Bigl([D_\mu, D_\nu ]\psi \Bigr)_\ell=&-i(\partial_\mu\tensor{A}{^\beta_\nu})\tensor{(t_\beta)}{_\ell^m}\psi_m-\tensor{A}{^\gamma_\mu}\tensor{A}{^\beta_\nu}\tensor{(t_\gamma t_\beta)}{_\ell^n}\psi_n(x)) \\
&+i(\partial_\nu\tensor{A}{^\beta_\mu})\tensor{(t_\beta)}{_\ell^m}\psi_m-\tensor{A}{^\gamma_\nu}\tensor{A}{^\beta_\mu}\tensor{(t_\gamma t_\beta)}{_\ell^n}\psi_n(x)) \\
=&-i\Bigl(\partial_\mu\tensor{A}{^\alpha_\nu}-\partial_\nu\tensor{A}{^\alpha_\mu}+\tensor{C}{^\alpha_{\gamma\beta}}\tensor{A}{^\gamma_\mu}\tensor{A}{^\beta_\nu}\Bigr)\tensor{(t_\alpha)}{_\ell^m}\psi_m \\
=&-i(t_\alpha)\indices{_\ell^m}\tensor{F}{^\alpha_{\mu\nu}}\psi_m
\end{align*}
これにより
\begin{align*}
\tensor{F}{^\alpha_{\mu\nu}}:= \partial_\mu\tensor{A}{^\alpha_\nu}-\partial_\nu\tensor{A}{^\alpha_\mu}+\tensor{C}{^\alpha_{\beta\gamma}}\tensor{A}{^\beta_\mu}\tensor{A}{^\gamma_\nu}
\end{align*}
となる.(15.1.12)から,$\tensor{F}{^\alpha_{\mu\nu}}$が随伴表現に属する物質場のように変換しなければならないことは自明だ.
\begin{align*}
\delta(D_\mu D_\nu \psi)_\ell=&i\epsilon^\alpha(x)(t_\alpha)\indices{_\ell^m}(D_\mu D_\nu \psi)_m \\
\delta \Bigl([D_\mu, D_\nu ]\psi \Bigr)_\ell=&i\epsilon^\alpha(x)(t_\alpha)\indices{_\ell^m}\Bigl([D_\mu, D_\nu ]\psi \Bigr)_m=\epsilon^\alpha(x)(t_\alpha t_\beta)\indices{_\ell^n}\tensor{F}{^\beta_{\mu\nu}}\psi_n \\
=&-i(t_\alpha)\indices{_\ell^m}\delta \tensor{F}{^\alpha_{\mu\nu}}\psi_m-i(t_\alpha)\indices{_\ell^m}\tensor{F}{^\alpha_{\mu\nu}}\delta \psi_m \\
=&-i(t_\alpha)\indices{_\ell^m}\delta \tensor{F}{^\alpha_{\mu\nu}}\psi_m+(t_\alpha t_\beta)\epsilon^\beta(x)\indices{_\ell^m}\tensor{F}{^\alpha_{\mu\nu}}\psi_m \\
\delta \tensor{F}{^\alpha_{\mu\nu}}(t_\alpha)\indices{_\ell^m}\psi_m=&i\epsilon^\alpha(x)([t_\alpha,t_\beta])\indices{_\ell^m}\tensor{F}{^\beta_{\mu\nu}}\psi_m \\
=&-\epsilon^\alpha(x)\tensor{C}{^\gamma_{\alpha\beta}}(t_\gamma)\indices{_\ell^m}\tensor{F}{^\beta_{\mu\nu}}\psi_m
\end{align*}
よって
\begin{align*}
\delta\tensor{F}{^\beta_{\mu\nu}}=\epsilon^\alpha(x)\tensor{C}{^\gamma_{\beta\alpha}}\tensor{F}{^\beta_{\mu\nu}}=i\epsilon^\alpha(\tensor{t}{^A_\alpha})\indices{^\gamma_\beta}\tensor{F}{^\beta_{\mu\nu}}
\end{align*}
となる.\par
目的によっては,これらの微小ゲージ変換を,パラメータが有限であるような変換に格上げして用いる.これは一般的な物質場$\psi_\ell(x)$に
\begin{align*}
\psi_\ell(x)\to \psi_{\ell\Lambda}(x)=\Bigl[\exp\Bigl(it_\alpha \Lambda^\alpha(x)\Bigr)\Bigr]\indices{_\ell^m}\psi_m(x)=U(\Lambda(x))\indices{_\ell^m}\psi_m(x)
\end{align*}
のように行列として働く.行列表記にすれば
\begin{align*}
\psi(x) \to \psi_{\Lambda}(x)=\exp\Bigl(it_\alpha \Lambda^\alpha(x)\Bigr) \psi(x) =U(\Lambda(x)) \psi(x)
\end{align*}
となる.共変微分も同様に
\begin{align*}
D_\mu \psi \to& U(\Lambda)D_\mu \psi \\
(\partial_\mu-it_\alpha \tensor{A}{^\alpha_\mu})\psi \to&(\partial_\mu -it_\alpha \tensor{A}{^\alpha_{\mu\Lambda}})\psi_\Lambda=U(\Lambda)(\partial_\mu-it_\alpha \tensor{A}{^\alpha_\mu})\psi
\end{align*}
と変換するのが望ましいので,$\tensor{A}{^\alpha_\mu}$の変換則$\tensor{A}{^\alpha_\mu}\to \tensor{A}{^\alpha_{\mu\Lambda}}$は
\begin{align*}
t_\alpha \tensor{A}{^\alpha_{\mu\Lambda}}=&U(\Lambda)t_\alpha \tensor{A}{^\alpha_\mu} U^{-1}(\Lambda)-i[\partial_\mu U(\Lambda)]U^{-1}(\Lambda) \\
=&U(\Lambda)t_\alpha \tensor{A}{^\alpha_\mu} U^{-1}(\Lambda)+iU(\Lambda)\partial_\mu U^{-1}(\Lambda) \\
=&U(\Lambda)[t_\alpha \tensor{A}{^\alpha_\mu} +i\partial_\mu]U^{-1}(\Lambda)
\end{align*}
となることを要請する.ここで
\begin{align*}
0=\partial_\mu 1=&\partial_\mu [U(\Lambda) U^{-1}(\Lambda)] \\
=&[\partial_\mu U(\Lambda)]U^{-1}(\Lambda)+U(\Lambda)\partial_\mu U^{-1}(\Lambda) \\
\therefore \quad & U(\Lambda)\partial_\mu U^{-1}(\Lambda)=-[\partial_\mu U(\Lambda)]U^{-1}(\Lambda)
\end{align*}
を用いた.場の強度の変換は
\begin{align*}
[D_\mu,D_\nu]\psi \to& [D_\mu,D_\nu] \psi_{\Lambda} \\
=&U(\Lambda) [D_\mu,D_\nu] \psi \\
=&-iU(\Lambda) t_\alpha F^{\alpha}_{\mu\nu} \psi \\
-it_\alpha F^{\alpha}_{\mu\nu} \psi \to & -it_\alpha F^{\alpha}_{\Lambda\mu\nu} \psi_\Lambda \\
=&-it_\alpha F^{\alpha}_{\Lambda\mu\nu} U(\Lambda) \psi \\
\therefore \quad t_{\alpha} F^\alpha_{\mu\nu} \to& t_\alpha F^{\alpha}_{\Lambda\mu\nu}=U(\Lambda) t_\alpha F^{\alpha}_{\mu\nu} U^{-1}(\Lambda)
\end{align*}
となる\footnote{これらの表記を見れば察するかもしれないが,ゲージ場や場の強度はしばしば$t_\alpha \tensor{A}{^\alpha_{\mu}}$というようにリー代数の行列とまとめて扱った方が便利である.実際,ゲージ理論の数理的な側面を取り扱う際にはゲージ場を$A^\mu:=t_\alpha \tensor{A}{^\alpha_{\mu}} \in \mathfrak{g}$としてリー代数$\mathfrak{g}$に値を持つ場として用いることが多い.一般の多様体$M$上の接続として取り扱うなどのさらに一般的な場合,近傍$U_i$上でリー代数に値を持つ1形式の場として$\mc{A}:=\tensor{A}{^\alpha_{\mu}}t_\alpha \otimes dx^\mu \in \mathfrak{g} \otimes \Omega^1(U_i)$と書いたりもする.}.\par
2章から6章で行った演算子形式の言葉で説明して,繋がりを明確にしよう.これはまさに対称性変換群$G$の元$T(\epsilon)$によって定まるユニタリー演算子
\begin{align*}
\widehat{U}\Bigl(T(\epsilon(x))\Bigr)=\exp\Bigl(i\epsilon^\alpha(x) \widehat{T}_\alpha\Bigr)
\end{align*}
により,場の演算子$\widehat{\psi}_\ell(x)$が
\begin{align*}
\widehat{U}\Bigl(T(\epsilon(x))\Bigr)^{-1} \widehat{\psi}_{\ell}(x) \widehat{U}\Bigl(T(\epsilon(x))\Bigr) =& \tensor{[U(\epsilon(x))]}{_\ell^m}\widehat{\psi}_m(x) \\
\exp\Bigl(-i\epsilon^\alpha(x) \widehat{T}_\alpha\Bigr)\widehat{\psi}(x) \exp\Bigl(i\epsilon^\alpha(x) \widehat{T}_\alpha\Bigr)=& \exp\Bigl(i\epsilon^\alpha(x) t_\alpha\Bigr)\widehat{\psi}(x)
\end{align*}
と変換されることを表している.微小変換で考えることにより,エルミート演算子$\widehat{T}_\alpha$による変換が具体的に見える.
\begin{align*}
\widehat{U}^{-1}\Bigl(T(\epsilon(x))\Bigr)\widehat{\psi}_\ell(x) \widehat{U}\Bigl(T(\epsilon(x))\Bigr)=&(1-i\epsilon^\alpha(x)\widehat{T}_\alpha) \widehat{\psi}_\ell(x) (1+i\epsilon^\beta(x) \widehat{T}_\beta) \\
=&\widehat{\psi}_\ell(x)-i\epsilon^\alpha(x)[\widehat{T}_\alpha,\widehat{\psi}_\ell(x)] \\
=\tensor{(1+i\epsilon^\alpha(x)t_\alpha)}{_\ell^m}\widehat{\psi}_m(x)=& \widehat{\psi}_\ell(x)+i\epsilon^\alpha(x) \tensor{(t_\alpha)}{_\ell^m}\widehat{\psi}_m(x) \\
\therefore \quad \Bigl[\widehat{T}_\alpha,\widehat{\psi}_\ell(x)\Bigr]=&-\tensor{(t_\alpha)}{_\ell^m}\widehat{\psi}_m(x)
\end{align*}
表現論の言葉で,これは$\widehat{\psi}$がテンソル演算子であるという.ゲージ場の変換も同様に
\begin{align*}
\widehat{U}\Bigl(T(\epsilon(x))\Bigr)^{-1}t_\alpha \tensor{\widehat{A}}{^{\alpha}_{\mu\nu}}(x) \widehat{U}\Bigl(T(\epsilon(x))\Bigr)=&U(\epsilon(x))[t_\alpha \tensor{\widehat{A}}{^\alpha_{\mu\nu}}+i\partial_\mu] U(\epsilon(x))^{-1} \\
\exp\Bigl(-i\epsilon^\alpha(x) \widehat{T}_\alpha\Bigr)t_\alpha \tensor{\widehat{A}}{^{\alpha}_{\mu\nu}}(x) \exp\Bigl(i\epsilon^\alpha(x) \widehat{T}_\alpha\Bigr)=& \exp\Bigl(i\epsilon^\alpha(x) t_\alpha\Bigr)\left[\tensor{\widehat{A}}{^{\alpha}_{\mu\nu}}(x)+i\partial_\mu \right]\exp\Bigl(-i\epsilon^\alpha(x) t_\alpha\Bigr)
\end{align*}
と書ける.


\vskip\baselineskip

(15.1.17)から,$\Lambda^\beta(x)$をうまく選べば,$\tensor{A}{^\alpha_{\mu\Lambda}}(x)$を任意の「一つの点」,たとえば$x=z$でゼロになるようにできることがわかる.(これには,$x=z$で$\Lambda^\alpha(z)=0$かつ$ \partial\Lambda^\alpha(x)/\partial x^\mu |_{x=z}=-\tensor{A}{^\alpha_\mu}(z)$となるように選べばよい.)また,常に$\Lambda^\beta(x)$を選んで,$\tensor{A}{^\alpha_{\mu\Lambda}}(x)$の任意の一つの時空成分が,少なくともある与えられた点の「有限な近傍」で,全ての$\alpha$についてゼロとなるようにできる.たとえば$\tensor{A}{^\alpha_{3\Lambda}}(x)$をゼロにするには,パラメータ$\Lambda^\beta(x)$に対する以下の「常」微分方程式の組を解かなければならない.
\begin{align*}
\partial_3 \exp(it_\beta \Lambda^\beta)=-i\exp(it_\beta \Lambda^\beta)t_\alpha \tensor{A}{^\alpha_3}
\end{align*}
これは少なくともある与えられた点の周りの有限な領域で必ず解を持つ.\par
しかし,一般には$\tensor{A}{^\alpha_{\alpha\Lambda}}$の4つの成分が「有限の領域」でゼロになるように$\Lambda^\alpha$を選ぶことはできない.この目的のためには,以下の「偏」微分方程式の組を満たさなければならない.
\begin{align*}
\partial_\mu \exp(it_\beta \Lambda^\beta)=-i\exp(it_\beta \Lambda^\beta)t_\alpha \tensor{A}{^\alpha_\mu}
\end{align*}
これはある種の可積分条件が満たされないと解を持たない.特に,もし$\tensor{A}{^\alpha_{\mu\Lambda}}$がある領域の全ての点でゼロとなると,$\tensor{F}{^\alpha_{\mu\nu\Lambda}}$もゼロとなる.しかし,場の強度は斉次的に変換するから,$\tensor{F}{^\alpha_{\mu\nu}}$がゼロとなるときにのみ$\tensor{F}{^\alpha_{\mu\nu\Lambda}}$もゼロとなる.したがって$\tensor{F}{^\alpha_{\mu\nu}}$をゼロにするようなゲージ場$\tensor{A}{^\alpha_\mu}$のみ,$\tensor{A}{^\alpha_{\mu\Lambda}}$がゼロになるようなゲージ変換が存在する.このようなゲージ場は「純ゲージ場」と呼ばれる.$\tensor{F}{^\alpha_{\mu\nu}}$がどこでもゼロとなるという条件は,$\tensor{A}{^\alpha_{\mu}}$が任意の単連結領域で純ゲージ場で表されるための必要十分条件である.

\vskip\baselineskip

ここではゲージ変換で単純に変換する量を構成したが,これは一般相対論での一般座標変換のもとで共変的に変換する量を構成するのと深い類似性がある.ゲージ場を使って物質場の共変微分$D_\mu \psi_\ell$を作り,それが物質場自身と同じゲージ変換性をもつようにしたのと同様に,アフィン接続$\tensor{\Gamma}{^\mu_{\nu\lambda}}(x)$を使い,テンソル$\tensor{T}{^{\rho\sigma\cdots}_{\kappa\lambda\cdots }}$の共変微分を
\begin{align*}
\tensor{(\nabla_\nu T)}{^{\rho\cdots}_{\kappa\cdots}}=\partial_\nu \tensor{T}{^{\rho\cdots}_{\kappa\cdots}}+\tensor{\Gamma}{^\rho_{\nu\lambda}}\tensor{T}{^{\lambda\cdots}_{\kappa\cdots}}+\cdots -\tensor{\Gamma}{^\mu_{\nu\kappa}}\tensor{T}{^{\rho\cdots}_{\mu\cdots}}-\cdots
\end{align*}
と構成すると,これ自身もテンソルとなる.ここでアフィン接続$\Gamma$はクリストッフェル記号
\begin{align}
\tensor{\Gamma}{^\alpha_{\beta\gamma}}=\frac{1}{2} g^{\alpha \rho}\left( \frac{\partial g_{\rho \beta}}{\partial x^\gamma}+ \frac{\partial g_{\rho \gamma}}{\partial x^\beta}- \frac{\partial g_{\beta\gamma}}{\partial x^\rho} \right)
\end{align}
である\footnote{厳密にいえばレヴィ・チビタ接続である.多様体$M$上に定まるアフィン接続は無数にあり,そこに対称接続だったり内積との整合性などの条件を課すと接続の形が決まる.}.また,ゲージ場の微分からゲージ場の強度$\tensor{F}{^\alpha_{\mu\nu}}$を作り,ゲージ群の随伴表現に属する物質場と同じゲージ変換性をもつようにできた.これに対応して,アフィン接続の微分から
\begin{align*}
\tensor{R}{^\lambda_{\mu\nu\kappa}}=\partial_\kappa \tensor{\Gamma}{^\lambda_{\mu\nu}}-\partial_\nu \tensor{\Gamma}{^\lambda_{\mu\kappa}}+\tensor{\Gamma}{^\eta_{\mu\nu}}\tensor{\Gamma}{^\lambda_{\kappa\eta}}-\tensor{\Gamma}{^\eta_{\mu\kappa}}\tensor{\Gamma}{^\lambda_{\nu\eta}}
\end{align*}
という量を構成すると,これはテンソルのように変換し,リーマンクリストッフェル曲率テンソルと呼ばれる.二つのゲージ共変微分$D_\mu,D_\nu$の交換子は,場の強度$\tensor{F}{^\alpha_{\mu\nu}}$を使って表せる.これと同様に,$x^\nu,x^\kappa$に関する二つの共変微分の交換子は曲率を使って,以下のように表せる.
\begin{align*}
\tensor{([\nabla_\kappa,\nabla_\nu]T)}{^{\lambda\cdots}_{\mu\cdots}}=\tensor{R}{^\lambda_{\sigma\nu\kappa}}\tensor{T}{^{\sigma\cdots}_{\mu\cdots}}+\cdots -\tensor{R}{^\sigma_{\mu\nu\kappa}}\tensor{T}{^{\lambda\cdots}_{\sigma\cdots}}-\cdots
\end{align*}
有限の単連結領域でゲージ場がゼロとなるようなゲージ変換が存在する必要十分条件は,場の強度テンソルがゼロとなることだった.同様に,有限の単連結領域でアフィン接続がゼロとなる座標系が存在する必要十分条件は,リーマンクリストッフェル曲率テンソルがゼロとなることだ.以下でこれを証明しよう.

\vskip\baselineskip

証明すべきは次の命題である.すなわち「ある領域$\Omega$がflat,すなわちミンコフスキー計量であるための必要十分条件は,領域$\Omega$上いたるところでリーマン曲率テンソル$\tensor{R}{^\mu_\nu_\rho_\sigma}$がゼロとなることである.」である.\par
必要であることはすぐわかる.なぜならば$\Omega$はミンコフスキー的であるから,$\Omega$の中のすべての点で計量テンソルが$\eta_{\mu\nu}$となるような座標系を設ければよい.このとき$\Gamma$は定義より$\Omega$のいたるところでゼロとなり,したがって$\tensor{R}{^\lambda_{\rho\mu\nu}}$もゼロである.\par
十分であることの証明.いま逆に$\tensor{R}{^\lambda_{\rho\mu\nu}}$が$\Omega$の中のすべての点でゼロとする.まず,計量テンソル$g_{\mu\nu}$が対称行列であることから,$\Omega$内の任意の点$P$において
\begin{align*}
\tensor{d}{^\alpha_\mu} \tensor{d}{^\beta_\nu}g^{\mu\nu}(P)=\eta^{\alpha\beta}
\end{align*}
となる直交行列$d$が存在する.(直交行列$\tensor{O}{^\alpha_\mu}$で$OgO^T$が対角行列
\begin{align*}
\tensor{O}{^\alpha_\mu}g^{\mu\nu}\tensor{O}{^\beta_\nu}=D^{\alpha\beta} \\
D^{\alpha\beta}=\left\{
\begin{matrix}
D^\alpha &\alpha=\beta\\
0 & \alpha\neq \beta
\end{matrix}
\right.
\end{align*}
となるものが存在する.$D^\alpha$のうち三つは正の値をとり,ひとつは負の値をとる.そこで$D^0$は負,$D^i(i=1,2,3)$は正であるようにラベル付けして,$\tensor{d}{^i_\mu}=\tensor{O}{^i_\mu}/\sqrt{D^i}(i=1,2,3),\tensor{d}{^0_\mu}=\tensor{O}{^0_\mu}/\sqrt{-D^0}$とすればよい.)\par
そこで,$\Omega$のなかで$\mu$について共変ベクトル量な$\tensor{D}{^\alpha_\mu}(x)$を次の$\alpha=0,1,2,3$の四つの微分方程式で定義する.
\begin{align}
\frac{\partial \tensor{D}{^\alpha_\mu}}{\partial x^\nu}=\tensor{\Gamma}{^\lambda_{\mu\nu}} \tensor{D}{^\alpha_\lambda}
\end{align}
初期条件として$\tensor{D}{^\alpha_\mu}(P)=\tensor{d}{^\alpha_\mu}$を満たすとする.この式は$\Omega$の中で解をもつ.なぜならば積分可能条件を満足しているからである.すなわち
\begin{align*}
\frac{\partial^2 \tensor{D}{^\alpha_\mu}}{\partial x^\rho \partial x^\nu}=&\frac{\partial}{\partial x^\rho}\left(\tensor{\Gamma}{^\lambda_{\mu\nu}} \tensor{D}{^\alpha_\lambda} \right) \\
=&\frac{\partial \tensor{\Gamma}{^\lambda_{\mu\nu}}}{\partial x^\rho} \tensor{D}{^\alpha_\lambda} +\tensor{\Gamma}{^\lambda_{\mu\nu}} \frac{\partial \tensor{D}{^\alpha_\lambda} }{\partial x^\rho} \\
=&\frac{\partial \tensor{\Gamma}{^\tau_{\mu\nu}}}{\partial x^\rho} \tensor{D}{^\alpha_\tau} +\tensor{\Gamma}{^\lambda_{\mu\nu}} \tensor{\Gamma}{^\tau_{\lambda\rho}}\tensor{D}{^\alpha_\tau} 
\end{align*}
よって
\begin{align*}
\frac{\partial^2 \tensor{D}{^\alpha_\mu}}{\partial x^\rho \partial x^\nu}-\frac{\partial^2 \tensor{D}{^\alpha_\mu}}{\partial x^\nu \partial x^\rho}=\tensor{R}{^\tau_{\mu\rho\nu}} \tensor{D}{^\alpha_\tau}
\end{align*}
これは仮定により,$\Omega$のいたるところでゼロとなる.したがって(2.4.2)は積分可能であり,逐次近似法を用いて解ける.($\int dD$が積分可能,すなわち始点と終点のみで値が決まるならば,互いに異なる経路を$C_1,C_2$として
\begin{align*}
\int_{C_1}dD =&\int_{C_2}dD \\
\oint_C dD=&0 \\
\int_{\partial S}dD =&\int_S d^2 D=0
\end{align*}
であり,二階外微分$d^2D$がゼロであるためには,偏微分は可換でなければならない.)このようにして求まる$\tensor{D}{^\alpha_\mu}$を用いると
\begin{align}
\frac{\partial}{\partial x^\rho}\left( g^{\mu\nu}\tensor{D}{^\alpha_\mu}\tensor{D}{^\beta_\nu} \right)=0
\end{align}
が満たされる.これを見るには,(2.4.2)を書き換えて
\begin{align}
\nabla_\nu \tensor{D}{^\alpha_\mu}=0
\end{align}
の形にかけばよい.$g^{\mu\nu}\tensor{D}{^\alpha_\mu}\tensor{D}{^\beta_\nu}$に$\nabla_\rho$をかければ,計量条件と(2.4.4)によりゼロとなる.しかし各$(\alpha,\beta)$について,これはスカラー量であるから,スカラー量に対する共変微分は普通の微分と等価であり(2.4.3)を得る.すなわち$g^{\mu\nu}\tensor{D}{^\alpha_\mu}\tensor{D}{^\beta_\nu}$は$\Omega$内で定数となるから,
\begin{align}
g^{\mu\nu}(x)\tensor{D}{^\alpha_\mu}(x)\tensor{D}{^\beta_\nu}(x)=g^{\mu\nu}(P)\tensor{d}{^\alpha_\mu}\tensor{d}{^\beta_\nu}=\eta^{\alpha\beta}
\end{align}
を得る.さて,(2.4.2)の右辺は$\mu,\nu$について対称であるから
\begin{align}
\frac{\partial \tensor{D}{^\alpha_\mu}}{\partial x^\nu}=\frac{\partial \tensor{D}{^\alpha_\nu}}{\partial x^\mu}
\end{align}
を満足する.したがって微分方程式
\begin{align}
\tensor{D}{^\alpha_\mu}=\frac{\partial\phi^{(\alpha)}}{\partial x^\mu} 
\end{align}
で4個のスカラー量$\phi^{(\alpha)}$を定義すれば,積分可能条件(2.4.6)より$\phi^{(\alpha)}$について解くことができる.そこでいま$\Omega$内の新しい座標系$X^\alpha (\alpha=0,1,2,3)$を
\begin{align}
X^\alpha=\phi^{(\alpha)}(x^0,x^1,x^2,x^3)
\end{align}
で定める.これを使うと線素は
\begin{align}
ds^2=g'_{\alpha\beta}(X)dX^\alpha dX^\beta
\end{align}
である.ここでテンソルの変換則から
\begin{align*}
g'_{\alpha\beta}(X)=\frac{\partial x^\mu}{\partial X^\alpha}\frac{\partial x^\nu}{\partial X^\beta}g_{\mu\nu}(x)
\end{align*}
同様に$g'^{\alpha\beta}(X)$は
\begin{align*}
g'^{\alpha\beta}(X)=&\frac{\partial X^\alpha}{\partial x^\mu}\frac{\partial X^\beta}{\partial x^\nu}g^{\mu\nu}(x) \\
=&\tensor{D}{^\alpha_\mu} \tensor{D}{^\beta_\nu} g^{\mu\nu}
\end{align*}
ところが,この右辺は条件(2.4.5)により$\Omega$のいたるところで
\begin{align*}
g'^{\alpha\beta}=\eta^{\alpha\beta}
\end{align*}
したがって
\begin{align*}
g'_{\alpha\beta}(X)=\eta_{\alpha\beta}
\end{align*}
も$\Omega$内の全ての点で成立する.つまり$X$系はローレンツ座標系であり,$\Omega$はミンコフスキー空間である.$\tensor{R}{^\lambda_{\rho\mu\nu}}$はテンソルであるから,これがゼロであるか否かということは座標系の選択に依らず不変な性質である.

\vskip\baselineskip

ちなみに,重力そのものを意味しているのはリーマンテンソルではなく,クリストッフェル記号である.座標系の取り方によって重力が現れたり消えたりするのはテンソル量ではありえない.かつ重力ポテンシャルの微分が重力であったことを思い出すと,計量テンソルの一階微分で構成されているクリストッフェル記号が重力自体を意味している.実際,クリストッフェル記号はテンソル量ではなく,座標系の取り方によりいつでもゼロにすることが可能である.やや重要な定理として「ある点$P$において座標系をうまくとって$g_{\mu\nu}(P)=\eta_{\mu\nu},\partial g_{\mu\nu}/\partial x^\rho =0$(すなわち局所ローレンツ系)とできる必要十分条件は,$\tensor{\Gamma}{^\mu_\rho_\sigma}$が$\rho,\sigma$に対して対称(すなわちレヴィ・チビタ接続)であることである」というものがある.以下でこれを証明してみよう.\par
必要性は次のようにして確かめられる.いま$x'$系を適当にとったとき,$\tensor{\Gamma}{^\lambda_{\mu\nu}}'(x')$がすべてゼロになったとする.このとき変換則から
\begin{align*}
\tensor{\Gamma}{^\lambda_{\mu\nu}}'(x')=0=\frac{\partial x'^\lambda}{\partial x^\rho} \frac{\partial x^\alpha}{\partial x'^\mu}\frac{\partial x^\beta}{\partial x'^\nu} \tensor{\Gamma}{^\rho_{\alpha\beta}} +\frac{\partial x'^\lambda}{\partial x^\rho}\frac{\partial x^\rho}{\partial x'^\mu \partial x'^\nu}
\end{align*}
両辺に$\frac{\partial x^\rho}{\partial x'^\lambda} \frac{\partial x'^\mu}{\partial x^\alpha}\frac{\partial x'^\nu}{\partial x^\beta} $をかけると
\begin{align*}
\tensor{\Gamma}{^\rho_{\alpha\beta}}=-\frac{\partial x'^\mu}{\partial x^\alpha}\frac{\partial x'^\nu}{\partial x^\beta }\frac{\partial x^\rho}{\partial x'^\mu \partial x'^\nu}
\end{align*}
右辺は$\alpha,\beta$について対称であるから,左辺もそうでなくてはならない.\par
十分性は次のようにして確かめられる.簡単のため,問題にしている点を座標原点$x^\mu=0$にとる.いま,新旧座標の関係を
\begin{align}
x^\mu=x'^\mu+\frac{1}{2}\tensor{a}{^\mu_{\alpha\beta}} x'^\alpha  x'^\beta
\end{align}
とする.$\tensor{a}{^\mu_{\alpha\beta}}$は$x'$に無関係の未定の定数で,添え字$\alpha,\beta$に対して対称とする.この関係式を$x'^\mu$について解いた答えのうち,$x=0$では$x'=0$であるような解をとることにする.したがって$x=0$の近くでは
\begin{align}
x'^\mu=x^\mu-\frac{1}{2}\tensor{a}{^\mu_{\alpha\beta}} x^\alpha  x^\beta
\end{align}
という形の関係がなりたつ.そこで原点では
\begin{align*}
\left. \frac{\partial x'^\mu}{\partial x^\alpha} \right|_{x=0}=\delta^{\mu}_{\alpha} ,\quad \left. \frac{\partial x^\mu}{\partial x'^\alpha} \right|_{x'=0}=\delta^{\mu}_{\alpha}
\end{align*}
したがって接続係数の座標変換より
\begin{align*}
\tensor{\Gamma}{^\lambda_{\mu\nu}}'(x')=&\frac{\partial x'^\lambda}{\partial x^\rho} \frac{\partial x^\alpha}{\partial x'^\mu}\frac{\partial x^\beta}{\partial x'^\nu} \tensor{\Gamma}{^\rho_{\alpha\beta}} +\frac{\partial x'^\lambda}{\partial x^\rho}\frac{\partial x^\rho}{\partial x'^\mu \partial x'^\nu} \\
=&\tensor{\Gamma}{^\lambda_{\mu\nu}}(x)+\tensor{a}{^\lambda_{\mu\nu}}
\end{align*}
いま,$\tensor{\Gamma}{^\lambda_{\mu\nu}}$が仮定より$\mu,\nu$について対称であるから,未定の定数$\tensor{a}{^\lambda_{\mu\nu}}$を
\begin{align*}
\tensor{a}{^\lambda_{\mu\nu}}=-\tensor{\Gamma}{^\lambda_{\mu\nu}}(0)
\end{align*}
と決めればよい.\par
ここで
\begin{align*}
\frac{\partial g_{\mu\nu}}{\partial x^\rho }=g_{\mu\lambda}\tensor{\Gamma}{^\lambda_{\nu\rho}}+g_{\nu\lambda}\tensor{\Gamma}{^\lambda_{\mu\rho}}
\end{align*}
とクリストッフェル記号の定義より,$\Gamma=0$と$\partial g_{\mu\nu}/\partial x=0$は同値である.これにより計量は定数になるのだから,さらに一次変換をすればミンコフスキー的な計量になる.したがって定理を得る.

\vskip\baselineskip

この二つの理論の間の類似は一つの重要な点で成り立たない.それは,一般相対論ではアフィン接続はそれ自身,計量テンソルの一階微分から構成されているのに対し,ゲージ理論ではゲージ場は他のどのような,より基本的な場でも表されてはいないことだ.

\newpage

\subsection{ゲージ理論のラグランジアンと単純リー群}
ゲージ場テンソル$\tensor{F}{^\alpha_{\mu\nu}}$,物質場$\psi$,そしてそれらの共変微分の変換則は,どれも変換パラメータ$\epsilon^\alpha(x)$の微分を含まない.したがって,ラグランジアンがこれらの量のみから構成され,$\epsilon^\alpha$が定数の大域的変換のもとで不変ならば,一般に座標に依存する$\epsilon^\alpha(x)$のもとでも不変だ.したがって,ラグランジアンが
\begin{align*}
\mc{L}=\mc{L}(\psi,D_\mu\psi,D_\nu D_\mu \psi ,\cdots ,\tensor{F}{^\alpha_{\mu\nu}},D_\rho\tensor{F}{^\alpha_{\mu\nu}},\cdots)
\end{align*}
と書けて,ゲージ不変条件
\begin{align*}
\delta \mc{L}=&\frac{\partial \mc{L}}{\partial \psi_\ell}i\epsilon^\alpha \tensor{(t_\alpha)}{_\ell^m}\psi_m +\frac{\partial \mc{L}}{\partial (D_\mu \psi)_\ell}i\epsilon^\alpha\tensor{(t_\alpha)}{_\ell^m}(D_\mu \psi)_m+\frac{\partial \mc{L}}{\partial (D_\nu D_\mu \psi)_\ell}i\epsilon^\alpha\tensor{(t_\alpha)}{_\ell^m}(D_\nu D_\mu \psi)_m+\cdots \\
&+\frac{\partial \mc{L}}{\partial \tensor{F}{^\beta_{\mu\nu}}}i\epsilon^\alpha (\tensor{t}{^A_\alpha})\indices{^\beta_\gamma}\tensor{F}{^\gamma_{\mu\nu}}+\frac{\partial \mc{L}}{\partial D_\rho \tensor{F}{^\beta_{\mu\nu}}}i\epsilon^\alpha (\tensor{t}{^A_\alpha})\indices{^\beta_\gamma}D_\rho \tensor{F}{^\gamma_{\mu\nu}}+\cdots =0 \\
\therefore \quad &\frac{\partial \mc{L}}{\partial \psi_\ell}i \tensor{(t_\alpha)}{_\ell^m}\psi_m +\frac{\partial \mc{L}}{\partial (D_\mu \psi)_\ell}i\tensor{(t_\alpha)}{_\ell^m}(D_\mu \psi)_m+\frac{\partial \mc{L}}{\partial (D_\nu D_\mu \psi)_\ell}i\tensor{(t_\alpha)}{_\ell^m}(D_\nu D_\mu \psi)_m+\cdots \\
&+\frac{\partial \mc{L}}{\partial \tensor{F}{^\beta_{\mu\nu}}}\tensor{C}{^\beta_{\gamma\alpha}}\tensor{F}{^\gamma_{\mu\nu}}+\frac{\partial \mc{L}}{\partial D_\rho \tensor{F}{^\beta_{\mu\nu}}}\tensor{C}{^\beta_{\gamma\alpha}}D_\rho \tensor{F}{^\gamma_{\mu\nu}}+\cdots=0
\end{align*}
を満たすと仮定する.また,ラグランジアンは$\tensor{F}{^\alpha_{\mu\nu}}$やゲージ共変微分$D_\mu$を通してのみゲージ場自身に依存しなくてはならない.特に,質量項$-\frac{1}{2}m^2_{\alpha\beta}A_{\alpha\mu}\tensor{A}{_\beta^\mu}$は許されない.\par
ここで,ラグランジアンの$\tensor{F}{^\alpha_{\mu\nu}}$にのみ依存する項を考える.電磁理論のように,任意のスピンが1で質量がゼロの粒子について,ラグランジアンは$\partial_\mu \tensor{A}{_\alpha_\nu}-\partial_\nu \tensor{A}{_\alpha_\mu}$にちての二次の自由粒子の項を含まなければならない.ゲージ不変性より,この自由粒子項は場の強度テンソル$\tensor{F}{^\alpha_{\mu\nu}}$について二次の項の一部として現れなければならない.この項は,ローレンツ不変性とパリティ保存を仮定すると,$g_{\alpha\beta}$を定数行列として
\begin{align*}
\mc{L}_A=-\frac{1}{4} g_{\alpha\beta} \tensor{F}{^\alpha_{\mu\nu}}\tensor{F}{^{\beta\mu\nu}}
\end{align*}
という形に決まる.(これは,$\mc{L}_A(x)\to \mc{L}_A(\mc{P}x)$としたときに現れる$\mc{P}\indices{^\mu_\nu}$(2.6節参照)を打ち消すために,$\mu,\nu$について計量テンソルによる縮約をしていなければならないからだ)もしパリティ(または$\mathsf{CP}$か$\mathsf{T}$)保存を仮定しなければ,ラグランジアンに$\theta_{\alpha\beta}$を別の定数行列として
\begin{align*}
\mc{L}'_A=-\frac{1}{2}\theta_{\alpha\beta}\epsilon^{\mu\nu\rho\sigma}\tensor{F}{^\alpha_{\mu\nu}}\tensor{F}{^\beta_{\rho\sigma}}
\end{align*}
という項も含めることができる.この項が$\mathsf{P}$対称性を破ることを見るには$\epsilon^{\mu\nu\rho\sigma}$が
\begin{align*}
\tensor{\Lambda}{^\mu_\alpha}\tensor{\Lambda}{^\nu_\beta}\tensor{\Lambda}{^\rho_\gamma}\tensor{\Lambda}{^\sigma_\delta}\epsilon^{\alpha\beta\gamma\delta}=\mathrm{det}(\Lambda)\epsilon^{\mu\nu\rho\sigma}
\end{align*}
の意味で擬テンソルであることを使えばよい.実際
\begin{align*}
\mc{L}'_A=&-\frac{1}{2}\theta_{\alpha\beta}\epsilon^{\mu\nu\rho\sigma}\tensor{F}{^\alpha_{\mu\nu}}\tensor{F}{^\beta_{\rho\sigma}} \\
\to &-\frac{1}{2}\theta_{\alpha\beta}\epsilon^{\mu\nu\rho\sigma}\tensor{\mc{P}}{_\mu^\tau}\tensor{\mc{P}}{_\nu^\lambda}\tensor{\mc{P}}{_\rho^\kappa}\tensor{\mc{P}}{_\sigma^\eta}\tensor{F}{^\alpha_{\tau\lambda}}\tensor{F}{^\beta_{\kappa\eta}} \\
=&-\frac{1}{2}\mathrm{det}(\mc{P})\theta_{\alpha\beta}\epsilon^{\mu\nu\rho\sigma}\tensor{F}{^\alpha_{\mu\nu}}\tensor{F}{^\beta_{\rho\sigma}} \\
=&+\frac{1}{2}\theta_{\alpha\beta}\epsilon^{\mu\nu\rho\sigma}\tensor{F}{^\alpha_{\mu\nu}}\tensor{F}{^\beta_{\rho\sigma}}=-\mc{L}'_A
\end{align*}
となり,不変でない.この項は22章で論じるように,実はある関数の全微分になっていて,場の方程式やファインマン則には影響しない.しかしそのような項は23章で見るように非摂動論的な量子効果を及ぼす.\par
(15.1.13)で定義された場の強度$\tensor{F}{^\alpha_{\mu\nu}}$の二次の項からは,(15.2.3)にゲージ場について三・四次の項が現れ相互作用が生じるが,ゲージ場$\tensor{A}{^\alpha_\mu}(x)$の運動項を,この相互作用項が生じないように導入するのは不可能だ.この点でも,非可換ゲージ理論は一般相対論に似ている.一般相対論では重力場のラグランジアンの運動項部分は,アインシュタインヒルベルトのラグランジアン密度$-\sqrt{g}R/8\pi G$に含まれていて,これにはまた場の自己相互作用も含まれている.この二つの場合が似ているのには理由がある.重力場がそれ自身と相互作用するのは,それがエネルギーと運動量を持つもの全てと相互作用するからで,また,ゲージ場が自分自身と相互作用するのは,それがゲージ群の自明でない表現として変換するもの全てと相互作用するからである.(ゲージ場はゲージ群の随伴表現として変換することを思い出そう.)これは電磁理論と対照的だ.光子は,それが相互作用する量子数である電荷を持たないので,電磁場の運動項を自己相互作用なしに$-\frac{1}{4}F_{\mu\nu}F^{\mu\nu}$と導入することが可能だ.(エネルギー運動量や電荷は,ネーター電荷である.重力場を含む一般的なゲージ場は,それぞれのゲージ変換群のネーター電荷を持つもの全てと相互作用すると言える.)\par
数値行列$g_{\alpha\beta}$は対称的にとることができ,ラグランジアンが実であるためには実行列でなければならない.この項がゲージ不変性(15.2.2)を満たすためには,
\begin{align*}
& \frac{\partial \mc{L}_A}{\partial \tensor{F}{^\beta_{\mu\nu}}}\tensor{C}{^\beta_{\gamma\alpha}}\tensor{F}{^\gamma_{\mu\nu}}=0 \\
\therefore \quad & g_{\alpha\beta}\tensor{F}{^\alpha_{\mu\nu}}\tensor{C}{^\beta_{\gamma\delta}}\tensor{F}{^{\gamma\mu\nu}}=0
\end{align*}
が成立していなければならない.関数$F$の間に何ら関数関係を課すことなくこれを満たすためには,行列$g_{\alpha\beta}$は
\begin{align*}
& g_{\alpha\beta}\tensor{F}{^\alpha_{\mu\nu}}\tensor{C}{^\beta_{\gamma\delta}}\tensor{F}{^{\gamma\mu\nu}}=\frac{1}{2}\left(g_{\alpha\beta}\tensor{C}{^\beta_{\gamma\delta}}+g_{\gamma\beta}\tensor{C}{^\beta_{\alpha\delta}}\right)\tensor{F}{^\alpha_{\mu\nu}}\tensor{F}{^{\gamma\mu\nu}}=0 \\
\therefore \quad &g_{\alpha\beta}\tensor{C}{^\beta_{\gamma\delta}}=-g_{\gamma\beta}\tensor{C}{^\beta_{\alpha\delta}}
\end{align*}
という条件を満たさなければならない.もう一つ,行列$g_{\alpha\beta}$についての重要な条件がある.量子電磁理論のように,正準量子化の法則と量子力学的スカラー積の正定値条件はラグランジアン(15.2.3)の中の行列$g_{\alpha\beta}$が正定値であることを要求する.

\newpage

\subsection{場の方程式と保存則}
(15.2.3)において行列$g_{\alpha\beta}$を$\delta_{\alpha\beta}$とすると,完全なラグランジアン密度は
\begin{align*}
\mc{L}=-\frac{1}{4}\tensor{F}{_{\alpha\mu\nu}}\tensor{F}{_\alpha^{\mu\nu}}+\mc{L}_M(\psi,D_\mu\psi)
\end{align*}
となる.ここで$\mc{L}_M(\psi,D_\mu \psi)$はゲージ場が無ければ「物質」のラグランジアン密度となるものだ.原則的には,$\mc{L}_M$が$\tensor{F}{_{\alpha\mu\nu}}$や高次の微分$D_\nu D_\mu \psi,D_\lambda \tensor{F}{_{\alpha\mu\nu}}$等に依存していてもよいが,これらを含んだ項は非くりこみ可能になるため,電磁理論と同じ理由で排除する.つまり,12.3節で論じたように,そのような項は通常のエネルギーでも,ある非常に大きな質量の逆ベキで抑えられているはずだ.このため,弱・電磁・強相互作用の標準理論は(15.3.1)の一般的な形のラグランジアンをもつ.\par
このゲージ場の運動方程式は
\begin{align*}
0=&\partial_\mu \frac{\partial \mc{L}}{\partial (\partial_\mu A_{\alpha\nu})}-\frac{\partial \mc{L}}{\partial A_{\alpha\nu}} \\
=&-\partial_\mu \tensor{F}{_\alpha^{\mu\nu}}-\left[\tensor{F}{_\gamma^{\nu\mu}}C_{\gamma\alpha\beta}A_{\beta\mu}-i\frac{\partial \mc{L}_M}{\partial D_\nu \psi}t_\alpha \psi\right]
\end{align*}
となる.これは
\begin{align*}
\partial_\mu \tensor{F}{_\alpha^{\mu\nu}}=-\mc{J}\indices{_\alpha^\nu}
\end{align*}
と見慣れた形に書ける.ただし$\mc{J}\indices{_\alpha^\nu}$は
\begin{align*}
\mc{J}\indices{_\alpha^\nu}\equiv \frac{\partial \mc{L}}{\partial A_{\alpha\nu}}=\tensor{F}{_\gamma^{\nu\mu}}C_{\gamma\alpha\beta}A_{\beta\mu}-i\frac{\partial \mc{L}_M}{\partial D_\nu \psi}t_\alpha \psi
\end{align*}
で定義されるカレントだ.このカレントは
\begin{align*}
\partial_\nu \mc{J}\indices{_\alpha^\nu}=0
\end{align*}
と,通常の意味で保存する.これは上の場の方程式からも,不変条件(15.2.2)からも導ける.\par
(15.3.2)と(15.3.4)の微分は通常の微分であり,ゲージ共変な微分$D_\nu$ではないから,これらの式のゲージ共変性は多少わかりにくい.それを明白にするためには,(15.3.2)を場の強度のゲージ共変微分
\begin{align*}
D_\lambda \tensor{F}{_\alpha^{\mu\nu}} \equiv & \partial_\lambda \tensor{F}{_\alpha^{\mu\nu}}-i(\tensor{t}{^A_\beta})_{\alpha\gamma}A_{\beta\lambda}\tensor{F}{_\gamma^{\mu\nu}} \\
=&\partial_\lambda \tensor{F}{_\alpha^{\mu\nu}}-C_{\alpha\gamma\beta}A_{\beta\lambda}\tensor{F}{_\gamma^{\mu\nu}}
\end{align*}
を使って書き換えるとよい.そうすると,(15.3.2)は
\begin{align*}
D_\mu \tensor{F}{_\alpha^{\mu\nu}}=-\tensor{J}{_\alpha^\nu}
\end{align*}
となる.ここで$\tensor{J}{_\alpha^\nu}$は
\begin{align*}
\tensor{J}{_\alpha^\nu} \equiv -i\frac{\partial \mc{L}_M}{\partial D_\nu \psi}t_\alpha \psi
\end{align*}
であり,物質場のみのカレントである.これは,もし$\mc{L}_M$がゲージ不変なら,この方程式はゲージ共変である.また(15.3.6)に$D_\nu$を施して,交換関係
\begin{align*}
[D_\nu,D_\mu]\tensor{F}{_\alpha^{\rho\sigma}}=-i(\tensor{t}{^A_\beta})\indices{_{\alpha\beta}}\tensor{F}{_{\gamma\nu\mu}}\tensor{F}{_\beta^{\rho\sigma}}=-C_{\gamma\alpha\beta}F_{\gamma\nu\mu}\tensor{F}{_\beta^{\rho\sigma}}
\end{align*}
を使うと(これは,場の強度テンソルが随伴表現に従うことと(15.1.12)より簡単にわかる),$\tensor{J}{_\alpha^\nu}$が,完全なカレント$\mc{J}\indices{_\alpha^\nu}$の満たす通常の保存則(15.3.4)ではなく,ゲージ共変な保存則
\begin{align*}
D_\nu \tensor{J}{_\alpha^\nu}=&-D_\nu D_\mu \tensor{F}{_\alpha^{\mu\nu}}=-\frac{1}{2}[D_\nu ,D_\mu]\tensor{F}{_\alpha^{\mu\nu}} \\
=&\frac{1}{2}C_{\gamma\alpha\beta}F_{\gamma\nu\mu}\tensor{F}{_\beta^{\mu\nu}}=0
\end{align*}
を満たすことが分かる.また,以下の恒等式を導くのも(ヤコビの恒等式(15.1.5)を使えば)簡単だ.
\begin{align*}
D_\mu F_{\alpha\nu\lambda}+D_\nu F_{\alpha\lambda\mu}+D_\lambda F_{\alpha\mu\nu}=0
\end{align*}
これは,ビアンキの恒等式と呼ばれ,ゲージ場が場の方程式を満たすかどうかに依らず成立する.

\newpage

\subsection{量子化}
これまでの二つの節で述べたゲージ理論の量子化に進む.ラグランジアン密度は(15.3.1)の形
\begin{align*}
\mc{L}=-\frac{1}{4}F_{\alpha\mu\nu}\tensor{F}{_\alpha^{\mu\nu}}+\mc{L}_M(\psi,D_\mu \psi)
\end{align*}
とする.ここで
\begin{align*}
F_{\alpha\mu\nu}\equiv& \partial_\mu A_{\alpha\nu}-\partial_\nu A_{\alpha\mu}+C_{\alpha\beta\gamma}A_{\beta\mu}A_{\gamma\nu} \\
D_\mu \psi \equiv & \partial_\mu \psi -it_\alpha A_{\alpha\mu}\psi
\end{align*}
としている.この理論では,交換子をポアソン括弧に$i$をかけたものに等しいとして安直に量子化を行うことはできない.問題は一つの拘束条件にある.7.6節に述べたディラックの用語を使うならば,この理論には第一種拘束条件
\begin{align*}
\Pi_{\alpha 0} \equiv \frac{\partial \mc{L}}{\partial (\partial_0 A^0_\alpha)}=F^{0\mu}_\alpha=0
\end{align*}
と,$A^0_\alpha$の場の方程式から導かれる第二種拘束条件
\begin{align*}
&-\partial_\mu \frac{\partial \mathcal{L}}{\partial(\partial_\mu A_{\alpha 0})}+\frac{\partial \mathcal{L}}{\partial A_{\alpha 0}}=0 \\
=&\partial_\mu F^{\mu 0}_\alpha +F^{\mu 0}_\gamma C_{\gamma\alpha\beta}A_{\beta\mu}+J^{0}_\alpha \\
=&\partial_k  \Pi_\alpha^k +\Pi^k_\gamma C_{\gamma\alpha\beta}A_{\beta\mu}+J^{0}_\alpha
\end{align*}







\end{document}