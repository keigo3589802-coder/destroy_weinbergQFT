\documentclass[dvipdfmx]{jsarticle}
\let\headfont=\gtfamily
\usepackage[dvips]{graphicx}
\usepackage{amsmath}
\usepackage{mathrsfs} % 花文字\mathscr{M}, 筆記体\mathcal{M}, 黒板文字\mathbb{M},ドイツ文字\mathfrak{M}
\usepackage{bm} %太文字
\usepackage{amssymb}
\usepackage{latexsym}
\usepackage{braket}
\usepackage{tikz}
\usepackage{tikz-feynhand}
\usepackage{ulem}
\usepackage{tensor}
\usepackage{bigdelim}
\usepackage{multirow}
\usepackage{tcolorbox}
\usepackage{here}
\tcbuselibrary{theorems,skins}
\usetikzlibrary{decorations}
\usepackage{color}
\usepackage{xcolor} %ここからハイパーリンクのパッケージの設定
\usepackage[%
  dvipdfmx,%
  bookmarks=true,%
  bookmarksnumbered=true,%
  colorlinks=true,%
  allcolors=blue%
]{hyperref}
\usepackage{pxjahyper}
\usepackage{cite} %ここまで

\usetikzlibrary{intersections, calc, arrows.meta}
 \usetikzlibrary{patterns}

\newfont{\bg}{cmr9 scaled\magstep4}
\newcommand{\bigzerol}{\smash{\lower1.0ex\hbox{\bg 0}}}
\newcommand{\bigzerou}{%
   \smash{\hbox{\bg 0}}}
\newcommand{\mcO}{\mathcal{O}}
\newcommand{\VAC}{\mathrm{VAC}}
\newcommand{\Slash}[1]{{\ooalign{\hfil/\hfil\crcr$#1$}}} %ファインマンのスラッシュ記号
\renewcommand{\mc}{\mathcal}

\tcbset{
  colback=blue!5!white,  % 背景色
  colframe=blue!75!black, % 枠線の色
  fonttitle=\bfseries,    % タイトルの太字
  boxrule=0.8pt,          % 枠線の太さ
  arc=2mm,                % 角の丸み
  left=4mm, right=4mm, top=2mm, bottom=2mm, % 余白
}

\newtcolorbox{sectiongoal}[1][]{title=この節でやるべきこと,#1}

% \textrm{Roman デフォルト}
% \textgt{Gothic 和文ゴシック体}*専門用語に
% \textbf{Boldface 太字}*専門用語(英語)に
% \textit{Italic 斜体}
% \textsl{Slanted ローマンを傾けただけ}
% \textsf{Sans Serif サンセリフ体}
% \texttt{Typewriter タイプライタ体、等幅}
% \textsc{Small Caps 小文字が大文字に}

\setlength{\textwidth}{\fullwidth}
\setlength{\textheight}{44\baselineskip}
\addtolength{\textheight}{\topskip}
\setlength{\voffset}{-0.6in}

\allowdisplaybreaks[4]

\makeatletter
  \renewcommand{\theequation}
  {\arabic{section}.\arabic{equation}}
  \@addtoreset{equation}{section}
 \makeatother

\title{\vspace{-1cm}\Huge{WeinbergQFT Part5}}
\author{坂井 啓悟(Sakai Keigo)}
\date{}
\begin{document}


\maketitle

\tableofcontents

\newpage

\setcounter{part}{4}
\part{量子場と反粒子}
\setcounter{section}{5}
\setcounter{subsection}{0}
\subsection{自由場}
3章で見たように,$S$行列は,相互作用が
\begin{align*}
V_I(t)=\int d^3 x \mc{H}_I(\mathbf{x},t)
\end{align*}
と書ければ明白にローレンツ不変である.ここで$\mc{H}_I$は
\begin{align*}
U_0(\Lambda,a)\mc{H}_I(x) U^{-1}_0(\Lambda,a)=\mc{H}_I(\Lambda x+a)
\end{align*}
の意味でローレンツスカラーであり,かつ付加的な条件としての因果律
\begin{align*}
[\mc{H}_I(x),\mc{H}_I(x')]=0 ,\quad \mathrm{where} \quad (x-x')^2 \geq 0
\end{align*}
を満たすとする.これよりもっと一般的なものも可能性があるが,そのどれもがこれとあまり違いがないらしい.(今のところ,ここでの$\Lambda$が固有順時ローレンツ変換に限られるか,あるいは空間反転を含むかは未解決の問題として残しておくことにする.)また,クラスター分解原理を容易に満たすためには,$\mc{H}_I(x)$を生成・消滅演算子から構成する.しかし,ここで一つの問題に直面する.すなわち,(4.2.12)に示されたように,ローレンツ変換のもとでそのような演算子の各々にはその演算子が持つ運動量に依存する行列$D(W(\Lambda,p))$がかかる.したがって直接生成・消滅演算子の積を構築してローレンツ不変なようにするのは困難である.どのようにすれば,そのような演算子を繋ぎ合わせてスカラーを作れるだろうか?以上より,我々がこの節で特にやるべきことは次の通りである.

\begin{sectiongoal}
\begin{itemize}
\item クラスター分解原理を満たす$\mc{H}_I(x)$を具体的に構成する.
\item ローレンツスカラーな$\mc{H}_I(x)$を具体的に構成する.
\item 因果律を満たす$\mc{H}_I(x)$を具体的に構成する.
\end{itemize}
\end{sectiongoal}

クラスター分解とローレンツスカラー性を満たす$\mc{H}_I(x)$を構成する方法に関して,早速答えを述べてしまおう.それは,$\mc{H}_I(x)$を\uwave{場},すなわち消滅場$\psi^+_\ell(x)$と生成場$\psi^-_\ell(x)$
\begin{align*}
\psi^+_\ell(x)=\sum_{\sigma n}\int d^3 \mathbf{p} u_\ell (x;\mathbf{p},\sigma,n) a(\mathbf{p},\sigma,n) \\
\psi^-_\ell(x)=\sum_{\sigma n}\int d^3 \mathbf{p} v_\ell (x;\mathbf{p},\sigma,n) a^\dagger(\mathbf{p},\sigma,n)
\end{align*}
から構成することである\footnote{ここで「場」という言葉は,時空点に対してある集合$E$の元を当てはめる写像$\psi:M \to E,x \mapsto \psi(x)$のことを指す.例えばスカラー場とは,ミンコフスキー時空の点$x\in \mathbb{R}^{3+1}$に対して,ローレンツ変換のもとでスカラー的に変換する演算子$\phi(x)$を当てはめる写像$\phi$を指す.これ以降,誤解のない限りもっぱら$\psi(x)$のことを場と呼ぶ.数学的に厳密な定義は,幾何学的にはファイバー束の切断だとか,代数的には層の切断だとかの定義があるが,ここでは何も気にしなくてよい.}.ここで係数$u_\ell (x;\mathbf{p},\sigma,n),v_\ell (x;\mathbf{p},\sigma,n)$は,ローレンツ変換のもとで位置に依らない行列が各々の場に
\begin{align*}
U_0(\Lambda,a)\psi^+_\ell (x)U^{-1}_0(\Lambda,a)=&\sum_{\bar{\ell}} D_{\ell\bar{\ell}}(\Lambda^{-1})\psi^+_{\bar{\ell}} (\Lambda x+a) \\
U_0(\Lambda,a)\psi^-_\ell (x)U^{-1}_0(\Lambda,a)=&\sum_{\bar{\ell}} D_{\ell\bar{\ell}}(\Lambda^{-1})\psi^-_{\bar{\ell}} (\Lambda x+a)
\end{align*}
とかかるように適当に選ぶ\footnote{この$D(\Lambda)$は2章で導入した小群の表現行列とは違うものであり,場$\psi_\ell$の$\ell$(例えばローレンツ添え字$\mu$やスピノル添え字$\alpha$など)を成分とする行列になっている.2章で導入した行列はスピン添え字$\sigma$を成分とする行列だった.}.(原理的には$\psi^+_\ell$と$\psi^-_\ell$で変換行列$D$が異なりえる.つまり
\begin{align*}
U_0(\Lambda,a)\psi^+_\ell (x)U^{-1}_0(\Lambda,a)=&\sum_{\bar{\ell}} D^+_{\ell\bar{\ell}}(\Lambda^{-1})\psi^+_{\bar{\ell}} (\Lambda x+a) \\
U_0(\Lambda,a)\psi^-_\ell (x)U^{-1}_0(\Lambda,a)=&\sum_{\bar{\ell}} D^-_{\ell\bar{\ell}}(\Lambda^{-1})\psi^-_{\bar{\ell}} (\Lambda x+a)
\end{align*}
であり,$D^+\neq D^-$かもしれない.しかし,後で見るように,これらの行列が同じになるように場を選ぶことが常に可能らしい.)別のローレンツ変換$\bar{\Lambda}$を適用すれば,(2.3.11)より
\begin{align*}
U_0(\bar{\Lambda},\bar{a})U_0(\Lambda,a)\psi^+_\ell (x)U^{-1}_0(\Lambda,a)U^{-1}_0(\bar{\Lambda},\bar{a})=&\sum_{\bar{\ell}} D_{\ell\bar{\ell}}(\Lambda^{-1}) U_0(\bar{\Lambda},\bar{a}) \psi^+_{\bar{\ell}} (\Lambda x+a) U_0^{-1}(\bar{\Lambda},\bar{a}) \\
=&\sum_{\bar{\ell}\bar{\ell}'} D_{\ell\bar{\ell}}(\Lambda^{-1}) D_{\bar{\ell}\bar{\ell}'}(\bar{\Lambda}^{-1}) \psi^+_{\bar{\ell}'} (\bar{\Lambda}\Lambda x+(\bar{\Lambda}a+\bar{a})) \\
=&\sum_{\bar{\ell}} \Bigl(D(\Lambda^{-1}) D(\bar{\Lambda}^{-1})\Bigr)_{\ell\bar{\ell}} \psi^+_{\bar{\ell}} (\bar{\Lambda}\Lambda x+(\bar{\Lambda}a+\bar{a})) \\
=U_0(\bar{\Lambda}\Lambda,\bar{\Lambda}a+\bar{a})\psi_{\ell}^+(x)U_0^{-1}(\bar{\Lambda}\Lambda,\bar{\Lambda}a+\bar{a})=&\sum_{\bar{\ell}}D_{\ell\bar{\ell}}((\bar{\Lambda}\Lambda)^{-1})\psi^+_{\bar{\ell}} (\bar{\Lambda}\Lambda x+(\bar{\Lambda}a+\bar{a})) \\
\therefore \quad D(\Lambda^{-1})D(\bar{\Lambda}^{-1})=&D((\bar{\Lambda}\Lambda)^{-1})
\end{align*}
となり,よって$\Lambda_1=\Lambda^{-1},\Lambda_2=\bar{\Lambda}^{-1}$とおくと
\begin{align*}
D(\Lambda_1)D(\Lambda_2)=D(\Lambda_1\Lambda_2)
\end{align*}
を得る.したがって$D$行列は斉次ローレンツ群の表現を与えることがわかる.\par
そのような表現はたくさんある.例えば,スカラー表現$D(\Lambda)=1$,ベクトル表現$\tensor{D(\Lambda)}{^\mu_\nu}=\tensor{\Lambda}{^\mu_\nu}$,および多くのテンソルと(5.4節で述べる)スピノル表現$D_{\alpha\beta}(\Lambda)$が含まれる.これらの特定の表現は,基底の選び方によって全ての$D(\Lambda)$を二個以上のブロックをもつ同一のブロック対角な形にすることはできない,という意味で既約である\footnote{ここで可約ディラック表現は議論に含んでいない.左カイラル表現と右カイラル表現が既約になっている.}しかし,現時点では$D(\Lambda)$が既約であることは要求せず,一般にブロック対角な行列で,ブロック内では任意の配列の既約表現になっているとする.
\begin{align*}
D(\Lambda)=\left(
\begin{matrix}
\tensor{\Lambda}{^\mu_\nu} &   &                 &   \\
                           & 1 &                 &   \\
                          &    & D_{\alpha\beta}(\Lambda) &   \\
                          &    &                 &\ddots
\end{matrix}
\right)
\end{align*}
すなわち,ここでの指標$\ell$は,個々の既約表現の成分($\mu,\alpha,\cdots$など)ばかりでなく,記述される粒子の型および異なるブロック内の既約表現に関する添え字を含む.後でこれらの場を,既約場に分解し,その各々が単一の粒子の種類$n$(およびその反粒子$n^c$)を記述し,ローレンツ群のもとで既約表現として変換するようにする.

\vskip\baselineskip


ローレンツ変換則(5.1.6)(5.1.6)を満たす場を構成する方法が一旦分かれば,相互作用密度を
\begin{align*}
\mc{H}_I(x)=\sum_{NM}\sum_{\ell_1'\cdots \ell_N'}\sum_{\ell_1\cdots \ell_M}g_{\ell_1' \cdots \ell'_N,\ell_1 \cdots \ell_M} \psi^-_{\ell_1'}(x)\cdots \psi^-_{\ell_N'}(x) \psi^+_{\ell_1}(x) \cdots \psi^+_{\ell_M}(x)
\end{align*}
と構成できる.ここで定数係数$g_{\ell_1' \cdots \ell'_N,\ell_1 \cdots \ell_M}$が任意のローレンツ変換$\Lambda$について
\begin{align*}
&U_0(\Lambda,a)\mc{H}_I(x)U_0^{-1}(\Lambda,a) \\
=&U_0(\Lambda,a)\left[\sum_{NM}\sum_{\ell_1'\cdots \ell_N'}\sum_{\ell_1\cdots \ell_M}g_{\ell_1' \cdots \ell'_N,\ell_1 \cdots \ell_M} \psi^-_{\ell_1'}(x)\cdots \psi^-_{\ell_N'}(x) \psi^+_{\ell_1}(x) \cdots \psi^+_{\ell_M}(x)\right] U_0^{-1}(\Lambda,a) \\
=&\sum_{NM}\sum_{\ell_1'\cdots \ell_N'}\sum_{\ell_1\cdots \ell_M}g_{\ell_1' \cdots \ell'_N,\ell_1 \cdots \ell_M} \\
&\times \sum_{\bar{\ell}_1' \cdots \bar{\ell}_N' } D_{\ell_1' \bar{\ell}_1'}(\Lambda^{-1})\psi^-_{\bar{\ell}_1'}(\Lambda x+a)\cdots D_{\ell_N' \bar{\ell}_N'}(\Lambda^{-1})\psi^-_{\bar{\ell}_N'}(\Lambda x+a) \\
&\times \sum_{\bar{\ell}_1 \cdot \bar{\ell}_M} D_{\ell_1 \bar{\ell}_1}(\Lambda^{-1}) \psi^+_{\bar{\ell}_1}(\Lambda x+a) \cdots D_{\ell_M \bar{\ell}_M}(\Lambda^{-1}) \psi^+_{\bar{\ell}_M}(\Lambda x+a) \\
=&\sum_{NM}\sum_{\bar{\ell}_1' \cdots \bar{\ell}_N' }\sum_{\bar{\ell}_1 \cdot \bar{\ell}_M}\left[\sum_{\ell_1'\cdots \ell_N'} \sum_{\ell_1\cdots \ell_M}D_{\ell_1' \bar{\ell}_1'}(\Lambda^{-1}) \cdots D_{\ell_N' \bar{\ell}_N'}(\Lambda^{-1}) D_{\ell_1 \bar{\ell}_1}(\Lambda^{-1})\cdots D_{\ell_M \bar{\ell}_M}(\Lambda^{-1}) g_{\ell_1' \cdots \ell'_N,\ell_1 \cdots \ell_M}\right] \\
&\times  \psi^-_{\bar{\ell}_1'}(\Lambda x+a)\cdots \psi^-_{\bar{\ell}_N'}(\Lambda x+a) \psi^+_{\bar{\ell}_1}(\Lambda x+a) \cdots \psi^+_{\bar{\ell}_M}(\Lambda x+a)
\end{align*}
これが(5.1.2)を満たし$\mc{H}_I(\Lambda x+a)$であるためには
\begin{align*}
\sum_{\ell_1'\cdots \ell_N'} \sum_{\ell_1\cdots \ell_M}D_{\ell_1' \bar{\ell}_1'}(\Lambda^{-1}) \cdots D_{\ell_N' \bar{\ell}_N'}(\Lambda^{-1}) D_{\ell_1 \bar{\ell}_1}(\Lambda^{-1})\cdots D_{\ell_M \bar{\ell}_M}(\Lambda^{-1}) g_{\ell_1' \cdots \ell'_N,\ell_1 \cdots \ell_M}=g_{\bar{\ell}_1' \cdots \bar{\ell}'_N,\bar{\ell}_1 \cdots \bar{\ell}_M}
\end{align*}
を満たすようにローレンツ共変に選べばよい.(これらの場の成分の微分$\partial_\mu \psi_\ell^\pm$は別の種類の場の成分と今はみなしている\footnote{微分が入るとローレンツ変換のもとでの変換性が変わる.異なる既約表現に属する場は別の種類とみなしている.}ので,ここでは微分を含んでいない.)この式を満たすように係数$g_{\ell_1' \cdots \ell'_N,\ell_1 \cdots \ell_M}$を見つける作業は,クレブシュ・ゴルダン係数を用いて三次元回転群の様々な表現を組み合わせて回転に関するスカラーを作る作業と原理的に同じである.(しかも実際に行うのもそれほど難しくない.)後で見るように,生成場と消滅場を組み合わせて,この相互作用密度が,空間的に離れた時空点で自分自身と交換(5.1.3)するようにできる.

\vskip\baselineskip


さて,係数関数$u_\ell(x;\mathbf{p},\sigma,n)$および$v_\ell(x;\mathbf{p},\sigma,n)$としては何をとればよいだろうか?(4.2.12)とその共役式から消滅演算子と生成演算子の変換則
\begin{align*}
U_0(\Lambda,b)a(\mathbf{p},\sigma,n) U_0^{-1}(\Lambda,b)=&\exp\Bigl(i(\Lambda p)\cdot b\Bigr)\sqrt{\frac{(\Lambda p)^0}{p^0}}\sum_{\bar{\sigma}}D_{\bar{\sigma}\sigma}^{(j_n)*}\Bigl( W(\Lambda,p) \Bigr) a(\mathbf{p}_\Lambda,\bar{\sigma},n) \\
=&\exp\Bigl(i(\Lambda p)\cdot b\Bigr)\sqrt{\frac{(\Lambda p)^0}{p^0}}\sum_{\bar{\sigma}}D_{\sigma\bar{\sigma}}^{(j_n)\dagger }\Bigl( W(\Lambda,p) \Bigr) a(\mathbf{p}_\Lambda,\bar{\sigma},n) \\
=&\exp\Bigl(i(\Lambda p)\cdot b\Bigr)\sqrt{\frac{(\Lambda p)^0}{p^0}}\sum_{\bar{\sigma}}D_{\sigma\bar{\sigma}}^{(j_n) }\Bigl( W^{-1}(\Lambda,p) \Bigr) a(\mathbf{p}_\Lambda,\bar{\sigma},n) \\
U_0(\Lambda,b)a^\dagger(\mathbf{p},\sigma,n) U_0^{-1}(\Lambda,b)=&\exp\Bigl(-i(\Lambda p)\cdot b\Bigr)\sqrt{\frac{(\Lambda p)^0}{p^0}}\sum_{\bar{\sigma}}D_{\sigma\bar{\sigma}}^{(j_n)*}\Bigl( W^{-1}(\Lambda,p) \Bigr) a^\dagger(\mathbf{p}_\Lambda,\bar{\sigma},n)
\end{align*}
が得られる.ここで$j_n$は種類$n$の粒子のスピン,また$\mathbf{p}_\Lambda$は$\Lambda p$の3元ベクトル部分である.(ここで$D^{(j_n)}_{\bar{\sigma}\sigma}$が小群$W(\Lambda,p)\in SO(3)$のユニタリー表現になっていることから
\begin{align*}
&D^{(j_n)}\Bigl(W^{-1}(\Lambda,p)\Bigr)D^{(j_n)}\Bigl(W(\Lambda,p)\Bigr)=D^{(j_n)}\Bigl(W^{-1}(\Lambda,p)W(\Lambda,p)\Bigr)=D^{(j_n)}(1)=1 \\
\therefore \quad &\left[D^{(j_n)}\Bigl(W(\Lambda,p)\Bigr)\right]^{-1}=D^{(j_n)}\Bigl(W^{-1}(\Lambda,p)\Bigr)
\end{align*}
とユニタリー性から
\begin{align*}
\left[D^{(j_n)}\Bigl(W(\Lambda,p)\Bigr)\right]^\dagger=\left[D^{(j_n)}\Bigl(W(\Lambda,p)\Bigr)\right]^{-1}=D^{(j_n)}\Bigl(W^{-1}(\Lambda,p)\Bigr)
\end{align*}
であることを用いた.)また,2.5節で見たように,体積要素$d^3\mathbf{p}/p^0$はローレンツ不変であり,よって(5.1.4)と(5.1.5)の$d^3\mathbf{p}$を
\begin{align*}
d^3\mathbf{p}=\frac{d^3\mathbf{p}}{p^0}p^0 =\frac{d^3 \mathbf{p}_\Lambda}{(\Lambda p)^0} p^0
\end{align*}
で置き換えてよい.これら全てを考慮すると
\begin{align*}
&U_0(\Lambda,b) \psi^+_\ell(x)U_0^{-1}(\Lambda ,b) \\
=&\sum_{\sigma n}\int d^3 \mathbf{p} u_\ell (x;\mathbf{p},\sigma,n) U_0(\Lambda,b)a(\mathbf{p},\sigma,n)U_0^{-1}(\Lambda,b) \\
=&\sum_{\sigma n}\int d^3 \mathbf{p} u_\ell (x;\mathbf{p},\sigma,n)\exp\Bigl(i(\Lambda p)\cdot b\Bigr)\sqrt{\frac{(\Lambda p)^0}{p^0}}\sum_{\bar{\sigma}}D_{\sigma\bar{\sigma}}^{(j_n)}\Bigl( W^{-1}(\Lambda,p) \Bigr) a(\mathbf{p}_\Lambda,\bar{\sigma},n) \\
=&\sum_{\sigma n}\int \frac{d^3 \mathbf{p}_\Lambda}{(\Lambda p)^0} p^0 u_\ell (x;\mathbf{p},\sigma,n)\exp\Bigl(i(\Lambda p)\cdot b\Bigr)\sqrt{\frac{(\Lambda p)^0}{p^0}}\sum_{\bar{\sigma}}D_{\sigma\bar{\sigma}}^{(j_n)}\Bigl( W^{-1}(\Lambda,p) \Bigr) a(\mathbf{p}_\Lambda,\bar{\sigma},n) \\
=&\sum_{\sigma \bar{\sigma} n}\int d^3 \mathbf{p}_\Lambda u_\ell (x;\mathbf{p},\sigma,n)\exp\Bigl(i(\Lambda p)\cdot b\Bigr)D_{\sigma\bar{\sigma}}^{(j_n)}\Bigl( W^{-1}(\Lambda,p) \Bigr) \sqrt{\frac{p^0}{(\Lambda p)^0}} a(\mathbf{p}_\Lambda,\bar{\sigma},n)
\end{align*}
と
\begin{align*}
&U_0(\Lambda,b) \psi^-_\ell(x)U_0^{-1}(\Lambda ,b) \\
=&\sum_{\sigma n}\int d^3 \mathbf{p} v_\ell (x;\mathbf{p},\sigma,n) U_0(\Lambda,b)a^\dagger(\mathbf{p},\sigma,n)U_0^{-1}(\Lambda,b) \\
=&\sum_{\sigma n}\int d^3 \mathbf{p} v_\ell (x;\mathbf{p},\sigma,n)\exp\Bigl(-i(\Lambda p)\cdot b\Bigr)\sqrt{\frac{(\Lambda p)^0}{p^0}}\sum_{\bar{\sigma}}D_{\sigma\bar{\sigma}}^{(j_n)*}\Bigl( W^{-1}(\Lambda,p) \Bigr) a^\dagger(\mathbf{p}_\Lambda,\bar{\sigma},n) \\
=&\sum_{\sigma n}\int \frac{d^3 \mathbf{p}_\Lambda}{(\Lambda p)^0} p^0 v_\ell (x;\mathbf{p},\sigma,n)\exp\Bigl(-i(\Lambda p)\cdot b\Bigr)\sqrt{\frac{(\Lambda p)^0}{p^0}}\sum_{\bar{\sigma}}D_{\sigma\bar{\sigma}}^{(j_n)*}\Bigl( W^{-1}(\Lambda,p) \Bigr) a^\dagger(\mathbf{p}_\Lambda,\bar{\sigma},n) \\
=&\sum_{\sigma \bar{\sigma} n}\int d^3 \mathbf{p}_\Lambda v_\ell (x;\mathbf{p},\sigma,n)\exp\Bigl(-i(\Lambda p)\cdot b\Bigr)D_{\sigma\bar{\sigma}}^{(j_n)*}\Bigl( W^{-1}(\Lambda,p) \Bigr) \sqrt{\frac{p^0}{(\Lambda p)^0}} a^\dagger(\mathbf{p}_\Lambda,\bar{\sigma},n)
\end{align*}
となる.($U_0(\Lambda,b),U^{-1}_0(\Lambda,b)$はヒルベルト空間に作用する演算子であり,積分や係数関数は透過して,同様にヒルベルト空間に作用する演算子$a,a^\dagger$だけを挟むことに注意.)したがって,場がローレンツ変換則(5.1.6)(5.1.7)を満たすための必要十分条件が,まず(5.1.6)から
\begin{align*}
(\mathrm{LHS})=&U_0(\Lambda,b) \psi^+_\ell(x)U_0^{-1}(\Lambda ,b) \\
=&\sum_{\sigma n}\int d^3 \mathbf{p}_\Lambda \left[ \sqrt{\frac{p^0}{(\Lambda p)^0}} \sum_{\bar{\sigma}} u_\ell (x;\mathbf{p},\bar{\sigma},n)e^{i(\Lambda p)\cdot b}D_{\bar{\sigma}\sigma}^{(j_n)}\Bigl( W^{-1}(\Lambda,p) \Bigr) \right] a(\mathbf{p}_\Lambda,\sigma,n) \\
(\mathrm{RHS})=&\sum_{\bar{\ell}} D_{\ell\bar{\ell}}(\Lambda^{-1})\psi^+_{\bar{\ell}} (\Lambda x+b) \\
=&\sum_{\sigma\bar{\ell}n} D_{\ell\bar{\ell}}(\Lambda^{-1}) \int d^3\mathbf{p} u_{\bar{\ell}}(\Lambda x+b;\mathbf{p},\sigma,n)a(\mathbf{p},\sigma,n) \\
=&\sum_{\sigma n} \int d^3\mathbf{p}_\Lambda \left[\sum_{\bar{\ell}}D_{\ell\bar{\ell}}(\Lambda^{-1}) u_{\bar{\ell}}(\Lambda x+b;\mathbf{p}_\Lambda,\sigma,n)\right] a(\mathbf{p}_\Lambda,\sigma,n) \quad (変数変換 \mathbf{p}\to \mathbf{p}_\Lambda)\\
\therefore \quad & \sum_{\bar{\ell}}D_{\ell\bar{\ell}}(\Lambda^{-1}) u_{\bar{\ell}}(\Lambda x+b;\mathbf{p}_\Lambda,\sigma,n)=\sqrt{\frac{p^0}{(\Lambda p)^0}} \sum_{\bar{\sigma}} u_\ell (x;\mathbf{p},\bar{\sigma},n)e^{i(\Lambda p)\cdot b}D_{\bar{\sigma}\sigma}^{(j_n)}\Bigl( W^{-1}(\Lambda,p) \Bigr)
\end{align*}
がわかり,同様に(5.1.7)より
\begin{align*}
(\mathrm{LHS})=&U_0(\Lambda,b) \psi^-_\ell(x)U_0^{-1}(\Lambda ,b) \\
=&\sum_{\sigma n}\int d^3 \mathbf{p}_\Lambda \left[ \sqrt{\frac{p^0}{(\Lambda p)^0}} \sum_{\bar{\sigma}} v_\ell (x;\mathbf{p},\bar{\sigma},n)e^{-i(\Lambda p)\cdot b}D_{\bar{\sigma}\sigma}^{(j_n)*}\Bigl( W^{-1}(\Lambda,p) \Bigr) \right] a^\dagger (\mathbf{p}_\Lambda,\sigma,n) \\
(\mathrm{RHS})=&\sum_{\bar{\ell}} D_{\ell\bar{\ell}}(\Lambda^{-1})\psi^-_{\bar{\ell}} (\Lambda x+b) \\
=&\sum_{\sigma\bar{\ell}n} D_{\ell\bar{\ell}}(\Lambda^{-1}) \int d^3\mathbf{p} v_{\bar{\ell}}(\Lambda x+b;\mathbf{p},\sigma,n)a^\dagger (\mathbf{p},\sigma,n) \\
=&\sum_{\sigma n} \int d^3\mathbf{p}_\Lambda \left[\sum_{\bar{\ell}}D_{\ell\bar{\ell}}(\Lambda^{-1}) v_{\bar{\ell}}(\Lambda x+b;\mathbf{p}_\Lambda,\sigma,n)\right] a^\dagger(\mathbf{p}_\Lambda,\sigma,n) \quad (変数変換 \mathbf{p}\to \mathbf{p}_\Lambda)\\
\therefore \quad & \sum_{\bar{\ell}}D_{\ell\bar{\ell}}(\Lambda^{-1}) v_{\bar{\ell}}(\Lambda x+b;\mathbf{p}_\Lambda,\sigma,n)=\sqrt{\frac{p^0}{(\Lambda p)^0}} \sum_{\bar{\sigma}} v_\ell (x;\mathbf{p},\bar{\sigma},n)e^{-i(\Lambda p)\cdot b}D_{\bar{\sigma}\sigma}^{(j_n)*}\Bigl( W^{-1}(\Lambda,p) \Bigr)
\end{align*}
を得る.これをもう少し便利なように書き換える.両辺の左から行列$D(\Lambda)$をかけて,右から行列$D^{(j_n)}(W)$をかけることにより,まず第一式目は
\begin{align*}
(\mathrm{LHS})=&\sum_{\ell\sigma}\sum_{\bar{\ell}}D_{\ell'\ell}(\Lambda)D_{\ell\bar{\ell}}(\Lambda^{-1}) u_{\bar{\ell}}(\Lambda x+b;\mathbf{p}_\Lambda,\sigma,n)D_{\sigma\sigma'}^{(j_n)}\Bigl( W(\Lambda,p) \Bigr) \\
=&\sum_{\ell\sigma}\delta_{\ell'\bar{\ell}} u_{\bar{\ell}}(\Lambda x+b;\mathbf{p}_\Lambda,\sigma,n)D_{\sigma \sigma'}^{(j_n)}\Bigl( W(\Lambda,p) \Bigr) \\
=&\sum_{\sigma}u_{\ell'}(\Lambda x+b;\mathbf{p}_\Lambda,\sigma,n)D_{\sigma \sigma'}^{(j_n)}\Bigl( W(\Lambda,p) \Bigr) \\
(\mathrm{RHS})=&\sqrt{\frac{p^0}{(\Lambda p)^0}} \sum_{\ell\sigma} D_{\ell'\ell}(\Lambda) \sum_{\bar{\sigma}} u_\ell (x;\mathbf{p},\bar{\sigma},n)e^{i(\Lambda p)\cdot b}D_{\bar{\sigma}\sigma}^{(j_n)}\Bigl( W^{-1}(\Lambda,p) \Bigr)D_{\sigma\sigma'}^{(j_n)}\Bigl( W(\Lambda,p) \Bigr) \\
=&\sqrt{\frac{p^0}{(\Lambda p)^0}} \sum_{\ell} D_{\ell'\ell}(\Lambda) \sum_{\bar{\sigma}} u_\ell (x;\mathbf{p},\bar{\sigma},n)e^{i(\Lambda p)\cdot b}\delta_{\bar{\sigma}\sigma'} \\
=&\sqrt{\frac{p^0}{(\Lambda p)^0}} \sum_{\ell} D_{\ell'\ell}(\Lambda) u_\ell (x;\mathbf{p},\sigma',n)e^{i(\Lambda p)\cdot b} \\
\therefore \quad & \sum_{\sigma}u_{\bar{\ell}}(\Lambda x+b;\mathbf{p}_\Lambda,\sigma,n)D_{\sigma \bar{\sigma}}^{(j_n)}\Bigl( W(\Lambda,p) \Bigr)=\sqrt{\frac{p^0}{(\Lambda p)^0}} \sum_{\ell} D_{\bar{\ell}\ell}(\Lambda) u_\ell (x;\mathbf{p},\bar{\sigma},n)e^{i(\Lambda p)\cdot b}
\end{align*}
となる.次に両辺の左から行列$D(\Lambda)$をかけて,右から行列$D^{(j_n)*}(W)$をかけることにより,第二式目は
\begin{align*}
(\mathrm{LHS})=&\sum_{\ell\sigma}\sum_{\bar{\ell}}D_{\ell'\ell}(\Lambda)\sum_{\bar{\ell}}D_{\ell\bar{\ell}}(\Lambda^{-1}) v_{\bar{\ell}}(\Lambda x+b;\mathbf{p}_\Lambda,\sigma,n)D_{\sigma\sigma'}^{(j_n)*}\Bigl( W(\Lambda,p) \Bigr) \\
=&\sum_{\ell\sigma}\delta_{\ell'\bar{\ell}} v_{\bar{\ell}}(\Lambda x+b;\mathbf{p}_\Lambda,\sigma,n)D_{\sigma \sigma'}^{(j_n)*}\Bigl( W(\Lambda,p) \Bigr) \\
=&\sum_{\sigma}v_{\ell'}(\Lambda x+b;\mathbf{p}_\Lambda,\sigma,n)D_{\sigma \sigma'}^{(j_n)*}\Bigl( W(\Lambda,p) \Bigr) \\
(\mathrm{RHS})=&\sqrt{\frac{p^0}{(\Lambda p)^0}} \sum_{\ell\sigma} D_{\ell'\ell}(\Lambda) \sum_{\bar{\sigma}} v_\ell (x;\mathbf{p},\bar{\sigma},n)e^{-i(\Lambda p)\cdot b}D_{\bar{\sigma}\sigma}^{(j_n)*}\Bigl( W^{-1}(\Lambda,p) \Bigr)D_{\sigma\sigma'}^{(j_n)*}\Bigl( W(\Lambda,p) \Bigr) \\
=&\sqrt{\frac{p^0}{(\Lambda p)^0}} \sum_{\ell} D_{\ell'\ell}(\Lambda) \sum_{\bar{\sigma}} v_\ell (x;\mathbf{p},\bar{\sigma},n)e^{-i(\Lambda p)\cdot b}\delta_{\bar{\sigma}\sigma'} \\
=&\sqrt{\frac{p^0}{(\Lambda p)^0}} \sum_{\ell} D_{\ell'\ell}(\Lambda) v_\ell (x;\mathbf{p},\sigma',n)e^{-i(\Lambda p)\cdot b} \\
\therefore \quad & \sum_{\sigma}v_{\bar{\ell}}(\Lambda x+b;\mathbf{p}_\Lambda,\sigma,n)D_{\sigma \bar{\sigma}}^{(j_n)*}\Bigl( W(\Lambda,p) \Bigr)=\sqrt{\frac{p^0}{(\Lambda p)^0}} \sum_{\ell} D_{\bar{\ell}\ell}(\Lambda) v_\ell (x;\mathbf{p},\bar{\sigma},n)e^{-i(\Lambda p)\cdot b}
\end{align*}
を得る.これらは,係数関数$u_\ell$と$v_\ell$を有限個の自由なパラメータを使って表すことを可能にする基本的な条件である.\par
この最終結果の2式を三段階で用いて,固有順時ローレンツ変換の三つの異なる型を順番に考察する.


\vskip\baselineskip

\textbf{並進} \par
まず(5.1.13)と(5.1.14)で$\Lambda=1$とおけば,$W(1,p)=L^{-1}(1p)1L(p)=1$であるから
\begin{align*}
u_\ell(x+b;\mathbf{p},\sigma,n)=&u_\ell (x;\mathbf{p},\sigma,n)e^{i p\cdot b} \\
v_\ell(x+b;\mathbf{p},\sigma,n)=&v_\ell (x;\mathbf{p},\sigma,n)e^{-i p\cdot b}
\end{align*}
であり,したがって$x$依存性が
\begin{align*}
u_\ell(x;\mathbf{p},\sigma,n)=&e^{ip\cdot x}u_\ell(0;\mathbf{p},\sigma,n)=:\frac{1}{(2\pi)^{3/2}}e^{ip\cdot x}u_\ell(\mathbf{p},\sigma,n) \\
v_\ell(x;\mathbf{p},\sigma,n)=&e^{-ip\cdot x}u_\ell(0;\mathbf{p},\sigma,n)=:\frac{1}{(2\pi)^{3/2}}e^{-ip\cdot x}v_\ell(\mathbf{p},\sigma,n)
\end{align*}
の形であることがわかる.よって場(5.1.4)(5.1.5)は
\begin{align*}
\psi^+_\ell(x)=&\sum_{\sigma n}\frac{1}{(2\pi)^{3/2}}\int d^3\mathbf{p} u_\ell (\mathbf{p},\sigma,n)e^{ip\cdot x} a(\mathbf{p},\sigma,n) \\
\psi^-_\ell(x)=&\sum_{\sigma n}\frac{1}{(2\pi)^{3/2}}\int d^3\mathbf{p} v_\ell (\mathbf{p},\sigma,n)e^{-ip\cdot x} a^\dagger(\mathbf{p},\sigma,n)
\end{align*}
というフーリエ変換の形である\footnote{因子$(2\pi)^{-3/2}$は$u_\ell,v_\ell$の定義で吸収したり別の因子にできる.Peskinなどでは$(2\pi)^{-3}$になっている.}.(5.1.15)(5.1.16)を用いると,まず(5.1.13)は
\begin{align*}
(\mathrm{LHS})=&\sum_{\sigma}u_{\bar{\ell}}(\Lambda x+b;\mathbf{p}_\Lambda,\sigma,n)D_{\sigma \bar{\sigma}}^{(j_n)}\Bigl( W(\Lambda,p) \Bigr) \\
=&\frac{1}{(2\pi)^{3/2}}\sum_{\sigma}e^{i(\Lambda p)\cdot (\Lambda x+b)}u_{\bar{\ell}}(\mathbf{p}_\Lambda,\sigma,n)D_{\sigma \bar{\sigma}}^{(j_n)}\Bigl( W(\Lambda,p) \Bigr) \\
=&\frac{1}{(2\pi)^{3/2}}\sum_{\sigma}e^{ip\cdot x}e^{i(\Lambda p)\cdot b}u_{\bar{\ell}}(\mathbf{p}_\Lambda,\sigma,n)D_{\sigma \bar{\sigma}}^{(j_n)}\Bigl( W(\Lambda,p) \Bigr) \quad (\Lambda p)\cdot(\Lambda x)=p\cdot x \\
(\mathrm{RHS})=&\sqrt{\frac{p^0}{(\Lambda p)^0}} \sum_{\ell} D_{\bar{\ell}\ell}(\Lambda) u_\ell (x;\mathbf{p},\bar{\sigma},n)e^{i(\Lambda p)\cdot b} \\
=&\frac{1}{(2\pi)^{3/2}}\sqrt{\frac{p^0}{(\Lambda p)^0}} \sum_{\ell} D_{\bar{\ell}\ell}(\Lambda) u_\ell (\mathbf{p},\bar{\sigma},n)e^{ip\cdot x}e^{i(\Lambda p)\cdot b} \\
\therefore \quad &\sum_{\bar{\sigma}}u_{\bar{\ell}}(\mathbf{p}_\Lambda,\bar{\sigma},n)D_{\bar{\sigma}\sigma}^{(j_n)}\Bigl( W(\Lambda,p) \Bigr)=\sqrt{\frac{p^0}{(\Lambda p)^0}} \sum_{\ell} D_{\bar{\ell}\ell}(\Lambda) u_\ell (\mathbf{p},\sigma,n)
\end{align*}
と書けて,(5.1.14)は
\begin{align*}
(\mathrm{LHS})=&\sum_{\sigma}v_{\bar{\ell}}(\Lambda x+b;\mathbf{p}_\Lambda,\sigma,n)D_{\sigma \bar{\sigma}}^{(j_n)*}\Bigl( W(\Lambda,p) \Bigr) \\
=&\frac{1}{(2\pi)^{3/2}}\sum_{\sigma}e^{-i(\Lambda p)\cdot (\Lambda x+b)}v_{\bar{\ell}}(\mathbf{p}_\Lambda,\sigma,n)D_{\sigma \bar{\sigma}}^{(j_n)*}\Bigl( W(\Lambda,p) \Bigr) \\
=&\frac{1}{(2\pi)^{3/2}}\sum_{\sigma}e^{-ip\cdot x}e^{-i(\Lambda p)\cdot b}v_{\bar{\ell}}(\mathbf{p}_\Lambda,\sigma,n)D_{\sigma \bar{\sigma}}^{(j_n)*}\Bigl( W(\Lambda,p) \Bigr) \quad (\Lambda p)\cdot(\Lambda x)=p\cdot x \\
(\mathrm{RHS})=&\sqrt{\frac{p^0}{(\Lambda p)^0}} \sum_{\ell} D_{\bar{\ell}\ell}(\Lambda) v_\ell (x;\mathbf{p},\bar{\sigma},n)e^{-i(\Lambda p)\cdot b} \\
=&\frac{1}{(2\pi)^{3/2}}\sqrt{\frac{p^0}{(\Lambda p)^0}} \sum_{\ell} D_{\bar{\ell}\ell}(\Lambda) v_\ell (\mathbf{p},\bar{\sigma},n)e^{-ip\cdot x}e^{-i(\Lambda p)\cdot b} \\
\therefore \quad &\sum_{\bar{\sigma}}v_{\bar{\ell}}(\mathbf{p}_\Lambda,\bar{\sigma},n)D_{\bar{\sigma} \sigma}^{(j_n)*}\Bigl( W(\Lambda,p) \Bigr)=\sqrt{\frac{p^0}{(\Lambda p)^0}} \sum_{\ell} D_{\bar{\ell}\ell}(\Lambda) v_\ell (\mathbf{p},\sigma,n)
\end{align*}
と書きなおせる.

\vskip\baselineskip

\textbf{ブースト}\par
次に,(5.1.19)(5.1.20)で$\mathbf{p}=0$とおき(つまり$p$を基準運動量$p=(0,0,0,m)$とする),$\Lambda$を質量$m$の粒子を静止状態からある4元運動量$q^\mu$の状態に移す基準ブースト$\Lambda=L(q):p\to q,\Lambda p=q$ととる.((2.5.24)参照)すると$L(p)=1$であり
\begin{align*}
W(\Lambda,p):=&L^{-1}(\Lambda p)\Lambda L(p) \\
=&L^{-1}(L(q)p) L(q) 1 \\
=&L^{-1}(q)L(q)=1
\end{align*}
である.よってこの特別な場合には(5.1.19)は
\begin{align*}
(\mathrm{LHS})=&\sum_{\bar{\sigma}}u_{\bar{\ell}}(\mathbf{p}_\Lambda,\sigma,n)D_{\bar{\sigma} \sigma}^{(j_n)}\Bigl( W(\Lambda,p) \Bigr) \\
=&u_{\bar{\ell}}(\mathbf{q},\sigma,n) \\
(\mathrm{RHS})=&\sqrt{\frac{p^0}{(\Lambda p)^0}} \sum_{\ell} D_{\bar{\ell}\ell}(\Lambda) u_\ell (\mathbf{p},\sigma,n) \\
=&\sqrt{\frac{m}{q^0}} \sum_{\ell} D_{\bar{\ell}\ell}(L(q)) u_\ell (0,\sigma,n) \\
\therefore \quad & u_{\bar{\ell}}(\mathbf{q},\sigma,n)=\sqrt{\frac{m}{q^0}} \sum_{\ell} D_{\bar{\ell}\ell}(L(q)) u_\ell (0,\sigma,n)
\end{align*}
を与え,また(5.1.20)は
\begin{align*}
(\mathrm{LHS})=&\sum_{\bar{\sigma}}v_{\bar{\ell}}(\mathbf{p}_\Lambda,\bar{\sigma},n)D_{\bar{\sigma} \sigma}^{(j_n)*}\Bigl( W(\Lambda,p) \Bigr) \\
=&v_{\bar{\ell}}(\mathbf{q},\sigma,n) \\
(\mathrm{RHS})=&\sqrt{\frac{p^0}{(\Lambda p)^0}} \sum_{\ell} D_{\bar{\ell}\ell}(\Lambda) v_\ell (\mathbf{p},\sigma,n) \\
=&\sqrt{\frac{m}{q^0}} \sum_{\ell} D_{\bar{\ell}\ell}(L(q)) v_\ell (0,\sigma,n) \\
\therefore \quad & v_{\bar{\ell}}(\mathbf{q},\sigma,n)=\sqrt{\frac{m}{q^0}} \sum_{\ell} D_{\bar{\ell}\ell}(L(q)) v_\ell (0,\sigma,n)
\end{align*}
を与える.言い換えれば,基準運動量(ゼロ運動量)での量$u_\ell(0,\sigma,n),v_\ell(0,\sigma,n)$が分かり,次に斉次ローレンツ群の与えられた表現$D(\Lambda)$が与えられれば,(2.5.24)の表現$D(L(q))$を用いて全ての$\mathbf{p}$についての関数$u_\ell(\mathbf{p},\sigma,n),v_\ell(\mathbf{p},\sigma,n)$がわかる.(斉次ローレンツ群の任意の表現に対する行列$D_{\bar{\ell}\ell}(L(q))$の陽な式は5.7節で与える.)

\vskip\baselineskip


\textbf{回転}\par
次に,再び$\mathbf{p}=0$ととるが,今度は$\Lambda$が$\mathbf{p}_\Lambda=0$を保つローレンツ変換とする.すなわち回転$\Lambda=R\in SO(3)$ととる.(p96で示したように)$W(R,p)=R$であり,よって(5.1.19)は
\begin{align*}
(\mathrm{LHS})=&\sum_{\bar{\sigma}}u_{\bar{\ell}}(0,\bar{\sigma},n)D_{\bar{\sigma}\sigma}^{(j_n)}\Bigl( W(R,p) \Bigr) \\
=&\sum_{\bar{\sigma}}u_{\bar{\ell}}(0,\bar{\sigma},n)D_{\bar{\sigma}\sigma}^{(j_n)}(R) \\
(\mathrm{RHS})=&\sqrt{\frac{p^0}{(R p)^0}} \sum_{\ell} D_{\bar{\ell}\ell}(R) u_\ell (0,\sigma,n) \\
=&\sum_{\ell} D_{\bar{\ell}\ell}(R) u_\ell (0,\sigma,n) \quad \because (Rp)^0=p^0=m\\
\therefore \quad & \sum_{\bar{\sigma}}u_{\bar{\ell}}(0,\bar{\sigma},n)D_{\bar{\sigma}\sigma}^{(j_n)}( R )=\sum_{\ell} D_{\bar{\ell}\ell}(R) u_\ell (0,\sigma,n)
\end{align*}
となり,(5.1.20)は
\begin{align*}
(\mathrm{LHS})=&\sum_{\bar{\sigma}}v_{\bar{\ell}}(0,\bar{\sigma},n)D_{\bar{\sigma}\sigma}^{(j_n)*}\Bigl( W(R,p) \Bigr) \\
=&\sum_{\bar{\sigma}}v_{\bar{\ell}}(0,\bar{\sigma},n)D_{\bar{\sigma}\sigma}^{(j_n)*}(R) \\
(\mathrm{RHS})=&\sqrt{\frac{p^0}{(R p)^0}} \sum_{\ell} D_{\bar{\ell}\ell}(R) v_\ell (0,\sigma,n) \\
=&\sum_{\ell} D_{\bar{\ell}\ell}(R) v_\ell (0,\sigma,n) \quad \because (Rp)^0=p^0=m\\
\therefore \quad & \sum_{\bar{\sigma}}v_{\bar{\ell}}(0,\bar{\sigma},n)D_{\bar{\sigma}\sigma}^{(j_n)*}( R )=\sum_{\ell} D_{\bar{\ell}\ell}(R) v_\ell (0,\sigma,n)
\end{align*}
となる.(2.5.20)より$R=1+\Theta$と微小展開すると
\begin{align*}
D^{(j_n)}_{\bar{\sigma}\sigma}(R)=&\delta_{\bar{\sigma}\sigma}+\frac{i}{2}\Theta_{ik}(J_{ik}^{(j_n)})_{\bar{\sigma}\sigma} \\
=&\delta_{\bar{\sigma}\sigma}+i\theta_i(J_{i}^{(j_n)})_{\bar{\sigma}\sigma} \quad \because \theta_i:=\frac{1}{2}\epsilon_{ijk}\Theta_{jk},\quad J^{(j_n)}_i:=\frac{1}{2}\epsilon_{ijk}J^{(j_n)}_{jk}
\end{align*}
と書けて,さらに$D(R)$も回転群の表現だから
\begin{align*}
D_{\bar{\ell}\ell}(R)=&\delta_{\bar{\ell}\ell}+\frac{i}{2}\Theta_{ik}(\mc{J}_{ik})_{\bar{\ell}\ell} \\
=&\delta_{\bar{\sigma}\sigma}+i\theta_i(\mc{J}_i)_{\bar{\ell}\ell} \quad \because \mc{J}_i:=\frac{1}{2}\epsilon_{ijk}\mc{J}_{jk}
\end{align*}
と微小展開できる.したがって$\theta_i$の係数を比較すれば,(5.1.25)は
\begin{align*}
(\mathrm{LHS})=&u_{\bar{\ell}}(0,\sigma,n)+i\theta_i\sum_{\bar{\sigma}}u_{\bar{\ell}}(0,\bar{\sigma},n)(J_i^{(j_n)})_{\bar{\sigma}\sigma} \\
(\mathrm{RHS})=&u_{\bar{\ell}}(0,\sigma,n)+i\theta_i \sum_{\ell}(\mc{J}_i)_{\bar{\ell}\ell}u_{\ell}(0,\sigma,n) \\
\therefore \quad & \sum_{\bar{\sigma}}u_{\bar{\ell}}(0,\bar{\sigma},n)\mathbf{J}^{(j_n)}_{\bar{\sigma}\sigma}=\sum_{\ell}\bm{\mc{J}}_{\bar{\ell}\ell}u_{\ell}(0,\sigma,n)
\end{align*}
と書けて,(5.1.26)も
\begin{align*}
(\mathrm{LHS})=&v_{\bar{\ell}}(0,\sigma,n)-i\theta_i\sum_{\bar{\sigma}}v_{\bar{\ell}}(0,\bar{\sigma},n)(J_i^{(j_n)*})_{\bar{\sigma}\sigma} \\
(\mathrm{RHS})=&u_{\bar{\ell}}(0,\sigma,n)+i\theta_i \sum_{\ell}(\mc{J}_i)_{\bar{\ell}\ell}v_{\ell}(0,\sigma,n) \\
\therefore \quad & \sum_{\bar{\sigma}}v_{\bar{\ell}}(0,\bar{\sigma},n)\mathbf{J}^{(j_n)*}_{\bar{\sigma}\sigma}=-\sum_{\ell}\bm{\mc{J}}_{\bar{\ell}\ell}v_{\ell}(0,\sigma,n)
\end{align*}
ここで$\mathbf{J}^{j}$と$\bm{\mc{J}}$はそれぞれ$D^{(j)}(R)$と$D(R)$における角運動量行列である.斉次ローレンツ群の\uwave{任意の}表現$D(\Lambda)$は,$\Lambda\in SO(3,1)$を回転$R\in SO(3)$に限れば明らかに回転群の表現を与える.よって(5.1.25)と(5.1.26)より以下で示されるように,場$\psi^\pm_{\ell}(x)$がある特定のスピン$j$の粒子を記述するとき,この(可約)表現$D(R)$はその既約成分のなかにスピン$j$表現$D^{(j)}(R)$を含まねばならない.

\vskip\baselineskip


わかりやすいように,$u_{\ell}(0,\sigma,n)$を成分$(\ell,\sigma)$をもつ行列$U_{\ell\sigma}$と書く.このとき,条件(5.1.23)((5.1.25)と同値)は(今は不要な添え字$n$を省略して)
\begin{align*}
\sum_{\bar{\sigma}}U_{\bar{\ell}\bar{\sigma}}D^{(j)}_{\bar{\sigma}\sigma}(R)=&\sum_{\ell}D_{\bar{\ell} \ell}(R) U_{\ell \sigma} \\
\therefore \quad UD^{(j)}(R)=&D(R)U
\end{align*}
となる.示すべき命題は次の通りである.ある非ゼロの線形写像(の表現行列)
\begin{align*}
U:V_j \to V_D
\end{align*}
が存在し,これが任意の回転$R\in SO(3)$について$UD^{(j)}(R)=D(R)U$を満たすとする.ここで$V_j$は回転群$R\in SO(3)$の,スピン$j$の表現空間$(\mathrm{dim}V_j=2j+1)$,$V_D$は場がもつ有限次元表現空間である.このとき,$V_D$は,$V_j$と同型な不変部分空間をもつ.すなわち$D$は$D^{(j)}$を既約成分として含む.\par
これを証明しよう.まず$\mathrm{Ker}U$と像$\mathrm{Im}U$はそれぞれ$V_j$と$V_D$の部分空間であることは明らかである.さらに与えられた関係式から,任意の$v\in \mathrm{Ker}U$と任意の$R\in SO(3)$について
\begin{align*}
U(D^{(j)}(R)v)=D(R)(U v)=D(R)0=0
\end{align*}
よって$D^{(j)}(R) v\in \mathrm{Ker}U$,したがって$\mathrm{Ker}U$は$D^{(j)}$に対する不変部分空間であることがわかる.同様に,任意の$w\in \mathrm{Im}U$と任意の$R\in SO(3)$について,$w=U(v)$と書けるから
\begin{align*}
D(R)w=D(R)U(v)=U(D^{(j)}(R)v) \in \mathrm{Im}U
\end{align*}
よって$\mathrm{Im}U$は$D$に対する不変部分空間であることがわかる.さて,$V_j$は既約表現空間であるから,既約の定義より不変部分空間は零部分空間$\{0\}$かあるいは$V_j$自身のどちらかでしかない.したがって,その不変部分空間$\mathrm{Ker}U \subset V_j$は$\{0\}$か$V_j$のどちらかである.しかし仮定より$U$は非ゼロであるから,$\mathrm{Ker}U\neq V_j$であり,よって$\mathrm{Ker}U=\{0\}$がわかる.すなわち$U$は単射である.単射であることから,$U:V_j\to V_D$はその像$\mathrm{Im}U \subset V_D$に全単射である.さらに$\mathrm{Im}U$は$V_D$の不変部分空間である.したがって$V_j$と$\mathrm{Im}U$は$D^{(j)}$と$D$の表現として同型であり,その間の同型写像は$U$である.言い換えれば,表現$D$は既約成分として$D^{(j)}$を含む\footnote{もはや自明だと思うが,$\mathrm{Im}U$の不変部分空間が自分自身のみであることを示す.$U:V_j\to \mathrm{Im}U$は全単射だから,逆写像$U^{-1}:\mathrm{Im}U\to V_j$を定義でき,これを用いると上での関係式から$D^{(j)}(R)U^{-1}=U^{-1}D(R)$がなりたつ.仮定より存在する$\mathrm{Im}U$の$D$-不変部分空間を$W$とし,$U^{-1}(W)\in V_j$を考える.$v\in U^{-1}(W)$は$w \in W$を用いて$v=U^{-1}(w)$と書けて,
\begin{align*}
D^{(j)}(R)v=D^{(j)}(R)U^{-1}(w)=U^{-1}(D(R)w) \in U^{-1} (W)
\end{align*}
を満たす.ここで$W$が$D$-不変部分空間であることから$D(R)w\in W$であることを用いた.よって$U^{-1}(W)$は$V_j$の不変部分空間である.$W$が$\mathrm{Im}U$の非自明かつ真部分空間$\{0\} \neq W \subsetneq \mathrm{Im}U$であると仮定すると,$U^{-1}$は同型写像であり次元を保つから$U^{-1}(W)$も$V_j$の非自明かつ真部分空間$\{0\} \neq U^{-1}(W) \subsetneq V_j$となる.しかしこれは$V_j$が既約であることに矛盾する.したがって$\mathrm{Im}U$は非自明な不変部分空間を持ちえず,つまり$\mathrm{Im}U$は既約である.$D(R)$の$D^{(j)}$成分は既約成分になっている.}.$V_{\ell\sigma}:=v_\ell(0,\sigma,n)$についても同様の議論ができる.これで証明が完了した.

\vskip\baselineskip

上の証明から明らかなように,$U_{\ell\sigma}=u_\ell(0,\sigma,n),V_{\ell\sigma}=v_\ell(0,\sigma,n)$は単に回転群のスピン$j$表現が$D(R)$の中にどのように埋め込まれるかを記述する.固有順時ローレンツ群の各々の\uwave{既約}表現は,回転群の任意の与えられた既約表現をせいぜい一回だけ含む.実際,Schurの補題を用いると,$D$が$SO(3)$の既約表現であり$U$が$D(R)U=UD^{(j)}(R)$を満たすならば,$V_j \ncong V_D$のとき$U=0$であり,$V_j \cong V_D$のとき同型写像$U:V_j\to V_D$は全体のスケールを除いて一意に決まる(つまり別の同型写像$U':V_j\to V_D$が存在しても$U'=\lambda U$となる.).$V_{\ell\sigma}$についても同様である.よって,もし場$\psi_\ell^+(x)$と$\psi_\ell^-(x)$が既約的に変換するなら,それは全体のスケールを除いて一意的である.より一般的には,消滅場と生成場の自由なパラメータの数は(全体のスケールを\uwave{含めて})場に含まれる既約表現の数に等しい.

\vskip\baselineskip


(5.1.23)と(5.1.24)を満たす$u_\ell(0,\sigma,n)$と$v_\ell(0,\sigma,n)$を用いて(5.1.21)と(5.1.22)で定義される係数関数$u_\ell(\mathbf{p},\sigma,n)$と$v_\ell(\mathbf{p},\sigma,n)$が,一般的な条件(5.1.22)(5.1.20)を満たすことを示すことができる.実際(5.1.22)は
\begin{align*}
(\mathrm{LHS})=&\sum_{\bar{\sigma}}u_{\bar{\ell}}(\mathbf{p}_\Lambda,\bar{\sigma},n)D_{\bar{\sigma}\sigma}^{(j_n)}\Bigl( W(\Lambda,p) \Bigr) \\
=&\sqrt{\frac{m}{(\Lambda p)^0}}\sum_{\ell\bar{\sigma}}D_{\bar{\ell}\ell}(L(\Lambda p))u_{\ell}(0,\bar{\sigma},n)D_{\bar{\sigma}\sigma}^{(j_n)}\Bigl( W(\Lambda,p) \Bigr) \\
=&\sqrt{\frac{m}{(\Lambda p)^0}}\sum_{\ell\bar{\ell}}D_{\bar{\ell}\ell}(L(\Lambda p))D_{\ell\ell'}(W(\Lambda,p))u_{\ell'}(0,\sigma,n) \quad \because (5.1.23),\quad W(\Lambda,p)\in SO(3)\\
=&\sqrt{\frac{m}{(\Lambda p)^0}}\sum_{\ell\bar{\ell}}D_{\bar{\ell}\ell}(L(\Lambda p))D_{\ell\ell'}(L^{-1}(\Lambda p)\Lambda L(p))u_{\ell'}(0,\sigma,n) \\
=&\sqrt{\frac{m}{(\Lambda p)^0}}\sum_{\ell\bar{\ell}}D_{\bar{\ell}\ell'}(\Lambda L(p))u_{\ell'}(0,\sigma,n) \\
=&\sqrt{\frac{p^0}{(\Lambda p)^0}}\sqrt{\frac{m}{p^0}}\sum_{\ell\bar{\ell}}D_{\bar{\ell}\ell}(\Lambda)D_{\ell\ell'}(L(p))u_{\ell'}(0,\sigma,n) \\
=&\sqrt{\frac{p^0}{(\Lambda p)^0}} \sum_{\ell} D_{\bar{\ell}\ell}(\Lambda) u_\ell (\mathbf{p},\sigma,n)=(\mathrm{RHS})
\end{align*}
で,(5.1.23)は
\begin{align*}
(\mathrm{LHS})=&\sum_{\bar{\sigma}}v_{\bar{\ell}}(\mathbf{p}_\Lambda,\bar{\sigma},n)D_{\bar{\sigma}\sigma}^{(j_n)*}\Bigl( W(\Lambda,p) \Bigr) \\
=&\sqrt{\frac{m}{(\Lambda p)^0}}\sum_{\ell\bar{\sigma}}D_{\bar{\ell}\ell}(L(\Lambda p))v_{\ell}(0,\bar{\sigma},n)D_{\bar{\sigma}\sigma}^{(j_n)*}\Bigl( W(\Lambda,p) \Bigr) \\
=&\sqrt{\frac{m}{(\Lambda p)^0}}\sum_{\ell\bar{\ell}}D_{\bar{\ell}\ell}(L(\Lambda p))D_{\ell\ell'}(W(\Lambda,p))v_{\ell'}(0,\sigma,n) \quad \because (5.1.23),\quad W(\Lambda,p)\in SO(3)\\
=&\sqrt{\frac{m}{(\Lambda p)^0}}\sum_{\ell\bar{\ell}}D_{\bar{\ell}\ell}(L(\Lambda p))D_{\ell\ell'}(L^{-1}(\Lambda p)\Lambda L(p))v_{\ell'}(0,\sigma,n) \\
=&\sqrt{\frac{m}{(\Lambda p)^0}}\sum_{\ell\bar{\ell}}D_{\bar{\ell}\ell'}(\Lambda L(p))v_{\ell'}(0,\sigma,n) \\
=&\sqrt{\frac{p^0}{(\Lambda p)^0}}\sqrt{\frac{m}{p^0}}\sum_{\ell\bar{\ell}}D_{\bar{\ell}\ell}(\Lambda)D_{\ell\ell'}(L(p))v_{\ell'}(0,\sigma,n) \\
=&\sqrt{\frac{p^0}{(\Lambda p)^0}} \sum_{\ell} D_{\bar{\ell}\ell}(\Lambda) v_\ell (\mathbf{p},\sigma,n)=(\mathrm{RHS})
\end{align*}
で示される.


\vskip\baselineskip


クラスター分解の原理に戻る.(5.1.17)(5.1.18)を(5.1.9)に代入すると
\begin{align*}
\mc{H}_I(x)=&\sum_{NM}\sum_{\ell_1'\cdots \ell_N'}\sum_{\ell_1\cdots \ell_M}g_{\ell_1' \cdots \ell'_N,\ell_1 \cdots \ell_M} \psi^-_{\ell_1'}(x)\cdots \psi^-_{\ell_N'}(x) \psi^+_{\ell_1}(x) \cdots \psi^+_{\ell_M}(x) \\
=&\sum_{NM}\sum_{\ell_1'\cdots \ell_N'}\sum_{\ell_1\cdots \ell_M}g_{\ell_1' \cdots \ell'_N,\ell_1 \cdots \ell_M} \\
&\times \sum_{\sigma_1'\cdots \sigma_N'}\sum_{n_1'\cdots n_N'}\int \frac{d^3\mathbf{p}'_1}{(2\pi)^{3/2}} \cdots \frac{d^3\mathbf{p}'_N}{(2\pi)^{3/2}} v_{\ell'_1}(\mathbf{p}'_1,\sigma'_1,n'_1) \cdots v_{\ell'_N}(\mathbf{p}'_N,\sigma'_N,n'_N) \\
&\qquad\qquad\qquad\qquad\qquad \times a^\dagger(\mathbf{p}_1',\sigma_1',n_1')\cdots a^\dagger(\mathbf{p}_N') \\
&\times \sum_{\sigma_1\cdots \sigma_M}\sum_{n_1 \cdots n_M}\int \frac{d^3\mathbf{p}_1}{(2\pi)^{3/2}} \cdots \frac{d^3\mathbf{p}_M}{(2\pi)^{3/2}} u_{\ell_1}(\mathbf{p}_1,\sigma_1,n_1) \cdots u_{\ell_M}(\mathbf{p}_M,\sigma_M,n_M) \\
&\qquad\qquad\qquad\qquad\qquad \times a(\mathbf{p}_1,\sigma_1,n_1)\cdots a(\mathbf{p}_M,\sigma_M,n_M) \\
&\times e^{i(p_1+\cdots +p_M-p'_1-\cdots -p'_N)\cdot x} \\
=&\sum_{NM}\sum_{\ell_1'\cdots \ell_N'}\sum_{\ell_1\cdots \ell_M}g_{\ell_1' \cdots \ell'_N,\ell_1 \cdots \ell_M} \\
&\times \sum_{\sigma_1'\cdots \sigma_N'}\sum_{n_1'\cdots n_N'}\int \frac{d^3\mathbf{p}'_1}{(2\pi)^{3/2}} \cdots \frac{d^3\mathbf{p}'_N}{(2\pi)^{3/2}} v_{\ell'_1}(\mathbf{p}'_1,\sigma'_1,n'_1) \cdots v_{\ell'_N}(\mathbf{p}'_N,\sigma'_N,n'_N) \\
&\qquad\qquad\qquad\qquad\qquad \times a^\dagger(\mathbf{p}_1',\sigma_1',n_1')\cdots a^\dagger(\mathbf{p}_N') \\
&\times \sum_{\sigma_1\cdots \sigma_M}\sum_{n_1 \cdots n_M}\int \frac{d^3\mathbf{p}_1}{(2\pi)^{3/2}} \cdots \frac{d^3\mathbf{p}_M}{(2\pi)^{3/2}} u_{\ell_1}(\mathbf{p}_1,\sigma_1,n_1) \cdots u_{\ell_M}(\mathbf{p}_M,\sigma_M,n_M) \\
&\qquad\qquad\qquad\qquad\qquad \times a(\mathbf{p}_1,\sigma_1,n_1)\cdots a(\mathbf{p}_M,\sigma_M,n_M) \\
&\times e^{i(\mathbf{p}_1+\cdots +\mathbf{p}_M-\mathbf{p}'_1-\cdots -\mathbf{p}'_N)\cdot \mathbf{x}}e^{-i(p_1+\cdots +p_M-p'_1-\cdots -p'_N)^0 t}
\end{align*}
$\mathbf{x}$について積分すると
\begin{align*}
V_I(t)=&\sum_{NM}\sum_{\ell_1'\cdots \ell_N'}\sum_{\ell_1\cdots \ell_M}g_{\ell_1' \cdots \ell'_N,\ell_1 \cdots \ell_M} \\
&\times \sum_{\sigma_1'\cdots \sigma_N'}\sum_{n_1'\cdots n_N'}\int \frac{d^3\mathbf{p}'_1}{(2\pi)^{3/2}} \cdots \frac{d^3\mathbf{p}'_N}{(2\pi)^{3/2}} v_{\ell'_1}(\mathbf{p}'_1,\sigma'_1,n'_1) \cdots v_{\ell'_N}(\mathbf{p}'_N,\sigma'_N,n'_N) \\
&\qquad\qquad\qquad\qquad\qquad \times a^\dagger(\mathbf{p}_1',\sigma_1',n_1')\cdots a^\dagger(\mathbf{p}_N') \\
&\times \sum_{\sigma_1\cdots \sigma_M}\sum_{n_1 \cdots n_M}\int \frac{d^3\mathbf{p}_1}{(2\pi)^{3/2}} \cdots \frac{d^3\mathbf{p}_M}{(2\pi)^{3/2}} u_{\ell_1}(\mathbf{p}_1,\sigma_1,n_1) \cdots u_{\ell_M}(\mathbf{p}_M,\sigma_M,n_M) \\
&\qquad\qquad\qquad\qquad\qquad \times a(\mathbf{p}_1,\sigma_1,n_1)\cdots a(\mathbf{p}_M,\sigma_M,n_M) \\
&\times e^{-i(p_1+\cdots +p_M-p'_1-\cdots -p'_N)^0 t} \\
&\times (2\pi)^3\delta^3(\mathbf{p}_1+\cdots +\mathbf{p}_M-\mathbf{p}'_1-\cdots -\mathbf{p}'_N)
\end{align*}
$t=0$とおくと(3.5.5)より$V_I(0)=V$であるから,
\begin{align*}
V=&\sum_{NM}\sum_{\ell_1'\cdots \ell_N'}\sum_{\ell_1\cdots \ell_M}g_{\ell_1' \cdots \ell'_N,\ell_1 \cdots \ell_M} \\
&\times \sum_{\sigma_1'\cdots \sigma_N'}\sum_{n_1'\cdots n_N'}\int \frac{d^3\mathbf{p}'_1}{(2\pi)^{3/2}} \cdots \frac{d^3\mathbf{p}'_N}{(2\pi)^{3/2}} v_{\ell'_1}(\mathbf{p}'_1,\sigma'_1,n'_1) \cdots v_{\ell'_N}(\mathbf{p}'_N,\sigma'_N,n'_N) \\
&\qquad\qquad\qquad\qquad\qquad \times a^\dagger(\mathbf{p}_1',\sigma_1',n_1')\cdots a^\dagger(\mathbf{p}_N') \\
&\times \sum_{\sigma_1\cdots \sigma_M}\sum_{n_1 \cdots n_M}\int \frac{d^3\mathbf{p}_1}{(2\pi)^{3/2}} \cdots \frac{d^3\mathbf{p}_M}{(2\pi)^{3/2}} u_{\ell_1}(\mathbf{p}_1,\sigma_1,n_1) \cdots u_{\ell_M}(\mathbf{p}_M,\sigma_M,n_M) \\
&\qquad\qquad\qquad\qquad\qquad \times a(\mathbf{p}_1,\sigma_1,n_1)\cdots a(\mathbf{p}_M,\sigma_M,n_M) \\
&\times (2\pi)^3\delta^3(\mathbf{p}_1+\cdots +\mathbf{p}_M-\mathbf{p}'_1-\cdots -\mathbf{p}'_N) \\
=&\sum_{NM} \int d^3\mathbf{p}'_1 \cdots d^3\mathbf{p}'_N d^3\mathbf{p}_1 \cdots d^3\mathbf{p}_M\sum_{\sigma_1'\cdots \sigma_N'}\sum_{n_1'\cdots n_N'} \sum_{\sigma_1\cdots \sigma_M}\sum_{n_1 \cdots n_M} \\
&\times a^\dagger(\mathbf{p}_1',\sigma_1',n_1')\cdots a^\dagger(\mathbf{p}_N') a(\mathbf{p}_1,\sigma_1,n_1)\cdots a(\mathbf{p}_M,\sigma_M,n_M) \\
&\times (2\pi)^{3-3N/2-3M/2} \delta^3(\mathbf{p}_1+\cdots +\mathbf{p}_M-\mathbf{p}'_1-\cdots -\mathbf{p}'_N) \sum_{\ell_1'\cdots \ell_N'} g_{\ell_1' \cdots \ell'_N,\ell_1 \cdots \ell_M} \\
&\times v_{\ell'_1}(\mathbf{p}'_1,\sigma'_1,n'_1) \cdots v_{\ell'_N}(\mathbf{p}'_N,\sigma'_N,n'_N) u_{\ell_1}(\mathbf{p}_1,\sigma_1,n_1) \cdots u_{\ell_M}(\mathbf{p}_M,\sigma_M,n_M) \\
=&\sum_{NM} \int d^3\mathbf{p}'_1 \cdots d^3\mathbf{p}'_N d^3\mathbf{p}_1 \cdots d^3\mathbf{p}_M\sum_{\sigma_1'\cdots \sigma_N'}\sum_{n_1'\cdots n_N'} \sum_{\sigma_1\cdots \sigma_M}\sum_{n_1 \cdots n_M} \\
&\times a^\dagger(\mathbf{p}_1',\sigma_1',n_1')\cdots a^\dagger(\mathbf{p}_N') a(\mathbf{p}_1,\sigma_1,n_1)\cdots a(\mathbf{p}_M,\sigma_M,n_M) \\
&\times \mc{V}_{NM}(\mathbf{p}_1'\sigma_1'n_1',\cdots,\mathbf{p}'_N\sigma_N' n'_N;\mathbf{p}_1 \sigma_1 n_1 ,\cdots,\mathbf{p}_M\sigma_M n_M)
\end{align*}
という形の相互作用になる.ここで係数関数$\mc{V}_{NM}$は
\begin{align*}
&\mc{V}_{NM}(\mathbf{p}_1'\sigma_1'n_1',\cdots,\mathbf{p}'_N\sigma_N' n'_N;\mathbf{p}_1 \sigma_1 n_1 ,\cdots,\mathbf{p}_M\sigma_M n_M) \\
=&\delta^3(\mathbf{p}_1+\cdots +\mathbf{p}_M-\mathbf{p}'_1-\cdots -\mathbf{p}'_N)\tilde{\mc{V}}_{NM}(\mathbf{p}_1'\sigma_1'n_1',\cdots,\mathbf{p}'_N\sigma_N' n'_N;\mathbf{p}_1 \sigma_1 n_1 ,\cdots,\mathbf{p}_M\sigma_M n_M)
\end{align*}
で,
\begin{align*}
&\tilde{\mc{V}}_{NM}(\mathbf{p}_1'\sigma_1'n_1',\cdots,\mathbf{p}'_N\sigma_N' n'_N;\mathbf{p}_1 \sigma_1 n_1 ,\cdots,\mathbf{p}_M\sigma_M n_M)\\
&:=\sum_{\ell_1'\cdots \ell_N'} g_{\ell_1' \cdots \ell'_N,\ell_1 \cdots \ell_M} \\
&\times v_{\ell'_1}(\mathbf{p}'_1,\sigma'_1,n'_1) \cdots v_{\ell'_N}(\mathbf{p}'_N,\sigma'_N,n'_N) u_{\ell_1}(\mathbf{p}_1,\sigma_1,n_1) \cdots u_{\ell_M}(\mathbf{p}_M,\sigma_M,n_M)
\end{align*}
この相互作用$V$は明らかに,$S$行列がクラスター分解の原理を満たすことを保証する形をしている.すなわち,$\mc{V}_{NM}$はただ一つのデルタ関数の因子をもつ.その係数$\tilde{\mc{V}}_{NM}$は,(少なくとも有限個($N+M$個)の場のタイプについては)ゼロ運動量でせいぜいbranch cutの特異性のみをもつ.\par
このように,クラスター分解原理とローレンツ不変性を合わせると,相互作用密度は消滅場と生成場から構成されねばならないという主張が自然なものになる.


\vskip\baselineskip


もしクラスター分解原理を満たすスカラー密度を構成すること\uwave{だけ}が必要なら,不変性の条件(5.1.10)(および適当な実条件)のみを満たす結合定数$g_{\ell_1' \cdots \ell'_N,\ell_1 \cdots \ell_M}$を使って,消滅演算子と生成演算子を組み合わせて任意の多項式(5.1.9)を作ればよい.しかし,$S$行列のローレンツ不変性のためには,相互作用密度$\mc{H}_I(x)$は交換関係(5.1.3)も満たさなければならない.この条件は生成・消滅演算子の任意の関数によって満たされるわけではない.なぜなら
\begin{align*}
\left[\psi_\ell^+(x),\psi^-_{\bar{\ell}}(y)\right]_{\mp}=&\frac{1}{(2\pi)^3}\sum_{\sigma n}\sum_{\bar{\sigma} \bar{n}}\int d^3\mathbf{p} \int d^3\mathbf{q} u_\ell(\mathbf{p},\sigma,n)v_{\bar{\ell}}(\mathbf{q},\bar{\sigma},\bar{n})e^{ip\cdot x}e^{-iq\cdot y}[a(\mathbf{p},\sigma,n),a^\dagger(\mathbf{q},\bar{\sigma},\bar{n})]_{\mp} \\
=&\frac{1}{(2\pi)^3}\sum_{\sigma n}\sum_{\bar{\sigma} \bar{n}}\int d^3\mathbf{p} \int d^3\mathbf{q} u_\ell(\mathbf{p},\sigma,n)v_{\bar{\ell}}(\mathbf{q},\bar{\sigma},\bar{n})e^{ip\cdot x}e^{-iq\cdot y} \delta^3(\mathbf{p}-\mathbf{q})\delta_{n\bar{n}} \delta_{\sigma\bar{\sigma}} \\
=&\frac{1}{(2\pi)^3}\sum_{\sigma n}\int d^3\mathbf{p} u_\ell(\mathbf{p},\sigma,n)v_{\bar{\ell}}(\mathbf{p},\sigma,n)e^{ip\cdot (x-y)}
\end{align*}
であり($\pm$の符号は,成分$\psi^+_\ell$と$\psi^-_\ell$により消滅および生成される粒子がボゾンかフェルミオンかによって,それぞれ交換子または反交換子を表す),一般にこれは$x-y$が空間的な場合$(x-y)^2 >0$でさえゼロにならないからである.相互作用密度を生成場\uwave{のみ}または消滅場\uwave{のみ}から作ることでこの問題を回避することも考え着くが,それは明らかに不可能である.なぜなら,その場合は相互作用がエルミートになれないからである.((5.1.27)を共役して自分自身に戻るためには,$a^\dagger,a$がそれぞれ$N,M$個含む項に対して$a^\dagger,a$がそれぞれ$M,N$個含む項も同時に含まなければならないからである.片方だけを含ませることはできない.)\par
この困難を解決する唯一の方法は,消滅場と生成場を組み合わせた線形結合
\begin{align*}
\psi_{\ell}(x):=\kappa_\ell \psi_\ell^+(x)+\lambda_\ell \psi^-_\ell(x)
\end{align*}
を用いることである.ここで定数$\kappa,\lambda$および場に含まれる他の任意の定数は,空間的な$x-y$に対して
\begin{align*}
[\psi_\ell(x),\psi_{\bar{\ell}}(y)]_{\mp}=[\psi_\ell(x),\psi^\dagger_{\bar{\ell}}(y)]_{\mp}=0
\end{align*}
となるように調節する.後の節で,様々な既約的に変換する場についてこれを実際に実行する方法を述べる.((5.1.31)で定数$\kappa,\lambda$を陽に含めたことで,先ほど述べた消滅場と生成場のスケールの任意性を,便利と思われる任意のやり方で自由に選んで固定できる.)ハミルトニアン密度$\mc{H}_I(x)$は,このような場とその共役場から構成され,かつフェルミオンを生成・消滅する任意の場の成分を偶数個含むなら,交換関係(5.1.3)を満たす.\par
条件(5.1.32)も(5.1.3)と同じく,しばしば因果律と呼ばれる.なぜなら,$x-y$が空間的なら$x$から$y$へ信号が届かず,よって点$x$での$\psi_\ell$の測定は点$y$での$\psi_{\ell'}$(あるいは$\psi^\dagger_{\ell'}$)の測定に影響を与えることはできないからである.因果率のそのような考察は,与えられた時空点でその成分の任意の一つが\uwave{観測可能}な電磁場$A^\mu$についてはもっともらしい.しかし,ここでは電子のディラック場$\psi$のようにいかなる意味でもかのく可能には思えない場も取り扱っている.(5.1.32)は,$S$行列のローレンツ不変性にとって必要であり,それは測定可能性や因果律に関する\uwave{付加的な仮定なし}でのことだというのが,ここでの観点である.

\vskip\baselineskip


因果律(5.1.32)を満たす場(5.1.31)を構成するのに,障害が一つある.これらの場によって消滅・生成させられる粒子は,電荷のような一つ以上のゼロでない値の保存する量子数を持っているかもしれない.例えば,種類$n$の粒子が電荷$Q$の値$q(n)$を持つと(3.3.38)(3.3.39)より
\begin{align*}
e^{iQ\theta}\Psi_{\mathbf{p} \sigma n}=&e^{iq(n)\theta}\Psi_{\mathbf{p} \sigma n} \\
e^{iQ\theta}a^\dagger(\mathbf{p},\sigma,n)e^{-iQ\theta}\Psi_{0}=&e^{iq(n)\theta}\Psi_{\mathbf{p} \sigma n} \\
e^{iQ\theta}a^\dagger(\mathbf{p} ,\sigma ,n)e^{-iQ\theta}=&e^{iq(n)\theta}a^\dagger(\mathbf{p} ,\sigma ,n)
\end{align*}
と\footnote{この1粒子状態のみを使った導出だけではもちろん,消滅演算子だけの追加の項の自由度があるが,ローレンツ変換のときと同様に多粒子状態でも正しい変換性(3.3.34)を示すためにこの形しか許されない.}なり,$\theta$について一次の項を比較すれば
\begin{align*}
\therefore\quad [Q,a^\dagger(\mathbf{p},\sigma,n)]=&q(n)a^\dagger(\mathbf{p},\sigma,n)
\end{align*}
が得られる.両辺エルミートをとって
\begin{align*}
[Q,a^\dagger(\mathbf{p},\sigma,n)]^\dagger=& (Qa^\dagger(\mathbf{p}, \sigma, n)-a^\dagger(\mathbf{p} , \sigma, n)Q)^\dagger \\
=&a(\mathbf{p},\sigma ,n)Q -Q a(\mathbf{p},\sigma,n) \\
=&-[Q,a(\mathbf{p},\sigma,n)] \\
=(q(n) a^\dagger(\mathbf{p},\sigma,n))^\dagger=&q(n)a(\mathbf{p},\sigma,n) \\
\therefore \quad [Q,a(\mathbf{p},\sigma,n)]=&-q(n)a(\mathbf{p},\sigma,n)
\end{align*}
で,以上より
\begin{align*}
[Q,a(\mathbf{p},\sigma,n)]=&-q(n)a(\mathbf{p},\sigma,n) \\
[Q,a^\dagger(\mathbf{p},\sigma,n)]=&+q(n)a^\dagger(\mathbf{p},\sigma,n)
\end{align*}
が得られる.(3.3.36)を満たすためには$\mc{H}_I(x)$が電荷の演算子$Q$(または他の対称性の生成子(バリオン数演算子や$SU(2)$対称性生成子など))と交換しなければならず,そのためには$\mc{H}_I(x)$は,簡単な交換関係
\begin{align*}
[Q,\psi_\ell(x)]=-q_\ell \psi_{\ell}(x)
\end{align*}
を持つ場$\psi_\ell$から作られれば十分である.なぜなら,そうすれば
\begin{align*}
q_{\ell_1}+q_{\ell_2}+\cdots + -q_{m_1} -q_{m_2} -\cdots=0
\end{align*}
を満たすように,場$\psi_{\ell_1},\psi_{\ell_2}\cdots$とその共役場$\psi^\dagger_{m_1},\psi^\dagger_{m_2}\cdots$の積の和として$\mc{H}_I(x)$が構成されていれば,$\mc{H}_I(x)$が$Q$と交換するようにできるからである.実際
\begin{align*}
\mc{H}_I(x)=\sum_{\ell_1 \cdots \ell_{M}}g_{\ell_1\cdots \ell_M ,m_1\cdots m_N}\psi_{\ell_1}(x)\cdots \psi_{\ell_M}(x)\psi^\dagger_{m_1}(x)\cdots \psi^\dagger_{m_N}(x)
\end{align*}
の形で構成されていれば,上の交換関係の両辺をエルミート共役して
\begin{align*}
[Q,\psi^\dagger_\ell(x)]=+q_\ell \psi^\dagger_{\ell}(x)
\end{align*}
となるから
\begin{align*}
&[Q,\mc{H}_I(x)] \\
=&\sum_{\ell_1 \cdots \ell_{M}}g_{\ell_1\cdots \ell_M ,m_1\cdots m_N}[Q,\psi_{\ell_1}(x)]\cdots \psi_{\ell_M}(x)\psi^\dagger_{m_1}(x)\cdots \psi^\dagger_{m_N}(x) +\cdots \\
&+\sum_{\ell_1 \cdots \ell_{M}}g_{\ell_1\cdots \ell_M ,m_1\cdots m_N}\psi_{\ell_1}(x)\cdots [Q,\psi_{\ell_M}(x)]\psi^\dagger_{m_1}(x)\cdots \psi^\dagger_{m_N}(x) \\
&+\sum_{\ell_1 \cdots \ell_{M}}g_{\ell_1\cdots \ell_M ,m_1\cdots m_N}\psi_{\ell_1}(x)\cdots \psi_{\ell_M}(x)[Q,\psi^\dagger_{m_1}(x)]\cdots \psi^\dagger_{m_N}(x) + \cdots  \\
&+\sum_{\ell_1 \cdots \ell_{M}}g_{\ell_1\cdots \ell_M ,m_1\cdots m_N}\psi_{\ell_1}(x)\cdots \psi_{\ell_M}(x)\psi^\dagger_{m_1}(x)\cdots [Q,\psi^\dagger_{m_N}(x)] \\
=&\sum_{\ell_1 \cdots \ell_{M}}(-q_{\ell_1}-\cdots +q_{\ell_M}+q_{m_1}+\cdots +q_{m_N})g_{\ell_1\cdots \ell_M ,m_1\cdots m_N}\psi_{\ell_1}(x)\cdots \psi_{\ell_M}(x)\psi^\dagger_{m_1}(x)\cdots \psi^\dagger_{m_N}(x) \\
=&0
\end{align*}
となって交換することがわかる.さて,(5.1.33)は,消滅場がある特定の(既約)成分$\psi_{\ell}^+(x)$について,その消滅場によって消滅させられる全ての粒子の種類$n$が,同一の電荷$q_\ell=q(n)$を持てば,またそのときに限り,満たされる.また,生成場のある特定の(既約)成分$\psi_\ell^-(x)$について,その場によって生成される全ての粒子の種類$\bar{n}$が電荷$q(\bar{n})=-q_\ell$を持つならば,またそのときに限り満たされる.実際
\begin{align*}
[Q,\psi_{\ell}(x)]=&\kappa_\ell [Q,\psi_{\ell}^+(x)]+\lambda_\ell[Q,\psi_{\ell}^-(x)] \\
=-q_\ell \psi_\ell(x)=&-\kappa_\ell q_\ell \psi_{\ell}^+(x)-\lambda_\ell q_\ell \psi_\ell^-(x)
\end{align*}
であるためには
\begin{align*}
[Q,\psi_\ell^+(x)]=&\sum_{\sigma n}\frac{1}{(2\pi)^{3/2}}\int d^3\mathbf{p} u_\ell (\mathbf{p},\sigma,n)e^{ip\cdot x} [Q,a(\mathbf{p},\sigma,n) ] \\
=&-\sum_{\sigma n}\frac{1}{(2\pi)^{3/2}}\int d^3\mathbf{p} u_\ell (\mathbf{p},\sigma,n)e^{ip\cdot x} q(n)a(\mathbf{p},\sigma,n)
\end{align*}
が
\begin{align*}
-q_\ell \psi_{\ell}^+(x)=&-q_{\ell}\sum_{\sigma n}\frac{1}{(2\pi)^{3/2}}\int d^3\mathbf{p} u_\ell (\mathbf{p},\sigma,n)e^{ip\cdot x} a(\mathbf{p},\sigma,n)
\end{align*}
でなければならないし,さらに
\begin{align*}
[Q,\psi_\ell^-(x)]=&\sum_{\sigma \bar{n}}\frac{1}{(2\pi)^{3/2}}\int d^3\mathbf{p} v_\ell (\mathbf{p},\sigma,n)e^{-ip\cdot x} [Q,a^\dagger (\mathbf{p},\sigma,\bar{n}) ] \\
=&+\sum_{\sigma \bar{n}}\frac{1}{(2\pi)^{3/2}}\int d^3\mathbf{p} v_\ell (\mathbf{p},\sigma,n)e^{-ip\cdot x} q(\bar{n})a^\dagger(\mathbf{p},\sigma,\bar{n})
\end{align*}
が
\begin{align*}
-q_\ell \psi_{\ell}^-(x)=&-q_{\ell}\sum_{\sigma \bar{n}}\frac{1}{(2\pi)^{3/2}}\int d^3\mathbf{p} v_\ell (\mathbf{p},\sigma,\bar{n})e^{-ip\cdot x} a^\dagger(\mathbf{p},\sigma,\bar{n})
\end{align*}
でなければならない.これは,消滅場で消滅させられる任意の種類$n$について$q(n)=q_\ell$のとき,かつ生成場で生成させられる任意の種類$\bar{n}$(消滅場の種類$n$と等しくなくてよい)について$q(\bar{n})=-q_\ell$のときになりたち,そうでなければならない.そうした理論が,電荷のような量子数を保存するためには,\uwave{ゼロでない}量子数を持つ粒子の種類は$q(n)=q_\ell$を与える種類$n$と$q(\bar{n})=-q_\ell$を与える$\bar{n}$が二重に存在していなければならない.この粒子の種類$\bar{n}$は,粒子の種類$n$の\textbf{反粒子}として知られ,全ての保存量子数が反対の値を持つ(上での議論を電荷に限らず,全ての保存量子数で行えばよい).これが反粒子の存在する理由である.

\vskip\baselineskip


もし表現$D(\Lambda)$が既約でないなら,場の基底を適当にとることで$D(\Lambda)$が主要な対角成分に沿ってブロック対角になるように分解し,異なるブロックに属する場(不変部分空間)はローレンツ変換のもとで互いに変換しあわないようにできる.
\begin{align*}
&U_0(\Lambda)\psi_\ell(x) U^{-1}_0(\Lambda)=U_0(\Lambda)\left(
\begin{matrix}
\phi(x) \\
\psi_a(x) \\
A^\mu(x) \\
\vdots
\end{matrix}
\right)U_0(\Lambda) \\
=&\sum_{\bar{\ell}}D_{\ell\bar{\ell}}(\Lambda^{-1})\psi_{\bar{\ell}}(\Lambda)=\left(
\begin{matrix}
1 &                 &             & \\
  & D_{ab}(\Lambda^{-1}) &             & \\
  &                 & \tensor{(\Lambda^{-1})}{^\mu_\nu} & \\
  &                 &                &   \ddots
\end{matrix}
\right) \left(
\begin{matrix}
\phi(\Lambda x) \\
\psi_b(\Lambda x) \\
A^\nu(\Lambda x) \\
\vdots
\end{matrix}
\right)
\end{align*}
また,ローレンツ変換は粒子の種類を変えない.(3.3節でのノートにも書いたが,ローレンツ変換は1粒子状態の運動量とスピン$z$成分(またはヘリシティ)のみを変えて,内部対称性は粒子の種類のみを変える変換なのだった.)よって,多くの既約成分と多くの粒子の種類を含む一つの大きな(可約表現に属する)場を考える代わりに,今後は単一のタイプの粒子(添え字$n$を省略して)のみを消滅させ,また対応する反粒子(場合によってはその粒子自身)のみを生成し,かつローレンツ群(空間反転を含んでいると考えてもそうでなくてもよい)のもとで\uwave{既約的}に変換する場のみを考える.ここで,一般には,ある場が他の場の微分になっている場合も含めて,多くの異なるそのような場を考えなければならないことを了解しておこう.次節から,最初にローレンツ群の最も単純な既約表現であるスカラー,ベクトル,およびディラックのスピノル表現に属する場について,係数関数$u_\ell(\mathbf{p},\sigma)$と$v_\ell(\mathbf{p},\sigma)$を決定し,定数$\kappa,\lambda$の相対値を定め,粒子と反粒子の性質の間の関係を導く.その後で,完全に一般的な既約表現についても解析を繰り返す.

\vskip\baselineskip

場の方程式について一言.(5.1.31)(5.1.17)(5.1.18)を見れば,一定の質量$m$を持つ場の全ての成分は,クライン・ゴルドン方程式
\begin{align*}
(\Box-m^2)\psi_\ell(x)=0
\end{align*}
を満たすことがわかる.実際,$p^0=\sqrt{\mathbf{p}^2+m^2}$であることを用いると$p^2+m^2=0$であり
\begin{align*}
(\Box-m^2)\psi_\ell(x)=&\kappa_\ell (\Box-m^2) \psi_\ell^+(x)+\lambda_\ell (\Box-m^2)\psi_\ell^-(x) \\
=&\kappa_\ell \frac{1}{(2\pi)^{3/2}}\sum_{\sigma}\int d^3\mathbf{p} u_\ell(\mathbf{p},\sigma)\Bigl[(\Box-m^2)e^{ip\cdot x}\Bigr] a(\mathbf{p},\sigma) \\
&+\lambda_\ell \frac{1}{(2\pi)^{3/2}}\sum_{\sigma}\int d^3\mathbf{p} v_\ell(\mathbf{p},\sigma,\bar{n})\Bigl[(\Box-m^2)e^{-ip\cdot x}\Bigr] a(\mathbf{p},\sigma) \\
=&\kappa_\ell \frac{1}{(2\pi)^{3/2}}\sum_{\sigma}\int d^3\mathbf{p} u_\ell(\mathbf{p},\sigma)\Bigl[(-p^2-m^2)e^{ip\cdot x}\Bigr] a(\mathbf{p},\sigma) \\
&+\lambda_\ell \frac{1}{(2\pi)^{3/2}}\sum_{\sigma }\int d^3\mathbf{p} v_\ell(\mathbf{p},\sigma)\Bigl[(-p^2-m^2)e^{-ip\cdot x}\Bigr] a(\mathbf{p},\sigma) \\
=&0
\end{align*}
となる.ある場は,場の成分が独立な粒子の状態より多いか否かによって,さらに他の場の方程式を満たす.(例えば,スカラー場は一つの粒子状態しかないので,これ以外の場の方程式は存在しない.質量のあるベクトル場は,$v^\mu$が4成分あるのに対して,スピン$z$成分$+1,0,-1$の3つしか独立な粒子状態がないので追加で一つの場の方程式(5.3.38)を満たす.ディラック場$\psi_a$は可約表現なので独立な自由度の数は単純には調べられないが,ディラック方程式が追加の場の方程式になっている.)\par
伝統的に場の量子論ではそのような場の方程式,あるいはそれを導くもとになるラグランジアンから出発し,それを用いて1粒子の消滅・生成演算子についての場の展開を導く.この本の進め方では,まず粒子から出発し,ローレンツ不変性の指図に従って自由場を導き,場の方程式はこれを構成する際の副産物としてほとんど付随的に出てくる!

\vskip\baselineskip


ここで技術的なことを述べておく.4.4節で与えた定理によれば,クラスター分解原理を保証する条件は,「相互作用が生成・消滅演算子の積の和で表せて,全ての消滅演算子が全ての生成演算子の右に来て,その係数はただ一つの運動量保存のデルタ関数を含むこと」である.このために,相互作用を「正規順序」形
\begin{align*}
V=\int d^3\mathbf{x} \, :\mc{F}(\psi(\mathbf{x}),\psi^\dagger(\mathbf{x})):,\quad \mc{H}(x)=:\mc{F}(\psi(x),\psi^\dagger(x)) :
\end{align*}
で書くべきである!ここで正規順序のコロン(:)は,(それで囲まれた表現がゼロでない交換子と反交換子は無視し,フェルミオン的演算子の交換についてはマイナス符号も含めて)全ての生成演算子は全ての消滅演算子の左に来るように書き直されることを表す.
\begin{align*}
:a_F(q)a_F^\dagger(q')a_B^\dagger(q''):=-a_F^\dagger(q')a_B^\dagger(q'')a_F(q)
\end{align*}
場の交換関係または反交換関係を用いると,場の任意のそのような正規順序関数は,ちょうどc数を係数とする場の通常の積の和として書き表せる.例えば(5.1.30)より
\begin{align*}
:\psi_\ell(x)\psi_{\bar{\ell}}(x):=&:\Bigl((\psi_\ell^+(x)+\psi_\ell^-(x))(\psi_{\bar{\ell}}^+(x)+\psi_{\bar{\ell}}^-(x))\Bigr): \\
=&:\Bigl(\psi_\ell^+(x) \psi^+_{\bar{\ell}}(x)+\psi_\ell^+(x) \psi^-_{\bar{\ell}}(x)+\psi_\ell^-(x) \psi^+_{\bar{\ell}}(x)+\psi_\ell^-(x) \psi^-_{\bar{\ell}}(x)\Bigr): \\
=&\psi_\ell^+(x) \psi^+_{\bar{\ell}}(x) \pm \psi_{\bar{\ell}}^-(x) \psi^+_\ell(x)+\psi_\ell^-(x) \psi^+_{\bar{\ell}}(x)+\psi_\ell^-(x) \psi^-_{\bar{\ell}}(x) \\
=&\psi_\ell^+(x) \psi^+_{\bar{\ell}}(x)+\psi_\ell^+(x) \psi^-_{\bar{\ell}}(x)+\psi_\ell^-(x) \psi^+_{\bar{\ell}}(x)+\psi_\ell^-(x) \psi^-_{\bar{\ell}}(x) \\
&-(\psi_{\ell}^+(x)\psi_{\bar{\ell}}^-(x) \mp \psi_{\bar{\ell}}^-(x)\psi_{\ell}^+(x)) \\
=&\psi_\ell(x) \psi_{\bar{\ell}}(x)-[\psi_{\ell}^+(x),\psi_{\bar{\ell}}^-(x)]_{\mp} \\
[\psi_{\ell}^+(x),\psi_{\bar{\ell}}^-(x)]_{\mp}=&\frac{1}{(2\pi)^3}\sum_{\sigma n}\int d^3\mathbf{p}u_\ell(\mathbf{p},\sigma,n)v_{\bar{\ell}}(\mathbf{p},\sigma,n)
\end{align*}
となる.したがって,$\mc{F}$をこのように書き換えると,それが因果律(5.1.32)を満たす場から構成され,フェルミオン場を偶数個だけ含むならば,正規順序化にもかかわらず,空間的な$x-y$に対して交換
\begin{align*}
:\mc{F}(\psi(x),\psi^\dagger(x))::\mc{F}(\psi(y),\psi^\dagger(y)):=&:\mc{F}(\psi(y),\psi^\dagger(y))::\mc{F}(\psi(x),\psi^\dagger(x)): \\
\mc{H}_I(x) \mc{H}_I(y)=&\mc{H}_I(y) \mc{H}_I(x)
\end{align*}
がなりたつ.



\newpage



\subsection{因果律を満たすスカラー場}
最初に,$D(\Lambda)=1$を持ち,ローレンツ群の全ての表現の中で最も単純な表現,すなわちスカラーとして変換する1成分の消滅場$\phi^+(x)$と生成場$\phi^-(x)$を考える.この表現が実際に表現であることは自明である(バカらしいかもしれないが,一応確かめておくと
\begin{align*}
D(\Lambda_1)D(\Lambda_2)=1=D(\Lambda_1\Lambda_2)
\end{align*}
となり,確かに表現になっている.).回転の部分群に限れば,これはちょうど回転群$SO(3)$のスカラー表現で,$\bm{\mc{J}}=0$であり,よって(5.1.25)(5.1.26)は$j=0$以外の解を持たない.($D(R)$は$D^{(j)}(R)$を既約成分をして含む表現であることを示したのだから,$D(R)$が既約スカラー表現であることから$D^{(j)}(R)$は$j=0$のスカラー表現$D^{(0)}(R)=1$でしかありえない.)この場合$\bar{\sigma},\sigma$はそれぞれ値ゼロのみをとる.こうしてスカラー場は,スピンゼロ粒子のみを記述できる!\par
前節で述べたように,既約な消滅場は一種類のみの粒子だけを消滅させ,既約な生成場も一種類のみの粒子を生成させると仮定する.さらに,当面の間(p278に入るまで)どちらの場も同一の種類の粒子を生成・消滅すると仮定する.これにより粒子の添え字の$n$を省略して,量$u_\ell(0,\sigma)$と$v_\ell(0,\sigma)$は$\ell,\sigma$は一つの成分$u(0),v(0)$しかとらない.これは単に数であり,全体のスケールは任意であったから,これらの定数がともに値$(2m)^{-1/2}$を持つように消滅・生成場の全体のスケールをとることができる.(これは慣習らしい.)すると(5.1.21)と(5.1.22)は単に
\begin{align*}
u(\mathbf{p})=&\sqrt{\frac{m}{p^0}}u(0)=\sqrt{\frac{m}{p^0}}\frac{1}{\sqrt{2m}} =\frac{1}{\sqrt{2p^0}} \\
v(\mathbf{p})=&\sqrt{\frac{m}{p^0}}v(0)=\sqrt{\frac{m}{p^0}}\frac{1}{\sqrt{2m}} =\frac{1}{\sqrt{2p^0}}
\end{align*}
を与える.そして場(5.1.17)(5.1.18)はスカラーの場合
\begin{align*}
\phi^+(x)=&\int d^3\mathbf{p} \frac{1}{(2\pi)^{3/2}}u(\mathbf{p})a(\mathbf{p})e^{ip\cdot x} \\
=&\int d^3\mathbf{p} \frac{1}{(2\pi)^{3/2}}\frac{1}{\sqrt{2p^0}}a(\mathbf{p})e^{ip\cdot x} \\
=&\int \frac{d^3\mathbf{p}}{(2\pi)^{3/2} \sqrt{2p^0}}a(\mathbf{p})e^{ip\cdot x} \\
\phi^-(x)=&\int d^3\mathbf{p} \frac{1}{(2\pi)^{3/2}}v(\mathbf{p})a^\dagger(\mathbf{p})e^{-ip\cdot x} \\
=&\int d^3\mathbf{p} \frac{1}{(2\pi)^{3/2}}\frac{1}{\sqrt{2p^0}}a^\dagger(\mathbf{p})e^{-ip\cdot x} \\
=&\int \frac{d^3\mathbf{p}}{(2\pi)^{3/2} \sqrt{2p^0}}a^\dagger(\mathbf{p})e^{-ip\cdot x}=\phi^{+\dagger}(x)
\end{align*}
となる.(この生成・消滅演算子はボゾンかフェルミオンかはまだわからないとする.)

\vskip\baselineskip

$\phi^+(x)$と$\phi^-(x)$はローレンツ変換のもとでスカラーとして振舞う
\begin{align*}
U_0(\Lambda)\phi^+(x)U^{-1}_0(\Lambda)=&\phi^+(\Lambda x) \\
U_0(\Lambda)\phi^-(x)U^{-1}_0(\Lambda)=&\phi^-(\Lambda x)
\end{align*}
ので,$\phi^+(x),\phi^-(x)$で作られたハミルトニアンはローレンツスカラーであるべきという条件(5.1.9)を自動的に満たす.$S$行列のローレンツ不変性のためには,もう一つの条件があるのだった.それは,$x-y$が空間的な距離のとき$\mc{H}_I(x)$と$\mc{H}_I(y)$が交換するという条件(5.1.3)である.もし$\mc{H}_I(x)$が$\phi^+(x)$だけの多項式になっているなら問題はない.なぜなら,全ての消滅演算子は交換または反交換(4.2.7)するので,あらゆる$x,y$に対して,粒子がボゾンかフェルミオンかに応じてそれぞれ,$\phi^+(x)$と$\phi^+(y)$とは交換または反交換する.
\begin{align*}
[\phi^+(x),\phi^+(y)]_{\mp}=&\int \frac{d^3\mathbf{p}}{(2\pi)^{3/2} \sqrt{2p^0}}\int \frac{d^3\mathbf{q}}{(2\pi)^{3/2} \sqrt{2q^0}}e^{ip\cdot x+iq\cdot y}[a(\mathbf{p}),a(\mathbf{q})]_{\mp} \\
=&0
\end{align*}
よって$\phi^+(x)$の多項式(あるいは,もしかしたらフェルミオンのスカラー場があればそのような偶数次の多項式)として作られた任意の$\mc{H}_I(x)$は,全ての$x$と$y$について$\mc{H}_I(y)$と交換する.もちろん,問題は$\mc{H}_I(x)$がエルミートであるためには,それが$\phi^+(x)$だけでなく$\phi^{+\dagger}(x)=\phi^-(x)$を含まねばならず,そのために一般の空間的距離だけ離れたとき$\phi^+(x)$は$\phi^-(y)$と交換あるいは反交換しないことである.(4.2.5)を用いると
\begin{align*}
[\phi^+(x),\phi^-(y)]_{\mp}=&\int \frac{d^3\mathbf{p}}{(2\pi)^{3/2} \sqrt{2p^0}}\int \frac{d^3\mathbf{p}'}{(2\pi)^{3/2} \sqrt{2p'^0}}e^{ip\cdot x-ip'\cdot y}[a(\mathbf{p}),a^\dagger(\mathbf{q})]_{\mp} \\
=&\int \frac{d^3\mathbf{p}}{(2\pi)^{3/2} \sqrt{2p^0}}\int \frac{d^3\mathbf{p}'}{(2\pi)^{3/2} \sqrt{2p'^0}}e^{ip\cdot x-ip'\cdot y} \delta^3(\mathbf{p}-\mathbf{p}') \\
=&\int \frac{d^3\mathbf{p}}{(2\pi)^3 2p^0}e^{ip\cdot (x- y)}=\Delta_+(x-y)
\end{align*}
という単一の積分になる.ここで
\begin{align*}
\Delta_+(x):=\int \frac{d^3\mathbf{p}}{(2\pi)^3 2p^0}e^{ip\cdot x}
\end{align*}
である.これは明白に固有順時ローレンツ変換のもとで不変である.実際
\begin{align*}
\Delta_+(\Lambda x)=&\int \frac{d^3\mathbf{p}}{(2\pi)^3 2p^0}e^{ip\cdot (\Lambda x)} \\
=&\int \frac{d^3\mathbf{p}_\Lambda}{(2\pi)^3 2(\Lambda p)^0}e^{i(\Lambda p)\cdot (\Lambda x)} \because 変数変換 \mathbf{p}\to \mathbf{p}_\Lambda\\
=&\int \frac{d^3\mathbf{p}_\Lambda}{(2\pi)^3 2(\Lambda p)^0}e^{ip\cdot x} \\
=&\int \frac{d^3\mathbf{p}}{(2\pi)^3 2p^0}e^{ip\cdot x} \quad \because (2.5.15) \\
=&\Delta_+(x)
\end{align*}
となる.したがって,空間的な$x$については不変二乗量$x^2>0$のみに依存する.ここで,$x$が空間的な場合,固有順時ローレンツ変換によって座標系を
\begin{align*}
x^0=0,\quad |\mathbf{x}|=\sqrt{x^2}
\end{align*}
となるように選ぶことが常にできる.(実際,空間的な$x^\mu$が与えられたとき,$x^\mu x_\mu >0$より$\sqrt{x^\mu x_\mu}$を定義することができるから,ローレンツ変換を
\begin{align*}
\gamma:=&\frac{|\mathbf{x}|}{\sqrt{x^\mu x_\mu}},\quad v:=\frac{x^0}{|\mathbf{x}|} ,\quad \mathbf{n}:=\frac{\mathbf{x}}{|\mathbf{x}|} \\
\tensor{\Lambda}{^i_j}=&\delta_{ij}+(\gamma-1)n_i n_j \\
\tensor{\Lambda}{^i_0}=&\tensor{\Lambda}{^0_i}=-\gamma v n_i \\
\tensor{\Lambda}{^0_0}=&\gamma \\
\Lambda=&\left(
\begin{matrix}
1+(\gamma-1)\mathbf{n} \mathbf{n}^T & -\gamma v \mathbf{n} \\
-\gamma v \mathbf{n}^T & \gamma
\end{matrix}
\right)
\end{align*}
と選べば
\begin{align*}
\left(
\begin{matrix}
\mathbf{x}' \\
x'^0
\end{matrix}
\right)=&\left(
\begin{matrix}
1+(\gamma-1)\mathbf{n} \mathbf{n}^T & -\gamma v \mathbf{n} \\
-\gamma v \mathbf{n}^T & \gamma
\end{matrix}
\right)\left(
\begin{matrix}
\mathbf{x} \\
x^0
\end{matrix}
\right) \\
=&\left(
\begin{matrix}
\mathbf{x}+(\gamma-1)\mathbf{n} (\mathbf{n}\cdot \mathbf{x})-\gamma v x^0 \mathbf{n} \\
-\gamma v (\mathbf{n}\cdot \mathbf{x}) +\gamma x^0
\end{matrix}
\right) \\
\mathbf{x}'=&\mathbf{x}+(\gamma-1)\mathbf{n} (\mathbf{n}\cdot \mathbf{x})-\gamma v x^0 \mathbf{n} \\
=&\mathbf{x}+(\gamma-1)\mathbf{x} -\gamma v x^0 \mathbf{n} \\
=&\frac{|\mathbf{x}|^2}{\sqrt{x^\mu x_\mu}}\mathbf{n} -\frac{(x^0)^2}{\sqrt{x^\mu x_\mu}} \mathbf{n} \\
=&\sqrt{x^\mu x_\mu} \mathbf{n} \\
x'^0=&-\gamma v (\mathbf{n}\cdot \mathbf{x}) +\gamma x^0 \\
=&-\gamma \frac{x^0}{|\mathbf{x}|} |\mathbf{x}| + \gamma x^0 \\
=&0
\end{align*}
となる.これが$O(3.1)$の元であることは
\begin{align*}
\Lambda^T \eta \Lambda =&\left(
\begin{matrix}
1+(\gamma-1)\mathbf{n} \mathbf{n}^T & -\gamma v \mathbf{n} \\
-\gamma v \mathbf{n}^T & \gamma
\end{matrix}
\right) \left(
\begin{matrix}
1 & 0 \\
0 & -1
\end{matrix}
\right)\left(
\begin{matrix}
1+(\gamma-1)\mathbf{n} \mathbf{n}^T & -\gamma v \mathbf{n} \\
-\gamma v \mathbf{n}^T & \gamma
\end{matrix}
\right) \\
=&\left(
\begin{matrix}
1+(\gamma-1)\mathbf{n} \mathbf{n}^T & +\gamma v \mathbf{n} \\
-\gamma v \mathbf{n}^T & -\gamma
\end{matrix}
\right)\left(
\begin{matrix}
1+(\gamma-1)\mathbf{n} \mathbf{n}^T & -\gamma v \mathbf{n} \\
-\gamma v \mathbf{n}^T & \gamma
\end{matrix}
\right) \\
=&\left(
\begin{matrix}
1+2(\gamma-1)\mathbf{n} \mathbf{n}^T +(\gamma-1)^2\mathbf{n} \mathbf{n}^T -\gamma^2 v^2 \mathbf{n}\mathbf{n}^T & -\gamma v \mathbf{n}-\gamma(\gamma-1)v\mathbf{n} +\gamma^2 v \mathbf{n} \\
-\gamma v \mathbf{n}^T-\gamma(\gamma-1)v\mathbf{n}^T +\gamma^2 v \mathbf{n}^T & \gamma^2 v^2 -\gamma^2
\end{matrix}
\right) \\
(i,j) 成分 =&1+2(\gamma-1)\mathbf{n} \mathbf{n}^T +(\gamma-1)^2\mathbf{n} \mathbf{n}^T -\gamma^2 v^2 \mathbf{n}\mathbf{n}^T \\
=&1+[2(\gamma-1)+(\gamma-1)^2 -\gamma^2 v^2]\mathbf{n}\mathbf{n}^T \\
=&1+[2(\gamma-1)+\gamma^2-2\gamma+1 -\gamma^2 v^2]\mathbf{n}\mathbf{n}^T \\
=&1+[\gamma^2(1 -v^2)-1]\mathbf{n}\mathbf{n}^T \\
=&1 \\
(0,i)成分=&(-\gamma -\gamma(\gamma-1) +\gamma^2 )v\mathbf{n} \\
=&0 \\
(0,0)成分=&\gamma^2 v^2 -\gamma^2=-1 \\
\therefore \quad \Lambda^T \eta \Lambda =&\eta
\end{align*}
であることから示せる.ここで
\begin{align*}
\gamma^2(1-v^2)=&\gamma^2\left(1-\frac{(x^0)^2}{|\mathbf{x}|^2}\right) \\
=&\gamma^2\frac{1}{|\mathbf{x}|^2}\left(\mathbf{x}^2-(x^0)^2\right) \\
=&\gamma^2 \frac{1}{|\mathbf{x}|^2}x^\mu x_\mu \\
=&1
\end{align*}
であることを用いた.$\tensor{\Lambda}{^0_0} \geq +1$であることは自明であり,さらに2.5節で与えた公式
\begin{align*}
M=&\left(
\begin{matrix}
A & B \\
C & D
\end{matrix}
\right) \\
\mathrm{det}M=&\det D \det\left(A-BD^{-1}C\right)
\end{align*}
によれば
\begin{align*}
A-BD^{-1} C=&[1+(\gamma-1)\mathbf{n} \mathbf{n}^T]-\frac{1}{\gamma} \gamma^2 v^2 \mathbf{n}\mathbf{n}^T \\
=&1+(\gamma(1-v^2) -1)\mathbf{n}\mathbf{n}^T \\
=&1+\left(\frac{1}{\gamma} -1\right)\mathbf{n}\mathbf{n}^T \\
\mathrm{det} \Lambda =&\gamma \mathrm{det}\left[1+\left(\frac{1}{\gamma} -1\right)\mathbf{n} \mathbf{n}^T\right] \\
=&\gamma \left[1+\left(\frac{1}{\gamma} -1\right)\mathbf{n}^T\mathbf{n}\right] \\
=&1
\end{align*}
なので固有順時ローレンツ変換$SO(3,1)$の元であることがわかる.最後の1行ではmatrix determinant lemma
\begin{align*}
\det(I+\mathbf{u}\mathbf{v}^T)=1+\mathbf{v}^T \mathbf{u}
\end{align*}
を用いた.)すると空間的な$x$に対する$\Delta_+(x)$を評価できる.つまりこの$\Lambda$を用いると
\begin{align*}
\Delta_+(x)=&\Delta_+(\Lambda x) \\
=&\int \frac{d^3\mathbf{p}}{(2\pi)^3 2p^0}e^{ip\cdot (\Lambda x)} \\
=&\frac{1}{(2\pi)^3}\int \frac{d^3\mathbf{p}}{2\sqrt{\mathbf{p}^2+m^2}}e^{i\mathbf{p}\cdot \mathbf{x}'} \\
=&\frac{1}{(2\pi)^3}\int_0^\infty \frac{p^2 dp}{2\sqrt{p^2+m^2}}\int_{-1}^{1} d\cos\theta e^{ipr\cos\theta} \int_0^{2\pi}d\phi \qquad (r:=|\mathbf{x}|=\sqrt{x^2}) \\
=&\frac{2\pi}{(2\pi)^3}\int_0^\infty \frac{p^2 dp}{2\sqrt{p^2+m^2}}\left[\frac{1}{ipr}e^{ipr\cos\theta}\right]_{-1}^{1} \\
=&\frac{2\pi}{(2\pi)^3}\int_0^\infty \frac{p^2 dp}{2\sqrt{p^2+m^2}}\frac{1}{ipr}(e^{ipr}-e^{-ipr}) \\
=&\frac{2\pi}{(2\pi)^3}\int_0^\infty \frac{p^2 dp}{2\sqrt{p^2+m^2}}\frac{2\sin(pr)}{pr} \\
=&\frac{4\pi}{(2\pi)^3}\int_0^\infty \frac{p^2 dp}{2\sqrt{p^2+m^2}}\frac{\sin(pr)}{pr}=\frac{4\pi}{(2\pi)^3}\int_0^\infty \frac{p^2 dp}{2\sqrt{p^2+m^2}}\frac{\sin(p\sqrt{x^2})}{p\sqrt{x^2}}
\end{align*}
を与える.さらに変形するとこれは
\begin{align*}
=&\frac{1}{4\pi^2 r}\int_0^\infty \frac{p^2 dp}{\sqrt{p^2+m^2}}\frac{\sin(pr)}{p} \\
=&\frac{1}{4\pi^2 r}\int_0^\infty \frac{p}{\sqrt{p^2+m^2}}\sin(pr) dp \\
=&\frac{-1}{4\pi^2 r}\frac{\partial}{\partial r}\int_0^\infty \frac{1}{\sqrt{p^2+m^2}}\cos(pr)dp \\
=&\frac{-1}{4\pi^2 r}\frac{\partial}{\partial r}\int_0^\infty \frac{1}{\sqrt{(q/r)^2+m^2}}\cos(q)dq \\
=&\frac{-1}{4\pi^2 r}\frac{\partial}{\partial r}\int_0^\infty \frac{1}{\sqrt{q^2+(mr)^2}}\cos(q)dq \\
=&\frac{-1}{4\pi^2 r}\frac{\partial}{\partial r} K_0(mr) \\
=&\frac{1}{4\pi^2 r}K_1(mr) \quad \because \frac{\partial}{\partial z}K_0(z)=-K_1(z)
\end{align*}
ここで第二種変形ベッセル関数
\begin{align*}
K_\nu(z)=&\frac{\Gamma\left(\nu+\frac{1}{2}\right)(2z)^\nu}{\sqrt{\pi}} \int_0^\infty \frac{\cos t}{(t^2+z^2)^{\nu+\frac{1}{2}}} dt \\
K_0(z)=&\int \frac{\cos t}{\sqrt{t^2+z^2}} dt
\end{align*}
を用いた.これは指数関数的に減衰する関数であり,$z\to \infty$の漸近形は
\begin{align*}
K_1(z)=e^{-z}\frac{\pi}{2z}+\cdots 
\end{align*}
となる.\par
これは明らかに(小さくとも)ゼロにならない.したがって$\phi^+(x)$と$\phi^-(x)$を素朴に積をとって作った$\mc{H}_I(x)$では空間的なときに因果律(5.1.3)を満たせない.どうすればいいだろうか?まずわかることとして,$\Delta_+(x)$はゼロではないが$x^2>0$に対して$x^\mu$の偶関数になることがわかる.(これは空間的な$x^\mu$に限り$x'^\mu=-x^\mu$となるような固有順時ローレンツ変換が存在することが背景にある.実際,上で作った$\Lambda:(\mathbf{x},x^0)\mapsto (\sqrt{x^2}\mathbf{n},0)$と,空間成分を反転させる空間回転$\mc{R}$
\begin{align*}
\mc{R}:=&\left(
\begin{matrix}
R & 0 \\
0 & 1
\end{matrix}
\right), \quad R:=2\mathbf{k} \mathbf{k}^T -1,\quad (\mathbf{k}\cdot \mathbf{n}=0,|\mathbf{k}|=1) \\
R \mathbf{n}=&(2\mathbf{k} \mathbf{k}^T-1)\mathbf{n}=-\mathbf{n} \\
\mathrm{det} R=&2\mathbf{k}^T \mathbf{k} -1 =1 \quad \therefore \quad \mc{R} \in SO(3,1)
\end{align*}
を用いて作ったローレンツ変換$\Lambda^{-1}\mc{R} \Lambda$は
\begin{align*}
x'=&\Lambda^{-1}\mc{R} \Lambda x \\
=&\Lambda^{-1}\mc{R} (\sqrt{x^\mu x_\mu}\mathbf{n} ,0)^T \\
=&\Lambda^{-1}(-\sqrt{x^\mu x_\mu}\mathbf{n} ,0)^T \\
=&-\Lambda^{-1}(\sqrt{x^\mu x_\mu}\mathbf{n} ,0)^T \\
=&-x
\end{align*}
となる.これは$SO(3,1)$の元の積であるから,$SO(3,1)$の元である.).$\phi^+(x)$だけを用いるのではなく,線形結合
\begin{align*}
\phi(x):=\kappa \phi^+(x)+\lambda \phi^-(x)
\end{align*}
から$\mc{H}_I(x)$を構成してみよう.すると(5.2.5)(5.2.6)を用いて,空間的な$x-y$に対して
\begin{align*}
[\phi(x),\phi^\dagger(y)]_{\mp}=&[\kappa \phi^+(x)+\lambda \phi^-(x), \kappa^* \phi^-(y)+\lambda^* \phi^+(y)]_{\mp} \\
=&|\kappa|^2[\phi^+(x),\phi^-(y)]_{\mp}+|\lambda|^2 [\phi^-(x),\phi^+(y)]_{\mp} \\
=&|\kappa|^2[\phi^+(x),\phi^-(y)]_{\mp} \mp |\lambda|^2 [\phi^+(y),\phi^-(x)]_{\mp} \\
=&|\kappa|^2\Delta_+(x-y)\mp |\lambda|^2\Delta_+(y-x) \\
=&|\kappa|^2\Delta_+(x-y)\mp |\lambda|^2\Delta_+(x-y) \\
=&(|\kappa|^2\mp |\lambda|^2)\Delta_+(x-y)
\end{align*}
と
\begin{align*}
[\phi(x),\phi(y)]_{\mp}=&[\kappa \phi^+(x)+\lambda \phi^-(x), \kappa \phi^+(y)+\lambda \phi^-(y)]_{\mp} \\
=&\kappa \lambda [\phi^+(x),\phi^-(y)]_{\mp}+\kappa\lambda [\phi^-(x),\phi^+(y)]_{\mp} \\
=&\kappa \lambda [\phi^+(x),\phi^-(y)]_{\mp} \mp \kappa\lambda [\phi^+(y),\phi^-(x)]_{\mp} \\
=&\kappa \lambda \Delta_+(x-y) \mp \kappa\lambda \Delta_+(y-x) \\
=&\kappa \lambda \Delta_+(x-y) \mp \kappa\lambda \Delta_+(x-y) \\
=&\kappa\lambda(1\mp 1)\Delta_+(x-y)
\end{align*}
が得られる.これらは,粒子が\uwave{ボゾン}で(すなわち適合するのは符号の上側)かつ$\kappa$と$\lambda$の大きさが
\begin{align*}
|\kappa|=|\lambda|
\end{align*}
と等しいなら,そしてそのときに限り\footnote{if and only if (iff)という慣用句の訳.同値性を表す.},両方ともゼロになる.ここで1粒子状態の位相を再定義$\Phi_{\mathbf{p}}\to e^{-i\theta} \Phi_{\mathbf{p}}$することで,それに伴って生成・消滅消滅演算子の再定義$a(\mathbf{p})\to e^{i\theta}a(\mathbf{p}),a^\dagger(\mathbf{p})\to e^{-i\theta}a^\dagger(\mathbf{p})$がされる.したがって
$\kappa=|\kappa|e^{i\alpha},\lambda=|\lambda|e^{i\beta}$とおいて,
\begin{align*}
\phi(x)=&|\kappa|e^{i\alpha}\int \frac{d^3\mathbf{p}}{(2\pi)^{3/2} \sqrt{2p^0}}e^{i\theta}a(\mathbf{p})e^{ip\cdot x}+|\lambda|e^{i\beta}\int \frac{d^3\mathbf{p}}{(2\pi)^{3/2} \sqrt{2p^0}}e^{-i\theta}a^\dagger(\mathbf{p})e^{ip\cdot x} \\
=&|\kappa|e^{i(\alpha+\theta)}\int \frac{d^3\mathbf{p}}{(2\pi)^{3/2} \sqrt{2p^0}}a(\mathbf{p})e^{ip\cdot x}+|\kappa|e^{i(\beta-\theta)}\int \frac{d^3\mathbf{p}}{(2\pi)^{3/2} \sqrt{2p^0}}a^\dagger(\mathbf{p})e^{ip\cdot x} \\
=&|\kappa|e^{i(\alpha+\theta)}\phi^+(x)+|\kappa|e^{i(\beta-\theta)} \phi^-(x)
\end{align*}
となる.再定義の位相を$\theta=\frac{1}{2}(\beta-\alpha)=\frac{1}{2}\mathrm{Arg}(\lambda/\kappa)$と選べば,
\begin{align*}
=&|\kappa|e^{-i(\alpha+\beta)/2}\phi^+(x)+|\kappa|e^{i(\alpha+\beta)/2} \phi^-(x) \\
=&|\kappa|e^{-i(\alpha+\beta)/2}(\phi^+(x)+\phi^-(x))
\end{align*}
と$\kappa$と$\lambda$の位相を等しくすることができる.$\phi(x)$を再定義して,全体にかかるスケール$|\kappa|e^{-i(\alpha+\beta)/2}$を吸収すれば
\begin{align*}
\phi(x)=\phi^+(x)+\phi^{+\dagger}(x)=\phi^\dagger(x)
\end{align*}
が得られる.相互作用密度$\mc{H}_I(x)$は,自己共役スカラー場$\phi(x)$の正規順序積の多項式になっていれば,空間的な距離$x-y$だけ離れた$\mc{H}_I(y)$と交換する!

\vskip\baselineskip

(5.2.10)の二つの項の相対位相をどう選ぶかは純粋に慣習の問題である.しかし,その慣習は一度採用し固定したら,この粒子に対するスカラー場$\phi(x)$が相互作用ハミルトニアン密度の中に現れるときには,常に使わればならない.例えば,相互作用密度が場(5.2.10)ばかりでなく,$\alpha$を任意の位相とする\uwave{同じ粒子}に対する別のスカラー場
\begin{align*}
\tilde{\phi}(x)=e^{i\alpha}\phi^+(x)+e^{-i\alpha}\phi^{+\dagger}(x)=\tilde{\phi}^\dagger(x)
\end{align*}
を含むとする.この$\tilde{\phi}(x)$は,$\phi$と同様,$x-y$が空間的なとき$\tilde{\phi}(x)$が$\tilde{\phi}(y)$と交換する
\begin{align*}
[\tilde{\phi}(x),\tilde{\phi}(y)]=0
\end{align*}
という意味で,因果律を満たしている.しかし$\tilde{\phi}$は空間的な距離にある$\phi(y)$とは交換しない!
\begin{align*}
[\tilde{\phi}(x),\phi(y)]_-=&e^{i\alpha}[\phi^+(x),\phi^-(y)]_-+e^{-i\alpha}[\phi^-(x),\phi^+(y)]_- \\
=&e^{i\alpha}[\phi^+(x),\phi^-(y)]_--e^{-i\alpha}[\phi^+(y),\phi^-(x)]_- \\
=&(e^{i\alpha}-e^{-i\alpha})\Delta_+(x-y) \neq 0
\end{align*}
したがって,同じ理論(ある一つ固定された$\mc{H}_I(x)$で記述される理論)の中で,これら二つの場を両方とも用いることはできない.(もちろん,上で定義した粒子の種類$n$に対して異なる粒子種類$\bar{n}\neq n$を生成・消滅させる場$\tilde{\phi}(x)$については$\delta_{\bar{n}n}=0$より(4.2.5)は自動的に交換し$[\phi(x),\tilde{\phi}(y)]=0$が自動的に満たされるから,ある一つの場についての慣習を固定しても別の粒子種類の場についての慣習は独立に選ぶことができる.あくまで,同じ粒子を記述する場は一意的でなければならないというだけである.)


\vskip\baselineskip


もし$\phi(x)$によって消滅・生成する粒子が電荷のような保存するゼロでない量子数を持っているなら,$\mc{H}_I(x)$の各項が演算子$a(\mathbf{p})$と$a^\dagger(\mathbf{p})$を等しい数\uwave{だけ}含むときに限り$\mc{H}_I(x)$は量子数を保存$[Q,\mc{H}_I(x)]=0$する(p271参照).しかしこれは,$\mc{H}_I(x)$が$\phi(x)=\phi^+(x)+\phi^{+\dagger}(x)$の多項式として作られているなら不可能である.(素朴に考えて,例えば$\phi(x)^2$を展開すると
\begin{align*}
\phi(x)^2=\phi^+(x)^2+\phi^+(x)\phi^-(x)+\phi^-(x)\phi^+(x)+\phi^-(x)^2
\end{align*}
$a(\mathbf{p})$と$a^\dagger(\mathbf{p})$が同数含まれる項(第2,3項目)以外にも$a^\dagger(\mathbf{p})$が二個含まれているが$a(\mathbf{p})$が含まれていない項(第4項目)などの項が出てくることになる.)別の言い方をすれば,$\mc{H}_I(x)$が電荷(あるいは他の対称性の生成子)の演算子$Q$と交換するためには,$Q$との交換関係が単純な場から$\mc{H}_I(x)$が作られていれば十分である.これは$\phi^+(x)$およびその共役場$\phi^-(x)$については
\begin{align*}
[Q,\phi^+(x)]_{-}=-q\phi^+(x) \\
[Q,\phi^{+\dagger}(x)]_-=+q\phi^{+\dagger}(x)
\end{align*}
と正しいが,自己共役場(5.2.8)については正しくない.実際
\begin{align*}
[Q,\phi(x)]=&[Q,\phi^+(x)]+[Q,\phi^{+\dagger}(x)] \\
=&-q\phi^+(x)+q\phi^-(x)\neq -q\phi(x) 
\end{align*}
となる.したがって,ゼロでない保存量子数をもつ粒子に関する場は自己共役場として$\mc{H}_I(x)$に用いることはできない.\par
この問題を取り扱うためには,等しい質量$m$を持つが電荷はそれぞれ$+q$と$-q$の2個のスピンゼロのボゾンが存在すると仮定しなければならない!$\phi^+(x)$と$\phi^{+c}(x)$がこれら二つの粒子の消滅場
\begin{align*}
\phi^+(x)=&\int \frac{d^3\mathbf{p}}{(2\pi)^{3/2} \sqrt{2p^0}}a(\mathbf{p})e^{ip\cdot x} \\
\phi^{+c}(x)=&\int \frac{d^3\mathbf{p}}{(2\pi)^{3/2} \sqrt{2p^0}}a^{c}(\mathbf{p})e^{ip\cdot x}
\end{align*}
を表すとすると\footnote{$a^c$は,$+q$を持つ粒子の種類を$n$とし,それに対する電荷$+q$の粒子の種類を$n^c$と書いたときの消滅演算子$a(\mathbf{p},n^c)=a^c(\mathbf{p})$のことである.添え字$c$は荷電共役を表す.粒子の種類が二種類存在するから,係数関数$u(0,0,n),u(0,0,n^c)$は粒子の種類に依存する複素定数だけ自由度がある.しかしその定数は$u$のスケールだけの自由度で吸収することができる.そうしなくとも,結局線形結合の係数$\kappa,\lambda$の再定義で吸収される.これはこの節以降のスカラー以外の場でも同様である.これができるのは,生成場と消滅場がそれぞれ1種類のみの粒子の生成・消滅を表すからであり,そうでなければ$n$に関する行列があらわれこの操作ができない.},
\begin{align*}
[Q,a(\mathbf{p})]_-=&-q a(\mathbf{p}) \\
[Q,a(\mathbf{p})]_-=&+qa^c(\mathbf{p})
\end{align*}
より
\begin{align*}
[Q,\phi^+(x)]_-=&-q\phi^+(x) \\
[Q,\phi^{+c}(x)]_-=&+q\phi^{+c}(x)
\end{align*}
となる.したがって$\phi(x)$を線形結合
\begin{align*}
\phi(x):=\kappa \phi^+(x)+\lambda \phi^{+c\dagger}(x)
\end{align*}
と定義すると,それは明らかに単独の$\phi^+(x)$と同じ交換関係
\begin{align*}
[Q,\phi(x)]_-=-q\phi(x)
\end{align*}
を持つ.すると,$\phi(x)$とその共役場$\phi^\dagger(y)$の交換子または反交換子は空間的な距離では
\begin{align*}
[\phi(x),\phi^\dagger(y)]_{\mp}=&[\kappa \phi^+(x)+\lambda \phi^{+c\dagger}(x),\kappa^* \phi^{+\dagger}(y)+\lambda^* \phi^{+c}(y)]_{\mp} \\
=&|\kappa|^2[\phi^+(x),\phi^{+\dagger}(y)]_{\mp}+\kappa\lambda^*[\phi^+(x),\phi^{+c}(y)]_{\mp} \\
&+\lambda \kappa^*[\phi^{+c\dagger}(x),\phi^{+\dagger}(y)]_{\mp} +|\lambda|^2[\phi^{+c\dagger}(x),\phi^{+c}(y)]_{\mp} \\
=&|\kappa|^2[\phi^+(x),\phi^{+\dagger}(y)]_{\mp}+|\lambda|^2[\phi^{+c\dagger}(x),\phi^{+c}(y)]_{\mp} \\
=&|\kappa|^2 \Delta_+(x-y) \mp |\lambda|^2 \Delta_+(y-x) \\
=&(|\kappa|^2\mp |\lambda|^2) \Delta_+(x-y)
\end{align*}
ここで
\begin{align*}
[\phi^+(x),\phi^{+c}(y)]_{\mp}=&\int \frac{d^3\mathbf{p}}{(2\pi)^{3/2} \sqrt{2p^0}}\int \frac{d^3\mathbf{q}}{(2\pi)^{3/2} \sqrt{2q^0}}e^{ip\cdot x}e^{iq\cdot y} [a(\mathbf{p}),a^{c}(\mathbf{q})]_{\mp} \\
=&0 \\
[\phi^{+c\dagger}(x),\phi^{+\dagger}(y)]_{\mp}=&\int \frac{d^3\mathbf{p}}{(2\pi)^{3/2} \sqrt{2p^0}}\int \frac{d^3\mathbf{q}}{(2\pi)^{3/2} \sqrt{2q^0}}e^{-ip\cdot x}e^{-iq\cdot y} [a^{c\dagger}(\mathbf{p}),a^\dagger(\mathbf{q})]_{\mp} \\
=&0 \\
[\phi^+(x),\phi^{+\dagger}(y)]_{\mp}=&[\phi^+(x),\phi^{-}(y)]_{\mp} \\
=&\int \frac{d^3\mathbf{p}}{(2\pi)^{3/2} \sqrt{2p^0}}\int \frac{d^3\mathbf{q}}{(2\pi)^{3/2} \sqrt{2q^0}}e^{ip\cdot x}e^{-iq\cdot y} [a(\mathbf{p}),a^{c}(\mathbf{q})]_{\mp} \\
=&\int \frac{d^3\mathbf{p}}{(2\pi)^{3/2} \sqrt{2p^0}}\int \frac{d^3\mathbf{q}}{(2\pi)^{3/2} \sqrt{2q^0}}e^{ip\cdot x}e^{-iq\cdot y} \delta^3(\mathbf{p}-\mathbf{q}) \\
=&\int \frac{d^3\mathbf{p}}{(2\pi)^3 2p^0}e^{ip\cdot (x- y)}=\int \frac{d^3\mathbf{p}}{(2\pi)^3 2\sqrt{\mathbf{p}^2+m^2}}e^{ip\cdot (x- y)} \\
=&\Delta_+(x-y) \\
[\phi^{+c\dagger}(x),\phi^{+c}(y)]_{\mp}=&[\phi^{-c}(x),\phi^{+c}(y)]_{\mp} \\
=&\int \frac{d^3\mathbf{p}}{(2\pi)^{3/2} \sqrt{2p^0}}\int \frac{d^3\mathbf{q}}{(2\pi)^{3/2} \sqrt{2q^0}}e^{-ip\cdot x}e^{+iq\cdot y} [a^{c\dagger}(\mathbf{p}),a^{c}(\mathbf{q})]_{\mp} \\
=&\mp \int \frac{d^3\mathbf{p}}{(2\pi)^{3/2} \sqrt{2p^0}}\int \frac{d^3\mathbf{q}}{(2\pi)^{3/2} \sqrt{2q^0}}e^{-ip\cdot x}e^{+iq\cdot y} [a^{c}(\mathbf{q}),a^{c\dagger}(\mathbf{p})]_{\mp} \\
=&\mp \int \frac{d^3\mathbf{p}}{(2\pi)^{3/2} \sqrt{2p^0}}\int \frac{d^3\mathbf{q}}{(2\pi)^{3/2} \sqrt{2q^0}}e^{ip\cdot x}e^{-iq\cdot y}\delta^3(\mathbf{p}-\mathbf{q}) \\
=&\mp \int \frac{d^3\mathbf{p}}{(2\pi)^3 2p^0}e^{ip\cdot (y- x)}=\int \frac{d^3\mathbf{p}}{(2\pi)^3 2\sqrt{\mathbf{p}^2+m_c^2}}e^{ip\cdot (y- x)} \\
=&\mp \Delta_+(y-x)
\end{align*}
となることを用いた.(後者二つは(5.2.6)から明らかだが,一応陽に示した.$m_c$は電荷$-q$の粒子の質量である.この二つの$\Delta_+$が等しい関数であるためには,$m=m_c$でなければならない.したがって,$\phi^+(x)$で消滅させられる電荷$+q$の粒子と,$\phi^{+c}(x)$で消滅させられる電荷$-q$の粒子は同じ質量を持つことを仮定する.)一方,$\phi^+$と$\phi^{+c\dagger}$は異なる粒子を消滅・生成するので,$\phi(x)$と$\phi(y)$は全ての$x,y$に対して自動的に交換または反交換する.実際,
\begin{align*}
[\phi(x),\phi(y)]_{\mp}=&[\kappa \phi^+(x)+\lambda \phi^{+c\dagger}(x),\kappa \phi^{+}(y)+\lambda \phi^{+c\dagger}(y)]_{\mp} \\
=&\kappa^2 [\phi^+(x),\phi^+(y)]_{\mp}+\kappa\lambda[\phi^+(x),\phi^{+c\dagger}(y)]_{\mp} \\
&+\kappa\lambda [\phi^{+c\dagger}(x),\phi^+(y)]_{\mp}+\lambda^2[\phi^{+c\dagger}(x),\phi^{+c\dagger}(y)]_{\mp} \\
=&0
\end{align*}
ここで
\begin{align*}
[\phi^+(x),\phi^+(y)]_{\mp}=&\int \frac{d^3\mathbf{p}}{(2\pi)^{3/2} \sqrt{2p^0}}\int \frac{d^3\mathbf{q}}{(2\pi)^{3/2} \sqrt{2q^0}}e^{ip\cdot x}e^{iq\cdot y} [a(\mathbf{p}),a(\mathbf{q})]_{\mp} \\
=&0 \\
[\phi^+(x),\phi^{+c\dagger}(y)]_{\mp}=&\int \frac{d^3\mathbf{p}}{(2\pi)^{3/2} \sqrt{2p^0}}\int \frac{d^3\mathbf{q}}{(2\pi)^{3/2} \sqrt{2q^0}}e^{ip\cdot x}e^{-iq\cdot y} [a(\mathbf{p}),a^{c\dagger}(\mathbf{q})]_{\mp} \\
=&0 \\
[\phi^{+c\dagger}(x),\phi^{+}(y)]_{\mp}=&\int \frac{d^3\mathbf{p}}{(2\pi)^{3/2} \sqrt{2p^0}}\int \frac{d^3\mathbf{q}}{(2\pi)^{3/2} \sqrt{2q^0}}e^{-ip\cdot x}e^{iq\cdot y} [a^{c\dagger}(\mathbf{p}),a(\mathbf{q})]_{\mp} \\
=&0 \\
[\phi^{+c\dagger}(x),\phi^{+c\dagger}(y)]_{\mp}=&\int \frac{d^3\mathbf{p}}{(2\pi)^{3/2} \sqrt{2p^0}}\int \frac{d^3\mathbf{q}}{(2\pi)^{3/2} \sqrt{2q^0}}e^{-ip\cdot x}e^{-iq\cdot y} [a^{c\dagger}(\mathbf{p}),a^{c\dagger}(\mathbf{q})]_{\mp} \\
=&0
\end{align*}
となることを用いた.もちろんこの両辺をエルミート共役すれば
\begin{align*}
[\phi^\dagger(x),\phi^\dagger(y)]_{\mp}=0
\end{align*}
も得られる.フェルミ統計はここでも再び排除される.なぜなら,$\phi(x)$が$\phi^\dagger(y)$が空間的な距離で\uwave{反交換}することは,下の符号を選ぶことになるから$\kappa=\lambda=0$でなければ不可能であり,その場合は場が単に消えてしまう$\phi(x)=0$からである.よってスピンのない粒子はボゾンでなければならず,交換子の符号$\mp$は上のものが選ばれる.\par
ボーズ統計の場合には,複素場$\phi(x)$が$\phi^\dagger(y)$と空間的な距離で交換$[\phi(x),\phi^\dagger(y)]=0$するには,粒子と反粒子が等しい質量を持つ$m=m_c$とともに$|\kappa|^2=|\lambda|^2$となることが必要十分である.これら二つの粒子の1粒子状態の相対位相を再定義$\Phi_{\mathbf{p},n}\to e^{i\theta} \Phi_{\mathbf{p},n},\Phi_{\mathbf{p},n}\to e^{-i\varphi} \Phi_{\mathbf{p},n^c}$することにより,それに伴って生成消滅演算子の再定義$a(\mathbf{p})\to e^{-i\theta}a(\mathbf{p}),a^{c\dagger}(\mathbf{p})\to e^{-i\varphi}a^{c\dagger}(\mathbf{p})$が起きる.したがって$\kappa=|\kappa|e^{i\alpha},\lambda=|\lambda|e^{i\beta}$と書くと
\begin{align*}
\phi(x)=&|\kappa|e^{i(\alpha-\theta)} \int \frac{d^3\mathbf{p}}{(2\pi)^{3/2} \sqrt{2p^0}}a(\mathbf{p})e^{ip\cdot x}+|\lambda|e^{i(\beta-\varphi)} \int \frac{d^3\mathbf{p}}{(2\pi)^{3/2} \sqrt{2p^0}}a^{c\dagger}(\mathbf{p})e^{-ip\cdot x}
\end{align*}
再定義の位相を$\theta=\alpha,\varphi=\beta$と選ぶと
\begin{align*}
\phi(x)=&|\kappa|\int \frac{d^3\mathbf{p}}{(2\pi)^{3/2} \sqrt{2p^0}}a(\mathbf{p})e^{ip\cdot x}+|\lambda|\int \frac{d^3\mathbf{p}}{(2\pi)^{3/2} \sqrt{2p^0}}a^{c\dagger}(\mathbf{p})e^{-ip\cdot x} \\
=&|\kappa|\left(\int \frac{d^3\mathbf{p}}{(2\pi)^{3/2} \sqrt{2p^0}}a(\mathbf{p})e^{ip\cdot x}+\int \frac{d^3\mathbf{p}}{(2\pi)^{3/2} \sqrt{2p^0}}a^{c\dagger}(\mathbf{p})e^{-ip\cdot x}\right)
\end{align*}
となり,場$\phi$のスケールの再定義により$|\kappa|$を取り除くことができ
\begin{align*}
\phi(x)=&\phi^+(x)+\phi^{+c\dagger}(x) \\
=&\int \frac{d^3\mathbf{p}}{(2\pi)^{3/2} \sqrt{2p^0}}a(\mathbf{p})e^{ip\cdot x}+\int \frac{d^3\mathbf{p}}{(2\pi)^{3/2} \sqrt{2p^0}}a^{c\dagger}(\mathbf{p})e^{-ip\cdot x} \\
=&\int \frac{d^3\mathbf{p}}{(2\pi)^{3/2} \sqrt{2p^0}}\Bigl[a(\mathbf{p})e^{ip\cdot x}+a^{c\dagger}(\mathbf{p})e^{-ip\cdot x}\Bigr]
\end{align*}
となる.これが因果律を満たす実質的に唯一のスカラー場である!この式は,自分自身の反粒子である純粋に中性なスピンゼロ粒子(この場合$a^c(\mathbf{p})=a(\mathbf{p})$で実スカラー場)についても,反粒子が自分とは異なっている粒子(その場合$a^c(\mathbf{p})\neq a(\mathbf{p})$で複素場)についても両方に使える.(5.1.6)(5.1.7)より,この場のローレンツ変換性は
\begin{align*}
U_0(\Lambda) \phi(x) U_0^{-1}(\Lambda)=D(\Lambda^{-1}) \phi(\Lambda x)=\phi(\Lambda x)
\end{align*}
となる.

\vskip\baselineskip

後で使うために,ここで複素スカラー場とその共役場の交換子は(上の計算で$\kappa=\lambda=1$とおけばいいけど,一応陽に示すと)
\begin{align*}
[\phi(x),\phi^\dagger(y)]_-=&[\phi^+(x)+\phi^{+c\dagger}(x), \phi^{+\dagger}(y)+ \phi^{+c}(y)]_{-} \\
=&[\phi^+(x),\phi^{+\dagger}(y)]_{\mp}+[\phi^+(x),\phi^{+c}(y)]_{-} \\
&+[\phi^{+c\dagger}(x),\phi^{+\dagger}(y)]_{-} +[\phi^{+c\dagger}(x),\phi^{+c}(y)]_{-} \\
=&[\phi^+(x),\phi^{+\dagger}(y)]_{-}+[\phi^{+c\dagger}(x),\phi^{+c}(y)]_{-} \\
=&\Delta_+(x-y) - \Delta_+(y-x) \\
=&\Delta(x-y)
\end{align*}
ただし
\begin{align*}
\Delta(x-y):=\Delta_+(x-y) - \Delta_+(y-x)=\int \frac{d^3\mathbf{p}}{(2\pi)^{3/2}2p^0}[e^{ip\cdot(x-y)}-e^{-ip\cdot(x-y)}]
\end{align*}
である.

\vskip\baselineskip


次に,この場に様々な反転対称変換を施した場合の効果を考察する.まず,4.2節の結果から,消滅・生成演算子の空間反転の効果は,$\eta$と$\eta^c$をそれぞれ粒子と反粒子の固有パリティとして,
\begin{align*}
\mathsf{P} a(\mathbf{p})\mathsf{P}^{-1} =&\eta^* a(-\mathbf{p}) \\
\mathsf{P} a^{c\dagger}(\mathbf{p}) \mathsf{P}^{-1}=&\eta^c a^{c\dagger}(-\mathbf{p})
\end{align*}
であることがわかる.これらの結果を消滅場(5.2.3)$\phi^+(x)$と生成場(5.2.4)の荷電共役場$\phi^{+c}(x)$に適用し,積分変数を$\mathbf{p}$から$-\mathbf{p}$に変更すると
\begin{align*}
\mathsf{P} \phi^+(x) \mathsf{P}^{-1} =&\int \frac{d^3\mathbf{p}}{(2\pi)^{3/2} \sqrt{2p^0}}\mathsf{P} a(\mathbf{p}) \mathsf{P}^{-1} e^{ip\cdot x} \\
=&\eta^*\int \frac{d^3\mathbf{p}}{(2\pi)^{3/2} \sqrt{2p^0}} a(-\mathbf{p}) e^{ip\cdot x} \\
=&\eta^*\int \frac{d^3\mathbf{p}}{(2\pi)^{3/2} \sqrt{2p^0}} a(-\mathbf{p}) e^{-i\mathbf{p} \cdot \mathbf{x}+ip^0 x^0} \\
=&\eta^*\int \frac{d^3\mathbf{p}}{(2\pi)^{3/2} \sqrt{2p^0}} a(\mathbf{p}) e^{+i\mathbf{p} \cdot \mathbf{x}+ip^0 x^0} \quad (\mathbf{p} \to -\mathbf{p}) \\
=&\eta^*\int \frac{d^3\mathbf{p}}{(2\pi)^{3/2} \sqrt{2p^0}} a(\mathbf{p}) e^{ip\cdot (\mc{P}x)} \\
=&\eta^* \phi^+(\mc{P}x) \\
\mathsf{P}\phi^{+c\dagger}(x)\mathsf{P}^{-1} =&\int \frac{d^3\mathbf{p}}{(2\pi)^{3/2} \sqrt{2p^0}}\mathsf{P} a^{c\dagger}(\mathbf{p}) \mathsf{P}^{-1} e^{-ip\cdot x} \\
=&\eta^c \int \frac{d^3\mathbf{p}}{(2\pi)^{3/2} \sqrt{2p^0}}a^{c\dagger}(-\mathbf{p}) e^{-ip\cdot x} \\
=&\eta^c \int \frac{d^3\mathbf{p}}{(2\pi)^{3/2} \sqrt{2p^0}}a^{c\dagger}(-\mathbf{p}) e^{+i\mathbf{p}\cdot \mathbf{x}-ip^0 x^0} \\
=&\eta^c \int \frac{d^3\mathbf{p}}{(2\pi)^{3/2} \sqrt{2p^0}}a^{c\dagger}(\mathbf{p}) e^{-i\mathbf{p}\cdot \mathbf{x}-ip^0 x^0} \quad (\mathbf{p}\to -\mathbf{p}) \\
=&\eta^c \int \frac{d^3\mathbf{p}}{(2\pi)^{3/2} \sqrt{2p^0}}a^{c\dagger}(\mathbf{p}) e^{-ip\cdot (\mc{P}x)} \\
=&\eta^c \phi^{+c\dagger}(\mc{P}x)
\end{align*}
が得られる\footnote{積分変数を反転させるときに,$d^3\mathbf{p}\to -d^3\mathbf{p}$としてマイナスを出さないように注意.積分範囲も$[-\infty,+\infty]\to [+\infty,-\infty]$と反転しているので,これを$[-\infty,+\infty]$へと戻すときに出てくるマイナスと打ち消しあう.}.ここで$\mc{P}x=(-\mathbf{x},x^0)$とおいた.一般にスカラー場$\phi(x)=\phi^+(x)+\phi^{+c}(x)$に空間反転を施すと
\begin{align*}
\mathsf{P}\phi(x) \mathsf{P}^{-1}=\eta^* \phi^+(\mc{P}x)+\eta^c \phi^{+c\dagger}(\mc{P}x)=:\phi_P(\mc{P}x)
\end{align*}
となり,元のスカラー場とは異なる場$\phi_P(x)=\eta^* \phi^+(x)+\eta^c \phi^{+c\dagger}(x)$が得られる.両方の場は個別に因果律を満たす.なぜなら空間的な$x,y$に対して
\begin{align*}
0=\mathsf{P}[\phi(x),\phi^\dagger(y)]_- \mathsf{P}^{-1}=&\mathsf{P}\phi(x)\mathsf{P}^{-1} \mathsf{P} \phi^\dagger(y) \mathsf{P}^{-1}-\mathsf{P}\phi^\dagger(y)\mathsf{P}^{-1} \mathsf{P} \phi(x) \mathsf{P}^{-1} \\
=&\phi_P(\mc{P}x) \phi_P^\dagger(\mc{P}y)-\phi_P^\dagger(\mc{P}y)\phi_P(\mc{P}x) \\
=&[\phi_P(\mc{P}x)\phi_P^\dagger(\mc{P}y)]_-
\end{align*}
となるから,
\begin{align*}
(\mc{P}x-\mc{P}y)^2=-(x^0-y^0)^2-((-\mathbf{x})-(-\mathbf{y}))^2=(x-y)^2>0
\end{align*}
より,$\phi_P(x)$は因果律を満たしている.(もちろん直接確認できる.パリティ位相因子は$|\eta|=|\eta^c|=1$なので,ボーズ統計の符号を選んでいるのでゼロになる.
\begin{align*}
[\phi_P(x),\phi^\dagger_P(y)]_-=&[\eta^* \phi^+(x)+\eta^c \phi^{+c\dagger}(x),\eta \phi^{+\dagger}(y)+\eta^{c*}\phi^{+c}(y)]_- \\
=&|\eta|^2[\phi^+(x),\phi^{+\dagger}(y)]_-+\eta^*\eta^{c*}[\phi^{+}(x),\phi^{+c}(y)]_- \\
&+\eta^{c} \eta [\phi^{+c\dagger}(x),\phi^{+\dagger}(y)]_-+|\eta_c|^2[\phi^{+c\dagger}(x),\phi^{+c}(y)]_- \\
=&[\phi^+(x),\phi^{+\dagger}(y)]_-+[\phi^{+c\dagger}(x),\phi^{+c}(y)]_- \\
=&\Delta_+(x-y)-\Delta_+(y-x) \\
=&0
\end{align*}
となる.)他にも
\begin{align*}
[\phi(x),\phi_P(y)]_-=&0 \\
[\phi^\dagger(x),\phi_P^\dagger(y)]_-=&0
\end{align*}
が自動的になりたつ.しかし,$\phi(x)$と$\phi^\dagger_P(x)$は一般の位相$\eta,\eta^c$では空間的なときに交換しない.
\begin{align*}
[\phi(x),\phi^\dagger_P(y)]=&[\phi^+(x)+\phi^{+c\dagger}(x),\eta \phi^{+\dagger}(y)+\eta^{c*}\phi^{+c}(y)]_- \\
=&\eta[\phi^+(x),\phi^{+\dagger}(y)]_-+\eta^{c*}[\phi^{+}(x),\phi^{+c}(y)]_- \\
&+ \eta [\phi^{+c\dagger}(x),\phi^{+\dagger}(y)]_-+\eta^{c*}[\phi^{+c\dagger}(x),\phi^{+c}(y)]_- \\
=&\eta[\phi^+(x),\phi^{+\dagger}(y)]_-+\eta^{c*}[\phi^{+c\dagger}(x),\phi^{+c}(y)]_- \\
=&\eta \Delta_+(x-y)-\eta^{c*}\Delta_+(y-x) \\
=&(\eta -\eta^{c*})\Delta_+(x-y) \\
\neq& 0
\end{align*}
もし$\phi$と$\phi^\dagger_P$が同じ相互作用の中に現れると,これにより因果律(5.1.3)が満たすことができず問題が起きる\footnote{これはp277で位相が慣習の問題だといった議論からも理解できる.場の各項の位相を変化させたものは,空間的なときは自分自身と交換し因果律を満たすことができるが,同じ相互作用の中で両方の場を用いることができない.$\eta$は位相因子だから,そのときの議論がそのまま使える.}.ローレンツ不変性をパリティ保存および相互作用のエルミート性とともに保持する唯一の方法は,$\phi_P$が$\phi$に比例する,つまり
\begin{align*}
\eta^c=\eta^*
\end{align*}
を要求することである.すなわち,スピンゼロ粒子とその反粒子を一個ずつ含む状態の固有パリティは
\begin{align*}
\mathsf{P} \Phi_{\mathbf{p} n;\mathbf{q} n^c}=&\eta \eta^c\Phi_{\mc{P}\mathbf{p} n;\mc{P}\mathbf{q} n^c} \\
=&\eta \eta^*\Phi_{\mc{P}\mathbf{p} n;\mc{P}\mathbf{q} n^c} \\
=&+\Phi_{\mc{P}\mathbf{p} n;\mc{P}\mathbf{q} n^c} \quad \because |\eta|=1
\end{align*}
となり偶でなければならない.すると単に
\begin{align*}
\mathsf{P}\phi(x)\mathsf{P}^{-1}=\eta^* \phi(\mc{P}x)
\end{align*}
となる.これらの結果は,またスピンゼロ粒子が自分自身の反粒子のときにもなりたつ.その場合は$\eta^c=\eta$であり,そのような粒子のパリティは実$\eta^*=\eta^c=\eta$で,さらに$\eta^2=\eta\eta^c=+1$より$\eta=\pm 1$であることを意味する.したがって実スカラー場で記述される粒子はスカラー粒子か擬スカラー粒子である.


\vskip\baselineskip



荷電共役もほぼ同様に取り扱うことができる.4.2節の結果から,1粒子状態に荷電共役の演算を施したときに現れる位相を$\xi$およびその反粒子では$\xi^c$として
\begin{align*}
\mathsf{C} a(\mathbf{p})\mathsf{C}^{-1} =&\xi^* a^c(\mathbf{P}) \\
\mathsf{C} a^c(\mathbf{p})\mathsf{C}^{-1}=&\xi^c a^\dagger(\mathbf{P})
\end{align*}
である.よって
\begin{align*}
\mathsf{C}\phi^+(x)\mathsf{C}^{-1}=&\int \frac{d^3\mathbf{p}}{(2\pi)^{3/2} \sqrt{2p^0}}\mathsf{C} a(\mathbf{p}) \mathsf{C}^{-1} e^{ip\cdot x} \\
=&\xi^* \int \frac{d^3\mathbf{p}}{(2\pi)^{3/2} \sqrt{2p^0}}a^c(\mathbf{p}) e^{ip\cdot x} \\
=&\xi^* \phi^{+c}(x) \\
\mathsf{C}\phi^{+c\dagger}(x)\mathsf{C}^{-1}=&\int \frac{d^3\mathbf{p}}{(2\pi)^{3/2} \sqrt{2p^0}}\mathsf{C} a^{c\dagger}(\mathbf{p}) \mathsf{C}^{-1} e^{-ip\cdot x} \\
=&\xi^c \int \frac{d^3\mathbf{p}}{(2\pi)^{3/2} \sqrt{2p^0}}a^\dagger(\mathbf{p}) e^{-ip\cdot x} \\
=&\xi^c \phi^{+\dagger}(x)
\end{align*}
となる.すると
\begin{align*}
\mathsf{C}\phi(x)\mathsf{C}^{-1}=\xi^* \phi^{+c}(x)+\xi^c\phi^{+\dagger}(x)=:\phi_C(x)
\end{align*}
となる.これは再び$[\phi_C(x),\phi^\dagger_C(y)]_-=0$であり自分自身では因果律を満たし,他にも
\begin{align*}
[\phi^\dagger(x),\phi_C(y)]_-=0
\end{align*}
を自動的に満たす.しかし
\begin{align*}
[\phi(x),\phi_C(y)]_-=&[\phi^{+}(x)+\phi^{+c\dagger}(x),\xi^* \phi^{+c}(y)+\xi^c\phi^{+\dagger}(y)]_- \\
=&\xi^c \Delta_+(x-y) -\xi^* \Delta_+(y-x) \\
=&(\xi^c-\xi^*)\Delta_+(x-y)
\end{align*}
となり,$\phi$と$\phi_C(x)$が同じ相互作用に現れるためには
\begin{align*}
\xi^c=\xi^*
\end{align*}
が満たされていなければならない.つまり,スピンゼロ粒子とその反粒子一個ずつからなる状態の固有荷電共役パリティは
\begin{align*}
\mathsf{C} \Phi_{\mathbf{p} n;\mathbf{q} n^c}=&\xi \xi^c\Phi_{\mathbf{p} n^c;\mathbf{q} n} \\
=&\xi \xi^*\Phi_{\mathbf{p} n^c;\mathbf{q} n} \\
=&+\Phi_{\mathbf{p} n^c;\mathbf{q} n} \quad \because |\xi|=1
\end{align*}
となって偶となる.このとき単に
\begin{align*}
\mathsf{C}\phi(x)\mathsf{C}^{-1}=\xi^*\phi^\dagger(x)
\end{align*}
となる.再びこれらの結果は粒子が自分自身の反粒子であり,$\xi^c=\xi$の場合にもなりたつ.この場合荷電共役パリティは通常のパリティと同様に実$\xi^*=\xi^c=\xi$で,$\xi^2=\xi \xi^c=+1$より$\xi=\pm 1$でなければならない.


\vskip\baselineskip


最後に時間反転に移る.4.2節から,$j=0,\sigma=0$なので
\begin{align*}
\mathsf{T} a(\mathbf{p})\mathsf{T}^{-1}=&\zeta^* a(-\mathbf{p}) \\
\mathsf{T} a^{c\dagger}(\mathbf{p})\mathsf{T}^{-1}=&\zeta^c a^{c\dagger}(-\mathbf{p})
\end{align*}
である.$\mathsf{T}$が反線形・反ユニタリーであることを思い出し
\begin{align*}
\mathsf{T} \phi^+(x) \mathsf{T}^{-1}=&\int \frac{d^3\mathbf{p}}{(2\pi)^{3/2} \sqrt{2p^0}}\mathsf{T} a(\mathbf{p}) e^{ip\cdot x}\mathsf{T}^{-1} \\
=&\int \frac{d^3\mathbf{p}}{(2\pi)^{3/2} \sqrt{2p^0}}\mathsf{T} a(\mathbf{p}) \mathsf{T}^{-1} e^{-ip\cdot x} \quad \because 反線形性\\
=&\zeta^*\int \frac{d^3\mathbf{p}}{(2\pi)^{3/2} \sqrt{2p^0}}a(-\mathbf{p}) e^{-ip\cdot x} \\
=&\zeta^*\int \frac{d^3\mathbf{p}}{(2\pi)^{3/2} \sqrt{2p^0}}a(-\mathbf{p}) e^{ip^0 x^0-i\mathbf{p}\cdot \mathsf{x}} \\
=&\zeta^*\int \frac{d^3\mathbf{p}}{(2\pi)^{3/2} \sqrt{2p^0}}a(\mathbf{p}) e^{ip^0 x^0+i\mathbf{p}\cdot \mathsf{x}} \quad (\mathbf{p}\to -\mathbf{p}) \\
=&\zeta^*\int \frac{d^3\mathbf{p}}{(2\pi)^{3/2} \sqrt{2p^0}}a(\mathbf{p}) e^{ip\cdot (-\mc{P}x)} \\
=&\zeta^* \phi^+(-\mc{P}x)=\zeta^* \phi^+(\mc{T}x) \\
\mathsf{T} \phi^{+c\dagger}(x) \mathsf{T}^{-1}=&\int \frac{d^3\mathbf{p}}{(2\pi)^{3/2} \sqrt{2p^0}}\mathsf{T} a^{c\dagger}(\mathbf{p}) e^{-ip\cdot x}\mathsf{T}^{-1} \\
=&\int \frac{d^3\mathbf{p}}{(2\pi)^{3/2} \sqrt{2p^0}}\mathsf{T} a^{c\dagger}(\mathbf{p})\mathsf{T}^{-1} e^{ip\cdot x} \\
=&\zeta^c\int \frac{d^3\mathbf{p}}{(2\pi)^{3/2} \sqrt{2p^0}} a^{c\dagger}(-\mathbf{p}) e^{ip\cdot x} \\
=&\zeta^c\int \frac{d^3\mathbf{p}}{(2\pi)^{3/2} \sqrt{2p^0}} a^{c\dagger}(-\mathbf{p}) e^{-ip^0x^0+i\mathbf{p}\cdot \mathbf{x}} \\
=&\zeta^c\int \frac{d^3\mathbf{p}}{(2\pi)^{3/2} \sqrt{2p^0}} a^{c\dagger}(-\mathbf{p}) e^{-ip^0x^0-i\mathbf{p}\cdot \mathbf{x}} \quad (\mathbf{p}\to -\mathbf{p}) \\
=&\zeta^c\int \frac{d^3\mathbf{p}}{(2\pi)^{3/2} \sqrt{2p^0}} a^{c\dagger}(-\mathbf{p}) e^{-ip\cdot(-\mc{P}x)} \\
=&\zeta^c \phi^{+c\dagger}(-\mc{P}x)=\zeta^c\phi^{+c\dagger}(\mc{T}x)
\end{align*}
がわかる.したがって
\begin{align*}
\mathsf{T}\phi(x)\mathsf{T}^{-1}=\zeta^* \phi^+(\mc{T}x)+\zeta^c \phi^{+c\dagger}(\mc{T}x)=:\phi_T(\mc{T}x)
\end{align*}
となる.再びこれは空間的な$x,y$のときに
\begin{align*}
[\phi(x),\phi^\dagger_T(y)]_-=&[\phi^{+}(x)+\phi^{+c\dagger}(x),\zeta \phi^{+\dagger}(y)+\zeta^{c*} \phi^{+c}(y)] \\
=&\zeta\Delta_+(x-y)-\zeta^{c*}\Delta_+(y-x) \\
=&(\zeta-\zeta^{c*})\Delta_+(x-y)
\end{align*}
を与え,したがってやはり
\begin{align*}
\zeta^c=\zeta^*
\end{align*}
でなければならない.そうすると
\begin{align*}
\mathsf{T}\phi(x)\mathsf{T}^{-1}=\zeta^* \phi(\mc{T}x)
\end{align*}
となる.


\vskip\baselineskip


最後に,場の係数について一言述べておく.Peskinなどの他の本でのスカラー場の定義を参照すると
\begin{align*}
\int \frac{d^3\mathbf{p}}{(2\pi)^3 \sqrt{2p^0}}\Bigl[a(\mathbf{p})e^{ip\cdot x}+a^{c\dagger}(\mathbf{p})e^{-ip\cdot x}\Bigr]
\end{align*}
であり,係数の$(2\pi)^{-3/2}$部分がWeinebrgと違う.この違いは,1粒子状態の規格化の違いにある.3.4節のノートの最後で述べたように,WeinbergとPeskinの1粒子状態の定義は
\begin{align*}
\phi(x)=\Psi_{p,\sigma}^{\mathrm{Peskin}}=\sqrt{(2\pi)^32E_p}\Psi_{p,\sigma}^{\mathrm{Weinberg}}
\end{align*}
という違いがあり,したがって生成消滅演算子も
\begin{align*}
a_\mathrm{P}=\sqrt{(2\pi)^3 2p^0}a_{\mathrm{W}} ,\quad a^{c\dagger}_\mathrm{P}=\sqrt{(2\pi)^3 2p^0}a^{c\dagger}_{\mathrm{W}}
\end{align*}
という違いが存在する.(PとWはそれぞれPeskin流とWeinberg流を指す.)つまり余分に$(2\pi)^{3/2}\sqrt{2p^0}$で割らなくてはならないことがわかる.しかし,そうすると$(2\pi)^{-3}$の因子は一致するが,$2p^0$の因子が合わなくなる.
\begin{align*}
\phi(x)=\int \frac{d^3\mathbf{p}}{(2\pi)^{3/2} \sqrt{2p^0}}\Bigl[a_{\mathrm{W}}(\mathbf{p})e^{ip\cdot x}+a^{c\dagger}_{\mathrm{W}}(\mathbf{p})e^{-ip\cdot x}\Bigr]=\int \frac{d^3\mathbf{p}}{(2\pi)^{3} 2p^0}\Bigl[a_{\mathrm{P}}(\mathbf{p})e^{ip\cdot x}+a^{c\dagger}_{\mathrm{P}}(\mathbf{p})e^{-ip\cdot x}\Bigr] ?
\end{align*}
これは,Peskin流のnotationにしたのに,$u(\mathbf{p}),v(\mathbf{p})$にそのnotationが反映されていないのが原因である.Weinberg流のスカラー場の導出において因子$1/\sqrt{2p^0}$は(2.5.18)で$N(p)=\sqrt{m/p^0}$と定義したことで(2.5.11)そして最終的に(5.1.21)で$\sqrt{m/p^0}$が出てくることが起源だったのを思い出そう.Peskinではこれを$N(p)=\sqrt{(2\pi)^3 2m}$という定数で与えているから,(5.1.21)の平方根部分は$1$になってしまい,$1/\sqrt{2p^0}$が最初から出てこない.つまり
\begin{align*}
u_{\bar{\ell}}(\mathbf{p},\sigma,n)=&\sum_{\ell} D_{\bar{\ell}\ell}(L(p)) u_\ell (0,\sigma,n) \\
v_{\bar{\ell}}(\mathbf{p},\sigma,n)=&\sum_{\ell} D_{\bar{\ell}\ell}(L(p)) v_\ell (0,\sigma,n)
\end{align*}
となり,例えばスカラー場の場合$u(\mathbf{p})=v(\mathbf{p})=1$となる.($\sqrt{m}$もないので$1/\sqrt{2m}$も不要になるからである.)
\begin{align*}
\phi(x)=&\int \frac{d^3\mathbf{p}}{(2\pi)^{3/2}}\Bigl[u(\mathbf{p})a_{\mathrm{W}}(\mathbf{p})e^{ip\cdot x}+v(\mathbf{p})a^{c\dagger}_{\mathrm{W}}(\mathbf{p})e^{-ip\cdot x}\Bigr] \\
=&\int \frac{d^3\mathbf{p}}{(2\pi)^{3} \sqrt{2p^0}}\Bigl[u(\mathbf{p})a_{\mathrm{P}}(\mathbf{p})e^{ip\cdot x}+v(\mathbf{p})a^{c\dagger}_{\mathrm{P}}(\mathbf{p})e^{-ip\cdot x}\Bigr]
\end{align*}
よって$1/\sqrt{2p^0}$の部分も辻褄があう.これを繰り返せば,これ以降で議論する場についても同様にPeskin流の場と辻褄があう.


\newpage


\subsection{因果律を満たすベクトル場}
今度はその次に簡単な種類な場を取り上げる.それは斉次ローレンツ群の自明でない表現のうち,最も簡単な4元ベクトルとして変換する場である.その例として,質量がゼロでない粒子$W^{\pm}$と$Z^0$がある.これらの粒子は低エネルギーでは\footnote{$SU(2)\times U(1)$対称性が自発的に破れる程度に低エネルギー.それ以上に高エネルギーではゲージ粒子は質量ゼロとなり,質量のある4元ベクトル場で取り扱うことができず,5.9節で述べる方法で取り扱わなければならない.}そのような4元ベクトル場として記述され,最新の素粒子物理においてますます大きな役割を果たしている.よってこの例は単に教育的に興味があるだけではない.\par
前と同様,消滅場と生成場はそれぞれ1粒子のみを生成・消滅させ,さらにしばらくの間は同一の種類の粒子を生成・消滅させる場合を考える.その後,場が粒子およびそれと異なる反粒子の両方を記述する可能性を考える.

\vskip\baselineskip

ローレンツ群の4元ベクトル表現とは,表現行列$D(\Lambda)$の行と列は4成分のインデックス$\mu,\nu$などの添え字をもち
\begin{align*}
\tensor{D(\Lambda)}{^\mu_\nu}:=\tensor{\Lambda}{^\mu_\nu}
\end{align*}
であるような表現をいう.ベクトル場の消滅部分と生成部分は(5.1.17)(5.1.18)より
\begin{align*}
\phi^{+\mu}(x)=&\sum_{\sigma}\frac{1}{(2\pi)^{3/2}} \int d^3\mathbf{p} u^\mu (\mathbf{p},\sigma)a(\mathbf{p},\sigma)e^{ip\cdot x} \\
\phi^{-\mu}(x)=&\sum_{\sigma}\frac{1}{(2\pi)^{3/2}} \int d^3\mathbf{p} v^\mu (\mathbf{p},\sigma)a^\dagger(\mathbf{p},\sigma)e^{-ip\cdot x}
\end{align*}
と書かれる.任意の運動量に対する係数関数$u^\mu(\mathbf{p},\sigma)$と$v^\mu(\mathbf{p},\sigma)$は(5.1.21)と(5.1.22)より,ゼロ運動量での係数関数を用いて与えられ,それは今度は(5.3.1)より
\begin{align*}
u^\mu(\mathbf{p},\sigma)=&\sqrt{\frac{m}{p^0}} \sum_{\nu=0}^3\tensor{D(L(p))}{^\mu_\nu} u^{\nu}(0,\sigma) \\
=&\sqrt{\frac{m}{p^0}} \tensor{L(p)}{^\mu_\nu} u^{\nu}(0,\sigma) \\
v^\mu(\mathbf{p},\sigma)=&\sqrt{\frac{m}{p^0}} \sum_{\nu=0}^3 \tensor{D(L(p))}{^\mu_\nu} v^{\nu}(0,\sigma) \\
=&\sqrt{\frac{m}{p^0}} \tensor{L(p)}{^\mu_\nu} v^{\nu}(0,\sigma)
\end{align*}
となる.(時空の指標$\mu,\nu$と同様に通常のアインシュタインの記法を採用している.)また,ゼロ運動量での係数関数は条件(5.1.25)(5.1.26),すなわち
\begin{align*}
\sum_{\bar{\sigma}}u^\mu(0,\bar{\sigma})\mathbf{J}^{(j)}_{\bar{\sigma}\sigma}=\tensor{\bm{\mc{J}}}{^\mu_\nu}u^\nu(0,\sigma)
\end{align*}
と
\begin{align*}
-\sum_{\bar{\sigma}}v^\mu(0,\bar{\sigma})\mathbf{J}^{(j)*}_{\bar{\sigma}\sigma}=\tensor{\bm{\mc{J}}}{^\mu_\nu}v^\nu(0,\sigma)
\end{align*}
に従う.4元ベクトル表現における回転の生成子$\tensor{\bm{\mc{J}}}{^\mu_\nu}$は(5.3.1)と2.7節で行った計算により
\begin{align*}
\tensor{\Lambda}{^\mu_\nu}=&\delta^\mu_\nu+\tensor{\omega}{^\mu_\nu} \\
=&\tensor{\left[1+\frac{i}{2}\omega_{\rho\sigma}\mc{J}^{\rho\sigma}\right]}{^\mu_\nu} \quad (\tensor{(\mc{J}^{\rho\sigma})}{^\mu_\nu}:=i\delta^\rho_\nu \eta^{\mu\sigma}-i\delta^\sigma_\nu \eta^{\mu\rho}) \\
\tensor{(\mc{J}_i)}{^\mu_\nu}=&\frac{1}{2}\epsilon_{ijk}\tensor{(\mc{J}^{jk})}{^\mu_\nu} \\
=&\frac{1}{2}\epsilon_{ijk}(i\delta^j_\nu \eta^{\mu k}-i\delta^k_\nu \eta^{\mu j}) \\
\end{align*}
したがって
\begin{align*}
\tensor{(\mc{J}_k)}{^0_0}=&\tensor{(\mc{J}_k)}{^0_i}=\tensor{(\mc{J}_k)}{^i_0}=0 \\
\tensor{(\mc{J}_i)}{^\ell_m}=&\frac{1}{2}\epsilon_{ijk}(i\delta^j_m \eta^{\ell k}-i\delta^k_m \eta^{\ell j}) \\
=&\frac{1}{2}i\epsilon_{im\ell} -\frac{1}{2}i\epsilon_{i\ell m} \\
=&-i\epsilon_{\ell m i}
\end{align*}
と与えられる.ここで$i,j\cdots$などのラテン文字は$1,2,3$の範囲をとる.特に$\bm{\mc{J}}^2$は
\begin{align*}
\tensor{(\bm{\mc{J}}^2)}{^\mu_\nu}=&\sum_{i=1}^3 \tensor{(\mc{J}_i)}{^\mu_\rho}\tensor{(\mc{J}_i)}{^\rho_\nu} \\
=&\sum_{i,k=1}^3 \tensor{(\mc{J}_i)}{^\mu_k}\tensor{(\mc{J}_i)}{^k_\nu} \\
\therefore \quad \tensor{(\bm{\mc{J}}^2)}{^0_0}=&\tensor{(\bm{\mc{J}}^2)}{^i_0}=\tensor{(\bm{\mc{J}}^2)}{^0_i}=0 \\
\tensor{(\bm{\mc{J}}^2)}{^i_j}=&\sum_{k,m=1}^3 \tensor{(\mc{J}_k)}{^i_m}\tensor{(\mc{J}_k)}{^m_j} \\
=&-\sum_{k,m=1}^3\epsilon_{imk}\epsilon_{mjk} \\
=&-\sum_{m=1}^3 (\delta_{im}\delta_{mj}-\delta_{ij}\delta_{mm}) \quad \because \sum_{k=1}^3\epsilon_{ijk}\epsilon_{k\ell m}=\delta_{ik}\delta_{j\ell}-\delta_{i\ell}\delta_{jk} \\
=&+2\delta_{ij}
\end{align*}
の形をとる.すると(5.3.6)(5.3.7)から
\begin{align*}
\sum_{\bar{\sigma}}u^\mu(0,\bar{\sigma})(\mathbf{J}^{(j)})^2_{\bar{\sigma}\sigma}=&\sum_i\sum_{\bar{\sigma}\sigma'}u^\mu(0,\bar{\sigma})(J_i^{(j)})_{\bar{\sigma}\sigma'}(J_i^{(j)})_{\sigma'\sigma} \\
=&\sum_i\sum_{\sigma'}\tensor{(\mc{J}_i)}{^\mu_\rho}u^\rho(0,\sigma')(J_i^{(j)})_{\sigma'\sigma} \\
=&\sum_i \tensor{(\mc{J}_i)}{^\mu_\rho}\tensor{(\mc{J}_i)}{^\rho_\nu}u^\nu(0,\sigma) \\
=&\tensor{(\bm{\mc{J}}^2)}{^\mu_\nu}u^\nu(0,\sigma) \\
\therefore \quad \sum_{\bar{\sigma}}u^0(0,\bar{\sigma})(\mathbf{J}^{(j)})^2_{\bar{\sigma}\sigma}=&0 \\
\sum_{\bar{\sigma}}u^i(0,\bar{\sigma})(\mathbf{J}^{(j)})^2_{\bar{\sigma}\sigma}=&\tensor{(\bm{\mc{J}}^2)}{^i_\nu}u^\nu(0,\sigma) \\
=&\tensor{(\bm{\mc{J}}^2)}{^i_j}u^j(0,\sigma) \\
=&2\delta^i_{j}u^j(0,\sigma) \\
=&2u^i(0,\sigma)
\end{align*}
および
\begin{align*}
\sum_{\bar{\sigma}}v^\mu(0,\bar{\sigma})(\mathbf{J}^{(j)*})^2_{\bar{\sigma}\sigma}=&\sum_i\sum_{\bar{\sigma}\sigma'}v^\mu(0,\bar{\sigma})(J_i^{(j)*})_{\bar{\sigma}\sigma'}(J_i^{(j)*})_{\sigma'\sigma} \\
=&-\sum_i\sum_{\sigma'}\tensor{(\mc{J}_i)}{^\mu_\rho}v^\rho(0,\sigma')(J_i^{(j)*})_{\sigma'\sigma} \\
=&+\sum_i \tensor{(\mc{J}_i)}{^\mu_\rho}\tensor{(\mc{J}_i)}{^\rho_\nu}v^\nu(0,\sigma) \\
=&\tensor{(\bm{\mc{J}}^2)}{^\mu_\nu}v^\nu(0,\sigma) \\
\therefore \quad \sum_{\bar{\sigma}}v^0(0,\bar{\sigma})(\mathbf{J}^{(j)*})^2_{\bar{\sigma}\sigma}=&0 \\
\sum_{\bar{\sigma}}v^i(0,\bar{\sigma})(\mathbf{J}^{(j)*})^2_{\bar{\sigma}\sigma}=&\tensor{(\bm{\mc{J}}^2)}{^i_\nu}v^\nu(0,\sigma) \\
=&\tensor{(\bm{\mc{J}}^2)}{^i_j}v^j(0,\sigma) \\
=&2\delta^i_{j}v^j(0,\sigma) \\
=&2v^i(0,\sigma)
\end{align*}
が導かれる.また,よく知られた$SO(3)$代数のカシミア元$(\mathbf{J}^{(j)*})^2_{\bar{\sigma}\sigma}=j(j+1)\delta_{\bar{\sigma}\sigma}$を思い出すと,(5.3.12)(5.3.13)は
\begin{align*}
j(j+1)u^0(0,\sigma)=&0 \\
j(j+1)u^i(0,\sigma)=&2u^i(0,\sigma)
\end{align*}
を与え,(5.3.14)(5.3.15)は
\begin{align*}
j(j+1)v^0(0,\sigma)=&0 \\
j(j+1)v^i(0,\sigma)=&2v^i(0,\sigma)
\end{align*}
を与える.したがって,ベクトル場で記述される粒子のスピンには二つの可能性だけがあることがわかる.すなわち,$j=0$か,さもなければ$j=1$(よって$j(j+1)=2$)のどちらかである.これ以外では$u^\mu(0,\sigma)=v^\mu(0,\sigma)=0$となって,そのような場はゼロになってしまうことがわかる.$j=0$の場合,$u^0$と$v^0$だけがゼロでない値をとり,$u^i=v^i=0$である.$j=1$の場合,$\mathbf{p}=0$で空間成分$u^i(0,\sigma)$と$v^i(\mathbf{p})$がゼロでない値をとり,$u^0(0,\sigma)=v^0(0,\sigma)=0$となる.これら二つの可能性を以下で詳細に見る.


\vskip\baselineskip


\textbf{スピンゼロ}\par
この場合$j=0$であるから$\sigma=0$のみをとり,この添え字を省略する.場の規格化を適当に選ぶことで,ゼロでない唯一の成分$u^0(0)$と$v^0(0)$が
\begin{align*}
u^0(0)=&i\sqrt{\frac{m}{2}} \\
v^0(0)=&-i\sqrt{\frac{m}{2}}
\end{align*}
を持つようにできる.すると(5.3.4)(5.3.5)より一般の運動量について
\begin{align*}
u^\mu(\mathbf{p},\sigma)=&\sqrt{\frac{m}{p^0}} \tensor{L(p)}{^\mu_\nu} u^{\nu}(0,\sigma) \\
=&i\sqrt{\frac{m}{p^0}}\sqrt{\frac{m}{2}}\tensor{L(p)}{^\mu_\nu}\left(
\begin{matrix}
0 \\
0 \\
0 \\
1
\end{matrix}
\right) \\
=&i\sqrt{\frac{1}{2p^0}}\tensor{L(p)}{^\mu_\nu}\left(
\begin{matrix}
0 \\
0 \\
0 \\
m
\end{matrix}
\right) \\
=&i\sqrt{\frac{1}{2p^0}} p^\mu \quad \because L(p):(0,0,0,m)\mapsto p^\mu
\end{align*}
と
\begin{align*}
v^\mu(\mathbf{p},\sigma)=&\sqrt{\frac{m}{p^0}} \tensor{L(p)}{^\mu_\nu} v^{\nu}(0,\sigma) \\
=&\sqrt{\frac{m}{p^0}} \tensor{L(p)}{^\mu_\nu} v^{\nu}(0,\sigma) \\
=&-i\sqrt{\frac{m}{p^0}}\sqrt{\frac{m}{2}}\tensor{L(p)}{^\mu_\nu}\left(
\begin{matrix}
0 \\
0 \\
0 \\
1
\end{matrix}
\right) \\
=&-i\sqrt{\frac{1}{2p^0}}\tensor{L(p)}{^\mu_\nu}\left(
\begin{matrix}
0 \\
0 \\
0 \\
m
\end{matrix}
\right) \\
=&-i\sqrt{\frac{1}{2p^0}} p^\mu
\end{align*}
を得る.このベクトル消滅・生成場は
\begin{align*}
\phi^{+\mu}(x)=&\frac{1}{(2\pi)^{3/2}} \int d^3\mathbf{p} u^\mu (\mathbf{p})a(\mathbf{p})e^{ip\cdot x} \\
=&\int \frac{d^3\mathbf{p}}{(2\pi)^{3/2}\sqrt{2p^0}}a(\mathbf{p})ip^\mu e^{ip\cdot x} \\
=&\partial^\mu \int \frac{d^3\mathbf{p}}{(2\pi)^{3/2}\sqrt{2p^0}}a(\mathbf{p})e^{ip\cdot x} \\
=&\partial^\mu \phi^+(x) \\
\phi^{-\mu}(x)=&\frac{1}{(2\pi)^{3/2}} \int d^3\mathbf{p} v^\mu (\mathbf{p})a^\dagger(\mathbf{p})e^{-ip\cdot x} \\
=&\int \frac{d^3\mathbf{p}}{(2\pi)^{3/2}\sqrt{2p^0}} (-ip^\mu) a^\dagger(\mathbf{p})e^{-ip\cdot x} \\
=&\partial^\mu \int \frac{d^3\mathbf{p}}{(2\pi)^{3/2}\sqrt{2p^0}} a^\dagger(\mathbf{p})e^{-ip\cdot x} \\
=&\partial^\mu \phi^{-}(x)
\end{align*}
となり,前節で定義されたスピンゼロ粒子のスカラー消滅・生成場$\phi^\pm$の微分に他ならない.スピンゼロ粒子に対する因果律を満たすベクトル場もまた,ちょうど因果律を満たす実スカラー場の微分
\begin{align*}
\phi^\mu(x):=\phi^{+\mu}(x)+\phi^{-\mu}(x) =\partial^\mu \phi(x)
\end{align*}
であることは明らかである.これは実際,空間的な$x,y$に対して
\begin{align*}
[\partial_\mu \phi(x),\partial_\nu \phi(y)]_{\mp}-=&\frac{\partial}{\partial x^\mu} \frac{\partial}{\partial y^\nu} [\phi(x),\phi(y)]_{\mp} \\
=&\frac{\partial}{\partial x^\mu} \frac{\partial}{\partial y^\nu} \Bigl(\Delta_+(x-y)\mp \Delta_+(y-x)\Bigr) \\
=&0 \quad (\mp が - のとき)
\end{align*}
となる.もちろん符号は上が選ばれ,この場で記述される粒子はボーズ統計に従うことも明らかである.(電荷などの保存量子数を持つ場合,複素スカラー場の微分
\begin{align*}
\phi^\mu(x):=\partial^\mu \phi^{+}(x)+\partial^\mu \phi^{+c\dagger}(x)=\partial^\mu \phi(x)
\end{align*}
になる.)よってこの場合については,これ以上調べる必要はない.


\vskip\baselineskip


\textbf{スピン1}\par
(5.3.6)と(5.3.7)から,$j=1$の表現が(2.5.22)より
\begin{align*}
(J_3^{(1)})_{\bar{\sigma}\sigma}=\sigma \delta_{\bar{\sigma}\sigma}
\end{align*}
であるから
\begin{align*}
\sum_{\bar{\sigma}}u^\mu(0,\bar{\sigma})(J_3^{(1)})_{\bar{\sigma}\sigma}=&\tensor{(\mc{J}_3)}{^\mu_\nu}u^\nu(0,\sigma) \\
\sum_{\bar{\sigma}}u^\mu(0,\bar{\sigma})(J_3^{(1)})_{\bar{\sigma}0}=&\tensor{(\mc{J}_3)}{^\mu_\nu}u^\nu(0,0) \\
\sum_{\bar{\sigma}}u^i(0,\bar{\sigma})(J_3^{(1)})_{\bar{\sigma}0}=&\tensor{(\mc{J}_3)}{^i_j}u^j(0,0) \quad \because u^0(0,\sigma)=0\\
(\mathrm{LHS})=\sum_{\bar{\sigma}}u^i(0,\bar{\sigma})0\delta_{\bar{\sigma}0}=&0 \\
(\mathrm{RHS})=\tensor{(\mc{J}_3)}{^i_j}u^j(0,0)=&-i\epsilon_{ij3}u^j(0,0) \\
=&-i\epsilon_{i23}u^2(0,0)-i\epsilon_{i13}u^1(0,0)
\end{align*}
よって$i=1,2$について入れてみると
\begin{align*}
u^1(0,0)=u^2(0,0)=0
\end{align*}
がわかる.同様にして
\begin{align*}
-\sum_{\bar{\sigma}}v^\mu(0,\bar{\sigma})(J_3^{(1)*})_{\bar{\sigma}\sigma}=&\tensor{(\mc{J}_3)}{^\mu_\nu}v^\nu(0,\sigma) \\
-\sum_{\bar{\sigma}}v^\mu(0,\bar{\sigma})(J_3^{(1)*})_{\bar{\sigma}0}=&\tensor{(\mc{J}_3)}{^\mu_\nu}v^\nu(0,0) \\
-\sum_{\bar{\sigma}}v^i(0,\bar{\sigma})(J_3^{(1)*})_{\bar{\sigma}0}=&\tensor{(\mc{J}_3)}{^i_j}v^j(0,0) \quad \because v^0(0,\sigma)=0\\
(\mathrm{LHS})=-\sum_{\bar{\sigma}}v^i(0,\bar{\sigma})0\delta_{\bar{\sigma}0}=&0 \\
(\mathrm{RHS})=\tensor{(\mc{J}_3)}{^i_j}v^j(0,0)=&-i\epsilon_{ij3}v^j(0,0) \\
=&-i\epsilon_{i23}v^2(0,0)-i\epsilon_{i13}v^1(0,0) \\
\therefore \quad v^1(0,0)=v^2(0,0)=0
\end{align*}
を得る.したがって$\sigma=0$の場合のベクトル$u^i(0,0),v^i(0,0)$が3軸方向を向いていることがわかる.(4元ベクトルの成分は常に$1,2,3,0$の順に並べることとして,)したがって場の適当な規格化によって,これらのベクトルが値
\begin{align*}
u^\mu(0,0)=v^\mu(0,0)=\frac{1}{\sqrt{2m}}\left(
\begin{matrix}
0 \\
0 \\
1 \\
0
\end{matrix}
\right)
\end{align*}
を持つようにできる.他の成分を知るためには,(5.3.6)(5.3.7)(5.3.9)を用いて昇降演算子$J_1^{(1)}\pm iJ_2^{(1)}$の$u,v$に対する効果を計算する.これは(2.5.21)より
\begin{align*}
(J_1^{(1)}\pm iJ_2^{(1)})_{\bar{\sigma}\sigma}=&\delta_{\bar{\sigma},\sigma \pm 1}\sqrt{(1\mp \sigma)(1\pm \sigma+1)} \\
=&\delta_{\bar{\sigma},\sigma \pm 1}\sqrt{(1\mp \sigma)(2\pm \sigma)} \\
\sum_{\bar{\sigma}}u^\mu(0,\bar{\sigma})(J_1^{(1)}\pm iJ^{(1)}_2)_{\bar{\sigma}\sigma}=&\tensor{(\mc{J}_1\pm i\mc{J}_2)}{^\mu_\nu}u^\nu(0,\sigma) \\
\sum_{\bar{\sigma}}u^\mu(0,\bar{\sigma})(J_1^{(1)}\pm iJ^{(1)}_2)_{\bar{\sigma}0}=&\tensor{(\mc{J}_1\pm i\mc{J}_2)}{^\mu_\nu}u^\nu(0,0) \\
\sum_{\bar{\sigma}}u^i(0,\bar{\sigma})(J_1^{(1)}\pm iJ^{(1)}_2)_{\bar{\sigma}0}=&\tensor{(\mc{J}_1\pm i\mc{J}_2)}{^i_j}u^j(0,0) \quad \because u^0(0,\sigma)=0 \\
(\mathrm{LHS})=\sum_{\bar{\sigma}}u^i(0,\bar{\sigma})(J_1^{(1)}\pm iJ^{(1)}_2)_{\bar{\sigma}0}=&\sum_{\bar{\sigma}}u^i(0,\bar{\sigma})\delta_{\bar{\sigma},\pm 1}\sqrt{2} \\
=&\sqrt{2}u^i(0,\pm 1) \\
(\mathrm{RHS})=\tensor{(\mc{J}_1\pm i\mc{J}_2)}{^i_j}u^j(0,0)=&(-i\epsilon_{ij1}\pm \epsilon_{ij2})u^j(0,0) \\
=&\frac{1}{\sqrt{2m}}(-i\epsilon_{i31} \pm \epsilon_{i32}) \\
\therefore\quad u^i(0,\pm 1)=&\frac{1}{\sqrt{2}}\frac{1}{\sqrt{2m}}(-i\epsilon_{i31} \pm \epsilon_{i32})
\end{align*}
よって$i=1,2,3$について
\begin{align*}
u^1(0,\pm 1)=&\frac{1}{\sqrt{2}}\frac{1}{\sqrt{2m}}(\mp 1) \\
u^2(0,\pm 1)=&\frac{1}{\sqrt{2}}\frac{1}{\sqrt{2m}}(- i) \\
u^3(0,\pm 1)=&0 \\
\therefore \quad u^\mu(0,+1)=&-\frac{1}{\sqrt{2}}\frac{1}{\sqrt{2m}}\left(
\begin{matrix}
1 \\
+i \\
0 \\
0
\end{matrix}
\right) \\
u^\mu(0,-1)=&\frac{1}{\sqrt{2}}\frac{1}{\sqrt{2m}}\left(
\begin{matrix}
1 \\
-i \\
0 \\
0
\end{matrix}
\right)
\end{align*}
同様にして
\begin{align*}
-\sum_{\bar{\sigma}}v^\mu(0,\bar{\sigma})(J_1^{(1)*}\pm iJ^{(1)*}_2)_{\bar{\sigma}\sigma}=&\tensor{(\mc{J}_1\pm i\mc{J}_2)}{^\mu_\nu}v^\nu(0,\sigma) \\
-\sum_{\bar{\sigma}}v^\mu(0,\bar{\sigma})(J_1^{(1)}\mp iJ^{(1)}_2)^*_{\bar{\sigma}0}=&\tensor{(\mc{J}_1\pm i\mc{J}_2)}{^\mu_\nu}v^\nu(0,0) \\
-\sum_{\bar{\sigma}}v^i(0,\bar{\sigma})(J_1^{(1)}\mp iJ^{(1)}_2)^*_{\bar{\sigma}0}=&\tensor{(\mc{J}_1\pm i\mc{J}_2)}{^i_j}v^j(0,0) \quad \because v^0(0,\sigma)=0 \\
(\mathrm{LHS})=-\sum_{\bar{\sigma}}v^i(0,\bar{\sigma})(J_1^{(1)}\mp iJ^{(1)}_2)^*_{\bar{\sigma}0}=&-\sum_{\bar{\sigma}}v^i(0,\bar{\sigma})\delta_{\bar{\sigma},\mp 1}\sqrt{2} \\
=&-\sqrt{2}v^i(0,\mp 1) \\
(\mathrm{RHS})=\tensor{(\mc{J}_1\pm i\mc{J}_2)}{^i_j}v^j(0,0)=&(-i\epsilon_{ij1}\pm \epsilon_{ij2})v^j(0,0) \\
=&\frac{1}{\sqrt{2m}}(-i\epsilon_{i31} \pm \epsilon_{i32}) \\
\therefore\quad v^i(0,\mp 1)=&-\frac{1}{\sqrt{2}}\frac{1}{\sqrt{2m}}(-i\epsilon_{i31} \pm \epsilon_{i32})
\end{align*}
よって$i=1,2,3$について
\begin{align*}
v^1(0,\mp 1)=&\frac{1}{\sqrt{2}}\frac{1}{\sqrt{2m}}(\pm 1) \\
v^2(0,\mp 1)=&\frac{1}{\sqrt{2}}\frac{1}{\sqrt{2m}}(+ i) \\
v^3(0,\mp 1)=&0 \\
\therefore \quad v^\mu(0,+1)=&-\frac{1}{\sqrt{2}}\frac{1}{\sqrt{2m}}\left(
\begin{matrix}
1 \\
-i \\
0 \\
0
\end{matrix}
\right) \\
v^\mu(0,-1)=&\frac{1}{\sqrt{2}}\frac{1}{\sqrt{2m}}\left(
\begin{matrix}
1 \\
+i \\
0 \\
0
\end{matrix}
\right)
\end{align*}
以上より
\begin{align*}
u^\mu(0,+1)=-v^\mu(0,-1)=&-\frac{1}{\sqrt{2}}\frac{1}{\sqrt{2m}}\left(
\begin{matrix}
1 \\
+i \\
0 \\
0
\end{matrix}
\right) \\
u^\mu(0,-1)=-v^\mu(0,+1)=&\frac{1}{\sqrt{2}}\frac{1}{\sqrt{2m}}\left(
\begin{matrix}
1 \\
-i \\
0 \\
0
\end{matrix}
\right)
\end{align*}
を与える.同等だが
\begin{align*}
u^\mu(0,+1)=&v^{\mu *}(0,+1)=-\frac{1}{\sqrt{2}}\frac{1}{\sqrt{2m}}\left(
\begin{matrix}
1 \\
+i \\
0 \\
0
\end{matrix}
\right)=\frac{1}{\sqrt{2m}}e^\mu(0,+1) \\
u^\mu(0,-1)=&v^{\mu *}(0,-1)=\frac{1}{\sqrt{2}}\frac{1}{\sqrt{2m}}\left(
\begin{matrix}
1 \\
-i \\
0 \\
0
\end{matrix}
\right)=\frac{1}{\sqrt{2m}}e^\mu(0,-1) \\
u^\mu(0,0)=&v^{\mu *}(0,0)=\frac{1}{\sqrt{2m}}\left(
\begin{matrix}
0 \\
0 \\
1 \\
0
\end{matrix}
\right)=\frac{1}{\sqrt{2m}}e^\mu(0,0)
\end{align*}
より$\sigma=0,+1,-$でまとめて
\begin{align*}
u^\mu(0,\sigma)=v^{\mu *}(0,\sigma)=\frac{1}{\sqrt{2m}} e^\mu(0,\sigma)
\end{align*}
と書ける.ここで
\begin{align*}
e^\mu(0,0):=\left(
\begin{matrix}
0 \\
0 \\
1 \\
0
\end{matrix}
\right) ,\quad e^\mu(0,+1):=-\frac{1}{\sqrt{2}}\left(
\begin{matrix}
1 \\
+i \\
0 \\
0
\end{matrix}
\right) e^\mu(0,-1):=\frac{1}{\sqrt{2}}\left(
\begin{matrix}
1 \\
-i \\
0 \\
0
\end{matrix}
\right)
\end{align*}
したがって(5.3.4)と(5.3.5)を適用して,今度は
\begin{align*}
u^\mu(\mathbf{p},\sigma)=v^{\mu *}(\mathbf{p},\sigma)=&\sqrt{\frac{m}{p^0}}\frac{1}{\sqrt{2m}}\tensor{L(p)}{^\mu_\nu}e^\nu(0,\sigma) \\
=&\frac{1}{\sqrt{2p^0}}e^\mu(\mathbf{p},\sigma)
\end{align*}
となる.ここで
\begin{align*}
e^\mu(\mathbf{p},\sigma):=\tensor{L(p)}{^\mu_\nu}e^\nu(0,\sigma)
\end{align*}
である.これを用いると消滅・生成場(5.3.2)(5.3.3)は
\begin{align*}
\phi^{+\mu}(x)=&\sum_{\sigma}\frac{1}{(2\pi)^{3/2}} \int d^3\mathbf{p} u^\mu (\mathbf{p},\sigma)a(\mathbf{p},\sigma)e^{ip\cdot x} \\
=&\sum_{\sigma}\int \frac{d^3\mathbf{p}}{(2\pi)^{3/2}\sqrt{2p^0}} e^\mu (\mathbf{p},\sigma)a(\mathbf{p},\sigma)e^{ip\cdot x} \\
\phi^{-\mu}(x)=&\sum_{\sigma}\frac{1}{(2\pi)^{3/2}} \int d^3\mathbf{p} v^\mu (\mathbf{p},\sigma)a^\dagger(\mathbf{p},\sigma)e^{-ip\cdot x} \\
=&\sum_{\sigma}\int \frac{d^3\mathbf{p}}{(2\pi)^{3/2}\sqrt{2p^0}} e^{\mu *} (\mathbf{p},\sigma)a(\mathbf{p},\sigma)e^{-ip\cdot x} \\
\phi^{-\mu \dagger}(x)=&\phi^{+\mu}(x)
\end{align*}
である.場$\phi^{+\mu}(x)$と$\phi^{+\nu}$はもちろん全ての$x,y$について交換または反交換する.
\begin{align*}
&[\phi^{+\mu}(x),\phi^{-\nu}(y)]_{\mp} \\
=&\sum_{\sigma}\int \frac{d^3\mathbf{p}}{(2\pi)^{3/2}\sqrt{2p^0}} \sum_{\sigma'}\int \frac{d^3\mathbf{q}}{(2\pi)^{3/2}\sqrt{2q^0}} e^\mu (\mathbf{p},\sigma)e^\nu (\mathbf{q},\sigma')e^{ip\cdot x} e^{iq\cdot y}[a(\mathbf{p},\sigma),a(\mathbf{q})]_{\mp} \\
=&0
\end{align*}
しかし$\phi^{+\mu}(x)$と$\phi^{-\nu}$はそうではない.それらの(ボゾンの場合の)交換子または(フェルミオンの場合の)反交換子は
\begin{align*}
&[\phi^{+\mu}(x),\phi^{-\nu}(y)]_{\mp}\\
=&\sum_{\sigma}\int \frac{d^3\mathbf{p}}{(2\pi)^{3/2}\sqrt{2p^0}} \sum_{\sigma'}\int \frac{d^3\mathbf{q}}{(2\pi)^{3/2}\sqrt{2q^0}} e^\mu (\mathbf{p},\sigma)e^{\nu *} (\mathbf{q},\sigma')e^{ip\cdot x} e^{-iq\cdot y}[a(\mathbf{p},\sigma),a^\dagger(\mathbf{q})]_{\mp} \\
=&\sum_{\sigma}\int \frac{d^3\mathbf{p}}{(2\pi)^{3/2}\sqrt{2p^0}} \sum_{\sigma'}\int \frac{d^3\mathbf{q}}{(2\pi)^{3/2}\sqrt{2q^0}} e^\mu (\mathbf{p},\sigma)e^{\nu *} (\mathbf{q},\sigma')e^{ip\cdot x} e^{-iq\cdot y} \delta^3(\mathbf{p}-\mathbf{q})\delta_{\sigma\sigma'} \\
=&\int \frac{d^3\mathbf{p}}{(2\pi)^{3}2p^0} \sum_{\sigma} e^\mu (\mathbf{p},\sigma)e^{\nu *} (\mathbf{p},\sigma)e^{ip\cdot (x-y)} \\
=&\int \frac{d^3\mathbf{p}}{(2\pi)^{3}2p^0} e^{ip\cdot (x-y)} \Pi^{\mu\nu}(\mathbf{p})
\end{align*}
となる.ここで
\begin{align*}
\Pi^{\mu\nu}(\mathbf{p}):=\sum_{\sigma} e^\mu (\mathbf{p},\sigma)e^{\nu *}(\mathbf{p},\sigma)
\end{align*}
である.(5.3.25)を用いると
\begin{align*}
\Pi^{\mu\nu}(0)=&\sum_{\sigma} e^\mu (0,\sigma)e^{\nu *}(0,\sigma) \\
=&\left(
\begin{matrix}
0 \\
0 \\
1 \\
0
\end{matrix}
\right)(0,0,1,0)+\frac{1}{2}\left(
\begin{matrix}
1 \\
+i \\
0 \\
0
\end{matrix}
\right)(1,-i,0,0)+\frac{1}{2}\left(
\begin{matrix}
1 \\
-i \\
0 \\
0
\end{matrix}
\right)(1,+i,0,0) \\
=&\left(
\begin{matrix}
0 & 0 & 0 & 0 \\
0 & 0 & 0 & 0 \\
0 & 0 & 1 & 0 \\
0 & 0 & 0 & 0
\end{matrix}
\right)+\frac{1}{2}\left(
\begin{matrix}
1 & -i & 0 & 0 \\
+i & 1 & 0 & 0 \\
0 & 0 & 0 & 0 \\
0 & 0 & 0 & 0
\end{matrix}
\right)+\frac{1}{2}\left(
\begin{matrix}
1 & +i & 0 & 0 \\
-i & 1 & 0 & 0 \\
0 & 0 & 0 & 0 \\
0 & 0 & 0 & 0
\end{matrix}
\right) \\
=&\left(
\begin{matrix}
1 & 0 & 0 & 0 \\
0 & 1 & 0 & 0 \\
0 & 0 & 1 & 0 \\
0 & 0 & 0 & 0
\end{matrix}
\right)
\end{align*}
となり,$\Pi^{\mu\nu}(0)$は時間方向と垂直な空間への射影行列であることが示される.さらに変形すると
\begin{align*}
=&\left(
\begin{matrix}
1 & 0 & 0 & 0 \\
0 & 1 & 0 & 0 \\
0 & 0 & 1 & 0 \\
0 & 0 & 0 & -1
\end{matrix}
\right)+\frac{1}{m^2}\left(
\begin{matrix}
0 \\
0 \\
0 \\
m
\end{matrix}
\right)(0,0,0,m) \\
=& \eta^{\mu\nu}+\frac{k^\mu k^\nu}{m^2}  \quad k^\mu:=(0,0,0,m)^T
\end{align*}
と書けて,$k^\mu$は$L(p):k^\mu \mapsto p^\mu$の基準運動量であるから
\begin{align*}
\Pi^{\mu\nu}(\mathbf{p})=&\sum_{\sigma} e^\mu (\mathbf{p},\sigma)e^{\nu *}(\mathbf{p},\sigma) \\
=&\sum_{\sigma} \tensor{L(p)}{^\mu_\rho}e^\rho (0,\sigma)\tensor{L(p)}{^\nu_\sigma}e^{\sigma *}(0,\sigma) ,\quad \because (2.5.24)L(p)^*=L(p) \\
=&\tensor{L(p)}{^\mu_\rho} \tensor{L(p)}{^\nu_\sigma} \Pi^{\rho\sigma}(0) \\
=&\tensor{L(p)}{^\mu_\rho} \tensor{L(p)}{^\nu_\sigma}\left(\eta^{\rho\sigma}+\frac{k^\rho k^\sigma}{m^2}\right) \\
=&\tensor{L(p)}{^\mu_\rho} \tensor{L(p)}{^\nu_\sigma}\eta^{\rho\sigma}+\frac{\tensor{L(p)}{^\mu_\rho}k^\rho \tensor{L(p)}{^\nu_\sigma}k^\sigma}{m^2} \\
=&\eta^{\mu\nu}+\frac{p^\mu p^\nu}{m^2} ,\quad \because L(p)\in SO(3,1) ,\tensor{L(p)}{^\mu_\rho} \tensor{L(p)}{^\nu_\sigma}\eta^{\rho\sigma}=\eta^{\mu\nu}
\end{align*}
であることがわかる.すると交換子あるいは反交換子(5.3.27)は
\begin{align*}
[\phi^{+\mu}(x),\phi^{-\nu}(y)]_{\mp}=&\int \frac{d^3\mathbf{p}}{(2\pi)^{3}2p^0} e^{ip\cdot (x-y)} \Pi^{\mu\nu}(\mathbf{p}) \\
=&\int \frac{d^3\mathbf{p}}{(2\pi)^{3}2p^0} e^{ip\cdot (x-y)} \left(\eta^{\mu\nu}+\frac{p^\mu p^\nu}{m^2}\right) \\
=&\left(\eta^{\mu\nu}-\frac{\partial^\mu_x \partial^\nu_x}{m^2}\right)\int \frac{d^3\mathbf{p}}{(2\pi)^{3}2p^0} e^{ip\cdot (x-y)} \\
=&\left(\eta^{\mu\nu}-\frac{\partial^\mu_x \partial^\nu_x}{m^2}\right)\Delta_+(x-y)
\end{align*}
と書ける.今の目的にとって,この表現で重要なことは,空間的な$x-y$に対してそれがゼロにならず,$x-y$について偶関数ということである.よって前節で因果律を満たす場を構成しょうとした際の理由付けを繰り返すことができる.すなわち,消滅場と生成場の線形結合
\begin{align*}
v^\mu(x):=\kappa \phi^{+\mu}(x)+\lambda \phi^{-\mu}(x)
\end{align*}
を作ると,空間的な$x-y$に対して
\begin{align*}
[v^\mu(x),v^\nu(y)]_{\mp}=&[\kappa \phi^{+\mu}(x)+\lambda \phi^{-\mu}(x), \kappa \phi^{+\nu}(y)+\lambda \phi^{-\nu}(y)]_{\mp} \\
=&\kappa^2 [\phi^{+\mu}(x),\phi^{+\nu}(y)]_{\mp}+\kappa \lambda[\phi^{+\mu}(x),\phi^{-\nu}(y)]_{\mp} \\
&+\lambda \kappa[\phi^{-\mu}(x),\phi^{+\nu}(y)]_{\mp}+\lambda^2 [\phi^{-\mu}(x),\phi^{-\nu}(y)]_{\mp} \\
=&\kappa \lambda [\phi^{+\mu}(x),\phi^{+\nu}(y)]_{\mp}+\kappa \lambda [\phi^{-\mu}(x),\phi^{+\nu}(y)]_{\mp} \\
=&\kappa \lambda [\phi^{+\mu}(x),\phi^{+\nu}(y)]_{\mp}\mp \kappa \lambda [\phi^{+\nu}(y),\phi^{-\mu}(x)]_{\mp} \\
=&\kappa \lambda \left(\eta^{\mu\nu}-\frac{\partial_x^\mu \partial_x^\nu}{m^2}\right)\Delta_+(x-y) \mp \kappa \lambda \left(\eta^{\nu\mu}-\frac{\partial_y^\nu \partial_y^\mu}{m^2}\right)\Delta_+(y-x) \\
=&\kappa \lambda \left(\eta^{\mu\nu}-\frac{\partial_x^\mu \partial_x^\nu}{m^2}\right)\Delta_+(x-y) \mp \kappa \lambda \left(\eta^{\mu\nu}-\frac{\partial_x^\mu \partial_x^\nu}{m^2}\right)\Delta_+(y-x) \\
=&\kappa\lambda \left(\eta^{\mu\nu}-\frac{\partial_x^\mu \partial_x^\nu}{m^2}\right) (\Delta_+(x-y)\mp \Delta_+(y-x)) \\
=&\kappa\lambda \left(\eta^{\mu\nu}-\frac{\partial_x^\mu \partial_x^\nu}{m^2}\right) [1\mp 1]\Delta_+(x-y) \quad \because (x-y)^2>0
\end{align*}
ここで,途中で$x$微分を$y$微分に変えるために公式$\frac{\partial}{\partial x}f(x-y)=-\frac{\partial}{\partial y}f(x-y)$を用いた.さらに
\begin{align*}
[v^\mu(x),v^{\nu\dagger}(y)]_{\mp}=&[\kappa \phi^{+\mu}(x)+\lambda \phi^{-\mu}(x), \kappa^* \phi^{-\nu}(y)+\lambda^* \phi^{+\nu}(y)]_{\mp} \\
=&|\kappa|^2 [\phi^{+\mu}(x),\phi^{-\nu}(y)]_{\mp}+|\lambda|^2 [\phi^{-\mu}(x),\phi^{+\nu}(y)]_{\mp} \\
=&|\kappa|^2[\phi^{+\mu}(x),\phi^{-\nu}(y)]_{\mp} \mp |\lambda|^2 [\phi^{+\nu}(y),\phi^{-\mu}(x)]_{\mp} \\
=&|\kappa|^2\left(\eta^{\mu\nu}-\frac{\partial^\mu_x \partial^\nu_x}{m^2}\right)\Delta_+(x-y) \mp |\lambda|^2 \left(\eta^{\nu\mu}-\frac{\partial^\nu_y \partial^\mu_y}{m^2}\right)\Delta_+(y-x) \\
=&|\kappa|^2\left(\eta^{\mu\nu}-\frac{\partial^\mu_x \partial^\nu_x}{m^2}\right)\Delta_+(x-y) \mp |\lambda|^2 \left(\eta^{\mu\nu}-\frac{\partial^\mu_x \partial^\nu_x}{m^2}\right)\Delta_+(y-x) \\
=&(|\kappa|^2\mp |\lambda|^2)\left(\eta^{\mu\nu}-\frac{\partial^\mu_x \partial^\nu_x}{m^2}\right)\Delta_+(x-y) \quad \because (x-y)^2>0
\end{align*}
となる.空間的な$x-y$に対して,この二つ両方がゼロとなり因果律を保つためには,スピン$1$粒子が\uwave{ボゾン}でかつ$|\kappa|=|\lambda|$であることが必要十分である!さらに前節で行ったように,1粒子状態の位相を再定義して$\Psi_{\mathbf{p},\sigma}\to e^{-i\theta}\Psi_{\mathbf{p},\sigma}$とすると$a(\mathbf{p},\sigma)\to e^{i\theta} a(\mathbf{p},\sigma),a^\dagger(\mathbf{p},\sigma)\to e^{-i\theta}a^\dagger(\mathbf{p},\sigma)$と再定義される.したがって$\kappa=|\kappa|e^{i\alpha},\lambda=|\lambda|e^{i\beta}$とおいて
\begin{align*}
v^\mu(x)=&|\kappa|e^{i(\alpha+\theta)}\sum_{\sigma}\int \frac{d^3\mathbf{p}}{(2\pi)^{3/2}\sqrt{2p^0}} e^\mu (\mathbf{p},\sigma)a(\mathbf{p},\sigma)e^{ip\cdot x} \\
&+|\lambda|e^{i(\beta-\theta)}\sum_{\sigma}\int \frac{d^3\mathbf{p}}{(2\pi)^{3/2}\sqrt{2p^0}} e^{\mu*} (\mathbf{p},\sigma)a^\dagger(\mathbf{p},\sigma)e^{-ip\cdot x}
\end{align*}
となり,再び$\theta=\frac{1}{2}(\beta-\alpha)=\frac{1}{2}\mathrm{Arg}(\lambda/\kappa)$とおけば同位相とすることができる.
\begin{align*}
v^\mu(x)=&|\kappa|e^{i(\alpha+\beta)/2}\sum_{\sigma}\int \frac{d^3\mathbf{p}}{(2\pi)^{3/2}\sqrt{2p^0}} e^\mu (\mathbf{p},\sigma)a(\mathbf{p},\sigma)e^{ip\cdot x} \\
&+|\lambda|e^{i(\alpha+\beta)/2}\sum_{\sigma}\int \frac{d^3\mathbf{p}}{(2\pi)^{3/2}\sqrt{2p^0}} e^{\mu*} (\mathbf{p},\sigma)a^\dagger(\mathbf{p},\sigma)e^{-ip\cdot x} \\
=&|\kappa|e^{i(\alpha+\beta)/2}[\phi^{+\mu}(x)+\phi^{-\mu}(x)]
\end{align*}
場全体のスケールの規格化を再定義して,共通因子を落とすことで
\begin{align*}
v^\mu(x)=&\phi^{+\mu}(x)+\phi^{-\mu}(x) \\
=&\int \frac{d^3\mathbf{p}}{(2\pi)^{3/2}\sqrt{2p^0}} \sum_{\sigma} \left[e^\mu (\mathbf{p},\sigma)a(\mathbf{p},\sigma)e^{ip\cdot x}+e^{\mu*} (\mathbf{p},\sigma)a^\dagger(\mathbf{p},\sigma)e^{-ip\cdot x}\right]
\end{align*}
となる.これが質量を持つスピン1粒子に対する因果律を満たす実質的に唯一の場である.この場は実,すなわち
\begin{align*}
v^\mu(x)=v^{\mu\dagger}(x)
\end{align*}
であることに留意する.

\vskip\baselineskip

前節と同様に,その場が記述する粒子が(電荷などの)ある保存量子数$Q$のゼロでない値をもつなら,そのような実の場から$Q$を保存する相互作用を構成することはできない.そのかわり,同じ質量とスピンをもつが逆符号の$Q$の値を持つ別のボゾンが存在すると仮定し,因果律を満たす場を
\begin{align*}
v^\mu(x):=&\phi^{+\mu}(x)+\phi^{+c \mu\dagger}(x) \\
=&\int \frac{d^3\mathbf{p}}{(2\pi)^{3/2}\sqrt{2p^0}} \sum_{\sigma} \left[e^\mu (\mathbf{p},\sigma)a(\mathbf{p},\sigma)e^{ip\cdot x}+e^{\mu*} (\mathbf{p},\sigma)a^{c\dagger}(\mathbf{p},\sigma)e^{-ip\cdot x}\right]
\end{align*}
と構成しなければならない.ここで,$a^c$は反粒子を生成する生成演算子を表すが,その反粒子とは$\phi^{+\mu}(x)$により消滅させられる粒子の荷電共役粒子である.これは因果律を満たす場であるが,もはや実ではなくなっている.この式はまた,$a^c(\mathbf{p},\sigma)=a(\mathbf{p},\sigma)$とおけば,自分自身が反粒子である純粋に中性なスピン1粒子についても使える.この場のローレンツ変換性は(5.1.3)(5.1.6)(5.1.7)より
\begin{align*}
U_0(\Lambda)v^\mu(x)U_0^{-1}(\Lambda)=\tensor{D(\Lambda^{-1})}{^\mu_\nu}v^\nu(\Lambda x)=\tensor{\Lambda}{_\nu^\mu}v^\nu(\Lambda x)
\end{align*}
($\tensor{\Lambda}{_\nu^\mu}=\tensor{(\Lambda^{-1})}{^\mu_\nu}$であることを思い出せば,ちゃんと変換行列は$\tensor{D(\Lambda^{-1})}{^\mu_\nu}$になっている.)

\vskip\baselineskip


実場でも複素場でも,ベクトル場とその共役場の交換子は
\begin{align*}
[v^\mu(x),v^{\nu\dagger}(y)]_-=&[\phi^{+\mu}(x)+\phi^{-c \mu}(x),\phi^{-\nu}(y)+\phi^{+c \nu}(y)]_- \\
=&[\phi^{+\mu}(x),\phi^{-\nu}(y)]_- +[\phi^{-c\mu}(x),\phi^{+c\nu}(y)]_- \\
=&\left(\eta^{\mu\nu}-\frac{\partial^\mu_x \partial^\nu_x}{m^2}\right)\Delta_+(x-y) - \left(\eta^{\nu\mu}-\frac{\partial^\nu_y \partial^\mu_y}{m^2}\right)\Delta_+(y-x) \\
=&\left(\eta^{\mu\nu}-\frac{\partial^\mu_x \partial^\nu_x}{m^2}\right)\left[\Delta_+(x-y)-\Delta_+(y-x)\right] \\
=&\left(\eta^{\mu\nu}-\frac{\partial^\mu_x \partial^\nu_x}{m^2}\right)\Delta(x-y)
\end{align*}
となる.ここで$\Delta(x-y)$は(5.2.13)である.


\vskip\baselineskip


質量がゼロでないスピン1粒子について上で構成した実場および複素場は興味深い場の方程式を満たす.まず,(5.3.26)の指数関数に現れる$p^\mu$は$p^2=-m^2$を満たすから,この場はスカラー場と全く同様にクラインゴルドン方程式
\begin{align*}
(\Box -m^2)v^\mu(x)=0
\end{align*}
を満たす.(これは5.1節の最後に言及した通りである.)加えて(5.3.24)より
\begin{align*}
e^\mu(\mathbf{p},\sigma) p_\mu =&\eta_{\mu\nu} \tensor{L(p)}{^\mu_\rho} e^\rho (0,\sigma) p^\nu \\
=&\eta_{\mu\nu} e^\mu (0,\sigma) \tensor{(L(p)^{-1})}{^\nu_\rho}p^\rho \quad \because \eta_{\mu\nu}\tensor{\Lambda}{^\mu_\rho}\tensor{\Lambda}{^\nu_\sigma}=\eta_{\rho\sigma} \\
=&\eta_{\mu\nu} e^\mu (0,\sigma) k^\nu \quad \because \tensor{L(p)}{^\mu_\nu}k^\nu=p^\mu \\
=&e^\mu(0,\sigma) k_\mu \\
=&0 \quad \because k^\mu=(0,0,0,m)
\end{align*}
が得られるので
\begin{align*}
\partial_\mu v^\mu(x)=&\int \frac{d^3\mathbf{p}}{(2\pi)^{3/2}\sqrt{2p^0}} \sum_{\sigma} \left[ip_\mu e^\mu (\mathbf{p},\sigma)a(\mathbf{p},\sigma)e^{ip\cdot x}-ip_\mu e^{\mu*} (\mathbf{p},\sigma)a^{c\dagger}(\mathbf{p},\sigma)e^{-ip\cdot x}\right] \\
=&0
\end{align*}
となり,もう一つの場の方程式
\begin{align*}
\partial_\mu v^\mu(x)=0
\end{align*}
を満たす.質量が小さい極限で(5.3.36)(5.3.38)は,ちょうど電磁理論のローレンツゲージと呼ばれるゲージでの4元ベクトルポテンシャルに対する方程式
\begin{align*}
0=\partial_\mu F^{\mu\nu}(x)=&\partial_\mu (\partial^\mu A^\nu -\partial^\nu A^\mu) \\
=&\Box A^\nu(x)=0 \quad (\partial_\mu A^\mu(x)=0)
\end{align*}
になる.\par
しかし,質量がゼロでないスピン1粒子のどのような理論からでも,その質量をゼロにすることによって電磁理論を得られるわけではない.何が問題点なのかは,$J_\mu$を任意の4元ベクトルカレントとして,相互作用密度$\mc{H}=J_\mu v^\mu$によるスピン1粒子の生成率を考察することで理解できる.このスピン1粒子を含まない真空の始状態$0$と,このスピン1粒子$v(\mathbf{p},\sigma)$を一つだけ放出する終状態$v(\mathbf{p},\sigma)$の間の$S$行列要素は(相互作用)
\begin{align*}
S_{v(\mathbf{p},\sigma),0}=&\left(\Phi_{v(\mathbf{p},\sigma)} ,\left[1-i\int d^4x \mc{H}(x)+\cdots \right] \Phi_0\right) \\
=&-i \int d^4x \Bigl( \Phi_{0},a(\mathbf{p},\sigma) J_\mu(x) v^\mu(x) \Phi_0\Bigr) \\
=&-i \int d^4x \Bigl( \Phi_0,J_\mu(x)\Phi_0\Bigr)\frac{e^{\mu *}(\mathbf{p},\sigma)}{(2\pi)^{3/2}\sqrt{2p^0}}e^{-ip\cdot x}+\cdots \\
=&-i \int d^4x \Bigl( \Phi_0,e^{ix^\mu P_\mu}J_\mu(0)e^{-ix^\mu P_\mu}\Phi_0\Bigr)\frac{e^{\mu *}(\mathbf{p},\sigma)}{(2\pi)^{3/2}\sqrt{2p^0}}e^{-ip\cdot x}+\cdots \\
=&-i \int d^4x \Bigl( \Phi_0,J_\mu(0)\Phi_0\Bigr)\frac{e^{\mu *}(\mathbf{p},\sigma)}{(2\pi)^{3/2}\sqrt{2p^0}}e^{-ip\cdot x}+\cdots \\
=&-i (2\pi)^4\delta^4(p)\braket{J_\mu(0)}\frac{e^{\mu *}(\mathbf{p},\sigma)}{(2\pi)^{3/2}\sqrt{2p^0}}+\cdots
\end{align*}
となり,これを二乗してスピン1粒子のスピン$z$成分についての和をとると
\begin{align*}
\sum_{\sigma}|\braket{J_\mu(0)}e^{\mu*}(\mathbf{p},\sigma)|^2=&\braket{J_\mu(0)} \braket{J_\nu(0)}^* \sum_{\sigma}e^{\mu}(\mathbf{p},\sigma)e^{\nu*}(\mathbf{p},\sigma) \\
=&\braket{J_\mu(0)} \braket{J_\nu(0)}^* \Pi^{\mu\nu}(\mathbf{p})
\end{align*}
に比例する生成率が得られる.ここで$\mathbf{p}$は放出されたスピン1粒子の運動量で,$\braket{J_\mu(0)}$はカレントの真空-真空行列要素である.$\Pi^{\mu\nu}(\mathbf{p})$の項$p^\mu p^\nu /m^2$のために,$m\to 0$ととったとき一般に放出率が無限大になってしまう.この破局を避ける唯一の方法は,$\braket{J_\mu(0)}p^\mu$がゼロになると考えることである.それは,座標空間ではちょうど$\partial_\mu J^\mu=0$の意味でカレントが保存しなければならないということである.実際,カレント保存の必要性は単に状態の数を数えるだけでもわかる.質量がゼロでないスピン1粒子は3個のスピン状態をもち,それらはヘリシティ$\sigma=+1,0,-1$をもつ状態にとれる.一方光子のような質量ゼロのスピン1粒子はヘリシティ$+1$と$-1$しか持てない.すなわちカレント保存の条件により制限される1自由度は,ちょうどスピン1粒子のヘリシティ0状態が質量ゼロの極限で放出されないことを保証する.


\vskip\baselineskip



空間・時間・荷電の種々の反転は,前節で議論したスカラー場とほとんど同じように扱える.\par
まず空間反転を調べると\footnote{前節と同様積分範囲の反転に注意}
\begin{align*}
\mathsf{P}\phi^{+\mu}(x) \mathsf{P}^{-1}=&\int \frac{d^3\mathbf{p}}{(2\pi)^{3/2}\sqrt{2p^0}} \sum_{\sigma} e^\mu (\mathbf{p},\sigma)\mathsf{P}a(\mathbf{p},\sigma)\mathsf{P}^{-1}e^{ip\cdot x} \\
=&\int \frac{d^3\mathbf{p}}{(2\pi)^{3/2}\sqrt{2p^0}} \sum_{\sigma} \eta^* e^\mu (\mathbf{p},\sigma)a(-\mathbf{p},\sigma)e^{ip\cdot x} \\
=&\int \frac{d^3\mathbf{p}}{(2\pi)^{3/2}\sqrt{2p^0}} \sum_{\sigma} \eta^* e^\mu (-\mathbf{p},\sigma)a(\mathbf{p},\sigma)e^{ip\cdot (\mc{P}x)} \quad (\mathbf{p}\to -\mathbf{p})
\end{align*}
と
\begin{align*}
\mathsf{P}\phi^{+c\mu\dagger}(x) \mathsf{P}^{-1} =&\int \frac{d^3\mathbf{p}}{(2\pi)^{3/2}\sqrt{2p^0}} \sum_{\sigma} e^{\mu*} (\mathbf{p},\sigma)\mathsf{P} a^{c\dagger}(\mathbf{p},\sigma)\mathsf{P}^{-1}e^{-ip\cdot x} \\
=&\int \frac{d^3\mathbf{p}}{(2\pi)^{3/2}\sqrt{2p^0}} \sum_{\sigma} \eta^c e^{\mu*} (\mathbf{p},\sigma)a^{c\dagger}(-\mathbf{p},\sigma)e^{-ip\cdot x} \\
=&\int \frac{d^3\mathbf{p}}{(2\pi)^{3/2}\sqrt{2p^0}} \sum_{\sigma}\eta^c e^{\mu*} (-\mathbf{p},\sigma)a^{c\dagger}(\mathbf{p},\sigma)e^{-ip\cdot (\mc{P}x)} \quad (\mathbf{p}\to -\mathbf{p})
\end{align*}
これをさらに調べるためには,$e^\mu(-\mathbf{p},\sigma)$についての式が必要である.このために
\begin{align*}
\mc{P} L(p) \mc{P}k=& \mc{P}L(p) k \quad \because \mc{P}k=(0,0,0,m)=k\\
=&\mc{P}p =(-\mathbf{p},p^0) \\
=&L(-\mathbf{p})k
\end{align*}
より,$\tensor{L(-\mathbf{p})}{^\mu_\nu}=\tensor{\mc{P}}{^\mu_\rho}\tensor{L(\mathbf{p})}{^\rho_\tau} \tensor{\mc{P}}{^\tau_\nu}$と書けることを用いる\footnote{もちろん$L(p)$の定義からも示せる.
\begin{align*}
L(p)=&\left(
\begin{matrix}
1+(\gamma-1)\hat{\mathbf{p}}\hat{\mathbf{p}}^T & \sqrt{\gamma^2-1}\hat{\mathbf{p}} \\
\sqrt{\gamma^2-1}\hat{\mathbf{p}}^T & \gamma
\end{matrix}
\right)
\end{align*}
より
\begin{align*}
L(-\mathbf{p})=&\left(
\begin{matrix}
1+(\gamma-1)\hat{\mathbf{p}}\hat{\mathbf{p}}^T & -\sqrt{\gamma^2-1}\hat{\mathbf{p}} \\
-\sqrt{\gamma^2-1}\hat{\mathbf{p}}^T & \gamma
\end{matrix}
\right)=\left(
\begin{matrix}
-1 & 0 \\
0 & 1
\end{matrix}
\right)\left(
\begin{matrix}
1+(\gamma-1)\hat{\mathbf{p}}\hat{\mathbf{p}}^T & \sqrt{\gamma^2-1}\hat{\mathbf{p}} \\
\sqrt{\gamma^2-1}\hat{\mathbf{p}}^T & \gamma
\end{matrix}
\right)\left(
\begin{matrix}
-1 & 0 \\
0 & 1
\end{matrix}
\right) \\
=&\mc{P} L(\mathbf{p})\mc{P}
\end{align*}
となる.}.(5.3.24)を用いると$e^\mu(0,\sigma)$は空間成分しか持たないので,全ての$\sigma$について
\begin{align*}
\tensor{\mc{P}}{^\mu_\nu} e^\nu(0,\sigma)=-e^{\mu}(0,\sigma)
\end{align*}
となるから,
\begin{align*}
e^\mu(-\mathbf{p},\sigma)=&\tensor{L(-\mathbf{p})}{^\mu_\nu}e^\nu(0,\sigma) \\
=&\tensor{\mc{P}}{^\mu_\rho}\tensor{L(\mathbf{p})}{^\rho_\tau} \tensor{\mc{P}}{^\tau_\nu} e^\nu(0,\sigma) \\
=&-\tensor{\mc{P}}{^\mu_\rho}\tensor{L(\mathbf{p})}{^\rho_\tau}e^{\tau}(0,\sigma) \\
=&-\tensor{\mc{P}}{^\mu_\nu}e^{\nu}(\mathbf{p},\sigma)
\end{align*}
を得る.よって
\begin{align*}
\mathsf{P}\phi^{+\mu}(x) \mathsf{P}^{-1} =&\int \frac{d^3\mathbf{p}}{(2\pi)^{3/2}\sqrt{2p^0}} \sum_{\sigma} \eta^* e^\mu (-\mathbf{p},\sigma)a(\mathbf{p},\sigma)e^{ip\cdot (\mc{P}x)} \\
=&-\eta^* \tensor{\mc{P}}{^\mu_\nu}\int \frac{d^3\mathbf{p}}{(2\pi)^{3/2}\sqrt{2p^0}} \sum_{\sigma} e^\nu (\mathbf{p},\sigma)a(\mathbf{p},\sigma)e^{ip\cdot (\mc{P}x)} \\
=&-\eta^*\tensor{\mc{P}}{^\mu_\nu}\phi^{+\mu}(\mc{P}x) \\
\mathsf{P}\phi^{+c\mu\dagger}(x) \mathsf{P}^{-1} =&\int \frac{d^3\mathbf{p}}{(2\pi)^{3/2}\sqrt{2p^0}} \sum_{\sigma}\eta^c e^{\mu*} (-\mathbf{p},\sigma)a^{c\dagger}(\mathbf{p},\sigma)e^{-ip\cdot (\mc{P}x)} \quad (\mathbf{p}\to -\mathbf{p}) \\
=&-\eta^c \tensor{\mc{P}}{^\mu_\nu} \int \frac{d^3\mathbf{p}}{(2\pi)^{3/2}\sqrt{2p^0}} \sum_{\sigma}e^{\nu*} (\mathbf{p},\sigma)a^{c\dagger}(\mathbf{p},\sigma)e^{-ip\cdot (\mc{P}x)} \quad (\mathbf{p}\to -\mathbf{p}) \\
=&-\eta^c \tensor{\mc{P}}{^\mu_\nu}\phi^{+c\mu\dagger}(\mc{P}x)
\end{align*}
となる.したがって
\begin{align*}
\mathsf{P}v^{\mu}(x) \mathsf{P}^{-1}=-\tensor{\mc{P}}{^\mu_\nu}\Bigl(\eta^* \phi^{+\nu}(\mc{P}x)+\eta^c \phi^{+c\mu \dagger}(\mc{P}x)\Bigr)
\end{align*}
これが$v^\mu(x)$と同じ相互作用に現れるためには,やはり
\begin{align*}
\eta^*=\eta^c
\end{align*}
でなければならず,このとき
\begin{align*}
\mathsf{P}v^\mu(x) \mathsf{P}^{-1}=-\eta^* \tensor{\mc{P}}{^\mu_\nu}v^\mu(\mc{P}x)
\end{align*}
となる.\par
また,時間反転については(4.2.15)より
\begin{align*}
\mathsf{T} a(\mathbf{p},\sigma)\mathsf{T}^{-1}=&\zeta^* (-1)^{1-\sigma} a(-\mathbf{p},-\sigma) \\
\mathsf{T} a^{c\dagger}(\mathbf{p},\sigma)\mathsf{T}^{-1}=&\zeta^c (-1)^{1-\sigma} a^{c\dagger}(-\mathbf{p},-\sigma)
\end{align*}
であるから,
\begin{align*}
\mathsf{T}\phi^{+\mu}(x) \mathsf{T}^{-1}=&\int \frac{d^3\mathbf{p}}{(2\pi)^{3/2}\sqrt{2p^0}} \sum_{\sigma} \mathsf{T} e^\mu (\mathbf{p},\sigma)a(\mathbf{p},\sigma)e^{ip\cdot x}\mathsf{T}^{-1} \\
=&\int \frac{d^3\mathbf{p}}{(2\pi)^{3/2}\sqrt{2p^0}} \sum_{\sigma} e^{\mu*} (\mathbf{p},\sigma)\mathsf{T}a(\mathbf{p},\sigma)\mathsf{T}^{-1}e^{-ip\cdot x} \quad \because \mathsf{T} の反線形性 \\
=&\int \frac{d^3\mathbf{p}}{(2\pi)^{3/2}\sqrt{2p^0}} \sum_{\sigma} \zeta^*(-1)^{1-\sigma} e^{\mu*} (\mathbf{p},\sigma)a(-\mathbf{p},-\sigma)e^{-ip\cdot x} \\
=&\int \frac{d^3\mathbf{p}}{(2\pi)^{3/2}\sqrt{2p^0}} \sum_{\sigma} \zeta^*(-1)^{1+\sigma} e^{\mu*} (-\mathbf{p},-\sigma)a(\mathbf{p},\sigma)e^{ip\cdot (-\mc{P}x)} \quad (\mathbf{p}\to -\mathbf{p},\sigma\to -\sigma)
\end{align*}
(ここで,$\sigma$の和を$\sigma'=-\sigma$の和にしても変数の範囲は$1,0,-1$で変わらないことを使っている.)さらに
\begin{align*}
\mathsf{T}\phi^{+c\mu\dagger}(x) \mathsf{T}^{-1} =&\int \frac{d^3\mathbf{p}}{(2\pi)^{3/2}\sqrt{2p^0}} \sum_{\sigma} \mathsf{T}e^{\mu*} (\mathbf{p},\sigma) a^{c\dagger}(\mathbf{p},\sigma)e^{-ip\cdot x} \mathsf{T}^{-1} \\
=&\int \frac{d^3\mathbf{p}}{(2\pi)^{3/2}\sqrt{2p^0}} \sum_{\sigma} e^{\mu} (\mathbf{p},\sigma)\mathsf{T} a^{c\dagger}(\mathbf{p},\sigma)\mathsf{T}^{-1} e^{ip\cdot x} \quad \because \mathsf{T} の反線形性 \\
=&\int \frac{d^3\mathbf{p}}{(2\pi)^{3/2}\sqrt{2p^0}} \sum_{\sigma}\zeta^c (-1)^{1-\sigma} e^{\mu} (\mathbf{p},\sigma) a^{c\dagger}(-\mathbf{p},-\sigma) e^{ip\cdot x} \\
=&\int \frac{d^3\mathbf{p}}{(2\pi)^{3/2}\sqrt{2p^0}} \sum_{\sigma}\zeta^c (-1)^{1+\sigma} e^{\mu} (-\mathbf{p},-\sigma)a^{c\dagger}(\mathbf{p},\sigma)e^{ip\cdot (-\mc{P}x)} \quad (\mathbf{p}\to -\mathbf{p},\sigma \to -\sigma)
\end{align*}
これをさらに調べるために,$(-1)^{1+\sigma}e^\mu(-\mathbf{p},-\sigma)$についての式が必要である.このために(5.3.25)から
\begin{align*}
(-1)^{1+1}e^{\mu*}(0,-1)=&+\frac{1}{\sqrt{2}}\left(
\begin{matrix}
1 \\
+i \\
0 \\
0
\end{matrix}
\right)=-e^{\mu}(0,+1) \\
(-1)^{1+0}e^{\mu*}(0,0)=&-\left(
\begin{matrix}
0 \\
0 \\
1 \\
0
\end{matrix}
\right)=-e^{\mu}(0,0) \\
(-1)^{1-1}e^{\mu*}(0,+1)=&-\frac{1}{\sqrt{2}}\left(
\begin{matrix}
1 \\
-i \\
0 \\
0
\end{matrix}
\right)=-e^\mu(0,-1) \\
\therefore\quad (-1)^{1+\sigma}e^{\mu*}(0,-\sigma)=&-e^{\mu}(0,\sigma)
\end{align*}
となることと,上で用いた関係式$\tensor{L(-\mathbf{p})}{^\mu_\nu}=\tensor{\mc{P}}{^\mu_\rho}\tensor{L(\mathbf{p})}{^\rho_\tau} \tensor{\mc{P}}{^\tau_\nu}$(あるいは(5.3.39))を用いると
\begin{align*}
(-1)^{1+\sigma}e^{\mu*}(-\mathbf{p},-\sigma)=&\tensor{L(-\mathbf{p})}{^\mu_\nu}(-1)^{1+\sigma}e^{\mu*}(0,-\sigma) \\
=&-\tensor{L(-\mathbf{p})}{^\mu_\nu}e^\nu(0,\sigma) \\
=&-e^{\mu}(-\mathbf{p},\sigma) \\
=&+\tensor{\mc{P}}{^\mu_\nu}e^\nu(\mathbf{p},\sigma) \quad \because (5.3.39)
\end{align*}
となる.以上より時間反転は
\begin{align*}
\mathsf{T}\phi^{+\mu}(x) \mathsf{T}^{-1}=&\int \frac{d^3\mathbf{p}}{(2\pi)^{3/2}\sqrt{2p^0}} \sum_{\sigma} \zeta^*(-1)^{1+\sigma} e^{\mu*} (-\mathbf{p},-\sigma)a(\mathbf{p},\sigma)e^{ip\cdot (-\mc{P}x)} \\
=&\zeta^* \tensor{\mc{P}}{^\mu_\nu}\int \frac{d^3\mathbf{p}}{(2\pi)^{3/2}\sqrt{2p^0}} \sum_{\sigma} e^{\nu}(\mathbf{p},\sigma)a(\mathbf{p},\sigma)e^{ip\cdot (-\mc{P}x)} \\
=&\zeta^* \tensor{\mc{P}}{^\mu_\nu} \phi^{+\mu}(-\mc{P}x)=\zeta^* \tensor{\mc{P}}{^\mu_\nu} \phi^{+\mu}(\mc{T}x) \\
\mathsf{T}\phi^{+c\mu\dagger}(x) \mathsf{T}^{-1} =&\int \frac{d^3\mathbf{p}}{(2\pi)^{3/2}\sqrt{2p^0}} \sum_{\sigma}\zeta^c (-1)^{1+\sigma} e^{\mu} (-\mathbf{p},-\sigma)a^{c\dagger}(\mathbf{p},\sigma)e^{ip\cdot (-\mc{P}x)} \\
=&\zeta^c \tensor{\mc{P}}{^\mu_\nu}\int \frac{d^3\mathbf{p}}{(2\pi)^{3/2}\sqrt{2p^0}} \sum_{\sigma} e^{\nu} (\mathbf{p},\sigma)a^{c\dagger}(\mathbf{p},\sigma)e^{ip\cdot (-\mc{P}x)} \\
=&\zeta^c \tensor{\mc{P}}{^\mu_\nu}\phi^{+c\nu\dagger}(-\mc{P}x)=\zeta^c \tensor{\mc{P}}{^\mu_\nu}\phi^{+c\nu\dagger}(\mc{T}x)
\end{align*}
となる.したがって
\begin{align*}
\mathsf{T}v^{\mu}(x) \mathsf{T}^{-1}=\tensor{\mc{P}}{^\mu_\nu}\Bigl(\zeta^* \phi^{+\nu}(\mc{T}x)+\zeta^c \phi^{+c\mu \dagger}(\mc{T}x)\Bigr)
\end{align*}
これが$v^\mu(x)$と同じ相互作用に現れるためには,やはり
\begin{align*}
\zeta^*=\zeta^c
\end{align*}
でなければならず,このとき
\begin{align*}
\mathsf{T}v^{\mu}(x) \mathsf{T}^{-1}=\zeta^*\tensor{\mc{P}}{^\mu_\nu}v^\nu(\mc{T}x)
\end{align*}
となる.\par
最後に荷電共役をすると,これは簡単に得られて
\begin{align*}
\mathsf{C}\phi^{+\mu}(x) \mathsf{C}^{-1}=&\int \frac{d^3\mathbf{p}}{(2\pi)^{3/2}\sqrt{2p^0}} \sum_{\sigma} e^\mu (\mathbf{p},\sigma)\mathsf{C}a(\mathbf{p},\sigma)\mathsf{C}^{-1}e^{ip\cdot x} \\
=&\int \frac{d^3\mathbf{p}}{(2\pi)^{3/2}\sqrt{2p^0}} \sum_{\sigma} \xi^* e^\mu (\mathbf{p},\sigma)a^c(\mathbf{p},\sigma)e^{ip\cdot x} \\
=&\int \frac{d^3\mathbf{p}}{(2\pi)^{3/2}\sqrt{2p^0}} \sum_{\sigma} \xi^* e^\mu (\mathbf{p},\sigma)a^c(\mathbf{p},\sigma)e^{ip\cdot x} \\
=&\xi^* \phi^{+c\mu}(x) \\
\mathsf{C}\phi^{+c\mu\dagger}(x) \mathsf{C}^{-1}=&\int \frac{d^3\mathbf{p}}{(2\pi)^{3/2}\sqrt{2p^0}} \sum_{\sigma} e^{\mu*} (\mathbf{p},\sigma)\mathsf{C}a^{c\dagger}(\mathbf{p},\sigma)\mathsf{C}^{-1}e^{-ip\cdot x} \\
=&\int \frac{d^3\mathbf{p}}{(2\pi)^{3/2}\sqrt{2p^0}} \sum_{\sigma} \xi^c e^{\mu*} (\mathbf{p},\sigma)a^\dagger(\mathbf{p},\sigma)e^{-ip\cdot x} \\
=&\int \frac{d^3\mathbf{p}}{(2\pi)^{3/2}\sqrt{2p^0}} \sum_{\sigma} \xi^c e^{\mu*} (\mathbf{p},\sigma)a^\dagger(\mathbf{p},\sigma)e^{-ip\cdot x} \\
=&\xi^c \phi^{+\mu\dagger}(x)
\end{align*}
したがって
\begin{align*}
\mathsf{T}v^{\mu}(x) \mathsf{T}^{-1}=\xi^* \phi^{+c\mu}(x)+\xi^c \phi^{+\mu \dagger}(x)
\end{align*}
これが$v^\mu(x)$と同じ相互作用に現れるためには,やはり
\begin{align*}
\xi^*=\xi^c
\end{align*}
でなければならず,このとき
\begin{align*}
\mathsf{T}v^{\mu}(x) \mathsf{T}^{-1}=\xi^* v^{\mu\dagger}(x)
\end{align*}
となる.\par
(5.3.44)の負符号により,極性ベクトルとして(すなわち空間反転のもとで通常のベクトルのように)変換するベクトル場は,行列$\tensor{\mc{P}}{^\mu_\nu}$に伴う空間反転に現れる通常の負符号や位相を除いて,
\begin{align*}
\mathsf{P}v^\mu(x) \mathsf{P}^{-1}=\tensor{\mc{P}}{^\mu_\nu}v^\mu(\mc{P}x)
\end{align*}
と変換しなければならないから,$\eta=-1$をもつスピン1粒子を記述することがわかる.


\newpage


\subsection{ディラック形式}
斉次ローレンツ群の表現とは,群の乗法則
\begin{align*}
D(\bar{\Lambda})D(\Lambda)=D(\bar{\Lambda}\Lambda)
\end{align*}
を満たす行列$D(\Lambda)$の集合を意味する.2.4節でユニタリー演算子$U(\Lambda)$について調べたのと全く同様に,微小変換
\begin{align*}
\tensor{\Lambda}{^\mu_\nu}=\delta^\mu_\nu +\tensor{\omega}{^\mu_\nu} ,\quad \omega_{\mu\nu}=-\omega_{\mu\nu}
\end{align*}
の場合を考察することで,これらの行列の性質を調べることができる.その場合,
\begin{align*}
D(\Lambda)=1+\frac{i}{2}\omega_{\mu\nu}\mc{J}^{\mu\nu}
\end{align*}
と書けて,$\mc{J}^{\mu\nu}=-\mc{J}^{\nu\mu}$を満たす$\{\mc{J}^{\mu\nu}\}_{0\leq \mu ,\nu \leq 3}$は交換関係(2.4.12)と同じ交換関係
\begin{align*}
i[\mc{J}^{\mu\nu},\mc{J}^{\rho\sigma}]=\eta^{\nu\rho}\mc{J}^{\mu\sigma}-\eta^{\mu\rho}\mc{J}^{\nu\sigma}-\eta^{\sigma\mu}\mc{J}^{\rho\nu}+\eta^{\sigma\nu}\mc{J}^{\rho\mu}
\end{align*}
を満たすような\uwave{有限次元行列}の集合である.このような交換関係を満たす$\mc{J}^{\mu\nu}$が見つかれば,$D(\Lambda)$は有限のパラメータでは
\begin{align*}
D(\Lambda)=\exp\left(\frac{i}{2}\omega_{\mu\nu}\mc{J}^{\mu\nu}\right)
\end{align*}
としてローレンツ群の表現を構成することができる.\par
そのような行列の集合$\{\mc{J}^{\mu\nu}\}_{0\leq \mu ,\nu \leq 3}$を見つけるには,まず\uwave{反}交換関係
\begin{align*}
\{\gamma^\mu,\gamma^\nu\}=2\eta^{\mu\nu}\bm{1}_N
\end{align*}
を満たす行列$\{\gamma^\mu\}_{0\leq \mu\leq 3}$を構成する\footnote{まだこの行列の大きさは指定していない.$4\times 4$でなくてもよいので適当に$N\times N$とおいている.もちろん,後から既約性を課すことで$4\times 4$行列と定める.}.これは
\begin{align*}
(\gamma^i)^2=&+\bm{1} ,\quad (\gamma^0)^2=-\bm{1} \\
\gamma^i \gamma^j=&-\gamma^j \gamma^i \quad (i\neq j),\quad \gamma^0 \gamma^i=-\gamma^i \gamma^0
\end{align*}
などの関係を表している.わかりやすく成分表示すれば
\begin{align*}
\sum_{b=1}^N\Bigl(\gamma^{\mu}_{ab}\gamma^{\nu}_{bc}+\gamma^{\nu}_{ab}\gamma^\nu_{bc}\Bigr)=2\eta^{\mu\nu}\delta_{ab}
\end{align*}
となっている.仮に
\begin{align*}
\mc{S}^{\mu\nu}=-\frac{i}{4}[\gamma^\mu,\gamma^\nu]
\end{align*}
と定義してみよう.(ここで$\mc{J}$ではなく$\mc{S}$としているのは,ローレンツ変換の$\tensor{\Lambda}{^\mu_\nu}$を展開したときに現れる$\tensor{(\mc{J}^{\rho\sigma})}{^\mu_\nu}$と区別するためである.)(5.4.5)を用いて
\begin{align*}
[\mc{S}^{\mu\nu},\gamma^\rho]=&-\frac{i}{4}[\gamma^\mu\gamma^\nu ,\gamma^\rho]+\frac{i}{4}[\gamma^\nu\gamma^\mu,\gamma^\rho] \\
=&-\frac{i}{4}\Bigl(\gamma^\mu\{\gamma^\nu, \gamma^\rho\}-\{\gamma^\mu,\gamma^\rho\}\gamma^\nu \Bigr) \\
&+\frac{i}{4}\Bigl(\gamma^\nu\{\gamma^\mu, \gamma^\rho\}-\{\gamma^\nu,\gamma^\rho\}\gamma^\mu \Bigr) \\
=&-\frac{i}{4}\Bigl(2\gamma^\mu\eta^{\nu\rho}-2\eta^{\mu\rho}\gamma^\nu \Bigr) \\
&+\frac{i}{4}\Bigl(2\gamma^\nu\eta^{\mu\rho}-2\eta^{\nu\rho}\gamma^\mu \Bigr) \quad \because (5.4.5) \\
=&-i\gamma^\mu \eta^{\nu\rho}+i\gamma^\nu \eta^{\mu\rho} \\
\therefore \quad i[\gamma^\mu,\mc{S}^{\rho\sigma}]=&\eta^{\mu\rho}\gamma^\sigma -\eta^{\mu\sigma}\gamma^\rho
\end{align*}
と書ける.これはローレンツ代数と運動量演算子の交換関係(2.4.13)と同じ構造であることがわかる.(ここで,反交換関係が利用できるときに便利な恒等式を使った.何度も使うのでここで紹介しておく.交換関係でもよく見る,所謂積の微分と似た構造
\begin{align*}
[A,BC]=&ABC-BCA=(ABC-BAC)+(BAC-BCA) \\
=&[A,B]C+B[A,C] \\
[AB,C]=&ABC-CAB=(ABC-ACB)+(ACB-CAB) \\
=&A[B,C]+[A,C]B
\end{align*}
と同様の
\begin{align*}
[A,BC]=&ABC-BCA=(ABC+BAC)-(BAC+BCA) \\
=&\{A,B\}C-B\{A,C\} \\
[AB,C]=&ABC-CAB=(ABC+ACB)-(ACB+CAB) \\
=&A\{B,C\}-\{A,C\}B
\end{align*}
がなりたつ.これらは$A,B,C$がフェルミオン的なものでもボゾン的なものでも恒等的になりたつ式だから,便利な方を選ぶと良い.)これを用いると
\begin{align*}
[\mc{S}^{\mu\nu},\mc{S}^{\rho\sigma}]=&-\frac{i}{4}[\mc{S}^{\mu\nu},\gamma^{\rho}\gamma^\sigma]+\frac{i}{4}[\mc{S}^{\mu\nu},\gamma^{\sigma}\gamma^\rho] \\
=&-\frac{i}{4}[\mc{S}^{\mu\nu},\gamma^\rho]\gamma^\sigma+\frac{i}{4}\gamma^\rho[\mc{S}^{\mu\nu},\gamma^\sigma] \\
&+\frac{i}{4}[\mc{S}^{\mu\nu},\gamma^\sigma]\gamma^\rho+ \frac{i}{4}\gamma^\sigma[\mc{S}^{\mu\nu},\gamma^\rho] \\
=&-\frac{i}{4}i(-\gamma^\mu \eta^{\nu\rho}+\gamma^\nu \eta^{\mu\rho})\gamma^\sigma+\frac{i}{4}i\gamma^\rho(-\gamma^\mu \eta^{\nu\sigma}+\gamma^\nu \eta^{\mu\sigma}) \\
&+\frac{i}{4}i(-\gamma^\mu\eta^{\nu\sigma}+\gamma^\nu \eta^{\mu\sigma})\gamma^\rho+ \frac{i}{4}\gamma^\sigma(-\gamma^\mu \eta^{\nu\rho}+\gamma^{\nu}\eta^{\mu\rho}) \\
=&-\frac{1}{4}(\eta^{\nu\rho}\gamma^\mu\gamma^\sigma -\eta^{\mu\rho}\gamma^\nu \gamma^\sigma +\eta^{\nu\sigma}\gamma^\rho \gamma^\mu -\eta^{\mu\sigma}\gamma^\rho\gamma^\nu \\
&\quad\quad -\eta^{\nu\sigma}\gamma^\mu\gamma^\rho +\eta^{\mu\sigma}\gamma^\nu \gamma^\rho -\eta^{\nu\rho}\gamma^\sigma \gamma^\mu +\eta^{\mu\rho}\gamma^\sigma\gamma^\nu) \\
=&-\frac{1}{4}\Bigl(\eta^{\nu\rho}[\gamma^\mu,\gamma^\sigma]-\eta^{\mu\rho}[\gamma^\nu,\gamma^\sigma]-\eta^{\mu\sigma}[\gamma^\rho,\gamma^\nu]+\eta^{\nu\sigma}[\gamma^\rho,\gamma^\mu]\Bigr) \\
=&-i\Bigl(\eta^{\nu\rho}\mc{S}^{\mu\sigma}-\eta^{\mu\rho}\mc{S}^{\nu\sigma}-\eta^{\sigma\mu}\mc{S}^{\rho\nu}+\eta^{\sigma\nu}\mc{S}^{\rho\mu}\Bigr) \\
\therefore \quad i[\mc{S}^{\mu\nu},\mc{S}^{\rho\sigma}]=&\eta^{\nu\rho}\mc{S}^{\mu\sigma}-\eta^{\mu\rho}\mc{S}^{\nu\sigma}-\eta^{\sigma\mu}\mc{S}^{\rho\nu}+\eta^{\sigma\nu}\mc{S}^{\rho\mu}
\end{align*}
となり,望みの交換関係(5.4.4)を満たすことがわかる.さらに$\gamma^\mu$は既約,すなわち,これらの行列のどれによっても不変な固有部分空間はないと仮定する.つまり,適当な相似変換によって全ての$\gamma^\mu$がブロック対角な形
\begin{align*}
\gamma^\mu=\left(
\begin{matrix}
A^\mu & 0 \\
0 & B^\mu
\end{matrix}
\right)
\end{align*}の形にはならない,と仮定する.もしそうできるならば,この$A^\mu,B^\mu$は同じ反交換関係(5.4.5)を満たす
\begin{align*}
2\eta^{\mu\nu}1=&\gamma^\mu \gamma^\nu+\gamma^\nu \gamma^\mu \\
=&\left(
\begin{matrix}
A^\mu A^\nu +A^\nu A^\mu & 0 \\
0 & B^\mu B^\nu +B^\nu B^\mu
\end{matrix}
\right) \\
\therefore \quad \{A^\mu ,A^\nu\}=&2\eta^{\mu\nu}1 ,\quad \{B^\mu,B^\nu\}=2\eta^{\mu\nu}1
\end{align*}
から,もっと小さい$A^\mu,B^\mu$のような行列によって$\mc{S}^{\mu\nu}=-\frac{i}{4}[A^\mu,A^\nu]$などのようにして(5.4.3)(5.4.6)のように変換する場の成分だけを選んで既約表現にできたはずだからである\footnote{これは$\gamma^\mu$が既約な行列であるというだけで,$\mc{J}^{\mu\nu}$が既約な行列であることを課していないことに注意.実際,ここから作られる$\mc{J}^{\mu\nu}$は(5.4.19)(5.4.20)のようにブロック対角な形になり可約である.}.

\vskip\baselineskip


(5.4.5)(あるいは$\eta^{\mu\nu}$をクロネッカーのデルタで置き換えたユークリッド版$\{\gamma^i,\gamma^j\}=2\delta{ij}$)のような関係を満たす行列の任意の集合$\{\gamma^\mu\}$は,クリフォード代数の表現になっているという.斉次ローレンツ群(より正確にはその被覆群)のこの特別な表現の数学的な重要性は,ローレンツ群の最も一般的な既約表現はテンソルか,(5.4.3)(5.4.6)のように変換するスピノルか,スピノルとテンソルの直積のどれかであるという事実(5.6節参照)から来ている.


\vskip\baselineskip


交換関係(5.4.7)を用いると,(5.4.3)から
\begin{align*}
\left[\frac{i}{2}\omega_{\mu\nu}\mc{S}^{\mu\nu},\gamma^\rho\right]=&\frac{i}{2}\omega_{\mu\nu} \left(-i\gamma^\mu \eta^{\nu\rho} +i\gamma^\nu \eta^{\mu\rho}\right) \quad \because \omega_{\mu\nu}=-\omega_{\nu\mu}\\
=&-\tensor{\omega}{^\rho_\nu}\gamma^\nu \\
=&-\frac{i}{2}\omega_{\mu\nu}\tensor{\left(\mc{J}^{\mu\nu}\right)}{^\rho_\sigma} \gamma^\sigma \quad (\tensor{\left(\mc{J}^{\mu\nu}\right)}{^\rho_\sigma}=i\delta^\mu_\sigma \eta^{\rho\nu}-i\delta^\nu_\sigma \eta^{\rho\mu})
\end{align*}
が得られ,よって$n$重交換子は
\begin{align*}
&\left[\frac{i}{2}\omega_{{\mu_1}{\nu_1}}\mc{S}^{{\mu_1}{\nu_1}},\left[\frac{i}{2}\omega_{{\mu_2}{\nu_2}}\mc{S}^{{\mu_2}{\nu_2}},\cdots,\left[\frac{i}{2}\omega_{{\mu_n}{\nu_n}}\mc{S}^{{\mu_n}{\nu_n}},\gamma^\rho\right] \cdots\right]\right] \\
=&\tensor{\left(-\frac{i}{2}\omega_{{\mu_1}{\nu_1}}\mc{J}^{{\mu_1}{\nu_1}}\right)}{^{\rho_1}_{\sigma_1}}\tensor{\left(-\frac{i}{2}\omega_{{\mu_2}{\nu_2}}\mc{J}^{{\mu_2}{\nu_2}}\right)}{^{\rho_2}_{\sigma_2}}\cdots \tensor{\left(-\frac{i}{2}\omega_{{\mu_n}{\nu_n}}\mc{J}^{{\mu_n}{\nu_n}}\right)}{^{\rho_n}_{\sigma_n}}\gamma^{{\sigma_n}} \\
=&\tensor{\left[\left(\frac{i}{2}\omega_{\mu\nu}\mc{J}^{\mu\nu}\right)^n\right]}{^\rho_\sigma}\gamma^\sigma
\end{align*}
と書ける.よって,BCH公式から
\begin{align*}
D(\Lambda)\gamma^{\rho}D^{-1}(\Lambda)=&\exp\left(+\frac{i}{2}\omega_{\mu\nu}\mc{S}^{\mu\nu}\right)\gamma^\rho \exp\left(-\frac{i}{2}\omega_{\mu\nu}\mc{S}^{\mu\nu}\right) \\
=&\gamma^\rho +\left[\frac{i}{2}\omega_{\mu\nu}\mc{S}^{\mu\nu},\gamma^\rho\right]+\frac{1}{2!}\left[\frac{i}{2}\omega_{{\mu_1}{\nu_1}}\mc{S}^{{\mu_1}{\nu_1}},\left[\frac{i}{2}\omega_{{\mu_2}{\nu_2}}\mc{S}^{{\mu_2}{\nu_2}},\gamma^\rho\right]\right] \\
&+\frac{1}{n!}\left[\frac{i}{2}\omega_{{\mu_1}{\nu_1}}\mc{S}^{{\mu_1}{\nu_1}},\left[\frac{i}{2}\omega_{{\mu_2}{\nu_2}}\mc{S}^{{\mu_2}{\nu_2}},\cdots,\left[\frac{i}{2}\omega_{{\mu_n}{\nu_n}}\mc{S}^{{\mu_n}{\nu_n}},\gamma^\rho\right] \cdots\right]\right]+\cdots \\
=&\sum_{n=0}^\infty \frac{1}{n!}\left[\frac{i}{2}\omega_{{\mu_1}{\nu_1}}\mc{J}^{{\mu_1}{\nu_1}},\left[\frac{i}{2}\omega_{{\mu_2}{\nu_2}}\mc{J}^{{\mu_2}{\nu_2}},\cdots,\left[\frac{i}{2}\omega_{{\mu_n}{\nu_n}}\mc{J}^{{\mu_n}{\nu_n}},\gamma^\rho\right] \cdots\right]\right] \\
=&\tensor{\left[\sum_{n=0}^\infty \frac{1}{n!}\left(-\frac{i}{2}\omega_{\mu\nu}\mc{J}^{\mu\nu}\right)^n\right]}{^\rho_\sigma}\gamma^\sigma \\
=&\tensor{\left[\exp\left(-\frac{i}{2}\omega_{\mu\nu}\mc{J}^{\mu\nu}\right)\right]}{^\rho_\sigma}\gamma^\sigma \\
=&\tensor{(\Lambda^{-})}{^\rho_\sigma}\gamma^\sigma \\
=&\tensor{\Lambda}{_\sigma^\rho}\gamma^\sigma
\end{align*}
となる.この意味で$\gamma^\rho$はベクトルであるという風にまとめることができる.同じ意味で,単位行列はスカラー
\begin{align*}
D(\Lambda) 1 D^{-1}(\Lambda)=1
\end{align*}
であることは自明である.$\mc{S}^{\rho\sigma}$については,(5.4.4)から
\begin{align*}
\left[\frac{i}{2}\omega_{\mu\nu}\mc{S}^{\mu\nu} ,\mc{S}^{\rho\sigma}\right]=&\frac{1}{2}\omega_{\mu\nu} \left(\eta^{\nu\rho}\mc{S}^{\mu\sigma}-\eta^{\mu\rho}\mc{S}^{\nu\sigma}-\eta^{\sigma\mu}\mc{S}^{\sigma\nu}+\eta^{\sigma\nu}\mc{S}^{\rho\mu}\right) \\
=&-\tensor{\omega}{^\rho_\mu} \mc{S}^{\mu\sigma}-\tensor{\omega}{^\sigma_\mu}\mc{S}^{\rho\mu} \\
=&-\frac{i}{2}\omega_{\mu\nu}\tensor{(\mc{J}^{\mu\nu})}{^\rho_\alpha}\mc{S}^{\alpha\sigma} -\frac{i}{2}\omega_{\mu\nu}\tensor{(\mc{J}^{\mu\nu})}{^\sigma_\alpha} \mc{S}^{\rho\alpha}
\end{align*}
となり,よって$n$重交換子は
\begin{align*}
&\left[\frac{i}{2}\omega_{{\mu_1}{\nu_1}}\mc{S}^{{\mu_1}{\nu_1}},\left[\frac{i}{2}\omega_{{\mu_2}{\nu_2}}\mc{S}^{{\mu_2}{\nu_2}},\cdots,\left[\frac{i}{2}\omega_{{\mu_n}{\nu_n}}\mc{S}^{{\mu_n}{\nu_n}},\mc{S}^{\rho\sigma}\right] \cdots\right]\right] \\
=&\sum_{k=0}^n \binom{n}{k}\tensor{\left[\left(-\frac{i}{2}\omega_{\mu\nu}\mc{J}^{\mu\nu}\right)^{n-k}\right]}{^\rho_\alpha}\tensor{\left[\left(-\frac{i}{2}\omega_{\mu\nu}\mc{J}^{\mu\nu}\right)^{k}\right]}{^\sigma_\beta}\mc{S}^{\alpha\beta} \\
=&\sum_{k=0}^n \frac{n!}{k!(n-k)!}\tensor{\left[\left(-\frac{i}{2}\omega_{\mu\nu}\mc{J}^{\mu\nu}\right)^{n-k}\right]}{^\rho_\alpha}\tensor{\left[\left(-\frac{i}{2}\omega_{\mu\nu}\mc{J}^{\mu\nu}\right)^{k}\right]}{^\sigma_\beta}\mc{S}^{\alpha\beta}
\end{align*}
となる.(二重交換子などで確かめるとよい.)よって
\begin{align*}
D(\Lambda)\mc{S}^{\rho\sigma}D^{-1}(\Lambda) =&\sum_{n=1}^\infty \frac{1}{n!}\left[\frac{i}{2}\omega_{{\mu_1}{\nu_1}}\mc{S}^{{\mu_1}{\nu_1}},\left[\frac{i}{2}\omega_{{\mu_2}{\nu_2}}\mc{S}^{{\mu_2}{\nu_2}},\cdots,\left[\frac{i}{2}\omega_{{\mu_n}{\nu_n}}\mc{S}^{{\mu_n}{\nu_n}},\mc{S}^{\rho\sigma}\right] \cdots\right]\right] \\
=&\sum_{n=0}^\infty \frac{1}{n!}\sum_{k=0}^n \frac{n!}{k!(n-k)!}\tensor{\left[\left(-\frac{i}{2}\omega_{\mu\nu}\mc{J}^{\mu\nu}\right)^{n-k}\right]}{^\rho_\alpha}\tensor{\left[\left(-\frac{i}{2}\omega_{\mu\nu}\mc{J}^{\mu\nu}\right)^{k}\right]}{^\sigma_\beta}\mc{S}^{\alpha\beta} \\
=&\sum_{n=0}^\infty \sum_{k=0}^n \frac{1}{k!(n-k)!}\tensor{\left[\left(-\frac{i}{2}\omega_{\mu\nu}\mc{J}^{\mu\nu}\right)^{n-k}\right]}{^\rho_\alpha}\tensor{\left[\left(-\frac{i}{2}\omega_{\mu\nu}\mc{J}^{\mu\nu}\right)^{k}\right]}{^\sigma_\beta}\mc{S}^{\alpha\beta} \\
=&\tensor{\left[\exp\left(-\frac{i}{2}\omega_{\mu\nu}\mc{J}^{\mu\nu}\right)\right]}{^\rho_\alpha} \tensor{\left[\exp\left(-\frac{i}{2}\omega_{\mu\nu}\mc{J}^{\mu\nu}\right)\right]}{^\sigma_\beta} \mc{S}^{\alpha\beta} \\
=&\tensor{\Lambda}{_\alpha^\rho} \tensor{\Lambda}{_\beta^\sigma}\mc{S}^{\alpha\beta}
\end{align*}
となる.この意味で$\mc{S}^{\rho\sigma}=-\frac{i}{4}[\gamma^\rho,\gamma^\sigma]$は反対称テンソルである.(こんな式変形しなくても,シンプルに$\gamma^\mu$の変換がベクトル的だから,その交換子を作ったら全体が反対称テンソルになるのは当たり前か.)行列$\gamma^\mu$は他の完全反対称テンソル
\begin{align*}
\mc{A}^{\rho\sigma\tau}:=\gamma^{[\rho} \gamma^\sigma \gamma^{\tau]} \\
\mc{P}^{\rho\sigma\tau \eta}:=\gamma^{[\rho}\gamma^\sigma \gamma^\tau \gamma^{\eta]}
\end{align*}
を構成するのに役立つ.ここで括弧は,括弧内の指標のあらゆる偶数回または奇数階の置換に対して,それぞれプラスまたはマイナスの符号をつけて和をとるという標準的な記法を意味している.例えば$\mc{A}^{\rho\sigma\tau}$は
\begin{align*}
\mc{A}^{\rho\sigma\tau}:=&\gamma^\rho \gamma^\sigma \gamma^\tau -\gamma^\rho \gamma^\tau \gamma^\sigma -\gamma^\sigma \gamma^\rho \gamma^\tau \\
&+\gamma^\tau \gamma^\rho \gamma^\sigma +\gamma^\sigma \gamma^\tau \gamma^\rho -\gamma^\tau \gamma^\sigma \gamma^\rho
\end{align*}
の略記になっている.後に示すように,これら$1,\gamma^\rho ,S^{\rho\sigma},\mc{A}^{\rho\sigma\tau},\mc{P}^{\rho\sigma\tau \eta}$はディラック行列から作られる全ての行列の集合の完全基底をなす.


\vskip\baselineskip


この形式は自動的にパリティ変換を含む.それは慣習的に
\begin{align*}
\beta:=i\gamma^0
\end{align*}
ととられている.定義よりこれは$\beta^2=-\gamma^0 \gamma^0=+1$となる.この行列をディラック行列に施すと
\begin{align*}
\beta \gamma^i \beta^{-1}=&-\gamma^0 \gamma^i \gamma^0 \\
=&+\gamma^i \gamma^0 \gamma^0 \\
=&-\gamma^i \\
\beta \gamma^0 \beta^{-1} =&-\gamma^0 \gamma^0 \gamma^0 \\
=&+\gamma^0
\end{align*}
となる.わかりやすくまとめて書くと
\begin{align*}
\beta \gamma^\mu \beta^{-1}=\tensor{\mc{P}}{^\mu_\nu}\gamma^\nu
\end{align*}
となり,確かにパリティ変換になっている.($\beta^2=1$なので$\beta=\beta^{-1}$であり,区別する必要性は特にない.)同じ相似変換を$\gamma$行列の任意の積に施すと,その積に含まれる空間的な添え字$i$の数が偶数か奇数化に応じて,それぞれプラスまたはマイナス符号がつくだけである.例えば
\begin{align*}
\beta (\gamma^i \gamma^j \gamma^0) \beta^{-1}=(-1)^2\gamma^i \gamma^j \gamma^0=+\gamma^i \gamma^j \gamma^0 
\end{align*}
となる.特に重要なこととして,$\mc{S}^{ij}=-\frac{i}{4}(\gamma^i\gamma^j-\gamma^j\gamma^i),\mc{S}^{i0}=-\frac{i}{4}(\gamma^i\gamma^0-\gamma^0\gamma^i)$より
\begin{align*}
\beta \mc{S}^{ij}\beta^{-1} =&\mc{S}^{ij} \\
\beta \mc{S}^{i0}\beta^{-1} =&-\mc{S}^{i0}
\end{align*}
となる.回転部分とブースト部分を$\mc{J}_i:=\frac{1}{2}\epsilon_{ijk}\mc{S}^{jk},\mc{K}_i:=\mc{S}_{i0}$で定めれば,これは(2.6.7)(2.6.8)と似た関係
\begin{align*}
\beta \mc{J}_i \beta^{-1} =\mc{J}_i \\
\beta \mc{K}_i \beta^{-1} =-\mc{K}_i
\end{align*}
になる.上での$\gamma^\mu$の変換性を用いれば
\begin{align*}
\beta \mc{S}^{\mu\nu} \beta=&-\frac{i}{4}\beta [\gamma^\mu,\gamma^\nu]\beta \\
=&-\frac{i}{4}(\tensor{\mc{P}}{^\mu_\rho}\tensor{\mc{P}}{^\nu_\sigma}\gamma^\rho\gamma^\sigma-\tensor{\mc{P}}{^\mu_\rho}\tensor{\mc{P}}{^\nu_\sigma}\gamma^\sigma \gamma^\rho) \\
=&\tensor{\mc{P}}{^\mu_\rho}\tensor{\mc{P}}{^\nu_\sigma}\mc{S}^{\rho\sigma}
\end{align*}
と変換することがわかる.つまり$\mc{S}^{\rho\sigma}$は反対称擬テンソルではなく反対称テンソルである.(擬テンソルならさらにマイナスがつく.)


\vskip\baselineskip


この節でこれまで説明したことは全て任意の時空の次元数$D$と任意の「計量」$\eta^{\mu\nu}$について成立する.しかし$D=4$の4次元時空では完全反対称テンソルは4個以上の添え字を持てないので,1連のテンソル$\bm{1},\gamma^\rho ,\mc{S}^{\rho\sigma},\mc{A}^{\rho\sigma\tau},\cdots$はテンソル(5.4.12)$\mc{P}^{\rho\sigma\tau\eta}$で終わるという特別な事情がある.(もし5個の添え字を持つ完全反対称テンソル$\mc{P}^{\rho\sigma\tau\eta\alpha}$が存在すると,どれか2つが必ず同じ添え字になってしまい,完全反対称テンソルは添え字が被ったものはゼロになるので必ずゼロになる.)\par
さらに,これらのテンソルの各々は,ローレンツとパリティの同時(あるいは区別の)変換のもとで異なる変換をするので,それらは全て線形独立である.これらのテンソルの線形独立な成分の数は,$\bm{1}$について1個,$\gamma^\rho$について4個,$\mc{S}^{\rho\sigma}$について6個,$\mc{A}^{\rho\sigma\tau}$について4個,$\mc{P}^{\rho\sigma\tau\eta}$について1個,全部で16個である.(一般則として,$n$個の添え字を持つ$d$次元の完全反対称テンソル$\mc{A}^{\mu_1\cdots \mu_n}$は二項係数$\binom{d}{n}=d!/n!(d-n)!$に等しい個数の独立成分を持つ.$d$個の$0,1\cdots ,d-1$の添え字から順序を無視した$n$個を選び出す場合の数を考えればよい.)独立な$\nu\times \nu$行列の独立な数はせいぜい$\nu^2$なので,それらは少なくとも$\sqrt{16}=4$の行と列の数を持たねばならない.$4\times 4$ディラック行列は次元の数がこの最小値なので,必然的に既約な行列になっている.もし可約ならば,これらの行列により不変に保たれる部分空間はさらに低い表現を与えることになる.(例えば,$6\times 6$ディラック行列は既約な$4\times 4$ディラック行列をブロック対角に含むことになる.)\par
よって$\gamma^\mu$を$4\times 4$行列にとることにする.この節の最後に,$1,\gamma^\rho,\mc{S}^{\rho\sigma},\mc{A}^{\rho\sigma\tau},\mc{P}^{\rho\sigma\tau\eta}$が実際に$4\times 4$行列の基底をなし,任意の$4\times 4$行列がこれらの線形結合で一意に展開できることを数学的に証明する.

\vskip\baselineskip

より一般的には,時空の次元が任意の偶数$d$の場合,$0,1,\cdots d-1$の添え字を持つ反対称テンソルを作ることができ,$n$階の反対称テンソルは$d!/n!(d-n)!$だけの独立成分を持つのだったから,全ての反対称テンソルの独立成分の総数は
\begin{align*}
\sum_{n=0}^d \frac{d!}{n!(d-n)!}=\sum_{n=0}^d \binom{d}{n}1^n 1^{d-n}=(1+1)^d=2^d
\end{align*}
となり,$2^d$個の独立成分を含む.$2^d$個の行列要素を持つ行列は$2^{d/2}\times 2^{d/2}$行列であるから,ガンマ行列は少なくとも$2^{d/2}$の行と列を持たねばならない.\par
次元数が奇数の空間または時空では,$n$階と$d-n$階の完全反対称テンソルの間には$r=0,1,\cdots ,d-1$について条件
\begin{align*}
\gamma^{[\mu_1}\gamma^{\mu_2}\dots \gamma^{\mu_r]}\propto \epsilon^{\mu_1\mu_2\cdots \mu_d} \gamma_{[\gamma_{r+1}}\gamma_{\mu_{r+2}}\cdots \gamma_{\gamma_d]}
\end{align*}
という線形の関係がある.

\vskip\baselineskip

さて,4次元時空に戻って,$4\times 4$のガンマ行列を選ぼう.1つの非常に便利な選択肢は
\begin{align*}
\gamma^0=-i\left(
\begin{matrix}
0 & \bm{1} \\
\bm{1} & 0
\end{matrix}
\right),\quad \gamma^i=-i\left(
\begin{matrix}
0 & \sigma^i \\
-\sigma^i & 0
\end{matrix}
\right)
\end{align*}
である.ここで$\bm{1}$は単位$2\times 2$行列,また$\sigma^i$の成分は通常のパウリ行列
\begin{align*}
\sigma_1=\left(
\begin{matrix}
0 & 1 \\
1 & 0
\end{matrix}
\right),\quad \sigma_2=\left(
\begin{matrix}
0 & -i \\
i & 0
\end{matrix}
\right),\quad \sigma_3=\left(
\begin{matrix}
1 & 0 \\
0 & -1
\end{matrix}
\right)
\end{align*}
である.($\sigma_i$はちょうど3次元空間のガンマ行列になっている.実際
\begin{align*}
\{\sigma_i,\sigma_j\}=2\delta_{ij}\bm{1}_{2}
\end{align*}
である.)ブロック化しているので$4\times 4$行列であることがわかりにくいが,陽に書くと
\begin{align*}
\gamma^0=&-i\left(
\begin{matrix}
0 & 0 & 1 & 0 \\
0 & 0 & 0 & 1 \\
1 & 0 & 0 & 0 \\
0 & 1 & 0 & 0
\end{matrix}
\right)\\
\gamma^1=&-i\left(
\begin{matrix}
0 & 0 & 0 & 1 \\
0 & 0 & 1 & 0 \\
0 & -1 & 0 & 0 \\
-1 & 0 & 0 & 0
\end{matrix}
\right) ,\quad \gamma^2=-i\left(
\begin{matrix}
0 & 0 & 0 & -i \\
0 & 0 & +i & 0 \\
0 & +i & 0 & 0 \\
-i & 0 & 0 & 0
\end{matrix}
\right),\quad \gamma^3=-i\left(
\begin{matrix}
0 & 0 & 1 & 0 \\
0 & 0 & 0 & -1 \\
-1 & 0 & 0 & 0 \\
0 & 1 & 0 & 0
\end{matrix}
\right)
\end{align*}
となっている.(5.4.17)から,ローレンツ群の生成子(5.4.6)を簡単に計算できて
\begin{align*}
\mc{S}^{ij}=&-\frac{i}{4}[\gamma^i,\gamma^j] \\
=&-\frac{i}{4}(-i)^2\left(
\begin{matrix}
0 & \sigma_i \\
-\sigma_i & 0
\end{matrix}
\right)\left(
\begin{matrix}
0 & \sigma_j \\
-\sigma_j & 0
\end{matrix}
\right)+\frac{i}{4}(-i)^2\left(
\begin{matrix}
0 & \sigma_j \\
-\sigma_j & 0
\end{matrix}
\right)\left(
\begin{matrix}
0 & \sigma_i \\
-\sigma_i & 0
\end{matrix}
\right) \\
=&+\frac{i}{4}\left(
\begin{matrix}
-\sigma_i\sigma_j & 0 \\
0 & -\sigma_i\sigma_j
\end{matrix}
\right)-\frac{i}{4}\left(
\begin{matrix}
-\sigma_j\sigma_i & 0 \\
0 & -\sigma_j\sigma_i
\end{matrix}
\right) \\
=&-\frac{i}{4}\left(
\begin{matrix}
[\sigma_i,\sigma_j] & 0 \\
0 & [\sigma_i,\sigma_j]
\end{matrix}
\right) \\
=&-\frac{i}{4}\left(
\begin{matrix}
2i\epsilon_{ijk}\sigma_k & 0 \\
0 & 2i\epsilon_{ijk}\sigma_k
\end{matrix}
\right) \quad \because [\sigma_i,\sigma_j]=2i\epsilon_{ijk}\sigma_k \\
=&\frac{1}{2}\epsilon_{ijk}\left(
\begin{matrix}
\sigma_k & 0 \\
0 & \sigma_k
\end{matrix}
\right) \\
\mc{S}^{i0}=&-\frac{i}{4}[\gamma^i,\gamma^0] \\
=&-\frac{i}{4}(-i)^2\left(
\begin{matrix}
0 & \sigma_i \\
-\sigma_i & 0
\end{matrix}
\right)\left(
\begin{matrix}
0 & \bm{1} \\
\bm{1} & 0
\end{matrix}
\right)+\frac{i}{4}(-i)^2\left(
\begin{matrix}
0 & \bm{1} \\
\bm{1} & 0
\end{matrix}
\right)\left(
\begin{matrix}
0 & \sigma_i \\
-\sigma_i & 0
\end{matrix}
\right) \\
=&+\frac{i}{4}\left(
\begin{matrix}
+\sigma_i & 0 \\
0 & -\sigma_i
\end{matrix}
\right)-\frac{i}{4}\left(
\begin{matrix}
-\sigma_i & 0 \\
0 & +\sigma_i
\end{matrix}
\right) \\
=&+\frac{i}{2}\left(
\begin{matrix}
\sigma_i & 0 \\
0 & -\sigma_i
\end{matrix}
\right) \\
\therefore \quad & \mc{S}^{ij}=\frac{1}{2}\epsilon_{ijk}\left(
\begin{matrix}
\sigma_k & 0 \\
0 & \sigma_k
\end{matrix}
\right) \\
& \mc{S}^{i0}=-\mc{S}^{0i}=+\frac{i}{2}\left(
\begin{matrix}
\sigma_i & 0 \\
0 & -\sigma_i
\end{matrix}
\right)
\end{align*}
となる.これらはブロック対角的である!よってディラック行列は固有順時ローレンツ群の\uwave{可約}表現を与え
\begin{align*}
\mc{S}^{ij}=&\mc{S}_{L}^{ij}\oplus \mc{S}_R^{ij}=\left(
\begin{matrix}
\frac{1}{2}\epsilon_{ijk}\sigma_k & 0 \\
0 & 0
\end{matrix}
\right)+\left(
\begin{matrix}
0 & 0 \\
0 & \frac{1}{2}\epsilon_{ijk}\sigma_k
\end{matrix}
\right) \\
\mc{S}^{i0}=&\mc{S}_L^{i0}\oplus \mc{S}_R^{i0}=+\frac{i}{2}\left(
\begin{matrix}
\sigma_i & 0 \\
0 & 0
\end{matrix}
\right)+\frac{i}{2}\left(
\begin{matrix}
0 & 0 \\
0 & -\sigma_i
\end{matrix}
\right)
\end{align*}
と分解できる.すなわち,$\mc{S}^{ij}_L=-i\epsilon_{ijk}\mc{S}^{k0}_L$と$\mc{S}^{ij}_R=+i\epsilon_{ijk}\mc{S}^{k0}_R$となる二つの既約表現の直和を与えることがわかる.後のために,さらに
\begin{align*}
\mc{J}_i:=&\frac{1}{2}\epsilon_{ijk}\mc{S}^{ij}=\left(
\begin{matrix}
\frac{1}{2}\sigma_i & 0 \\
0 & \frac{1}{2}\sigma_i
\end{matrix}
\right) \quad \because \epsilon_{ijk}\epsilon_{ij\ell}=2\delta_{k\ell} \\
=&\mc{J}_{Li} \oplus \mc{J}_{Ri}=\left(
\begin{matrix}
\frac{1}{2}\sigma_i & 0 \\
0 & 0
\end{matrix}
\right)+\left(
\begin{matrix}
0 & 0 \\
0 & \frac{1}{2}\sigma_i
\end{matrix}
\right) \\
\mc{K}_i:=&\mc{S}_{i0}=\left(
\begin{matrix}
-\frac{i}{2}\sigma_i & 0 \\
0 & +\frac{i}{2}\sigma_i
\end{matrix}
\right) \\
=&\mc{K}_{Li} \oplus \mc{K}_{Ri}=\left(
\begin{matrix}
-\frac{i}{2}\sigma_i & 0 \\
0 & 0
\end{matrix}
\right)+\left(
\begin{matrix}
0 & 0 \\
0 & +\frac{i}{2}\sigma_i
\end{matrix}
\right)
\end{align*}
と回転部分とブースト部分に分けると
\begin{align*}
\mc{A}_i:=&\frac{1}{2}(\mc{J}_i+i\mc{K}_i)=\left(
\begin{matrix}
\sigma_i/2 & 0 \\
0 & 0
\end{matrix}
\right) \\
\mc{B}_i:=&\frac{1}{2}(\mc{J}_i-i\mc{K}_i)=\left(
\begin{matrix}
0& 0 \\
0 & \sigma_i/2
\end{matrix}
\right)
\end{align*}
と二つの独立な$SU(2)$スピンに分解できる.つまり,この$\mc{A}_i$と$\mc{B}_i$はそれぞれ
\begin{align*}
[\mc{A}_i,\mc{A}_j]=&i\epsilon_{ijk}\mc{A}_k \\
[\mc{B}_i,\mc{B}_j]=&i\epsilon_{ijk}\mc{B}_k \\
[\mc{A}_i,\mc{B}_j]=&0
\end{align*}
となり,別々に$\mathfrak{su}(2)$代数をなす.これはローレンツ代数$\mathfrak{so}(3,1)$が$\mathfrak{su}(2)_A\oplus \mathfrak{su}(2)_B$と同型になっていることを示している\footnote{数学的厳密性を考えると実際は同型ではない.これについては5.6節で詳しく議論することにする.}.この分解を用いると,ローレンツ群の表現は
\begin{align*}
D(\Lambda)=&\exp\left(+\frac{i}{2}\omega_{\mu\nu}\mc{S}^{\mu\nu}\right) \\
=&\exp\Bigl(+i\theta_i \mc{J}_i -i\omega_i \mc{K}_i\Bigr) \qquad (\theta_i:=\frac{1}{2}\epsilon_{ijk}\omega_{ij},\quad \omega_i:=\omega_{i0}) \\
=&\exp\left(+i(\theta_i+i\omega_i)\left(\frac{\mc{J}_i+i\mc{K}_i}{2}\right)+i(\theta_i-i\omega_i)\left(\frac{\mc{J}_i-i\mc{K}_i}{2}\right)\right) \\
=&\exp\left(+i\alpha_i \mc{A}_i +i\beta_i \mc{B}_i\right) \qquad (\alpha_i:=\theta_i+i\omega_i=:\beta^*_i)
\end{align*}
とできる.\par
このディラック行列の表示(5.4.17)は,ワイル表示と呼ばれている.(少しWeinberg特有の表記になっているが.)他にもディラック表示やマヨラナ表示などが知られている.しかしこれらは全てユニタリ同値(ユニタリー行列による相似変換で互いに移り替わる)であり,したがって全ての表示の仕方は等価である.一般に,$\{\gamma^\mu\}_{\mu=0,\cdots ,3}$と$\{\tilde{\gamma}^{\mu}\}_{\mu=0,\cdots,3}$がともにガンマ行列の代数(5.4.5)を満たすならば,どのような表示も相似変換$M\gamma^\mu M^{-1}=\tilde{\gamma}^\mu$で移り替わるという意味で同値であり,さらに追加の条件$(\gamma^0)^\dagger=-\gamma^0,(\gamma^i)^\dagger=+\gamma^i$を満たすガンマ行列の組同士は全てユニタリ同値($M^\dagger=M^{-1}$)であることを示すことができる.その証明をこの節の最後で与える.この事実により,用途に応じてディラック行列の表示を使い分けることが許される.


\vskip\baselineskip


完全反対称テンソル(5.4.11)と(5.4.12)はもう少し簡単な形に書くのが便利である.行列(5.4.12)は完全反対称テンソル(つまり添え字の順番で符号のみが変化し,符号を除いた値は(被りのない)添え字の順番によって変わらない)なので,$\epsilon^{0123}=+1$の\footnote{$\epsilon^{ijk}$は$\epsilon^{123}=\epsilon_{123}=+1$なので特に明記する必要がないが,$\eta^{00}=-1$によって$\epsilon^{0123}=-\epsilon_{0123}=+1$は下付きと上付きのどちらを$+1$にするかで符号が異なるので,毎回断りを入れなければならない.}完全反対称量として定義された擬テンソル$\epsilon^{\rho\sigma\tau\eta}$に比例する.
\begin{align*}
\mc{P}^{\rho\sigma\tau\eta}=\epsilon^{\rho\sigma\tau\eta}\mc{P}
\end{align*}
ここで$\mc{P}$は比例定数行列である.$\rho,\sigma,\tau,\eta$を$0,1,2,3$とおくと
\begin{align*}
\mc{P}^{0123}=&\gamma^0\gamma^1\gamma^2\gamma^3\pm(0,1,2,3の置換) \\
=&4!\gamma^0\gamma^1\gamma^2\gamma^3 \\
=\mc{P}\epsilon^{0123}=&\mc{P}
\end{align*}
であることがわかる.ここで,置換の項は全て$0,1,2,3$の偶置換ならば$+1$で,奇置換なら$-1$であり,各項の4つのガンマ行列の積は全て添え字が異なるのだから並べ替えたときに奇数回入れ替えたら$-1$が,偶数回入れ替えたら$+1$が出るので,係数と打ち消しあって結局$4!$個の$\gamma^0\gamma^1\gamma^2\gamma^3$が出てくることを用いた.したがって
\begin{align*}
\mc{P}^{\rho\sigma\tau\eta}=4!\epsilon^{\rho\sigma\tau\eta}\gamma^0\gamma^1\gamma^2\gamma^3=4!i\epsilon^{\rho\sigma\tau\eta}\gamma_5
\end{align*}
を得る.ここで
\begin{align*}
\gamma_5:=-i\gamma^0 \gamma^1 \gamma^2 \gamma^3
\end{align*}
である.行列$\gamma_5$は4つの異なる$\gamma^\mu$から構成されているから,任意の$\mu$に対して,$\gamma^\mu$と$\gamma_5$の入れ替えを考えると,自分自身$\gamma^\mu$以外の3つの$\gamma^\rho(\rho\neq \mu)$との反可換性により
\begin{align*}
\gamma_5\gamma^\mu=&(-1)^3 \gamma^\mu \gamma_5=-\gamma^\mu \gamma_5 \\
\therefore\quad \{\gamma_5,\gamma^\mu\}=&0
\end{align*}
のように全ての$\gamma^\mu$と反可換することがわかる.(適当に$\mu$を1や2として交換してみれば具体的でわかりやすい.
\begin{align*}
\gamma_5\gamma^2=&-i\gamma^0\gamma^1\gamma^2\gamma^3 \gamma^2 \\
=&-i(-1)\gamma^0 \gamma^1\gamma^2 \gamma^2 \gamma^3 \qquad (\gamma^3 と \gamma^2 の入れ替え)\\
=&-i(-1)^2\gamma^0 \gamma^2\gamma^1 \gamma^2 \gamma^3 \qquad (\gamma^1 と \gamma^2 の入れ替え)\\
=&-i(-1)^3\gamma^2 \gamma^0\gamma^1 \gamma^2 \gamma^3 \qquad (\gamma^0 と \gamma^2 の入れ替え)\\
=&(-1)^3\gamma^2 \gamma_5=-\gamma^2\gamma_5
\end{align*}
となる.これが全ての$\mu$で同様にできる.)これを用いると,$\gamma_5$は$\mc{S}^{\rho\sigma}$と交換することが分かる.
\begin{align*}
\mc{S}^{\rho\sigma}\gamma_5=&-\frac{i}{4}(\gamma^\rho \gamma^\sigma -\gamma^\sigma \gamma^\rho)\gamma_5 \\
=&-\frac{i}{4}(-\gamma^\rho\gamma_5 \gamma^\sigma-(-1)\gamma^\sigma \gamma_5 \gamma^\rho) \\
=&-\frac{i}{4}((-1)^2\gamma_5\gamma^\rho \gamma^\sigma-\gamma_5(-1)^2\gamma^\sigma \gamma^\rho) \\
=&-\frac{i}{4}\gamma_5(\gamma^\rho \gamma^\sigma -\gamma^\sigma \gamma^\rho) \\
=&\gamma_5 \mc{S}^{\rho\sigma} \\
\therefore\quad [\mc{S}^{\rho\sigma},\gamma_5]=&0
\end{align*}
しかしパリティ変換$\beta$のもとで符号を変化させる.
\begin{align*}
\beta \gamma_5 \beta^{-1}=&(i\gamma^0)\gamma_5 (i\gamma^0) \\
=&-\gamma_5 \beta \beta \\
=&-\gamma_5
\end{align*}
したがって固有順時ローレンツ変換のもとでスカラーだが,パリティの下で符号が反転するので
\begin{align*}
D(\Lambda)\gamma_5 D(\Lambda)=&\gamma_5 \\
\beta \gamma_5 \beta^{-1}=&-\gamma_5
\end{align*}
の意味で擬スカラーである.\par
同様に,$\mc{A}^{\rho\sigma\tau}$はある行列$\mc{A}_\eta$と縮約した$\epsilon^{\rho\sigma\tau\eta}$に比例しなければならない.なぜなら,添え字$\rho\sigma\tau$は完全反対称性により$0,1,2,3$のどれか一つ以外をとる(被りがあればゼロになる)が,無い添え字一つを固定したら後は添え字の順番で符号のみが変化し値は変わらない.したがって,$\rho\sigma\tau$に無い添え字$\eta$(例えば$\rho\sigma\tau=013$で$\eta=2$)に依存する行列値$\mc{A}_\eta$を縮約すればよい.
\begin{align*}
\mc{A}^{\rho\sigma\tau}=\epsilon^{\rho\sigma\tau\eta}\mc{A}_\eta
\end{align*}
$\rho,\sigma,\tau$を$0,1,2$とおけば
\begin{align*}
\mc{A}^{012}=&\gamma^0 \gamma^1 \gamma^2\pm(0,1,2の置換) \\
=&3!\gamma^0\gamma^1\gamma^2 \\
=&3!\gamma^0\gamma^1\gamma^2 \gamma^3 \gamma^3 \quad \because \bm{1}=(\gamma^i)^2\\
=&3!\gamma^0\gamma^1\gamma^2 \gamma^3 \gamma_3 \quad \because \gamma^i=\gamma_i\\
=&3!i\gamma_5 \gamma_3 \\
=\epsilon^{012\eta}\mc{A}_\eta =&\epsilon^{0123}\mc{A}_3=\mc{A}_3 \\
\therefore \quad \mc{A}_3=&3!i\gamma_5 \gamma_3
\end{align*}
$0,1,3$とおけば
\begin{align*}
\mc{A}^{013}=&\gamma^0 \gamma^1 \gamma^3\pm(0,1,3の置換) \\
=&3!\gamma^0\gamma^1\gamma^3 \\
=&3!\gamma^0\gamma^1\gamma^2 \gamma^2 \gamma^3 \quad \because \bm{1}=(\gamma^i)^2\\
=&- 3!\gamma^0\gamma^1\gamma^2 \gamma^3 \gamma^2 \quad \because \gamma^i\gamma^j=-\gamma^j\gamma^i\quad (i\neq j)\\
=&-3!\gamma^0\gamma^1\gamma^2 \gamma^3 \gamma_2 \quad \because \gamma^i=\gamma_i\\
=&-3!i\gamma_5 \gamma_2 \\
=\epsilon^{013\eta}\mc{A}_\eta =&\epsilon^{0132}\mc{A}_2=-\mc{A}_2 \\
\therefore \quad \mc{A}_2=&3!i\gamma_5 \gamma_2 
\end{align*}
$0,2,3$とおけば
\begin{align*}
\mc{A}^{023}=&\gamma^0 \gamma^2 \gamma^3\pm(0,2,3の置換) \\
=&3!\gamma^0\gamma^2\gamma^3 \\
=&3!\gamma^0\gamma^1\gamma^1 \gamma^2 \gamma^3 \quad \because \bm{1}=(\gamma^i)^2\\
=&(-1)^2 3!\gamma^0\gamma^1\gamma^2 \gamma^3 \gamma^1 \quad \because \gamma^i\gamma^j=-\gamma^j\gamma^i\quad (i\neq j)\\
=&+3!\gamma^0\gamma^1\gamma^2 \gamma^3 \gamma_1 \quad \because \gamma^i=\gamma_i\\
=&+3!i\gamma_5 \gamma_1 \\
=\epsilon^{023\eta}\mc{A}_\eta =&\epsilon^{0231}\mc{A}_1=+\mc{A}_1 \\
\therefore \quad \mc{A}_1=&3!i\gamma_5 \gamma_1 
\end{align*}
$1,2,3$とおけば
\begin{align*}
\mc{A}^{123}=&\gamma^1 \gamma^2 \gamma^3\pm(1,2,3の置換) \\
=&3!\gamma^1\gamma^2\gamma^3 \\
=&-3!\gamma^0\gamma^0\gamma^1 \gamma^2 \gamma^3 \quad \because \bm{1}=-(\gamma^0)^2\\
=&-(-1)^3 3!\gamma^0\gamma^1\gamma^2 \gamma^3 \gamma^0 \quad \because \gamma^0\gamma^j=-\gamma^j\gamma^0 \\
=&-(-1)^4!\gamma^0\gamma^1\gamma^2 \gamma^3 \gamma_0 \quad \because \gamma^0=-\gamma_0\\
=&-3!i\gamma_5 \gamma_0 \\
=\epsilon^{123\eta}\mc{A}_\eta =&\epsilon^{1230}\mc{A}_0=-\mc{A}_0 \\
\therefore \quad \mc{A}_0=&3!i\gamma_5 \gamma_0
\end{align*}
以上より$\mc{A}_\eta=3!i\gamma_5 \gamma_\eta$であり
\begin{align*}
\mc{A}^{\rho\sigma\tau}=3!i\epsilon^{\rho\sigma\tau\eta}\gamma_5 \gamma_\eta
\end{align*}
を得る.この$\gamma_5 \gamma^\eta$という量は,固有順時ローレンツ変換のもとでベクトルとして振舞うが,パリティのもとでベクトル的な変換に加えて余分に符号を変える.
\begin{align*}
D(\Lambda)(\gamma_5 \gamma_\rho) D^{-1}(\Lambda)=&\tensor{\Lambda}{_\sigma^\rho}\gamma_5 \gamma^\sigma \\
\beta (\gamma_5 \gamma^\rho) \beta^{-1}=&-\tensor{\mc{P}}{^\rho_\sigma}(\gamma_5 \gamma^\rho)
\end{align*}
この意味でこれは軸性ベクトルである.\par
よって16個の独立な$4\times 4$行列は,スカラー$\bm{1}$,ベクトル$\gamma^\rho$,反対称テンソル$\mc{S}^{\rho\sigma}$,「軸性」ベクトル$\gamma_5 \gamma_\eta$,擬スカラー$\gamma_5$の各成分にとれる.つまり任意の$4\times 4$行列$M$は
\begin{align*}
M=a_I\bm{1} +a_\mu \gamma^\mu +\frac{1}{2}a_{\mu\nu}\mc{S}^{\mu\nu}+a_{5\mu}i\gamma^\mu \gamma_5 +a_5 \gamma_5
\end{align*}
と一意的に展開することができる.\par
行列$\gamma_5$は二乗が単位行列になっている.実際
\begin{align*}
\gamma_5^2=&(-i\gamma^0 \gamma^1 \gamma^2 \gamma^3)(-i\gamma^0 \gamma^1 \gamma^2 \gamma^3) \\
=&-\gamma^0 \gamma^1 \gamma^2 \gamma^3 \gamma^0 \gamma^1 \gamma^2 \gamma^3 \\
=&(-1)^2\gamma^0 \gamma^1 \gamma^2 \gamma^0 \gamma^3 \gamma^1 \gamma^2 \gamma^3 \qquad (\gamma^3 \gamma^0 =-\gamma^0 \gamma^3) \\
=&(-1)^3\gamma^0 \gamma^1 \gamma^0 \gamma^2 \gamma^3 \gamma^1 \gamma^2 \gamma^3 \qquad (\gamma^2 \gamma^0 =-\gamma^0 \gamma^2) \\
=&(-1)^4\gamma^0 \gamma^0 \gamma^1 \gamma^2 \gamma^3 \gamma^1 \gamma^2 \gamma^3 \qquad (\gamma^1 \gamma^0 =-\gamma^0 \gamma^1) \\
=&(-1)^5\gamma^1 \gamma^2 \gamma^3 \gamma^1 \gamma^2 \gamma^3 \qquad \because (\gamma^0)^2=-\bm{1} \\
=&(-1)^6\gamma^1 \gamma^2 \gamma^1 \gamma^3 \gamma^2 \gamma^3 \qquad (\gamma^3\gamma^1=-\gamma^1\gamma^3) \\
=&(-1)^7\gamma^1 \gamma^1 \gamma^2 \gamma^3 \gamma^2 \gamma^3 \qquad (\gamma^2\gamma^1=-\gamma^1\gamma^2) \\
=&(-1)^7 \gamma^2 \gamma^3 \gamma^2 \gamma^3 \qquad \because (\gamma^1)^2=+\bm{1} \\
=&(-1)^8 \gamma^2 \gamma^2 \gamma^3 \gamma^3 \qquad (\gamma^3\gamma^2=-\gamma^2\gamma^3) \\
=&(-1)^8 \bm{1} \qquad \because (\gamma^2)^2=(\gamma^3)+\bm{1} \\
=&+\bm{1}
\end{align*}
である.先程示した
\begin{align*}
\{\gamma_5 ,\gamma^\mu\}=0
\end{align*}
と合わせると,これは$\mu$を5次元にまで拡張した
\begin{align*}
\{\gamma^\mu ,\gamma^\nu\}=2\eta^{\mu\nu}
\end{align*}
になっている.(実際にこれは,$\mu,\nu$にともに5を入れると$\gamma_5^2=\bm{1}$を表し,$\mu$を$0,1,\cdots 3$として$\nu$を5とすると$\{\gamma_5 ,\gamma^\mu\}=0$を表す.)この意味で$\gamma_5$という表記は的を得ている.なぜならば$\gamma^0,\gamma^1,\gamma^2,\gamma^3,\gamma_5$は5次元のクリフォード代数を与えているからである.ガンマ行列の$4\times 4$ワイル表示(5.4.17)に対して,行列$\gamma_5$は
\begin{align*}
\gamma_5=&-i(-i)^4\left(
\begin{matrix}
0 & \bm{1} \\
\bm{1} & 0
\end{matrix}
\right)\left(
\begin{matrix}
0 & \sigma_1 \\
-\sigma_1 & 0
\end{matrix}
\right)\left(
\begin{matrix}
0 & \sigma_2 \\
-\sigma_2 & 0
\end{matrix}
\right)\left(
\begin{matrix}
0 & \sigma_3 \\
-\sigma_3 & 0
\end{matrix}
\right) \\
=&-i\left(
\begin{matrix}
-\sigma_1 & 0 \\
0 & \sigma_1
\end{matrix}
\right)\left(
\begin{matrix}
0 & \sigma_2 \\
-\sigma_2 & 0
\end{matrix}
\right)\left(
\begin{matrix}
0 & \sigma_3 \\
-\sigma_3 & 0
\end{matrix}
\right) \\
=&-i\left(
\begin{matrix}
0 & -\sigma_1\sigma_2 \\
-\sigma_1\sigma_2 & 0
\end{matrix}
\right)\left(
\begin{matrix}
0 & \sigma_3 \\
-\sigma_3 & 0
\end{matrix}
\right) \\
=&-i\left(
\begin{matrix}
\sigma_1\sigma_2\sigma_3 & 0 \\
0 & -\sigma_1\sigma_2\sigma_3
\end{matrix}
\right) \\
\sigma_1\sigma_2\sigma_3=&\left(
\begin{matrix}
0 & 1 \\
1 & 0
\end{matrix}
\right)\left(
\begin{matrix}
0 & -i \\
i & 0
\end{matrix}
\right)\left(
\begin{matrix}
1 & 0 \\
0 & -1
\end{matrix}
\right)=i\bm{1}_2 \\
\therefore \quad \gamma_5=&\left(
\begin{matrix}
\bm{1} & 0 \\
0 & -\bm{1}
\end{matrix}
\right)
\end{align*}
となる.したがって,このワイル表示は$\mc{S}^{\rho\sigma}$と$\gamma_5$をブロック対角型にするので便利である!後で見るように,この性質のために,この表示は超相対論的極限$v\to c=1$の粒子を扱う際に特に有用となる.(この$\gamma_5$はカイラリティを表し,ワイル表示はカイラル表示とも呼ばれる.)\par
$\gamma_5$は$\pm 1$の固有値を持ち,ローレンツ変換$D(\Lambda)$は$\gamma_5$と交換するので,$\gamma_5$の固有値$+1$の部分空間と固有値$-1$の部分空間はローレンツ変換のもとで混ざりあわない.したがって固有値$\pm 1$の不変部分空間で既約分解することができ,実際(5.4.17)(5.4.18)はそのように分解されている.


\vskip\baselineskip


ここで構成した斉次ローレンツ群の表現$D(\Lambda)$はユニタリー行列ではないことに注意!なぜなら,生成子$\mc{S}^{\rho\sigma}$が全てエルミート行列で表現されているわけではないからである.(もしエルミートならば$D(\Lambda)^\dagger=\exp(\frac{i}{2}\omega_{\mu\nu}\mc{S}^{\mu\nu})^\dagger=\exp(-\frac{i}{2}\omega_{\mu\nu}\mc{S}^{\mu\nu})=D^{-1}(\Lambda)$となり,ユニタリーだっただろう.)特に,(5.4.17)の表示を見れば,$\mc{S}^{ij}$はエルミートであるが$\mc{S}^{i0}$は\uwave{反}エルミートである.
\begin{align*}
(\mc{S}^{ij})^\dagger=\mc{S}^{ij} ,\quad (\mc{S}^{i0})^\dagger=-\mc{S}^{i0}
\end{align*}
そのような実条件は,(5.4.13)の行列$\beta:=i\gamma^0$を導入すれば,明白にローレンツ不変な形で書けて便利になる.この行列は,(5.4.17)の表示では
\begin{align*}
\beta=\left(
\begin{matrix}
0& \bm{1} \\
\bm{1} &0 
\end{matrix}
\right)
\end{align*}
の形をとる.(5.4.17)を見れば
\begin{align*}
(\gamma^0)^\dagger=&-\gamma^0,\quad (\gamma^i)^\dagger=+\gamma^i \\
\therefore \quad (\gamma^\mu)^\dagger=&-\tensor{\mc{P}}{^\mu_\nu}\gamma^\nu
\end{align*}
であり,(5.4.14)より
\begin{align*}
\beta \gamma^\mu \beta=&\tensor{\mc{P}}{^\mu_\nu}\gamma^\nu=-(\gamma^\mu)^\dagger \\
\therefore\quad \beta (\gamma^\mu)^\dagger \beta=&-\gamma^\mu
\end{align*}
がわかる.これより
\begin{align*}
\beta (\mc{S}^{\rho\sigma})^\dagger \beta=&\beta \left(+\frac{i}{4}(\gamma^\rho \gamma^\sigma -\gamma^\sigma \gamma^\rho)^\dagger\right) \beta \\
=&\beta \left(+\frac{i}{4}\Bigl((\gamma^\sigma)^\dagger (\gamma^\rho)^\dagger -(\gamma^\rho)^\dagger ( \gamma^\rho)^\dagger\Bigr)\right) \beta \\
=&+\frac{i}{4}\Bigl(\beta(\gamma^\sigma)^\dagger\beta \beta (\gamma^\rho)^\dagger \beta -\beta (\gamma^\rho)^\dagger \beta \beta (\gamma^\rho)^\dagger\beta \Bigr) \\
=&+\frac{i}{4}(\gamma^\sigma \gamma^\rho -\gamma^\rho \gamma^\sigma) \\
=&-\frac{i}{4}(\gamma^\rho \gamma^\sigma-\gamma^\sigma \gamma^\rho )\\
=&-frac{i}{4}[\gamma^\rho,\gamma^\sigma]=\mc{S}^{\rho\sigma} \\
\therefore \quad \beta (\mc{S}^{\rho\sigma})^\dagger \beta=&\mc{S}^{\rho\sigma}
\end{align*}
を得る\footnote{$\mc{S}^{\rho\sigma}$という行列についてのエルミート(転置$+$複素共役)であり,添え字$\rho,\sigma$についての転置は含んでいないことに注意.この$\rho,\sigma$は基底$\{\mc{S}^{\rho\sigma}\}$同士を区別するためのラベルであり,行列の成分の意味では使われていないからである.}.よって,行列$D(\Lambda)$はユニタリーではないが,擬ユニタリーである.
\begin{align*}
\beta D(\Lambda)^\dagger \beta =&\beta \exp\left(+\frac{i}{2}\omega_{\mu\nu}\mc{S}^{\mu\nu}\right)^\dagger \beta \\
=&\beta \exp\left(-\frac{i}{2}\omega_{\mu\nu}(\mc{S}^{\mu\nu})^\dagger\right) \beta \\
=&\beta \sum_{n=0}^\infty \frac{1}{n!} \left[-\frac{i}{2}\omega_{\mu\nu}(\mc{S}^{\mu\nu})^\dagger\right]^n\beta \\
=&\sum_{n=0}^\infty \frac{1}{n!} \left[-\frac{i}{2}\omega_{\mu\nu}\beta (\mc{S}^{\mu\nu})^\dagger \beta \right]^n \quad \because \beta=\beta^{-1}\\
=&\sum_{n=0}^\infty \frac{1}{n!} \left[-\frac{i}{2}\omega_{\mu\nu}\mc{S}^{\mu\nu}\right]^n \\
=&D^{-1}(\Lambda) \\
\therefore\quad \beta D(\Lambda)^\dagger \beta =& D^{-1}(\Lambda)
\end{align*}
また,$\gamma_5$はエルミートであり,さらに$\beta=i\gamma^0$と反可換するから
\begin{align*}
\beta \gamma^\dagger_5 \beta =-\gamma_5
\end{align*}
であり,よって
\begin{align*}
\beta(\gamma_5 \gamma_\mu)^\dagger \beta=&\beta(\gamma_\mu)^\dagger (\gamma_5)^\dagger \beta \\
=&\beta(\gamma_\mu)^\dagger\beta \beta (\gamma_5)^\dagger \beta \\
=&+\gamma^\mu \gamma_5 \\
=&-\gamma_5 \gamma^\mu \\
\therefore \quad \beta(\gamma_5 \gamma_\mu)^\dagger \beta=&-\gamma_5 \gamma^\mu
\end{align*}
が導かれる.



\vskip\baselineskip


ディラック行列およびそれに関連した行列は,また重要な対称性を持つ.(5.4.17)と(5.4.18)より$\gamma^\mu$は$\mu=0,2$について対称で,$\mu=1,3$について反対称である.
\begin{align*}
(\gamma^0)^T=&-i\left(
\begin{matrix}
0 & 0 & 1 & 0 \\
0 & 0 & 0 & 1 \\
1 & 0 & 0 & 0 \\
0 & 1 & 0 & 0
\end{matrix}
\right)=\gamma^0, \quad (\gamma^2)^T=-i\left(
\begin{matrix}
0 & 0 & 0 & -i \\
0 & 0 & +i & 0 \\
0 & +i & 0 & 0 \\
-i & 0 & 0 & 0
\end{matrix}
\right)=\gamma^2\\
(\gamma^1)^T=&-i\left(
\begin{matrix}
0 & 0 & 0 & 1 \\
0 & 0 & 1 & 0 \\
0 & -1 & 0 & 0 \\
-1 & 0 & 0 & 0
\end{matrix}
\right)=-\gamma^1 ,\quad (\gamma^3)^T=-i\left(
\begin{matrix}
0 & 0 & 1 & 0 \\
0 & 0 & 0 & -1 \\
-1 & 0 & 0 & 0 \\
0 & 1 & 0 & 0
\end{matrix}
\right)=-\gamma^3
\end{align*}
したがって,これを
\begin{align*}
(\gamma^\mu)^T =& -(\gamma^2 i\gamma^0) \gamma^\mu (\gamma^2i\gamma^0)^{-1} \\
=&-\mc{C} \gamma^\mu \mc{C}^{-1}
\end{align*}
と書くことができる.ここで
\begin{align*}
\mc{C}:=\gamma^2 \beta =-i\left(
\begin{matrix}
0 & \sigma_2 \\
-\sigma_2 & 0
\end{matrix}
\right)\left(
\begin{matrix}
0 & \bm{1} \\
\bm{1} & 0
\end{matrix}
\right)=-i\left(
\begin{matrix}
\sigma_2 & 0 \\
0 & -\sigma_2
\end{matrix}
\right)=\left(
\begin{matrix}
0 & -1 & 0 & 0 \\
1 & 0 & 0 & 0 \\
0 & 0 & 0 & 1 \\
0 & 0 & -1 & 0
\end{matrix}
\right)
\end{align*}
である.(実際,$\mu=0,2$では$\gamma^\mu$は$\gamma^2 \beta$と反可換するから,$(\gamma^\mu)^T =+\gamma^\mu$となる.$\mu=1,3$は$\gamma^2\beta$と二回反可換するから可換であり,したがって$(\gamma^\mu)^T=-\gamma^\mu$となる.これは転置での性質にほかならない.)この行列$\mc{C}$は
\begin{align*}
\mc{C}^2=&\gamma_2\beta \gamma_2 \beta =-\gamma_2 \gamma_2 \beta \beta=-\bm{1} \\
\mc{C}^\dagger=&+i\left(
\begin{matrix}
\sigma_2^\dagger & 0 \\
0 & -\sigma_2^\dagger
\end{matrix}
\right)=+i\left(
\begin{matrix}
\sigma_2 & 0 \\
0 & -\sigma_2
\end{matrix}
\right)=-\mc{C} \\
\therefore \quad \mc{C}^{-1}=&\mc{C}^\dagger=-\mc{C}
\end{align*}
というユニタリーかつ反エルミートな性質がある.ここから
\begin{align*}
(\mc{S}^{\rho\sigma})^T=&-\frac{i}{4}(\gamma^\rho\gamma^\sigma-\gamma^\sigma \gamma^\rho)^T \\
=&-\frac{i}{4}\Bigl((\gamma^\sigma)^T(\gamma^\rho)^T -(\gamma^\rho)^T (\gamma^\sigma)^T\Bigr) \\
=&-\frac{i}{4}\Bigl([-\mc{C}\gamma^\sigma \mc{C}^{-1}][-\mc{C}\gamma^\rho \mc{C}^{-1}] -[-\mc{C}\gamma^\rho \mc{C}^{-1}] [-\mc{C}\gamma^\sigma \mc{C}^{-1}]\Bigr) \\
=&-\mc{C}\left[\frac{i}{4}(\gamma^\sigma\gamma^\rho- \gamma^\rho\gamma^\sigma)\right]\mc{C}^{-1} \\
=&-\mc{C}\left[-\frac{i}{4}(\gamma^\rho\gamma^\sigma- \gamma^\sigma\gamma^\rho)\right]\mc{C}^{-1} \\
=&-\mc{C}\mc{S}^{\rho\sigma}\mc{C}^{-1} \\
\therefore \quad (\mc{S}^{\rho\sigma})^T=&-\mc{C}\mc{S}^{\rho\sigma}\mc{C}^{-1}
\end{align*}
や
\begin{align*}
(\gamma_5)^T=+\mc{C}\gamma_5 \mc{C}^{-1}
\end{align*}
($\gamma_5$は$\gamma^2$と$\gamma^0$のどちらとも反可換であるから$\mc{C}$と可換なので当然.)
\begin{align*}
(\gamma_5 \gamma^\mu)^T=&(\gamma^\mu)^T \gamma_5 \\
=&[-\mc{C}\gamma^\mu \mc{C}^{-1}]\mc{C}\gamma_5 \mc{C}^{-1} \\
=&-\mc{C}\gamma^\mu \gamma_5 \mc{C}^{-1} \\
=&+\mc{C} \gamma_5 \gamma^\mu \mc{C}^{-1}
\end{align*}
が導かれる.これらの符号は次節で様々なカレントの荷電共役の性質を考察する際に重要になる.もtろん,上記の随伴と転置の結果を組み合わせて,ディラック行列および関連行列の複素共役行列を得ることができる.
\begin{align*}
(\gamma^\mu)^*=&[(\gamma^\mu)^\dagger]^T \\
=&[-\beta \gamma^\mu \beta]^T \\
=&-\beta (\gamma^\mu)^T \beta \quad \because \beta^T =\beta\\
=&+ \beta \mc{C} \gamma^\mu \mc{C}^{-1} \beta \\
(\mc{S}^{\rho\sigma})^*=&[(\mc{S}^{\rho\sigma})^\dagger]^T \\
=&[+\beta \mc{S}^{\rho\sigma}\beta]^T \\
=&+\beta (\mc{S}^{\rho\sigma})^T \beta \\
=&-\beta \mc{C}\mc{S}^{\rho\sigma}\mc{C}^{-1}\beta \\
\gamma_5^*=&[\gamma_5^\dagger]^T \\
=&[-\beta\gamma_5 \beta]^T \\
=&-\beta \mc{C} \gamma_5 \mc{C}^{-1} \beta \\
(\gamma_5 \gamma^\mu)^*=&[(\gamma_5\gamma^\mu)^\dagger]^T \\
=&[-\beta \gamma_5 \gamma^\mu \beta]^T \\
=&-\beta (\gamma_5 \gamma^\mu )^T\beta \\
=&-\beta \mc{C} \gamma_5 \gamma^\mu \mc{C}^{-1} \beta
\end{align*}
これらの公式は,何度も使うことになるので,もはや暗記したほうがいい(とはいえ,Peskinなどの他の本とは定義が異なるので注意すること.)

\vskip\baselineskip

最後に,4次元時空でのガンマ行列の一般的な代数構造について説明しておこう.以下で,ガンマ行列の積が$4\times 4$行列の基底をなすことの証明を与える.\par
ガンマ行列の定義は(5.4.5),すなわち二つの異なる$\gamma$は反可換であり,二乗すると$\pm 1$になり,自分自身とは可換であるということである.これを用いると,任意の数の$\gamma$行列の積は
\begin{align*}
\Gamma=&\gamma^{\mu_1}\gamma^{\mu_2}\cdots \gamma^{\mu_N} \\
=&\pm (\gamma^0)^{n_0} (\gamma^1)^{n_1} (\gamma^2)^{n_2} (\gamma^3)^{n_3}
\end{align*}
という形になる.実際,反可換か可換のどちらかであるから,入れ替えて$\gamma^0$が一番左に,$\gamma^1$が次に,…というように並び替えられる.ここで$n_0,\cdots ,n_3$はそれぞれ$\gamma^0,\cdots \gamma^3$の数($\sum_i n_i=N$)であり,$\pm$は異なるガンマ行列の入れ替えが奇数回あれば$-1$となる符号である.さらに,ガンマ行列の二乗は$\pm 1$だから,各$n_\mu$は0か1にできる.\par
以上より,ガンマ行列の任意個数の積は,$\{n_\mu\}_{\mu=0,\cdots ,3}$の組み合わせで区別でき,符号を除いて$2^4=16$個の行列
\begin{align*}
\Gamma=\Bigl\{\bm{1},\quad \gamma^\mu ,\quad \gamma^\mu \gamma^\nu \, (\mu <\nu ), \quad \gamma^\mu \gamma^\nu \gamma^\rho \, (\mu < \nu < \rho),\quad  \gamma^0\gamma^1 \gamma^2 \gamma^3 \Bigr\}
\end{align*}
のいずれかに等しい.それぞれ$1,4,6,4,1$個あり,全部で16個ある.(これは$4\times 4$行列の自由度に等しい.)ここで
\begin{align*}
\gamma^5:=-i\gamma^0\gamma^1 \gamma^2 \gamma^3
\end{align*}
を定めると,これは
\begin{align*}
(\gamma_5)^2=+1,\quad \gamma_5 \gamma^\mu =-\gamma^\mu \gamma_5
\end{align*}
を満たす.さらに
\begin{align*}
\sigma^{\mu\nu}:=i\gamma^\mu \gamma^\nu
\end{align*}
も定める.これにより上の行列の集合は
\begin{align*}
\Gamma^A=\Bigl\{\bm{1},\quad \gamma^\mu ,\quad \sigma^{\mu\nu} \, (\mu <\nu ), \quad i\gamma^\mu \gamma_5,\quad  \gamma_5 \Bigr\}
\end{align*}
で書きなおせる.ここで$A=1,\cdots, 16$はこの行列の集合の元を指定する添え字であり,例えば$\Gamma^1=\bm{1},\Gamma^3=\gamma^1$などである.\par
上付き添え字を下したものを
\begin{align*}
\Gamma_A=\Bigl\{\bm{1},\quad \gamma_\mu ,\quad \sigma_{\mu\nu} \, (\mu <\nu ), \quad i\gamma_\mu \gamma_5,\quad  \gamma_5 \Bigr\}
\end{align*}
で定めると,これは
\begin{align*}
\Gamma_A=\left\{
\begin{array}{ll}
\Gamma^A \quad &(\Gamma^A=\bm{1},\gamma^i,\sigma^{ij},i\gamma^i \gamma_5 ,\gamma_5)\\
-\Gamma^A & (それ以外)
\end{array}
\right.
\end{align*}
になっている.さらにこれは各$A$について
\begin{align*}
\Gamma^A \Gamma_A=&\bm{1} \\
\therefore \quad (\Gamma^A)^{-1}=&\Gamma_A
\end{align*}
となっている.($A$について和をとってはいないことに注意.)実際
\begin{align*}
\bm{1}\cdot \bm{1}=&+\bm{1} \\
\gamma^i \cdot \gamma^i=&+\bm{1} \\
\gamma^0\cdot (-\gamma^0)=&+\bm{1} \\
i\gamma^i \gamma^j \cdot i\gamma^i \gamma^j=&-\gamma^i \gamma^j \gamma^i \gamma^j=+\gamma^i \gamma^i \gamma^j \gamma^j=+\bm{1} \\
i\gamma^0 \gamma^i \cdot (-i\gamma^0 \gamma^i)=&+\gamma^0\gamma^i \gamma^0 \gamma^i=-\gamma^0 \gamma^0 \gamma^i \gamma^i=+\bm{1} \\
i\gamma^i \gamma_5 \cdot i\gamma^i \gamma_5=&-\gamma^i \gamma_5 \gamma^i \gamma_5= +\gamma^i \gamma^i \gamma_5 \gamma_5 =+\bm{1} \\
i\gamma^0 \gamma_5 \cdot (-i\gamma^0 \gamma_5)=&=+\gamma^0 \gamma_5 \gamma^0 \gamma_5=-\gamma^0 \gamma^0 \gamma_5 \gamma_5 =+\bm{1} \\
\gamma_5 \cdot \gamma_5=&+\bm{1}
\end{align*}
となっている.\par
ここで,$\bm{1}$以外の$\Gamma^A$に対して
\begin{align*}
\Gamma^B \Gamma^A \Gamma_B =-\Gamma^A
\end{align*}
を満たす$\Gamma^B$が必ず存在する.(この式を変形すると$\Gamma^B \Gamma^A=-\Gamma^A \Gamma^B$であるから,$\Gamma^A=\gamma^\mu ,\sigma^{\mu\nu},i\gamma^\mu \gamma_5 ,\gamma_5$に対して,それぞれ$\Gamma^B=\gamma_5 ,\gamma^\mu ,\gamma_5 ,\gamma^\mu$ととればよい.これ以外にもとりかたはある.)両辺のトレースをとると
\begin{align*}
(\mathrm{LHS})=&\mathrm{tr}\left[\Gamma^B \Gamma^A \Gamma_B\right] =\mathrm{tr}\left[\Gamma_B\Gamma^B \Gamma^A\right]=\mathrm{tr}\left[\Gamma^A\right] \\
(\mathrm{RHS})=&-\mathrm{tr}\left[\Gamma^A\right] \\
\therefore \quad \mathrm{tr}\left[\Gamma^A\right]=&0 \quad (\Gamma^A\neq \bm{1})
\end{align*}
がわかる.\par
次に,2個の積のトレース
\begin{align*}
\mathrm{tr}\left[\Gamma^A\Gamma_B\right]
\end{align*}
を考える.$A=B$のとき,$\Gamma^A\Gamma_A=\bm{1}$より
\begin{align*}
\mathrm{tr}\left[\Gamma^A\Gamma_B\right]=\mathrm{tr}\left[\bm{1}\right]=4
\end{align*}
となる.$A\neq B$のとき,$\Gamma^A\Gamma_B$は必ず$\gamma^\mu$を1個以上含む形になっている.(逆行列の一意性より$A\neq B$ならば$\Gamma^A\Gamma_B\neq \bm{1}$であり,定義より$\Gamma^A$も$\Gamma_B$もガンマ行列の積から作られているから,$\Gamma^A\Gamma_B$も必ずガンマ行列を1個以上含んでいる.)したがってこの積$\Gamma^A\Gamma_B$は因子$\pm 1, \pm i$を除いて,ある$\Gamma^C$に等しい.よって
\begin{align*}
\mathrm{tr}\left[\Gamma^A\Gamma_B\right]\propto \mathrm{tr}\left[\Gamma^c\right]=0 \\
\therefore\quad \mathrm{tr}\left[\Gamma^A\Gamma_B\right]=0 \quad (A\neq B)
\end{align*}
が得られる.以上より
\begin{align*}
\mathrm{tr}\left[\Gamma^A\Gamma_B\right]=4\delta^A_B
\end{align*}
とまとめられる.\par
以上の話から,$\Gamma^A$の線形独立性
\begin{align*}
\sum_{A=1}^{16} c_A \Gamma^A=0 \quad \Rightarrow \forall A, c_A=0
\end{align*}
が示すことができる.実際仮定が満たされていると,この両辺に$\Gamma_B$をかけてトレースをとって
\begin{align*}
0=&\mathrm{tr}\left[\sum_{A=1}^{16} c_A \Gamma^A\Gamma_B\right]=\sum_{A=1}^{16} c_A \mathrm{tr}\left[\Gamma^A\Gamma_B\right]=\sum_{A=1}^{16} c_A 4\delta^A_B=4c_B \\
\therefore\quad c_B=&0
\end{align*}
となり,$B$は任意であるから,全ての$c_B$がゼロであることがわかる.これより,任意の$4\times 4$行列$M$は16個の$\Gamma$で一意的に展開できる.
\begin{align*}
M=&\sum_{A=1}^{16} a_A \Gamma^A \\
=&a_I\bm{1} +a_\mu \gamma^\mu +\frac{1}{2}a_{\mu\nu}\sigma^{\mu\nu}+a_{5\mu}i\gamma^\mu \gamma_5 +a_5 \gamma_5
\end{align*}
展開係数は
\begin{align*}
a_A=\frac{1}{4}\mathrm{tr} \left[\Gamma_A M\right]
\end{align*}
すなわち
\begin{align*}
a_I=\frac{1}{4}\mathrm{tr}M,\quad a_\mu=\frac{1}{4}\mathrm{tr}\left[\gamma_\mu M\right]
\end{align*}
などで得られる.さらに
\begin{align*}
\sigma^{\mu\nu}=&i\gamma^\mu \gamma^\nu \\
=&\frac{i}{2}\left(\gamma^\mu \gamma^\nu-\gamma^\nu \gamma^\mu\right)+\frac{i}{2}\left(\gamma^\mu \gamma^\nu-\gamma^\nu \gamma^\mu\right) \\
=&-2\mc{S}^{\mu\nu}+i\eta^{\mu\nu}\bm{1}
\end{align*}
であるから,$\bm{1},\gamma^\mu,\mc{S}^{\mu\nu},i\gamma^\mu \gamma_5 ,\gamma_5$でも一意的に展開することができる.
\begin{align*}
M=a_I\bm{1} +a_\mu \gamma^\mu +\frac{1}{2}a_{\mu\nu}\mc{S}^{\mu\nu}+a_{5\mu}i\gamma^\mu \gamma_5 +a_5 \gamma_5
\end{align*}
これが示したかったことである.

\vskip\baselineskip

次に,途中で述べたようにガンマ行列の公理$\{\gamma^\mu,\gamma^\nu\}=2\eta^{\mu\nu}$を満たす4つの行列の組$\{\gamma^\mu\}_{\mu=0,\cdots 3}$は全て相似変換で関係しており,さらに追加の条件$(\gamma^0)=-\gamma^0,(\gamma^i)^\dagger=+\gamma^i$を満たすもの同士は特にユニタリ同値であることを証明する.\par
$\Gamma^A$と$\Gamma^B$の積は(符号因子$\eta_{AB}=\pm 1,\pm i$を除いて)$\Gamma_C$になっている.これは
\begin{align*}
\Gamma^A \Gamma^B =&\eta_{AB}\Gamma^C \quad (\eta_{AB})^4=1 \\
\therefore \quad \Gamma^B =&\eta_{AB}\Gamma_A \Gamma^C
\end{align*}
と書ける.つまり,$\Gamma^A$と$\Gamma^C$の二つを指定したとき,$\Gamma^A \Gamma^B =\eta_{AB}\Gamma^C$を満たす$\Gamma^B$はただ一つに定まることを示している.(16個の行列集合のうち2つが同じ式を満たす,ということがない.)これより,16個の$\{\Gamma^A\}$にある$\Gamma^X$をかけてできる16個の行列$\{\Gamma^X \Gamma^A\}$は被りがなく全て異なり,$\{\Gamma^X \Gamma^A\}$は組として$\{\Gamma^A\}$と(符号を除き)同型である.\par
ところで,ある$4\times 4$行列$M$が4つの$\{\gamma^\mu\}$のいずれとも可換であるならば,$\Gamma^A$は$\gamma^\mu$の積であるから,16個の$\{\Gamma^A\}$と$M$は可換である.
\begin{align*}
M\Gamma^A=\Gamma^A M
\end{align*}
任意の$4\times 4$行列は$\Gamma^A$で展開できるのだったから,そのような$M$はあらゆる$4\times 4$行列と可換である.
\begin{align*}
MN=M\left[\sum_{A}c_A \Gamma^A\right] =\left[\sum_{A}c_A \Gamma^A\right]M=NM, \quad \forall N \in M(4,\mathbb{C})
\end{align*}
このような行列$M$は単位行列に比例する.(もしそうでなければ,$M$は$\bm{1}$以外の$\Gamma^A$を含む形で展開されるが,$\bm{1}$以外の$\Gamma^A$には必ず反可換な$\Gamma^B$が存在するのだったから,少なくともそのような$\Gamma^B$の存在が反例となり矛盾する.)\par
さて,以上で道具が揃ったので証明に入ろう.ガンマ行列の公理を満たす2つの行列の組$\{\gamma^\mu\},\{\gamma'^\mu\}$があるとする.それぞれから$\{\Gamma^A\},\{\Gamma'^A\}$を構成することができる.(添え字の順番は$1,\gamma^\mu ,\cdots$で共通させる.)すると,上記の性質は両方に対して成り立つ.(上での計算は,具体的な行列表示を用いずに純粋な代数の性質から行ったものだから,全て成り立つ.)そこで,任意の行列$K\in M(4,\mathbb{C})$に対して,行列$M$を
\begin{align*}
M:=\sum_{A=1}^{16} \Gamma'^A K \Gamma_A
\end{align*}
で定める.この$M$は性質
\begin{align*}
\Gamma'^X M\Gamma_X=M
\end{align*}
を満たす.実際
\begin{align*}
\Gamma'_A \Gamma_X =(\Gamma^A)^{-1} (\Gamma^X)^{-1} =(\Gamma^X \Gamma^A)^{-1}=\frac{1}{\eta_{AX}}(\Gamma^B)^{-1}=\frac{1}{\eta_{AX}}\Gamma_B
\end{align*}
より
\begin{align*}
\Gamma'^X M\Gamma_X=& \sum_{A=1}^{16} \Gamma'^X \Gamma'^A K \Gamma_A \Gamma_X \\
=&\sum_{B=1} \eta'_{XA} \Gamma'^B K \frac{1}{\eta_{XA}}\Gamma_B \\
=&\sum_{B=1}^{16} \eta_{XA} \Gamma'^B K \frac{1}{\eta_{XA}} \Gamma_B \\
=&\sum_{B=1}^{16} \Gamma'^B K \Gamma_B =M
\end{align*}
1行目から2行目にかけて,組
\begin{align*}
\{\Gamma^A\}_{A=1,\cdots,16},\quad \{\Gamma'^A\}_{A=1,\cdots ,16}
\end{align*}
と組
\begin{align*}
\{\Gamma^X \Gamma^A\}_{A=1,\cdots, 16}=\{\eta_{XA}\Gamma^B\}_{B=1,\cdots ,16},\quad \{\Gamma'^X\Gamma'^A\}_{A=1,\cdots ,16}=\{\eta'_{XA}\Gamma'^B\}_{B=1,\cdots ,16}
\end{align*}
が組として同型であるから,$A$についての和と$B$についての和が同一であることを用いた.2行目から3行目にかけて,$\Gamma^A$と$\Gamma'A$で添え字の順番が同じであるから積の構造が同じで,したがって符号が同じ$\eta_{XA}=\eta'_{XA}$であることを用いた.よって右から$\Gamma^X$をかけると
\begin{align*}
\Gamma'^X M=M \Gamma^X
\end{align*}
となる.この$M$が相似変換行列の候補となる.\par
$M=0$でないことを示そう.このために,適当に$K$をとれば$M\neq 0$となることを背理法を用いて示す.\par
任意の$K$に対して$M=0$であると仮定する.$K$を行列要素$I^{(\xi\eta)}$(16個の成分のうち$\xi$行$\eta$列目のみが1で,他が全てゼロの行列,例えば
\begin{align*}
I^{(23)}=\left(
\begin{matrix}
0 & 0 & 0 & 0 \\
0 & 0 & 1 & 0 \\
0 & 0 & 0 & 0 \\
0 & 0 & 0 & 0 
\end{matrix}
\right) ,\quad (I^{(23)})_{\alpha\beta}=\delta_{\alpha 2}\delta_{\beta 3}
\end{align*}
という行列)とおいて$M$を計算してやると,仮定より$M=0$なので
\begin{align*}
0=M_{\alpha\beta}=&\sum_{A=1}^{16} \sum_{\gamma,\delta}(\Gamma'^A)_{\alpha\gamma}(I^{(\xi\eta)})_{\gamma\delta} (\Gamma_A)_{\delta\beta} \\
=&\sum_{A=1}^{16} (\Gamma'^A)_{\alpha\xi} (\Gamma_A)_{\eta\beta} \quad (\xi=\gamma,\eta=\delta のみ生き残る)
\end{align*}
これは,行列の関係として,各$(\alpha,\xi)$を固定して両辺を$(\eta,\beta)$を成分とする行列とみてやると
\begin{align*}
0=\sum_{A=1}^{16}(\Gamma'^A)_{\alpha\xi} \Gamma_A
\end{align*}
と書ける.しかし,明らかに$\Gamma'_A$はゼロでない行列要素$(\Gamma'^A)_{\alpha\xi}\neq 0$を含み,したがってこれは$\Gamma_A$が独立でないことを示す.これは今まで示してきた定理と矛盾し,仮定が間違っていることが示せた.証明完了.つまり$M\neq 0$となるような$K$が存在する.\par
次に,$M$が逆行列をもつような$K$が存在することを示そう.つまり$M\neq 0$かつ$\mathrm{det}M=\neq 0$を満たすような$K$が存在することを示す.\par
今までの議論は$\{\Gamma^A\}$と$\{\Gamma'^A\}$の役割を入れ替えても同様に成り立つ.したがって任意の行列$\tilde{K}$に対して
\begin{align*}
\tilde{M}:=\sum_{A=1}^{16} \Gamma^A \tilde{K} \Gamma'_A
\end{align*}
で定めた行列$\tilde{M}$は
\begin{align*}
\Gamma^X \tilde{M}=\tilde{M}\Gamma'^X
\end{align*}
を満たす.左から$M$をかけて,$\Gamma'X M=M \Gamma^X$を用いると
\begin{align*}
\Gamma^X \tilde{M}M=&\tilde{M} \Gamma'^X M \\
=&\tilde{M}M \Gamma^X
\end{align*}
この関係は,どの$\Gamma^X(X=1,\cdots,16)$についてもなりたつから,上で行った議論により$\tilde{M}M$は単位行列$\bm{1}$に比例していなければならない.
\begin{align*}
\tilde{M}M=\kappa \bm{1}
\end{align*}
行列$\tilde{K}$を適当に選べば,やはり$\tilde{M}\neq0$とでき,さらに$\kappa \neq 0$となるように$K$を選べる.実際$M$の定義より
\begin{align*}
\kappa\bm{1}=\tilde{M}M=\sum_{A=1}^{16} (\tilde{M}\Gamma'^A)K\Gamma_A
\end{align*}
と書けて,$K$を行列要素にすれば再び線形独立性を用いて$\kappa\neq 0$が示せる.\par
したがって,$M$は逆行列$\kappa^{-1}\tilde{M}$を持つ.すなわち前に示した$\Gamma'^XM =M\Gamma^X$より
\begin{align*}
\Gamma'^X=M\Gamma^X M^{-1}
\end{align*}
特に(添え字の順が同じなので$\Gamma^A=\gamma^\mu$となる$A=2,3,4,5$を持ってきて)
\begin{align*}
\gamma'^\mu=M\gamma^\mu M^{-1}
\end{align*}
がなりたつ.すなわち,2組のガンマ行列は同値である.\par
さらに$\{\gamma^\mu\}$と$\{\gamma'^\mu\}$が共に,追加の条件$(\gamma^0)^\dagger=-\gamma^0,(\gamma^i)^\dagger=+\gamma^i$を満たすならば,$M$はユニタリ行列と選ぶことができる.両辺エルミートをとると
\begin{align*}
(\gamma'^\mu)^\dagger=&(M^{-1})^\dagger (\gamma^\mu)^\dagger M^\dagger \\
\therefore\quad \gamma'^\mu =&(M^{-1})^\dagger \gamma^\mu M^\dagger
\end{align*}
(ここで,$\gamma^\mu$は$\mu=1,2,3$でエルミートであり,$\mu=0$では反エルミート性からマイナスが出てくるが両辺で符号が打ち消しあうことを用いた.)$\gamma'^\mu=M\gamma^\mu M^{-1}$をもう一度用いて
\begin{align*}
\gamma^\mu=&M^{-1} \gamma'^\mu M \\
=&M^{-1} (M^{-1})^\dagger \gamma^\mu M^\dagger M
\end{align*}
すなわち
\begin{align*}
M^\dagger M \gamma^\mu =\gamma^\mu M^\dagger M
\end{align*}
最初の議論より,このような$M^\dagger M$は単位行列に比例する.
\begin{align*}
M^\dagger M =\lambda \bm{1}
\end{align*}
両辺の行列式をとると,$\mathrm{det}M\neq 0$であるように選んでいるのだったから
\begin{align*}
\mathrm{det}(M^\dagger M)=&(\mathrm{det}M)^* \mathrm{det}M \quad \because \mathrm{det}(M^\dagger)=(\mathrm{det}M)^*\\
=&|\mathrm{det}M|^2>0 \\
=\lambda\mathrm{det}\bm{1}=&\lambda
\end{align*}
よって$\lambda$は正の実数である.したがって平方根$\lambda^{1/2}$を定義することができ,$\lambda^{1/2}$を$M$に押し込めて$M'=\lambda^{-1/2}M$とすると
\begin{align*}
M'^\dagger M'=& \lambda^{-1} M^\dagger M=\lambda^{-1}\lambda \bm{1}=\bm{1} \\
\therefore \quad M'^\dagger=&M^{-1}
\end{align*}
したがって$M'$はユニタリ行列であり,相似変換$\gamma'^\mu=M\gamma^\mu M^{-1}$は
\begin{align*}
\gamma'^\mu=M\gamma^\mu M^{-1}=(\lambda^{-1/2}M)\gamma^\mu (\lambda^{-1/2}M)^{-1}=M'\gamma^\mu M'^{-1}
\end{align*}
とできる.これは$\{\gamma^\mu\}$と$\{\gamma'^\mu\}$がユニタリ同値であることを示している.証明終わり.\par
証明を追ってもらえたらわかると思うが,追加の条件$(\gamma^0)^\dagger=-\gamma^0,(\gamma^i)^\dagger=+\gamma^i$は$\{\gamma^\mu\}$と$\{\gamma'^\mu\}$の各$\mu$についてエルミート性・反エルミート性が一致していれば他の条件になっていてもかまわない.例えばWeinbergでは$\gamma^0$が反エルミートになるようにしている一方,Peskinなどでは$\gamma^i$の方が反エルミートになるように,つまり$(\gamma^0)^\dagger=+\gamma^0,(\gamma^i)^\dagger=-\gamma^i$と定義しているが,後者の条件を代わりに課しても,ユニタリ性を同様に証明することができる.


\vskip\baselineskip


具体的にディラック表示とワイル(カイラル)表示との間を見てみよう.ディラック表示は(Weinberg流のワイル表示に合わせて,$-i$倍して)
\begin{align*}
\gamma^0_{\mathrm{D}}=-i\left(
\begin{matrix}
\bm{1} & 0 \\
0 & -\bm{1}
\end{matrix}
\right) , \quad \gamma^i_{\mathrm{D}}=-i\left(
\begin{matrix}
0 & \sigma_i \\
-\sigma_i & 0
\end{matrix}
\right)
\end{align*}
と書かれる.これは実際にガンマ行列の公理$\{\gamma^\mu_{\mathrm{D}},\gamma^\nu_{\mathrm{D}}\}=2\eta^{\mu\nu}\bm{1}$と追加条件$(\gamma^0_{\mathrm{D}})^\dagger=-\gamma^0_{\mathrm{D}},(\gamma^i_{\mathrm{D}})^\dagger=+\gamma^i_{\mathrm{D}}$を満たす.したがってこれはワイル表示(5.4.17)
\begin{align*}
\gamma^0_{\mathrm{W}}=-i\left(
\begin{matrix}
0 & \bm{1} \\
\bm{1} & 0
\end{matrix}
\right),\quad \gamma^i_{\mathrm{W}}=-i\left(
\begin{matrix}
0 & \sigma^i \\
-\sigma^i & 0
\end{matrix}
\right)
\end{align*}
との間にユニタリ同値がある.実際,ユニタリ行列
\begin{align*}
U=&\frac{1}{\sqrt{2}}\left(
\begin{matrix}
\bm{1} & \bm{1} \\
-\bm{1} & \bm{1}
\end{matrix}
\right) ,\quad U^\dagger U=\frac{1}{2}\left(
\begin{matrix}
\bm{1} & -\bm{1} \\
\bm{1} & \bm{1}
\end{matrix}
\right)\left(
\begin{matrix}
\bm{1} & \bm{1} \\
-\bm{1} & \bm{1}
\end{matrix}
\right)=\left(
\begin{matrix}
\bm{1} & 0 \\
0 & \bm{1}
\end{matrix}
\right)=\bm{1}
\end{align*}
を用いて
\begin{align*}
\gamma^\mu_{\mathrm{D}}=&U \gamma_{\mathrm{W}}^\mu U^{-1}
\end{align*}
というユニタリな相似変換で結ばれている.計算してみると
\begin{align*}
U \gamma^0_{\mathrm{W}} U^{-1} =&-i\frac{1}{2}\left(
\begin{matrix}
\bm{1} & \bm{1} \\
-\bm{1} & \bm{1}
\end{matrix}
\right)\left(
\begin{matrix}
0 & \bm{1} \\
\bm{1} & 0
\end{matrix}
\right)\left(
\begin{matrix}
\bm{1} & -\bm{1} \\
\bm{1} & \bm{1}
\end{matrix}
\right) \\
=&-i\frac{1}{2}\left(
\begin{matrix}
\bm{1} & \bm{1} \\
\bm{1} & -\bm{1}
\end{matrix}
\right)\left(
\begin{matrix}
\bm{1} & -\bm{1} \\
\bm{1} & \bm{1}
\end{matrix}
\right) \\
=&-i\left(
\begin{matrix}
\bm{1} & 0 \\
0 & -\bm{1}
\end{matrix}
\right)=\gamma^0_{\mathrm{D}} \\
U \gamma^i_{\mathrm{W}} U^{-1}=&-i\frac{1}{2}\left(
\begin{matrix}
\bm{1} & \bm{1} \\
-\bm{1} & \bm{1}
\end{matrix}
\right)\left(
\begin{matrix}
0 & \sigma_i \\
-\sigma_i & 0
\end{matrix}
\right)\left(
\begin{matrix}
\bm{1} & -\bm{1} \\
\bm{1} & \bm{1}
\end{matrix}
\right) \\
=&-i\frac{1}{2}\left(
\begin{matrix}
-\sigma_i & \sigma_i \\
-\sigma_i & -\sigma_i
\end{matrix}
\right)\left(
\begin{matrix}
\bm{1} & -\bm{1} \\
\bm{1} & \bm{1}
\end{matrix}
\right) \\
=&-i\left(
\begin{matrix}
0 & \sigma_i \\
-\sigma_i & 0
\end{matrix}
\right)=\gamma^i_{\mathrm{D}}
\end{align*}
となっている.\par
また,他にもマヨラナ表示(これも従来のものに$-i$倍したもの)
\begin{align*}
\gamma^0_{\mathrm{M}}=-i&\left(
\begin{matrix}
0 & \sigma_2 \\
\sigma_2 & 0
\end{matrix}
\right) \\
\gamma^1_{\mathrm{M}}=-i&\left(
\begin{matrix}
i\sigma_3 & 0 \\
0 & i\sigma_3
\end{matrix}
\right) , \quad \gamma^2_{\mathrm{M}}=-i\left(
\begin{matrix}
0 & -\sigma_2 \\
\sigma_2 & 0
\end{matrix}
\right),\quad \gamma^3_{\mathrm{M}}=-i\left(
\begin{matrix}
i\sigma_1 & 0\\
0 & i\sigma_1
\end{matrix}
\right)
\end{align*}
がある.これは行列要素を詳しく見ると
\begin{align*}
\gamma^0_{\mathrm{M}}=-i&\left(
\begin{matrix}
0 & 0 & 0 & -i  \\
0 & 0 & i & 0  \\
0 & -i & 0 & 0  \\
i & 0 & 0 & 0  \\
\end{matrix}
\right) \\
\gamma^1_{\mathrm{M}}=-i&\left(
\begin{matrix}
i & 0 & 0 & 0  \\
0 & -i & 0 & 0  \\
0 & 0 & i & 0  \\
0 & 0 & 0 & -i  \\
\end{matrix}
\right) , \quad \gamma^2_{\mathrm{M}}=-i\left(
\begin{matrix}
0 & 0 & 0 & i  \\
0 & 0 & -i & 0  \\
0 & -i & 0 & 0  \\
i & 0 & 0 & 0  \\
\end{matrix}
\right),\quad \gamma^3_{\mathrm{M}}=-i\left(
\begin{matrix}
0 & i & 0 & 0  \\
i & 0 & 0 & 0  \\
0 & 0 & 0 & i  \\
0 & 0 & i & 0  \\
\end{matrix}
\right)
\end{align*}
となっていて,全て純虚数(の$-i$倍になっているので実数)になっている.これを用いたディラック方程式を解くと,解であるスピノルが必ず実数になり,したがってマヨラナスピノルが解になるという性質がある.マヨラナ表示のガンマ行列も公理$\{\gamma^\mu_{\mathrm{M}},\gamma^\nu_{\mathrm{M}}\}=2\eta^{\mu\nu}\bm{1}$を満たし,さらに$(\gamma^0_{\mathrm{M}})^\dagger=-\gamma^0_{\mathrm{M}},(\gamma^i_{\mathrm{M}})^\dagger=+\gamma^i_{\mathrm{M}}$を満たすので,ワイル表示とユニタリ同値になっている.実際,ユニタリ行列
\begin{align*}
U=&\frac{1}{2}\left(
\begin{matrix}
-1 & -i & -1 & i  \\
-i & 1 & i & 1  \\
1 & -i & -1 & -i  \\
-i & -1 & -i & 1  \\
\end{matrix}
\right) \\
U^\dagger U=&\frac{1}{4}\left(
\begin{matrix}
-1 & i & 1 & i  \\
i & 1 & i & -1  \\
-1 & -i & -1 & i  \\
-i & 1 & i & 1  \\
\end{matrix}
\right)\left(
\begin{matrix}
-1 & -i & -1 & i  \\
-i & 1 & i & 1  \\
1 & -i & -1 & -i  \\
-i & -1 & -i & 1  \\
\end{matrix}
\right)=\left(
\begin{matrix}
1 & 0 & 0 & 0  \\
0 & 1 & 0 & 0  \\
0 & 0 & 1 & 0  \\
0 & 0 & 0 & 1  \\
\end{matrix}
\right)
\end{align*}
を用いると
\begin{align*}
U \gamma^0_{\mathrm{W}} U^{-1}=&-i\frac{1}{4}\left(
\begin{matrix}
-1 & -i & -1 & i  \\
-i & 1 & i & 1  \\
1 & -i & -1 & -i  \\
-i & -1 & -i & 1  \\
\end{matrix}
\right)\left(
\begin{matrix}
0 & 0 & 1 & 0 \\
0 & 0 & 0 & 1 \\
1 & 0 & 0 & 0 \\
0 & 1 & 0 & 0
\end{matrix}
\right)\left(
\begin{matrix}
-1 & i & 1 & i  \\
i & 1 & i & -1  \\
-1 & -i & -1 & i  \\
-i & 1 & i & 1  \\
\end{matrix}
\right) \\
=&-i\frac{1}{4}\left(
\begin{matrix}
-1 & i & -1 & -i \\
i & 1 & -i & 1 \\
-1 & -i & 1 & -i \\
-i & 1 & -i & -1
\end{matrix}
\right)\left(
\begin{matrix}
-1 & i & 1 & i  \\
i & 1 & i & -1  \\
-1 & -i & -1 & i  \\
-i & 1 & i & 1  \\
\end{matrix}
\right) \\
=&-i\left(
\begin{matrix}
0 & 0 & 0 & -i \\
0 & 0 & i & 0 \\
0 & -i & 0 & 0 \\
i & 1 & 0 & 0
\end{matrix}
\right)=\gamma^0_{\mathrm{M}} \\
U \gamma^1_{\mathrm{W}} U^{-1}=&-i\frac{1}{4}\left(
\begin{matrix}
-1 & -i & -1 & i  \\
-i & 1 & i & 1  \\
1 & -i & -1 & -i  \\
-i & -1 & -i & 1  \\
\end{matrix}
\right)\left(
\begin{matrix}
0 & 0 & 0 & 1 \\
0 & 0 & 1 & 0 \\
0 & -1 & 0 & 0 \\
-1 & 0 & 0 & 0
\end{matrix}
\right)\left(
\begin{matrix}
-1 & i & 1 & i  \\
i & 1 & i & -1  \\
-1 & -i & -1 & i  \\
-i & 1 & i & 1  \\
\end{matrix}
\right) \\
=&\left(
\begin{matrix}
-i & 1 & -i & -1 \\
-1 & -i & 1 & -i \\
i & 1 & -i & 1 \\
-1 & i & -1 & -i
\end{matrix}
\right)\left(
\begin{matrix}
-1 & i & 1 & i  \\
i & 1 & i & -1  \\
-1 & -i & -1 & i  \\
-i & 1 & i & 1  \\
\end{matrix}
\right) \\
=&\left(
\begin{matrix}
i & 0 & 0 & 0 \\
0 & -i & 0 & 0 \\
0 & 0 & i & 0 \\
0 & 0 & 0 & -i
\end{matrix}
\right)=\gamma^1_{\mathrm{M}} \\
U \gamma^2_{\mathrm{W}} U^{-1}=&-i\frac{1}{4}\left(
\begin{matrix}
-1 & -i & -1 & i  \\
-i & 1 & i & 1  \\
1 & -i & -1 & -i  \\
-i & -1 & -i & 1  \\
\end{matrix}
\right)\left(
\begin{matrix}
0 & 0 & 0 & -i \\
0 & 0 & +i & 0 \\
0 & +i & 0 & 0 \\
-i & 0 & 0 & 0
\end{matrix}
\right)\left(
\begin{matrix}
-1 & i & 1 & i  \\
i & 1 & i & -1  \\
-1 & -i & -1 & i  \\
-i & 1 & i & 1  \\
\end{matrix}
\right) \\
=&-i\frac{1}{4}\left(
\begin{matrix}
1 & -i & 1 & i \\
-i & -1 & i & -1 \\
-1 & -i & 1 & -i \\
-i & 1 & -i & -1
\end{matrix}
\right)\left(
\begin{matrix}
-1 & i & 1 & i  \\
i & 1 & i & -1  \\
-1 & -i & -1 & i  \\
-i & 1 & i & 1  \\
\end{matrix}
\right) \\
=&=-i\left(
\begin{matrix}
0 & 0 & 0 & i  \\
0 & 0 & -i & 0  \\
0 & -i & 0 & 0  \\
i & 0 & 0 & 0  \\
\end{matrix}
\right)=\gamma^2_{\mathrm{M}} \\
U \gamma^3_{\mathrm{W}} U^{-1}=&-i\frac{1}{4}\left(
\begin{matrix}
-1 & -i & -1 & i  \\
-i & 1 & i & 1  \\
1 & -i & -1 & -i  \\
-i & -1 & -i & 1  \\
\end{matrix}
\right)\left(
\begin{matrix}
0 & 0 & 1 & 0 \\
0 & 0 & 0 & -1 \\
-1 & 0 & 0 & 0 \\
0 & 1 & 0 & 0
\end{matrix}
\right)\left(
\begin{matrix}
-1 & i & 1 & i  \\
i & 1 & i & -1  \\
-1 & -i & -1 & i  \\
-i & 1 & i & 1  \\
\end{matrix}
\right) \\
=&-i\frac{1}{4}\left(
\begin{matrix}
1 & i & -1 & i \\
-i & 1 & -i & -1 \\
1 & -i & 1 & i \\
i & 1 & -i & 1
\end{matrix}
\right)\left(
\begin{matrix}
-1 & i & 1 & i  \\
i & 1 & i & -1  \\
-1 & -i & -1 & i  \\
-i & 1 & i & 1  \\
\end{matrix}
\right) \\
=&=-i\left(
\begin{matrix}
0 & i & 0 & 0  \\
i & 0 & 0 & 0  \\
0 & 0 & 0 & i  \\
0 & 0 & i & 0  \\
\end{matrix}
\right)=\gamma^3_{\mathrm{M}}
\end{align*}
となって,実際にユニタリ同値である.ここでワイル表示を陽に書いたもの
\begin{align*}
\gamma^0_{\mathrm{W}}=&-i\left(
\begin{matrix}
0 & 0 & 1 & 0 \\
0 & 0 & 0 & 1 \\
1 & 0 & 0 & 0 \\
0 & 1 & 0 & 0
\end{matrix}
\right)\\
\gamma^1_{\mathrm{W}}=&-i\left(
\begin{matrix}
0 & 0 & 0 & 1 \\
0 & 0 & 1 & 0 \\
0 & -1 & 0 & 0 \\
-1 & 0 & 0 & 0
\end{matrix}
\right) ,\quad \gamma^2_{\mathrm{W}}=-i\left(
\begin{matrix}
0 & 0 & 0 & -i \\
0 & 0 & +i & 0 \\
0 & +i & 0 & 0 \\
-i & 0 & 0 & 0
\end{matrix}
\right),\quad \gamma^3_{\mathrm{W}}=-i\left(
\begin{matrix}
0 & 0 & 1 & 0 \\
0 & 0 & 0 & -1 \\
-1 & 0 & 0 & 0 \\
0 & 1 & 0 & 0
\end{matrix}
\right)
\end{align*}
を用いた.


\newpage


\subsection{因果律を満たすディラック場}
さて,ローレンツ群のスピノル表現に従って変換する,粒子を消滅させ反粒子を生成する場を構成しよう.一般にこれらは(5.1.17)と(5.1.18)で与えられる形
\begin{align*}
\psi^+_\ell(x)=&\frac{1}{(2\pi)^{3/2}}\sum_{\sigma}\int d^3\mathbf{p} u_\ell(\mathbf{p},\sigma) e^{ip\cdot x}a(\mathbf{p},\sigma) \\
\psi^-_\ell(x)=&\frac{1}{(2\pi)^{3/2}}\sum_{\sigma}\int d^3\mathbf{p} v_\ell(\mathbf{p},\sigma) e^{-ip\cdot x}a^{c\dagger}(\mathbf{p},\sigma)
\end{align*}
をとる.(今回は最初から反粒子を導入している.)ただし粒子の種類を表す添え字$n$をやはり省略した.これらの式に現れる係数関数$u_\ell(\mathbf{p},\sigma),v_\ell(\mathbf{p},\sigma)$を計算するためには,まず(5.1.25)(5.1.26)を用いてゼロ運動量での$u_\ell(0,\sigma),v_\ell(0,\sigma)$を見つけ,(5.1.21)(5.1.22)を適用して任意の運動量での$u_\ell(\mathbf{p},\sigma),v_\ell(\mathbf{p},\sigma)$を計算しなければならない.ただし両式の$D_{\bar{\ell}\ell}(\Lambda)$として,前節で議論した斉次ローレンツ群の$4\times 4$スピノル表現
\begin{align*}
D_{\bar{\ell}\ell}(\Lambda)=\exp\left(+\frac{i}{2}\omega_{\mu\nu}\mc{S}^{\mu\nu}\right)_{\bar{\ell}\ell}
\end{align*}
をとる.つまり添え字$\ell$は4成分をとる.

\vskip\baselineskip


(5.4.19)を使うと,前節で計算しているように
\begin{align*}
\mc{J}_i:=&\frac{1}{2}\epsilon_{ijk}\mc{S}^{ij}=\left(
\begin{matrix}
\frac{1}{2}\sigma_i & 0 \\
0 & \frac{1}{2}\sigma_i
\end{matrix}
\right)
\end{align*}
であるから,ゼロ運動量の条件(5.1.25)(5.1.26)は,まず(5.1.25)を見てみると
\begin{align*}
\sum_{\bar{\sigma}}u_{\bar{\ell}}(0,\bar{\sigma})\mathbf{J}^{(j)}_{\bar{\sigma}\sigma}=&\sum_{\ell}\bm{\mc{J}}_{\bar{\ell}\ell}u_{\ell}(0,\sigma) \\
\left(
\begin{matrix}
\sum_{\bar{\sigma}}u_{+}(0,\bar{\sigma})\mathbf{J}^{(j)}_{\bar{\sigma}\sigma} \\
\sum_{\bar{\sigma}}u_{-}(0,\bar{\sigma})\mathbf{J}^{(j)}_{\bar{\sigma}\sigma}
\end{matrix}
\right)
=&\left(
\begin{matrix}
\frac{1}{2}\bm{\sigma} & 0 \\
0 & \frac{1}{2}\bm{\sigma}
\end{matrix}
\right)\left(
\begin{matrix}
u_{+}(0,\sigma) \\
u_{-}(0,\sigma)
\end{matrix}
\right) \\
=&\left(
\begin{matrix}
\frac{1}{2}\bm{\sigma} u_{+}(0,\sigma) \\
\frac{1}{2}\bm{\sigma} u_{-}(0,\sigma)
\end{matrix}
\right) \\
\therefore\quad \sum_{\bar{\sigma}}u_{\bar{m}\pm }(0,\bar{\sigma})\mathbf{J}^{(j)}_{\bar{\sigma}\sigma}=&\sum_m\frac{1}{2}\bm{\sigma}_{\bar{m}m} u_{m\pm}(0,\sigma)
\end{align*}
ここで,4成分の添え字$\ell$を1対の添え字$(m,\pm)$で置き換えた.$m$は二つの値をとる添え字で,(5.4.19)の部分行列の行と列を表し,もう一つの添え字$\pm$は(5.4.19)の上行列と下行列を区別する.つまり
\begin{align*}
u_\ell=\left(
\begin{matrix}
u_1 \\
u_2 \\
u_3 \\
u_4
\end{matrix}
\right)=\left(
\begin{matrix}
\left[
\begin{matrix}
u_{1+} \\
u_{2+}
\end{matrix}
\right] \\
\left[
\begin{matrix}
u_{1-} \\
u_{2-}
\end{matrix}
\right]
\end{matrix}
\right)
\end{align*}
となっている.同様に(5.1.26)も見てみると
\begin{align*}
v_\ell=\left(
\begin{matrix}
v_1 \\
v_2 \\
v_3 \\
v_4
\end{matrix}
\right)=\left(
\begin{matrix}
\left[
\begin{matrix}
v_{1+} \\
v_{2+}
\end{matrix}
\right] \\
\left[
\begin{matrix}
v_{1-} \\
v_{2-}
\end{matrix}
\right]
\end{matrix}
\right)
\end{align*}
と分解し,
\begin{align*}
-\sum_{\bar{\sigma}}v_{\bar{\ell}}(0,\bar{\sigma})\mathbf{J}^{(j)*}_{\bar{\sigma}\sigma}=&\sum_{\ell}\bm{\mc{J}}_{\bar{\ell}\ell}u_{\ell}(0,\sigma) \\
-\left(
\begin{matrix}
\sum_{\bar{\sigma}}v_{+}(0,\bar{\sigma})\mathbf{J}^{(j)*}_{\bar{\sigma}\sigma} \\
\sum_{\bar{\sigma}}v_{-}(0,\bar{\sigma})\mathbf{J}^{(j)*}_{\bar{\sigma}\sigma}
\end{matrix}
\right)
=&\left(
\begin{matrix}
\frac{1}{2}\bm{\sigma} & 0 \\
0 & \frac{1}{2}\bm{\sigma}
\end{matrix}
\right)\left(
\begin{matrix}
v_{+}(0,\sigma) \\
v_{-}(0,\sigma)
\end{matrix}
\right) \\
=&\left(
\begin{matrix}
\frac{1}{2}\bm{\sigma} v_{+}(0,\sigma) \\
\frac{1}{2}\bm{\sigma} v_{-}(0,\sigma)
\end{matrix}
\right) \\
\therefore\quad -\sum_{\bar{\sigma}}v_{\bar{m}\pm }(0,\bar{\sigma})\mathbf{J}^{(j)*}_{\bar{\sigma}\sigma}=&\sum_m\frac{1}{2}\bm{\sigma}_{\bar{m}m} v_{m\pm}(0,\sigma)
\end{align*}
となる.これを言い換えれば,$u_{m\pm}(0,\sigma)$と$v_{m\pm}(0,\sigma)$を,行列$U_{\pm},V_{\pm}$の$m,\sigma$要素だと見なすことによって
\begin{align*}
(U_{\pm})_{m\sigma}:=&u_{m\pm}(0,\sigma),\quad (V_{\pm})_{m\sigma}:=v_{m\pm}(0,\sigma) \\
U_\pm =&\left(
\begin{matrix}
u_{1\pm}(0,j) & u_{1\pm}(0,j-1) &\cdots & u_{1\pm}(0,-j+1) & u_{1\pm}(0,-j) \\
u_{2\pm}(0,1/2) & u_{2\pm}(0,-1/2) &\cdots & u_{2\pm}(0,-j+1) & u_{2\pm}(0,-j)
\end{matrix}
\right) \\
V_\pm =&\left(
\begin{matrix}
v_{1\pm}(0,j) & v_{1\pm}(0,j-1) & \cdots & v_{1\pm}(0,-j+1) & v_{1\pm}(0,-j) \\
v_{2\pm}(0,j) & v_{2\pm}(0,j-1) & \cdots & v_{2\pm}(0,-j+1) & v_{2\pm}(0,-j)
\end{matrix}
\right)
\end{align*}
とすると,上の関係式は
\begin{align*}
U_\pm \mathbf{J}^{(j)}=&\frac{1}{2}\bm{\sigma} U_\pm \\
-V_{\pm} \mathbf{J}^{(j)*}=&\frac{1}{2}\bm{\sigma} V_\pm
\end{align*}
と書き換えられる.$\mathbf{J}^{(j)}$と$-\mathbf{J}^{(j)*}$はともに回転群の$\mathfrak{so}(3)$代数の$(2j+1)$次元既約表現を与える.しかも$2\times 2$行列$\frac{1}{2}\bm{\sigma}$も回転群の$\mathfrak{so}(3)$代数の既約表現を与える.5.1節で一般的に述べたように,Schurの補題を用いると,$U_\pm$や$V_\pm$が(5.5.3)(5.5.4)のように二つの\uwave{既約}表現を結び付けているとき,その既約表現が同値でないならばその行列$U_\pm,V_\pm$はゼロになり(つまりここでは興味がない可能性),同値ならば$U_\pm,V_\pm$はスケールを除いて一意的に決まり,それは正方行列かつ正則行列(つまり同じ次元のベクトル空間同士の同型写像の表現行列)でなければならない.これにより$\mathbf{J}^{(j)}$と$-\mathbf{J}^{(j)*}$は2次元表現の行列でなければならず,よってディラック場はスピン$j=\frac{1}{2}$(よって$2j+1=2$次元表現)の粒子しか記述できない!したがって$\sigma$は$+1/2$と$-1/2$のみをとり,行列$U_{\pm},V_\pm$は$2\times 2$行列
\begin{align*}
U_\pm =&\left(
\begin{matrix}
u_{1\pm}(0,1/2) & u_{1\pm}(0,-1/2)\\
u_{2\pm}(0,1/2) & u_{2\pm}(0,-1/2)
\end{matrix}
\right) \\
V_\pm =&\left(
\begin{matrix}
v_{1\pm}(0,1/2) & v_{1\pm}(0,-1/2)\\
v_{2\pm}(0,1/2) & v_{2\pm}(0,-1/2)
\end{matrix}
\right)
\end{align*}
となる.そして行列$\mathbf{J}^{(1/2)}$と$-\mathbf{J}^{(1/2)*}$は相似変換を除いて行列$\frac{1}{2}\bm{\sigma}$と同じでなければならない.
\begin{align*}
U_\pm \mathbf{J}^{(j)}U_\pm^{-1}=&\frac{1}{2}\bm{\sigma} \\
-V_{\pm} \mathbf{J}^{(j)*}V_{\pm}^{-1}=&\frac{1}{2}\bm{\sigma}
\end{align*}
実際,回転の生成子の標準的な表現(2.5.21)(2.5.22)を見れば,(量子力学で回転群の二次元表現でやったと思うけど)
\begin{align*}
J_1^{(1/2)}+iJ_2^{(1/2)}=&\left(
\begin{matrix}
0 & \sqrt{(\frac{1}{2}+\frac{1}{2})(\frac{1}{2}-\frac{1}{2}+1)} \\
0 & 0
\end{matrix}
\right)=\left(
\begin{matrix}
0 & 1 \\
0 & 0
\end{matrix}
\right) \\
J_1^{(1/2)}-iJ_2^{(1/2)}=&\left(
\begin{matrix}
0 & 0 \\
\sqrt{(\frac{1}{2}+\frac{1}{2})(\frac{1}{2}-\frac{1}{2}+1)} & 0
\end{matrix}
\right)=\left(
\begin{matrix}
0 & 0 \\
1 & 0
\end{matrix}
\right)\\
J_3^{(1/2)}=&\left(
\begin{matrix}
1/2 & 0 \\
0 & -1/2
\end{matrix}
\right) \\
\therefore\quad J^{(1/2)}_1=&\frac{1}{2}\left(
\begin{matrix}
0 & 1 \\
1 & 0
\end{matrix}
\right)=\frac{1}{2}\sigma_1 ,\quad J^{(1/2)}_2=\frac{1}{2}\left(
\begin{matrix}
0 & -i \\
i & 0
\end{matrix}
\right)=\frac{1}{2}\sigma_2 ,\quad J^{(1/2)}_{3}=\frac{1}{2}\left(
\begin{matrix}
1 & 0 \\
0 & -1
\end{matrix}
\right)=\frac{1}{2}\sigma_3 \\
\mathbf{J}^{(1/2)}=&\frac{1}{2}\bm{\sigma}
\end{align*}
となっている.また
\begin{align*}
-\mathbf{J}^{(1/2)*}=-\frac{1}{2}\bm{\sigma}^*=+\frac{1}{2}\sigma_2 \bm{\sigma}\sigma_2 \quad \because \sigma_2\bm{\sigma}^*\sigma_2=-\bm{\sigma}
\end{align*}
したがって(5.5.3)(5.5.4)は
\begin{align*}
U_{\pm}\frac{1}{2}\bm{\sigma} =&\frac{1}{2}\bm{\sigma} U_{\pm} \\
V_{\pm}\sigma_2 \bm{\sigma} \sigma_2=&\frac{1}{2}\bm{\sigma}V_\pm ,\quad \therefore \quad (V_{\pm}\sigma_2) \bm{\sigma}=\frac{1}{2}\bm{\sigma}(V_\pm \sigma_2)
\end{align*}
すると$U_{\pm}$と$V_\pm \sigma_2$はパウリ行列$\bm{\sigma}$と交換しなければならない$2\times 2$行列であることがわかる.そのようなものは単位行列以外にない\footnote{Schurの補題より,どちらも回転群の同一の既約表現であることから言える.もちろん純粋に代数的にも言える.2.7節で示したように任意の$2\times 2$行列は単位行列とパウリ行列で一意的に展開
\begin{align*}
A=\alpha \bm{1}+\beta_i \sigma_i 
\end{align*}
できるため,この行列とパウリ行列の交換子をとってみると
\begin{align*}
[A,\sigma_j]=\beta_i[\sigma_i,\sigma_j]=2i\beta_i \epsilon_{ijk}\sigma_k=2i(\bm{\beta}\times \mathbf{e}_j)\cdot \bm{\sigma}
\end{align*}
これが全ての$j=1,2,3$で交換$[A,\sigma_j]=0$するためには,$\bm{\beta}\times \mathbf{e}_j=0$がなりたたなければならない.($\mathbf{w}\cdot \bm{\sigma}=0$ならば$(\mathbf{w}\cdot \bm{\sigma})^2=|\mathbf{w}|^2\bm{1}=0$より$\mathbf{w}=0$である.)これは$\bm{\beta}=0$でなければならない.}.したがって
\begin{align*}
(U_{\pm})_{m\sigma}=&c_{\pm}\delta_{m\sigma} ,\quad (V_\pm \sigma_2)_{m\sigma}=-id_\pm \delta_{m\sigma} \\
\therefore\quad u_{m\pm}(0,\sigma)=&c_\pm \delta_{m\sigma} ,\quad v_{m\pm}(0,\sigma)=-id_\pm (\sigma_2)_{m\sigma}
\end{align*}
となる.(ここで$v$の虚数$-i$は$\sigma_2$に現れる虚数をキャンセルするために入れた.定数は今のところ任意なのでこれで良い.)言い換えれば,
\begin{align*}
U_+ =&\left(
\begin{matrix}
u_{1+}(0,1/2) & u_{1+}(0,-1/2)\\
u_{2+}(0,1/2) & u_{2+}(0,-1/2)
\end{matrix}
\right)=\left(
\begin{matrix}
c_+ & 0 \\
0 & c_+
\end{matrix}
\right) \\
U_- =&\left(
\begin{matrix}
u_{1-}(0,1/2) & u_{1-}(0,-1/2)\\
u_{2-}(0,1/2) & u_{2-}(0,-1/2)
\end{matrix}
\right)=\left(
\begin{matrix}
c_- & 0 \\
0 & c_-
\end{matrix}
\right)
\end{align*}
より
\begin{align*}
u\left(0,\frac{1}{2}\right)=\left(
\begin{matrix}
\left[
\begin{matrix}
u_{1+}(0,1/2) \\
u_{2+}(0,1/2)
\end{matrix}
\right] \\
\left[
\begin{matrix}
u_{1-}(0,1/2) \\
u_{2-}(0,1/2)
\end{matrix}
\right]
\end{matrix}
\right)=\left(
\begin{matrix}
c_+ \\
0 \\
c_- \\
0
\end{matrix}
\right),\quad u\left(0,-\frac{1}{2}\right)=\left(
\begin{matrix}
\left[
\begin{matrix}
u_{1+}(0,-1/2) \\
u_{2+}(0,-1/2)
\end{matrix}
\right] \\
\left[
\begin{matrix}
u_{1-}(0,-1/2) \\
u_{2-}(0,-1/2)
\end{matrix}
\right]
\end{matrix}
\right)=\left(
\begin{matrix}
0 \\
c_+ \\
0 \\
c_-
\end{matrix}
\right)
\end{align*}
が得られ,$v(0,\sigma)$についても
\begin{align*}
V_+=&\left(
\begin{matrix}
v_{1+}(0,1/2) & v_{1+}(0,-1/2)\\
v_{2+}(0,1/2) & v_{2+}(0,-1/2)
\end{matrix}
\right)=\left(
\begin{matrix}
0 & -d_+ \\
d_+ & 0
\end{matrix}
\right) \\
V_-=&\left(
\begin{matrix}
v_{1-}(0,1/2) & v_{1-}(0,-1/2)\\
v_{2-}(0,1/2) & v_{2-}(0,-1/2)
\end{matrix}
\right)=\left(
\begin{matrix}
0 & -d_- \\
d_- & 0
\end{matrix}
\right)
\end{align*}
より
\begin{align*}
v\left(0,\frac{1}{2}\right)=\left(
\begin{matrix}
\left[
\begin{matrix}
v_{1+}(0,1/2) \\
v_{2+}(0,1/2)
\end{matrix}
\right] \\
\left[
\begin{matrix}
v_{1-}(0,1/2) \\
v_{2-}(0,1/2)
\end{matrix}
\right]
\end{matrix}
\right)=\left(
\begin{matrix}
0 \\
d_+ \\
0 \\
d_-
\end{matrix}
\right),\quad v\left(0,-\frac{1}{2}\right)=\left(
\begin{matrix}
\left[
\begin{matrix}
v_{1+}(0,-1/2) \\
v_{2+}(0,-1/2)
\end{matrix}
\right] \\
\left[
\begin{matrix}
v_{1-}(0,-1/2) \\
v_{2-}(0,-1/2)
\end{matrix}
\right]
\end{matrix}
\right)=-\left(
\begin{matrix}
d_+ \\
0 \\
d_- \\
0
\end{matrix}
\right)
\end{align*}
となる.また,有限運動量でのスピノルは(5.1.21)(5.1.22)より
\begin{align*}
u(\mathbf{p},\sigma)=&\sqrt{\frac{m}{p^0}}D\Bigl(L(p)\Bigr)u(0,\sigma) \\
v(\mathbf{p},\sigma)=&\sqrt{\frac{m}{p^0}}D\Bigl(L(p)\Bigr)v(0,\sigma)
\end{align*}
となる.これを少し詳しく見てみると
\begin{align*}
\cosh \theta:=\frac{\sqrt{\mathbf{p}^2+m^2}}{m} ,\quad \sinh \theta=\frac{|\mathbf{p}|}{m}
\end{align*}
とおくと(これはちゃんと双曲線関数の$\cosh^2\theta -\sinh^2\theta=1$を満たしている)2.5節で与えた基準ブースト$L(p)$は,
\begin{align*}
\tensor{L}{^i_k}(p)=&\delta_{ik}+(\gamma-1)\hat{p}_i \hat{p}_k=\delta_{ik}+(\cosh\theta-1)\hat{p}_i \hat{p}_k \\
\tensor{L}{^i_0}(p)=&\tensor{L}{^0_i}=\hat{p}_i \sqrt{\gamma^2-1}=\hat{p}_i\sinh \theta \\
\tensor{L}{^0_0}(p)=&\gamma=\cosh\theta
\end{align*}
と書きなおせる.このパラメータ$\theta$は加法的,すなわち
\begin{align*}
\tensor{[L(\bar{\theta})L(\theta)]}{^\mu_\rho}=&\tensor{L(\bar{\theta})}{^\mu_\nu}\tensor{L(\theta)}{^\nu_\rho}=\tensor{L(\bar{\theta})}{^\mu_0}\tensor{L(\theta)}{^0_\rho}+\tensor{L(\bar{\theta})}{^\mu_i}\tensor{L(\theta)}{^i_\rho} \\
\tensor{[L(\bar{\theta})L(\theta)]}{^0_0}=&\tensor{L(\bar{\theta})}{^0_0}\tensor{L(\theta)}{^0_0}+\tensor{L(\bar{\theta})}{^0_i}\tensor{L(\theta)}{^i_0} \\
=&\cosh\bar{\theta}\cosh\theta+\sinh\bar{\theta}\sinh\theta \\
=&\cosh(\bar{\theta}+\theta)=\tensor{L(\bar{\theta}+\theta)}{^0_0} \quad \because 加法定理 \\
\tensor{[L(\bar{\theta})L(\theta)]}{^i_k}=&\tensor{L(\bar{\theta})}{^i_0}\tensor{L(\theta)}{^0_k}+\tensor{L(\bar{\theta})}{^i_j}\tensor{L(\theta)}{^j_k} \\
=&\hat{p}_i\hat{p}_k\sinh\bar{\theta}\sinh\theta+\Bigl\{\delta_{ij}+(\cosh\bar{\theta}-1)\hat{p}_i\hat{p}_j\Bigr\}\Bigl\{\delta_{jk}+(\cosh\theta-1)\hat{p}_j\hat{p}_k\Bigr\} \\
=&\delta_{ik}+\Bigl[\sinh\bar{\theta}\sinh\theta+(\cosh\bar{\theta}+\cosh\theta-2)+(\cosh\bar{\theta}-1)(\cosh\theta-1)\Bigr]\hat{p}_i\hat{p}_k \\
=&\delta_{ik}+\Bigl[\sinh\bar{\theta}\sinh\theta+\cosh\bar{\theta}\cosh\theta -1\Bigr]\hat{p}_i\hat{p}_k \\
=&\delta_{ik}+\Bigl[\cosh(\bar{\theta}+\theta) -1\Bigr]\hat{p}_i\hat{p}_k=\tensor{L(\bar{\theta}+\theta)}{^i_k} \quad \because 加法定理 \\
\tensor{[L(\bar{\theta})L(\theta)]}{^i_0}=&\tensor{L(\bar{\theta})}{^i_0}\tensor{L(\theta)}{^0_0}+\tensor{L(\bar{\theta})}{^i_j}\tensor{L(\theta)}{^j_0} \\
=&\hat{p}_i \sinh\bar{\theta}\cosh\theta+\Bigl\{\delta_{ij}+(\cosh\bar{\theta}-1)\hat{p}_i\hat{p}_j\Bigr\}\hat{p}_j \sinh\theta \\
=&\hat{p}_i \sinh\bar{\theta}\cosh\theta +\hat{p}_j \sinh\theta-\hat{p}_j \sinh\theta +\hat{p}_j \cosh\bar{\theta}\sinh\theta \\
=&\hat{p}_i [\sinh\bar{\theta}\cosh\theta+\cosh\bar{\theta}\sinh\theta] \\
=&\hat{p}_i \sinh(\bar{\theta}+\theta) =\tensor{L(\bar{\theta}+\theta)}{^i_0} \\
\therefore\quad L(\bar{\theta})L(\theta)=&L(\bar{\theta}+\theta)
\end{align*}
であることがわかる.微小な運動量$\mathbf{p}$について(つまり$\theta$が微小として)$\theta$の一次までで展開すると
\begin{align*}
\tensor{L}{^i_k}(p)=&\delta_{ik}+(\cosh\theta-1)\hat{p}_i \hat{p}_k \\
=&\delta_{ik}+\mc{O}(\theta^2) \\
\tensor{L}{^i_0}(p)=&\tensor{L}{^0_i}=\hat{p}_i\sinh \theta=\hat{p}_i\theta +\mc{O}(\theta^2) \\
\tensor{L}{^0_0}(p)=&\cosh\theta=1+\mc{O}(\theta^2)
\end{align*}
とできる.ここで双曲線関数のマクローリン展開
\begin{align*}
\cosh\theta=1+\frac{1}{2!}\theta^2+\cdots \\
\sinh\theta=\theta+\frac{1}{3!}\theta^3+\cdots
\end{align*}
を用いた.これは,$\tensor{L(\mathbf{p})}{^\mu_\nu}=\delta^\mu_\nu+\tensor{\omega}{^\mu_\nu}$と展開したときのパラメータ$\omega$が
\begin{align*}
\tensor{\omega}{^i_j}=0,\quad \tensor{\omega}{^0_i}=\tensor{\omega}{^i_0}=\hat{p}_i\theta, \quad \tensor{\omega}{^0_0}=0
\end{align*}
であることを示している.よって$\omega_{\mu\nu}=\eta_{\mu\rho}\tensor{\omega}{^\rho_\nu}$は
\begin{align*}
\omega_{i0}=-\omega_{0i}=\tensor{\omega}{^i_0}=\hat{p}_i\theta,\omega_{00}=\omega_{ij}=0
\end{align*}
であり,これを用いるとローレンツ群の表現$D(L(p))$は,$\theta \ll 1$において
\begin{align*}
D(L(\theta))=&1+\frac{i}{2}\omega_{\mu\nu}\mc{S}^{\mu\nu} \\
=&1+i\omega_{i0}\mc{S}^{i0} \\
=&1-i\hat{p}_i\theta \mc{K}_i \quad \because \mc{K}_i=\mc{S}_{i0}
\end{align*}
となる.$D(L(\bar{\theta}))D(L(\theta))=D(L(\bar{\theta})L(\theta))=D(L(\bar{\theta}+\theta))$より,2.2節の(2.2.26)を導いた議論を思い出すと,有限の大きさの$\theta$では
\begin{align*}
D\Bigl(L(\theta)\Bigr)=\exp\Bigl(-i\theta \hat{\mathbf{p}}\cdot \bm{\mc{K}}\Bigr) \qquad (\theta=\sinh^{-1}\left(|\mathbf{p}|/m\right))
\end{align*}
となる.

\vskip\baselineskip


今や,定数$c_\pm$と$d_\pm$についての説明だけが残っている.一般にはこれらは全く任意であり,望めば$c_-$と$d_-$,あるいは$c_+$と$d_+$をゼロに選んで,ディラック場がゼロでない成分を上2成分だけ,あるいは下2成分だけ持つようにすることさえできる.$c_\pm$または$d_\pm$の相対的値について何か言える唯一の物理的原理はパリティの保存である\footnote{例えば,標準模型のニュートリノはパリティを破る弱い相互作用に関連するディラック場であるから,上記で述べているような上2成分あるいは下2成分のみをとるようにできて,左巻きニュートリノ(上2成分のみ)が現れる.}.空間反転のもとで粒子消滅演算子と反粒子生成演算子は
\begin{align*}
\mathsf{P}a(\mathbf{p},\sigma)\mathsf{P}^{-1}=&\eta^* a(-\mathbf{p},\sigma) \\
\mathsf{P}a^{c\dagger}(\mathbf{p},\sigma)\mathsf{P}^{-1}=&\eta^c a^{c\dagger}(-\mathbf{p},\sigma)
\end{align*}
の変換を受けるのだった.よって\footnote{前と同様積分範囲の反転に注意}
\begin{align*}
\mathsf{P} \psi^+_\ell(x)\mathsf{P}^{-1}=&\sum_\sigma \int \frac{d^3\mathbf{p}}{(2\pi)^{3/2}} u_\ell(\mathbf{p},\sigma)e^{ip\cdot x}\mathsf{P} a(\mathbf{p},\sigma) \mathsf{P}^{-1} \\
=&\eta^* \sum_\sigma \int \frac{d^3\mathbf{p}}{(2\pi)^{3/2}} u_\ell(\mathbf{p},\sigma)e^{ip\cdot x}a(-\mathbf{p},\sigma) \\
=&\eta^* \sum_\sigma \int \frac{d^3\mathbf{p}}{(2\pi)^{3/2}} u_\ell(-\mathbf{p},\sigma)e^{ip\cdot (\mc{P}x)}a(\mathbf{p},\sigma)\qquad (\mathbf{p}\to -\mathbf{p}) \\
\mathsf{P} \psi^{-c}_\ell(x)\mathsf{P}^{-1}=&\sum_\sigma \int \frac{d^3\mathbf{p}}{(2\pi)^{3/2}} v_\ell(\mathbf{p},\sigma)e^{-ip\cdot x}\mathsf{P} a^{c\dagger}(\mathbf{p},\sigma) \mathsf{P}^{-1} \\
=&\eta^c \sum_\sigma \int \frac{d^3\mathbf{p}}{(2\pi)^{3/2}} v_\ell(\mathbf{p},\sigma)e^{-ip\cdot x}a^{c\dagger}(-\mathbf{p},\sigma) \\
=&\eta^c \sum_\sigma \int \frac{d^3\mathbf{p}}{(2\pi)^{3/2}} v_\ell(-\mathbf{p},\sigma)e^{-ip\cdot (\mc{P}x)}a^{c\dagger}(\mathbf{p},\sigma)\qquad (\mathbf{p}\to -\mathbf{p})
\end{align*}
となる.また(5.4.16)$\beta \mc{K}_i\beta =-\mc{K}_i$を用いると
\begin{align*}
D\Bigl(L(-\mathbf{p})\Bigr)=&\exp\Bigl(+i\theta \hat{\mathbf{p}}\cdot \bm{\mc{K}}\Bigr) \\
=&\beta \exp\Bigl(-i\theta \hat{\mathbf{p}}\cdot \bm{\mc{K}}\Bigr)\beta \\
=&\beta D\Bigl(L(p)\Bigr)\beta \quad \because \theta(-\mathbf{p})=\sinh^{-1}\left(|-\mathbf{p}|/m\right)=\sinh^{-1}\left(|\mathbf{p}|/m\right)=\theta(\mathbf{p})
\end{align*}
であるから,(5.1.21)(5.1.22)より
\begin{align*}
u(-\mathbf{p},\sigma)=&\sqrt{\frac{m}{p^0}}D\Bigl(L(-\mathbf{p})\Bigr) u(0,\sigma) \\
=&\sqrt{\frac{m}{p^0}}\beta D\Bigl(L(\mathbf{p})\Bigr) \beta u(0,\sigma) \\
v(-\mathbf{p},\sigma)=&\sqrt{\frac{m}{p^0}}D\Bigl(L(-\mathbf{p})\Bigr) v(0,\sigma) \\
=&\sqrt{\frac{m}{p^0}}\beta D\Bigl(L(\mathbf{p})\Bigr) \beta v(0,\sigma)
\end{align*}
となる.これが$u(\mathbf{p},\sigma),v(\mathbf{p},\sigma)$で書けるためには
\begin{align*}
\beta u(0,\sigma)=b_u u(0,\sigma),\quad \beta v(0,\sigma)=b_v v(0,\sigma)
\end{align*}
と,それぞれ$u(0,\sigma)$と$v(0,\sigma)$に比例していなければならない.このようにすれば,実際
\begin{align*}
u(-\mathbf{p},\sigma)=&\sqrt{\frac{m}{p^0}}\beta D\Bigl(L(\mathbf{p})\Bigr) \beta u(0,\sigma) \\
=&b_u \beta \sqrt{\frac{m}{p^0}} D\Bigl(L(\mathbf{p})\Bigr) u(0,\sigma) \\
=&b_u\beta u(\mathbf{p},\sigma) \\ 
v(-\mathbf{p},\sigma)=&\sqrt{\frac{m}{p^0}}\beta D\Bigl(L(\mathbf{p})\Bigr) \beta v(0,\sigma) \\
=&b_v \beta \sqrt{\frac{m}{p^0}} D\Bigl(L(\mathbf{p})\Bigr) v(0,\sigma) \\
=&b_v\beta v(\mathbf{p},\sigma)
\end{align*}
となって$u(\mathbf{p},\sigma),v(\mathbf{p},\sigma)$で書くことができる.また$\beta^2=\bm{1}$より
\begin{align*}
u(0,\sigma)=& \beta^2 u(0,\sigma) \\
=&b_u \beta u(0,\sigma) \\
=&b_u^2 u(0,\sigma) \\
v(0,\sigma)=&\beta^2 v(0,\sigma) \\
=&b_v \beta v(0,\sigma) \\
=&b_v^2 v(0,\sigma) \\
\therefore \quad & b_u^2=b_v^2=1
\end{align*}
となり,$b_u,b_v$は符号因子$\pm 1$でなければならない.$\beta$の具体的な行列表示(5.4.29)より条件(5.5.14)は
\begin{align*}
\beta u\left(0,\frac{1}{2}\right)=&b_u u\left(0,\frac{1}{2}\right) \\
\left(
\begin{matrix}
0 & 0 & 1 & 0 \\
0 & 0 & 0 & 1 \\
1 & 0 & 0 & 0 \\
0 & 1 & 0 & 0 \\
\end{matrix}
\right)\left(
\begin{matrix}
c_+ \\
0 \\
c_- \\
0
\end{matrix}
\right)=&\left(
\begin{matrix}
c_- \\
0 \\
c_+ \\
0
\end{matrix}
\right)=b_u \left(
\begin{matrix}
c_+ \\
0 \\
c_- \\
0
\end{matrix}
\right) \\
\therefore\quad c_-=&b_u c_+ \\
\beta v\left(0,\frac{1}{2}\right)=&b_v v\left(0,\frac{1}{2}\right) \\
\left(
\begin{matrix}
0 & 0 & 1 & 0 \\
0 & 0 & 0 & 1 \\
1 & 0 & 0 & 0 \\
0 & 1 & 0 & 0 \\
\end{matrix}
\right)\left(
\begin{matrix}
0 \\
d_+ \\
0 \\
d_-
\end{matrix}
\right)=&\left(
\begin{matrix}
0 \\
d_- \\
0 \\
d_+
\end{matrix}
\right)=b_v \left(
\begin{matrix}
0 \\
d_+ \\
0 \\
d_-
\end{matrix}
\right) \\
\therefore \quad d_-=&b_vd_+
\end{align*}
がわかる.(他の$u(0,-1/2)$などを用いた式も同様の条件しか与えない.$b_u^2=b_v^2=1$よりもう片方の条件$c_+=b_uc_-$などは自明になる.)したがって$u(0,\sigma),v(0,\sigma)$はそれぞれ
\begin{align*}
u\left(0,\frac{1}{2}\right)=&\left(
\begin{matrix}
c_+ \\
0 \\
c_- \\
0
\end{matrix}
\right)=c_+ \left(
\begin{matrix}
1 \\
0 \\
b_u \\
0
\end{matrix}
\right) ,\quad u\left(0,-\frac{1}{2}\right)=\left(
\begin{matrix}
0 \\
c_+ \\
0 \\
c_-
\end{matrix}
\right)=c_+\left(
\begin{matrix}
0 \\
1 \\
0 \\
b_u
\end{matrix}
\right) \\
v\left(0,\frac{1}{2}\right)=&\left(
\begin{matrix}
0 \\
d_+ \\
0 \\
d_-
\end{matrix}
\right)=d_+ \left(
\begin{matrix}
0 \\
1 \\
0 \\
b_v
\end{matrix}
\right) ,\quad u\left(0,-\frac{1}{2}\right)=-\left(
\begin{matrix}
d_+ \\
0 \\
d_- \\
0
\end{matrix}
\right)=-d_+\left(
\begin{matrix}
1 \\
0 \\
b_v \\
0
\end{matrix}
\right)
\end{align*}
となる.場の全体のスケールを再定義することで
\begin{align*}
u\left(0,\frac{1}{2}\right)=&\frac{1}{\sqrt{2}} \left(
\begin{matrix}
1 \\
0 \\
b_u \\
0
\end{matrix}
\right) ,\quad u\left(0,-\frac{1}{2}\right)=\frac{1}{\sqrt{2}}\left(
\begin{matrix}
0 \\
1 \\
0 \\
b_u
\end{matrix}
\right) \\
v\left(0,\frac{1}{2}\right)=&\frac{1}{\sqrt{2}} \left(
\begin{matrix}
0 \\
1 \\
0 \\
b_v
\end{matrix}
\right) ,\quad u\left(0,-\frac{1}{2}\right)=-\frac{1}{\sqrt{2}}\left(
\begin{matrix}
1 \\
0 \\
b_v \\
0
\end{matrix}
\right)
\end{align*}
とできる.

\vskip\baselineskip

さて,いつも通り消滅場と生成場を,空間的な距離で自分自身およびその共役場と交換または反交換する線形結合
\begin{align*}
\psi(x):=\kappa \psi^+(x)+\lambda \psi^{-c}(x)
\end{align*}
にまとめてみよう.このとき交換子あるいは反交換子は
\begin{align*}
[\psi_\ell(x),\psi^{\dagger}_{\bar{\ell}}(y)]_{\mp}=&\left[\kappa \psi_\ell^+(x)+\lambda \psi^{-c}_\ell(x),\kappa^* \psi_{\bar{\ell}}^{+\dagger}(y)+\lambda^* \psi^{-c\dagger}_{\bar{\ell}}(y)\right]_{\mp} \\
=&|\kappa|^2 [\psi_\ell^+(x),\psi_{\bar{\ell}}^{+\dagger}(y)]_{\mp}+\kappa \lambda^*[\psi_\ell^{+}(x),\psi^{-c\dagger}_{\bar{\ell}}(y)]_{\mp} \\
&+\lambda \kappa^*[\psi^{-c}_\ell(x),\psi_{\bar{\ell}}^{+\dagger}(y)]_{\mp}+|\lambda|^2[\psi^{-c}_\ell(x),\psi_{\bar{\ell}}^{-c\dagger}(y)]_{\mp} \\
=&|\kappa|^2 [\psi_\ell^+(x),\psi_{\bar{\ell}}^{+\dagger}(y)]_{\mp}+|\lambda|^2[\psi^{-c}_\ell(x),\psi_{\bar{\ell}}^{-c\dagger}(y)]_{\mp} \\
=&\int \frac{d^3\mathbf{p}}{(2\pi)^{3}} \left[|\kappa|^2N_{\ell\bar{\ell}}(\mathbf{p})e^{ip\cdot (x-y)} \mp |\lambda|^2 M_{\ell\bar{\ell}}(\mathbf{p})e^{-ip\cdot(x-y)}\right]
\end{align*}
となる.ここで
\begin{align*}
[\psi_\ell^+(x),\psi^{-c\dagger}_{\bar{\ell}}(y)]_{\mp}=&\sum_{\sigma} \int \frac{d^3\mathbf{p}}{(2\pi)^{3/2}} \sum_{\sigma'}\int \frac{d^3\mathbf{q}}{(2\pi)^{3/2}} u_\ell(\mathbf{p},\sigma) v^*_{\bar{\ell}}(\mathbf{q},\sigma')e^{ip\cdot x+iq\cdot y} \left[a(\mathbf{p},\sigma), a^{c}(\mathbf{q},\sigma')\right]_{\mp} \\
=&0 \\
[\psi^{-c}_\ell(x),\psi_{\bar{\ell}}^{+\dagger}(y)]_{\mp}=&\sum_{\sigma} \int \frac{d^3\mathbf{p}}{(2\pi)^{3/2}} \sum_{\sigma'}\int \frac{d^3\mathbf{q}}{(2\pi)^{3/2}} v_\ell(\mathbf{p},\sigma) u^*_{\bar{\ell}}(\mathbf{q},\sigma')e^{-ip\cdot x-iq\cdot y} \left[a^{c\dagger}(\mathbf{p},\sigma), a^{\dagger}(\mathbf{q},\sigma')\right]_{\mp} \\
=&0 \\
[\psi^{-c}_\ell(x),\psi_{\bar{\ell}}^{-c\dagger}(y)]_{\mp}=&\sum_{\sigma} \int \frac{d^3\mathbf{p}}{(2\pi)^{3/2}} \sum_{\sigma'}\int \frac{d^3\mathbf{q}}{(2\pi)^{3/2}} v_\ell(\mathbf{p},\sigma) v^*_{\bar{\ell}}(\mathbf{q},\sigma')e^{-ip\cdot x+iq\cdot y} \left[a^{c\dagger}(\mathbf{p},\sigma), a^{c}(\mathbf{q},\sigma')\right]_{\mp} \\
=&\mp \sum_{\sigma} \int \frac{d^3\mathbf{p}}{(2\pi)^{3/2}} \sum_{\sigma'}\int \frac{d^3\mathbf{q}}{(2\pi)^{3/2}} v_\ell(\mathbf{p},\sigma) v^*_{\bar{\ell}}(\mathbf{q},\sigma')e^{-ip\cdot x+iq\cdot y} \left[a^{c}(\mathbf{q},\sigma'),a^{c\dagger}(\mathbf{p},\sigma)\right]_{\mp} \\
=&\mp \sum_{\sigma} \int \frac{d^3\mathbf{p}}{(2\pi)^{3/2}} \sum_{\sigma'}\int \frac{d^3\mathbf{q}}{(2\pi)^{3/2}} v_\ell(\mathbf{p},\sigma) v^*_{\bar{\ell}}(\mathbf{q},\sigma')e^{-ip\cdot x+iq\cdot y} \delta^3(\mathbf{p}-\mathbf{q})\delta_{\sigma\sigma'}\\
=&\mp\sum_{\sigma} \int \frac{d^3\mathbf{p}}{(2\pi)^{3}} v_\ell(\mathbf{p},\sigma) v^*_{\bar{\ell}}(\mathbf{p},\sigma)e^{-ip\cdot (x-y)} \\
=&\mp\int \frac{d^3\mathbf{p}}{(2\pi)^{3}} M_{\ell\bar{\ell}}(\mathbf{p})e^{-ip\cdot (x-y)} \\
[\psi_\ell^+(x),\psi_{\bar{\ell}}^{+\dagger}(y)]_{\mp}=&\sum_{\sigma} \int \frac{d^3\mathbf{p}}{(2\pi)^{3/2}} \sum_{\sigma'}\int \frac{d^3\mathbf{q}}{(2\pi)^{3/2}} u_\ell(\mathbf{p},\sigma) u^*_{\bar{\ell}}(\mathbf{q},\sigma')e^{ip\cdot x-iq\cdot y} \left[a(\mathbf{p},\sigma), a^{\dagger}(\mathbf{q},\sigma')\right]_{\mp} \\
=&\sum_{\sigma} \int \frac{d^3\mathbf{p}}{(2\pi)^{3/2}} \sum_{\sigma'}\int \frac{d^3\mathbf{q}}{(2\pi)^{3/2}} u_\ell(\mathbf{p},\sigma) u^*_{\bar{\ell}}(\mathbf{q},\sigma')e^{ip\cdot x-iq\cdot y} \delta^3(\mathbf{p}-\mathbf{q})\delta_{\sigma\sigma'} \\
=&\sum_{\sigma} \int \frac{d^3\mathbf{p}}{(2\pi)^{3}} u_\ell(\mathbf{p},\sigma) u^*_{\bar{\ell}}(\mathbf{p},\sigma)e^{ip\cdot (x-y)} \\
=&\int \frac{d^3\mathbf{p}}{(2\pi)^{3}} N_{\ell\bar{\ell}}(\mathbf{p})e^{ip\cdot (x-y)}
\end{align*}
であることを使い,
\begin{align*}
N_{\ell\bar{\ell}}(\mathbf{p}):=&\sum_\sigma u_\ell(\mathbf{p},\sigma)u^*_{\bar{\ell}}(\mathbf{p},\sigma) \\
M_{\ell\bar{\ell}}(\mathbf{p}):=&\sum_\sigma v_\ell(\mathbf{p},\sigma)v^*_{\bar{\ell}}(\mathbf{p},\sigma)
\end{align*}
と定義した.ゼロ運動量でこれは
\begin{align*}
N(0)=&\sum_\sigma u(0,\sigma)u^*(0,\sigma) \\
=&u(0,1/2)u^*(0,1/2)+u(0,-1/2)u^*(0,-1/2) \\
=&\frac{1}{2}\left(
\begin{matrix}
1 \\
0 \\
b_u \\
0
\end{matrix}
\right)(1, 0, b_u , 0) +\frac{1}{2}\left(
\begin{matrix}
0 \\
1 \\
0 \\
b_u
\end{matrix}
\right)(0, 1 , 0, b_u) \\
=&\frac{1}{2}\left(
\begin{matrix}
1   & 0 & b_u & 0 \\
0   & 0 & 0   & 0 \\
b_u & 0 & 1   & 0 \\
0   & 0 & 0   & 0
\end{matrix}
\right)+\frac{1}{2}\left(
\begin{matrix}
0   & 0 & 0     & 0  \\
0   & 1 & 0     & b_u \\
0   & 0 & 0     & 0 \\
0   & b_u & 0   & 1
\end{matrix}
\right) \qquad \because b_u^2=1 \\
=&\frac{1}{2}\left(
\begin{matrix}
1   & 0 & b_u & 0 \\
0   & 1 & 0   & b_u \\
b_u & 0 & 1   & 0 \\
0   & b_u & 0   & 1
\end{matrix}
\right) \\
=& \frac{\bm{1}+b_u \beta}{2} \\
M(0)=&\sum_\sigma v(0,\sigma)v^*(0,\sigma) \\
=&v(0,1/2)v^*(0,1/2)+v(0,-1/2)v^*(0,-1/2) \\
=&\frac{1}{2}\left(
\begin{matrix}
0 \\
1 \\
0 \\
b_v
\end{matrix}
\right)(0, 1 , 0, b_v)+\frac{1}{2}\left(
\begin{matrix}
1 \\
0 \\
b_v \\
0
\end{matrix}
\right)(1, 0, b_v , 0) \\
=&\frac{1}{2}\left(
\begin{matrix}
0   & 0 & 0     & 0  \\
0   & 1 & 0     & b_v \\
0   & 0 & 0     & 0 \\
0   & b_v & 0   & 1
\end{matrix}
\right)+\frac{1}{2}\left(
\begin{matrix}
1   & 0 & b_v & 0 \\
0   & 0 & 0   & 0 \\
b_v & 0 & 1   & 0 \\
0   & 0 & 0   & 0
\end{matrix}
\right) \qquad \because b_v^2=1 \\
=&\frac{1}{2}\left(
\begin{matrix}
1   & 0 & b_v & 0 \\
0   & 1 & 0   & b_v \\
b_v & 0 & 1   & 0 \\
0   & b_v & 0   & 1
\end{matrix}
\right) \\
=& \frac{\bm{1}+b_v \beta}{2}
\end{align*}
がわかる.すると(5.5.6)(5.5.7)から
\begin{align*}
N_{\ell\bar{\ell}}(\mathbf{p})=&\sum_\sigma u_\ell(\mathbf{p},\sigma)u^*_{\bar{\ell}}(\mathbf{p},\sigma) \\
=&\sum_{\ell'\bar{\ell}'}\sum_\sigma \sqrt{\frac{m}{p^0}}D_{\ell\ell'}\Bigl(L(p)\Bigr)u_{\ell'}(0,\sigma)\sqrt{\frac{m}{p^0}} D^*_{\bar{\ell}\bar{\ell}'}\Bigl(L(p)\Bigr)u^*_{\bar{\ell}'}(0,\sigma) \\
=&\frac{m}{p^0}\sum_{\ell'\bar{\ell}'}\sum_\sigma D_{\ell\ell'}\Bigl(L(p)\Bigr)u_{\ell'}(0,\sigma)u^*_{\bar{\ell}'}(0,\sigma)D^\dagger_{\bar{\ell}'\bar{\ell}}\Bigl(L(p)\Bigr) \\
=&\frac{m}{p^0}\sum_{\ell'\bar{\ell}'}\sum_\sigma D_{\ell\ell'}\Bigl(L(p)\Bigr)N_{\ell'\bar{\ell}'}(0)D^\dagger_{\bar{\ell}'\bar{\ell}}\Bigl(L(p)\Bigr) \\
M_{\ell\bar{\ell}}(\mathbf{p})=&\sum_\sigma v_\ell(\mathbf{p},\sigma)v^*_{\bar{\ell}}(\mathbf{p},\sigma) \\
=&\sum_{\ell'\bar{\ell}'}\sum_\sigma \sqrt{\frac{m}{p^0}}D_{\ell\ell'}\Bigl(L(p)\Bigr)v_{\ell'}(0,\sigma)\sqrt{\frac{m}{p^0}} D^*_{\bar{\ell}\bar{\ell}'}\Bigl(L(p)\Bigr)v^*_{\bar{\ell}'}(0,\sigma) \\
=&\frac{m}{p^0}\sum_{\ell'\bar{\ell}'}\sum_\sigma D_{\ell\ell'}\Bigl(L(p)\Bigr)v_{\ell'}(0,\sigma)v^*_{\bar{\ell}'}(0,\sigma)D^\dagger_{\bar{\ell}'\bar{\ell}}\Bigl(L(p)\Bigr) \\
=&\frac{m}{p^0}\sum_{\ell'\bar{\ell}'}\sum_\sigma D_{\ell\ell'}\Bigl(L(p)\Bigr)M_{\ell'\bar{\ell}'}(0)D^\dagger_{\bar{\ell}'\bar{\ell}}\Bigl(L(p)\Bigr) \\
\therefore \quad N(\mathbf{p})=&\frac{m}{p^0}D\Bigl(L(p)\Bigr)N(0)D^\dagger\Bigl(L(p)\Bigr) \\
=&\frac{m}{2p^0}D\Bigl(L(p)\Bigr)[\bm{1}+b_u\beta]D^\dagger\Bigl(L(p)\Bigr) \\
M(\mathbf{p})=&\frac{m}{p^0}D\Bigl(L(p)\Bigr)M(0)D^\dagger\Bigl(L(p)\Bigr) \\
=&\frac{m}{2p^0}D\Bigl(L(p)\Bigr)[\bm{1}+b_v\beta]D^\dagger\Bigl(L(p)\Bigr)
\end{align*}
を得る.擬ユニタリー条件(5.4.32)を用いると
\begin{align*}
&D\Bigl(L(p)\Bigr)\beta D^\dagger\Bigl(L(p)\Bigr)\beta=D\Bigl(L(p)\Bigr)D^{-1}\Bigl(L(p)\Bigr)=1 \\
\therefore \quad & D\Bigl(L(p)\Bigr)\beta D^\dagger\Bigl(L(p)\Bigr)=\beta
\end{align*}
と
\begin{align*}
&D^\dagger\Bigl(L(p)\Bigr)=\beta D^{-1}\Bigl(L(p)\Bigr) \beta \\
\therefore \quad &D\Bigl(L(p)\Bigr) D^\dagger \Bigl(L(p)\Bigr)=D\Bigl(L(p)\Bigr)\beta D^{-1}\Bigl(L(p)\Bigr) \beta
\end{align*}
がわかる.また$\beta=i\gamma^0$を思い出すとローレンツ変換則(5.4.8)を用いて
\begin{align*}
D\Bigl(L(p)\Bigr)\beta D^{-1}\Bigl(L(p)\Bigr)=&iD\Bigl(L(p)\Bigr)\gamma^0 D^{-1}\Bigl(L(p)\Bigr) \\
=&i\tensor{L(p)}{_\mu^0}\gamma^\mu \qquad \because D(\Lambda)\gamma^\nu D^{-1}(\Lambda)=\tensor{\Lambda}{_\mu^\nu}\gamma^\mu \, で \, \Lambda=L(p),\nu=0 \\
=&\frac{-i}{m}k_\nu \tensor{L(p)}{_\mu^\nu}\gamma^\mu \quad (k^\mu=(0,0,0,m)^T) \\
=&\frac{-i}{m} \gamma^\mu \tensor{L(p)}{_\mu^\nu}k_\nu \\
=&\frac{-i}{m}\gamma_\mu \tensor{L(p)}{^\mu_\nu}k^\nu \quad \because u^\lambda \tensor{\Lambda}{_\lambda^\sigma} v_\sigma=u^\lambda \eta_{\lambda\mu}\tensor{\Lambda}{^\mu_\rho}\eta^{\rho\sigma} v_\sigma=u_\mu \tensor{\Lambda}{^\mu_\rho} v^\rho \\
=&\frac{-i}{m}\gamma_\mu p^\mu
\end{align*}
となる.これらを合わせると
\begin{align*}
N(\mathbf{p})=&\frac{m}{2p^0}D\Bigl(L(p)\Bigr)[\bm{1}+b_u\beta]D^\dagger\Bigl(L(p)\Bigr) \\
=&\frac{m}{2p^0}\Bigl[D\Bigl(L(p)\Bigr)D^\dagger\Bigl(L(p)\Bigr)+b_uD\Bigl(L(p)\Bigr)\beta D^\dagger\Bigl(L(p)\Bigr)\Bigr] \\
=&\frac{m}{2p^0}\Bigl[D\Bigl(L(p)\Bigr)\beta D^{-1}\Bigl(L(p)\Bigr)\beta +b_u\beta \Bigr] \\
=&\frac{m}{2p^0}\Bigl[\frac{-i}{m}p_\mu \gamma^\mu \beta +b_u\beta \Bigr] \\
=&\frac{1}{2p^0}\Bigl[-ip_\mu \gamma^\mu +b_u m\Bigr]\beta \\
M(\mathbf{p})=&\frac{1}{2p^0}\Bigl[-ip_\mu \gamma^\mu +b_v m\Bigr]\beta
\end{align*}
($M(\mathbf{p})$についてはもう一度計算しなくても,(5.5.24)(5.5.25)を観察すれば,$N(\mathbf{p})$の結果を$b_u\to b_v$とすればすぐ得られることがわかる.)(5.5.20)でこれを使えば
\begin{align*}
[\psi_\ell(x),\psi^{\dagger}_{\bar{\ell}}(y)]_{\mp}=&\int \frac{d^3\mathbf{p}}{(2\pi)^{3}} \Bigl[|\kappa|^2N_{\ell\bar{\ell}}(\mathbf{p})e^{ip\cdot (x-y)} \mp |\lambda|^2 M_{\ell\bar{\ell}}(\mathbf{p})e^{-ip\cdot(x-y)}\Bigr] \\
=&\int \frac{d^3\mathbf{p}}{(2\pi)^{3}} \Bigl[|\kappa|^2\frac{1}{2p^0}\Bigl(-ip_\mu \gamma^\mu +b_u m\Bigr)\beta e^{ip\cdot (x-y)} \mp |\lambda|^2 \frac{1}{2p^0}\Bigl(-ip_\mu \gamma^\mu +b_u m\Bigr)\beta e^{-ip\cdot(x-y)}\Bigr]_{\ell\bar{\ell}} \\
=&|\kappa|^2 \left[\int \frac{d^3\mathbf{p}}{(2\pi)^{3} 2p^0}\Bigl(-ip_\mu \gamma^\mu +b_u m\Bigr)\beta e^{ip\cdot (x-y)}\right]_{\ell\bar{\ell}} \\
&\mp |\lambda|^2 \left[\int \frac{d^3\mathbf{p}}{(2\pi)^{3} 2p^0}\Bigl(-ip_\mu \gamma^\mu +b_u m\Bigr)\beta e^{-ip\cdot(x-y)}\right]_{\ell\bar{\ell}} \\
=&|\kappa|^2 \left[\Bigl(-\gamma^\mu \partial^x_\mu +b_u m\Bigr)\beta\int \frac{d^3\mathbf{p}}{(2\pi)^{3} 2p^0} e^{ip\cdot (x-y)}\right]_{\ell\bar{\ell}} \\
&\mp |\lambda|^2 \left[\Bigl(+\gamma^\mu \partial^x_\mu +b_u m\Bigr)\beta \int \frac{d^3\mathbf{p}}{(2\pi)^{3} 2p^0} e^{-ip\cdot(x-y)}\right]_{\ell\bar{\ell}} \\
=&|\kappa|^2 \left[\Bigl(-\gamma^\mu \partial^x_\mu +b_u m\Bigr)\beta\Delta_+(x-y)\right]_{\ell\bar{\ell}} \\
&\mp |\lambda|^2 \left[\Bigl(+\gamma^\mu \partial^x_\mu +b_u m\Bigr)\beta \Delta_+(y-x)\right]_{\ell\bar{\ell}} \\
=&-|\kappa|^2 \Bigl[\gamma^\mu \partial_\mu^x \beta\Bigr]_{\ell\bar{\ell}} \Delta_+(x-y) \mp |\lambda|^2 \Bigl[\gamma^\mu \partial_\mu^x \beta\Bigr]_{\ell\bar{\ell}} \Delta_+(y-x) \\
&+m|\kappa|^2b_u \beta_{\ell\bar{\ell}}\Delta_+(x-y) \mp |\lambda|^2 b_v \beta_{\ell\bar{\ell}}\Delta_+(y-x)
\end{align*}
となる.空間的な$x-y$な場合,$\Delta_+(x-y)$は偶関数で$\Delta_+(x-y)=\Delta_+(y-x)$であり
\begin{align*}
=&-(|\kappa|^2\pm |\lambda|^2) \Bigl[\gamma^\mu \partial_\mu^x \beta\Bigr]_{\ell\bar{\ell}}\Delta_+(x-y) \\
&+m(|\kappa|^2b_u \mp |\lambda|^2 b_v) \beta_{\ell\bar{\ell}}\Delta_+(x-y)
\end{align*}
となる\footnote{本文で$\partial_\mu \Delta_+(x-y)$が奇関数だという文があるが,これは所謂
\begin{align*}
\frac{\partial \Delta_+}{\partial x^\mu}(x-y)=-\frac{\partial \Delta_+}{\partial x^\mu}(y-x)
\end{align*}
の意味で奇関数ということであり,
\begin{align*}
\frac{\partial}{\partial x^\mu}\Delta_+(x-y)=\frac{\partial}{\partial x^\mu}\Delta_+(y-x)
\end{align*}
はやはり偶関数である.$\partial_\mu \Delta_+(x-y)$という表記だとどちらを指しているかが分かりにくい.強いて書くなら,前者は$(\partial_\mu \Delta_+)(x-y)$であり,後者は$\partial_\mu[\Delta_+(x-y)]$である.このノートのこの文脈では後者の意味で微分を用いた.たびたび述べているが,微分の表記には毎回気を付けて.}.よって交換子または反交換子が空間的な$x-y$でゼロとなるには,$\gamma^\mu$の線形独立性より第一項目と第二項目が個別にゼロにならなくてはならず,それは
\begin{align*}
|\kappa|^2=\mp |\lambda|^2
\end{align*}
かつ
\begin{align*}
|\kappa|^2b_u=\pm |\lambda|^2 b_v
\end{align*}
であれば,またそのときに限り満たされる.明らかに,前者の条件は符号を下側$\pm=+$と選ぶときだけ可能である.すなわち,ディラック場によって記述される粒子はフェルミオンでなければならない!(符号の違いはボゾンかフェルミオンかに対応しているという4.2節の説明を思い出す.)すると$|\kappa|^2=|\lambda|^2$という条件が前者から与えられ,さらに後者からは(前者の条件を使って)$b_u=-b_v$という条件が与えられる.スカラーの場合と全く同様にして,(まぁ一応ここでもやるけど)一粒子状態の位相を再定義すると,消滅・生成演算子の位相が変更されて$a(\mathbf{p},\sigma)\to e^{-i\theta}a(\mathbf{p},\sigma),a^{c\dagger}(\mathbf{p},\sigma)\to e^{-i\varphi}a^{c\dagger}(\mathbf{p},\sigma)$となる.$\kappa=|\kappa|e^{i\alpha},\lambda=|\lambda|e^{i\beta}$と書くと
\begin{align*}
\psi_\ell(x)=&|\kappa|e^{i(\alpha-\theta)}\sum_\sigma \int \frac{d^3\mathbf{p}}{(2\pi)^{3/2}} u_\ell(\mathbf{p},\sigma)e^{ip\cdot x} a(\mathbf{p},\sigma)+ |\lambda|e^{i(\beta-\varphi)}\sum_\sigma \int \frac{d^3\mathbf{p}}{(2\pi)^{3/2}} v_\ell(\mathbf{p},\sigma)e^{-ip\cdot x} a^{c\dagger}(\mathbf{p},\sigma)
\end{align*}
となるから,やはり$\theta=\alpha,\varphi=\beta$とおけば位相を取り除ける.場全体のスケールを再定義して,$|\kappa|=|\lambda|$も取り除ける.そうすることで
\begin{align*}
\psi_\ell(x)=&\sum_\sigma \int \frac{d^3\mathbf{p}}{(2\pi)^{3/2}} u_\ell(\mathbf{p},\sigma)e^{ip\cdot x} a(\mathbf{p},\sigma)+ \sum_\sigma \int \frac{d^3\mathbf{p}}{(2\pi)^{3/2}} v_\ell(\mathbf{p},\sigma)e^{-ip\cdot x} a^{c\dagger}(\mathbf{p},\sigma)
\end{align*}
となる.最後に,$b_u=-b_v$についての問題が残っている.これらは符号因子であるから,$b_u=-b_v=+1$の場合と$b_u=-b_v=-1$の場合が存在する.前者の場合は,$\psi(x)$に現れる係数関数のゼロ運動量での値が
\begin{align*}
u\left(0,\frac{1}{2}\right)=&\frac{1}{\sqrt{2}} \left(
\begin{matrix}
1 \\
0 \\
1 \\
0
\end{matrix}
\right) ,\quad u\left(0,-\frac{1}{2}\right)=\frac{1}{\sqrt{2}}\left(
\begin{matrix}
0 \\
1 \\
0 \\
1
\end{matrix}
\right) \\
v\left(0,\frac{1}{2}\right)=&\frac{1}{\sqrt{2}} \left(
\begin{matrix}
0 \\
1 \\
0 \\
-1
\end{matrix}
\right) ,\quad v\left(0,-\frac{1}{2}\right)=-\frac{1}{\sqrt{2}}\left(
\begin{matrix}
1 \\
0 \\
-1 \\
0
\end{matrix}
\right)
\end{align*}
である場合だといえる.後者の場合,$\psi(x)$に現れる係数関数のゼロ運動量での値が
\begin{align*}
u\left(0,\frac{1}{2}\right)=&\frac{1}{\sqrt{2}} \left(
\begin{matrix}
1 \\
0 \\
-1 \\
0
\end{matrix}
\right) ,\quad u\left(0,-\frac{1}{2}\right)=\frac{1}{\sqrt{2}}\left(
\begin{matrix}
0 \\
1 \\
0 \\
-1
\end{matrix}
\right) \\
v\left(0,\frac{1}{2}\right)=&\frac{1}{\sqrt{2}} \left(
\begin{matrix}
0 \\
1 \\
0 \\
1
\end{matrix}
\right) ,\quad v\left(0,-\frac{1}{2}\right)=-\frac{1}{\sqrt{2}}\left(
\begin{matrix}
1 \\
0 \\
1 \\
0
\end{matrix}
\right)
\end{align*}
となり,前者と後者で別の場合を考えることになりそうだが,実際はそうではない.これは,$\psi(x)$を$\gamma_5 \psi(x)$と再定義することで
\begin{align*}
\gamma_5 \psi(x)=&\sum_\sigma \int \frac{d^3\mathbf{p}}{(2\pi)^{3/2}} u(\mathbf{p},\sigma)e^{ip\cdot x} a(\mathbf{p},\sigma)+ \sum_\sigma \int \frac{d^3\mathbf{p}}{(2\pi)^{3/2}} v(\mathbf{p},\sigma)e^{-ip\cdot x} a^{c\dagger}(\mathbf{p},\sigma) \\
\therefore \quad \psi(x)=&\sum_\sigma \int \frac{d^3\mathbf{p}}{(2\pi)^{3/2}} \gamma_5 u(\mathbf{p},\sigma)e^{ip\cdot x} a(\mathbf{p},\sigma)+ \sum_\sigma \int \frac{d^3\mathbf{p}}{(2\pi)^{3/2}} \gamma_5 v(\mathbf{p},\sigma)e^{-ip\cdot x} a^{c\dagger}(\mathbf{p},\sigma)
\end{align*}
$D(\Lambda)$と$\gamma_5$は交換(5.4.23)するという結果を用いて
\begin{align*}
\gamma_5 u(\mathbf{p},\sigma)=&\sqrt{\frac{m}{p^0}}\gamma_5 D\Bigl(L(p)\Bigr) u(0,\sigma)\\
=&\sqrt{\frac{m}{p^0}}D\Bigl(L(p)\Bigr)\gamma_5 u(0,\sigma) \\
\gamma_5 v(\mathbf{p},\sigma)=&\sqrt{\frac{m}{p^0}}\gamma_5 D\Bigl(L(p)\Bigr) v(0,\sigma) \\
=&\sqrt{\frac{m}{p^0}}D\Bigl(L(p)\Bigr)\gamma_5 v(0,\sigma)
\end{align*}
この$\gamma_5 u(0,\sigma),\gamma_5 v(0,\sigma)$という量は,ちょうど$b_u,b_v$部分の符号のみを反転させる効果を及ぼす.
\begin{align*}
\gamma_5 u(0,1/2)=&\frac{1}{\sqrt{2}}\left(
\begin{matrix}
1 & 0 & 0 & 0 \\
0 & 1 & 0 & 0 \\
0 & 0 & -1 & 0 \\
0 & 0 & 0 & -1 \\
\end{matrix}
\right)\left(
\begin{matrix}
1 \\
0 \\
-1 \\
0
\end{matrix}
\right) \\
=&\frac{1}{\sqrt{2}}\left(
\begin{matrix}
1 \\
0 \\
1 \\
0
\end{matrix}
\right) \\
\gamma_5 u(0,-1/2)=&\frac{1}{\sqrt{2}}\left(
\begin{matrix}
1 & 0 & 0 & 0 \\
0 & 1 & 0 & 0 \\
0 & 0 & -1 & 0 \\
0 & 0 & 0 & -1 \\
\end{matrix}
\right)\left(
\begin{matrix}
0 \\
1 \\
0 \\
-1
\end{matrix}
\right) \\
=&\frac{1}{\sqrt{2}}\left(
\begin{matrix}
0 \\
1 \\
0 \\
1
\end{matrix}
\right) \\
\gamma_5 v(0,1/2)=&\frac{1}{\sqrt{2}}\left(
\begin{matrix}
1 & 0 & 0 & 0 \\
0 & 1 & 0 & 0 \\
0 & 0 & -1 & 0 \\
0 & 0 & 0 & -1 \\
\end{matrix}
\right)\left(
\begin{matrix}
0 \\
1 \\
0 \\
1
\end{matrix}
\right) \\
=&\frac{1}{\sqrt{2}}\left(
\begin{matrix}
0 \\
1 \\
0 \\
-1
\end{matrix}
\right) \\
\gamma_5 v(0,-1/2)=&-\frac{1}{\sqrt{2}}\left(
\begin{matrix}
1 & 0 & 0 & 0 \\
0 & 1 & 0 & 0 \\
0 & 0 & -1 & 0 \\
0 & 0 & 0 & -1 \\
\end{matrix}
\right)\left(
\begin{matrix}
1 \\
0 \\
1 \\
0
\end{matrix}
\right) \\
=&\frac{1}{\sqrt{2}}\left(
\begin{matrix}
1 \\
0 \\
-1 \\
0
\end{matrix}
\right)
\end{align*}
これは$b_u=-b_v=+1$としたときの$u(0,\sigma),v(0,\sigma)$にほかならない.これを$D(L(p))$でブーストした$\gamma_5 u(\mathbf{p},\sigma),\gamma_5 v(\mathbf{p},\sigma)$も前者の場合の係数関数$u(\mathbf{p},\sigma),v(\mathbf{p},\sigma)$と一致し,したがって,どちらの場合も等価であり,単に前者の場合のみを考えればよい.\par
一旦まとめよう.ディラック場はいまや
\begin{align*}
\psi_\ell(x)=&\sum_\sigma \int \frac{d^3\mathbf{p}}{(2\pi)^{3/2}} \Bigl[u_\ell(\mathbf{p},\sigma)e^{ip\cdot x} a(\mathbf{p},\sigma)+ v_\ell(\mathbf{p},\sigma)e^{-ip\cdot x} a^{c\dagger}(\mathbf{p},\sigma)\Bigr]
\end{align*}
であり,ゼロ運動量での係数関数は(統一されて)
\begin{align*}
u\left(0,\frac{1}{2}\right)=&\frac{1}{\sqrt{2}} \left(
\begin{matrix}
1 \\
0 \\
1 \\
0
\end{matrix}
\right) ,\quad u\left(0,-\frac{1}{2}\right)=\frac{1}{\sqrt{2}}\left(
\begin{matrix}
0 \\
1 \\
0 \\
1
\end{matrix}
\right) \\
v\left(0,\frac{1}{2}\right)=&\frac{1}{\sqrt{2}} \left(
\begin{matrix}
0 \\
1 \\
0 \\
-1
\end{matrix}
\right) ,\quad v\left(0,-\frac{1}{2}\right)=\frac{-1}{\sqrt{2}}\left(
\begin{matrix}
1 \\
0 \\
-1 \\
0
\end{matrix}
\right)
\end{align*}
である.スピン和は(5.5.24)(5.5.25)で$b_u=-b_v=+1$とおいて
\begin{align*}
N(\mathbf{p})=&\sum_\sigma u(\mathbf{p},\sigma)u^*(\mathbf{p},\sigma) =\frac{1}{2p^0}\Bigl[-ip_\mu \gamma^\mu + m\Bigr]\beta \\
M(\mathbf{p})=&\sum_\sigma v(\mathbf{p},\sigma)v^*(\mathbf{p},\sigma) =\frac{1}{2p^0}\Bigl[-ip_\mu \gamma^\mu -b_v m\Bigr]\beta
\end{align*}
となる.\uwave{反}交換子は(5.5.20)より$\mp=+ ,\kappa=\lambda=1,b_u=-b_v=+1$とおいて
\begin{align*}
[\psi_\ell(x),\psi^{\dagger}_{\bar{\ell}}(y)]_{+}=&-|\kappa|^2 \Bigl[\gamma^\mu \partial_\mu^x \beta\Bigr]_{\ell\bar{\ell}} \Delta_+(x-y) + |\lambda|^2 \Bigl[\gamma^\mu \partial_\mu^x \beta\Bigr]_{\ell\bar{\ell}} \Delta_+(y-x) \\
&+m|\kappa|^2b_u \beta_{\ell\bar{\ell}}\Delta_+(x-y) + |\lambda|^2 b_v \beta_{\ell\bar{\ell}}\Delta_+(y-x) \\
=&-\Bigl[\gamma^\mu \partial_\mu^x \beta\Bigr]_{\ell\bar{\ell}} \Delta_+(x-y) + \Bigl[\gamma^\mu \partial_\mu^x \beta\Bigr]_{\ell\bar{\ell}} \Delta_+(y-x) \\
&+m \beta_{\ell\bar{\ell}}\Delta_+(x-y) - m \beta_{\ell\bar{\ell}}\Delta_+(y-x) \\
=&\Bigl[(-\gamma^\mu \partial_\mu+m)\beta \Bigr]_{\ell\bar{\ell}}\Bigl(\Delta_+(x-y)-\Delta_+(y-x)\Bigr) \\
=&\Bigl[(-\gamma^\mu \partial_\mu+m)\beta \Bigr]_{\ell\bar{\ell}}\Delta(x-y)
\end{align*}
となる.(5.1.6)(5.1.7)より,ローレンツ変換のもとでの変換性は
\begin{align*}
U_0(\Lambda) \psi(x) U_0^{-1}(\Lambda)=D(\Lambda^{-1})\psi(\Lambda x)=\exp\left(-\frac{i}{2}\omega_{\mu\nu}\mc{S}^{\mu\nu}\right)\psi(\Lambda x)
\end{align*}
となる.

\vskip\baselineskip



空間反転のもとで場$\psi(x)$は$\psi(\mc{P}x)$に比例する場へ変換しなければならないという要求に戻ろう.この節の最初に行った計算を思い返すと,いまや(5.5.14)より$u(-\mathbf{p},\sigma)=+\beta u(\mathbf{p},\sigma),v(-\mathbf{p},\sigma)=-\beta v(\mathbf{p},\sigma)$であるから
\begin{align*}
\mathsf{P} \psi^{+}_\ell(x) \mathsf{P}^{-1} =&\eta^* \sum_\sigma \int \frac{d^3\mathbf{p}}{(2\pi)^{3/2}} u_\ell(-\mathbf{p},\sigma)e^{ip\cdot (\mc{P}x)}a(\mathbf{p},\sigma) \\
=&\eta^* \beta \sum_\sigma \int \frac{d^3\mathbf{p}}{(2\pi)^{3/2}} u_\ell(\mathbf{p},\sigma)e^{ip\cdot (\mc{P}x)}a(\mathbf{p},\sigma) \\
=&\eta^* \beta \psi^{+}(\mc{P}x)+ \\
\mathsf{P} \psi^{-c}_\ell(x)\mathsf{P}^{-1}=&\eta^c \sum_\sigma \int \frac{d^3\mathbf{p}}{(2\pi)^{3/2}} v_\ell(-\mathbf{p},\sigma)e^{-ip\cdot (\mc{P}x)}a^{c\dagger}(\mathbf{p},\sigma) \\
=&-\eta^c \beta \sum_\sigma \int \frac{d^3\mathbf{p}}{(2\pi)^{3/2}} v_\ell(\mathbf{p},\sigma)e^{-ip\cdot (\mc{P}x)}a^{c\dagger}(\mathbf{p},\sigma) \\
=&-\eta^c \beta \psi^{-c}(\mc{P}x)
\end{align*}
より
\begin{align*}
\mathsf{P}\psi(x)\mathsf{P}^{-1}=\eta^* \beta \psi^{+}(\mc{P}x)-\eta^c \beta \psi^{-c}(\mc{P}x)
\end{align*}
となる.これが$\psi(\mc{P}x)$に依存するためには
\begin{align*}
\eta^c=-\eta^*
\end{align*}
でなければならない.すなわち,スピン$\frac{1}{2}$粒子とその反粒子からなる状態の固有パリティは$\eta\eta^c=-\eta\eta^*=-1$となり,奇となる.これを用いると,フェルミオンと反フェルミオンの束縛状態の固有パリティは,3.3節での計算を繰り返して
\begin{align*}
\mathsf{P}\Phi_{JM}^{\ell s}=&\int d^3 \mathbf{p} \sum_{m,\sigma}C_{\ell s}(JM;m,\sigma)Y_{\ell m}(\hat{\mathbf{p}}) \sum_{\sigma_1,\sigma_2} C_{s_1,s_2}(s\sigma;\sigma_1,\sigma_2) \mathsf{P} \Phi_{\mathbf{p},\sigma_1;-\mathbf{p},\sigma_2} \\
=&\eta\eta^c \int d^3 \mathbf{p} \sum_{m,\sigma}C_{\ell s}(JM;m,\sigma)Y_{\ell m}(\hat{\mathbf{p}}) \sum_{\sigma_1,\sigma_2} C_{s_1,s_2}(s\sigma;\sigma_1,\sigma_2) \Phi_{-\mathbf{p},\sigma_1;+\mathbf{p},\sigma_2} \\
=&\eta\eta^c \int d^3 \mathbf{p} \sum_{m,\sigma}C_{\ell s}(JM;m,\sigma)Y_{\ell m}(-\hat{\mathbf{p}}) \sum_{\sigma_1,\sigma_2} C_{s_1,s_2}(s\sigma;\sigma_1,\sigma_2) \Phi_{+\mathbf{p},\sigma_1;-\mathbf{p},\sigma_2} \\
=&(-1)^\ell \eta\eta^c \int d^3 \mathbf{p} \sum_{m,\sigma}C_{\ell s}(JM;m,\sigma)Y_{\ell m}(\hat{\mathbf{p}}) \sum_{\sigma_1,\sigma_2} C_{s_1,s_2}(s\sigma;\sigma_1,\sigma_2) \Phi_{+\mathbf{p},\sigma_1;-\mathbf{p},\sigma_2} \\
=&(-1)^\ell \eta\eta^c \Phi_{JM}^{\ell s}
\end{align*}
となり,$-(-1)^\ell$であることがわかる.$\ell=0$のs波束縛状態であるとき,パリティは$-1$である.これが$\rho^0$粒子や$J/\psi$粒子などの中間子がクォーク-反クォーク対のs波束縛状態として解釈できる理由である.(実際,$\rho^0$は$\frac{1}{\sqrt{2}}(u\bar{u}-d\bar{d})$であり,$J/\psi$は$c\bar{c}$である.)以上より,因果律を満たすディラック場の空間反転のもとでの変換性は
\begin{align*}
\mathsf{P}\psi(x)\mathsf{P}^{-1}=\eta^*\beta \psi(\mc{P}x)
\end{align*}
であることがわかる.

\vskip\baselineskip


時間と荷電の反転に進む前に,以下のことに言及しておく.$u(\mathbf{p},\sigma),v(\mathbf{p},\sigma)$は$-ip_\mu \gamma^\mu/m$の固有ベクトルで,その固有値はそれぞれ$+1,-1$である.実際
\begin{align*}
\frac{-i}{m}p_\mu \gamma^\mu u(\mathbf{p},\sigma)=&D\Bigl(L(p)\Bigr)\beta D^{-1}\Bigl(L(p)\Bigr)u(\mathbf{p},\sigma) \quad \because (5.5.26) \\
=&\sqrt{\frac{m}{p^0}}D\Bigl(L(p)\Bigr)\beta u(0,\sigma) \quad \because (5.5.6) \\
=&+\sqrt{\frac{m}{p^0}}D\Bigl(L(p)\Bigr)u(0,\sigma) \quad \because (5.5.14),b_u=+1 \\
=&+u(\mathbf{p},\sigma) \quad \because (5.5.6)
\end{align*}
なので$u(\mathbf{p},\sigma)$は固有値$+1$の固有ベクトルであり,一方
\begin{align*}
\frac{-i}{m}p_\mu \gamma^\mu v(\mathbf{p},\sigma)=&D\Bigl(L(p)\Bigr)\beta D^{-1}\Bigl(L(p)\Bigr)v(\mathbf{p},\sigma) \quad \because (5.5.26) \\
=&\sqrt{\frac{m}{p^0}}D\Bigl(L(p)\Bigr)\beta v(0,\sigma) \quad \because (5.5.7) \\
=&-\sqrt{\frac{m}{p^0}}D\Bigl(L(p)\Bigr)v(0,\sigma) \quad \because (5.5.14),b_v=-1 \\
=&-v(\mathbf{p},\sigma) \quad \because (5.5.7)
\end{align*}
なので$v(\mathbf{p},\sigma)$は固有値$-1$の固有ベクトルである.したがって
\begin{align*}
(ip_\mu \gamma^\mu +m)u(\mathbf{p},\sigma)=0 ,\quad (-ip_\mu \gamma^\mu +m)v(\mathbf{p},\sigma)=0
\end{align*}
がなりたつ.これを用いると,場(5.5.34)は
\begin{align*}
(\gamma^\mu \partial_\mu +m)\psi(x)=&(\gamma^\mu \partial_\mu +m)\sum_\sigma \int \frac{d^3\mathbf{p}}{(2\pi)^{3/2}} \Bigl[u_\ell(\mathbf{p},\sigma)e^{ip\cdot x} a(\mathbf{p},\sigma)+ v_\ell(\mathbf{p},\sigma)e^{-ip\cdot x} a^{c\dagger}(\mathbf{p},\sigma)\Bigr] \\
=&\sum_\sigma \int \frac{d^3\mathbf{p}}{(2\pi)^{3/2}} \Bigl[(ip_\mu \gamma^\mu +m) u_\ell(\mathbf{p},\sigma)e^{ip\cdot x} a(\mathbf{p},\sigma) \\
&\qquad \qquad + (-ip_\mu \gamma^\mu +m)v_\ell(\mathbf{p},\sigma)e^{-ip\cdot x} a^{c\dagger}(\mathbf{p},\sigma)\Bigr] \\
=&0
\end{align*}
という微分方程式を満たすことが\uwave{導かれる}.これは有名なスピン$\frac{1}{2}$自由粒子のディラック方程式である\footnote{(5.4.17)を見ればわかる通り,ここでのガンマ行列はPeskinなどで普通使われるガンマ行列と違い,余分に$-i$をかけたものになっている.したがって従来のガンマ行列を$\tilde{\gamma}^\mu:=i\gamma^\mu$とおいてディラック方程式に用いると,$0=(-i\tilde{\gamma}^\mu\partial_\mu +m)\psi(x)$となり,したがって(チルダを外して)我々がよく見知った形$(i\gamma^\mu\partial_\mu -m)\psi(x)=0$が復元される.}.ここで採用した観点から言えば,自由粒子のディラック方程式はまさに,固有順時ローレンツ群の二つの既約表現(上成分と下成分)をまとめて空間反転のもとでも単純に変換する場を作る際に用いた慣習から自然に導かれたものであり,したがってそれらの慣習をローレンツ不変な形で記録したものである.

\vskip\baselineskip


ディラック場の荷電共役と時間反転の性質を導出しよう.まず,荷電共役を考える.
\begin{align*}
\mathsf{C} a(\mathbf{p},\sigma)\mathsf{C}^{-1}=&\xi^* a^c(\mathbf{p},\sigma) \\
\mathsf{C} a^{c\dagger}(\mathbf{p},\sigma)\mathsf{C}^{-1}=&\xi^c a^\dagger(\mathbf{p},\sigma)
\end{align*}
であるから,前と同様の操作を繰り返せば
\begin{align*}
\mathsf{C} \psi^+(x)\mathsf{C}^{-1}=&\sum_\sigma \int \frac{d^3\mathbf{p}}{(2\pi)^{3/2}} u(\mathbf{p},\sigma)e^{ip\cdot x}\mathsf{C} a(\mathbf{p},\sigma) \mathsf{C}^{-1} \\
=&\xi^* \sum_\sigma \int \frac{d^3\mathbf{p}}{(2\pi)^{3/2}} u(\mathbf{p},\sigma)e^{ip\cdot x}a^c(\mathbf{p},\sigma) \\
\mathsf{C} \psi^{-c}_\ell(x)\mathsf{C}^{-1}=&\sum_\sigma \int \frac{d^3\mathbf{p}}{(2\pi)^{3/2}} v(\mathbf{p},\sigma)e^{-ip\cdot x}\mathsf{C} a^{c\dagger}(\mathbf{p},\sigma) \mathsf{C}^{-1} \\
=&\xi^c \sum_\sigma \int \frac{d^3\mathbf{p}}{(2\pi)^{3/2}} v(\mathbf{p},\sigma)e^{-ip\cdot x}a^{\dagger}(\mathbf{p},\sigma)
\end{align*}
となる.これがそれぞれ$\psi^{-c}(x)$と$\psi^{+}(x)$に関係しているためには,$u(\mathbf{p},\sigma)$が$v^*(\mathbf{p},\sigma)$に,また$v(\mathbf{p},\sigma)$が$u^*(\mathbf{p},\sigma)$に関係していなければならない.これらの係数関数はゼロ運動量では実数だが,有限運動量での係数関数を得るには,複素行列$D(L(p))$をかけなければならない.(5.4.41)より,一般に実数パラメータ$\omega_{\mu\nu}$について
\begin{align*}
\left[\exp\left(\frac{i}{2}\omega_{\mu\nu}\mc{S}^{\mu\nu}\right)\right]^*=&\exp\left(-\frac{i}{2}\omega_{\mu\nu}(\mc{S}^{\mu\nu})^*\right) \\
=&\exp\left(+\frac{i}{2}\omega_{\mu\nu}\beta \mc{C}\mc{S}^{\mu\nu}\mc{C}^{-1} \beta \right) \quad \because (5.4.41)\\
=&\sum_{n=0}^\infty \left[\frac{i}{2}\omega_{\mu\nu}\beta \mc{C}\mc{S}^{\mu\nu}\mc{C}^{-1} \beta \right]^n \\
=&\sum_{n=0}^\infty \beta \mc{C}\left[\frac{i}{2}\omega_{\mu\nu}\mc{S}^{\mu\nu} \right]^n\mc{C}^{-1} \beta \quad \because \mc{C}^{-1} \beta \beta \mc{C}=1 \\
=&\beta \mc{C} \exp\left(\frac{i}{2}\omega_{\mu\nu}\mc{S}^{\mu\nu}\right) \mc{C}^{-1} \beta \\
\therefore\quad D(\Lambda)^*=&\beta \mc{C} D(\Lambda)\mc{C}^{-1} \beta 
\end{align*}
という関係が得られる.よって特に
\begin{align*}
D(L(p))^*=\beta \mc{C} D(L(p))\mc{C}^{-1} \beta 
\end{align*}
がわかる.また
\begin{align*}
\mc{C}^{-1} \beta u\left(0,\frac{1}{2}\right)=&-\mc{C} u\left(0,\frac{1}{2}\right) \quad \because (5.5.14),b_u=+1 ,\mc{C}^{-1}=-\mc{C} \\
=&-\frac{1}{\sqrt{2}}\left(
\begin{matrix}
0 & -1 & 0 & 0 \\
1 & 0 & 0 & 0 \\
0 & 0 & 0 & 1 \\
0 & 0 & -1 & 0
\end{matrix}
\right)\left(
\begin{matrix}
1 \\
0 \\
1 \\
0
\end{matrix}
\right) \\
=&-\frac{1}{\sqrt{2}}\left(
\begin{matrix}
0 \\
1 \\
0 \\
-1
\end{matrix}
\right)=-v\left(0,\frac{1}{2}\right) \\
\mc{C}^{-1} \beta u\left(0,-\frac{1}{2}\right)=&-\mc{C}u\left(0,-\frac{1}{2}\right) \\
=&-\frac{1}{\sqrt{2}}\left(
\begin{matrix}
0 & -1 & 0 & 0 \\
1 & 0 & 0 & 0 \\
0 & 0 & 0 & 1 \\
0 & 0 & -1 & 0
\end{matrix}
\right)\left(
\begin{matrix}
0 \\
1 \\
0 \\
1
\end{matrix}
\right) \\
=&-\frac{1}{\sqrt{2}}\left(
\begin{matrix}
-1 \\
0 \\
1 \\
0
\end{matrix}
\right)=-v\left(0,-\frac{1}{2}\right) \\
\mc{C}^{-1}\beta v\left(0,\frac{1}{2}\right)=&+\mc{C} v\left(0,\frac{1}{2}\right) \quad \because (5.5.14),b_v=-1 ,\mc{C}^{-1}=-\mc{C} \\
=&\frac{1}{\sqrt{2}} \left(
\begin{matrix}
0 & -1 & 0 & 0 \\
1 & 0 & 0 & 0 \\
0 & 0 & 0 & 1 \\
0 & 0 & -1 & 0
\end{matrix}
\right)\left(
\begin{matrix}
0 \\
1 \\
0 \\
-1
\end{matrix}
\right) \\
=&\frac{1}{\sqrt{2}}\left(
\begin{matrix}
-1 \\
0 \\
-1 \\
0
\end{matrix}
\right)=-u\left(0,\frac{1}{2}\right) \\
\mc{C}^{-1}\beta v\left(0,-\frac{1}{2}\right)=&+\mc{C} v\left(0,-\frac{1}{2}\right) \\
=&\frac{-1}{\sqrt{2}}\left(
\begin{matrix}
0 & -1 & 0 & 0 \\
1 & 0 & 0 & 0 \\
0 & 0 & 0 & 1 \\
0 & 0 & -1 & 0
\end{matrix}
\right)\left(
\begin{matrix}
1 \\
0 \\
-1 \\
0
\end{matrix}
\right) \\
=&\frac{-1}{\sqrt{2}}\left(
\begin{matrix}
0 \\
1 \\
0 \\
1
\end{matrix}
\right)=-u\left(0,-\frac{1}{2}\right) \\
\therefore \quad \mc{C}^{-1}\beta u(0,\sigma)=&-v(0,\sigma),\quad \mc{C}^{-1}\beta v(0,\sigma)=-u(0,\sigma)
\end{align*}
であるから
\begin{align*}
u^*(\mathbf{p},\sigma)=&\left[\sqrt{\frac{m}{p^0}}D(L(p))u(0,\sigma)\right]^* \\
=&\sqrt{\frac{m}{p^0}}D^*(L(p))u(0,\sigma) \quad \because u^*(0,\sigma)=u(0,\sigma) \\
=&\sqrt{\frac{m}{p^0}}\beta\mc{C} D(L(p))\mc{C}^{-1}\beta u(0,\sigma) \\
=&-\sqrt{\frac{m}{p^0}}\beta\mc{C}D(L(p)) v(0,\sigma) \\
=&-\beta \mc{C} v(\mathbf{p},\sigma) \\
v^*(\mathbf{p},\sigma)=&\left[\sqrt{\frac{m}{p^0}}D(L(p))v(0,\sigma)\right]^* \\
=&\sqrt{\frac{m}{p^0}}D^*(L(p))v(0,\sigma) \quad \because v^*(0,\sigma)=v(0,\sigma) \\
=&\sqrt{\frac{m}{p^0}}\beta\mc{C} D(L(p))\mc{C}^{-1}\beta v(0,\sigma) \\
=&-\sqrt{\frac{m}{p^0}}\beta\mc{C}D(L(p)) u(0,\sigma) \\
=&-\beta \mc{C} u(\mathbf{p},\sigma)
\end{align*}
を得る.したがって,
\begin{align*}
\mc{C}^{-1}\beta=&-\mc{C}\beta \\
=&-(\gamma^2i\gamma^0)(i\gamma^0) \\
=&+(i\gamma^0)(\gamma^2i\gamma^0) \quad \because \gamma^0\gamma^2=-\gamma^2\gamma^0 \\
=&\beta \mc{C}
\end{align*}
であることを使えば
\begin{align*}
u(\mathbf{p},\sigma)=&-\mc{C}^{-1}\beta v^*(\mathbf{p},\sigma) \\
=&-\beta \mc{C} \beta v^*(\mathbf{p},\sigma) \\
v(\mathbf{p},\sigma)=&-\beta \mc{C} \beta u^*(\mathbf{p},\sigma)
\end{align*}
を得る.これを荷電共役で用いれば
\begin{align*}
\mathsf{C} \psi^+(x)\mathsf{C}^{-1}=&\xi^* \sum_\sigma \int \frac{d^3\mathbf{p}}{(2\pi)^{3/2}} u(\mathbf{p},\sigma)e^{ip\cdot x}a^c(\mathbf{p},\sigma) \\
=&-\xi^*\beta \mc{C} \sum_\sigma \int \frac{d^3\mathbf{p}}{(2\pi)^{3/2}} v^*(\mathbf{p},\sigma)e^{ip\cdot x}a^c(\mathbf{p},\sigma) \\
=&-\xi^*\beta \mc{C} \left[ \sum_\sigma \int \frac{d^3\mathbf{p}}{(2\pi)^{3/2}} v(\mathbf{p},\sigma)e^{-ip\cdot x}a^{c\dagger}(\mathbf{p},\sigma)\right]^* \\
=&-\xi^*\beta \mc{C} \psi^{-c*}(x) \\
\mathsf{C} \psi^{-c}_\ell(x)\mathsf{C}^{-1}=&\xi^c \sum_\sigma \int \frac{d^3\mathbf{p}}{(2\pi)^{3/2}} v_\ell(\mathbf{p},\sigma)e^{-ip\cdot x}a^{\dagger}(\mathbf{p},\sigma) \\
=&-\xi^c\beta \mc{C} \sum_\sigma \int \frac{d^3\mathbf{p}}{(2\pi)^{3/2}} u^*(\mathbf{p},\sigma)e^{-ip\cdot x}a^{\dagger}(\mathbf{p},\sigma) \\
=&-\xi^c\beta \mc{C} \left[\sum_\sigma \int \frac{d^3\mathbf{p}}{(2\pi)^{3/2}} u(\mathbf{p},\sigma)e^{ip\cdot x}a(\mathbf{p},\sigma)\right]^* \\
=&-\xi^c\beta \mc{C} \psi^{+*}(x)
\end{align*}
となる.したがって
\begin{align*}
\mathsf{C}\psi(x)\mc{C}^{-1}=-\xi^*\beta \mc{C} \psi^{-c*}(x)-\xi^c\beta \mc{C} \psi^{+*}(x)
\end{align*}
を得る.場が荷電共役のもとで,それと空間的距離で交換する別の場に変換するには,やはり粒子と反粒子の荷電共役パリティが
\begin{align*}
\xi^c=\xi^*
\end{align*}
で関係している必要がある.このとき
\begin{align*}
\mathsf{C}\psi(x)\mc{C}^{-1}=-\xi^*\beta \mc{C} \psi^{*}(x)
\end{align*}
と変換する.(ここで,場のエルミート共役を$\psi^\dagger$ではなく$\psi^*$と呼んでいる.これは,$\mathsf{C}$は演算子であるから,$\mc{C}\psi(x)\mc{C}^{-1}$が4成分ベクトル的には$\psi(x)$と同じく列ベクトルのまま
\begin{align*}
\mathsf{C}\psi(x)\mathsf{C}^{-1}=\mathsf{C}\left(
\begin{matrix}
\psi_1(x) \\
\psi_2(x) \\
\psi_3(x) \\
\psi_4(x)
\end{matrix}
\right)\mathsf{C}^{-1}=\left(
\begin{matrix}
\mathsf{C} \psi_1(x)\mathsf{C}^{-1} \\
\mathsf{C} \psi_2(x)\mathsf{C}^{-1} \\
\mathsf{C} \psi_3(x)\mathsf{C}^{-1} \\
\mathsf{C} \psi_4(x)\mathsf{C}^{-1} \\
\end{matrix}
\right)
\end{align*}
であり,もし$\psi^\dagger$とすると,転置をして行ベクトルのように見えてしまうからである.)

\vskip\baselineskip

我々はこの節の最初から粒子と反粒子を区別してきたが,その二つが実際は同一$a(\mathbf{p},\sigma)=a^c(\mathbf{p},\sigma)$という可能性は排除していない.そのようなスピン$1/2$粒子は\textbf{マヨラナ・フェルミオン}と呼ばれる.このとき,マヨラナ・フェルミオン場は
\begin{align*}
\psi(x)=&\sum_\sigma \int \frac{d^3\mathbf{p}}{(2\pi)^{3/2}} \Bigl[u(\mathbf{p},\sigma)e^{ip\cdot x} a(\mathbf{p},\sigma)+ v(\mathbf{p},\sigma)e^{-ip\cdot x} a^{c\dagger}(\mathbf{p},\sigma)\Bigr] \\
=&\sum_\sigma \int \frac{d^3\mathbf{p}}{(2\pi)^{3/2}} \Bigl[u(\mathbf{p},\sigma)e^{ip\cdot x} a(\mathbf{p},\sigma)+ v(\mathbf{p},\sigma)e^{-ip\cdot x} a^{\dagger}(\mathbf{p},\sigma)\Bigr] \quad \because a(\mathbf{p},\sigma)=a^c(\mathbf{p},\sigma) \\
=&\sum_\sigma \int \frac{d^3\mathbf{p}}{(2\pi)^{3/2}} \Bigl[u(\mathbf{p},\sigma)e^{ip\cdot x} a(\mathbf{p},\sigma)- \beta \mc{C} u^* (\mathbf{p},\sigma)e^{-ip\cdot x} a^{\dagger}(\mathbf{p},\sigma)\Bigr] \quad \because (5.5.44)
\end{align*}
となり,この場全体をエルミート共役をし,さらに$-\beta\mc{C}$をかけると($\beta$と$\mc{C}$はどちらも実行列であり,さらに$\beta\mc{C}\beta\mc{C}=-\mc{C}\beta \beta \mc{C}=-\mc{C}^2=+\bm{1}$に注意)
\begin{align*}
\psi^*(x)=&\sum_\sigma \int \frac{d^3\mathbf{p}}{(2\pi)^{3/2}} \Bigl[u^*(\mathbf{p},\sigma)e^{-ip\cdot x} a^{\dagger}(\mathbf{p},\sigma)- \beta \mc{C} u (\mathbf{p},\sigma)e^{ip\cdot x} a(\mathbf{p},\sigma)\Bigr] \\
-\beta \mc{C}\psi^*(x)=&\sum_\sigma \int \frac{d^3\mathbf{p}}{(2\pi)^{3/2}} \Bigl[-\beta\mc{C}u^*(\mathbf{p},\sigma)e^{-ip\cdot x} a^\dagger(\mathbf{p},\sigma)+ \beta \mc{C} \beta \mc{C} u (\mathbf{p},\sigma)e^{ip\cdot x} a(\mathbf{p},\sigma)\Bigr] \\
=&\sum_\sigma \int \frac{d^3\mathbf{p}}{(2\pi)^{3/2}} \Bigl[-\beta\mc{C}u^*(\mathbf{p},\sigma)e^{-ip\cdot x} a^\dagger(\mathbf{p},\sigma)+ u (\mathbf{p},\sigma)e^{ip\cdot x} a(\mathbf{p},\sigma)\Bigr] \\
=&\sum_\sigma \int \frac{d^3\mathbf{p}}{(2\pi)^{3/2}} \Bigl[-\beta\mc{C}u^*(\mathbf{p},\sigma)e^{-ip\cdot x} a^\dagger(\mathbf{p},\sigma)+ u (\mathbf{p},\sigma)e^{ip\cdot x} a(\mathbf{p},\sigma)\Bigr] \\
=&\psi(x)
\end{align*}
となる.つまりマヨラナ・フェルミオン場は実条件
\begin{align*}
\psi(x)=-\beta \mc{C}\psi^*(x)
\end{align*}
を満たす.マヨラナ・フェルミオンについては空間反転の固有パリティは$\eta=\eta^c$より$\eta^2=\eta\eta^c=-1$,つまり虚数$\eta=\pm i$でなければならない.一方荷電共役パリティは$\xi=\xi^c$で$\xi^2=\xi\xi^c=+1$,つまり実数$\xi=\pm 1$でなければならない.\par
粒子とその反粒子からなる2粒子状態の固有荷電共役位相には,フェルミオンとボゾンで大きな違いがある.そのような2粒子状態は,$\Phi_0$を真空として
\begin{align*}
\Phi:=&\sum_{\sigma,\sigma'}\int d^3\mathbf{p} \int d^3\mathbf{p}' \chi(\mathbf{p},\sigma;\mathbf{p}',\sigma') \Phi_{\mathbf{p},\sigma,n;\mathbf{p}',\sigma',n^c} \\
=&\sum_{\sigma,\sigma'}\int d^3\mathbf{p} \int d^3\mathbf{p}' \chi(\mathbf{p},\sigma;\mathbf{p}',\sigma') a^\dagger(\mathbf{p},\sigma)a^{c\dagger}(\mathbf{p}',\sigma')\Phi_0
\end{align*}
と書ける.ここで$\chi$は二粒子束縛系の波動関数である.荷電共役でこの状態は
\begin{align*}
\mathsf{C}\Phi=&\sum_{\sigma,\sigma'}\int d^3\mathbf{p} \int d^3\mathbf{p}' \chi(\mathbf{p},\sigma;\mathbf{p}',\sigma') \mathsf{C} a^\dagger(\mathbf{p},\sigma)\mathsf{C}^{-1} \mathsf{C} a^{c\dagger}(\mathbf{p}',\sigma')\mathsf{C}^{-1} \mathsf{C}\Phi_0 \\
=&\xi \xi^c \sum_{\sigma,\sigma'}\int d^3\mathbf{p} \int d^3\mathbf{p}' \chi(\mathbf{p},\sigma;\mathbf{p}',\sigma') a^{c\dagger}(\mathbf{p},\sigma)a^{\dagger}(\mathbf{p}',\sigma') \Phi_0 \quad \because \mathsf{C}\Phi_0=\Phi_0
\end{align*}
と変換する.積分と和の変数を入れ替えて,生成演算子の反交換性と(5.5.46)$\xi\xi^c=+1$を用いると
\begin{align*}
=&+ \sum_{\sigma,\sigma'}\int d^3\mathbf{p}' \int d^3\mathbf{p} \chi(\mathbf{p}',\sigma';\mathbf{p},\sigma) a^{c\dagger}(\mathbf{p}',\sigma')a^{\dagger}(\mathbf{p},\sigma) \Phi_0 \\
=&- \sum_{\sigma',\sigma}\int d^3\mathbf{p}' \int d^3\mathbf{p} \chi(\mathbf{p}',\sigma';\mathbf{p},\sigma) a^{\dagger}(\mathbf{p},\sigma) a^{c\dagger}(\mathbf{p}',\sigma') \Phi_0 \\
=&- \sum_{\sigma,\sigma'}\int d^3\mathbf{p} \int d^3\mathbf{p}' \chi(\mathbf{p}',\sigma';\mathbf{p},\sigma) a^{\dagger}(\mathbf{p},\sigma) a^{c\dagger}(\mathbf{p}',\sigma') \Phi_0 \\
=&\mp\sum_{\sigma,\sigma'}\int d^3\mathbf{p} \int d^3\mathbf{p}' \chi(\mathbf{p},\sigma;\mathbf{p}',\sigma') a^{\dagger}(\mathbf{p},\sigma) a^{c\dagger}(\mathbf{p}',\sigma') \Phi_0 \\
=&\mp \Phi
\end{align*}
と書き換えられる.すなわち,ディラック場で記述される粒子とその反粒子からなる状態の固有荷電共役パリティは奇である.それは,状態の波動関数$\chi$が粒子反粒子の運動量とスピンの交換について偶$\chi(\mathbf{p},\sigma;\mathbf{p}',\sigma')=+\chi(\mathbf{p}',\sigma';\mathbf{p},\sigma)$の状態に荷電共役演算子を施すと符号$-1$が,奇$\chi(\mathbf{p},\sigma;\mathbf{p}',\sigma')=-\chi(\mathbf{p}',\sigma';\mathbf{p},\sigma)$の状態に荷電共役演算子を施すと符号$+1$が出る,という意味で「奇」である.この典型的な例は,電子と陽電子の束縛状態であるポジトロニウムである.二つの最低エネルギー状態は,全スピンが$s=0$と$s=1$のほとんど縮退したs波状態$(\ell=0)$の対であり,それぞれパラ・ポジトロニウムおよびオルソ・ポジトロニウムとして知られている.これら二つの状態の波動関数は運動量の入れ替えについて偶であり,スピンの入れ替えについてはパラならば$s=0$なので奇,オルソならば$s=1$なので偶であるから,それぞれ$\mathsf{C}=+1$および$\mathsf{C}=-1$をもつ.(わかりにくければ,3.3節で導入した2粒子系の陽な形式で計算するとよい.実際
\begin{align*}
\mathsf{C}\Phi_{JM}^{\ell s}=&\int d^3 \mathbf{p} \sum_{m,\sigma}C_{\ell s}(JM;m,\sigma)Y_{\ell m}(\hat{\mathbf{p}}) \sum_{\sigma_1,\sigma_2} C_{s_1,s_2}(s\sigma;\sigma_1,\sigma_2) \mathsf{C} \Phi_{\mathbf{p},\sigma_1,n;-\mathbf{p},\sigma_2,n^c} \\
=&\int d^3 \mathbf{p} \sum_{m,\sigma}C_{\ell s}(JM;m,\sigma)Y_{\ell m}(\hat{\mathbf{p}}) \sum_{\sigma_1,\sigma_2} C_{s_1,s_2}(s\sigma;\sigma_1,\sigma_2) \mathsf{C} a^{\dagger}(\mathbf{p},\sigma_1)a^{c\dagger}(-\mathbf{p},\sigma_2) \Phi_{0} \\
=&\xi\xi^c \int d^3 \mathbf{p} \sum_{m,\sigma}C_{\ell s}(JM;m,\sigma)Y_{\ell m}(\hat{\mathbf{p}}) \sum_{\sigma_1,\sigma_2} C_{s_1,s_2}(s\sigma;\sigma_1,\sigma_2) a^{c\dagger}(\mathbf{p},\sigma_1)a^{\dagger}(-\mathbf{p},\sigma_2) \Phi_0 \\
=&-\xi\xi^c \int d^3 \mathbf{p} \sum_{m,\sigma}C_{\ell s}(JM;m,\sigma)Y_{\ell m}(\hat{\mathbf{p}}) \sum_{\sigma_1,\sigma_2} C_{s_1,s_2}(s\sigma;\sigma_1,\sigma_2) a^{\dagger}(-\mathbf{p},\sigma_2) a^{c\dagger}(\mathbf{p},\sigma_1) \Phi_0 \\
=&- \int d^3 \mathbf{p} \sum_{m,\sigma}C_{\ell s}(JM;m,\sigma)Y_{\ell m}(-\hat{\mathbf{p}}) \sum_{\sigma_1,\sigma_2} C_{s_1,s_2}(s\sigma;\sigma_2,\sigma_1) a^{\dagger}(\mathbf{p},\sigma_1) a^{c\dagger}(-\mathbf{p},\sigma_2) \Phi_0 \\
=&-(-1)^{s+\ell-1} \int d^3 \mathbf{p} \sum_{m,\sigma}C_{\ell s}(JM;m,\sigma)Y_{\ell m}(\hat{\mathbf{p}}) \sum_{\sigma_1,\sigma_2} C_{s_1,s_2}(s\sigma;\sigma_1,\sigma_2) a^{\dagger}(\mathbf{p},\sigma_1) a^{c\dagger}(-\mathbf{p},\sigma_2) \Phi_0 \\
=&(-1)^{s+\ell} \Phi_{JM}^{\ell s}
\end{align*}
ここでクレブシュゴルダン係数の性質
\begin{align*}
C_{j'j''}(jm,m_1m_2)=(-1)^{j-j'-j''}C_{j''j'}(jm,m_2m_1)
\end{align*}
で$s_1=s_2=\frac{1}{2}$で用いた.また$Y_{\ell m}(-\hat{\mathbf{p}})=(-1)^{\ell}Y_{\ell m}(\hat{\mathbf{p}})$も使った.$\ell=0$かつ$s=0$のパラ・ポジトロニウムは$\mathsf{C}=+1$であり,$s=1$のオルソ・ポジトロニウムは$\mathsf{C}=-1$になっている.)これらの値はポジトロニウムの崩壊モードで確かめられている.すなわち,パラ・ポジトロニウムは光子の対(光子1つが$\mathsf{C}=-1$をもつことが分かっている.10.1節参照.よって光子の対は$\mathsf{C}=+1$を持つ)に急激に崩壊するが,オルソ・ポジトロニウム$\mathsf{C}=-1$は光子3個またはそれ以上の光子に遥かにゆっくりと崩壊できるだけである.(つまり荷電共役対称性により$\mathsf{C}=+1$の光子二つへ崩壊する遷移が禁止されている\footnote{光子1つへの遷移は禁止されないように思える.しかしこれはオルソ・ポジトロニウムが静止しているならば不可能である.なぜなら光子は必ずゼロでない3元運動量を持つから,運動量保存により全運動量がゼロのポジトロニウムから1つの光子が生まれることは不可能である.3つ以上の光子は,全運動量がちょうど打ち消しあうように光子が生まれていれば可能になる.}.)


\vskip\baselineskip


さて,時間反転に移ろう.(4.2.15)より粒子消滅演算子と反粒子生成演算子の変換性は
\begin{align*}
\mathsf{T} a(\mathbf{p},\sigma)\mathsf{T}^{-}=\zeta^* (-1)^{\frac{1}{2}-\sigma} a(-\mathbf{p},-\sigma) \\
\mathsf{T} a^{c\dagger}(\mathbf{p},\sigma)\mathsf{T}^{-}=\zeta^c (-1)^{\frac{1}{2}-\sigma} a^{c\dagger}(-\mathbf{p},-\sigma)
\end{align*}
を思い出す.すると
\begin{align*}
\mathsf{T} \psi^+_\ell(x)\mathsf{T}^{-1}=&\sum_\sigma \int \frac{d^3\mathbf{p}}{(2\pi)^{3/2}} \mathsf{T} u_\ell(\mathbf{p},\sigma)e^{ip\cdot x} a(\mathbf{p},\sigma) \mathsf{T}^{-1} \\
=&\sum_\sigma \int \frac{d^3\mathbf{p}}{(2\pi)^{3/2}} u^*_\ell(\mathbf{p},\sigma)e^{-ip\cdot x}\mathsf{T} a(\mathbf{p},\sigma) \mathsf{T}^{-1} \\
=&\xi^* \sum_\sigma \int \frac{d^3\mathbf{p}}{(2\pi)^{3/2}}(-1)^{\frac{1}{2}-\sigma} u^*_\ell(\mathbf{p},\sigma)e^{-ip\cdot x}a(-\mathbf{p},-\sigma) \\
=&\xi^* \sum_\sigma \int \frac{d^3\mathbf{p}}{(2\pi)^{3/2}} (-1)^{\frac{1}{2}+\sigma} u^*_\ell(-\mathbf{p},-\sigma)e^{ip\cdot (-\mc{P}x)}a(\mathbf{p},\sigma)\qquad (\mathbf{p}\to -\mathbf{p},\sigma\to -\sigma) \\
\mathsf{T} \psi^{-c}_\ell(x)\mathsf{T}^{-1}=&\sum_\sigma \int \frac{d^3\mathbf{p}}{(2\pi)^{3/2}} \mathsf{T} v_\ell(\mathbf{p},\sigma)e^{-ip\cdot x} a^{c\dagger}(\mathbf{p},\sigma) \mathsf{T}^{-1} \\
=&\sum_\sigma \int \frac{d^3\mathbf{p}}{(2\pi)^{3/2}} v^*_\ell(\mathbf{p},\sigma)e^{ip\cdot x}\mathsf{T} a^{c\dagger}(\mathbf{p},\sigma) \mathsf{T}^{-1} \\
=&\xi^c \sum_\sigma \int \frac{d^3\mathbf{p}}{(2\pi)^{3/2}} (-1)^{\frac{1}{2}-\sigma} v^*_\ell(\mathbf{p},\sigma)e^{-ip\cdot x}a^{c\dagger}(-\mathbf{p},-\sigma) \\
=&\xi^c \sum_\sigma \int \frac{d^3\mathbf{p}}{(2\pi)^{3/2}} (-1)^{\frac{1}{2}+\sigma}v^*_\ell(-\mathbf{p},-\sigma)e^{-ip\cdot (-\mc{P}x)}a^{c\dagger}(\mathbf{p},\sigma)\qquad (\mathbf{p}\to -\mathbf{p},\sigma\to -\sigma)
\end{align*}
を得る.ここでそれぞれの最後の行では,$\sigma=+\frac{1}{2},-\frac{1}{2}$についての和を反転させて$\sigma'=-\sigma=-\frac{1}{2},+\frac{1}{2}$についての和に変えた.これをさらに計算するためには,$u^*_\ell(-\mathbf{p},-\sigma),v^*_\ell(-\mathbf{p},-\sigma)$をそれぞれ$u_\ell(\mathbf{p},\sigma),v_\ell(\mathbf{p},\sigma)$で表した式が必要である.そのためには$D(L(p))^*$についての以前の結果$D(L(p))^*=\beta \mc{C} D(L(p))\mc{C}^{-1}\beta$とともに,$\mc{K}_i$が$\beta$と反交換し$\gamma_5$と交換することを用いて((5.5.12)を導いたときの$D(L(-\mathbf{p}))=\beta D(L(\mathbf{p}))\beta$を思い出して)
\begin{align*}
D^*(L(-\mathbf{p}))=&\beta D(L(\mathbf{p}))^*\beta \\
=&\gamma_5 \gamma_5 \beta D(L(\mathbf{p}))^*\beta \quad \because \gamma^2_5=\bm{1}\\
=&\gamma_5 \beta D(L(\mathbf{p}))^*\beta \gamma_5 \quad \because (5.4.23) \\
=&\gamma_5 \beta \Bigl[\beta\mc{C} D(L(\mathbf{p}))\mc{C}^{-1}\beta \Bigr]\beta \gamma_5 \\
=&\gamma_5 \mc{C} D(L(\mathbf{p}))\mc{C}^{-1} \gamma_5
\end{align*}
と書いておく.(5.4.36)と(5.5.35)(5.5.36)より
\begin{align*}
\gamma_5 \mc{C}^{-1} u\left(0,\frac{1}{2}\right)=&-\gamma_5 \mc{C} u\left(0,\frac{1}{2}\right) \quad \because \mc{C}^{-1}=-\mc{C} \\
=&-\frac{1}{\sqrt{2}}\left(
\begin{matrix}
1 & 0 & 0 & 0 \\
0 & 1 & 0 & 0 \\
0 & 0 & -1 & 0 \\
0 & 0 & 0 & -1
\end{matrix}
\right)\left(
\begin{matrix}
0 & -1 & 0 & 0 \\
1 & 0 & 0 & 0 \\
0 & 0 & 0 & 1 \\
0 & 0 & -1 & 0
\end{matrix}
\right)\left(
\begin{matrix}
1 \\
0 \\
1 \\
0
\end{matrix}
\right) \\
=&-\frac{1}{\sqrt{2}}\left(
\begin{matrix}
1 & 0 & 0 & 0 \\
0 & 1 & 0 & 0 \\
0 & 0 & -1 & 0 \\
0 & 0 & 0 & -1
\end{matrix}
\right)\left(
\begin{matrix}
0 \\
1 \\
0 \\
-1
\end{matrix}
\right)\\
=&-\frac{1}{\sqrt{2}}\left(
\begin{matrix}
0 \\
1 \\
0 \\
1
\end{matrix}
\right) =-u\left(0,-\frac{1}{2}\right)=(-1)^{\frac{1}{2}+\frac{1}{2}} u\left(0,-\frac{1}{2}\right) \\
\gamma_5 \mc{C}^{-1} u\left(0,-\frac{1}{2}\right)=&-\gamma_5\mc{C}u\left(0,-\frac{1}{2}\right) \\
=&-\frac{1}{\sqrt{2}}\left(
\begin{matrix}
1 & 0 & 0 & 0 \\
0 & 1 & 0 & 0 \\
0 & 0 & -1 & 0 \\
0 & 0 & 0 & -1
\end{matrix}
\right)\left(
\begin{matrix}
0 & -1 & 0 & 0 \\
1 & 0 & 0 & 0 \\
0 & 0 & 0 & 1 \\
0 & 0 & -1 & 0
\end{matrix}
\right)\left(
\begin{matrix}
0 \\
1 \\
0 \\
1
\end{matrix}
\right) \\
=&-\frac{1}{\sqrt{2}}\left(
\begin{matrix}
1 & 0 & 0 & 0 \\
0 & 1 & 0 & 0 \\
0 & 0 & -1 & 0 \\
0 & 0 & 0 & -1
\end{matrix}
\right)\left(
\begin{matrix}
-1 \\
0 \\
1 \\
0
\end{matrix}
\right) \\
=&-\frac{1}{\sqrt{2}}\left(
\begin{matrix}
-1 \\
0 \\
-1 \\
0
\end{matrix}
\right) =u\left(0,\frac{1}{2}\right)=(-1)^{\frac{1}{2}+\left(-\frac{1}{2}\right)}u\left(0,\frac{1}{2}\right) \\
\gamma_5 \mc{C}^{-1} v\left(0,\frac{1}{2}\right)=&-\gamma_5 \mc{C} v\left(0,\frac{1}{2}\right) \\
=&-\frac{1}{\sqrt{2}} \left(
\begin{matrix}
1 & 0 & 0 & 0 \\
0 & 1 & 0 & 0 \\
0 & 0 & -1 & 0 \\
0 & 0 & 0 & -1
\end{matrix}
\right)\left(
\begin{matrix}
0 & -1 & 0 & 0 \\
1 & 0 & 0 & 0 \\
0 & 0 & 0 & 1 \\
0 & 0 & -1 & 0
\end{matrix}
\right)\left(
\begin{matrix}
0 \\
1 \\
0 \\
-1
\end{matrix}
\right) \\
=&-\frac{1}{\sqrt{2}}\left(
\begin{matrix}
1 & 0 & 0 & 0 \\
0 & 1 & 0 & 0 \\
0 & 0 & -1 & 0 \\
0 & 0 & 0 & -1
\end{matrix}
\right)\left(
\begin{matrix}
-1 \\
0 \\
-1 \\
0
\end{matrix}
\right) \\
=&-\frac{1}{\sqrt{2}}\left(
\begin{matrix}
-1 \\
0 \\
1 \\
0
\end{matrix}
\right) =-v\left(0,-\frac{1}{2}\right)=(-1)^{\frac{1}{2}+\frac{1}{2}}v\left(0,-\frac{1}{2}\right) \\
\gamma_5 \mc{C}^{-1} v\left(0,-\frac{1}{2}\right)=&-\gamma_5 \mc{C} v\left(0,-\frac{1}{2}\right) \\
=&-\frac{-1}{\sqrt{2}}\left(
\begin{matrix}
1 & 0 & 0 & 0 \\
0 & 1 & 0 & 0 \\
0 & 0 & -1 & 0 \\
0 & 0 & 0 & -1
\end{matrix}
\right)\left(
\begin{matrix}
0 & -1 & 0 & 0 \\
1 & 0 & 0 & 0 \\
0 & 0 & 0 & 1 \\
0 & 0 & -1 & 0
\end{matrix}
\right)\left(
\begin{matrix}
1 \\
0 \\
-1 \\
0
\end{matrix}
\right) \\
=&\frac{+1}{\sqrt{2}}\left(
\begin{matrix}
1 & 0 & 0 & 0 \\
0 & 1 & 0 & 0 \\
0 & 0 & -1 & 0 \\
0 & 0 & 0 & -1
\end{matrix}
\right)\left(
\begin{matrix}
0 \\
1 \\
0 \\
1
\end{matrix}
\right)\\
=&\frac{1}{\sqrt{2}}\left(
\begin{matrix}
0 \\
1 \\
0 \\
-1
\end{matrix}
\right)=v\left(0,\frac{1}{2}\right)=(-1)^{\frac{1}{2}+\left(-\frac{1}{2}\right)}v\left(0,\frac{1}{2}\right) \\
\therefore \quad \gamma_5 \mc{C}^{-1}u(0,\sigma)=&(-1)^{\frac{1}{2}+\sigma}u(0,-\sigma),\quad \gamma_5 \mc{C}^{-1} v(0,\sigma)=(-1)^{\frac{1}{2}+\sigma}v(0,-\sigma)
\end{align*}
よって
\begin{align*}
u(0,\sigma)=&(-1)^{\frac{1}{2}+\sigma}\mc{C}\gamma_5 u(0,-\sigma) \\
=&(-1)^{\frac{1}{2}+\sigma}\gamma_5\mc{C} u(0,-\sigma) \\
=&-(-1)^{\frac{1}{2}+\sigma}\gamma_5\mc{C}^{-1} u(0,-\sigma) \\
v(0,\sigma)=&(-1)^{\frac{1}{2}+\sigma} \mc{C}\gamma_5 v(0,-\sigma) \\
=&(-1)^{\frac{1}{2}+\sigma} \gamma_5\mc{C} v(0,-\sigma) \\
=&-(-1)^{\frac{1}{2}+\sigma}\gamma_5 \mc{C}^{-1} v(0,-\sigma) \\
\therefore \quad (-1)^{\frac{1}{2}+\sigma}\gamma_5\mc{C}^{-1} u(0,-\sigma)=&-u(0,\sigma) ,\quad (-1)^{\frac{1}{2}+\sigma}\gamma_5 \mc{C}^{-1} v(0,-\sigma)=-v(0,\sigma)
\end{align*}
を得る.($\mc{C}$が二つのガンマ行列の積だから$\gamma_5 \mc{C}=\mc{C} \gamma_5$である.)これを用いて
\begin{align*}
(-1)^{\frac{1}{2}+\sigma}u^*(-\mathbf{p},-\sigma)=&(-1)^{\frac{1}{2}+\sigma}\left[\sqrt{\frac{m}{p^0}}D(L(-\mathbf{p}))u(0,-\sigma)\right]^* \\
=&\sqrt{\frac{m}{p^0}}(-1)^{\frac{1}{2}+\sigma}D^*(L(-\mathbf{p}))u(0,-\sigma) \quad \because u^*(0,\sigma)=u(0,\sigma) \\
=&\sqrt{\frac{m}{p^0}}(-1)^{\frac{1}{2}+\sigma}\gamma_5 \mc{C} D(L(\mathbf{p}))\mc{C}^{-1}\gamma_5 u(0,-\sigma) \\
=&-\sqrt{\frac{m}{p^0}}\gamma_5 \mc{C} D(L(\mathbf{p})) u(0,\sigma) \\
=&-\gamma_5 \mc{C} u(\mathbf{p},\sigma) \\
(-1)^{\frac{1}{2}+\sigma}v^*(-\mathbf{p},-\sigma)=&(-1)^{\frac{1}{2}+\sigma}\Bigl[\sqrt{\frac{m}{p^0}}D(L(-\mathbf{p}))v(0,-\sigma)\Bigr]^* \\
=&\sqrt{\frac{m}{p^0}}(-1)^{\frac{1}{2}+\sigma}D^*(L(-\mathbf{p}))v(0,-\sigma) \quad \because v^*(0,\sigma)=v(0,\sigma) \\
=&-\sqrt{\frac{m}{p^0}}(-1)^{\frac{1}{2}+\sigma}\gamma_5 \mc{C} D(L(\mathbf{p}))\mc{C}^{-1}\gamma_5 v(0,-\sigma) \\
=&-\sqrt{\frac{m}{p^0}}\gamma_5 \mc{C} D(L(\mathbf{p})) v(0,\sigma) \\
=&-\gamma_5 \mc{C} v(\mathbf{p},\sigma)
\end{align*}
を得る.したがって時間反転での消滅・生成場の変換が
\begin{align*}
\mathsf{T} \psi^+(x)\mathsf{T}^{-1}=&\xi^* \sum_\sigma \int \frac{d^3\mathbf{p}}{(2\pi)^{3/2}} (-1)^{\frac{1}{2}+\sigma} u^*(-\mathbf{p},-\sigma)e^{ip\cdot (-\mc{P}x)}a(\mathbf{p},\sigma) \\
=&-\xi^* \gamma_5 \mc{C} \sum_\sigma \int \frac{d^3\mathbf{p}}{(2\pi)^{3/2}} u(\mathbf{p},\sigma)e^{ip\cdot (-\mc{P}x)}a(\mathbf{p},\sigma) \\
=&-\xi^*\gamma_5 \mc{C} \psi^+(-\mc{P}x)=-\xi^*\gamma_5 \mc{C} \psi^+(\mc{T}x) \\
\mathsf{T} \psi^{-c}(x)\mathsf{T}^{-1}=&\xi^c \sum_\sigma \int \frac{d^3\mathbf{p}}{(2\pi)^{3/2}} (-1)^{\frac{1}{2}+\sigma}v^*(-\mathbf{p},-\sigma)e^{-ip\cdot (-\mc{P}x)}a^{c\dagger}(\mathbf{p},\sigma) \\
=&-\xi^c \gamma_5 \mc{C} \sum_\sigma \int \frac{d^3\mathbf{p}}{(2\pi)^{3/2}} v(\mathbf{p},\sigma)e^{-ip\cdot (-\mc{P}x)}a^{c\dagger}(\mathbf{p},\sigma) \\
=&-\xi^*\gamma_5 \mc{C} \psi^{-c}(-\mc{P}x)=-\xi^*\gamma_5 \mc{C} \psi^{-c}(\mc{T}x)
\end{align*}
となる.これよりディラック場は時間反転のもとで
\begin{align*}
\mathsf{T} \psi(x) \mathsf{T}^{-1}= -\xi^*\gamma_5 \mc{C} \psi^+(\mc{T}x)-\xi^*\gamma_5 \mc{C} \psi^{-c}(\mc{T}x)
\end{align*}
と変換する.これが$\psi(\mc{T}x)$に比例するためには
\begin{align*}
\xi^c=\xi^*
\end{align*}
でなければならない.このとき
\begin{align*}
\mathsf{T} \psi(x)\mathsf{T}^{-1}=-\xi^*\gamma_5 \mc{C} \psi(\mc{T}x)
\end{align*}
となることがわかる.

\vskip\baselineskip


ディラック場の変換性が分かった.今度は,ディラック場とその共役場から,スカラー相互作用密度がどのようにして構成できるかを考えよう.すでに(5.4.32)で述べたようにスピノル表現はユニタリーではない.つまり$D^\dagger(\Lambda)D(\Lambda)\neq \bm{1}$である.したがって$\psi^\dagger \psi$はスカラーにならない.実際
\begin{align*}
U_0(\Lambda)\Bigl[ \psi^\dagger(x)\psi(x)\Bigr] U_0(\Lambda)=&\left[U_0(\Lambda) \psi (x)U_0^{-1}(\Lambda)\right] U_0(\Lambda) \psi(x) U_0(\Lambda^{-1})\\
=&\left[D(\Lambda^{-1})\psi(\Lambda x)\right]^\dagger D(\Lambda^{-1})\psi(\Lambda x) \\
=&\psi^\dagger (\Lambda x) D^\dagger (\Lambda^{-1}) D(\Lambda^{-1}) \psi(\Lambda x) \\
\neq& \psi^\dagger (\Lambda x)\psi(\Lambda x)
\end{align*}
である.この問題に対処するためには,新しい種類の共役場
\begin{align*}
\bar{\psi}:=\psi^\dagger \beta
\end{align*}
を定義するのが便利である.これはディラック共役場と呼ばれる.擬ユニタリー条件を用いると,$\bar{\psi}$を用いて構成したフェルミオンの双1次形式$\bar{\psi}M\psi$は($M_{\ell\bar{\ell}}$は4成分スピノル添え字をもつ任意の$4\times 4$行列),ローレンツ変換のもとで
\begin{align*}
U_0(\Lambda)\Bigl[\bar{\psi}(x)M\psi(x)\Bigr] U^{-1}_0(\lambda)=&U_0 (\Lambda) \psi^\dagger(x) U_0^{-1}(\Lambda) \beta M U_0(\Lambda) \psi(x) U_0^{-1}(\Lambda) \\
=&\psi^\dagger(\Lambda x) D^\dagger(\Lambda^{-1})\beta M D(\Lambda^{-1})\psi(\Lambda x) \\
=&\psi^\dagger(\Lambda x) \beta D^{-1}(\Lambda^{-1}) M D(\Lambda^{-1})\psi(\Lambda x) \quad \because (5.4.32) \\
=&\bar{\psi}(\Lambda x) D^{-1}(\Lambda^{-1}) M D(\Lambda^{-1})\psi(\Lambda x) \\
=&\bar{\psi}(\Lambda x) \Bigl[ D(\Lambda) M D^{-1}(\Lambda)\Bigr] \psi(\Lambda x)
\end{align*}
と変換することがわかる.(ここで$U_0(\Lambda)$は演算子であり,演算子でなくただの行列である$\beta$や$M$は透過して交換することに注意.)また,空間反転のもとで
\begin{align*}
\mathsf{P}\Bigl[\bar{\psi}(x)M\psi(x)\Bigr] \mathsf{P}^{-1}=&\mathsf{P}\psi^\dagger(x)\mathsf{P}^{-1} \beta M \mathsf{P} \psi(x)\mathsf{P} \\
=&[\eta^*\beta \psi(\mc{P}x)]^\dagger \beta M \beta \eta^*\psi(\mc{P}x) \\
=&|\eta|^2 \psi^\dagger(\mc{P}x) \beta \beta M \beta \psi(\mc{P}x) \\
=&\bar{\psi}(\mc{P}x) \Bigl[\beta M \beta \Bigr]\psi(\mc{P}x)
\end{align*}
と変換することがわかる.任意の$4\times 4$行列$M$は基底$\bm{1},\gamma^\mu,\mc{S}^{\mu\nu},\gamma_5 \gamma^\mu,\gamma_5$で一意的に展開できるのだったから,それら基底の変換性のみを見れば任意の双1次形式の変換性がわかる.前節での結果をもう一度書き下すと
\begin{align*}
D(\Lambda) \bm{1} D^{-1}(\Lambda)=&\bm{1} \\
\beta \bm{1} \beta =& \bm{1}
\end{align*}
より$\bar{\psi}\bm{1}\psi=\bar{\psi}\psi$はスカラーとして変換される.
\begin{align*}
D(\Lambda) \gamma^\mu D^{-1}(\Lambda)=& \tensor{\Lambda}{_\nu^\mu} \gamma^\nu \\
\beta \gamma^\mu \beta=&\tensor{\mc{P}}{_\nu^\mu}\gamma^\mu
\end{align*}
より$\bar{\psi}\gamma^\mu \psi$は4元ベクトルとして変換される.
\begin{align*}
D(\Lambda) \mc{S}^{\mu\nu} D^{-1}(\Lambda)=& \tensor{\Lambda}{_\rho^\mu}\tensor{\Lambda}{_\sigma^\nu} \mc{S}^{\rho\sigma} \\
\beta \mc{S}^{\mu\nu} \beta=&\tensor{\mc{P}}{_\rho^\mu}\tensor{\mc{P}}{_\sigma^\mu}\mc{S}^{\rho\sigma}
\end{align*}
より$\bar{\psi}\mc{S}^{\mu\nu}\psi$は反対称テンソルとして変換される.
\begin{align*}
D(\Lambda) \gamma_5 \gamma^\mu D^{-1}(\Lambda)=& \tensor{\Lambda}{_\nu^\mu} \gamma_5 \gamma^\nu \\
\beta \gamma_5 \gamma^\mu \beta=&-\tensor{\mc{P}}{_\nu^\mu}\gamma_5 \gamma^\mu \\
=&(\mathrm{det}\mc{P})\tensor{\mc{P}}{_\nu^\mu}\gamma_5 \gamma^\nu
\end{align*}
より$\bar{\psi}\gamma_5 \gamma^\mu\psi$は軸性ベクトルとして変換される.
\begin{align*}
D(\Lambda) \gamma_5 D^{-1}(\Lambda)=&\gamma_5 \\
\beta \gamma_5 \beta =& -\gamma_5 \\
=&(\mathrm{det}\mc{P}) \gamma_5
\end{align*}
より$\bar{\psi}\gamma_5\psi$は擬スカラーとして変換される.($\beta$はパリティの変換に対応していることを思い出そう.)これらの結果は,双1次形式を形成する二つのフェルミオン場が異なる粒子の種類に属するときにも成立する.ただし,この場合には空間反転により
\begin{align*}
\mathsf{P}\Bigl[\bar{\psi}_1(x)M\psi_2(x)\Bigr] \mathsf{P}^{-1}=&\mathsf{P}\psi_1^\dagger(x)\mathsf{P}^{-1} \beta M \mathsf{P} \psi_2(x)\mathsf{P} \\
=&[\eta_1^*\beta \psi_1(\mc{P}x)]^\dagger \beta M \beta \eta_2^*\psi_2(\mc{P}x) \\
=&\eta_1 \eta_2^* \psi_1^\dagger(\mc{P}x) \beta \beta M \beta \psi_2(\mc{P}x) \\
=&(\eta_1/\eta_2)\bar{\psi}_1(\mc{P}x) \Bigl[\beta M \beta \Bigr]\psi_2(\mc{P}x) \quad \because \eta_2^*=1/\eta_2
\end{align*}
固有パリティの比が出る.\par

\vskip\baselineskip


これらの双1次形式の荷電共役変換性も調べる.(5.5.47)を用いると
\begin{align*}
\mathsf{C} \Bigl[\bar{\psi}(x) M \psi(x)\Bigr]\mathsf{C}^{-1}=&\mathsf{C} \psi^\dagger(x) \mathsf{C}^{-1} \beta M \mathsf{C} \psi(x) \mathsf{C}^{-1} \\
=&[-\xi^* \beta \mc{C} \psi^*(x)]^\dagger \beta M [-\xi^* \beta \mc{C} \psi^*(x)] \\
=&+|\xi|^2 (\beta \mc{C}\psi(x))^T \beta M (\beta \mc{C} \psi^*(x)) \\
=&+\psi^T(x)(\beta \mc{C})^T \beta M (\beta \mc{C}) \psi^*(x) \\
=&\sum_{\ell\ell'}\psi_{\ell}(x)\Bigl[(\beta\mc{C})^T \beta M (\beta \mc{C})\Bigr]_{\ell\ell'}\psi^*_{\ell'}(x) \\
=&-\sum_{\ell\ell'}\psi^*_{\ell'}(x)\Bigl[(\beta\mc{C})^T \beta M (\beta \mc{C})\Bigr]_{\ell\ell'}\psi_{\ell}(x) \\
=&-\sum_{\ell\ell'}\psi^*_{\ell'}(x)\Bigl[(\beta\mc{C})^T \beta M (\beta \mc{C})\Bigr]^T_{\ell'\ell}\psi_{\ell}(x) \\
=&-\psi^\dagger (x)\Bigl[(\beta \mc{C})^T \beta M (\beta \mc{C})\Bigr]^T \psi(x) \\
=&-\psi^\dagger (x)\Bigl[\mc{C}^T \beta M^T \beta \beta \mc{C}^T\Bigr] \psi(x) \\
=&\psi^\dagger (x)\beta \mc{C}^{-1} M^T \mc{C} \psi(x) \quad \because \mc{C}^T=\mc{C}^{-1}=-\mc{C},\mc{C}\beta =-\beta \mc{C} \\
=&\bar{\psi}(x) \mc{C}^{-1} M^T \mc{C} \psi(x)
\end{align*}
となる.ここで,6行目から7行目にかけては,反交換関係(5.5.39)より
\begin{align*}
[\psi_\ell(x),\psi^*_{\ell'}(x)]_+=&0 \qquad \because \Delta(0)=\Delta_+(0)-\Delta_+(0)=0 \\
\therefore \quad \psi_\ell(x)\psi_{\ell'}^*(x)=&-\psi_{\ell'}^*(x)\psi_\ell(x)
\end{align*}
となり,$\psi_\ell(x)$と$\psi_{\ell'}^*(x)$が反交換することを用いた.($\psi_\ell(x)$は行列ではなく4成分ベクトルなので$\dagger$と$*$の違いはないことに注意.間にある行列は成分表示すればただの数なので,順番を入れ替えてもよい.)$M$はやはり$\bm{1},\gamma^\mu,\mc{S}^{\mu\nu},\gamma_5 \gamma^\mu,\gamma_5$で展開できるので,それぞれの変換性を見てやれば(5.4.35)(5.4.37)(5.4.39)(5.4.38)より
\begin{align*}
\mc{C}^{-1} \bm{1}^T \mc{C}=&+\bm{1} \\
\mc{C}^{-1} (\gamma^\mu)^T \mc{C}=&-\gamma^\mu \\
\mc{C}^{-1} (\mc{S}^{\mu\nu})^T \mc{C}=&-\mc{S}^{\mu\nu} \\
\mc{C}^{-1} (\gamma_5 \gamma^\mu)^T \mc{C}=&+\gamma_5 \gamma^\mu \\
\mc{C}^{-1} (\gamma_5 )^T \mc{C}=&+\gamma_5
\end{align*}
であるから,行列$\bm{1},\gamma_5 \gamma^\mu,\gamma_5$に対しては$+1$が出る.
\begin{align*}
\mathsf{C} \Bigl[\bar{\psi}(x) \psi(x)\Bigr]\mathsf{C}^{-1}=&+\bar{\psi}(x)\psi(x) \\
\mathsf{C} \Bigl[\bar{\psi}(x) \gamma_5 \gamma^\mu \psi(x)\Bigr]\mathsf{C}^{-1}=&+\bar{\psi}(x)\gamma_5 \gamma^\mu \psi(x) \\
\mathsf{C} \Bigl[\bar{\psi}(x) \gamma_5 \psi(x)\Bigr]\mathsf{C}^{-1}=&+\bar{\psi}(x)\gamma_5 \psi(x)
\end{align*}
一方,$\gamma^\mu,\mc{S}^{\mu\nu}$に対しては$-1$が出る.
\begin{align*}
\mathsf{C} \Bigl[\bar{\psi}(x) \gamma^\mu \psi(x)\Bigr]\mathsf{C}^{-1}=&-\bar{\psi}(x) \gamma^\mu \psi(x) \\
\mathsf{C} \Bigl[\bar{\psi}(x) \mc{S}^{\mu\nu} \psi(x)\Bigr]\mathsf{C}^{-1}=&-\bar{\psi}(x)\mc{S}^{\mu\nu} \psi(x)
\end{align*}
よってカレント$\bar{\psi}M\psi$と相互作用するボゾン場$\phi,v_\mu,g_{\mu\nu},u_\mu,\varphi$
\begin{align*}
\mc{L} \propto \bar{\psi}\psi \phi, \quad \bar{\psi} \gamma^\mu \psi v_\mu ,\quad \bar{\psi} \mc{S}^{\mu\nu}\psi g_{\mu\nu} ,\quad \bar{\psi}\gamma_5 \gamma^\mu \psi u_\mu ,\quad \bar{\psi} \gamma_5 \psi \varphi
\end{align*}
は,スカラー,擬スカラー,軸性ベクトルに結合する$\phi,\varphi,u_\mu$については$\mathsf{C}=+1$を,またベクトルと反対称テンソルに結合する$v_\mu,g_{\mu\nu}$については$\mathsf{C}=-1$を持たねばならない.$\mathsf{C}$対称性を保存する強い相互作用および電磁相互作用を考えると,$\bar{\psi}\gamma_5 \psi$に結合する擬スカラー粒子である$\pi^0$が$\mathsf{C}=+1$をもち,光子が$\mathsf{C}=-1$を持つことがここからわかる.


\newpage


\subsection{斉次ローレンツ群の一般的な既約表現}
スカラー場,ベクトル場,ディラック場という特別な場合から,斉次ローレンツ群の一般的な既約表現にしたがって変換する場の場合に議論を一般化する.全ての場は,これらの既約な場の直和として構成できる.

\vskip\baselineskip


リー代数についての数学的な定義を整理しておこう.\par
リー代数とは,次を満たす$\mathfrak{g}$をリー代数と呼ぶ.\\
1.$\mathfrak{g}$は体$\mathbb{K}$上のベクトル空間である.\\
2.$\mathfrak{g}$上にリー括弧積と呼ばれる双線形写像$[\, , \, ]_{\mathfrak{g}}:\mathfrak{g}\times \mathfrak{g}\to \mathfrak{g}$が定義されており,以下の性質を満たす.\par
(a)任意の$a\in \mathfrak{g}$に対して$[a,a]_{\mathfrak{g}}=0$である.\par
(b)$a,b,c \in \mathfrak{g}$に対して,ヤコビ恒等式
\begin{align*}
[a,[b,c]_{\mathfrak{g}}]_{\mathfrak{g}}+[b,[c,a]_{\mathfrak{g}}]_{\mathfrak{g}}+[c,[a,b]_{\mathfrak{g}}]_{\mathfrak{g}}=0
\end{align*}
がなりたつ.\par
リー代数$\mathfrak{g}$はベクトル空間であり,したがって結合律を満たす積$ab$の存在は仮定されていない.したがってリー括弧積は通常の意味での交換子ではない.

\vskip\baselineskip

次に,リー代数の表現を定義しよう.\par
次を満たす組$(\rho,V)$をリー代数$\mathfrak{g}$の(線形)表現という.\\
1.$V$は複素体$\mathbb{C}$(体$\mathbb{K}$が実数体$\mathbb{R}$ならば$\mathbb{R}$も可)上のベクトル空間である.\\
2.$V$上の自己準同型線形写像の全体がなすベクトル空間を$\mathfrak{gl}(V):=\mathrm{End}(V)$とするとき,写像$\rho:\mathfrak{g}\to \mathfrak{gl}(V)$が以下の性質を満たす.\par
(a)$\rho:\mathfrak{g}\to \mathfrak{gl}(V)$はベクトル空間としての準同型写像
\begin{align*}
\rho(\alpha a+\beta b)=\alpha \rho(a)+\beta \rho(b) \quad (\forall a,b \in \mathfrak{g} ,\forall \alpha,\beta \in \mathbb{K})
\end{align*}
である.\par
(b)$\rho:\mathfrak{g}\to \mathfrak{gl}(V)$は$\mathfrak{g}$上のリー括弧積$[ \, , \, ]_{\mathfrak{g}}$を合成に関する交換子$[ \, , \, ]_{\circ}$に移す.
\begin{align*}
\rho([a,b]_{\mathfrak{g}})=[\rho(a),\rho(b)]_{\circ} := \rho(a)\circ \rho(b)-\rho(b)\circ \rho(a)
\end{align*}

\vskip\baselineskip


$\mathrm{dim}V=n$とし,$\{e_{(i)}\}_{i=1}^n$を$V$の基底とする.各$a\in \mathfrak{g}$に対する$\rho(a):V\to V$は自己同型線形写像であるから,$\rho(a)(e_{(j)})\in V$を基底で一意的に展開することができる.
\begin{align*}
\rho(a)(e_{(i)})=\sum_{j=1}^n[M(a)]_{ji} e_{(j)}
\end{align*}
$[M(a)]_{ij}$は$n\times n$行列であり,基底の線形独立性より$a\in \mathfrak{g}$が定められれば一意的に定まる行列である.(これを線形写像の表現行列と呼ぶ.)二つの線形写像の合成の結果は,行列の積として表される.
\begin{align*}
(\rho(a)\circ \rho(b))(e_{(i)})=&\rho(a) \Bigl(\rho(b)(e_{(i)})\Bigr) \\
=&\rho(a)\left(\sum_{j=1}^n[M(b)]_{ji} e_{(j)}\right) \\
=&\sum_{j=1}^n[M(b)]_{ji} \rho(a)(e_{(j)}) \\
=&\sum_{j=1}^n[M(b)]_{ji} \sum_{k=1}^n[M(a)]_{kj}e_{(k)} \\
=&\sum_{j=1}^n(M(a)M(b))_{ji} e_{(j)}
\end{align*}
したがって,リー括弧積は
\begin{align*}
\rho([a,b]_{\mathfrak{g}})(e_{(i)})=&(\rho(a)\circ \rho(b)-\rho(b)\circ \rho(a)) (e_{(i)}) \\
=&(\rho(a)\circ \rho(b)) (e_{(i)})-(\rho(b)\circ \rho(a)) (e_{(i)}) \\
=&\sum_{j=1}^n(M(a)M(b)-M(b)M(a))_{ji} e_{(j)} \\
=&\sum_{j=1}^n\Bigl([M(a),M(b)]\Bigr)_{ji} e_{(j)}
\end{align*}
表現とベクトル空間の基底を定めることで,抽象的なリー括弧積$[a,b]_{\mathfrak{g}}$を,行列の積に関する交換子$[M(a),M(b)]$として表現することが可能になった.\par
我々が調べていた行列$\tensor{(\mc{J}^{\mu\nu})}{^\rho_\sigma},(\mc{S}^{\mu\nu})_{\ell\bar{\ell}}$は,この意味で代数そのものではなく行列による表現であり,その根底には抽象的なリー代数が存在している.


\vskip\baselineskip

以上で,リー代数の定義が完了したから,ローレンツ代数について説明する.\par
ローレンツ代数$\mathfrak{so}(3,1)$とは,$J^{\mu\nu}=-J^{\nu\mu}$を満たす基底$\{J^{\mu\nu}\}_{0\leq \mu < \nu \leq 3}$により張られる6次元$\mathbb{R}$係数実ベクトル空間
\begin{align*}
\mathfrak{so}(3,1)=\left\{ \frac{1}{2}\sum_{\mu,\nu=0}^3\omega_{\mu\nu} J^{\mu\nu} \middle|\omega_{\mu\nu}=-\omega_{\nu\mu}, \omega_{\mu\nu} \in \mathbb{R} (0\leq \mu<\nu\leq 3) \right\}
\end{align*}
で,リー括弧積は
\begin{align*}
i[J^{\mu\nu},J^{\rho\sigma}]_{\mathfrak{so}(3,1)}=\eta^{\nu\rho}J^{\mu\sigma}-\eta^{\mu\rho}J^{\nu\sigma}-\eta^{\sigma\mu}J^{\rho\nu}+\eta^{\sigma\nu}J^{\rho\mu}
\end{align*}
で与えられる.$\mathfrak{so}(3,1)$の新しい基底$\{J_i,K_i\}_{i=1,2,3}$を
\begin{align*}
J_i:=\frac{1}{2}\epsilon_{ijk}J_{jk}=(J_{23},J_{31},J_{12}) ,\quad K_i:=J_{i0}=(J_{10},J_{20},J_{30})
\end{align*}
で定めると,任意の$\mathfrak{so}(3,1)$の元は
\begin{align*}
\frac{1}{2}\omega_{\mu\nu} J^{\mu\nu}=\theta_i J_i -\omega_i K_i
\end{align*}
と書きなおすことができる.ここで$\theta_i:=\frac{1}{2}\epsilon_{ijk}\omega_{ij},\omega_i:=\omega_{i0}$である.2.4節で導出したのと全く同様の計算により,この基底に対するリー括弧積は
\begin{align*}
[J_i,J_j]_{\mathfrak{so}(3,1)}=&i\epsilon_{ijk}J_k \\
[J_i,K_j]_{\mathfrak{so}(3,1)}=&i\epsilon_{ijk}K_k \\
[K_i,K_j]_{\mathfrak{so}(3,1)}=&-i\epsilon_{ijk}J_k
\end{align*}
となることがわかる.すなわち$\{J_i,K_i\}$で張られるベクトル空間
\begin{align*}
\mathfrak{so}(3,1)'=\left\{ \sum_{i=1}^3 (\theta_i J_i-\omega_i K_i) \middle| \theta_i,\omega_i \in \mathbb{R}(i=1,2,3) \right\}
\end{align*}
で,この基底$\{J_i,K_i\}_{i=1,2,3}$に対するリー括弧積を
\begin{align*}
[J_i,J_j]_{\mathfrak{so}(3,1)'}=&i\epsilon_{ijk}J_k \\
[J_i,K_j]_{\mathfrak{so}(3,1)'}=&i\epsilon_{ijk}K_k \\
[K_i,K_j]_{\mathfrak{so}(3,1)'}=&-i\epsilon_{ijk}J_k
\end{align*}
と与えることにより定まるリー代数$\mathfrak{so}(3,1)'$が,ローレンツ代数$\mathfrak{so}(3,1)$と同型であることが理解できる.(この二つは同型であり普通は同一視されるべきものであるが,基底が異なるため,同型写像で結ばれる異なる二つのベクトル空間として扱い,プライムを付けて区別することにする.すなわち,同型写像
\begin{align*}
\Phi:\mathfrak{so}(3,1)\to& \mathfrak{so}(3,1)' \\
\frac{1}{2}\omega_{\mu\nu}J^{\mu\nu} \mapsto& \theta_i J_i -\omega_i K_i \quad \left(\theta_i:=\frac{1}{2}\epsilon_{ijk}\omega_{ij} ,\omega_i:=\omega_{i0}\right)
\end{align*}
で結ばれている.これが実際に可逆であることも$\omega_{ij}=\epsilon_{ijk}\theta_i,\omega_{i0}=-\omega_{0i}=\omega_{i}$よりすぐわかる.)\par
さて,2.7節で述べたようにトポロジー的には$SO(3,1)\simeq SL(2,\mathbb{C})/\mathbb{Z}_2$である.リー代数は単位元まわりの群の振る舞いを記述するから,それぞれのリー群$SO(3,1),SL(2,\mathbb{C})$から得られるそれぞれリー代数はほぼ同一であると考えられる.(つまり,二重被覆性は大域的構造から得られる性質であり,単位元まわりの情報のみを考える上では$\mathbb{Z}_2$の情報は無視してよい.)実際,実リー代数レベルで$\mathfrak{so}(3,1)\simeq \mathfrak{sl}(2,\mathbb{C})_{\mathbb{R}}$であることがすぐにわかる.これを確認しよう.ここで$\mathfrak{sl}(2,\mathbb{C})$は,基底$\{S_i\}_{i=1,2,3}$で張られる3次元$\mathbb{C}$係数複素ベクトル空間
\begin{align*}
\mathfrak{sl}(2,\mathbb{C})=\left\{ \sum_{i=1}^3\gamma_i S_i \middle| \gamma_i \in \mathbb{C} (i=1,2,3) \right\}
\end{align*}
で,リー括弧積は
\begin{align*}
[S_i,S_j]_{\mathfrak{sl}(2,\mathbb{C})}=i\epsilon_{ijk} S_k
\end{align*}
で与えられる.$\mathfrak{sl}(2,\mathbb{C})$の任意の元は,係数を実部と虚部に分けることで
\begin{align*}
\sum_{i=1}^3 \gamma_i S_i=&\sum_{i=1}^3(\alpha_i +i\beta_i) S_i \\
=&\sum_{i=1}^3\alpha_i S_i +\sum_{i=1}^3\beta_i (iS_i)
\end{align*}
となる.したがって,$\mathfrak{sl}(2,\mathbb{C})$を基底$\{S_i,iS_i\}_{i=1,2,3}$で張られる\uwave{$\mathbb{R}$係数}6次元実ベクトル空間とみなすことができる.
\begin{align*}
\mathfrak{sl}(2,\mathbb{C})_{\mathbb{R}}=\left\{ \sum_{i=1}^3(\alpha_i S_i+\beta_i(iS_i)) \middle| \alpha_i,\beta \in \mathbb{R} (i=1,2,3) \right\}
\end{align*}
リー括弧積は
\begin{align*}
[S_i,S_j]_{\mathfrak{sl}(2,\mathbb{C})_{\mathbb{R}}}=&i\epsilon_{ijk} S_k \\
[S_i,(iS_j)]_{\mathfrak{sl}(2,\mathbb{C})_{\mathbb{R}}}=&i\epsilon_{ijk}(iS_k) \\
[(iS_i),(iS_j)]_{\mathfrak{sl}(2,\mathbb{C})_{\mathbb{R}}}=&-i\epsilon_{ijk}S_k
\end{align*}
であるから,したがって$J_i=S_i,K_i=iS_i(i=1,2,3)$とおくことで$\mathfrak{so}(3,1)'$と同じリー括弧積の構造が得られることがわかる.$\mathfrak{sl}(2,\mathbb{C})_{\mathbb{R}}\simeq \mathfrak{so}(3,1)'$が理解できる.ここで$\mathfrak{sl}(2,\mathbb{C})_{\mathbb{R}}$は$\mathfrak{sl}(2,\mathbb{R})$ではないことに注意.前者は6次元実ベクトル空間であるが,後者は3次元実ベクトル空間
\begin{align*}
\mathfrak{sl}(2,\mathbb{R})=\left\{ \sum_{i=1}^3\gamma_i S_i \middle| \gamma_i \in \mathbb{R} (i=1,2,3) \right\}
\end{align*}
である.$\mathfrak{sl}(2,\mathbb{R})$は$\mathfrak{sl}(2,\mathbb{C})$の実形である.

\vskip\baselineskip

我々が知りたいのはローレンツ群の実線形表現ではなく,もっと一般的な複素線形表現である.複素表現を得るためには二つの方法があり,そのうちの1つは実表現を得てから複素表現にするという方法,もう一つは最初からリー代数を複素化し,その複素表現をとるという方法である.この二つは可換図式をなし,したがって同型である.そこで我々はリー代数の複素化を通じて,ローレンツ代数の既約な複素表現を得る方法を採用する.

\vskip\baselineskip

一般に,$\mathbb{R}$上のリー代数$\mathfrak{g}(\mathrm{dim}\mathfrak{g}=n)$の基底の一つを$\{a_i\}_{i=1}^n$とするとき,
\begin{align*}
\mathfrak{g}_{\mathbb{C}}:=\mathfrak{g}\otimes_{\mathbb{R}} \mathbb{C}=\left\{ \sum_{j=1}^{n} (\alpha_i+i\beta_i)a_j \middle| \alpha_j,\beta_j\in \mathbb{R} (j=1,\cdots,n) \right\}
\end{align*}
とする.このとき,$\mathbb{C}$上のベクトル空間$\mathfrak{g}_{\mathbb{C}}$を$\mathfrak{g}$の複素化と呼び,
\begin{align*}
\mathfrak{g}_{\mathbb{C}}=\left\{a+ib \middle| a,b\in\mathfrak{g}\right\}
\end{align*}
がなりたつ.$\mathfrak{g}_{\mathbb{C}}$上のリー括弧積は
\begin{align*}
[a\otimes z ,b\otimes w]_{\mathfrak{g}_{\mathbb{C}}}:=[a,b]_{\mathfrak{g}}\otimes zw
\end{align*}
で定められる.これがヤコビ恒等式を満たすことはすぐ示すことができる.与えられた実表現$\rho:\mathfrak{g}\to \mathfrak{gl}(V)$に対し,テンソル積の普遍性により一意的に複素線形写像
\begin{align*}
\rho^{\mathbb{C}}:\mathfrak{g}_{\mathbb{C}} \to \mathfrak{gl}(V)
\end{align*}
が存在して
\begin{align*}
\rho^{\mathbb{C}}(a\otimes z)=z \rho(a)
\end{align*}
を満たす.この$\rho^{\mathbb{C}}$がリー代数の準同型写像であることも確かめられる.
\begin{align*}
\rho^{\mathbb{C}}([a\otimes z ,b \otimes w]_{\mathfrak{g}_{\mathbb{C}}})=&\rho^{\mathbb{C}}([a,b]_{\mathfrak{g}}\otimes zw) \\
=&zw\rho([a,b]_{\mathfrak{g}}) \\
=&zw[\rho(a),\rho(b)]_{\circ} \\
=&[z\rho(a),w\rho(b)]_{\circ} \\
=&[\rho^{\mathbb{C}}(a\otimes z),\rho^{\mathbb{C}}(b\otimes w)]_{\circ}
\end{align*}
これでリー代数の複素化の定義が完了した.

\vskip\baselineskip

$\mathfrak{so}(3,1)$を複素化した代数$\mathfrak{so}(3,1)_{\mathbb{C}}$を考える.これは複素係数を許したベクトル空間である.この空間を何らかの直和へと分解したい.そこで,$J^{\mu\nu}$は$\mu,\nu$に対して完全反対称テンソルであるから,
\begin{align*}
J^{\mu\nu}=&\frac{1}{2}\left(J^{\mu\nu}+\frac{i}{2}\epsilon^{\mu\nu\rho\sigma}J_{\rho\sigma}\right)+\frac{1}{2}\left(J^{\mu\nu}-\frac{i}{2}\epsilon^{\mu\nu\rho\sigma}J_{\rho\sigma}\right) \\
=&J^{(+)\mu\nu}+J^{(-)\mu\nu}
\end{align*}
と分解することができることに気付けばよい.(複素化しなければ,係数に虚数単位が現れることが許されない.)これらは
\begin{align*}
\frac{i}{2}\epsilon_{\alpha\beta\mu\nu}\left(J^{\mu\nu}\pm \frac{i}{2}\epsilon^{\mu\nu\rho\sigma}J_{\rho\sigma}\right)=&\frac{i}{2}\epsilon_{\alpha\beta\mu\nu}J^{\mu\nu}\mp \frac{1}{4}\epsilon_{\alpha\beta\mu\nu}\epsilon^{\mu\nu\rho\sigma}J_{\rho\sigma} \\
=&\frac{i}{2}\epsilon_{\alpha\beta\mu\nu}J^{\mu\nu}\pm \frac{1}{2}(\delta^\rho_\alpha \delta^\sigma_\beta -\delta^\sigma_\alpha \delta^\rho_\beta)J_{\rho\sigma} \\
=&\frac{i}{2}\epsilon_{\alpha\beta\mu\nu}J^{\mu\nu}+J_{\alpha\beta} \\
=&J_{\alpha\beta}\pm \frac{i}{2}\epsilon_{\alpha\beta\mu\nu}J^{\mu\nu} \\
\therefore\quad J^{(\pm)\mu\nu}=&\pm \frac{i}{2}\epsilon^{\mu\nu\rho\sigma}J_{\rho\sigma}^{(\pm)}
\end{align*}
という意味で,$J^{(+)}$は自己双対反対称テンソル,$J^{(-)}$は反自己双対反対称テンソルである\footnote{なぜこのように分解するかというと,これら自身がそれぞれローレンツ群の互いに異なる既約表現として振舞う,という事情があるからである.すなわち,自己双対部分が基底として張る部分空間$\mathfrak{g}$と反自己双対部分が基底として張る部分空間$\mathfrak{h}$はリー括弧積で再び戻ってくる
\begin{align*}
[\mathfrak{so}(3,1)'_{\mathbb{C}},\mathfrak{g}]_{\mathfrak{so}(3,1)'_{\mathbb{C}}}\subset \mathfrak{g} \\
[\mathfrak{so}(3,1)'_{\mathbb{C}},\mathfrak{h}]_{\mathfrak{so}(3,1)'_{\mathbb{C}}}\subset \mathfrak{h}
\end{align*}
という意味で$\mathfrak{so}(3,1)'_{\mathbb{C}}$のイデアルをなす.(具体的に計算して確かめることはここではしない.後に行う(反)自己共役場についての計算が同等の計算になっている.)基底$\{J^{\mu\nu}\}$が$\{J^{(+)\mu\nu}\},\{J^{(-)\mu\nu}\}$に一意的に分解できるから,$\mathfrak{so}(3,1)'_{\mathbb{C}}$はベクトル空間として$\mathfrak{g}$と$\mathfrak{h}$の直和であることが理解でき,さらに$[\mathfrak{g},\mathfrak{h}]_{\mathfrak{so}(3,1)'_{\mathbb{C}}}\subset \mathfrak{g}\cap \mathfrak{h}=\{0\}$より$\mathfrak{so}(3,1)_{\mathbb{C}}$はリー代数として直和$\mathfrak{g}\oplus \mathfrak{h}$に同型であることが理解できる.}.ここで
\begin{align*}
\epsilon_{\alpha\beta\mu\nu}\epsilon^{\mu\nu\rho\sigma}=&A(\delta^\rho_\alpha \delta^\sigma_\beta -\delta^\sigma_\alpha \delta^\rho_\beta) \\
\epsilon_{01\mu\nu}\epsilon^{01\mu\nu}=&\epsilon_{0123}\epsilon^{0123}+\epsilon_{0132}\epsilon^{0132} \\
=&-2 \quad \because \epsilon^{0123}=-\epsilon_{0123}=+1 \\
=&A \\
\therefore \quad \epsilon_{\alpha\beta\mu\nu}\epsilon^{\mu\nu\rho\sigma}=&-2(\delta^\rho_\alpha \delta^\sigma_\beta -\delta^\sigma_\alpha \delta^\rho_\beta)
\end{align*}
を用いた.それぞれの成分を調べると
\begin{align*}
J^{(+)i0}=&\frac{1}{2}\left(J^{i0}+\frac{i}{2}\epsilon^{i0jk}J_{jk}\right) \\
=&\frac{1}{2}\left(J^{i0}-\frac{i}{2}\epsilon^{ijk}J_{jk}\right) \\
=&\frac{1}{2}\left(-J_{i0}-\frac{i}{2}\epsilon_{ijk}J_{jk}\right) \\
=&\frac{1}{2}\left(-K_i-iJ_i\right)=-\frac{i}{2}(J_i-iK_i)=-K_i^{(+)} \\
J^{(+)ij}=&\frac{1}{2}\left(J^{ij}+\frac{i}{2}\epsilon^{ijk0}J_{k0}+\frac{i}{2}\epsilon^{ij0k}J_{0k}\right) \\
=&\frac{1}{2}\left(J^{ij}-i\epsilon^{0ijk}J_{k0}\right) \\
=&\frac{1}{2}(\epsilon_{ijk}J_k -i\epsilon_{ijk}K_k) \\
=&\epsilon_{ijk}\frac{1}{2}(J_k-iK_k)=\epsilon_{ijk}J_i^{(+)} \\
J^{(-)i0}=&\frac{1}{2}\left(J^{i0}-\frac{i}{2}\epsilon^{i0jk}J_{jk}\right) \\
=&\frac{1}{2}\left(J^{i0}+\frac{i}{2}\epsilon^{ijk}J_{jk}\right) \\
=&\frac{1}{2}\left(-J_{i0}+\frac{i}{2}\epsilon_{ijk}J_{jk}\right) \\
=&\frac{1}{2}\left(-K_i+iJ_i\right)=\frac{i}{2}(J_i+iK_i)=-K^{(-)}_{i} \\
J^{(-)ij}=&\frac{1}{2}\left(J^{ij}-\frac{i}{2}\epsilon^{ijk0}J_{k0}-\frac{i}{2}\epsilon^{ij0k}J_{0k}\right) \\
=&\frac{1}{2}\left(J^{ij}+i\epsilon^{0ijk}J_{k0}\right) \\
=&\frac{1}{2}(\epsilon_{ijk}J_k +i\epsilon_{ijk}K_k)\\
=&\epsilon_{ijk}\frac{1}{2}(J_k+iK_k)=\epsilon_{ijk}J^{(-)}_k
\end{align*}
であるから,基底$\{J^{\mu\nu}\}$を次の6つの新しい基底で書き換えるのが最も自然である.
\begin{align*}
A_i=&J^{(-)}_i=iK^{(-)}_{i}=\frac{1}{2}(J_i+iK_i) \\
B_i=&J^{(+)}_i=-iK^{(+)}_{i}=\frac{1}{2}(J_i-iK_i)
\end{align*}
これを用いると,$\mathfrak{so}(3,1)'$を複素化した$\mathfrak{so}(3,1)'_{\mathbb{C}}$の任意の元は,$\theta_i,\omega_i\in \mathbb{C}(i=1,2,3)$として
\begin{align*}
\theta_{i}J_i-\omega_i K_i =&\left(\theta_i+i\omega_i\right)\frac{J_i+iK_i}{2}+\left(\theta_i-i\omega_i\right)\frac{J_i-iK_i}{2} \\
=&\alpha_i A_i +\beta_i B_i \quad (\alpha=\theta_i +i\omega_i =\beta^*_i)
\end{align*}
と書くことができる.この$A_i,B_i$は
\begin{align*}
[A_i,A_j]_{\mathfrak{so}(3,1)'_{\mathbb{C}}}=&\frac{1}{4}\Bigl([J_i,J_j]_{\mathfrak{so}(3,1)'_{\mathbb{C}}}+i[J_i,K_j]_{\mathfrak{so}(3,1)'_{\mathbb{C}}}+i[K_i,J_j]_{\mathfrak{so}(3,1)'_{\mathbb{C}}}-[K_i,K_j]_{\mathfrak{so}(3,1)'_{\mathbb{C}}}\Bigr) \\
=&\frac{1}{4}\Bigl(i\epsilon_{ijk}J_k+i\epsilon_{ijk}K_k+i\epsilon_{ijk}K_k+i\epsilon_{ijk}J_k\Bigr) \\
=&\epsilon_{ijk}\frac{1}{2}(J_i+iK_i)=i\epsilon_{ijk}A_k \\
[B_i,B_j]_{\mathfrak{so}(3,1)'_{\mathbb{C}}}=&\frac{1}{4}\Bigl([J_i,J_j]_{\mathfrak{so}(3,1)'_{\mathbb{C}}}-i[J_i,K_j]_{\mathfrak{so}(3,1)'_{\mathbb{C}}}-i[K_i,J_j]_{\mathfrak{so}(3,1)'_{\mathbb{C}}}-[K_i,K_j]_{\mathfrak{so}(3,1)'_{\mathbb{C}}}\Bigr) \\
=&\frac{1}{4}\Bigl(i\epsilon_{ijk}J_k-i\epsilon_{ijk}K_k-i\epsilon_{ijk}K_k+i\epsilon_{ijk}J_k\Bigr) \\
=&\epsilon_{ijk}\frac{1}{2}(J_k-iK_k)=i\epsilon_{ijk}B_k \\
[A_i,B_j]_{\mathfrak{so}(3,1)'_{\mathbb{C}}}=&\frac{1}{4}\Bigl([J_i,J_j]_{\mathfrak{so}(3,1)'_{\mathbb{C}}}-i[J_i,K_j]_{\mathfrak{so}(3,1)'_{\mathbb{C}}}+i[K_i,J_j]_{\mathfrak{so}(3,1)'_{\mathbb{C}}}+[K_i,K_j]_{\mathfrak{so}(3,1)'_{\mathbb{C}}}\Bigr) \\
=&\frac{1}{4}\Bigl(i\epsilon_{ijk}J_k-i\epsilon_{ijk}K_k+i\epsilon_{ijk}K_k-i\epsilon_{ijk}J_k\Bigr) \\
=&0
\end{align*}
を満たし,したがって$\{J_i,K_i\}_{i=1,2,3}$が張る複素リー代数$\mathfrak{so}(3,1)'_{\mathbb{C}}$は,基底$\{A_i\}_{i=1,2,3}$と$\{B_i\}_{i=1,2,3}$が張る二つの$\mathfrak{sl}(2,\mathbb{C})$の直和に同型
\begin{align*}
\mathfrak{so}(3,1)'_{\mathbb{C}} \simeq \mathfrak{sl}(2,\mathbb{C})_A \oplus \mathfrak{sl}(2,\mathbb{C})_B
\end{align*}
であり,実際その同型写像は
\begin{align*}
\Phi:\mathfrak{so}(3,1)'_{\mathbb{C}} \to& \mathfrak{sl}(2,\mathbb{C})_A \oplus \mathfrak{sl}(2,\mathbb{C})_B \\
\theta_i J_i-\omega_i K_i \mapsto& (\theta_i +i\omega_i)A_i + (\theta_i-i\omega_i)B_i
\end{align*}
で与えられる.明らかにこれは可逆である.$\mathfrak{sl}(2,\mathbb{C})$代数の有限次元既約表現の同値類の分類は量子力学で行ったように,$2j+1$次元表現で完全に分類されている.したがって$\mathfrak{so}(1,3)$の既約表現の同値類は,二つの$\mathfrak{sl}(2,\mathbb{C})$の既約表現の同値類を定める整数・半整数の組$(A,B)$で定められる$(2A+1)(2B+1)$次元表現で完全に分類され,その基底は
\begin{align*}
a=&+A,A-1,\cdots ,-A+1,-A \\
b=&+B,B-1,\cdots ,-B+1,-B
\end{align*}
でラベル付けされる.つまり,$\mathfrak{sl}(2,\mathbb{C})_A \oplus \mathfrak{sl}(2,\mathbb{C})_B$の有限次元既約表現の同値類は$(\rho_A,V_A)$と$(\rho_B,V_B)$の直和表現($\mathrm{dim}V_A=2A+1,\mathrm{dim}V_B=2B+1$)
\begin{align*}
\rho_{AB}=&\rho_A\otimes \mathbf{1}_B+\mathbf{1}_A \otimes \rho_B:\mathfrak{so}(1,3)\to \mathfrak{gl}(V_A\otimes V_B)_{\mathbb{C}} \\
\rho_A:&\mathfrak{sl}(2,\mathbb{C})_A \to \mathrm{End}(V_A,\mathbb{C}) \\
\rho_B:&\mathfrak{sl}(2,\mathbb{C})_B \to \mathrm{End}(V_B,\mathbb{C})
\end{align*}
で与えられる.ここで$\rho_{AB}(A_i),\rho_{AB}(B_i)$の表現行列$\mc{A}_i,\mc{B}_i$は
\begin{align*}
(\mc{A}_i)_{aa'bb'}=&\left(J_i^{(A)}\right)_{aa'}\delta_{bb'},\quad (\mc{B}_i)_{aa'bb'}=\delta_{aa'}\left(J_i^{(B)}\right)_{bb'} \\
\left(J^{(A)}_1\pm iJ^{(A)}_2\right)_{aa'}=&\delta_{a+1,a'}\sqrt{(A\mp a)(A\pm a+1)} ,\quad \left(J_3^{(A)}\right)_{aa'}=a\delta_{aa'} \\
\left(J^{(B)}_1\pm iJ^{(B)}_2\right)_{bb'}=&\delta_{b+1,b'}\sqrt{(B\mp b)(B\pm b+1)} ,\quad \left(J_3^{(B)}\right)_{bb'}=b\delta_{bb'}
\end{align*}
となる.\par
以上より,$\mathfrak{so}(3,1)$の複素表現を構成することができる.同型写像
\begin{align*}
\Phi_1:\mathfrak{so}(3,1)\to& \mathfrak{so}(3,1)' \\
\frac{1}{2}\omega_{\mu\nu}J^{\mu\nu} \mapsto& \theta_i J_i -\omega_i K_i \quad \left(\theta_i:=\frac{1}{2}\epsilon_{ijk}\omega_{ij} ,\omega_i:=\omega_{i0}\right)
\end{align*}
と,包含写像$\iota:\mathfrak{so}(3,1)'\to \mathfrak{so}(3,1)'_{\mathbb{C}}$と,同型写像
\begin{align*}
\Phi_2:\mathfrak{so}(3,1)_{\mathbb{C}} \to& \mathfrak{sl}(2,\mathbb{C})_A \oplus \mathfrak{sl}(2,\mathbb{C})_B \\
\theta_i J_i-\omega_i K_i \mapsto& (\theta_i +i\omega_i)A_i + (\theta_i-i\omega_i)B_i
\end{align*}
と,線形表現$(\rho_{AB},V_A\otimes V_B)$
\begin{align*}
\rho_{AB}=&\rho_A\otimes \mathbf{1}_B+\mathbf{1}_A \otimes \rho_B:\mathfrak{sl}(2,\mathbb{C})_A +\mathfrak{sl}(2,\mathbb{C})_B \to \mathfrak{gl}(V_A\otimes V_B)_{\mathbb{C}} \\
\rho_A:&\mathfrak{sl}(2,\mathbb{C})_A \to \mathrm{End}(V_A,\mathbb{C}) \\
\rho_B:&\mathfrak{sl}(2,\mathbb{C})_B \to \mathrm{End}(V_B,\mathbb{C})
\end{align*}
の合成写像
\begin{align*}
\rho:=\rho_{AB}\circ \Phi_2 \circ \iota \circ \Phi_1:\mathfrak{so}(3,1)\to \mathfrak{gl}(V_A\otimes V_B)_{\mathbb{C}}
\end{align*}
を用いることによって,$(\rho,V_A\otimes V_B)$は$\mathfrak{so}(3,1)$の複素表現をなす.すなわち,任意の$\mathfrak{so}(3,1)$の元は$\theta_i ,\omega_i \in \mathbb{R} (i=1,2,3)$として
\begin{align*}
\rho\left(\frac{1}{2}\omega_{\mu\nu}J^{\mu\nu}\right)=&(\rho_{AB}\circ \Phi_2 \circ \iota \circ \Phi_1) \left(\frac{1}{2}\omega_{\mu\nu}J^{\mu\nu}\right) \\
=&(\rho_{AB}\circ \Phi_2 \circ \iota) \left(\theta_i J_i -\omega_i K_i\right) \\
=&(\rho_{AB}\circ \Phi_2) \left(\theta_i J_i -\omega_i K_i\right) \\
=&\rho_{AB}\Bigl((\theta_i +i\omega_i)A_i +(\theta_i-i\omega_i)B_i\Bigr) \\
=&(\theta_i+i\omega_i)\rho_{AB}(A_i) + (\theta_i-i\omega_i)\rho_{AB}(B_i) \\
=:&(\theta_i+i\omega_i)\rho_{A}(A_i) \otimes \bm{1}_B + (\theta_i-i\omega_i)\bm{1}\otimes \rho_B(B_i)
\end{align*}
と表現される.最後の式の表現行列を考えよう.$\{\bm{e}^A_{a}\}_{a=A,\cdots -A}$を$V_A$の基底,$\{\bm{e}^B_{b}\}_{b=B,\cdots ,-B}$を$V_B$の基底とすれば,$V_A \otimes V_B$の任意の元は$\{\psi_{ab}^{(AB)} \} \in \mathbb{C}$を$V_A\otimes V_B$の成分として
\begin{align*}
\sum_{a=A,\cdots, -A} \sum_{b=B,\cdots ,-B} \psi^{(AB)}_{ab} \bm{e}^A_{a} \otimes \bm{e}^B_{b}
\end{align*}
と書ける.したがって$\rho\left(\frac{1}{2}\omega_{\mu\nu}J^{\mu\nu}\right) :V_A \otimes V_B \to V_A \otimes V_B$の作用をあらわに書けば
\begin{align*}
& \rho\left(\frac{1}{2}\omega_{\mu\nu}J^{\mu\nu}\right)\left(\sum_{a,b} \psi^{(AB)}_{ab} \bm{e}^A_{a} \otimes \bm{e}^B_{b}\right) \\
=&\sum_{a,b} \psi^{(AB)}_{ab} \rho\left(\frac{1}{2}\omega_{\mu\nu}J^{\mu\nu}\right)\left(\bm{e}^A_{a} \otimes \bm{e}^B_{b}\right) \\
=&\sum_{a=A,\cdots, -A} \sum_{b=B,\cdots ,-B}\psi^{(AB)}_{ab} \Bigl[(\theta_i+i\omega_i)\rho_{A}(A_i) \otimes \bm{1}_B + (\theta_i-i\omega_i)\bm{1}\otimes \rho_B(B_i)\Bigr] \left(\bm{e}^A_{a}\otimes \bm{e}^B_{b}\right) \\
=&\sum_{a,b}\psi^{(AB)}_{ab} \left[(\theta_i+i\omega_i) \sum_{a'}\left(J_i^{(A)}\right)_{a'a}\bm{e}^A_{a'}\otimes \bm{e}^B_{b}  + (\theta_i-i\omega_i) \sum_{b'}\left(J_i^{(B)}\right)_{b'b}\bm{e}^A_{a}\otimes \bm{e}^B_{b'}\right] \\
=&\sum_{a',b'} \left[(\theta_i+i\omega_i) \sum_{a,b}\left(J_i^{(A)}\right)_{a'a}\delta_{b'b}\psi^{(AB)}_{ab} + (\theta_i-i\omega_i) \sum_{a,b}\delta_{a'a}\left(J_i^{(B)}\right)_{b'b}\psi^{(AB)}_{ab}\right]\bm{e}^A_{a'}\otimes \bm{e}^B_{b'} \\
=&\sum_{a,b} \left\{\sum_{a,b}\left[(\theta_i+i\omega_i) \left(J_i^{(A)}\right)_{aa'}\delta_{bb'}+ (\theta_i-i\omega_i) \delta_{aa'}\left(J_i^{(B)}\right)_{bb'}\right] \psi^{(AB)}_{a'b'}\right\}\bm{e}^A_{a}\otimes \bm{e}^B_{b} \\
=&\sum_{a,b} \left\{\sum_{a'b'}\left[\frac{1}{2}\omega_{\mu\nu}\left(\mc{J}^{(AB)\mu\nu}\right)\right]_{aa'bb'} \psi^{(AB)}_{a'b'}\right\}\bm{e}^A_{a} \otimes \bm{e}^B_{b}
\end{align*}
すなわち
\begin{align*}
&(\theta_i+i\omega_i)(\mc{A}_i)_{aa'bb'} + (\theta_i-i\omega_i)(\mc{B}_i)_{aa'bb'} \\
=&(\theta_i+i\omega_i)\left(J^{(A)}_i\right)_{aa'}\delta_{bb'}+(\theta_i-i\omega_i)\delta_{aa'}\left(J^{(B)}\right)_{bb'} \\
=&\frac{1}{2}\omega_{\mu\nu}\left(\mc{J}^{(AB)\mu\nu}\right)_{aa'bb'}
\end{align*}
が$\rho\left(\frac{1}{2}\omega_{\mu\nu}J^{\mu\nu}\right)$の表現行列となっている.ここで行列$\{\mc{J}^{(AB)\mu\nu}\}_{0\leq \mu <\nu \leq 3}$は,上の構成法より
\begin{align*}
\bm{\mc{A}}=&\frac{1}{2}(\bm{\mc{J}}+i\bm{\mc{K}}),\quad \bm{\mc{B}}=\frac{1}{2}(\bm{\mc{J}}-i\bm{\mc{K}}) \\
\left(\mc{J}^{(AB)ij}\right)_{aa'bb'}:=&\epsilon_{ijk}(\mc{J}_k)_{aa'bb'} \\
=&\epsilon_{ijk}\Bigl((\mc{A}_k)_{aa'bb'}+(\mc{B}_k)_{aa'bb'}\Bigr) \\
=&\epsilon_{ijk}\left(\left(J^{(A)}_k\right)_{aa'}\delta_{bb'}+\delta_{aa'}\left(J^{(B)}_k\right)_{bb'}\right) \\
\left(\mc{J}^{(AB)i0}\right)_{aa'bb'}:=&-(\mc{K}_i)_{aa'bb'} \\
=&i(\mc{A}_i)_{aa'bb'}-(\mc{B}_i)_{aa'bb'} \\
=&i\left(\left(J^{(A)}_i\right)_{aa'}\delta_{bb'}-\delta_{aa'}\left(J^{(B)}\right)_{bb'}\right)
\end{align*}
で定められ,これは実際に$\mathfrak{so}(3,1)$の交換関係
\begin{align*}
i[\mc{J}^{(AB)\mu\nu},\mc{J}^{(AB)\rho\sigma}]=\eta^{\nu\rho}\mc{J}^{(AB)\mu\sigma}-\eta^{\mu\rho}\mc{J}^{(AB)\nu\sigma}-\eta^{\sigma\mu}\mc{J}^{(AB)\rho\nu}+\eta^{\sigma\nu}\mc{J}^{(AB)\rho\mu}
\end{align*}
を満たす.(構成法より明らかなので証明略.)したがって$\mathfrak{so}(3,1)$の表現をなす.



\vskip\baselineskip

$\mathfrak{so}(3,1)\simeq \mathfrak{sl}(2,\mathbb{C})_{\mathbb{R}}$を複素化したものは,$\mathfrak{sl}(2,\mathbb{C})\oplus \mathfrak{sl}(2,\mathbb{C})$に同型であることを見た.一般に,複素リー代数$\mathfrak{g}$を実係数で見た$\mathfrak{g}_{\mathbb{R}}$を再び複素化した代数$\mathfrak{g}_{\mathbb{R}}\otimes_{\mathbb{R}} \mathbb{C}$は
\begin{align*}
\mathfrak{g}_{\mathbb{R}} \otimes_{\mathbb{R}} \mathbb{C} \simeq \mathfrak{g} \oplus \mathfrak{g}
\end{align*}
と分解できる.(証明追記予定)



\vskip\baselineskip



以上により,整数あるいは半整数の組$(A,B)$を定めることで$\mathbf{J}^{(A)},\mathbf{J}^{(B)}$の表現が一意的に定まり,したがって$\mathfrak{so}(3,1)$の表現も一意的に決まることが分かった.これを用いればローレンツ群$SO(3,1)$の有限次元表現も得ることができる.
\begin{align*}
D^{(AB)}_{aa'bb'}(\Lambda)=&\left[\exp\left(\frac{i}{2}\omega_{\mu\nu}\mc{J}^{(AB)\mu\nu}\right)\right]_{aa'bb'} \\
=&\left[\exp\left(i(\theta_i+i\omega_i)\mc{A}_i +i(\theta_i-i\omega_i)\mc{B}_i\right)\right]_{aa'bb'} \\
=&\left[\exp\left(i(\theta_i+i\omega_i)J^{(A)}_i1_B +i(\theta_i-i\omega_i)1_A J^{(B)}_i\right)\right]_{aa'bb'} \\
=&\left[\exp(i(\theta_i+i\omega_i)J^{(A)})\right]_{aa'}\left[\exp(i(\theta_i-i\omega_i)J^{(B)})\right]_{bb'} \\
=&D^{A0}_{aa'}(\Lambda) D^{0B}_{bb'}(\Lambda)
\end{align*}
2行目から3行目にかけて,BCH公式と$[\mc{A}_i,\mc{B}_j]=[J^{(A)}_i1_B ,1_A J^{(B)}_j]=0$であることを用いた.また$D^{A0}$と$D^{0B}$はそれぞれ$D^{AB}$の$B=0$(つまり$J^{(B)}=0$)と$A=0$(つまり$J^{(A)}=0$)の場合に相当することを用いた.$(A,B)$の組を定めることによって,ローレンツ群のあらゆる表現を網羅することができる.したがって以下では,$(A,B)$の組を与えて定まったローレンツ群の既約表現のことを$(A,B)$表現と呼ぶ\footnote{$\{\mc{J}^{(AB)\mu\nu}\}$は$\mathfrak{so}(3,1)$の表現になっているが,$D^{(AB)}(\Lambda)$は大域的には$SO(3,1)$の表現ではなく,一般に普遍被覆群$SL(2,\mathbb{C})$の表現にしかなっていない.リー代数は単位元近傍の情報しか含んでおらず,したがって$SO(3,1)$と$SL(2,\mathbb{C})$との大域的構造の違いを反映しない.これは量子力学で回転群$SO(3)$の表現を与えたとき,スピン$j$が半整数のスピノル表現はその普遍被覆群$SU(2)$の表現であり$SO(3)$の表現ではなかったのと同様の理由である.}.\par
この表現にしたがって変換する\footnote{$(A,B)$表現にしたがって変換する,という言葉は,ユニタリー演算子による変換が,有限次元$(A,B)$表現行列による変換に翻訳できる,という意味である.}場$\psi^{(AB)}_{ab}(x)$は
\begin{align*}
U_0(\Lambda)\psi^{(AB)}_{ab}(x)U^{-1}(\Lambda)=&D^{(AB)}_{aa'bb'}(\Lambda^{-1})\psi^{(AB)}_{a'b'}(\Lambda x) \\
=&\left[\exp\left(\frac{i}{2}\omega_{\mu\nu}\mc{J}^{(AB)\mu\nu}\right)\right]_{aa'bb'}\psi^{(AB)}_{a'b'}(\Lambda x) \\
=&\left[\exp(-i(\theta_i+i\omega_i)J^{(A)})\right]_{aa'}\left[\exp(-i(\theta_i-i\omega_i)J^{(B)})\right]_{bb'}\psi^{(AB)}_{a'b'}(\Lambda x)
\end{align*}
と変換し,$\omega_{\mu\nu}$が微小であるとして係数比較すれば
\begin{align*}
\psi^{(AB)}_{ab}(\Lambda x)=&\psi^{(AB)}_{ab}\left(x^\rho+\frac{i}{2}\omega_{\mu\nu}\tensor{(\mc{J}^{\mu\nu})}{^\rho_\sigma}x^\sigma\right) \\
=&\psi^{(AB)}_{ab}(\Lambda x)+\frac{i}{2}\omega_{\mu\nu}\tensor{(\mc{J}^{\mu\nu})}{^\rho_\sigma}x^\sigma \frac{\partial\psi^{(AB)}_{ab}}{\partial x^\rho}(x) \\
=&\psi^{(AB)}_{ab}(\Lambda x)+\frac{i}{2}\omega_{\mu\nu}\left(i\delta^\mu_\sigma \eta^{\rho\nu}-i\delta^\nu_\sigma \eta^{\rho\mu}\right)x^\sigma \frac{\partial\psi^{(AB)}_{ab}}{\partial x^\rho}(x) \\
=&\psi^{(AB)}_{ab}(\Lambda x)+\frac{i}{2}\omega_{\mu\nu}\left(ix^\mu \partial^\nu-ix^\nu \partial^\mu\right) \psi^{(AB)}_{ab}(x) \\
=&\psi^{(AB)}_{ab}(\Lambda x)-\frac{i}{2}\omega_{\mu\nu}L^{\mu\nu} \psi^{(AB)}_{ab}(x) \quad \left(L^{\mu\nu}:=-ix^\mu \partial^\nu +ix^\mu \partial^\nu \right)
\end{align*}
であることを用いて
\begin{align*}
(\mathrm{LHS})=&(1+\frac{i}{2}\omega_{\mu\nu}J^{\mu\nu})\psi_{ab}^{(AB)}(x)(1-\frac{i}{2}\omega_{\mu\nu}J^{\mu\nu}) \\
=&\psi_{ab}^{(AB)}(x)+\frac{i}{2}\omega_{\mu\nu}[J^{\mu\nu},\psi_{ab}^{(AB)}(x)] \\
(\mathrm{RHS})=&(1-\frac{i}{2}\omega_{\mu\nu}\mc{J}^{(AB)\mu\nu})_{aa'bb'}\psi_{a'b'}^{(AB)}(\Lambda x) \\
=&\psi_{ab}^{(AB)}(\Lambda x)-\frac{i}{2}\omega_{\mu\nu}\mc{J}^{(AB)\mu\nu}_{aa'bb'}\psi_{a'b'}^{(AB)}(\Lambda x) \\
=&\psi_{ab}^{(AB)}(x)-\frac{i}{2}\omega_{\mu\nu}(\mc{J}^{(AB)\mu\nu}+L^{\mu\nu})_{aa'bb'}\psi_{a'b'}^{(AB)}(x) \\
\therefore \quad [J^{\mu\nu},\psi^{(AB)}_{ab}(x)]=&-(\mc{J}^{(AB)\mu\nu}+L^{\mu\nu})_{aa'bb'}\psi^{(AB)}_{a'b'}(x)
\end{align*}
となる.物理的に,$L^{\mu\nu}$は4次元的軌道角運動量であり,$\mc{J}^{(AB)\mu\nu}$は内部スピン角運動量に対応している.$J^{\mu\nu}$は状態空間にはたらく全角運動量演算子である.(右辺にマイナスがついているのは,$\psi^{(AB)}(x)$が所謂,正エネルギー解部分が消滅演算子になっており,全角運動量$\mc{J}^{(AB)\mu\nu}+L^{\mu\nu}$を運ぶ粒子を消滅させるからであると考えられる.)\par

\vskip\baselineskip

以上の手続きは非常に長かったので,あまり正確でない記述で要点だけを抑えよう.$SO(3,1)$の有限次元表現がもし得られたとすれば,それは
\begin{align*}
&\exp\left(\frac{i}{2}\omega_{\mu\nu}\mc{J}^{\mu\nu}\right) \\
=&\exp\Bigl(i\theta_i \mc{J}_i -i\omega_i \mc{K}_i\Bigr) \qquad \left(\mc{J}_i =\frac{1}{2}\epsilon_{ijk}\mc{J}_{jk} ,\mc{K}_i=\mc{J}_{i0} \right)\\
=&\exp\Bigl(i\alpha_i \mc{A}_i +i\beta_i \mc{B}_i\Bigr) \qquad \left( \mc{A}_i=\frac{1}{2}(\mc{J}_i+i\mc{K}_i) ,\mc{B}_i=\frac{1}{2}(\mc{J}_i-i\mc{K}_i)\right)
\end{align*}
と書くことができ,この行列$\mc{A}_i,\mc{B}_i$がなす交換関係は二つの$\mathfrak{sl}(2,\mathbb{C})\simeq \mathfrak{su}(2)_{\mathbb{C}}$がなす二つの独立なスピン代数と同じ
\begin{align*}
[\mc{A}_i,\mc{A}_j]=&i\epsilon_{ijk} \mc{A}_k \\
[\mc{B}_i,\mc{B}_j]=&i\epsilon_{ijk} \mc{B}_k \\
[\mc{A}_i,\mc{B}_j]=&0
\end{align*}
になっている.量子力学でやったように$\mathfrak{sl}(2,\mathbb{C})\simeq \mathfrak{su}(2)_{\mathbb{C}}$の有限次元既約表現の同値類は$j=0,\frac{1}{2},1,\frac{3}{2},\cdots$でラベル付けされた$2j+1$次元表現となるのだった.したがって,ローレンツ群の有限次元既約表現は$A=0,\frac{1}{2},1,\frac{3}{2},\cdots$と$B=0,\frac{1}{2},1,\frac{3}{2},\cdots$の組で同型を除いて一意的に定まる.
\begin{align*}
(\mc{A}_i)_{aa'bb'}=&\left(J_i^{(A)}\right)_{aa'}\delta_{bb'},\quad (\mc{B}_i)_{aa'bb'}=\delta_{aa'}\left(J_i^{(B)}\right)_{bb'} \\
\left(J^{(A)}_1\pm iJ^{(A)}_2\right)_{aa'}=&\delta_{a+1,a'}\sqrt{(A\mp a)(A\pm a+1)} ,\quad \left(J_3^{(A)}\right)_{aa'}=a\delta_{aa'} \\
\left(J^{(B)}_1\pm iJ^{(B)}_2\right)_{bb'}=&\delta_{b+1,b'}\sqrt{(B\mp b)(B\pm b+1)} ,\quad \left(J_3^{(B)}\right)_{bb'}=b\delta_{bb'}
\end{align*}
これが$(A,B)$表現である.

\vskip\baselineskip

この$J^{(j)}_i$はエルミート行列であることがすぐに確かめられる.実際,任意の$j=0,\frac{1}{2},1,\cdots$に関して
\begin{align*}
\left(J_1^{(j)}\right)_{\sigma'\sigma}=&\frac{1}{2}\left(\delta_{\sigma',\sigma+ 1}\sqrt{(j- \sigma)(j+ \sigma+1)}+\delta_{\sigma',\sigma-1}\sqrt{(j+ \sigma)(j- \sigma+1)}\right) \\
\left(J_2^{(j)}\right)_{\sigma'\sigma}=&\frac{1}{2i}\left(\delta_{\sigma',\sigma+ 1}\sqrt{(j- \sigma)(j+ \sigma+1)}-\delta_{\sigma',\sigma-1}\sqrt{(j+ \sigma)(j- \sigma+1)}\right) \\
\left(J_3^{(j)}\right)_{\sigma'\sigma}=&\delta_{\sigma'\sigma}\sigma
\end{align*}
と書かれ,明らかに$(J_i^{(j)\dagger})_{\sigma\sigma'}=(J^{(j)*}_i)_{\sigma'\sigma}$は$(J^{(j)}_i)_{\sigma\sigma'}$に一致する.したがって$(\mc{A}^\dagger_{i})_{aa'bb'}=(\mc{A}^*_i)_{a'ab'b}$と$(\mc{B}^\dagger_{i})_{aa'bb'}=(\mc{B}^*_i)_{a'ab'b}$もそれぞれ$(\mc{A}_i)_{aa'bb'}$と$(\mc{B}_i)_{aa'bb'}$に一致するという意味でエルミート行列である.\par
しかし,このように構成された$D^{(AB)}_{aa'bb'}(\Lambda)$はユニタリ表現にはならないことに注意する.これは,$\bm{\mc{A}},\bm{\mc{B}}$がエルミート行列であるが,
\begin{align*}
\bm{\mc{J}}=&\bm{\mc{A}}+\bm{\mc{B}} \\
\bm{\mc{K}}=&-i(\bm{\mc{A}}-\bm{\mc{B}})
\end{align*}
となるから,$\bm{\mc{J}}$はエルミート行列になるが$\bm{\mc{K}}$はエルミートではなく反エルミート$\bm{\mc{K}}^\dagger=-\bm{\mc{K}}$になるからである.つまり
\begin{align*}
[D^{(AB)}(\Lambda)]^\dagger=&\exp(i\theta_i \mc{J}_i -i\omega_i \mc{K}_i)^\dagger \\
=&\exp(-i\theta_i \mc{J}^\dagger_i +i\omega_i \mc{K}_i^\dagger) \\
=&\exp(-i\theta_i \mc{J}_i -i\omega_i \mc{K}_i^) \\
\neq &\exp(-i\theta_i \mc{J}_i +i\omega_i \mc{K}_i) =[D^{(AB)}(\Lambda)]^{-1}
\end{align*}
となる.(もちろんそのまま確認することもできる.上で与えた表現$D^{(AB)}(\Lambda)$をエルミート共役すれば
\begin{align*}
[D^{(AB)}(\Lambda)]^\dagger=&\exp\left(i(\theta_i+i\omega_i)\mc{A}_i +i(\theta_i-i\omega_i)\mc{B}_i\right)^\dagger \\
=&\exp\left(-i(\theta_i-i\omega_i)\mc{A}_i -i(\theta_i+i\omega_i)\mc{B}_i\right) \\
\neq& \exp\left(-i(\theta_i+i\omega_i)\mc{A}_i -i(\theta_i-i\omega_i)\mc{B}_i\right)=[D^{(AB)}(\Lambda)]^{-1}
\end{align*}
となる.)これは,斉次ローレンツ群$SO(3,1)$が,コンパクト群である4次元回転群$SO(4)=SU(2)\times SU(2)$と同じではなく非コンパクト群であるという事実に由来する.有限次元のユニタリー表現を持てるのはコンパクト群のみである.(非コンパクト部分,つまりブースト部分が自明な表現になっている場合を除く.つまり2章で質量ゼロ粒子の小群の解析でやったように,$ISO(2)=SO(2)\times \mathbb{R}^2$群の非コンパクト部分$\mathbb{R}^2$が自明な作用するとすれば有限次元ユニタリ表現が得られる.)非ユニタリーな表現を取り扱うことに問題はない.なぜなら,ここで関与しているのは場であって波動関数ではないので,ローレンツ不変な正のノルムを持っている必要はない\footnote{相対論的量子力学で,ディラックの波動関数$\psi(x)$から得られるノルム$\psi^\dagger \psi$は正のノルムになるが$\psi^\dagger D^\dagger(\Lambda)D(\Lambda)\psi\neq \psi^\dagger \psi$よりローレンツ不変でない.したがって相対論的粒子の確率密度$\rho(x)=\psi^\dagger(x)\psi(x)$はローレンツ不変でなくなる.具体的には$j^\mu:=\bar{\psi}\gamma^\mu \psi$が4元ベクトルとして変換し,その時間成分が$\bar{\psi} \gamma^0 \psi=i\rho$を与えることから,$j^i$成分が混ざりこむ形に変換されてしまう.ただし全確率$P(t)=\int \rho(\mathbf{x},t) d^3\mathbf{x}$はローレンツ不変$P(t')=P(t)$になる.これを見るには超曲面要素$d\Sigma_\mu=(0,0,0,d^3\mathbf{x})$を用いて$P(t)=\int j^\mu d\Sigma_\mu$と書けることを使えば理解できる.}.\par
一方,回転部分群$SO(3)$の表現はエルミート行列
\begin{align*}
\bm{\mc{J}}=&\bm{\mc{A}}+\bm{\mc{B}}
\end{align*}
で生成されるユニタリー表現
\begin{align*}
D^{(AB)}(R)=\exp(i\theta_i \mc{J}_i)
\end{align*}
である.$\bm{\mc{A}},\bm{\mc{B}}$がそれぞれ$\mathfrak{su}(2)$の表現になっていることを思い出せば,これは量子力学でやったように
\begin{align*}
j=A+B,A+B-1,\cdots, |A-B|
\end{align*}
のスピンのように回転する成分を持つことが分かる.このことは,$(A,B)$表現を我々に馴染み深いテンソルとスピノルに同定する手助けになる.

\vskip\baselineskip

以下で具体的な表現に属する場を調べてみよう.\par
・$A=0,B=0$\par
$(0,0)$場は明らかに$j=0$成分のみを持つスカラーである.実際
\begin{align*}
\bm{\mc{A}}=&0,\quad \bm{\mc{B}}=0 \\
D^{(0,0)}(\Lambda)=&1_A 1_B \\
U_0(\Lambda)\psi^{(0,0)}_{ab}(x) U_0^{-1}(\Lambda)=&D^{(0,0)}_{aa'bb'}(\Lambda^{-1})\psi^{(0,0)}_{a'b'}(\Lambda x)=\psi^{(0,0)}_{ab}(\Lambda x)
\end{align*}
であり,これは実際にスカラー場である.もちろん$\mc{J}^{(0,0)\mu\nu}=0$である.この場合$a,b$はともに単一の成分$a=b=0$しか持たないから省略してしまって$\phi(x)=\psi_{ab}^{(0,0)}(x)$と書きなおせば
\begin{align*}
U_0(\Lambda)\phi(x) U_0^{-1}(\Lambda)=\phi(\Lambda x)
\end{align*}
となる.
\begin{tcolorbox}[title= ${(0,0)}$表現(スカラー場)]
\begin{align*}
D(\Lambda)=&1 \\
U_0(\Lambda)\phi(x) U_0^{-1}(\Lambda)=&\phi(\Lambda x)
\end{align*}
\end{tcolorbox}

\vskip\baselineskip

・$A=\frac{1}{2},B=0$\par
$(\frac{1}{2},0)$または$(0,\frac{1}{2})$場は$j=+\frac{1}{2}$のみを持てるから,これらはワイルスピノルの上の2成分(例えばワイル表示$\gamma_5=+1$部分空間)と下の2成分($\gamma_5=-1$)である.実際$(\frac{1}{2},0)$表現は,上の公式にしたがって表現行列を調べてみると
\begin{align*}
J_1^{(1/2)}+iJ_2^{(1/2)}=&\left(
\begin{matrix}
0 & \sqrt{(\frac{1}{2}+\frac{1}{2})(\frac{1}{2}-\frac{1}{2}+1)} \\
0 & 0
\end{matrix}
\right)=\left(
\begin{matrix}
0 & 1 \\
0 & 0
\end{matrix}
\right) \\
J_1^{(1/2)}-iJ_2^{(1/2)}=&\left(
\begin{matrix}
0 & 0 \\
\sqrt{(\frac{1}{2}+\frac{1}{2})(\frac{1}{2}-\frac{1}{2}+1)} & 0
\end{matrix}
\right)=\left(
\begin{matrix}
0 & 0 \\
1 & 0
\end{matrix}
\right)\\
J_3^{(1/2)}=&\left(
\begin{matrix}
1/2 & 0 \\
0 & -1/2
\end{matrix}
\right) \\
\therefore\quad J^{(1/2)}_1=&\frac{1}{2}\left(
\begin{matrix}
0 & 1 \\
1 & 0
\end{matrix}
\right)=\frac{1}{2}\sigma_1 ,\quad J^{(1/2)}_2=\frac{1}{2}\left(
\begin{matrix}
0 & -i \\
i & 0
\end{matrix}
\right)=\frac{1}{2}\sigma_2 ,\quad J^{(1/2)}_{3}=\frac{1}{2}\left(
\begin{matrix}
1 & 0 \\
0 & -1
\end{matrix}
\right)=\frac{1}{2}\sigma_3 \\
\mathbf{J}^{(1/2)}=&\frac{1}{2}\bm{\sigma}
\end{align*}
より,$\bm{\mc{A}}=J^{(\frac{1}{2})}1_B$はパウリ行列であることがわかり
\begin{align*}
\bm{\mc{A}}=&\frac{\bm{\sigma}}{2}1_B,\quad \bm{\mc{B}}=0 \\
D^{(\frac{1}{2}0)}(\Lambda)=&\exp\left(i(\theta_i-i\omega_i)\frac{1}{2}\sigma_i\right)1_B \\
=&\exp\left(i\theta_i\frac{\sigma_i}{2}+i\omega_i\left(-i\frac{\sigma_i}{2}\right)\right)1_B \\
U_0(\Lambda)\psi^{(\frac{1}{2}0)}_{ab}(x) U_0^{-1}(\Lambda)=&D^{(\frac{1}{2}0)}_{aa'bb'}(\Lambda^{-1})\psi^{(\frac{1}{2}0)}_{a'b'}(\Lambda x) \\
=&\exp\left(-i\theta_i\frac{\sigma_i}{2}-i\omega_i\left(-i\frac{\sigma_i}{2}\right)\right)_{aa'}\psi^{(\frac{1}{2}0)}_{a'b}(\Lambda x)
\end{align*}
と変換する.添え字$a$は2成分$a=\frac{1}{2},-\frac{1}{2}$をとり,添え字$b$は単一の成分$b=0$しか持たないから省略してしまって$\psi^{L}_{a}(x)=\psi^{(\frac{1}{2}0)}_{ab}(x)$と書きなおせば
\begin{align*}
U_0(\Lambda)\psi^{L}_{a}(x)U_0^{-1}(\Lambda)=\exp\left(-i\theta_i\frac{\sigma_i}{2}-i\omega_i\left(-i\frac{\sigma_i}{2}\right)\right)_{aa'}\psi^{L}_{a'}(\Lambda x)
\end{align*}
となる.
\begin{tcolorbox}[title=${(\frac{1}{2},0)}$表現(左ワイルスピノル)]
\begin{align*}
&D_{aa'}(\Lambda)=\exp\left(i\theta_i\frac{\sigma_i}{2}+i\omega_i\left(-i\frac{\sigma_i}{2}\right)\right)_{aa'} \\
&U_0(\Lambda)\psi^L_{a}(x) U_0^{-1}(\Lambda)=D_{aa'}(\Lambda^{-1})\psi^L_{a'}(\Lambda x)
\end{align*}
\end{tcolorbox}

\vskip\baselineskip

・$A=0,B=\frac{1}{2}$\par
一方$(0,\frac{1}{2})$表現は$\bm{\mc{B}}=1_AJ^{(\frac{1}{2})}$がパウリ行列となるから
\begin{align*}
\bm{\mc{A}}=&0,\quad \bm{\mc{B}}=1_A \frac{\bm{\sigma}}{2} \\
D^{(0\frac{1}{2})}(\Lambda)=&1_A\exp\left(i(\theta_i+i\omega_i)\frac{1}{2}\sigma_i\right) \\
=&1_A \exp\left(i\theta_i\frac{\sigma_i}{2}+i\omega_i\left(+i\frac{\sigma_i}{2}\right)\right) \\
U_0(\Lambda)\psi^{(0\frac{1}{2})}_\ell(x) U_0^{-1}(\Lambda)=&D^{(0\frac{1}{2})}_{aa'bb'}(\Lambda^{-1})\psi^{(\frac{1}{2}0)}_{a'b'}(\Lambda x) \\
=&\exp\left(-i\theta_i\frac{\sigma_i}{2}-i\omega_i\left(+i\frac{\sigma_i}{2}\right)\right)_{bb'}\psi^{(0\frac{1}{2})}_{ab'}(\Lambda x)
\end{align*}
と変換する.添え字$b$は$b=\frac{1}{2},-\frac{1}{2}$をとり,添え字$a$は単一の成分$a=0$しか持たないから省略してしまって,$\psi^R_{b}(x)=\psi^{(0\frac{1}{2})}_{ab}(x)$と書きなおせば
\begin{align*}
U_0(\Lambda)\psi^{R}_{b}(x)U_0^{-1}(\Lambda)=\exp\left(-i\theta_i\frac{\sigma_i}{2}-i\omega_i\left(+i\frac{\sigma_i}{2}\right)\right)_{bb'}\psi^{R}_{b'}(\Lambda x)
\end{align*}
となる.
\begin{tcolorbox}[title=${(0,\frac{1}{2})}$表現(右ワイルスピノル)]
\begin{align*}
&D_{bb'}(\Lambda)=\exp\left(i\theta_i\frac{\sigma_i}{2}+i\omega_i\left(+i\frac{\sigma_i}{2}\right)\right)_{bb'} \\
&U_0(\Lambda)\psi^R_{b}(x) U_0^{-1}(\Lambda)=D_{bb'}(\Lambda^{-1})\psi^R_{b'}(\Lambda x)
\end{align*}
\end{tcolorbox}
どちらも$\omega_i=0$の回転部分群では2成分スピノルとして変換することがわかる.その直和$\psi:=\psi^L\oplus \psi^R=\left(
\begin{matrix}
\psi^L \\
\psi^R
\end{matrix}
\right)$は
\begin{align*}
U_0(\Lambda)\psi(x)U_0^{-1}(\Lambda)=&U_0(\Lambda)\psi^L(x)U_0^{-1}(\Lambda )\oplus U_0(\Lambda)\psi^R(x)U^{-1}_0(\Lambda) \\
=&\left(
\begin{matrix}
\exp\left(-i\theta_i\frac{\sigma_i}{2}-i\omega_i\left(-i\frac{\sigma_i}{2}\right)\right)\psi^L(\Lambda x) \\
\exp\left(-i\theta_i\frac{\sigma_i}{2}-i\omega_i\left(+i\frac{\sigma_i}{2}\right)\right)\psi^R(\Lambda x)
\end{matrix}
\right) \\
=&\exp\left(-i\theta_i \left(
\begin{matrix}
\frac{\sigma_i}{2} & 0 \\
0 & \frac{\sigma_i}{2}
\end{matrix}
\right)-i\omega_i\left(
\begin{matrix}
-i\frac{\sigma_i}{2} & 0 \\
0 & +i\frac{\sigma_i}{2}
\end{matrix}
\right) \right) \left(
\begin{matrix}
\psi^L(\Lambda x) \\
\psi^R(\Lambda x)
\end{matrix}
\right) \\
=&D(\Lambda^{-1})\psi(\Lambda x)
\end{align*}
と書ける.この$D(\Lambda^{-1})$はまさに前節で議論した4成分スピノル表現である.したがって4成分スピノルは可約表現であり,その既約分解は$\left(\frac{1}{2},0\right)\oplus \left(0,\frac{1}{2}\right)$であることがわかる.そして特にワイル表示では(5.4.28)の$\gamma_5=+1$の固有ベクトルである上成分が$\left(\frac{1}{2},0\right)$既約成分,$\gamma_5=-1$の固有ベクトルである下成分が$\left(0,\frac{1}{2}\right)$既約成分であることがわかる.\par
また,$\left(\frac{1}{2},0\right)$表現に属する場を複素共役した場$\psi^{L*}_{a}(x)$の変換性を見てみると
\begin{align*}
U_0(\Lambda)\psi^{L*}_{a}(x) U_0^{-1}(\Lambda)=&\exp\left(+i\theta_i\frac{\sigma_i^*}{2}+i\omega_i\left(+i\frac{\sigma_i^*}{2}\right)\right)_{aa'}\psi^{L*}_{a'}(\Lambda x)
\end{align*}
となる.さらに$\sigma_2\bm{\sigma}^*\sigma_2= -\bm{\sigma}$を用いると$(\sigma_2)_{aa'}\psi^{L*}_{a'}(x)$の変換性は
\begin{align*}
U_0(\Lambda)\left(\sigma_2 \psi^{L*}\right)_{a}(x) U_0^{-1}(\Lambda)=&\left[\sigma_2 \exp\left(+i\theta_i\frac{\sigma_i^*}{2}+i\omega_i\left(+i\frac{\sigma_i^*}{2}\right)\right)\right]_{aa'}\psi^{L*}_{a'}(\Lambda x) \\
=&\left[\exp\left(-i\theta_i\frac{\sigma_i^*}{2}-i\omega_i\left(+i\frac{\sigma_i^*}{2}\right)\right)\sigma_2\right]_{aa'}\psi^{L*}_{a'}(\Lambda x) \\
=&\exp\left(-i\theta_i\frac{\sigma_i^*}{2}-i\omega_i\left(+i\frac{\sigma_i^*}{2}\right)\right)_{aa'}\left(\sigma_2 \psi^{L*}\right)_{a'}(\Lambda x)
\end{align*}
となり,これはまさに$\left(0,\frac{1}{2}\right)$表現であることを示している.同様に,$(0,\frac{1}{2})$表現に属する$\psi^{R}(x)$を用いて$(\sigma_2)_{bb'}\psi^{R*}_{b'}(x)$と作った場は$(\frac{1}{2},0)$表現に属する.あるいは$\epsilon=i\sigma_2$を用いて
\begin{align*}
\epsilon_{aa'}\psi^{L*}_{a'},\quad \epsilon_{bb'}\psi^{R*}
\end{align*}
はそれぞれ$(0,\frac{1}{2})$表現と$(\frac{1}{2},0)$表現となる,と書ける.\par
最後に,$\mc{J}^{(\frac{1}{2}0)},\mc{J}^{(0\frac{1}{2})\mu\nu}$に対応する行列を求めておこう.このために,5.4節で導入したガンマ行列を
\begin{align*}
\gamma^0 =&-i\beta =-i\left(
\begin{matrix}
0 & \sigma_0 \\
\sigma_0 & 0
\end{matrix}
\right)=-i\left(
\begin{matrix}
0 & -\sigma^0 \\
-\sigma^0 & 0
\end{matrix}
\right),\quad \gamma^i=-i\left(
\begin{matrix}
0 & \sigma_i \\
-\sigma_i & 0
\end{matrix}
\right)=-i\left(
\begin{matrix}
0 & \sigma^i \\
-\sigma^i & 0
\end{matrix}
\right) \\
\gamma^\mu =&-i\left(
\begin{matrix}
0 & \sigma^\mu \\
\bar{\sigma}^\mu & 0
\end{matrix}
\right), \quad (\sigma_\mu =(\bm{\sigma},1),\bar{\sigma}_\mu=(-\bm{\sigma},1),\sigma^\mu =(\bm{\sigma},-1),\bar{\sigma}^\mu=(-\bm{\sigma},-1))
\end{align*}
と4次元版パウリ行列を用いて書き表し,このとき4成分スピノル表現が
\begin{align*}
\mc{S}^{\mu\nu}=&-\frac{i}{4}(\gamma^\mu \gamma^\nu -\gamma^\nu \gamma^\mu) \\
=&-\frac{i}{4}\left[-\left(
\begin{matrix}
0 & \sigma^\mu \\
\bar{\sigma}^\mu & 0
\end{matrix}
\right)\left(
\begin{matrix}
0 & \sigma^\nu \\
\bar{\sigma}^\nu & 0
\end{matrix}
\right)+\left(
\begin{matrix}
0 & \sigma^\nu \\
\bar{\sigma}^\nu & 0
\end{matrix}
\right)\left(
\begin{matrix}
0 & \sigma^\mu \\
\bar{\sigma}^\mu & 0
\end{matrix}
\right)\right] \\
=&\frac{i}{4}\left(
\begin{matrix}
\sigma^\mu \bar{\sigma}^\nu-\sigma^\nu \bar{\sigma}^\mu & 0 \\
0 & \bar{\sigma}^\mu\sigma^\nu-\bar{\sigma}^\nu\sigma^\mu
\end{matrix}
\right) \\
=&\frac{1}{2}\left(
\begin{matrix}
\sigma^{\mu\nu} & 0 \\
0 & \bar{\sigma}^{\mu\nu}
\end{matrix}
\right)
\end{align*}
と書く.ここで$\sigma^{\mu\nu},\bar{\sigma}^{\mu\nu}$は
\begin{align*}
\sigma^{\mu\nu}:=&\frac{i}{2}(\sigma^\mu \bar{\sigma}^\nu-\sigma^\nu \bar{\sigma}^\mu) \\
\bar{\sigma}^{\mu\nu}:=&\frac{i}{2}(\bar{\sigma}^\mu\sigma^\nu-\bar{\sigma}^\nu\sigma^\mu)
\end{align*}
と定義される.この表記から,
\begin{align*}
D(\Lambda)=&\exp\left(\frac{i}{2}\omega_{\mu\nu}\mc{S}^{\mu\nu}\right) \\
=&\exp\left(\frac{i}{2}\omega_{\mu\nu}\left(
\begin{matrix}
\sigma^{\mu\nu}/2 & 0 \\
0 & \bar{\sigma}^{\mu\nu}/2
\end{matrix}
\right)\right) \\
=&\left(
\begin{matrix}
\exp\left(\frac{i}{2}\omega_{\mu\nu}\frac{\sigma^{\mu\nu}}{2}\right) & 0 \\
0 & \exp\left(\frac{i}{2}\omega_{\mu\nu}\frac{\bar{\sigma}^{\mu\nu}}{2}\right)
\end{matrix}
\right)\\
=&\exp\left(\frac{i}{2}\omega_{\mu\nu}\frac{\sigma^{\mu\nu}}{2}\right)\oplus \exp\left(\frac{i}{2}\omega_{\mu\nu}\frac{\bar{\sigma}^{\mu\nu}}{2}\right) \\
=&D^{(\frac{1}{2}0)}(\Lambda) \oplus D^{(0\frac{1}{2})}(\Lambda)
\end{align*}
したがって,$\mc{J}^{(\frac{1}{2}0)\mu\nu}=\frac{1}{2}\sigma^{\mu\nu},\mc{J}^{(0\frac{1}{2})\mu\nu}=\frac{1}{2}\bar{\sigma}^{\mu\nu}$であることがわかる.



\vskip\baselineskip

・$A=\frac{1}{2},B=\frac{1}{2}$\par
$\left(\frac{1}{2},\frac{1}{2}\right)$表現は,$j=\frac{1}{2}+\frac{1}{2}=1$部分と$j=\frac{1}{2}-\frac{1}{2}=0$成分をもつことがわかる.前者は回転のもとで通常の3元ベクトル$\mathbf{v}$として振る舞い,後者はスカラー$v^0$として振舞うから,その組み合わせは4元ベクトル$v^\mu=(\mathbf{v},v^0)$であることがわかる.$\left(\frac{1}{2},\frac{1}{2}\right)$表現は
\begin{align*}
\bm{\mc{A}}=&\frac{\bm{\sigma}}{2},\quad \bm{\mc{B}}=\frac{\bm{\sigma}}{2} \\
D^{\left(\frac{1}{2}\frac{1}{2}\right)}(\Lambda)=&\exp\left(i(\theta_i-i\omega_i)\frac{\sigma_i}{2}\right)\exp\left(i(\theta_i+i\omega_i)\frac{\sigma_i}{2}\right) \\
=&\exp\left(i\theta_i\frac{\sigma_i}{2}+\omega_i\frac{\sigma_i}{2}\right)\exp\left(i\theta_i\frac{\sigma_i}{2}-\omega_i\frac{\sigma_i}{2}\right) \\
U_0(\Lambda)\psi_{ab}^{\left(\frac{1}{2}\frac{1}{2}\right)}(x)U^{-1}_0(\Lambda)=&D_{aa'bb'}^{\left(\frac{1}{2}\frac{1}{2}\right)}(\Lambda^{-1})\psi_{a'b'}^{\left(\frac{1}{2}\frac{1}{2}\right)}(\Lambda x) \\
=&\exp\left(-i\theta_i\frac{\sigma_i}{2}-\omega_i\frac{\sigma_i}{2}\right)_{aa'}\exp\left(-i\theta_i\frac{\sigma_i}{2}+\omega_i\frac{\sigma_i}{2}\right)_{bb'}\psi_{a'b'}^{\left(\frac{1}{2}\frac{1}{2}\right)}(\Lambda x)
\end{align*}
となる.このままではわかりにくいので,$\sigma_2$を用いて$\psi_{ab}=(\sigma_2)_{bb'}\psi'_{ab'}$とおいて
\begin{align*}
&U_0(\Lambda)\psi_{ab}^{\left(\frac{1}{2}\frac{1}{2}\right)}(x)U^{-1}_0(\Lambda)=(\sigma_2)_{bb'}U_0(\Lambda)\psi_{ab'}'^{\left(\frac{1}{2}\frac{1}{2}\right)}(x)U^{-1}_0(\Lambda) \\
=&\exp\left(-i\theta_i\frac{\sigma_i}{2}-\omega_i\frac{\sigma_i}{2}\right)_{aa'}\left[\exp\left(-i\theta_i\frac{\sigma_i}{2}+\omega_i\frac{\sigma_i}{2}\right)\sigma_2\right]_{bb'}\psi_{a'b'}'^{\left(\frac{1}{2}\frac{1}{2}\right)}(\Lambda x) \\
=&\exp\left(-i\theta_i\frac{\sigma_i}{2}-\omega_i\frac{\sigma_i}{2}\right)_{aa'}\left[\sigma_2\exp\left(+i\theta_i\frac{\sigma_i^*}{2}-\omega_i\frac{\sigma_i^*}{2}\right)\right]_{bb'}\psi_{a'b'}'^{\left(\frac{1}{2}\frac{1}{2}\right)}(\Lambda x)
\end{align*}
ここで$\sigma_2\bm{\sigma}\sigma_2=-\bm{\sigma}^*$であることを用いた.したがって
\begin{align*}
U_0(\Lambda)\psi_{ab'}'^{\left(\frac{1}{2}\frac{1}{2}\right)}(x)U^{-1}_0(\Lambda)=&\exp\left(-i\theta_i\frac{\sigma_i}{2}-\omega_i\frac{\sigma_i}{2}\right)_{aa'}\exp\left(+i\theta_i\frac{\sigma_i^*}{2}-\omega_i\frac{\sigma_i^*}{2}\right)_{bb'}\psi_{a'b'}'^{\left(\frac{1}{2}\frac{1}{2}\right)}(\Lambda x) \\
=&\exp\left(-i\theta_i\frac{\sigma_i}{2}-\omega_i\frac{\sigma_i}{2}\right)_{aa'}\left[\exp\left(-i\theta_i\frac{\sigma_i}{2}-\omega_i\frac{\sigma_i}{2}\right)\right]^\dagger_{b'b}\psi_{a'b'}'^{\left(\frac{1}{2}\frac{1}{2}\right)}(\Lambda x) \\
=&\exp\left(-i\theta_i\frac{\sigma_i}{2}-\omega_i\frac{\sigma_i}{2}\right)_{aa'}\psi_{a'b'}'^{\left(\frac{1}{2}\frac{1}{2}\right)}(\Lambda x) \left[\exp\left(-i\theta_i\frac{\sigma_i}{2}-\omega_i\frac{\sigma_i}{2}\right)^\dagger\right]_{b'b}
\end{align*}
さらに,2.7節で説明したように任意の$2\times 2$行列は$\psi'_{ab}=(\sigma_\mu)_{ab}\psi^\mu$と一意に展開できることを用いると
\begin{align*}
(\sigma_\mu)_{ab}U_0(\Lambda)\psi^\mu(x)U^{-1}_0(\Lambda)=&\exp\left(-i\theta_i\frac{\sigma_i}{2}-\omega_i\frac{\sigma_i}{2}\right)_{aa'} (\sigma_\mu)_{a'b'}\left[\exp\left(-i\theta_i\frac{\sigma_i}{2}-\omega_i\frac{\sigma_i}{2}\right)\right]^\dagger_{b'b}\psi^{\mu}(\Lambda x) \\
=&\tensor{\left[\exp\left(-\frac{i}{2}\omega_{\rho\sigma}\mc{J}^{\rho\sigma}\right)\right]}{^\nu_\mu} \sigma_\nu \psi^{\mu}(\Lambda x) \\
=&\sigma_\mu \tensor{(\Lambda^{-1})}{^\mu_\nu} \psi^{\nu}(\Lambda x)
\end{align*}
ここで,2.7節で与えた関係式
\begin{align*}
\left[\exp\left(i\theta_k \frac{\sigma_k}{2}+\beta_k \frac{\sigma_k}{2}\right)\right] \sigma_\mu \left[\exp\left(i\theta_k \frac{\sigma_k}{2}+\beta_k \frac{\sigma_k}{2}\right)\right]^\dagger=\tensor{\left[\exp\left(\frac{i}{2}\omega_{\rho\sigma}\mc{J}^{\rho\sigma}\right)\right]}{^\nu_\mu} \sigma_\nu
\end{align*}
を用いた.$\sigma_\mu$が基底をなすことを用いれば
\begin{align*}
U_0(\Lambda)\psi^\mu(x)U^{-1}_0(\Lambda)=\tensor{(\Lambda^{-1})}{^\mu_\nu} \psi^{\nu}(\Lambda x)=\tensor{\Lambda}{_\nu^\mu} \psi^{\nu}(\Lambda x)
\end{align*}
であることがわかる.以上より,$\left(\frac{1}{2},\frac{1}{2}\right)$表現にしたがって変換する場$\psi_{ab}^{(\frac{1}{2}\frac{1}{2})}$は4元ベクトル場$\psi^\mu$と
\begin{align*}
\psi_{ab}^{(\frac{1}{2}\frac{1}{2})}(x)=&-i(\sigma_2)_{bb'}(\sigma_\mu)_{ab'} \psi^\mu(x) \\
=&i(\sigma_\mu \sigma_2)_{ab} \psi^\mu(x)
\end{align*}
で結ぶことができ,(同型を同一視するという意味で)$\left(\frac{1}{2},\frac{1}{2}\right)$場は4元ベクトル場である.$\epsilon_{\frac{1}{2},-\frac{1}{2}}=+1$となる反対称テンソル$\epsilon_{ab}=i(\sigma_2)_{ab}$を用いて,別のシンプルな表式
\begin{align*}
\psi_{ab}^{(\frac{1}{2}\frac{1}{2})}(x)=(\sigma_\mu\epsilon)_{ab} \psi^\mu(x)
\end{align*}
で結ぶこともできる.逆に解くと
\begin{align*}
\psi^\mu(x)=&-\frac{1}{2}(\epsilon \sigma^\mu)_{ab}\psi_{ab}^{(\frac{1}{2}\frac{1}{2})}(x) \\
=&-\frac{1}{2}(\epsilon \sigma^\mu\epsilon \epsilon^T)_{ab}\psi_{ab}^{(\frac{1}{2}\frac{1}{2})}(x) \\
=&\frac{1}{2}((\bar{\sigma}^\mu)^T \epsilon^T)_{ab}\psi_{ab}^{(\frac{1}{2}\frac{1}{2})}(x) \\
=&\frac{1}{2}(\epsilon \bar{\sigma}^\mu)_{ba}\psi_{ab}^{(\frac{1}{2}\frac{1}{2})}(x) \\
=&\frac{1}{2}\mathrm{tr}\left[\epsilon \bar{\sigma}^\mu \psi^{(\frac{1}{2}\frac{1}{2})}(x)\right]
\end{align*}
となる.(ここで,
\begin{align*}
\sigma_2 \sigma_\mu \sigma_2 =(\sigma_2 \bm{\sigma}\sigma_2,\bm{1})=(-\bm{\sigma}^* ,1)=\bar{\sigma}_\mu^*=\bar{\sigma}_\mu^T \\
\sigma_2 \bar{\sigma}_\mu \sigma_2 =(-\sigma_2 \bm{\sigma}\sigma_2,\bm{1})=(\bm{\sigma}^* ,1)=\sigma_\mu^*=\sigma_\mu^T
\end{align*}
より$\epsilon \sigma^\mu \epsilon=-(\bar{\sigma}^\mu)^T$を用いた.)実際上に代入すると
\begin{align*}
(\sigma_\mu\epsilon)_{ab} \psi^\mu(x)=&-\frac{1}{2} (\sigma_\mu\epsilon)_{ab} (\epsilon \sigma^\mu)_{cd}\psi_{cd}^{(\frac{1}{2}\frac{1}{2})}(x) \\
=&-\frac{1}{2} (\sigma_\mu)_{ae}\epsilon_{eb} \epsilon_{cf} (\sigma^\mu)_{fd}\psi_{cd}^{(\frac{1}{2}\frac{1}{2})}(x) \\
=&(\epsilon_{af}\epsilon_{ed})\epsilon_{eb} \epsilon_{cf} \psi_{cd}^{(\frac{1}{2}\frac{1}{2})}(x) \\
=&\delta_{ac}\delta_{bd}\psi_{cd}^{(\frac{1}{2}\frac{1}{2})}(x) \\
=&\psi_{ab}^{(\frac{1}{2}\frac{1}{2})}(x)
\end{align*}
となり戻る.ここでフィールツ恒等式
\begin{align*}
(\sigma_\mu)_{ab}(\sigma^\mu)_{cd}=-2\epsilon_{ac}\epsilon_{bd}
\end{align*}
を用いた.(これを証明するには
\begin{align*}
&(\sigma_\mu)_{ab}(\sigma^\mu)_{cd}=-\sigma_0 \otimes \sigma_0 +\sigma_1\otimes \sigma_1 +\sigma_2 \otimes \sigma_2 +\sigma_3 \otimes \sigma_3 \\
=&-\left(
\begin{matrix}
1 & 0 \\
0 & 1
\end{matrix}
\right) \otimes \left(
\begin{matrix}
1 & 0 \\
0 & 1
\end{matrix}
\right) +\left(
\begin{matrix}
0 & 1 \\
1 & 0
\end{matrix}
\right) \otimes \left(
\begin{matrix}
0 & 1 \\
1 & 0
\end{matrix}
\right) +\left(
\begin{matrix}
0 & -i \\
i & 0
\end{matrix}
\right)\otimes \left(
\begin{matrix}
0 & -i \\
i & 0
\end{matrix}
\right) +\left(
\begin{matrix}
1 & 0 \\
0 & -1
\end{matrix}
\right) \otimes \left(
\begin{matrix}
1 & 0 \\
0 & -1
\end{matrix}
\right) \\
=&-\left(
\begin{matrix}
1 & 0 & 0 & 0 \\
0 & 1 & 0 & 0 \\
0 & 0 & 1 & 0 \\
0 & 0 & 0 & 1
\end{matrix}
\right) +\left(
\begin{matrix}
0 & 0 & 0 & 1 \\
0 & 0 & 1 & 0 \\
0 & 1 & 0 & 0 \\
1 & 0 & 0 & 0
\end{matrix}
\right)+\left(
\begin{matrix}
0 & 0 & 0 & -1 \\
0 & 0 & 1 & 0 \\
0 & 1 & 0 & 0 \\
-1 & 0 & 0 & 0
\end{matrix}
\right)+\left(
\begin{matrix}
1 & 0 & 0 & 0 \\
0 & -1 & 0 & 0 \\
0 & 0 & -1 & 0 \\
0 & 0 & 0 & 1
\end{matrix}
\right) \\
=&\left(
\begin{matrix}
0 & 0 & 0 & 0 \\
0 & -2 & 2 & 0 \\
0 & 2 & -2 & 0 \\
0 & 0 & 0 & 0
\end{matrix}
\right)
\end{align*}
となり,この$4\times 4$行列を$2\times 2$ブロック行列に分割すると,それぞれ
\begin{align*}
\left(
\begin{matrix}
(a=1/2,b=1/2) & (a=1/2,b=-1/2) \\
(a=-1/2,b=1/2) &(a=-1/2,b=-1/2) 
\end{matrix}
\right)
\end{align*}
であり,さらにその$2\times2$行列の中は
\begin{align*}
\left(
\begin{matrix}
(c=1/2,d=1/2) & (c=1/2,d=-1/2) \\
(c=-1/2,d=1/2) &(c=-1/2,d=-1/2) 
\end{matrix}
\right)
\end{align*}
になっていることに気付けば,この行列は
\begin{align*}
(a,b,c,d):&\left(+\frac{1}{2},+\frac{1}{2},-\frac{1}{2},-\frac{1}{2}\right) \to -2 \\
&\left(+\frac{1}{2},-\frac{1}{2},-\frac{1}{2},+\frac{1}{2}\right) \to 2 \\
&\left(-\frac{1}{2},+\frac{1}{2},+\frac{1}{2},-\frac{1}{2}\right) \to 2 \\
&\left(-\frac{1}{2},-\frac{1}{2},+\frac{1}{2},+\frac{1}{2}\right) \to -2
\end{align*}
でそれ以外ゼロなものであるとわかる.したがって$-2\epsilon_{ac}\epsilon_{bd}$であると書ける.)後に使うので,もう一つの種類のフィールツ恒等式
\begin{align*}
(\sigma_\mu)_{ab}(\sigma^\mu)_{cd}=-2\delta_{ad}\delta_{bc}
\end{align*}
も紹介しておく.(これを証明するには
\begin{align*}
&(\bar{\sigma}^\mu)_{ab}(\sigma_\mu)_{cd}=-\sigma_0 \otimes \sigma_0 -\sigma_1\otimes \sigma_1 -\sigma_2 \otimes \sigma_2 -\sigma_3 \otimes \sigma_3 \\
=&-\left(
\begin{matrix}
1 & 0 \\
0 & 1
\end{matrix}
\right) \otimes \left(
\begin{matrix}
1 & 0 \\
0 & 1
\end{matrix}
\right) -\left(
\begin{matrix}
0 & 1 \\
1 & 0
\end{matrix}
\right) \otimes \left(
\begin{matrix}
0 & 1 \\
1 & 0
\end{matrix}
\right) -\left(
\begin{matrix}
0 & -i \\
i & 0
\end{matrix}
\right)\otimes \left(
\begin{matrix}
0 & -i \\
i & 0
\end{matrix}
\right) -\left(
\begin{matrix}
1 & 0 \\
0 & -1
\end{matrix}
\right) \otimes \left(
\begin{matrix}
1 & 0 \\
0 & -1
\end{matrix}
\right) \\
=&-\left(
\begin{matrix}
1 & 0 & 0 & 0 \\
0 & 1 & 0 & 0 \\
0 & 0 & 1 & 0 \\
0 & 0 & 0 & 1
\end{matrix}
\right) -\left(
\begin{matrix}
0 & 0 & 0 & 1 \\
0 & 0 & 1 & 0 \\
0 & 1 & 0 & 0 \\
1 & 0 & 0 & 0
\end{matrix}
\right)-\left(
\begin{matrix}
0 & 0 & 0 & -1 \\
0 & 0 & 1 & 0 \\
0 & 1 & 0 & 0 \\
-1 & 0 & 0 & 0
\end{matrix}
\right)-\left(
\begin{matrix}
1 & 0 & 0 & 0 \\
0 & -1 & 0 & 0 \\
0 & 0 & -1 & 0 \\
0 & 0 & 0 & 1
\end{matrix}
\right) \\
=&\left(
\begin{matrix}
-2 & 0 & 0 & 0 \\
0 & 0 & -2 & 0 \\
0 & -2 & 0 & 0 \\
0 & 0 & 0 & -2
\end{matrix}
\right)
\end{align*}
となり,この$4\times 4$行列を$2\times 2$ブロック行列に分割すると,それぞれ
\begin{align*}
\left(
\begin{matrix}
(a=1/2,b=1/2) & (a=1/2,b=-1/2) \\
(a=-1/2,b=1/2) &(a=-1/2,b=-1/2) 
\end{matrix}
\right)
\end{align*}
であり,さらにその$2\times2$行列の中は
\begin{align*}
\left(
\begin{matrix}
(c=1/2,d=1/2) & (c=1/2,d=-1/2) \\
(c=-1/2,d=1/2) &(c=-1/2,d=-1/2) 
\end{matrix}
\right)
\end{align*}
になっていることに気付けば,この行列は
\begin{align*}
(a,b,c,d):&\left(+\frac{1}{2},+\frac{1}{2},+\frac{1}{2},+\frac{1}{2}\right) \to -2 \\
&\left(+\frac{1}{2},-\frac{1}{2},-\frac{1}{2},+\frac{1}{2}\right) \to -2 \\
&\left(-\frac{1}{2},+\frac{1}{2},+\frac{1}{2},-\frac{1}{2}\right) \to -2 \\
&\left(-\frac{1}{2},-\frac{1}{2},-\frac{1}{2},-\frac{1}{2}\right) \to -2
\end{align*}
でそれ以外ゼロなものであるとわかる.したがって$-2\delta_{ad}\delta_{bc}$であると書ける.)
\begin{tcolorbox}[title=${(\frac{1}{2},\frac{1}{2})}$表現(4元ベクトル)]
\begin{align*}
&D^{(\frac{1}{2}\frac{1}{2})}_{aa'bb'}(\Lambda)=\exp\left(+i\theta_i\frac{\sigma_i}{2}+\omega_i\frac{\sigma_i}{2}\right)_{aa'}\exp\left(+i\theta_i\frac{\sigma_i}{2}-\omega_i\frac{\sigma_i}{2}\right)_{bb'} \\
&U_0(\Lambda) \psi^{(\frac{1}{2}\frac{1}{2})}_{ab}(x) U_0^{-1}(\Lambda)=D^{(\frac{1}{2}\frac{1}{2})}_{aa'bb'}(\Lambda^{-1})\psi_{a'b'}^{\left(\frac{1}{2}\frac{1}{2}\right)}(\Lambda x) \\
&\qquad \Downarrow \psi^\mu(x):=\frac{1}{2}\mathrm{tr}\left[\epsilon \bar{\sigma}^\mu \psi^{(\frac{1}{2}\frac{1}{2})}(x)\right]  \qquad \Uparrow \psi^{(\frac{1}{2}\frac{1}{2})}_{ab}(x):=(\sigma_\mu \epsilon)_{ab} \psi^\mu(x) \\
&\tensor{D}{^\mu_\nu}(\Lambda)=\tensor{\Lambda}{^\mu_\nu} \\
&U_0(\Lambda)\psi^\mu(x) U_0^{-1}(\Lambda)=\tensor{D}{^\mu_\nu}(\Lambda^{-1})\psi^\nu(\Lambda x)
\end{align*}
\end{tcolorbox}

\vskip\baselineskip

・$A=1,B=0$\par
(1,0)表現についても与えておこう.これは$j=1$のみの成分をもち,表現行列は
\begin{align*}
J_1^{(1)}+iJ_2^{(1)}=&\left(
\begin{matrix}
0 & \sqrt{(1-0)(1+0+1)} & 0 \\
0 & 0 & \sqrt{(1-(-1))(1+(-1)+1)} \\
0 & 0 & 0
\end{matrix}
\right)=\left(
\begin{matrix}
0 & \sqrt{2} & 0 \\
0 & 0 & \sqrt{2} \\
0 & 0 & 0 
\end{matrix}
\right) \\
J_1^{(1)}-iJ_2^{(1)}=&\left(
\begin{matrix}
0 & 0 & 0\\
\sqrt{(1+(+1))(1-(+1)+1)} & 0 & 0 \\
0 & \sqrt{(1+0)(1-0+1)} & 0 
\end{matrix}
\right)=\left(
\begin{matrix}
0 & 0 & 0 \\
\sqrt{2} & 0 & 0 \\
0 & \sqrt{2} & 0
\end{matrix}
\right)\\
J_3^{(1/2)}=&\left(
\begin{matrix}
1 & 0 & 0\\
0 & 0 & 0 \\
0 & 0& -1
\end{matrix}
\right) \\
\therefore\quad J^{(1)}_1=&\frac{1}{\sqrt{2}}\left(
\begin{matrix}
0 & 1 & 0\\
1 & 0 & 1 \\
0 & 1 & 0 
\end{matrix}
\right) ,\quad J^{(1)}_2=\frac{1}{\sqrt{2}}\left(
\begin{matrix}
0 & -i & 0 \\
i & 0 & -i \\
0 & i & 0
\end{matrix}
\right) ,\quad J^{(1)}_{3}=\left(
\begin{matrix}
1 & 0 & 0 \\
0 & 0 & 0 \\
0 & 0 & -1
\end{matrix}
\right)
\end{align*}
となる.このままでは扱いづらいので,ユニタリー同値な別の表現
\begin{align*}
J_1=&\left(
\begin{matrix}
0 & 0 & 0\\
0 & 0 & -i \\
0 & i & 0 
\end{matrix}
\right) ,\quad J_2=\left(
\begin{matrix}
0 & 0 & i \\
0 & 0 & 0 \\
-i & 0 & 0
\end{matrix}
\right) ,\quad J_{3}=\left(
\begin{matrix}
0 & -i & 0 \\
i & 0 & 0 \\
0 & 0 & 0
\end{matrix}
\right)
\end{align*}
を使う.この表現はユニタリ行列
\begin{align*}
U=&\left(
\begin{matrix}
-\frac{1}{\sqrt{2}} & \frac{i}{\sqrt{2}} & 0 \\
 0& 0 & 1 \\
\frac{1}{\sqrt{2}} & \frac{i}{\sqrt{2}} & 0
\end{matrix}
\right) \\
UU^\dagger=&\left(
\begin{matrix}
-\frac{1}{\sqrt{2}} & \frac{i}{\sqrt{2}} & 0 \\
 0& 0 & 1 \\
\frac{1}{\sqrt{2}} & \frac{i}{\sqrt{2}} & 0
\end{matrix}
\right) \left(
\begin{matrix}
-\frac{1}{\sqrt{2}} & 0 & \frac{1}{\sqrt{2}} \\
-\frac{i}{\sqrt{2}}& 0 & -\frac{i}{\sqrt{2}} \\
0 & 1 & 0
\end{matrix}
\right)=\left(
\begin{matrix}
1 & 0 & 0 \\
0 & 1 & 0 \\
0 & 0 & 1
\end{matrix}
\right)=\bm{1}
\end{align*}
で移り変わる.実際
\begin{align*}
U^{-1}J_1^{(1)}U =&\frac{1}{\sqrt{2}}\left(
\begin{matrix}
-\frac{1}{\sqrt{2}} & 0 & \frac{1}{\sqrt{2}} \\
-\frac{i}{\sqrt{2}}& 0 & -\frac{i}{\sqrt{2}} \\
0 & 1 & 0
\end{matrix}
\right)\left(
\begin{matrix}
0 & 1 & 0\\
1 & 0 & 1 \\
0 & 1 & 0 
\end{matrix}
\right)\left(
\begin{matrix}
-\frac{1}{\sqrt{2}} & \frac{i}{\sqrt{2}} & 0 \\
 0& 0 & 1 \\
\frac{1}{\sqrt{2}} & \frac{i}{\sqrt{2}} & 0
\end{matrix}
\right) \\
=&\frac{1}{\sqrt{2}}\left(
\begin{matrix}
0 & 0 & 0\\
0 & -\sqrt{2}i & 0 \\
1 & 0 & 1 
\end{matrix}
\right)\left(
\begin{matrix}
-\frac{1}{\sqrt{2}} & \frac{i}{\sqrt{2}} & 0 \\
 0& 0 & 1 \\
\frac{1}{\sqrt{2}} & \frac{i}{\sqrt{2}} & 0
\end{matrix}
\right) \\
=&\left(
\begin{matrix}
0 & 0 & 0\\
0 & 0 & -i \\
0 & i & 0
\end{matrix}
\right)=J_1 \\
U^{-1}J_2^{(1)}U =&\frac{1}{\sqrt{2}}\left(
\begin{matrix}
-\frac{1}{\sqrt{2}} & 0 & \frac{1}{\sqrt{2}} \\
-\frac{i}{\sqrt{2}}& 0 & -\frac{i}{\sqrt{2}} \\
0 & 1 & 0
\end{matrix}
\right)\left(
\begin{matrix}
0 & -i & 0\\
i & 0 & -i \\
0 & i & 0 
\end{matrix}
\right)\left(
\begin{matrix}
-\frac{1}{\sqrt{2}} & \frac{i}{\sqrt{2}} & 0 \\
 0& 0 & 1 \\
\frac{1}{\sqrt{2}} & \frac{i}{\sqrt{2}} & 0
\end{matrix}
\right) \\
=&\frac{1}{\sqrt{2}}\left(
\begin{matrix}
0 & \sqrt{2}i & 0\\
0 & 0 & 0 \\
i & 0 & -i 
\end{matrix}
\right)\left(
\begin{matrix}
-\frac{1}{\sqrt{2}} & \frac{i}{\sqrt{2}} & 0 \\
 0& 0 & 1 \\
\frac{1}{\sqrt{2}} & \frac{i}{\sqrt{2}} & 0
\end{matrix}
\right) \\
=&\left(
\begin{matrix}
0 & 0 & i\\
0 & 0 & 0 \\
-i & 0 & 0
\end{matrix}
\right)=J_2 \\
U^{-1}J_3^{(1)}U =&\left(
\begin{matrix}
-\frac{1}{\sqrt{2}} & 0 & \frac{1}{\sqrt{2}} \\
-\frac{i}{\sqrt{2}}& 0 & -\frac{i}{\sqrt{2}} \\
0 & 1 & 0
\end{matrix}
\right)\left(
\begin{matrix}
1 & 0 & 0\\
0 & 0 & 0 \\
0 & 0 & -1 
\end{matrix}
\right)\left(
\begin{matrix}
-\frac{1}{\sqrt{2}} & \frac{i}{\sqrt{2}} & 0 \\
 0& 0 & 1 \\
\frac{1}{\sqrt{2}} & \frac{i}{\sqrt{2}} & 0
\end{matrix}
\right) \\
=&\left(
\begin{matrix}
-\frac{1}{\sqrt{2}} & 0 & -\frac{1}{\sqrt{2}} \\
-\frac{i}{\sqrt{2}}& 0 & \frac{i}{\sqrt{2}} \\
0 & 0 & 0
\end{matrix}
\right)\left(
\begin{matrix}
-\frac{1}{\sqrt{2}} & \frac{i}{\sqrt{2}} & 0 \\
 0& 0 & 1 \\
\frac{1}{\sqrt{2}} & \frac{i}{\sqrt{2}} & 0
\end{matrix}
\right) \\
=&\left(
\begin{matrix}
0 & -i & 0\\
i & 0 & 0 \\
0 & 0 & 0
\end{matrix}
\right)=J_3
\end{align*}
となる.したがって$D^{(10)}(\Lambda)$は
\begin{align*}
\bm{\mc{A}}=&\mathbf{J}^{(1)}1_B,\quad \bm{\mc{B}}=0 \\
D^{(10)}(\Lambda)=&\exp\Bigl(i(\theta_i-i\omega_i)J^{(1)}_i\Bigr)1_B \\
=&U \exp\Bigl(i(\theta_i-i\omega_i)J_i\Bigr)U^{-1}1_B
\end{align*}
と書きなおせる.したがって$(1,0)$表現にしたがって変換する場は
\begin{align*}
U_0(\Lambda) \psi^{(10)}_{ab}(x) U_0^{-1}(\Lambda)=&D^{(10)}_{aa'bb'}(\Lambda^{-1})\psi^{(10)}_{a'b'}(\Lambda x) \\
=&\left[ U \exp\Bigl(-i(\theta_i-i\omega_i)J_i\Bigr)U^{-1}\right]_{aa'}\psi^{(10)}_{a'b}(\Lambda x)
\end{align*}
となる.添え字$a$は3成分$a=+1,0,-1$をとり,$b$は単一の成分$b=0$しかもたないので,$U^{-1}_{ja'}\psi^{(10)}_{a'b}=\psi_{j}$と再定義して(ここで$U$は添え字$i=1,2,3$をもつ空間から$a=1,0,-1$をとるベクトル空間への変換を表すので$U_{aj}$と添え字づけた)
\begin{align*}
U_0(\Lambda) \psi_{j}(x) U_0^{-1}(\Lambda)=\exp\Bigl(-i(\theta_i-i\omega_i)J_i\Bigr)_{jk}\psi_{k}(\Lambda x)
\end{align*}
となる.ここで少し天下りであるが反対称テンソル$F_{\mu\nu}=-F_{\nu\mu}$で,かつ自己双対条件
\begin{align*}
F^{(+)\mu\nu}=&+ \frac{i}{2}\epsilon^{\mu\nu\rho\sigma}F_{\rho\sigma}^{(+)}
\end{align*}
を満たす場$F^{(+)\mu\nu}(x)$を考える.この場は$F^{(+)ij}=+i\epsilon_{ijk}F^{(+)k0}$を満たすから,自由度は$\{F^{(+)i0}\}_{i=1,2,3}$の3成分しかない.この場は2階のテンソルであることから,変換則を調べると,その無限小変換は
\begin{align*}
U_0(\Lambda)F^{(+)\mu\nu}(x)U_0^{-1}(\Lambda)=&\tensor{(\Lambda^{-1})}{^\mu_\rho}\tensor{(\Lambda^{-1})}{^\nu_\sigma}F^{(+)\rho\sigma}(\Lambda x) \\
=&F^{(+)\mu\nu}(\Lambda x)-\tensor{\omega}{^\mu_\rho}F^{(+)\rho\nu}(\Lambda x)-\tensor{\omega}{^\nu_\rho}F^{(+)\mu\rho}(\Lambda x)
\end{align*}
となり,特に$\{F^{(+)i0}\}$の変換は
\begin{align*}
U_0(\Lambda)F^{(+)i0}(x)U_0^{-1}(\Lambda)=&F^{(+)i0}(\Lambda x)-\tensor{\omega}{^i_\rho}F^{(+)\rho0}(\Lambda x)-\tensor{\omega}{^0_\rho}F^{(+)i\rho}(\Lambda x) \\
=&F^{(+)i0}(\Lambda x)-\tensor{\omega}{^i_j}F^{(+)j0}(\Lambda x)-\tensor{\omega}{^0_j}F^{(+)ij}(\Lambda x) \\
=&F^{(+)i0}(\Lambda x)-\tensor{\omega}{^i_j}F^{(+)j0}(\Lambda x)-i\tensor{\omega}{^0_j}\epsilon_{ijk}F^{(+)k0}(\Lambda x)
\end{align*}
ここで$\tensor{\omega}{^0_0}=0,F^{(-)00}=0$と反自己双対条件を用いた.$\omega_{ij}=\tensor{\omega}{^i_j}=\epsilon_{ijk}\theta_k,\omega_{i0}=\tensor{\omega}{^0_i}=\omega_i$を用いると
\begin{align*}
U_0(\Lambda)F^{(+)i0}(x)U_0^{-1}(\Lambda)=&F^{(+)i0}(\Lambda x)-\theta_k \epsilon_{ijk}F^{(+)j0}(\Lambda x)-i\omega_j\epsilon_{ijk}F^{(+)k0}(\Lambda x) \\
=&F^{(+)i0}(\Lambda x)-i\theta_j (-i\epsilon_{jik})F^{(+)k0}(\Lambda x)-\omega_j(-i\epsilon_{jik})F^{(+)k0}(\Lambda x) \\
=&F^{(-)i0}(\Lambda x)-i(\theta_j-i\omega_j)(-i\epsilon_{jik})F^{(+)k0}(\Lambda x) \\
=&F^{(-)i0}(\Lambda x)-i(\theta_j-i\omega_j)(J_j)_{ik}F^{(+)k0}(\Lambda x) \\
=&\left(1-i(\theta_j-i\omega_j)J_j\right)_{ik} F^{(+)k0}(\Lambda x)
\end{align*}
ここで,上の行列表示を見て$(J_i)_{jk}=-i\epsilon_{ijk}$であることを用いた.これを有限変換に拡張すると
\begin{align*}
U_0(\Lambda)F^{(+)i0}(x)U_0^{-1}(\Lambda)=&\exp\Bigl(-i(\theta_j-i\omega_j)J_j\Bigr)_{ik} F^{(+)k0}(\Lambda x)
\end{align*}
となる.これは明らかに自己双対な2階の反対称テンソル場$F^{(+)i0}(x)$が$(1,0)$表現にしたがって変換されることを示している.(この計算から明らかなように,自己双対性を満たす2階の反対称テンソルはローレンツ変換のもとで閉じている,すなわち実際に不変部分空間であることがわかる.)すなわち$(1,0)$表現に属する場は(同型を同一視するという意味で)自己双対2階反対称テンソル場$F^{(+)\mu\nu}$である.$(1,0)$表現は回転のもとで$j=1$の変換性を示す部分$F^{(+)j0}$のみを持ち,実際に自由度は3次元である.
\begin{tcolorbox}[title=${(1,0)}$表現(自己双対2階反対称テンソル)]
\begin{align*}
&\tensor{D}{^\mu^\nu_\rho_\sigma}(\Lambda)=\tensor{\Lambda}{^\mu_\rho}\tensor{\Lambda}{^\nu_\sigma} \\
&U_0(\Lambda)F^{(+)\mu\nu}(x) U_0^{-1}(\Lambda)=\tensor{D}{^\mu^\nu_\rho_\sigma}(\Lambda^{-1})F^{(+)\rho\sigma}(\Lambda x)
\end{align*}
\end{tcolorbox}


\vskip\baselineskip


・$A=0,B=1$\par
同様に$(0,1)$表現に属する場も与えよう.$D^{(01)}(\Lambda)$は
\begin{align*}
\bm{\mc{A}}=&0,\quad \bm{\mc{B}}=1_A \mathbf{J}^{(1)} \\
D^{(01)}(\Lambda)=&1_A\exp\Bigl(i(\theta_i+i\omega_i)J^{(1)}_i\Bigr) \\
=&1_AU \exp\Bigl(i(\theta_i-i\omega_i)J_i\Bigr)U^{-1}
\end{align*}
と書きなおせる.したがって$(0,1)$表現にしたがって変換する場は
\begin{align*}
U_0(\Lambda) \psi^{(01)}_{ab}(x) U_0^{-1}(\Lambda)=&D^{(01)}_{aa'bb'}(\Lambda^{-1})\psi^{(01)}_{a'b'}(\Lambda x) \\
=&\left[ U \exp\Bigl(-i(\theta_i+i\omega_i)J_i\Bigr)U^{-1}\right]_{bb'}\psi^{(01)}_{ab'}(\Lambda x)
\end{align*}
となる.添え字$b$は3成分$b=+1,0,-1$をとり,$a$は単一の成分$a=0$しかもたないので,$U^{-1}_{bb'}\psi^{(01)}_{ab'}=\psi_{b}$と再定義して
\begin{align*}
U_0(\Lambda) \psi_{b}(x) U_0^{-1}(\Lambda)=\exp\Bigl(-i(\theta_i+i\omega_i)J_i\Bigr)_{bb'}\psi_{b'}(\Lambda x)
\end{align*}
となる.ここで再び天下りであるが反対称テンソル$F_{\mu\nu}=-F_{\nu\mu}$で,かつ反自己双対条件
\begin{align*}
F^{(-)\mu\nu}=&- \frac{i}{2}\epsilon^{\mu\nu\rho\sigma}F_{\rho\sigma}^{(-)}
\end{align*}
を満たす場$F^{(-)\mu\nu}(x)$を考える.この場は$F^{(-)ij}=-i\epsilon_{ijk}F^{(-)k0}$を満たすから,自由度は$\{F^{(-)i0}\}_{i=1,2,3}$の3成分しかない.この場は2階のテンソルであることから,変換則を調べると,その無限小変換は
\begin{align*}
U_0(\Lambda)F^{(-)\mu\nu}(x)U_0^{-1}(\Lambda)=F^{(-)\mu\nu}(\Lambda x)-\tensor{\omega}{^\mu_\rho}F^{(-)\rho\nu}(\Lambda x)-\tensor{\omega}{^\nu_\rho}F^{(-)\mu\rho}(\Lambda x)
\end{align*}
となり,特に$\{F^{(-)i0}\}_{i=1,2,3}$の変換は
\begin{align*}
U_0(\Lambda)F^{(-)i0}(x)U_0^{-1}(\Lambda)=&F^{(-)i0}(\Lambda x)-\tensor{\omega}{^i_\rho}F^{(-)\rho0}(\Lambda x)-\tensor{\omega}{^0_\rho}F^{(-)i\rho}(\Lambda x) \\
=&F^{(-)i0}(\Lambda x)-\tensor{\omega}{^i_j}F^{(-)j0}(\Lambda x)-\tensor{\omega}{^0_j}F^{(-)ij}(\Lambda x) \\
=&F^{(-)i0}(\Lambda x)-\tensor{\omega}{^i_j}F^{(-)j0}(\Lambda x)+i\tensor{\omega}{^0_j}\epsilon_{ijk}F^{(-)k0}(\Lambda x)
\end{align*}
ここで$\tensor{\omega}{^0_0}=0,F^{(-)00}=0$と反自己双対条件を用いた.$\omega_{ij}=\tensor{\omega}{^i_j}=\epsilon_{ijk}\theta_k,\omega_{i0}=\tensor{\omega}{^0_i}=\omega_i$を用いると
\begin{align*}
U_0(\Lambda)F^{(-)i0}(x)U_0^{-1}(\Lambda)=&F^{(-)i0}(\Lambda x)-\theta_k \epsilon_{ijk}F^{(-)j0}(\Lambda x)+i\omega_j\epsilon_{ijk}F^{(-)k0}(\Lambda x) \\
=&F^{(-)i0}(\Lambda x)-i\theta_j (-i\epsilon_{jik})F^{(-)k0}(\Lambda x)+\omega_j(-i\epsilon_{jik})F^{(-)k0}(\Lambda x) \\
=&F^{(-)i0}(\Lambda x)-i(\theta_j+i\omega_j)(-i\epsilon_{jik})F^{(-)k0}(\Lambda x) \\
=&F^{(-)i0}(\Lambda x)-i(\theta_j+i\omega_j)(J_j)_{ik}F^{(-)k0}(\Lambda x) \\
=&\left(1-i(\theta_j+i\omega_j)J_j\right)_{ik} F^{(-)k0}(\Lambda x)
\end{align*}
ここで,上の行列表示を見て$(J_i)_{jk}=-i\epsilon_{ijk}$であることを用いた.これを有限変換に拡張すると
\begin{align*}
U_0(\Lambda)F^{(-)i0}(x)U_0^{-1}(\Lambda)=&\exp\Bigl(-i(\theta_j+i\omega_j)J_j\Bigr)_{ik} F^{(-)k0}(\Lambda x)
\end{align*}
となる.これは明らかに反自己双対な2階の反対称テンソル場$F^{(-)i0}(x)$が$(0,1)$表現にしたがって変換されることを示している.すなわち$(0,1)$表現に属する場は(同型を同一視するという意味で)反自己双対2階反対称テンソル場$F^{(-)\mu\nu}$である.$(0,1)$表現は回転のもとで$j=1$の変換性を示す部分$F^{(-)j0}$のみを持ち,実際に自由度は3次元である.
\begin{tcolorbox}[title=${(0,1)}$表現(反自己双対2階反対称テンソル)]
\begin{align*}
&\tensor{D}{^\mu^\nu_\rho_\sigma}(\Lambda)=\tensor{\Lambda}{^\mu_\rho}\tensor{\Lambda}{^\nu_\sigma} \\
&U_0(\Lambda)F^{(-)\mu\nu}(x) U_0^{-1}(\Lambda)=\tensor{D}{^\mu^\nu_\rho_\sigma}(\Lambda^{-1})F^{(-)\rho\sigma}(\Lambda x)
\end{align*}
\end{tcolorbox}

前にローレンツ代数の基底$J^{\mu\nu}$で行ったように,任意の2階反対称テンソル$F^{\mu\nu}$は自己双対と反自己双対に分解できる.
\begin{align*}
F^{\mu\nu}=&\frac{1}{2}\left(F^{\mu\nu}+\frac{i}{2}\epsilon^{\mu\nu\rho\sigma}F_{\rho\sigma}\right)+\frac{1}{2}\left(F^{\mu\nu}-\frac{i}{2}\epsilon^{\mu\nu\rho\sigma}F_{\rho\sigma}\right) \\
=&F^{(+)\mu\nu}+F^{(-)\mu\nu}
\end{align*}
(このように分離した第一項目と第二項目がそれぞれ自己・反自己双対条件を満たすのは,既に見た.)したがって,2階の反対称テンソル場はローレンツ群の可約表現$(1,0)\oplus (0,1)$に属している.実際に2階の反対称テンソルは自由度が6次元であり,$(1,0)\oplus (0,1)$の自由度に一致している.

\vskip\baselineskip

$\mc{A}$と$\mc{B}$はそれぞれ独立な$SU(2)$スピン行列として表されるのだったから,直積による合成則も簡単に調べることができる.一般に,$(A,B)$表現と$(A',B')$表現の直積$(A,B)\otimes (A',B')$は$(A+A',B+B'),(A+A'-1,B+B'),(A+A',B+B'-1),\cdots ,(|A-A'|,|B-B'|)$の直和に分解される.これを用いれば,既に知っている低い階数の表現から,高い階数の表現を得ることができる.\par
例えば,$\left(\frac{1}{2},0\right)\otimes \left(\frac{1}{2},0\right)=(1,0)\oplus (0,0)$と分解できる.これは
\begin{align*}
\psi_{aa'}=&\frac{1}{2}(\psi_{aa'}+\psi_{a'a})+\frac{1}{2}(\psi_{aa'}-\psi_{a'a})\\
=&\frac{1}{2}(\psi_{aa'}+\psi_{a'a})+\frac{1}{2}\epsilon_{aa'}\epsilon_{cc'}\psi_{cc'}
\end{align*}
となる.($a,a'$は,どちらも$\mc{A}_i$の$\left(\frac{1}{2},0\right)$変換性を示す添え字としている.)ここで,任意の$2\times 2$行列は$\sigma_\mu$で展開でき,特に反対称な$2\times 2$行列は$\epsilon_{\frac{1}{2},-\frac{1}{2}}=+1$となる反対称テンソル$\epsilon_{ab}=i(\sigma_2)_{ab}$に比例することを用いて
\begin{align*}
\psi_{ab}-\psi_{ba}=&C\epsilon_{ab} \\
\epsilon_{ab}(\psi_{ab}-\psi_{ba})=&2\epsilon_{ab}\psi_{ab} \\
=&C\epsilon_{ab}\epsilon_{ab}=2C \quad (両辺 \epsilon_{ab} 縮約)\\
\therefore \quad \psi_{ab}-\psi_{ba}=&\epsilon_{ab}\epsilon_{cd}\psi_{cd}
\end{align*}
となることを用いた.第二項目がスカラー$(0,0)$表現であることはすぐにわかる.実際,$\epsilon=i\sigma_2$が
\begin{align*}
&\exp\left(-i\theta_i\frac{\sigma_i}{2}-i\omega_i\left(-i\frac{\sigma_i}{2}\right)\right)\epsilon \exp\left(-i\theta_i\frac{\sigma_i}{2}-i\omega_i\left(-i\frac{\sigma_i}{2}\right)\right)^T \\
=&i\exp\left(-i\theta_i\frac{\sigma_i}{2}-i\omega_i\left(-i\frac{\sigma_i}{2}\right)\right)\sigma_2\exp\left(-i\theta_i\frac{\sigma_i}{2}-i\omega_i\left(-i\frac{\sigma_i}{2}\right)\right)^T \\
=&i\exp\left(-i\theta_i\frac{\sigma_i}{2}-i\omega_i\left(-i\frac{\sigma_i}{2}\right)\right)\sigma_2\exp\left(-i\theta_i\frac{(\sigma_i)^T}{2}-i\omega_i\left(-i\frac{(\sigma_i)^T}{2}\right)\right)\\
=&i\exp\left(-i\theta_i\frac{\sigma_i}{2}-i\omega_i\left(-i\frac{\sigma_i}{2}\right)\right)\exp\left(+i\theta_i\frac{(\sigma_i)^\dagger}{2}+i\omega_i\left(-i\frac{(\sigma_i)^\dagger}{2}\right)\right)\sigma_2 \quad \because \sigma_2 \bm{\sigma}\sigma_2=-\bm{\sigma}^*\\
=&i\exp\left(-i\theta_i\frac{\sigma_i}{2}-i\omega_i\left(-i\frac{\sigma_i}{2}\right)\right)\exp\left(+i\theta_i\frac{\sigma_i}{2}+i\omega_i\left(-i\frac{\sigma_i}{2}\right)\right)\sigma_2 \quad \because \bm{\sigma}^\dagger=\bm{\sigma} \\
=&i\sigma_2=\epsilon
\end{align*}
の意味で不変テンソルであることから
\begin{align*}
&U_0(\Lambda) \epsilon_{aa'}\epsilon_{cc'}\psi_{cc'}(x) U^{-1}_0(\Lambda) \\
=&\epsilon_{aa'}\epsilon_{cc'}U_0\psi_{cc'}(x) U_0^{-1} \\
=&\epsilon_{aa'}\epsilon_{cc'}\exp\left(-i\theta_i\frac{\sigma_i}{2}-i\omega_i\left(-i\frac{\sigma_i}{2}\right)\right)_{cd}\exp\left(-i\theta_i\frac{\sigma_i}{2}-i\omega_i\left(-i\frac{\sigma_i}{2}\right)\right)_{c'd'}\psi_{dd'}(\Lambda x) \\
=&\epsilon_{aa'}\left[\exp\left(-i\theta_i\frac{\sigma_i}{2}-i\omega_i\left(-i\frac{\sigma_i}{2}\right)\right)\epsilon \exp\left(-i\theta_i\frac{\sigma_i}{2}-i\omega_i\left(-i\frac{\sigma_i}{2}\right)\right)^T\right]_{dd'}\psi_{dd'}(\Lambda x) \\
=&\epsilon_{aa'}\epsilon_{dd'}\psi_{dd'}(\Lambda x)
\end{align*}
となる.第一項目が$(1,0)$表現に属することを見るためには,
\begin{align*}
\sigma^{\mu\nu}=&\frac{i}{2}(\sigma^\mu \bar{\sigma}^\nu-\sigma^\nu \bar{\sigma}^\mu) \\
\bar{\sigma}^{\mu\nu}=&\frac{i}{2}(\bar{\sigma}^\mu\sigma^\nu-\bar{\sigma}^\nu\sigma^\mu)
\end{align*}
と定義された$\sigma^{\mu\nu},\bar{\sigma}^{\mu\nu}$を思い出す.これらは($A_i,B_i$がそれぞれ$J^{\mu\nu}$のそれぞれ反自己双対,自己双対部分であったことを思い出せばすぐわかる通り)$\sigma^{\mu\nu}$は反自己双対性を満たし,$\bar{\sigma}^{\mu\nu}$は自己双対性を満たす.実際
\begin{align*}
\sigma^{ij}=&\frac{i}{2}(\sigma_i (-\sigma_j)-\sigma_j(- \sigma_i))=-\frac{i}{2}[\sigma_i,\sigma_j]=\sigma_{ijk}\sigma_k \\
\sigma^{i0}=&\frac{i}{2}(\sigma_i (-\bm{1})-(-\bm{1})(- \sigma_i))=-i\sigma_i \\
\therefore \quad \sigma^{ij}=&-i\epsilon_{ijk}\sigma^{k0}
\end{align*}
となり,同等の条件として
\begin{align*}
\sigma^{\mu\nu}=-\frac{i}{2}\epsilon^{\mu\nu\rho\sigma}\sigma_{\rho\sigma}
\end{align*}
を得る.したがって$\sigma^{\mu\nu}$は反自己双対性をもつ.同様の計算により$\bar{\sigma}^{\mu\nu}$は
\begin{align*}
\bar{\sigma}^{ij}=&\frac{i}{2}((-\sigma_i) \sigma_j-(-\sigma_j) \sigma_i)=-\frac{i}{2}[\sigma_i,\sigma_j]=\sigma_{ijk}\sigma_k \\
\sigma^{i0}=&\frac{i}{2}((-\sigma_i) (-\bm{1})-(-\bm{1}) \sigma_i)=i\sigma_i \\
\therefore \quad \sigma^{ij}=&+i\epsilon_{ijk}\sigma^{k0}
\end{align*}
より自己双対性
\begin{align*}
\bar{\sigma}^{\mu\nu}=+\frac{i}{2}\epsilon^{\mu\nu\rho\sigma}\bar{\sigma}_{\rho\sigma}
\end{align*}
をもつ.さらに$(\sigma^{\mu\nu}\epsilon)_{ab}$は$a,b$に関して対称行列であることがわかる.なぜなら,
\begin{align*}
\sigma_2 \sigma_\mu \sigma_2 =(\sigma_2 \bm{\sigma}\sigma_2,\bm{1})=(-\bm{\sigma}^* ,1)=\bar{\sigma}_\mu^*=\bar{\sigma}_\mu^T \\
\sigma_2 \bar{\sigma}_\mu \sigma_2 =(-\sigma_2 \bm{\sigma}\sigma_2,\bm{1})=(\bm{\sigma}^* ,1)=\sigma_\mu^*=\sigma_\mu^T
\end{align*}
と反対称性$\epsilon^T=-\epsilon,\epsilon=i\sigma_2$を用いれば
\begin{align*}
[\sigma^{\mu\nu}\epsilon]^T=&\epsilon^T (\sigma^{\mu\nu})^T \\
=&-i\sigma_2\frac{i}{2}\Bigl((\bar{\sigma}^\nu)^T(\sigma^\mu)^T-(\bar{\sigma}^\mu)^T(\sigma^\nu)^T\Bigr) \\
=&-i\frac{i}{2}\Bigl(\sigma^\nu \bar{\sigma}^\mu -\sigma^\mu \bar{\sigma}^\nu\Bigr)\sigma_2 \\
=&\frac{i}{2}\Bigl(\sigma^\mu \bar{\sigma}^\nu-\sigma^\nu \bar{\sigma}^\mu\Bigr)\epsilon \\
=&\sigma^{\mu\nu}\epsilon
\end{align*}
となるからである.同様に$\bar{\sigma}^{\mu\nu}\epsilon$も対称
\begin{align*}
[\bar{\sigma}^{\mu\nu}\epsilon]^T=&\epsilon^T (\bar{\sigma}^{\mu\nu})^T \\
=&-i\sigma_2\frac{i}{2}\Bigl((\sigma^\nu)^T(\bar{\sigma}^\mu)^T-(\sigma^\mu)^T(\bar{\sigma}^\nu)^T\Bigr) \\
=&-i\frac{i}{2}\Bigl(\bar{\sigma}^\nu \sigma^\mu -\bar{\sigma}^\mu \sigma^\nu\Bigr)\sigma_2 \\
=&\frac{i}{2}\Bigl(\bar{\sigma}^\mu \sigma^\nu-\bar{\sigma}^\nu \sigma^\mu\Bigr)\epsilon \\
=&\bar{\sigma}^{\mu\nu}\epsilon
\end{align*}
である.よって
\begin{align*}
F^{(+)\mu\nu}:=(\bar{\sigma}^{\mu\nu}\epsilon)_{aa'} \frac{1}{2}(\psi_{aa'}+\psi_{a'a})
\end{align*}
と縮約することによって自己双対な2階の反対称テンソル場を作ることができる.これは$(1,0)$表現に属すのだったから,対称スピノル部分は$(1,0)$場である.\par
他の例としては$\left(\frac{1}{2},\frac{1}{2}\right)\otimes \left(\frac{1}{2},\frac{1}{2}\right)=(1,1)\oplus (1,0)\oplus (0,1) \oplus (0,0)$と分解され
\begin{align*}
\psi^{\mu\nu}=\frac{1}{2}\left(\psi^{\mu\nu}+\psi^{\nu\mu}-\frac{1}{2}\eta^{\mu\nu}\eta_{\rho\sigma}\psi^{\rho\sigma}\right)+\frac{1}{2}\left(\psi^{\mu\nu}-\psi^{\nu\mu}\right)+\frac{1}{4}\eta^{\mu\nu}\eta_{\rho\sigma}\psi^{\rho\sigma}
\end{align*}
と分解される.第二項目は2階の反対称テンソルであり,自己双対部分と反自己双対部分に$(1,0)\oplus (0,1)$とさらに分解することができる.最後の項はスカラー$(0,0)$表現である.実際にスカラー場であることは
\begin{align*}
U_0(\Lambda)\Bigl[\eta^{\mu\nu}\eta_{\rho\sigma}\psi^{\rho\sigma}(x)\Bigr]U_0^{-1}(\Lambda)=&\eta^{\mu\nu}\eta_{\rho\sigma}U_0(\Lambda)\psi^{\rho\sigma}(x) U_0^{-1}(\Lambda) \\
=&\eta^{\mu\nu}\eta_{\rho\sigma} \tensor{(\Lambda^{-1})}{^\rho_\alpha}\tensor{(\Lambda^{-1})}{^\sigma_\beta} \psi^{\alpha\beta}(\Lambda x) \\
=&\eta^{\mu\nu}\eta_{\alpha\beta}\psi^{\alpha\beta}(\Lambda x) \quad \because \eta_{\rho\sigma}\tensor{\Lambda}{^\rho_\alpha}\tensor{\Lambda}{^\sigma_\beta}=\eta_{\alpha\beta}
\end{align*}
からわかる.第一項目が$(1,1)$表現であり,実際に2階のトレースレス\footnote{トレースレス性を要求しないと既約にならない.既約な対称テンソル$T^{\mu\nu}$が存在すると仮定すると,不変テンソルと縮約させた$\eta_{\mu\nu}T^{\mu\nu}$は(上で確かめたように)スカラーとして変換する場となり,これは$\{T^{\mu\nu}\}_{0\leq \mu,\nu \leq 3}$の線形結合で作られる不変部分空間である.したがって仮定の既約性と矛盾する.これはもっと高階のテンソルについても同様であり,$n$階の対称テンソルが既約であるためには任意の二つの添え字$\mu_i,\mu_j(i\neq j)$についてのトレースがゼロ
\begin{align*}
\eta_{{\mu_i}{\mu_j}}T^{{\mu_1}\cdots {\mu_i} \cdots {\mu_j}\cdots {\mu_n}}=0
\end{align*}
という条件を課す必要がある.そうでなければ縮約により作られた量は$n-2$階のテンソルとなり,不変部分空間をなすために既約性と矛盾することとなる.}
\begin{align*}
\eta_{\mu\nu}\left(\psi^{\mu\nu}+\psi^{\nu\mu}-\frac{1}{2}\eta^{\mu\nu}\eta_{\rho\sigma}\psi^{\rho\sigma}\right)=0
\end{align*}
かつ対称なテンソルである.$(1,1)$場は$j=2,1,0$の成分をもち,それらは$\psi_{ij},\psi_{j0}=\psi_{0j},\psi_{00}$に対応する.上と同様にして,$(1,1)$対称テンソル場として変換することは
\begin{align*}
&U_0(\Lambda)\left(\psi^{\mu\nu}(x)+\psi^{\nu\mu}(x)-\frac{1}{2}\eta^{\mu\nu}\eta_{\rho\sigma}\psi^{\rho\sigma}(x)\right) U_0^{-1}(\Lambda) \\
=&\tensor{\Lambda}{^\mu_\alpha}\tensor{\Lambda}{^\nu_\beta}\psi^{\alpha\beta}(\Lambda x)+\tensor{\Lambda}{^\nu_\beta}\tensor{\Lambda}{^\mu_\alpha}\psi^{\beta\alpha}(\Lambda x)-\frac{1}{2}\eta^{\mu\nu}\eta_{\rho\sigma}\psi^{\rho\sigma}(\Lambda x) \\
=&\tensor{\Lambda}{^\mu_\alpha}\tensor{\Lambda}{^\nu_\beta}\psi^{\alpha\beta}(\Lambda x)+\tensor{\Lambda}{^\mu_\alpha}\tensor{\Lambda}{^\nu_\beta}\psi^{\beta\alpha}(\Lambda x)-\tensor{\Lambda}{^\mu_\alpha}\tensor{\Lambda}{^\nu_\beta}\frac{1}{2}\eta^{\alpha\beta}\eta_{\rho\sigma}\psi^{\rho\sigma}(\Lambda x) \quad \because \eta_{\rho\sigma}\tensor{\Lambda}{^\rho_\alpha}\tensor{\Lambda}{^\sigma_\beta}=\eta_{\alpha\beta} \\
=&\tensor{\Lambda}{^\mu_\alpha}\tensor{\Lambda}{^\nu_\beta}\left(\psi^{\alpha\beta}(\Lambda x)+\psi^{\beta\alpha}(\Lambda x)-\frac{1}{2}\eta^{\alpha\beta}\eta_{\rho\sigma}\psi^{\rho\sigma}(\Lambda x)\right)
\end{align*}
からわかる.最後の結果は実際に$\mu,\nu$について対称テンソルになっているので,ローレンツ変換のもとで閉じていることがわかる.\par
一般に,$(A,A)$表現に属する場は,$\left(\frac{1}{2},\frac{1}{2}\right)$の$2A$個のテンソル積から適当な対称化と反対称化と適切なトレースの引き去りによって構成することができる.そのようなものは$j=2A,2A-1,\cdots ,0$の整数スピンのみを持つ項$T_{i_1i_2\cdots i_A},T_{i_12_i\cdots i_{A-1}0},\cdots ,T_{00\cdots 0}$を含んでおり,階数$2A$でトレースレスな対称テンソルに対応する.実際に次元を見てみよう.階数$2A$の対称テンソルの独立成分は\footnote{対称テンソルであるから,独立な成分は$2A$個の添え字$\mu_1,\cdots ,\mu_{2A}$の順序を区別しない(重複を許す)組の総数になる.すなわち$\mu=0,1,2,3$でラベル付けされた4つの箱があって,$2A$個の区別できない玉を箱に入れる方法は何通りかを調べる問題である.これは$\binom{(2A+4)-1}{2A}$通り存在する.}
\begin{align*}
\binom{(2A+4)-1}{2A}=\frac{(2A+3)!}{(2A)!(2A+3-2A)!}=\frac{(2A+3)!}{6(2A)!}=\frac{(2A+3)(2A+2)(2A+1)}{6}
\end{align*}
個になる.そこからトレースレス条件$\eta_{\mu_1\mu_2}T^{\mu_1\mu_2\cdots \mu_{2A}}=0$が\footnote{最初の2個の添え字についてのトレースのみを考えればいいのは,完全対称だからである.}(これは$2A-2$階の完全対称テンソルだから,その次元を上と同様に調べれば)
\begin{align*}
\binom{2A+1}{3}=\frac{(2A+1)!}{3!(2A-2)!}=\frac{(2A+1)!}{6(2A-2)!}=\frac{(2A+1)(2A)(2A-1)}{6}
\end{align*}
個ある.したがってトレースレスな完全対称テンソルの独立成分は
\begin{align*}
\frac{(2A+3)!}{6(2A)!}-\frac{(2A+1)!}{6(2A-2)!}=&\frac{(2A+3)(2A+2)(2A+1)-(2A+1)(2A)(2A-1)}{6} \\
=&\frac{(2A+1)([4A^2+10A+6]-[4A^2-2A])}{6} \\
=&(2A+1)^2
\end{align*}
個となる.これは$(A,A)$場の次元にぴったり合致する.\par
以上のことをまとめれば,$N$階の一般のテンソルは$N$個の$\left(\frac{1}{2},\frac{1}{2}\right)$4元ベクトル表現の直積として変換する.したがって上で述べたように,それは
\begin{align*}
\left(\frac{N}{2},\frac{N}{2}\right)\oplus \left(\frac{N}{2},\frac{N}{2}-1\right)\oplus \left(\frac{N}{2}-1,\frac{N}{2}\right)\oplus \cdots (0,0)
\end{align*}
と既約分解される.このようにして,$A+B$が整数である任意の既約$(A,B)$表現が構成できる.$A+B$が半整数であるようなスピン表現は,同様にこれらのテンソル表現とディラック表現$\left(\frac{1}{2},0\right)\oplus \left(0,\frac{1}{2}\right)$との直積から構成できる.例えば,ベクトル$\left(\frac{1}{2},\frac{1}{2}\right)$表現とディラック表現$\left(\frac{1}{2},0\right)\oplus \left(0,\frac{1}{2}\right)$表現の直積をとることにより,可約表現
\begin{align*}
\left(\frac{1}{2},\frac{1}{2}\right)\otimes \left[ \left(\frac{1}{2},0\right)\oplus \left(0,\frac{1}{2}\right) \right]=\left(\frac{1}{2},1\right)\oplus \left(\frac{1}{2},0\right)\oplus \left(1,\frac{1}{2}\right)\oplus \left(0,\frac{1}{2}\right)
\end{align*}
に従って変換するスピノルベクトル$\psi^\mu_\alpha(x)$が得られる.実際,$\left(\frac{1}{2},\frac{1}{2}\right)\otimes (\frac{1}{2},0)$は
\begin{align*}
\psi_{aba'}=&\frac{1}{2}(\psi_{aba'}+\psi_{a'ba})+\frac{1}{2}(\psi_{aba'}-\psi_{a'ba}) \\
=&\frac{1}{2}(\psi_{aba'}+\psi_{a'ba})+\frac{1}{2}\epsilon_{aa'}\epsilon_{cc'}\psi_{cbc'}
\end{align*}
と分解される.ここで$a,a'$は$\left(\frac{1}{2},0\right)$表現にしたがって変換する添え字であり,$b$は$\left(0,\frac{1}{2}\right)$表現にしたがって変換する添え字である.前と同様に$a,a'$について反対称な項は不変テンソル$\epsilon$に比例するようにできることを用いてた.第二項目は添え字$a,a'$についてスカラーであり,したがって$\left(0,\frac{1}{2}\right)$表現に属する場である.$\left(\frac{1}{2},\frac{1}{2}\right)$表現の添え字を$\mu$に変換するには$(\epsilon \sigma^\mu)_{ab}$を縮約すれば良いのだったから
\begin{align*}
\psi^{L\mu}_{a'}=&(\epsilon \sigma^\mu)_{ab}\psi_{aba'} \\
=&(\epsilon \sigma^\mu)_{ab}\frac{1}{2}(\psi_{aba'}+\psi_{a'ba})+(\epsilon \sigma^\mu)_{ab}\frac{1}{2}\epsilon_{aa'}\epsilon_{cc'}\psi_{cbc'} \\
=&(\epsilon \sigma^\mu)_{ab}\frac{1}{2}(\psi_{aba'}+\psi_{a'ba})+\sigma^\mu_{a'b}\frac{1}{2}\epsilon_{cc'}\psi_{cbc'}
\end{align*}
となる.同様に$\left(\frac{1}{2},\frac{1}{2}\right)\otimes (0,\frac{1}{2})$も$\epsilon \sigma^\mu \epsilon=-(\bar{\sigma}^\mu)^T$より
\begin{align*}
\psi_{abb'}=&\frac{1}{2}(\psi_{abb'}+\psi_{ab'b})+\frac{1}{2}(\psi_{abb'}-\psi_{ab'b}) \\
=&\frac{1}{2}(\psi_{aba'}+\psi_{a'ba})+\frac{1}{2}\epsilon_{bb'}\epsilon_{cc'}\psi_{acc'} \\
\psi^{R\mu}_{b'}=&(\epsilon \sigma^\mu)_{ab}\psi_{abb'} \\
=&(\epsilon \sigma^\mu)_{ab}\frac{1}{2}(\psi_{abb'}+\psi_{ab'b})+(\epsilon \sigma^\mu \epsilon)_{ab'}\frac{1}{2}\epsilon_{cc'}\psi_{acc'} \\
=&(\epsilon \sigma^\mu)_{ab}\frac{1}{2}(\psi_{abb'}+\psi_{ab'b})-\bar{\sigma}^\mu_{b'a}\frac{1}{2}\epsilon_{cc'}\psi_{acc'}
\end{align*}
となる.この二つを直和したもの
\begin{align*}
\Psi^\mu=\left(
\begin{matrix}
\psi^{L\mu} \\
\psi^{R\mu}
\end{matrix}
\right)
\end{align*}
を考え,$\gamma^\mu$をかけると
\begin{align*}
\gamma^\mu \Psi_\mu=&-i\left(
\begin{matrix}
0 & \sigma^\mu \\
\bar{\sigma}^\mu & 0
\end{matrix}
\right)\left(
\begin{matrix}
\psi^L_\mu \\
\psi^R_\mu
\end{matrix}
\right) =-i\left(
\begin{matrix}
\sigma^\mu \psi^R_\mu \\
\bar{\sigma}^\mu \psi^L_\mu
\end{matrix}
\right) \\
(\sigma^\mu)_{a'b'} \psi^R_{\mu b'}=&(\sigma^\mu)_{a'b'} (\epsilon \sigma_\mu)_{ab}\frac{1}{2}(\psi_{abb'}+\psi_{ab'b})-(\sigma^\mu \bar{\sigma}_\mu)_{a'a}\frac{1}{2}\epsilon_{cc'}\psi_{acc'} \\
=&(\sigma^\mu)_{a'b'} \epsilon_{ac} (\sigma_\mu)_{cb}\frac{1}{2}(\psi_{abb'}+\psi_{ab'b})+2\delta_{a'a}\epsilon_{cc'}\psi_{acc'} \\
=&-2\epsilon_{a'c}\epsilon_{bb'}\epsilon_{ac}\frac{1}{2}(\psi_{abb'}+\psi_{ab'b})-2\epsilon_{cc'}\psi_{a'cc'} \\
=&-2\epsilon_{cc'}\psi_{a'cc'} \\
(\bar{\sigma}^\mu)_{b'a'}\psi^L_{\mu a'}=&(\bar{\sigma}^\mu)_{b'a'}(\epsilon \sigma_\mu)_{ab}\frac{1}{2}(\psi_{aba'}+\psi_{a'ba})+(\bar{\sigma}^\mu \sigma_\mu)_{b'b}\frac{1}{2}\epsilon_{cc'}\psi_{cbc'} \\
=&(\bar{\sigma}^\mu)_{b'a'}\epsilon_{ac} (\sigma_\mu)_{cb}\frac{1}{2}(\psi_{aba'}+\psi_{a'ba})-2\delta_{b'b}\epsilon_{cc'}\psi_{cbc'} \\
=&-2\delta_{b'b}\delta_{a'c}\epsilon_{ac} \frac{1}{2}(\psi_{aba'}+\psi_{a'ba})-2\epsilon_{cc'}\psi_{cb'c'} \\
=&-2\delta_{b'b}\epsilon_{aa'} \frac{1}{2}(\psi_{aba'}+\psi_{a'ba})-2\epsilon_{cc'}\psi_{cb'c'} \\
=&-2\epsilon_{cc'}\psi_{cb'c'}
\end{align*}
ここでフィールツ恒等式および
\begin{align*}
\sigma^\mu \bar{\sigma}_\mu=\bar{\sigma}^\mu \sigma_\mu=-\bm{\sigma}^2-\bm{1}=-4\cdot \bm{1}
\end{align*}
を用いた.それぞれの最後の行では,反対称行列と対称行列の縮約が自明にゼロになることを用いている.これらは$\psi^{L\mu},\psi^{R\mu}$のちょうど第二項目$\left(0,\frac{1}{2}\right),\left(\frac{1}{2},0\right)$部分であることがわかる.すなわち$\gamma^\mu \Psi_\mu$は$\left(\frac{1}{2},0\right)\oplus \left(0,\frac{1}{2}\right)$表現に属する.$\left(\frac{1}{2},\frac{1}{2}\right)\otimes \left[ \left(\frac{1}{2},0\right)\oplus \left(0,\frac{1}{2}\right) \right]$の元$\Psi^\mu$に$\gamma^\mu \Psi_\mu=0$を条件として与えることによって,$\left(1,\frac{1}{2}\right)\oplus \left(\frac{1}{2},1\right)$の部分を分離することができる.そのように条件を課して構成された$\left(1,\frac{1}{2}\right)\oplus \left(\frac{1}{2},1\right)$可約表現に属する場$\Psi^\mu$をラリタ・シュウィンガー場という.

\vskip\baselineskip


この節ではここまで,固有順時ローレンツ群$SO(3,1)$の表現のみを考察してきた.空間反転$\tensor{\mc{P}}{^\mu_\nu}$を含むローレンツ群$O(3,1)$の任意の表現では,奇数個の空間添え字を持つテンソルの符号を逆転させる行列$D(\mc{P})=:\beta$が存在しなければならない.(2.6.7)(2.6.8)を導くときの計算を繰り返せば
\begin{align*}
\beta \bm{\mc{J}}\beta^{-1}=+\bm{\mc{J}} \\
\beta \bm{\mc{K}}\beta^{-1}=-\bm{\mc{K}}
\end{align*}
を得る.これはすなわち
\begin{align*}
\beta \bm{\mc{A}} \beta^{-1} =&\frac{1}{2}(\beta \bm{\mc{J}} \beta^{-1} +i\beta \bm{\mc{K}} \beta^{-1}) \\
=&\frac{1}{2}(\bm{\mc{J}}-i\bm{\mc{K}}) \\
=&\bm{\mc{B}} \\
\beta \bm{\mc{B}} \beta^{-1} =&\frac{1}{2}(\beta \bm{\mc{J}} \beta^{-1} -i\beta \bm{\mc{K}} \beta^{-1}) \\
=&\frac{1}{2}(\bm{\mc{J}}+i\bm{\mc{K}}) \\
=&\bm{\mc{A}} \\
\therefore \quad \beta D^{(AB)}(\Lambda) \beta^{-1} =&D^{(BA)}(\Lambda)
\end{align*}
を意味する.これにより固有順時ローレンツ群の既約$(A,B)$表現で$A\neq B$なものに対しては,空間反転を含むローレンツ群の表現を与えない.$A=B$のときはパリティを含むことができるが,それは$(0,0)$のスカラー表現,$\left(\frac{1}{2},\frac{1}{2}\right)$のベクトル表現,および$(A,A)$のトレースレスの対称テンソル表現である.$A\neq B$の場合,空間反転を含むローレンツ群の既約表現は直和$(A,B)\oplus (B,A)$であり,その次元は$2(2A+1)(2B+1)$である.$\beta$はこの二つを入れ替える表現行列をなしており,それは
\begin{align*}
\beta :&\left(
\begin{matrix}
\psi^{(AB)} \\
\psi^{(BA)}
\end{matrix}
\right)\to \left(
\begin{matrix}
\psi^{(BA)} \\
\psi^{(AB)}
\end{matrix}
\right) \\
\therefore \quad \beta =&\left(
\begin{matrix}
0 & \bm{1} \\
\bm{1} & 0
\end{matrix}
\right)
\end{align*}
というものになる.$A\neq B$の中で典型的な例の一つは$\left(\frac{1}{2},0\right)\oplus \left(0,\frac{1}{2}\right)$のディラック表現であり,その場合の$\beta$は$4\times 4$行列(5.4.29)になっている.他の例としては$(1,0)\oplus (0,1)$表現であり,それはまさに自己双対部分と反自己双対部分を含む2階の反対称テンソル$F^{\mu\nu}$である.



\newpage

\subsection{一般の因果律を満たす場}
今度は,前節で述べた一般的な既約表現$(A,B)$に従って変換する因果律を満たす場の構成に進む.添え字$\ell$はここでは範囲$a=A,A-1,\cdots, -A$と$b=B,B-1,\cdots ,-B$をとる一対の添え字$a,b$で置き換えられ,よって今の場合$\kappa,\lambda$を任意定数として場を
\begin{align*}
\psi_{ab}(x)=\sum_{\sigma}\int \frac{d^3\mathbf{p}}{(2\pi)^{3/2}}\Bigl[\kappa a(\mathbf{p},\sigma)e^{ip\cdot x}u_{ab}(\mathbf{p},\sigma)+\lambda a^{c\dagger}(\mathbf{p},\sigma)e^{-ip\cdot x}v_{ab}(\mathbf{p},\sigma)\Bigr]
\end{align*}
と書いておく.ここではこの粒子が自分自身の反粒子である可能性を残しておく.その場合$a^c(\mathbf{p},\sigma)=a(\mathbf{p},\sigma)$である.

\vskip\baselineskip


まずやるべきことは,ゼロ運動量の係数関数$u_{ab}(0,\sigma),v_{ab}(0,\sigma)$を求めることなのだった.これらを求めるための基本条件(5.1.25)(5.1.26)は,ここでは
\begin{align*}
\sum_{\bar{\sigma}} u_{\bar{a}\bar{b}}(0,\bar{\sigma})\mathbf{J}^{(j)}_{\bar{\sigma}\sigma}=&\sum_{a,b}\bm{\mc{J}}_{\bar{a}\bar{b},ab}u_{ab}(0,\sigma) \\
-\sum_{\bar{\sigma}} v_{\bar{a}\bar{b}}(0,\bar{\sigma})\mathbf{J}^{(j)*}_{\bar{\sigma}\sigma}=&\sum_{a,b}\bm{\mc{J}}_{\bar{a}\bar{b},ab}v_{ab}(0,\sigma)
\end{align*}
である.これは(5.4.14)(5.6.15)の$\bm{\mc{J}}_{\bar{a}\bar{b},ab}=\mathbf{J}^{(A)}_{\bar{a}a}\delta_{\bar{b}b}+\delta_{\bar{b}b}\mathbf{J}^{(B)}_{\bar{b}b}$を用いると
\begin{align*}
\sum_{\bar{\sigma}} u_{\bar{a}\bar{b}}(0,\bar{\sigma})\mathbf{J}^{(j)}_{\bar{\sigma}\sigma}=&\sum_{a}\mathbf{J}^{(A)}_{\bar{a}a}u_{a\bar{b}}(0,\sigma)+\sum_{b}\mathbf{J}^{(B)}_{\bar{b}b}u_{\bar{a}b}(0,\sigma) \\
-\sum_{\bar{\sigma}} v_{\bar{a}\bar{b}}(0,\bar{\sigma})\mathbf{J}^{(j)*}_{\bar{\sigma}\sigma}=&\sum_{a}\mathbf{J}^{(A)}_{\bar{a}a}v_{a\bar{b}}(0,\sigma)+\sum_{b}\mathbf{J}^{(B)}_{\bar{b}b}v_{\bar{a}b}(0,\sigma)
\end{align*}
となる.まず上の式を見てみれば,これはまさにクレブシュゴルダン係数$C_{AB}(j\sigma;ab)$を定義する条件になってることに気付く.すなわち,もし基底ベクトル$\Psi_{ab}$(量子力学では$\ket{A,a}\otimes \ket{B,b}$と書いたもの)が微小回転$R=1+\theta$のもとで
\begin{align*}
\delta \Psi_{ab}=i\sum_{\bar{a}}\bm{\theta}\cdot \mathbf{J}^{(A)}_{\bar{a}a}\Psi_{\bar{a}b}+i\sum_{\bar{b}}\bm{\theta}\cdot \mathbf{J}^{(B)}_{\bar{b}b}\Psi_{a\bar{b}}
\end{align*}
と変換するならば,同じ回転のもとで状態
\begin{align*}
\tensor{\Psi}{^j_\sigma}:= \sum_{ab}C_{AB}(j\sigma;ab)\Psi_{ab}
\end{align*}
は,範囲$j=A+B,A+B-1,\cdots,|A-B|$のスピンをもち
\begin{align*}
\delta \tensor{\Psi}{^j_\sigma}=i\sum_{\bar{\sigma}} \bm{\theta}\cdot \mathbf{J}^{(j)}_{\bar{\sigma}\sigma}\tensor{\Psi}{^j_{\bar{\sigma}}}
\end{align*}
と変換するのだった.この両辺を係数比較すると
\begin{align*}
(\mathrm{LHS})=&\sum_{ab}C_{AB}(j\sigma;ab)\delta \Psi_{ab} \\
=&i\sum_{ab}\sum_{\bar{a}}\bm{\theta}\cdot \mathbf{J}^{(A)}_{\bar{a}a}C_{AB}(j\sigma;ab)\Psi_{\bar{a}b}+i\sum_{ab}\sum_{\bar{b}}\bm{\theta}\cdot \mathbf{J}^{(B)}_{\bar{b}b}C_{AB}(j\sigma;ab)\Psi_{a\bar{b}} \\
=&i\sum_{a}\sum_{\bar{a}\bar{b}}\bm{\theta}\cdot \mathbf{J}^{(A)}_{\bar{a}a}C_{AB}(j\sigma;a\bar{b})\Psi_{\bar{a}\bar{b}}+i\sum_{b}\sum_{\bar{a}\bar{b}}\bm{\theta}\cdot \mathbf{J}^{(B)}_{\bar{b}b}C_{AB}(j\sigma;\bar{a}b)\Psi_{\bar{a}\bar{b}} \\
(\mathrm{RHS})=&i \sum_{\bar{\sigma}} \sum_{\bar{a}\bar{b}} \bm{\theta}\cdot \mathbf{J}^{(j)}_{\bar{\sigma}\sigma}C_{AB}(j\bar{\sigma};\bar{a}\bar{b})\Psi_{\bar{a}\bar{b}} \\
\therefore \quad \sum_{\bar{\sigma}} C_{AB}(j\bar{\sigma};\bar{a}\bar{b})\mathbf{J}^{(j)}_{\bar{\sigma}\sigma}=&\sum_{a}\mathbf{J}^{(A)}_{\bar{a}a}C_{AB}(j\sigma;a\bar{b})+\sum_{b}\mathbf{J}^{(B)}_{\bar{b}b}C_{AB}(j\sigma;\bar{a}b)
\end{align*}
となる.これはちょうど(5.7.2)になっている.よって可能な比例因子を除いて$u_{ab}(0,\sigma)$はちょうど$C_{AB}(j\sigma;ab)$に等しい.この定数を慣習上
\begin{align*}
u_{ab}(0,\sigma)=\frac{1}{\sqrt{2m}}C_{AB}(j\sigma;ab)
\end{align*}
となるように選ぶ.(これがスカラー$(0,0)$表現の$u(0)=1/\sqrt{2m}$を含んでいることはすぐにわかる.スピノル表現についても,まず$(\frac{1}{2},0)$表現について($j=\frac{1}{2}$しかないので)
\begin{align*}
u_{\frac{1}{2}0}\left(0,\frac{1}{2}\right)=&\frac{1}{\sqrt{2m}}C_{\frac{1}{2}0}\left(\frac{1}{2}\frac{1}{2};\frac{1}{2}0\right)=\frac{1}{\sqrt{2m}} \\
u_{-\frac{1}{2}0}\left(0,\frac{1}{2}\right)=&\frac{1}{\sqrt{2m}}C_{\frac{1}{2}0}\left(\frac{1}{2}\frac{1}{2};-\frac{1}{2}0\right)=0 \\
u_{\frac{1}{2}0}\left(0,-\frac{1}{2}\right)=&\frac{1}{\sqrt{2m}}C_{\frac{1}{2}0}\left(\frac{1}{2}-\frac{1}{2};\frac{1}{2}0\right)=0 \\
u_{-\frac{1}{2}0}\left(0,-\frac{1}{2}\right)=&\frac{1}{\sqrt{2m}}C_{\frac{1}{2}0}\left(\frac{1}{2}-\frac{1}{2};-\frac{1}{2}0\right)=\frac{1}{\sqrt{2m}} \\
\therefore \quad u^{(\frac{1}{2}0)}\left(0,\frac{1}{2}\right)=&\frac{1}{\sqrt{2m}}\left(
\begin{matrix}
1 \\
0
\end{matrix}
\right), \quad u^{(\frac{1}{2}0)}\left(0,-\frac{1}{2}\right)=\frac{1}{\sqrt{2m}}\left(
\begin{matrix}
0 \\
1
\end{matrix}
\right)
\end{align*}
となる.これは(5.5.35)のそれぞれ上2成分になっている.$(0,\frac{1}{2})$表現についても
\begin{align*}
u_{0\frac{1}{2}}\left(0,\frac{1}{2}\right)=&\frac{1}{\sqrt{2m}}C_{0\frac{1}{2}}\left(\frac{1}{2}\frac{1}{2};0\frac{1}{2}\right)=\frac{1}{\sqrt{2m}} \\
u_{0-\frac{1}{2}}\left(0,\frac{1}{2}\right)=&\frac{1}{\sqrt{2m}}C_{0\frac{1}{2}}\left(\frac{1}{2}\frac{1}{2};0-\frac{1}{2}\right)=0 \\
u_{\frac{1}{2}0}\left(0,-\frac{1}{2}\right)=&\frac{1}{\sqrt{2m}}C_{\frac{1}{2}0}\left(\frac{1}{2}-\frac{1}{2};0\frac{1}{2}\right)=0 \\
u_{0-\frac{1}{2}}\left(0,-\frac{1}{2}\right)=&\frac{1}{\sqrt{2m}}C_{0\frac{1}{2}}\left(\frac{1}{2}-\frac{1}{2};0-\frac{1}{2}\right)=\frac{1}{\sqrt{2m}} \\
\therefore \quad u^{(0\frac{1}{2})}\left(0,\frac{1}{2}\right)=&\frac{1}{\sqrt{2m}}\left(
\begin{matrix}
1 \\
0
\end{matrix}
\right), \quad u^{(0\frac{1}{2})}\left(0,-\frac{1}{2}\right)=\frac{1}{\sqrt{2m}}\left(
\begin{matrix}
0 \\
1
\end{matrix}
\right)
\end{align*}
となる.これは(5.5.35)のそれぞれ下2成分になっている.(全体の$\sqrt{m}$の因子はスケールの任意性より一致していなくてもよい.)直和をとると
\begin{align*}
\therefore \quad u\left(0,\frac{1}{2}\right)=&u^{(\frac{1}{2}0)}\left(0,\frac{1}{2}\right)\oplus u^{(0\frac{1}{2})}\left(0,\frac{1}{2}\right) =\frac{1}{\sqrt{2m}}\left(
\begin{matrix}
1 \\
0 \\
1 \\
0
\end{matrix}
\right) \\
u\left(0,-\frac{1}{2}\right)=&u^{(\frac{1}{2}0)}\left(0,-\frac{1}{2}\right)\oplus u^{(0\frac{1}{2})}\left(0,-\frac{1}{2}\right)=\frac{1}{\sqrt{2m}}\left(
\begin{matrix}
0 \\
1 \\
0 \\
1
\end{matrix}
\right)
\end{align*}
となり,全体のスケールを除いて(5.5.35)を復元する.ベクトルについても,$j=0$で
\begin{align*}
u_{\frac{1}{2}\frac{1}{2}}(0,0)=&\frac{1}{\sqrt{2m}}C_{\frac{1}{2}\frac{1}{2}}(00;\frac{1}{2}\frac{1}{2})=0 \\
u_{\frac{1}{2}-\frac{1}{2}}(0,0)=&\frac{1}{\sqrt{2m}}C_{\frac{1}{2}\frac{1}{2}}(00;\frac{1}{2}-\frac{1}{2})=\frac{1}{\sqrt{2m}}\frac{1}{\sqrt{2}} \\
u_{-\frac{1}{2}\frac{1}{2}}(0,0)=&\frac{1}{\sqrt{2m}}C_{\frac{1}{2}\frac{1}{2}}(00;-\frac{1}{2}\frac{1}{2})=-\frac{1}{\sqrt{2m}}\frac{1}{\sqrt{2}} \\
u_{-\frac{1}{2}-\frac{1}{2}}(0,0)=&\frac{1}{\sqrt{2m}}C_{\frac{1}{2}\frac{1}{2}}(00;-\frac{1}{2}-\frac{1}{2})=0 \\
\therefore \quad u_{ab}(0,0)=&\frac{1}{2\sqrt{m}}\left(
\begin{matrix}
0 & 1 \\
-1 & 0
\end{matrix}
\right)
\end{align*}
で,$j=1$では$\sigma=+1,0,-1$があるので
\begin{align*}
u_{\frac{1}{2}\frac{1}{2}}(0,0)=&\frac{1}{\sqrt{2m}}C_{\frac{1}{2}\frac{1}{2}}(10;\frac{1}{2}\frac{1}{2})=0 \\
u_{\frac{1}{2}-\frac{1}{2}}(0,0)=&\frac{1}{\sqrt{2m}}C_{\frac{1}{2}\frac{1}{2}}(10;\frac{1}{2}-\frac{1}{2})=\frac{1}{\sqrt{2m}}\frac{1}{\sqrt{2}} \\
u_{-\frac{1}{2}\frac{1}{2}}(0,0)=&\frac{1}{\sqrt{2m}}C_{\frac{1}{2}\frac{1}{2}}(10;-\frac{1}{2}\frac{1}{2})=\frac{1}{\sqrt{2m}}\frac{1}{\sqrt{2}} \\
u_{-\frac{1}{2}-\frac{1}{2}}(0,0)=&\frac{1}{\sqrt{2m}}C_{\frac{1}{2}\frac{1}{2}}(10;-\frac{1}{2}-\frac{1}{2})=0 \\
\therefore \quad u_{ab}(0,0)=&\frac{1}{2\sqrt{m}}\left(
\begin{matrix}
0 & 1 \\
1 & 0
\end{matrix}
\right)
\end{align*}
と
\begin{align*}
u_{\frac{1}{2}\frac{1}{2}}(0,1)=&\frac{1}{\sqrt{2m}}C_{\frac{1}{2}\frac{1}{2}}(11;\frac{1}{2}\frac{1}{2})=\frac{1}{\sqrt{2m}} \\
u_{\frac{1}{2}-\frac{1}{2}}(0,1)=&\frac{1}{\sqrt{2m}}C_{\frac{1}{2}\frac{1}{2}}(11;\frac{1}{2}-\frac{1}{2})=0 \\
u_{-\frac{1}{2}\frac{1}{2}}(0,1)=&\frac{1}{\sqrt{2m}}C_{\frac{1}{2}\frac{1}{2}}(11;-\frac{1}{2}\frac{1}{2})=0 \\
u_{-\frac{1}{2}-\frac{1}{2}}(0,1)=&\frac{1}{\sqrt{2m}}C_{\frac{1}{2}\frac{1}{2}}(11;-\frac{1}{2}-\frac{1}{2})=0 \\
\therefore \quad u_{ab}(0,1)=&\frac{1}{\sqrt{2m}}\left(
\begin{matrix}
1 & 0 \\
0 & 0
\end{matrix}
\right)
\end{align*}
と
\begin{align*}
u_{\frac{1}{2}\frac{1}{2}}(0,-1)=&\frac{1}{\sqrt{2m}}C_{\frac{1}{2}\frac{1}{2}}(1-1;\frac{1}{2}\frac{1}{2})=0 \\
u_{\frac{1}{2}-\frac{1}{2}}(0,-1)=&\frac{1}{\sqrt{2m}}C_{\frac{1}{2}\frac{1}{2}}(1-1;\frac{1}{2}-\frac{1}{2})=0 \\
u_{-\frac{1}{2}\frac{1}{2}}(0,-1)=&\frac{1}{\sqrt{2m}}C_{\frac{1}{2}\frac{1}{2}}(1-1;-\frac{1}{2}\frac{1}{2})=0 \\
u_{-\frac{1}{2}-\frac{1}{2}}(0,-1)=&\frac{1}{\sqrt{2m}}C_{\frac{1}{2}\frac{1}{2}}(1-1;-\frac{1}{2}-\frac{1}{2})=1 \\
\therefore \quad u_{ab}(0,-1)=&\frac{1}{\sqrt{2m}}\left(
\begin{matrix}
0 & 0 \\
0 & 1
\end{matrix}
\right)
\end{align*}
がある.これらを$\mu$添え字に直すには,前節の計算により
\begin{align*}
u^\mu(0,\sigma)=&\frac{1}{2}\mathrm{tr}[\epsilon \bar{\sigma}^\mu u(0,\sigma)] \\
=&\frac{1}{2}\mathrm{tr}[\bar{\sigma}^\mu u(0,\sigma)\epsilon ]
\end{align*}
を計算すれば良かったのだった.したがって,$j=0$の場合
\begin{align*}
u^\mu(0,0)=&\frac{1}{2}\frac{1}{\sqrt{2m}}\mathrm{tr}\left[\bar{\sigma}^\mu \left(
\begin{matrix}
0 & 1 \\
-1 & 0
\end{matrix}
\right)\left(
\begin{matrix}
0 & 1 \\
-1 & 0
\end{matrix}
\right) \right] \\
=&-\frac{1}{2}\frac{1}{\sqrt{2m}}\mathrm{tr}\left[\bar{\sigma}^\mu \left(
\begin{matrix}
1 & 0 \\
0 & 1
\end{matrix}
\right)\right] \\
=&-\frac{1}{\sqrt{2m}}\left(
\begin{matrix}
0 \\
0 \\
0 \\
1
\end{matrix}
\right)
\end{align*}
となり(最後のベクトルは$\mu=1,2,3,0$について並べたもの),これは5.3節の最初にスピンゼロについて考えたのと(スケールを除いて)同じになる.$j=1$については
\begin{align*}
u^\mu(0,0)=&\frac{1}{2}\frac{1}{2\sqrt{m}}\mathrm{tr}\left[\bar{\sigma}^\mu \left(
\begin{matrix}
0 & 1 \\
1 & 0
\end{matrix}
\right) \left(
\begin{matrix}
0 & 1 \\
-1 & 0
\end{matrix}
\right) \right] \\
=&\frac{1}{2}\frac{1}{2\sqrt{m}}\mathrm{tr}\left[\bar{\sigma}^\mu \left(
\begin{matrix}
-1 & 0 \\
0 & 1
\end{matrix}
\right)\right] \\
=&\frac{1}{2\sqrt{m}}\left(
\begin{matrix}
0 \\
0 \\
1 \\
0
\end{matrix}
\right)=\frac{1}{\sqrt{2}}\left[\frac{1}{\sqrt{2m}}\left(
\begin{matrix}
0 \\
0 \\
1 \\
0
\end{matrix}
\right)\right] \\
u^\mu(0,1)=&\frac{1}{2}\frac{1}{\sqrt{2m}}\mathrm{tr}\left[\bar{\sigma}^\mu \left(
\begin{matrix}
1 & 0 \\
0 & 0
\end{matrix}
\right) \left(
\begin{matrix}
0 & 1 \\
-1 & 0
\end{matrix}
\right)\right] \\
=&\frac{1}{2}\frac{1}{\sqrt{2m}}\mathrm{tr}\left[\bar{\sigma}^\mu \left(
\begin{matrix}
0 & 1 \\
0 & 0
\end{matrix}
\right)\right] \\
=&\frac{1}{2}\frac{1}{\sqrt{2m}}\left(
\begin{matrix}
-1 \\
-i \\
0 \\
0
\end{matrix}
\right)=-\frac{1}{2}\frac{1}{\sqrt{2m}}\left(
\begin{matrix}
1 \\
i \\
0 \\
0
\end{matrix}
\right) =\frac{1}{\sqrt{2}}\left[-\frac{1}{\sqrt{2}}\frac{1}{\sqrt{2m}}\left(
\begin{matrix}
1 \\
-i \\
0 \\
0
\end{matrix}
\right)\right]\\
u^\mu(0,-1)=&\frac{1}{2}\frac{1}{\sqrt{2m}}\mathrm{tr}\left[\bar{\sigma}^\mu \left(
\begin{matrix}
0 & 0 \\
0 & 1
\end{matrix}
\right)\left(
\begin{matrix}
0 & 1 \\
-1 & 0
\end{matrix}
\right) \right] \\
=&\frac{1}{2}\frac{1}{\sqrt{2m}}\mathrm{tr}\left[\bar{\sigma}^\mu \left(
\begin{matrix}
0 & 0 \\
-1 & 0
\end{matrix}
\right)\right] \\
=&\frac{1}{2}\frac{1}{\sqrt{2m}}\left(
\begin{matrix}
1 \\
-i \\
0 \\
0
\end{matrix}
\right)=\frac{1}{\sqrt{2}}\left[\frac{1}{\sqrt{2}}\frac{1}{\sqrt{2m}}\left(
\begin{matrix}
1 \\
-i \\
0 \\
0
\end{matrix}
\right)\right]
\end{align*}
となる.ここで,$\bar{\sigma}^\mu=(-\bm{\sigma},-1)$を用いた.これは(5.3.20)~(5.3.22)とスケール除き一致している.)5.1節で述べたように,Schurの補題よりこの$u_{ab}(0,\sigma)$は各$(A,B)$既約表現につきスケールを除いて一意的に定まる.\par
次に$v_{ab}(0,\sigma)$を決めよう.角運動量行列の複素共役は
\begin{align*}
\left(J_1^{(j)}\right)_{\sigma'\sigma}=&\frac{1}{2}\left(\delta_{\sigma',\sigma+ 1}\sqrt{(j- \sigma)(j+ \sigma+1)}+\delta_{\sigma',\sigma-1}\sqrt{(j+ \sigma)(j- \sigma+1)}\right) \\
=&\frac{1}{2}\left( \delta_{-\sigma',-\sigma- 1}\sqrt{(j+( -\sigma))(j- (-\sigma) +1)}+\delta_{-\sigma',-\sigma+1}\sqrt{(j-(- \sigma))(j+(- \sigma)+1)}\right) \\
=&\frac{1}{2}\Bigl( (-1)^{\sigma'-(\sigma-1)}\delta_{-\sigma',-\sigma+1}\sqrt{(j-(- \sigma))(j+(- \sigma)+1)} \quad \because \sigma'=\sigma-1\\
&+(-1)^{(\sigma'-(\sigma+1))}\delta_{-\sigma',-\sigma- 1}\sqrt{(j+( -\sigma))(j- (-\sigma) +1)}\Bigr) \quad \because \sigma'=\sigma+1 \\
&=-(-1)^{\sigma'-\sigma}\left(J_1^{(j)}\right)_{-\sigma',-\sigma}^* \\
\left(J_2^{(j)}\right)_{\sigma'\sigma}=&\frac{1}{2i}\left(\delta_{\sigma',\sigma+ 1}\sqrt{(j- \sigma)(j+ \sigma+1)}-\delta_{\sigma',\sigma-1}\sqrt{(j+ \sigma)(j- \sigma+1)}\right) \\
=&\frac{1}{2i}\left( \delta_{-\sigma',-\sigma- 1}\sqrt{(j+( -\sigma))(j- (-\sigma) +1)}-\delta_{-\sigma',-\sigma+1}\sqrt{(j-(- \sigma))(j+(- \sigma)+1)}\right) \\
=&\frac{1}{2i}\Bigl( -(-1)^{\sigma'-(\sigma-1)}\delta_{-\sigma',-\sigma+1}\sqrt{(j-(- \sigma))(j+(- \sigma)+1)} \quad \because \sigma'=\sigma-1\\
&+(-1)^{(\sigma'-(\sigma+1))}\delta_{-\sigma',-\sigma- 1}\sqrt{(j+( -\sigma))(j- (-\sigma) +1)}\Bigr) \\
=&(-1)^{\sigma'-\sigma}\frac{1}{2i}\Bigl( \delta_{-\sigma',-\sigma+1}\sqrt{(j-(- \sigma))(j+(- \sigma)+1)} \\
&-\delta_{-\sigma',-\sigma- 1}\sqrt{(j+( -\sigma))(j- (-\sigma) +1)}\Bigr) \quad \because \sigma'=\sigma+1 \\
=&-(-1)^{\sigma'-\sigma}\left(J_2^{(j)}\right)_{-\sigma',-\sigma}^* \\
\left(J_3^{(j)}\right)_{\sigma'\sigma}=&\delta_{\sigma'\sigma}\sigma \\
=&-(-\sigma)\delta_{-\sigma',-\sigma}=-(-1)^{\sigma'-\sigma}(-\sigma)\delta_{-\sigma',-\sigma} \\
=&-(-1)^{\sigma'-\sigma}\left(J_3^{(j)}\right)_{-\sigma',-\sigma}^*
\end{align*}
より
\begin{align*}
-\left(\mathbf{J}^{(j)}\right)^*_{\sigma'\sigma}=(-1)^{\sigma'-\sigma}\left(\mathbf{J}^{(j)}\right)_{-\sigma',-\sigma}
\end{align*}
となる.これを(5.7.3)に用いると左辺は
\begin{align*}
-\sum_{\bar{\sigma}} v_{\bar{a}\bar{b}}(0,\bar{\sigma})\mathbf{J}^{(j)*}_{\bar{\sigma}\sigma} =&\sum_{\bar{\sigma}} (-1)^{\bar{\sigma}-\sigma}v_{\bar{a}\bar{b}}(0,\bar{\sigma})\mathbf{J}^{(j)}_{-\bar{\sigma},-\sigma} \\
=&(-1)^{-\sigma}\sum_{\sigma} (-1)^{-\bar{\sigma}}v_{\bar{a}\bar{b}}(0,-\bar{\sigma})\mathbf{J}^{(j)}_{\bar{\sigma},-\sigma} \quad (\bar{\sigma}\to \bar{\sigma})
\end{align*}
となり,したがって
\begin{align*}
(-1)^{-\sigma}\sum_{\sigma} (-1)^{-\bar{\sigma}}v_{\bar{a}\bar{b}}(0,-\bar{\sigma})\mathbf{J}^{(j)}_{\bar{\sigma},-\sigma}=&\sum_{a}\mathbf{J}^{(A)}_{\bar{a}a}v_{a\bar{b}}(0,-\sigma)+\sum_{b}\mathbf{J}^{(B)}_{\bar{b}b}v_{\bar{a}b}(0,-\sigma) \\
(-1)^{\sigma}\sum_{\sigma} (-1)^{-\bar{\sigma}}v_{\bar{a}\bar{b}}(0,-\bar{\sigma})\mathbf{J}^{(j)}_{\bar{\sigma},\sigma} =&\sum_{a}\mathbf{J}^{(A)}_{\bar{a}a}v_{a\bar{b}}(0,\sigma)+\sum_{b}\mathbf{J}^{(B)}_{\bar{b}b}v_{\bar{a}b}(0,\sigma) \\
\sum_{\sigma} (-1)^{j-\bar{\sigma}}v_{\bar{a}\bar{b}}(0,-\bar{\sigma})\mathbf{J}^{(j)}_{\bar{\sigma},\sigma} =&\sum_{a}\mathbf{J}^{(A)}_{\bar{a}a}(-1)^{j-\sigma}v_{a\bar{b}}(0,\sigma)+\sum_{b}\mathbf{J}^{(B)}_{\bar{b}b}(-1)^{j-\sigma}v_{\bar{a}b}(0,\sigma)
\end{align*}
となる.これは(5.7.2)そのものであることがわかる.Schurの補題による一意性から,スケールを調節すれば$(-1)^{j-\sigma}v(0,-\sigma)$は$u(0,\sigma)$に等しくすることができて,
\begin{align*}
(-1)^{j-\sigma}v_{ab}(0,-\sigma)=&u_{ab}(0,\sigma) \\
\therefore \quad v_{ab}(0,\sigma)=&(-1)^{j+\sigma}u_{ab}(0,-\sigma)
\end{align*}
(ここで,$j,\sigma$はともに整数もしくはともに半整数であるから,$j-\sigma$は整数であり$(-1)^{2(j-\sigma)}=1$であることを用いた.)これもスカラーの場合は$v(0)=1/\sqrt{2m}$で,実際に5.2節の最初に与えたものに一致する.ディラック表現の場合,$(\frac{1}{2},0)$表現は
\begin{align*}
v^{(\frac{1}{2}0)}\left(0,\frac{1}{2}\right)=&(-1)^{\frac{1}{2}+\frac{1}{2}}u^{(\frac{1}{2}0)}\left(0,-\frac{1}{2}\right)=\frac{1}{\sqrt{2m}}\left(
\begin{matrix}
0 \\
-1
\end{matrix}
\right) \\
v^{(\frac{1}{2}0)}\left(0,-\frac{1}{2}\right)=&(-1)^{\frac{1}{2}-\frac{1}{2}}u^{(\frac{1}{2}0)}\left(0,\frac{1}{2}\right)=\frac{1}{\sqrt{2m}}\left(
\begin{matrix}
1 \\
0
\end{matrix}
\right)
\end{align*}
となって(5.5.36)の上2成分に$-\sqrt{m}$のスケールを除いて一致する.$(0,\frac{1}{2})$表現についても
\begin{align*}
v^{(0\frac{1}{2})}\left(0,\frac{1}{2}\right)=&(-1)^{\frac{1}{2}+\frac{1}{2}}u^{(0\frac{1}{2})}\left(0,-\frac{1}{2}\right)=\frac{1}{\sqrt{2m}}\left(
\begin{matrix}
0 \\
-1
\end{matrix}
\right) \\
v^{(0\frac{1}{2})}\left(0,-\frac{1}{2}\right)=&(-1)^{\frac{1}{2}-\frac{1}{2}}u^{(0\frac{1}{2})}\left(0,\frac{1}{2}\right)=\frac{1}{\sqrt{2m}}\left(
\begin{matrix}
1 \\
0
\end{matrix}
\right)
\end{align*}
となって,(5.5.36)の下2成分とスケール$\sqrt{m}$倍を除いて一致する.直和は
\begin{align*}
v\left(0,\frac{1}{2}\right)=&\left(- v^{(\frac{1}{2}0)}\left(0,\frac{1}{2}\right)\right) \oplus v^{(0\frac{1}{2})}\left(0,\frac{1}{2}\right)=\frac{1}{\sqrt{2m}}\left(
\begin{matrix}
0 \\
1 \\
0 \\
-1
\end{matrix}
\right) \\
v\left(0,-\frac{1}{2}\right)=&\left(-v^{(\frac{1}{2}0)}\left(0,-\frac{1}{2}\right)\right) \oplus v^{(0\frac{1}{2})}\left(0,-\frac{1}{2}\right)=\frac{1}{\sqrt{2m}}\left(
\begin{matrix}
1 \\
0 \\
-1 \\
0
\end{matrix}
\right)
\end{align*}
となり(5.5.36)と一致する.(各表現についてそれぞれスケールだけ任意性があるので,相対的なマイナス倍は問題はない.このマイナスはパリティの下で対称性があるように要請されるのだった.既約表現同士の相対的なスケールや位相因子は何らかの対称性で制限しなければなんでもよい.)

\vskip\baselineskip


有限運動量の係数関数を計算するためにはブーストを実行しなければならない.固定した方向$\hat{\mathbf{p}}:=\mathbf{p}/|\mathbf{p}|$について,5.5節で行ったようにして
\begin{align*}
\cosh \theta:=\frac{\sqrt{\mathbf{p}^2+m^2}}{m} ,\quad \sinh \theta=\frac{|\mathbf{p}|}{m}
\end{align*}
とおくと2.5節で与えた基準ブースト$L(p)$が
\begin{align*}
\tensor{L}{^i_k}(p)=&\delta_{ik}+(\gamma-1)\hat{p}_i \hat{p}_k=\delta_{ik}+(\cosh\theta-1)\hat{p}_i \hat{p}_k \\
\tensor{L}{^i_0}(p)=&\tensor{L}{^0_i}=\hat{p}_i \sqrt{\gamma^2-1}=\hat{p}_i\sinh \theta \\
\tensor{L}{^0_0}(p)=&\gamma=\cosh\theta
\end{align*}
となり,これは$L(\theta)L(\bar{\theta})=L(\bar{\theta}+\theta)$を満たす意味で加法的である.これを用いると
\begin{align*}
D\Bigl(L(p)\Bigr)=\exp (-i\hat{\mathbf{p}}\cdot \bm{\mc{K}}\theta)
\end{align*}
が導ける.これは斉次ローレンツ群の任意の表現について成り立つ.既約な$(A,B)$表現については(5.6.7)(5.6.8)より
\begin{align*}
i\bm{\mc{K}}=\bm{\mc{A}}-\bm{\mc{B}}
\end{align*}
であるから,$[\mc{A}_i,\mc{B}_j]=0$であるからBCH公式より
\begin{align*}
D\Bigl(L(p)\Bigr)=&\exp (-\hat{\mathbf{p}}\cdot \bm{\mc{A}}\theta+\hat{\mathbf{p}}\cdot \bm{\mc{B}}\theta) \\
=&\exp (-\hat{\mathbf{p}}\cdot \bm{\mc{A}}\theta)\exp(+\hat{\mathbf{p}}\cdot \bm{\mc{B}}\theta)
\end{align*}
となる.成分を明記すれば,$(\mc{A}_i)_{aa'bb'}=J^{(A)}_{aa'}\delta_{bb'},(\mc{B}_i)_{aa'bb'}=\delta_{aa'}J^{(B)}_{bb'}$であるから
\begin{align*}
D\Bigl(L(p)\Bigr)_{a'b',ab}=&\exp (-\hat{\mathbf{p}}\cdot \mathbf{J}^{(A)}\theta)_{a'a}\exp(+\hat{\mathbf{p}}\cdot \mathbf{J}^{(B)}\theta)_{b'b}
\end{align*}
と書ける.すると(5.7.4)(5.7.6)より有限運動量での係数関数は
\begin{align*}
u_{ab}(\mathbf{p},\sigma)=&\sqrt{\frac{m}{p^0}}\sum_{a'b'}D\Bigl(L(p)\Bigr)_{ab,a'b'}u_{a'b'}(0,\sigma) \\
=&\frac{1}{\sqrt{2p^0}}\sum_{a'b'} \exp (-\hat{\mathbf{p}}\cdot \mathbf{J}^{(A)}\theta)_{a'a}\exp(+\hat{\mathbf{p}}\cdot \mathbf{J}^{(B)}\theta)_{b'b} C_{AB}(j\sigma ;a'b')
\end{align*}
および
\begin{align*}
v_{ab}(\mathbf{p},\sigma)=&\sqrt{\frac{m}{p^0}}\sum_{a'b'}D\Bigl(L(p)\Bigr)_{ab,a'b'}v_{a'b'}(0,\sigma) \\
=&(-1)^{j+\sigma}\sqrt{\frac{m}{p^0}}\sum_{a'b'}D\Bigl(L(p)\Bigr)_{ab,a'b'}u_{a'b'}(0,-\sigma) \\
=&(-1)^{j+\sigma}u_{ab}(\mathbf{p},-\sigma)
\end{align*}
となる.


\vskip\baselineskip


この形式ではローレンツスカラーな相互作用密度を構成するのは非常に簡単になる.前節で述べたように,斉次ローレンツ群の$(A,B)$表現は直積$(A,0)\otimes (0,B)$とみなせて,
\begin{align*}
D^{(AB)}_{aa'bb'}(\Lambda^{-1})=D^{(A0)}_{aa'}(\Lambda^{-1})D^{(0B)}_{bb'}(\Lambda^{-1})
\end{align*}
であるから,一般のローレンツ変換則は
\begin{align*}
U_0(\Lambda) \psi_{ab}(x) U^{-1}_0(\Lambda)=&\sum_{a'b'}D^{(AB)}_{aa'bb'}(\Lambda^{-1}) \psi_{a'b'}(\Lambda x) \\
=&\sum_{a'b'}D^{(A0)}_{aa'}(\Lambda^{-1})D^{(0B)}_{bb'}(\Lambda^{-1}) \psi_{a'b'}(\Lambda x)
\end{align*}
となる.さらに(5.6.14)(5.6.15)より$(A,0)$および$(0,B)$表現の行列生成子はそれぞれスピン$A$とスピン$B$に対するスピン行列$\mathbf{J}^{(A)},\mathbf{J}^{(B)}$であることがわかる.このようにして
\begin{align*}
\sum_{a_1 a_2 \cdots a_n}g_{a_1a_2 \cdots a_n,b_1 b_2 \cdots b_n}\psi_{a_1 b_1}^{(A_1B_1)}(x)\psi_{a_2 b_2}^{(A_2B_2)}(x)\cdots \psi_{a_n b_n}^{(A_n B_n)}(x)
\end{align*}
の形のスカラーを,以下のやり方で構成できる.すなわち,単に係数$g_{a_1a_2 \cdots a_n,b_1 b_2 \cdots b_n}$を,スピン$A_1,A_2,\cdots ,A_n$を結合してスカラーを作るための係数とスピン$B_1,B_2,\cdots ,B_n$を結合してスカラーを作るための係数との積にとればよい\footnote{微分がかかった場は$(\frac{1}{2},\frac{1}{2})\otimes (A,B)$として振舞うから,それは再び既約分解することができるから,微分を含む相互作用の構成もこの処方で扱うことができる.}.\par
例えば,変換の型$(A_1,B_1),(A_2,B_2),(A_3,B_3)$をもつ3個の場の積から作られる最も一般的なローレンツ・スカラーは,$g$を1個の自由パラメータとして
\begin{align*}
\sum_{a_1 a_2 a_3}\left(
\begin{matrix}
A_1 & A_2 & A_3 \\
a_1 & a_2 & a_3
\end{matrix}
\right)\left(
\begin{matrix}
B_1 & B_2 & B_3 \\
b_1 & b_2 & b_3
\end{matrix}
\right)\psi_{a_1 b_1}^{(A_1B_1)}(x)\psi_{a_2 b_2}^{(A_2B_2)}(x)\psi_{a_3 b_3}^{(A_3B_3)}(x)
\end{align*}
である.ここでウィグナーの$3j$記号
\begin{align*}
\left(
\begin{matrix}
j_1 & j_2 & j_3 \\
m_1 & m_2 & m_3
\end{matrix}
\right):=&\sum_{m_3'}C_{j_1j_2}(j_3m_3' ;m_1 m_2)C_{j_3j_3}(00;m_3'm_3) \\
=&\frac{(-1)^{j_3+m_3}}{\sqrt{2j_3+1}}C_{j_1j_2}(j_3-m_3;m_1,m_2)
\end{align*}
を用いた.これは三つのスピンから回転に関してスカラーを作るための結合を表す\footnote{表記を見ればわかる通り,$j_1,j_2$を結合させて$j_3$を作り,$j_3$同士を結合させてスピン0状態を作るものである.ここで公式
\begin{align*}
C_{jj}(00;m-m)=\frac{(-1)^{j-m}}{\sqrt{2j+1}}
\end{align*}
を用いた.Weinberg量子力学講義参照.}.\par
例として,二つの$(\frac{1}{2},0)$を結合させてスカラーを作ろう.このためには
\begin{align*}
\sum_{a_1a_2}C_{\frac{1}{2}\frac{1}{2}}(00;a_1a_2)\psi^L_{a_1}(x)\psi'^L_{a_2}(x)=\sum_{a_1a_2} \frac{1}{\sqrt{2}}\epsilon_{a_1a_2}\psi^L_{a_1}(x)\psi'^L_{a_2}(x)
\end{align*}
としてやればよい.二つの$(0,\frac{1}{2})$を結合させる場合も同様であり
\begin{align*}
\sum_{b_1b_2}C_{\frac{1}{2}\frac{1}{2}}(00;b_1b_2)\psi^R_{b_1}(x)\psi'^R_{b_2}(x)=\sum_{b_1b_2} \frac{1}{\sqrt{2}}\epsilon_{b_1b_2}\psi^R_{b_1}(x)\psi'^R_{b_2}(x)
\end{align*}
となる.$(\frac{1}{2},0)$と$(0,\frac{1}{2})$を結合させてスカラーを作るには,前節より$(\epsilon\psi^{R*})_{b}$は$(\frac{1}{2},0)$として振舞うのだったから,上の構成を繰り返して
\begin{align*}
\sum_{a_1b_1}\epsilon_{a_1b_1}\epsilon_{b_1b_2}\psi^{R*}_{b_2}(x)\psi^L_{a_1}(x) =-\sum_{a}\psi^{R*}_{a}(x)\psi^L_{a}(x)
\end{align*}
となる.あるいは$(\epsilon\psi^{L*})_{a}$は$(0,\frac{1}{2})$として振舞うのだから
\begin{align*}
\sum_{a}\psi^{L*}_{a}(x)\psi^R_{a}(x)
\end{align*}
の形となる.これらを合わせて,
\begin{align*}
\bar{\psi}(x)\psi(x)=&(\psi^{L*}(x) ,\psi^{R*}(x)) \left(
\begin{matrix}
0 & 1 \\
1 & 0
\end{matrix}
\right)\left(
\begin{matrix}
\psi^L(x) \\
\psi^R(x)
\end{matrix}
\right) \\
=&\sum_{a}\Bigl[\psi^{R*}_{a}(x)\psi^L_{a}(x)+\psi^{L*}_{a}(x)\psi^R_{a}(x)\Bigr]
\end{align*}
が4成分スピノル表示で作れるスカラーであることがわかる.

\vskip\baselineskip


$S$行列がローレンツ不変であるためには,相互作用密度$\mc{H}_I(x)$が上のようにスカラーであるだけでは十分でなく,$\mc{H}_I(x)$が$\mc{H}_I(y)$と空間的な距離$x-y$で交換することも必要である.この条件をいかに満足させるかを見るには,同じ種類の粒子の二つの場,


\newpage


\subsection{$\mathsf{CPT}$定理}
有名な$\mathsf{CPT}$定理を証明する.

\newpage

\subsection{質量ゼロの場}
ここまではゼロでない質量を持つ粒子の場\uwave{のみ}を取り扱ってきた.これらの場のうち,5.2節と5.5節で議論したスカラー場やディラック場のようないくつかの場については,質量ゼロの極限に移行するのに何ら特別な問題はない.しかし,5.3節で見たように,スピン1の粒子に対するベクトル場の質量ゼロ極限をとる際には,この極限で偏極ベクトルの少なくとも一つが発散するという困難が\uwave{存在する}.実際,この節で,スピン$j\geq 1$の物理的な質量ゼロ粒子の生成・消滅演算子は,有限質量の場合には構成できる既約な$(A,B)$場の全てを構成するのには使えないことを見る.この場の型に関する特別な制限により,自然にゲージ不変性の導入に導かれる.

\vskip\baselineskip


質量がゼロでない粒子と全く同様にして,質量ゼロ粒子の一般的な自由場を,運動量$\mathbf{p}$と\uwave{ヘリシティ}$\sigma$をもつ粒子の消滅演算子$a(\mathbf{p},\sigma)$および反粒子の生成演算子$a^{c\dagger}(\mathbf{p},\sigma)$を用いて
\begin{align*}
\psi_\ell(x)=\int d^3\mathbf{p}\sum_{\sigma}\Bigl[\kappa a(\mathbf{p})u_{\ell}(x;\mathbf{p},\sigma)+\lambda a^{c\dagger}(x;\mathbf{p},\sigma)v_{\ell}(\mathbf{p},\sigma)\Bigr]
\end{align*}
として構成する.ここで$p^0=|\mathbf{p}|$であり,1粒子状態のローレンツ変換性
\begin{align*}
U(\Lambda,\alpha)\Psi_{\mathbf{p},\sigma}=e^{-i(\Lambda p)\cdot \alpha}\sqrt{\frac{(\Lambda p)^0}{p^0}}e^{i\sigma\theta(\Lambda,p)}\Psi_{\mathbf{p}_\Lambda,\sigma}
\end{align*}
より
\begin{align*}
U(\Lambda,\alpha)a^{\dagger}(\mathbf{p},\sigma)U^{-1}(\Lambda,\alpha)=&e^{-i(\Lambda p)\cdot \alpha}\sqrt{\frac{(\Lambda p)^0}{p^0}}e^{i\sigma\theta(\Lambda,p)} a^\dagger(\mathbf{p},\sigma) \\
U(\Lambda)a^{c\dagger}(\mathbf{p},\sigma)U^{-1}(\Lambda)=&e^{-i(\Lambda p)\cdot \alpha}\sqrt{\frac{(\Lambda p)^0}{p^0}}e^{i\sigma\theta(\Lambda,p)} a^{c\dagger}(\mathbf{p},\sigma)
\end{align*}
という変換性がわかる.消滅演算子に関しても
\begin{align*}
U(\Lambda,\alpha)a(\mathbf{p},\sigma)U^{-1}(\Lambda,\alpha)=&e^{i(\Lambda p)\cdot \alpha}\sqrt{\frac{(\Lambda p)^0}{p^0}}e^{-i\sigma\theta(\Lambda,p)} a(\mathbf{p},\sigma) \\
U(\Lambda,\alpha)a^{c}(\mathbf{p},\sigma)U^{-1}(\Lambda,\alpha)=&e^{i(\Lambda p)\cdot \alpha}\sqrt{\frac{(\Lambda p)^0}{p^0}}e^{-i\sigma\theta(\Lambda,p)} a^{c}(\mathbf{p},\sigma)
\end{align*}
と変換する.ここで$p_\Lambda=\Lambda p$であり,また$\theta(p,\Lambda)$は(2.5.43)で定義された角度である.よて場が斉次ローレンツ群の表現$D(\Lambda)$に従って
\begin{align*}
U(\Lambda) \psi_\ell(x) U^{-1}(\Lambda) =\sum_{\bar{\ell}} D_{\ell\bar{\ell}}(\Lambda^{-1})\psi_{\bar{\ell}}(\Lambda x)
\end{align*}
と変換するようにしたいのならば,
\begin{align*}
U(\Lambda,a)\psi^+_{\ell}(x) U^{-1}(\Lambda,a)=&\int d^3\mathbf{p} \sum_{\sigma} u_\ell(x;\mathbf{p},\sigma) U(\Lambda,\alpha)a(\mathbf{p},\sigma)U^{-1}(\Lambda,\alpha) \\
=&\int d^3\mathbf{p} \sum_{\sigma} u_\ell(x;\mathbf{p},\sigma) e^{i(\Lambda p)\cdot \alpha}\sqrt{\frac{(\Lambda p)^0}{p^0}}e^{i\sigma\theta(p,\Lambda) }a(\mathbf{p}_\Lambda,\sigma) \\
=&\int \frac{d^3\mathbf{p_\Lambda}}{(\Lambda p)^0} p^0 e^{i(\Lambda p)\cdot \alpha} \sqrt{\frac{(\Lambda p)^0}{p^0}} \sum_{\sigma} u_\ell(x;\mathbf{p},\sigma) e^{i\sigma\theta(p,\Lambda) }a(\mathbf{p}_\Lambda,\sigma) \\
=&\int d^3\mathbf{p_\Lambda} \sum_{\sigma} \left[e^{i(\Lambda p)\cdot \alpha}\sqrt{\frac{p^0}{(\Lambda p)^0}} u_\ell(x;\mathbf{p},\sigma) e^{i\sigma\theta(p,\Lambda) }\right]a(\mathbf{p}_\Lambda,\sigma) \\
=\sum_{\bar{\ell}} D_{\ell\bar{\ell}}(\Lambda^{-1})\psi^+_{\bar{\ell}}(\Lambda x+a)=&\int d^3\mathbf{p} \sum_{\sigma} \left[\sum_{\bar{\ell}} D_{\ell\bar{\ell}}(\Lambda^{-1})u_{\bar{\ell}}(\Lambda x+a;\mathbf{p},\sigma)\right]a(\mathbf{p},\sigma) \\
=&\int d^3\mathbf{p_\Lambda} \sum_{\sigma} \left[\sum_{\bar{\ell}} D_{\ell\bar{\ell}}(\Lambda^{-1})u_{\bar{\ell}}(\Lambda x+a;\mathbf{p}_\Lambda,\sigma)\right]a(\mathbf{p}_\Lambda,\sigma) \\
\end{align*}
より一般的な条件
\begin{align*}
e^{i(\Lambda p)\cdot \alpha}\sqrt{\frac{p^0}{(\Lambda p)^0}} u_\ell(x;\mathbf{p},\sigma) e^{i\sigma\theta(p,\Lambda) }=&\sum_{\bar{\ell}} D_{\ell\bar{\ell}}(\Lambda^{-1})u_{\bar{\ell}}(\Lambda x+a;\mathbf{p}_\Lambda,\sigma)
\end{align*}
が得られる.$\Lambda=1$のとき$\theta=0,D(\Lambda)=1$であるから
\begin{align*}
e^{ip\cdot \alpha} u_{\ell}(x;\mathbf{p},\sigma)=&u_{\ell}(x+a;\mathbf{p},\sigma) \\
\therefore \quad u_\ell(x;\mathbf{p},\sigma)=&e^{ip\cdot x}u_\ell(0;\mathbf{p},\sigma)=:\frac{1}{(2\pi)^{3/2}}e^{ip\cdot x} u_{\ell}(\mathbf{p},\sigma)
\end{align*}
という依存性になる.$v_\ell(x;\mathbf{p},\sigma)$に関しても同様にして
\begin{align*}
U(\Lambda,a)\psi^{c-}_{\ell}(x) U^{-1}(\Lambda,a)=&\int d^3\mathbf{p} \sum_{\sigma} v_\ell(x;\mathbf{p},\sigma) U(\Lambda,\alpha)a^{c\dagger}(\mathbf{p},\sigma)U^{-1}(\Lambda,\alpha) \\
=&\int d^3\mathbf{p} \sum_{\sigma} v_\ell(x;\mathbf{p},\sigma) e^{-i(\Lambda p)\cdot \alpha}\sqrt{\frac{(\Lambda p)^0}{p^0}}e^{-i\sigma\theta(p,\Lambda) }a^{c\dagger}(\mathbf{p}_\Lambda,\sigma) \\
=&\int \frac{d^3\mathbf{p_\Lambda}}{(\Lambda p)^0} p^0 e^{-i(\Lambda p)\cdot \alpha} \sqrt{\frac{(\Lambda p)^0}{p^0}} \sum_{\sigma} v_\ell(x;\mathbf{p},\sigma) e^{-i\sigma\theta(p,\Lambda) }a^{c\dagger}(\mathbf{p}_\Lambda,\sigma) \\
=&\int d^3\mathbf{p_\Lambda} \sum_{\sigma} \left[e^{-i(\Lambda p)\cdot \alpha}\sqrt{\frac{p^0}{(\Lambda p)^0}} v_\ell(x;\mathbf{p},\sigma) e^{-i\sigma\theta(p,\Lambda) }\right]a^{c\dagger}(\mathbf{p}_\Lambda,\sigma) \\
=\sum_{\bar{\ell}} D_{\ell\bar{\ell}}(\Lambda^{-1})\psi^{c-}_{\bar{\ell}}(\Lambda x+a)=&\int d^3\mathbf{p} \sum_{\sigma} \left[\sum_{\bar{\ell}} D_{\ell\bar{\ell}}(\Lambda^{-1})v_{\bar{\ell}}(\Lambda x+a;\mathbf{p},\sigma)\right]a^{c\dagger}(\mathbf{p},\sigma) \\
=&\int d^3\mathbf{p_\Lambda} \sum_{\sigma} \left[\sum_{\bar{\ell}} D_{\ell\bar{\ell}}(\Lambda^{-1})v_{\bar{\ell}}(\Lambda x+a;\mathbf{p}_\Lambda,\sigma)\right]a^{c\dagger}(\mathbf{p}_\Lambda,\sigma) \\
\end{align*}
より,一般的な条件
\begin{align*}
e^{-i(\Lambda p)\cdot \alpha}\sqrt{\frac{p^0}{(\Lambda p)^0}} v_\ell(x;\mathbf{p},\sigma) e^{-i\sigma\theta(p,\Lambda) }=&\sum_{\bar{\ell}} D_{\ell\bar{\ell}}(\Lambda^{-1})v_{\bar{\ell}}(\Lambda x+a;\mathbf{p}_\Lambda,\sigma)
\end{align*}
が得られ,同様に
\begin{align*}
e^{-ip\cdot \alpha} v_{\ell}(x;\mathbf{p},\sigma)=&v_{\ell}(x+a;\mathbf{p},\sigma) \\
\therefore \quad v_\ell(x;\mathbf{p},\sigma)=&e^{-ip\cdot x}v_\ell(0;\mathbf{p},\sigma)=:\frac{1}{(2\pi)^{3/2}}e^{-ip\cdot x} v_{\ell}(\mathbf{p},\sigma)
\end{align*}
という依存性になる.したがって場は
\begin{align*}
\psi_\ell(x)=\int \frac{d^3\mathbf{p}}{(2\pi)^{3/2}}\sum_{\sigma}\Bigl[\kappa a(\mathbf{p})u_{\ell}(\mathbf{p},\sigma)e^{ip\cdot x}+\lambda a^{c\dagger}(\mathbf{p},\sigma)v_{\ell}(\mathbf{p},\sigma)e^{-ip\cdot x}\Bigr]
\end{align*}
となる.また係数関数の変換性は
\begin{align*}
\sqrt{\frac{p^0}{(\Lambda p)^0}} u_\ell(\mathbf{p},\sigma) e^{i\sigma\theta(p,\Lambda) }=&\sum_{\bar{\ell}} D_{\ell\bar{\ell}}(\Lambda^{-1})u_{\bar{\ell}}(\mathbf{p}_\Lambda,\sigma) \\
\sqrt{\frac{p^0}{(\Lambda p)^0}} v_\ell(\mathbf{p},\sigma) e^{-i\sigma\theta(p,\Lambda) }=&\sum_{\bar{\ell}} D_{\ell\bar{\ell}}(\Lambda^{-1})v_{\bar{\ell}}(\mathbf{p}_\Lambda,\sigma
)
\end{align*}
つまり
\begin{align*}
\therefore \quad u_\ell(\mathbf{p}_\Lambda,\sigma) e^{-i\sigma\theta(p,\Lambda)}=&\sqrt{\frac{p^0}{(\Lambda p)^0}}\sum_{\bar{\ell}} D_{\ell\bar{\ell}}(\Lambda)u_{\bar{\ell}}(\mathbf{p},\sigma) \\
v_\ell(\mathbf{p}_\Lambda,\sigma)e^{i\sigma\theta(p,\Lambda)} =&\sqrt{\frac{p^0}{(\Lambda p)^0}}\sum_{\bar{\ell}} D_{\ell\bar{\ell}}(\Lambda)v_{\bar{\ell}}(\mathbf{p},\sigma)
\end{align*}
となる.基準運動量として$k=(0,0,\kappa,\kappa)$と,基準ブースト$\Lambda=L(p)$とおくと
\begin{align*}
u_{\ell}(\mathbf{p},\sigma)=\sqrt{\frac{|\mathbf{k}|}{p^0}} \sum_{\bar{\ell}}D_{\ell\bar{\ell}}(L(p))u_{\bar{\ell}}(\mathbf{k},\sigma) \\
v_{\ell}(\mathbf{p},\sigma)=\sqrt{\frac{|\mathbf{k}|}{p^0}} \sum_{\bar{\ell}}D_{\ell\bar{\ell}}(L(p))v_{\bar{\ell}}(\mathbf{k},\sigma)
\end{align*}
となる.また,$\mathbf{p}$を基準運動量$\mathbf{k}$とし,$\Lambda$を小群の任意の元$W(\Lambda,p)=L^{-1}(\Lambda p)\Lambda L(p)$ととれば,$(Wk)^0=k^0$より
\begin{align*}
u_\ell(\mathbf{k},\sigma) e^{-i\sigma\theta}=&\sqrt{\frac{k^0}{(W k)^0}}\sum_{\bar{\ell}} D_{\ell\bar{\ell}}\Bigl(W(\Lambda,p)\Bigr)u_{\bar{\ell}}(\mathbf{k},\sigma) \\
=&\sum_{\bar{\ell}} D_{\ell\bar{\ell}}\Bigl(W(\Lambda,p)\Bigr)u_{\bar{\ell}}(\mathbf{k},\sigma) \\
v_\ell(\mathbf{k},\sigma)e^{i\sigma\theta} =&\sqrt{\frac{k^0}{(W k)^0}}\sum_{\bar{\ell}} D_{\ell\bar{\ell}}\Bigl(W(\Lambda,p)\Bigr)v_{\bar{\ell}}(\mathbf{p},\sigma) \\
=&\sum_{\bar{\ell}} D_{\ell\bar{\ell}}\Bigl(W(\Lambda,p)\Bigr)v_{\bar{\ell}}(\mathbf{p},\sigma)
\end{align*}
となる.(ここで,
\begin{align*}
W(W(\Lambda,p),k)=&L^{-1}(W(\Lambda,p)k) W(\Lambda,p) L(k) \\
=&L^{-1}(k) W(\Lambda,p) L(k) \\
=&W(\Lambda,p) \quad \because L(k)=1
\end{align*}
より$\theta(W(\Lambda,p),k)=\theta(\Lambda,p)$を用いた.)この関係式を満たすような$u(\mathbf{k},\sigma),v(\mathbf{k},\sigma)$を見つけ,それをブーストすることにより一般運動量における係数関数が求まる.

\vskip\baselineskip


(2.5.28)の小群の2種類の元を別々に考察することにより,(5.9.10)(5.9.11)の内容を引き出せる.
\begin{align*}
W(\alpha,\beta,\theta)=S(\alpha,\beta)R(\theta)
\end{align*}
であるから,$\alpha=\beta=0$では,$W$は(2.5.27)で与えられる$z$軸周りの角度$\theta$の回転
\begin{align*}
\tensor{R}{^\mu_\nu}(\theta)=\left(
\begin{matrix}
\cos\theta & \sin\theta & 0 & 0 \\
-\sin\theta & \cos\theta & 0 & 0 \\
0 & 0 & 1 & 0 \\
0 & 0 & 0 & 1
\end{matrix}
\right)
\end{align*}
であり,そのとき
\begin{align*}
u_{\ell}(\mathbf{k},\sigma)e^{i\sigma\theta(k,W)}=&\sum_{\bar{\ell}}D_{\ell\bar{\ell}}\Bigl(R(\theta)\Bigr)u_{\bar{\ell}} (\mathbf{k},\sigma) \\
v_{\ell}(\mathbf{k},\sigma)e^{-i\sigma\theta(k,W)}=&\sum_{\bar{\ell}}D_{\ell\bar{\ell}}\Bigl(R(\theta)\Bigr)v_{\bar{\ell}} (\mathbf{k},\sigma)
\end{align*}
となる.$\theta=0$の場合,(2.5.26)で与えられる$x-y$平面の回転とブーストの組み合わせ
\begin{align*}
\tensor{S}{^\mu_\nu}(\alpha,\beta)=\left(
\begin{matrix}
1 & 0 & -\alpha & \alpha \\
0 & 1 & -\beta & \beta \\
\alpha & \beta & 1-\gamma & \gamma \\
\alpha & \beta &-\gamma & 1+\gamma
\end{matrix}
\right) ,\quad \gamma=\frac{\alpha^2+\beta^2}{2}
\end{align*}
に対しては
\begin{align*}
u_{\ell}(\mathbf{k},\sigma)=&\sum_{\bar{\ell}}D_{\ell\bar{\ell}}\Bigl(S(\alpha,\beta)\Bigr)u_{\bar{\ell}} (\mathbf{k},\sigma) \\
v_{\ell}(\mathbf{k},\sigma)=&\sum_{\bar{\ell}}D_{\ell\bar{\ell}}\Bigl(S(\alpha,\beta)\Bigr)v_{\bar{\ell}} (\mathbf{k},\sigma)
\end{align*}
となる.もし$D(\Lambda)$がローレンツ群の実表現ならば,$v$についての条件(5.9.13)(5.9.15)はちょうど$u$についての方程式の複素共役になっている.したがってスケールの定数$\kappa,\lambda$を適当に調節すれば係数関数を
\begin{align*}
v_\ell(\mathbf{k},\sigma)=&u^*_{\ell}(\mathbf{k},\sigma) \\
v_\ell(\mathbf{p},\sigma)=&\sqrt{\frac{|\mathbf{k}|}{p^0}} \sum_{\bar{\ell}}D_{\ell\bar{\ell}}(L(p))v_{\bar{\ell}}(\mathbf{k},\sigma) \\
=&\sqrt{\frac{|\mathbf{k}|}{p^0}} \left[\sum_{\bar{\ell}}D_{\ell\bar{\ell}}(L(p))u_{\bar{\ell}}(\mathbf{k},\sigma)\right]^* \\
=&u^*_{\ell}(\mathbf{p},\sigma)
\end{align*}
と選ぶことができる.\par
問題は,斉次ローレンツ群の一般の表現$D(\Lambda)$に対して,(5.9.12)(5.9.14)を同時に満たす$u_\ell$を求めることができないことである.これは,$m\neq 0$の場合では可能だった表現ですら,できない.

\vskip\baselineskip

実際に何が問題となるのかを見るために,ヘリシティ$\sigma=+1,-1$の質量ゼロ粒子に対する4元ベクトル$(\frac{1}{2},\frac{1}{2})$場を構成してみよう.4元ベクトル表現は
\begin{align*}
\tensor{D}{^\mu_\nu}(\Lambda)=\tensor{\Lambda}{^\mu_\nu}
\end{align*}
で与えられるのだった.(これは実表現だから実際に$v$を$u^*$とできる.)ここでの係数関数$u_\mu$は,偏極ベクトル$e_\mu$を用いて
\begin{align*}
u_\mu(\mathbf{p},\sigma)=:\frac{1}{\sqrt{2p^0}}e_\mu(\mathbf{p},\sigma) \quad (\sigma=+1,-1)
\end{align*}
と書くのが慣習である.するとこの偏極ベクトル$e_\mu(\mathbf{p},\sigma)$は(5.9.8)より,基準運動量$\mathbf{k}$における偏極ベクトル$e_\mu(\mathbf{k},\sigma)$から
\begin{align*}
\frac{1}{\sqrt{2p^0}}e_\mu(\mathbf{p},\sigma)=&u_\mu(\mathbf{p},\sigma) \\
=&\sqrt{\frac{|\mathbf{k}|}{p^0}}\tensor{L}{^\mu_\nu}(\mathbf{p})u^\nu(\mathbf{k},\sigma) \\
=&\sqrt{\frac{|\mathbf{k}|}{p^0}}\tensor{L}{^\mu_\nu}(\mathbf{p})\frac{1}{\sqrt{2k^0}}e^\nu(\mathbf{k},\sigma) \\
=&\frac{1}{\sqrt{2p^0}}e^\nu(\mathbf{k},\sigma) \quad \because k^0=|\mathbf{k}|=\kappa \\
\therefore \quad e^\mu(\mathbf{p},\sigma)=&\tensor{L}{^\mu_\nu}(p)e^\nu(\mathbf{k},\sigma)
\end{align*}
として得られる.この$e^\mu(\mathbf{k},\sigma)$は条件(5.9.12)(5.9.14)より
\begin{align*}
e^\mu(\mathbf{k},\sigma)e^{i\sigma \theta}=&\tensor{R}{^\mu_\nu}(\theta)e^\nu(\mathbf{k},\sigma) \\
e^{\mu}(\mathbf{k},\sigma)=&\tensor{S}{^\mu_\nu}(\alpha,\beta)e^{\nu}(\mathbf{k},\sigma)
\end{align*}
を解くことにより得られる.前者の式は$\sigma=+1,-1$それぞれで調べることで解ける.$\sigma=1$では
\begin{align*}
\left(
\begin{matrix}
e^1 \\
e^2 \\
e^3 \\
e^0
\end{matrix}
\right)e^{i\theta}=&\left(
\begin{matrix}
\cos\theta & \sin\theta & 0 & 0 \\
-\sin\theta & \cos\theta & 0 & 0 \\
0 & 0 & 1 & 0 \\
0 & 0 & 0 & 1
\end{matrix}
\right)\left(
\begin{matrix}
e^1 \\
e^2 \\
e^3 \\
e^0
\end{matrix}
\right) \\
=&\left(
\begin{matrix}
e^1 \cos\theta +e^2 \sin\theta \\
e^2 \cos\theta -e^1 \sin\theta \\
e^3 \\
e^0
\end{matrix}
\right) \\
e^1(\cos\theta +i \sin\theta)=&e^1 \cos\theta +e^2 \sin\theta \\
e^2(\cos\theta +i \sin\theta)=&e^2 \cos\theta -e^1 \sin\theta \\
\therefore\quad & e^3=e^0=0 ,\quad e^1=ie^2
\end{align*}
定数$\kappa,\lambda$に吸収できる全体のスケールを調整して規格化をすると
\begin{align*}
e^\mu(\mathbf{k},+1)=\frac{1}{\sqrt{2}}\left(
\begin{matrix}
1 \\
+i \\
0 \\
0
\end{matrix}
\right)
\end{align*}
となる.$\sigma=-1$についても
\begin{align*}
\left(
\begin{matrix}
e^1 \\
e^2 \\
e^3 \\
e^0
\end{matrix}
\right)e^{-i\theta}=&\left(
\begin{matrix}
\cos\theta & \sin\theta & 0 & 0 \\
-\sin\theta & \cos\theta & 0 & 0 \\
0 & 0 & 1 & 0 \\
0 & 0 & 0 & 1
\end{matrix}
\right)\left(
\begin{matrix}
e^1 \\
e^2 \\
e^3 \\
e^0
\end{matrix}
\right) \\
=&\left(
\begin{matrix}
e^1 \cos\theta +e^2 \sin\theta \\
e^2 \cos\theta -e^1 \sin\theta \\
e^3 \\
e^0
\end{matrix}
\right) \\
e^1(\cos\theta -i \sin\theta)=&e^1 \cos\theta +e^2 \sin\theta \\
e^2(\cos\theta -i \sin\theta)=&e^2 \cos\theta -e^1 \sin\theta \\
\therefore\quad & e^3=e^0=0 ,\quad e^1=-ie^2
\end{align*}
より
\begin{align*}
e^\mu(\mathbf{k},+1)=\frac{1}{\sqrt{2}}\left(
\begin{matrix}
1 \\
+i \\
0 \\
0
\end{matrix}
\right)
\end{align*}
となる.すなわち
\begin{align*}
e^\mu(\mathbf{k},\pm 1)=\frac{1}{\sqrt{2}}\left(
\begin{matrix}
1 \\
\pm i \\
0 \\
0
\end{matrix}
\right)
\end{align*}
が(5.9.12)より要求される.しかし,そうすると第二の条件(5.9.20)より
\begin{align*}
\tensor{S}{^\mu_\nu}(\alpha,\beta)e^{\nu}(\mathbf{k},\sigma)=&\frac{1}{\sqrt{2}}\left(
\begin{matrix}
1 & 0 & -\alpha & \alpha \\
0 & 1 & -\beta & \beta \\
\alpha & \beta & 1-\gamma & \gamma \\
\alpha & \beta &-\gamma & 1+\gamma
\end{matrix}
\right)\left(
\begin{matrix}
1 \\
\pm i \\
0 \\
0
\end{matrix}
\right) \\
=&\frac{1}{\sqrt{2}}\left(
\begin{matrix}
1 \\
\pm i \\
\alpha\pm i\beta \\
\alpha\pm i\beta
\end{matrix}
\right) \\
=e^\mu(\mathbf{k},\pm 1)=&\frac{1}{\sqrt{2}}\left(
\begin{matrix}
1 \\
\pm i \\
0 \\
0
\end{matrix}
\right) \\
\therefore \quad \alpha \pm i\beta=&0
\end{align*}
となり,$\alpha \pm i\beta=0$を要求するが,それは明らかに一般の実数$\alpha,\beta$については不可能である.つまり,基本的な要求(5.9.14)が満たせず,したがって(5.9.10)が満たせない!その代わりに
\begin{align*}
\tensor{S}{^\mu_\nu}(\alpha,\beta)e^{\nu}(\mathbf{k},\sigma)=&\frac{1}{\sqrt{2}}\left(
\begin{matrix}
1 \\
\pm i \\
\alpha\pm i\beta \\
\alpha\pm i\beta
\end{matrix}
\right) \\
=&\frac{1}{\sqrt{2}}\left(
\begin{matrix}
1 \\
\pm i \\
0 \\
0
\end{matrix}
\right)+\frac{1}{\sqrt{2}\kappa }(\alpha \pm i\beta ) \left(
\begin{matrix}
0 \\
0 \\
\kappa \\
\kappa
\end{matrix}
\right) \\
=&e^\mu(\mathbf{k},\pm 1) +\frac{\alpha \pm i\beta }{\sqrt{2}|\mathbf{k}|}k^\mu
\end{align*}
であるから,$W(\theta,\alpha,\beta)=S(\alpha,\beta)R(\theta)$と併用して
\begin{align*}
\tensor{D}{^\mu_\nu}\Bigl( W(\theta ,\alpha,\beta)\Bigr)e^\nu(\mathbf{k},\pm 1)=&\tensor{S}{^\mu_\lambda}(\alpha,\beta) \tensor{R}{^\lambda_\nu}(\theta)e^{\nu}(\mathbf{k},\pm 1) \\
=&\tensor{S}{^\mu_\lambda}(\alpha,\beta) e^{\lambda}(\mathbf{k},\pm 1)e^{\pm i\theta } \\
=&e^{\pm i\theta }\left\{e^\mu(\mathbf{k},\pm 1) +\frac{\alpha \pm i\beta }{\sqrt{2}|\mathbf{k}|}k^\mu\right\} \neq e^{\pm i\theta}e^\mu(\mathbf{k},\pm 1)
\end{align*}
が得られる.このようにして,質量ゼロでヘリシティ$\sigma=\pm 1$の粒子の消滅・生成演算子から4元ベクトル場を構成することはできないという結論に到達する!

\vskip\baselineskip

とりあえずこの困難は置いといて,先へ進む.(5.9.21)を(5.9.18)でブーストして,任意の運動量における偏極ベクトル$e_\mu(\mathbf{p},\sigma)$を定義し,場を
\begin{align*}
a_\mu(x)=\int \frac{d^3\mathbf{p}}{(2\pi)^{3/2}\sqrt{2p^0}}\sum_{\sigma=\pm 1}\Bigl[e_\mu(\mathbf{p},\sigma)e^{ip\cdot x}a(\mathbf{p},\sigma)+e_\mu(\mathbf{p},\sigma)^* e^{-ip\cdot x}a^{c\dagger}(\mathbf{p},\sigma)\Bigr]
\end{align*}
ととる.


\end{document}