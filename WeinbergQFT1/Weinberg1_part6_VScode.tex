\documentclass[dvipdfmx]{jsarticle}
\let\headfont=\gtfamily
\usepackage[dvips]{graphicx}
\usepackage{amsmath}
\usepackage{mathrsfs} % 花文字\mathscr{M}, 筆記体\mathcal{M}, 黒板文字\mathbb{M},ドイツ文字\mathfrak{M}
\usepackage{bm} %太文字
\usepackage{amssymb}
\usepackage{latexsym}
\usepackage{braket}
\usepackage{tikz}
\usepackage{tikz-feynhand}
\usepackage{ulem}
\usepackage{tensor}
\usepackage{bigdelim}
\usepackage{multirow}
\usepackage{tcolorbox}
\usepackage{here}
\tcbuselibrary{theorems,skins}
\usetikzlibrary{decorations}
\usepackage{color}
\usepackage{xcolor} %ここからハイパーリンクのパッケージの設定
\usepackage[%
  dvipdfmx,%
  bookmarks=true,%
  bookmarksnumbered=true,%
  colorlinks=true,%
  allcolors=blue%
]{hyperref}
\usepackage{pxjahyper}
\usepackage{cite} %ここまで

\usetikzlibrary{intersections, calc, arrows.meta}
 \usetikzlibrary{patterns}

\newfont{\bg}{cmr9 scaled\magstep4}
\newcommand{\bigzerol}{\smash{\lower1.0ex\hbox{\bg 0}}}
\newcommand{\bigzerou}{%
   \smash{\hbox{\bg 0}}}
\newcommand{\mcO}{\mathcal{O}}
\newcommand{\VAC}{\mathrm{VAC}}
\newcommand{\Slash}[1]{{\ooalign{\hfil/\hfil\crcr$#1$}}} %ファインマンのスラッシュ記号
\renewcommand{\mc}{\mathcal}

\tcbset{
  colback=blue!5!white,  % 背景色
  colframe=blue!75!black, % 枠線の色
  fonttitle=\bfseries,    % タイトルの太字
  boxrule=0.8pt,          % 枠線の太さ
  arc=2mm,                % 角の丸み
  left=4mm, right=4mm, top=2mm, bottom=2mm, % 余白
}

\newtcolorbox{sectiongoal}[1][]{title=この節でやるべきこと,#1}

% \textrm{Roman デフォルト}
% \textgt{Gothic 和文ゴシック体}*専門用語に
% \textbf{Boldface 太字}*専門用語(英語)に
% \textit{Italic 斜体}
% \textsl{Slanted ローマンを傾けただけ}
% \textsf{Sans Serif サンセリフ体}
% \texttt{Typewriter タイプライタ体、等幅}
% \textsc{Small Caps 小文字が大文字に}

\setlength{\textwidth}{\fullwidth}
\setlength{\textheight}{44\baselineskip}
\addtolength{\textheight}{\topskip}
\setlength{\voffset}{-0.6in}

\allowdisplaybreaks[4]

\makeatletter
  \renewcommand{\theequation}
  {\arabic{section}.\arabic{equation}}
  \@addtoreset{equation}{section}
 \makeatother

\title{\vspace{-1cm}\Huge{WeinbergQFT Part5}}
\author{坂井 啓悟(Sakai Keigo)}
\date{}
\begin{document}


\maketitle

\tableofcontents

\newpage

\setcounter{part}{5}
\part{ファインマン図}
\setcounter{section}{5}
\setcounter{subsection}{0}
\subsection{ファインマン則の導出}
正規順序積同士の積についての便利な公式を与えておこう.\par
再び4.2節で導入した生成・消滅演算子の省略記法
\begin{align*}
a(q):=a(\mathbf{p}\sigma n)
\end{align*}
を用いて,生成消滅演算子の生成汎関数
\begin{align*}
N[\eta,\xi]\equiv :\exp\left(\int dq \eta(q)a^\dagger(q)+\int dq a(q)\xi(q)\right):
\end{align*}
を考える.ここで$\eta(q)=\eta(\mathbf{p},\sigma,n),\xi(q)=\xi(\mathbf{p},\sigma,n)$は,$n$がボゾン的な粒子の種類であれば互いに可換なc数であり,$n$がフェルミオン的な粒子の種類であれば,ボゾン的なものとは可換だがフェルミオン的なもの同士では反可換なc数とする.任意の生成・消滅演算子からなる正規順序多項式$:F[a,a^\dagger]:$は,適当な微分作用素を用いてこの生成汎関数から得ることができる.すなわち,ある形式的な作用素$\hat{F}\left(\frac{\delta}{\delta \eta},\frac{\delta}{\delta \xi}\right)$を用いて
\begin{align*}
:F[a,a^\dagger]:=\left. \hat{F}\left(\frac{\overrightarrow{\delta}}{\delta \eta},\frac{\overleftarrow{\delta}}{\delta \xi}\right) N[\eta,\xi] \right|_{\eta=\xi=0}
\end{align*}
と書ける.例えば
\begin{align*}
:a(q_1)a^\dagger(q_2) a(q_3):= \pm a^\dagger(q_2) a(q_1) a(q_3)=\left. \pm \frac{\overrightarrow{\delta}}{\delta \eta(q_1)} N[\eta,\xi] \frac{\overleftarrow{\delta}}{\delta \xi(q_1)} \frac{\overleftarrow{\delta}}{\delta \xi(q_3)}\right|_{\eta=\xi=0}
\end{align*}
として得られる.さて,二つの生成汎関数同士の積は
\begin{align*}
&N[\eta_1 ,\xi_1] N[\eta_2,\xi_2] \\
=&:\exp\left(\int dq \eta_1(q)a^\dagger(q)+\int dq \xi_1(q) a(q)\right)::\exp\left(\int dp \eta_2(q)a^\dagger(q)+\int dq a(q)\xi_2(q)\right): \\
=&e^{\int dq \eta_1(q)a^\dagger(q)}e^{\int dq a(q)\xi_1(q)}e^{\int dq \eta_2(q)a^\dagger(q)}e^{\int dq a(q)\xi_2(q)} \\
=&e^{\int dq \eta_1(q)a^\dagger(q)}e^{\int dq \eta_2(q)a^\dagger(q)}e^{\int dq a(q)\xi_1(q)}e^{\int dq a(q)\xi_2(q)}\exp\left(\int dq dq' \xi_1(q)[a(q),a^\dagger(q')]_{\mp}\eta_2(q')\right) \\
=&e^{\int dq (\eta_1(q)+\eta_2(q))a^\dagger(q)}e^{\int dq a(q)(\xi_1(q)+ \xi_2(q))}\exp\left(\int dq dq' \xi_1(q)[a(q),a^\dagger(q')]_{\mp}\eta_2(q')\right) \\
=&e^{\int dq (\eta_1(q)+\eta_2(q))a^\dagger(q)}e^{\int dq a(q)(\xi_1(q)+ \xi_2(q))}\exp\left(\int dq dq' \frac{\overleftarrow{\delta}}{\delta a_1(q)}[a(q),a^\dagger(q')]_{\mp}\frac{\overrightarrow{\delta}}{\delta a^\dagger_2 (q')}\right) \\
=&N[\eta_1+\eta_2,\xi_1+\xi_2]\exp\left(\int dq dq' \frac{\overleftarrow{\delta}}{\delta a_1(q)}[a(q),a^\dagger(q')]_{\mp}\frac{\overrightarrow{\delta}}{\delta a^\dagger_2 (q')}\right) \\
=&N[\eta_1+\eta_2,\xi_1+\xi_2]\exp\left(\int dq dq' \frac{\overleftarrow{\delta}}{\delta a_1(q)}[a(q),a^\dagger(q')]_{\mp}\frac{\overrightarrow{\delta}}{\delta a^\dagger_2 (q')} \right)
\end{align*}
となる.(ただし$\frac{\overleftarrow{\delta}}{\delta a_1(q)}$と$\frac{\overrightarrow{\delta}}{\delta a^\dagger_2 (q')}$は$\eta_1(q),\xi_2(q)$の係数にある$a(q)$と$a^\dagger(q)$にのみ微分が作用することを意味する.)ここで
\begin{align*}
:\exp\left(\int dq \eta(q)a^\dagger(q)+\int dq a(q)\xi(q)\right):=&e^{\int dq \eta(q)a^\dagger(q)}e^{\int dq a(q)\xi(q)}
\end{align*}
であることと,BCH公式より
\begin{align*}
&e^{\int dq a(q)\xi_1(q)}e^{\int dq \eta_2(q)a^\dagger(q)} \\
=&e^{\int dq \eta_2(q)a^\dagger(q)}\left[e^{-\int dq \eta_2(q)a^\dagger(q)}e^{\int dq a(q)\xi_1(q)}e^{\int dq \eta_2(q)a^\dagger(q)}\right] \\
=&e^{\int dq \eta_2(q)a^\dagger(q)}\exp\left(\int dq a(q)\xi_1(q)-\int dq dq' [\eta_2(q')a^\dagger(q'),a(q)\xi_1(q)]+\cdots\right) \\
=&e^{\int dq \eta_2(q)a^\dagger(q)}\exp\left(\int dq a(q)\xi_1(q)+\int dq \xi_1(q)\eta_2(q) \right) \\
=&e^{\int dq \eta_2(q)a^\dagger(q)}e^{\int dq a(q)\xi_1(q)}\exp\left(\int dq \xi_1(q)\eta_2(q) \right)
\end{align*}
となることを用いた.(途中で
\begin{align*}
[\eta_2(q')a^\dagger(q'),a(q)\xi_1(q)]=&\eta_2(q')a^\dagger(q')a(q)\xi_1(q)-a(q)\xi_1(q)\eta_2(q')a^\dagger(q') \\
=&(-1)^{\pi(q)\pi(q')+\pi(q)+\pi(q')} \xi_1(q) a^\dagger(q')a(q)\eta_2(q') \\
&- (-1)^{\pi(q)+\pi(q')}\xi_1(q) a(q)a^\dagger(q')\eta_2(q') \\
=&-(-1)^{\pi(q)+\pi(q')}\xi_1(q)\Bigl(a(q)a^\dagger(q') - (-1)^{\pi(q)\pi(q')}a^\dagger(q')a(q)\Bigr) \eta_2(q') \\
=&-(-1)^{\pi(q)+\pi(q')}\xi_1(q)[a(q),a^\dagger(q')]_{\mp} \eta_2(q') \\
=&-(-1)^{\pi(q)+\pi(q')}\xi_1(q)\delta(q-q') \eta_2(q') \\
=&-(-1)^{2\pi(q)}\xi_1(q)\delta(q-q') \eta_2(q') \\
=&-\xi_1(q) \delta(q-q') \eta_2(q')
\end{align*}
を用いた.これはc数であるから高次の交換子はゼロである.また$\delta(q-q')=\delta^3(\mathbf{p}-\mathbf{p}')\delta_{\sigma\sigma'}\delta_{nn'}$の$\delta_{nn'}$により,$[a(q),a^\dagger(q')]_{\mp}$に現れる$a(q)$と$a^\dagger(q')$は同じ粒子の種類でありそうでなければゼロとなるから$\xi_1(q')$と$\eta_2(q)$はともにボゾン的かともにフェルミオン的である.)したがって,二つの正規順序多項式の積は
\begin{align*}
&:F[a,a^\dagger]::G[a,a^\dagger]: \\
=& \left. \hat{F}\left(\frac{\overrightarrow{\delta}}{\delta \eta_1},\frac{\overleftarrow{\delta}}{\delta \xi_1}\right)\hat{G}\left(\frac{\overrightarrow{\delta}}{\delta \eta_2},\frac{\overleftarrow{\delta}}{\delta \xi_2}\right) N[\eta_1,\xi_1]N[\eta_2,\xi_2] \right|_{\eta=\xi=0} \\
=& \hat{F}\left(\frac{\overrightarrow{\delta}}{\delta \eta_1},\frac{\overleftarrow{\delta}}{\delta \xi_1}\right)\hat{G}\left(\frac{\overrightarrow{\delta}}{\delta \eta_2},\frac{\overleftarrow{\delta}}{\delta \xi_2}\right) \\
& \qquad \times \left.N[\eta_1+\eta_2,\xi_1+\xi_2]\exp\left(\int dq \frac{\overleftarrow{\delta}}{\delta a_1(q)}\frac{\overrightarrow{\delta}}{\delta a^\dagger_2 (q)} \right) \right|_{\eta=\xi=0} \\
=&\left. \hat{F}\left(\frac{\overrightarrow{\delta}}{\delta \eta},\frac{\overleftarrow{\delta}}{\delta \xi}\right)\hat{G}\left(\frac{\overrightarrow{\delta}}{\delta \eta},\frac{\overleftarrow{\delta}}{\delta \xi}\right) \exp\left(\int dq \frac{\overleftarrow{\delta}}{\delta a_1(q)}\frac{\overrightarrow{\delta}}{\delta a^\dagger_2 (q)} \right) N[\eta,\xi] \right|_{\eta=\xi=0} \\
=&\left. \exp\left(\int dq \frac{\overleftarrow{\delta}}{\delta a_1(q)}\frac{\overrightarrow{\delta}}{\delta a^\dagger_2 (q)} \right) \hat{F}\left(\frac{\overrightarrow{\delta}}{\delta \eta},\frac{\overleftarrow{\delta}}{\delta \xi}\right)\hat{G}\left(\frac{\overrightarrow{\delta}}{\delta \eta},\frac{\overleftarrow{\delta}}{\delta \xi}\right) N[\eta,\xi] \right|_{\eta=\xi=0} \\
=&\exp\left(\int dq \frac{\overleftarrow{\delta}}{\delta a_1(q)}\frac{\overrightarrow{\delta}}{\delta a^\dagger_2 (q)} \right) :F[a,a^\dagger]G[a,a^\dagger]: \\
=&\exp\left(\int dq \frac{\overleftarrow{\delta}}{\delta a^{(F)}(q)}\frac{\overrightarrow{\delta}}{\delta a^{(G)\dagger} (q)} \right) :F[a,a^\dagger]G[a,a^\dagger]: \\
=&:F[a,a^\dagger]\exp\left(\sum_{\sigma n}\int d^3\mathbf{p} \frac{\overleftarrow{\delta}}{\delta a^{(F)}(\mathbf{p},\sigma,n)}\frac{\overrightarrow{\delta}}{\delta a^{(G)\dagger} (\mathbf{p},\sigma,n)} \right) G[a,a^\dagger]:
\end{align*}
という公式が得られる.ここで$\delta/\delta \eta_1$によって置き換えられた$a(q)$は$\eta_1(q)$のもともと係数であるものだったのだから,それはまさに$\delta/\delta a_1(q)$によって微分されるものであり,したがって$\delta/\delta a_1(q)$はそのまま$F[a,a^\dagger]$の$a(q)$を微分する$\delta/\delta a^{(F)}(q)$に置き換えられることを用いた.他の部分も同様である.この公式は,書き換えると
\begin{align*}
&:F[a,a^\dagger]::G[a,a^\dagger]: \\
=&:F[a,a^\dagger]\exp\left(\sum_n \sum_{\sigma \sigma'}\int d^3\mathbf{p} d^3\mathbf{p}' \frac{\overleftarrow{\delta}}{\delta a^{(F)}(\mathbf{p},\sigma,n)}[a(\mathbf{p},\sigma,n),a^{\dagger}(\mathbf{p}',\sigma',n)]_{\mp}\frac{\overrightarrow{\delta}}{\delta a^{(G)\dagger} (\mathbf{p}',\sigma',n)} \right) G[a,a^\dagger]:
\end{align*}
となり,$:F[a,a^\dagger]:$の右側に並べられた消滅演算子を$:G[a,a^\dagger]$の左側に並べられた生成演算子を超えて並び替えることにより$:F[a,a^\dagger]G[a,a^\dagger]:$にするとき,「(反)交換関係$[a(\mathbf{p},\sigma,n),a^\dagger(\mathbf{p}',\sigma',n)]_{\mp}=\delta^3(\mathbf{p}-\mathbf{p}')\delta_{\sigma\sigma'}$を用いて(反)交換させていく」という描像を少し見方を変えて,「$a(\mathbf{p},\sigma,n)$と$a^\dagger(\mathbf{p}',\sigma',n)$の組を対にして消す代わりに$[a(\mathbf{p},\sigma,n),a^\dagger(\mathbf{p}',\sigma',n)]_{\mp}=\delta^3(\mathbf{p}-\mathbf{p}')\delta_{\sigma\sigma'}$を生み出す」という描像にするものである.\par
実際に確かめると
\begin{align*}
:F[a,a^\dagger]:=&a(q_1)a(q_2) \\
:G[a,a^\dagger]:=&a^\dagger(q_3)a^\dagger(q_4) \\
:F::G:=&a(q_1)a(q_2)a^\dagger(q_3)a^\dagger(q_4) \\
=&\delta(q_2-q_3)a(q_1)a^\dagger(q_4) +(-1)^{\pi(q_2)\pi(q_3)}a(q_1)a^\dagger(q_3)a(q_2)a^\dagger(q_4) \\
=&\delta(q_2-q_3)a(q_1)a^\dagger(q_4) +(-1)^{\pi(q_2)\pi(q_3)}a(q_1)a^\dagger(q_3)\delta(q_2-q_4)\\
&+(-1)^{\pi(q_2)\pi(q_3)}(-1)^{\pi(q_2)\pi(q_4)}a(q_1)a^\dagger(q_3)a^\dagger(q_4)a(q_2) \\
=&\delta(q_2-q_3)\delta(q_1-q_4) +(-1)^{\pi(q_1)\pi(q_4)}\delta(q_2-q_3)a^\dagger(q_4)a(q_1) \\
&+(-1)^{\pi(q_2)\pi(q_3)}\delta(q_1-q_3)\delta(q_2-q_4)+(-1)^{\pi(q_2)\pi(q_3)}(-1)^{\pi(q_1)\pi(q_3)}a^\dagger(q_3)a(q_1)\delta(q_2-q_4) \\
&+(-1)^{\pi(q_2)\pi(q_3)}(-1)^{\pi(q_2)\pi(q_4)}\delta(q_1-q_3)a^\dagger(q_4)a(q_2) \\
&+(-1)^{\pi(q_2)\pi(q_3)}(-1)^{\pi(q_2)\pi(q_4)}(-1)^{\pi(q_1)\pi(q_3)}a^\dagger(q_3)a(q_1)a^\dagger(q_4)a(q_2) \\
=&\delta(q_2-q_3)\delta(q_1-q_4) +(-1)^{\pi(q_1)\pi(q_4)}\delta(q_2-q_3)a^\dagger(q_4)a(q_1) \\
&+(-1)^{\pi(q_2)\pi(q_3)}\delta(q_1-q_3)\delta(q_2-q_4)+(-1)^{\pi(q_2)\pi(q_3)}(-1)^{\pi(q_1)\pi(q_3)}a^\dagger(q_3)a(q_1)\delta(q_2-q_4) \\
&+(-1)^{\pi(q_2)\pi(q_3)}(-1)^{\pi(q_2)\pi(q_4)}\delta(q_1-q_3)a^\dagger(q_4)a(q_2) \\
&+(-1)^{\pi(q_2)\pi(q_3)}(-1)^{\pi(q_2)\pi(q_4)}(-1)^{\pi(q_1)\pi(q_3)}\delta(q_1-q_4)a^\dagger(q_3)a(q_2) \\
&+(-1)^{\pi(q_2)\pi(q_3)}(-1)^{\pi(q_2)\pi(q_4)}(-1)^{\pi(q_1)\pi(q_3)}(-1)^{\pi(q_1)\pi(q_4)}a^\dagger(q_3)a^\dagger(q_4)a(q_1)a(q_2) \\
=&:a(q_1)a(q_2)a^\dagger(q_3)a^\dagger(q_4): \\
&+\int dp :a(q_1)a(q_2) \frac{\overleftarrow{\delta}}{\delta a(p)}\frac{\overrightarrow{\delta}}{\delta a^\dagger(p)} a^\dagger(q_3)a^\dagger(q_4): \\
&+\frac{1}{2!}\int dp_1 dp_2:a(q_1)a(q_2) \frac{\overleftarrow{\delta}}{\delta a(p_1)}\frac{\overrightarrow{\delta}}{\delta a^\dagger(p_1)}\frac{\overleftarrow{\delta}}{\delta a(p_2)}\frac{\overrightarrow{\delta}}{\delta a^\dagger(p_2)} a^\dagger(q_3)a^\dagger(q_4):
\end{align*}
となる.\par
さて,我々は場を通じてのみ生成・消滅演算子に依存する関数を考えたいのだった.一般に粒子の種類$n$の場は
\begin{align*}
\psi_\ell(x)=\int \frac{d^3\mathbf{p}}{(2\pi)^{3/2}} \sum_\sigma \left[u_\ell(\mathbf{p},\sigma)e^{ip\cdot x}a(\mathbf{p},\sigma)+v_{\ell}(q)e^{-ip\cdot x}a^{c\dagger}(\mathbf{p},\sigma)\right]
\end{align*}
と書ける\footnote{線形結合の係数$\kappa,\lambda$はともに$1$にしている.後の節で示すように,これは常に可能である.}から,$\delta a,\delta a^\dagger$の変分に対する場の変分は
\begin{align*}
\delta \psi_{\ell}(x)=&\int \frac{dq}{(2\pi)^{3/2}} \left[\delta a(\mathbf{p},\sigma) u_\ell(\mathbf{p},\sigma)e^{ip\cdot x}+\delta a^{c\dagger}(\mathbf{p},\sigma) v_{\ell}(\mathbf{p},\sigma)e^{-ip\cdot x}\right] \\
\therefore \quad & \frac{\overleftarrow{\delta}\psi_{\ell}(x)}{\delta a(\mathbf{p},\sigma)}=\frac{u_\ell(\mathbf{p},\sigma)}{(2\pi)^{3/2}}e^{ip\cdot x} ,\quad \frac{\overleftarrow{\delta}\psi^\dagger_{\ell}(x)}{\delta a^{\dagger}(\mathbf{p},\sigma)}=\frac{u^\dagger_\ell(\mathbf{p},\sigma)}{(2\pi)^{3/2}}e^{-ip\cdot x} \\
& \frac{\overleftarrow{\delta}\psi^\dagger_{\ell}(x)}{\delta a^{c}(\mathbf{p},\sigma)}=\frac{v^\dagger_\ell(\mathbf{p},\sigma)}{(2\pi)^{3/2}}e^{ip\cdot x} ,\quad \frac{\overleftarrow{\delta}\psi_{\ell}(x)}{\delta a^{c\dagger}(\mathbf{p},\sigma)}=\frac{v_\ell(\mathbf{p},\sigma)}{(2\pi)^{3/2}}e^{-ip\cdot x}
\end{align*}
となるから,チェーンルールより,自分自身が反粒子でない場合$a\neq a^c$では
\begin{align*}
\frac{\overleftarrow{\delta}}{\delta a(\mathbf{p},\sigma)}=&\sum_{\ell}\int d^4x \frac{\overleftarrow{\delta} \psi_\ell(x)}{\delta a(\mathbf{p},\sigma)}\frac{\overleftarrow{\delta}}{\delta \psi_\ell(x)} =\sum_{\ell}\int d^4x \frac{u_\ell(\mathbf{p},\sigma)}{(2\pi)^{3/2}}e^{ip\cdot x}\frac{\overleftarrow{\delta}}{\delta \psi_\ell(x)} \\
\frac{\overrightarrow{\delta}}{\delta a^\dagger(\mathbf{p},\sigma)}=&\sum_{\ell}\int d^4x \frac{\overrightarrow{\delta} \psi^\dagger_\ell(x)}{\delta a^\dagger(\mathbf{p},\sigma)}\frac{\overrightarrow{\delta}}{\delta \psi^\dagger_\ell(x)} =\sum_{\ell}\int d^4x \frac{u^\dagger_\ell(\mathbf{p},\sigma)}{(2\pi)^{3/2}}e^{-ip\cdot x}\frac{\overrightarrow{\delta}}{\delta \psi^\dagger_\ell(x)} \\
\frac{\overleftarrow{\delta}}{\delta a^c(\mathbf{p},\sigma)}=&\sum_{\ell}\int d^4x \frac{\overleftarrow{\delta} \psi^\dagger_\ell(x)}{\delta a^c(\mathbf{p},\sigma)}\frac{\overleftarrow{\delta}}{\delta \psi^\dagger_\ell(x)} =\sum_{\ell}\int d^4x \frac{v^\dagger_\ell(\mathbf{p},\sigma)}{(2\pi)^{3/2}}e^{ip\cdot x}\frac{\overleftarrow{\delta}}{\delta \psi^\dagger_\ell(x)} \\
\frac{\overrightarrow{\delta}}{\delta a^{c\dagger}(\mathbf{p},\sigma)}=&\sum_{\ell}\int d^4x \frac{\overrightarrow{\delta} \psi_\ell(x)}{\delta a^{c\dagger}(\mathbf{p},\sigma)}\frac{\overrightarrow{\delta}}{\delta \psi_\ell(x)} =\sum_{\ell}\int d^4x \frac{v_\ell(\mathbf{p},\sigma)}{(2\pi)^{3/2}}e^{-ip\cdot x}\frac{\overrightarrow{\delta}}{\delta \psi_\ell(x)}
\end{align*}
であり,自分自身が反粒子の場合$a=a^c$では($\psi$と$\psi^\dagger$が独立でないから)
\begin{align*}
\frac{\overleftarrow{\delta}}{\delta a(\mathbf{p},\sigma)}=&\sum_{\ell}\int d^4x \frac{\overleftarrow{\delta} \psi_\ell(x)}{\delta a(\mathbf{p},\sigma)}\frac{\overleftarrow{\delta}}{\delta \psi_\ell(x)} =\sum_{\ell}\int d^4x \frac{u_\ell(\mathbf{p},\sigma)}{(2\pi)^{3/2}}e^{ip\cdot x}\frac{\overleftarrow{\delta}}{\delta \psi_\ell(x)} \\
\frac{\overrightarrow{\delta}}{\delta a^\dagger(\mathbf{p},\sigma)}=&\sum_{\ell}\int d^4x \frac{\overrightarrow{\delta} \psi_\ell(x)}{\delta a^\dagger(\mathbf{p},\sigma)}\frac{\overrightarrow{\delta}}{\delta \psi_\ell(x)}=\sum_{\ell}\int d^4x \frac{v_\ell(\mathbf{p},\sigma)}{(2\pi)^{3/2}}e^{-ip\cdot x}\frac{\overrightarrow{\delta}}{\delta \psi_\ell(x)}
\end{align*}
となる.これを用いれば,まず反粒子が自分自身でない場合では(粒子の種類$n$に関する和の$n$と$n^c$の部分だけ取り出して)
\begin{align*}
&\sum_\sigma \int d^3\mathbf{p} \frac{\overleftarrow{\delta}}{\delta a^{(F)}(\mathbf{p},\sigma)}\frac{\overrightarrow{\delta}}{\delta a^{(G)\dagger} (\mathbf{p},\sigma)}+\sum_\sigma \int d^3\mathbf{p} \frac{\overleftarrow{\delta}}{\delta a^{(F)c}(\mathbf{p},\sigma)}\frac{\overrightarrow{\delta}}{\delta a^{(G)c\dagger} (\mathbf{p},\sigma)} \\
=&\frac{1}{(2\pi)^3}\sum_{\ell m} \sum_\sigma \int d^3\mathbf{p} \int d^4x d^4y e^{ip\cdot x }e^{-ip\cdot y} u_\ell(\mathbf{p},\sigma)u^\dagger_{m}(\mathbf{p},\sigma)\frac{\overleftarrow{\delta}}{\delta \psi^{(F)}_\ell(x)} \frac{\overrightarrow{\delta}}{\delta \psi^{\dagger(G)}_{m}(y)} \\
&+\frac{1}{(2\pi)^3}\sum_{\ell m}\sum_\sigma \int d^3\mathbf{p} \int d^4x d^4y e^{iq\cdot (x -y)} v^\dagger_\ell(\mathbf{p},\sigma)v_{m}(\mathbf{p},\sigma)\frac{\overleftarrow{\delta}}{\delta \psi^{(F)\dagger}_\ell(x)} \frac{\overrightarrow{\delta}}{\delta \psi^{(G)}_{m}(y)} \\
=&\sum_{\ell m}\int d^4x d^4y \sum_{\sigma}\int \frac{d^3\mathbf{p}}{(2\pi)^3} u_\ell(\mathbf{p},\sigma)u^\dagger_{m}(\mathbf{p},\sigma) e^{ip\cdot (x - y)} \frac{\overleftarrow{\delta}}{\delta \psi^{(F)}_\ell(x)} \frac{\overrightarrow{\delta}}{\delta \psi^{\dagger(G)}_{m}(y)} \\
&+\sum_{\ell m}\int d^4x d^4y \sum_{\sigma}\int \frac{d^3\mathbf{p}}{(2\pi)^3} v^\dagger_\ell(\mathbf{p},\sigma)v_{m}(\mathbf{p},\sigma) e^{ip\cdot (x - y)} \frac{\overleftarrow{\delta}}{\delta \psi^{\dagger(F)}_\ell(x)} \frac{\overrightarrow{\delta}}{\delta \psi^{(G)}_{m}(y)} \\
=&\sum_{\ell m}\int d^4x d^4y [\psi^+_{\ell}(x),\psi^{+\dagger}_{m}(y)]_{\mp} \frac{\overleftarrow{\delta}}{\delta \psi^{(F)}_\ell(x)} \frac{\overrightarrow{\delta}}{\delta \psi^{\dagger(G)}_{m}(y)} \\
&+\sum_{\ell m}\int d^4x d^4y [\psi^{c-\dagger}_{\ell}(x),\psi^{c-}_{m}(y)]_{\mp} \frac{\overleftarrow{\delta}}{\delta \psi^{\dagger(F)}_\ell(x)} \frac{\overrightarrow{\delta}}{\delta \psi^{(G)}_{m}(y)}
\end{align*}
となる.ここで
\begin{align*}
\psi^+_\ell(x)=&\frac{1}{(2\pi)^{3/2}}\int d^3\mathbf{p} u_\ell(\mathbf{p},\sigma) e^{ip\cdot x}a(\mathbf{p},\sigma) \\
\psi^{c-}_\ell(x)=&\frac{1}{(2\pi)^{3/2}}\int d^3\mathbf{p} v_\ell(\mathbf{p},\sigma) e^{-ip\cdot x}a^{c\dagger}(\mathbf{p},\sigma)\\
\psi^{+\dagger}_\ell(x)=&\frac{1}{(2\pi)^{3/2}}\int d^3\mathbf{p} u^\dagger_\ell(\mathbf{p},\sigma) e^{-ip\cdot x}a^\dagger(\mathbf{p},\sigma) \\
\psi^{c-\dagger}_\ell(x)=&\frac{1}{(2\pi)^{3/2}}\int d^3\mathbf{p} v^\dagger_\ell(\mathbf{p},\sigma) e^{ip\cdot x}a^c(\mathbf{p},\sigma) \\
[\psi^+_{\ell}(x),\psi^{+\dagger}_{m}(y)]_{\mp}=&\sum_{\sigma\sigma'}\int \frac{d^3\mathbf{p}d^3\mathbf{p}'}{(2\pi)^3} u_\ell(\mathbf{p},\sigma)u^\dagger_{m}(\mathbf{p}',\sigma')e^{ip\cdot x-ip'\cdot y}[a(\mathbf{p},\sigma),a^\dagger(\mathbf{p}',\sigma')]_{\mp} \\
=&\sum_{\sigma\sigma'}\int \frac{d^3\mathbf{p}d^3\mathbf{p}'}{(2\pi)^3} u_\ell(\mathbf{p},\sigma)u^\dagger_{m}(\mathbf{p}',\sigma')e^{ip\cdot x-ip'\cdot y}\delta^3(\mathbf{p}-\mathbf{p}')\delta_{\sigma\sigma'} \\
=&\sum_{\sigma}\int \frac{d^3\mathbf{p}}{(2\pi)^3} u_\ell(\mathbf{p},\sigma)u^\dagger_{m}(\mathbf{p},\sigma)e^{ip\cdot(x-y)} \\
[\psi^{c-\dagger}_{\ell}(x),\psi^{c-}_{m}(y)]_{\mp}=&\sum_{\sigma\sigma'}\int \frac{d^3\mathbf{p}d^3\mathbf{p}'}{(2\pi)^3} v^\dagger_\ell(\mathbf{p},\sigma)v_{m}(\mathbf{p}',\sigma')e^{ip\cdot x-ip'\cdot y}[a^{c}(\mathbf{p},\sigma),a^{c\dagger}(\mathbf{p}',\sigma')]_{\mp} \\
=&\sum_{\sigma\sigma'}\int \frac{d^3\mathbf{p}d^3\mathbf{p}'}{(2\pi)^3} v^\dagger_\ell(\mathbf{p},\sigma)v_{m}(\mathbf{p}',\sigma')e^{ip\cdot x-ip'\cdot y}\delta^3(\mathbf{p}-\mathbf{p}')\delta_{\sigma\sigma'} \\
=&\sum_{\sigma}\int \frac{d^3\mathbf{p}}{(2\pi)^3} v^\dagger_\ell(\mathbf{p},\sigma)v_{m}(\mathbf{p},\sigma)e^{ip\cdot(x-y)}
\end{align*}
であることを用いた.自分自身が反粒子の場合は
\begin{align*}
&\sum_\sigma \int d^3\mathbf{p}\frac{\overleftarrow{\delta}}{\delta a^{(F)}(\mathbf{p},\sigma)}\frac{\overrightarrow{\delta}}{\delta a^{(G)\dagger} (\mathbf{p},\sigma)} \\
=&\sum_{\ell m}\int d^4x d^4y \sum_{\sigma}\int \frac{d^3\mathbf{p}}{(2\pi)^3} u_\ell(\mathbf{p},\sigma,n)v_{\bar{\ell}}(\mathbf{p},\sigma,n) e^{ip\cdot (x - y)} \frac{\overleftarrow{\delta}}{\delta \psi^{(F)}_\ell(x)} \frac{\overrightarrow{\delta}}{\delta \psi^{(G)}_{\bar{\ell}}(y)} \\
=&\sum_{\ell m}\int d^4x d^4y [\psi^+_{\ell}(x),\psi^{-}_{\bar{\ell}}(y)]_{\mp} \frac{\overleftarrow{\delta}}{\delta \psi^{(F)}_\ell(x)} \frac{\overrightarrow{\delta}}{\delta \psi^{(G)}_{\bar{\ell}}(y)}
\end{align*}
となる.したがって
\begin{align*}
&:F[\psi(x),\psi^\dagger(x)]::G[\psi(y),\psi^\dagger(y)]: \\
=&:F[\psi(x),\psi^\dagger(x)]\exp\biggl(\sum_{\ell m}\int d^4u d^4v \frac{\overleftarrow{\delta}}{\delta \psi^{(F)}_\ell(u)} [\psi^+_{\ell}(u),\psi^{+\dagger}_{m}(v)]_{\mp} \frac{\overrightarrow{\delta}}{\delta \psi^{\dagger(G)}_{m}(v)} \\
&\qquad \qquad \qquad \qquad +\sum_{\ell m}\int d^4u d^4v \frac{\overleftarrow{\delta}}{\delta \psi^{\dagger(F)}_\ell(u)} [\psi^{c-\dagger}_{\ell}(u),\psi^{c-}_{m}(v)]_{\mp} \frac{\overrightarrow{\delta}}{\delta \psi^{(G)}_{m}(v)}\biggr) G[\psi(y),\psi^\dagger(y)]:
\end{align*}
という公式が得られる.

\vskip\baselineskip

$S$行列に現れる時間順序積は
\begin{align*}
T\{\mc{H}_I(x_1),\mc{H}_I(x_2),\cdots,\mc{H}_I(x_n)\}=&\cdots \mc{H}_I(x_i)\theta(x^0_i-x^0_j)\mc{H}_I(x_j)\cdots \\
&+\cdots \mc{H}_I(x_j)\theta(x^0_j-x^0_i)\mc{H}_I(x_i)\cdots
\end{align*}
という,$x_i$と$x_j$の相互作用の積と,その入れ替えをした積が足し合わされる.ここでは簡単のため,2個の時間順序積
\begin{align*}
T\{\mc{H}_I(x),\mc{H}_I(y)\}=\theta(x-y)\mc{H}_I(x)\mc{H}_I(y)+\theta(y-x)\mc{H}_I(y)\mc{H}_I(x)
\end{align*}
を考えよう.5.1節で述べたように,相互作用は場とその共役場の関数を正規順序積にしたもの$\mc{H}_I(x)\equiv :\mc{F}[\psi(x),\psi^\dagger(x)]:$で与えられ,かつ$\mc{H}_I(x)$は多項式だから単項式$\mc{H}_i(x)$の和で書くことができるはずであるから$\mc{F}[\psi,\psi^\dagger]$も同様に場と共役場の単項式で分解しよう.
\begin{align*}
\mc{F}[\psi(x),\psi^\dagger(x)]=&\sum_{i}g_i \mc{F}_i[\psi(x),\psi^\dagger(x)] \\
\mc{H}_i(x)\equiv& :\mc{F}_i[\psi(x),\psi^\dagger(x)]: \\
\therefore \quad \mc{H}_I(x)=&\sum_{i}g_i \mc{H}_i(x)
\end{align*}
したがって,2個の時間順序積は
\begin{align*}
&T\{\mc{H}_I(x),\mc{H}_I(y)\} \\
=&\sum_{ij}g_ig_j \Bigl[\theta(x-y):\mc{F}_i[\psi(x),\psi^\dagger(x)]::\mc{F}_j[\psi(y),\psi^\dagger(y)]: \\
&\qquad +\theta(y-x):\mc{F}_j[\psi(y),\psi^\dagger(y)]::\mc{F}_i[\psi(x),\psi^\dagger(x)]:\Bigr]
\end{align*}
という形になる.上で与えた公式より
\begin{align*}
&:\mc{F}_i[\psi(x),\psi^\dagger(x)]::\mc{F}_j[\psi(y),\psi^\dagger(y)]: \\
=&:\mc{F}_i[\psi(x),\psi^\dagger(x)]\exp\Biggl(\sum_{\ell m}\int d^4u d^4v \frac{\overleftarrow{\delta}}{\delta \psi_\ell(u)} [\psi^+_{\ell}(u),\psi^{+\dagger}_{m}(v)]_{\mp} \frac{\overrightarrow{\delta}}{\delta \psi^{\dagger}_{m}(v)} \\
&\qquad \qquad \qquad \qquad +\sum_{\ell m}\int d^4u d^4v \frac{\overleftarrow{\delta}}{\delta \psi^{\dagger}_\ell(u)} [\psi^{c-\dagger}_{\ell}(u),\psi^{c-}_{m}(v)]_{\mp} \frac{\overrightarrow{\delta}}{\delta \psi_{m}(v)}\Biggr) \mc{F}_j[\psi(y),\psi^\dagger(y)]:
\end{align*}
を用いれば,第一項目の$\mc{F}_i(x)[\psi(x),\psi^\dagger(x)]$の中にある$\psi_\ell(x)$と,$\mc{F}_j[\psi(y),\psi^\dagger(y)]$の中にある$\psi^\dagger_m(y)$の間から
\begin{align*}
\theta(x-y)[\psi^{+}_{\ell}(x),\psi^{+\dagger}_{m}(y)]_{\mp}
\end{align*}
が現れることがわかる.同時に時間順序積の第二項目も
\begin{align*}
&:\mc{F}_j[\psi(y),\psi^\dagger(y)]::\mc{F}_i[\psi(x),\psi^\dagger(x)]: \\
=&:\mc{F}_j[\psi(y),\psi^\dagger(y)]\exp\Biggl(\sum_{\ell m}\int d^4u d^4v \frac{\overleftarrow{\delta}}{\delta \psi_m(u)} [\psi^+_{m}(u),\psi^{+\dagger}_{\ell}(v)]_{\mp} \frac{\overrightarrow{\delta}}{\delta \psi^{\dagger}_{\ell}(v)} \\
&\qquad \qquad \qquad \qquad +\sum_{\ell m}\int d^4u d^4v \frac{\overleftarrow{\delta}}{\delta \psi^{\dagger}_m(u)} [\psi^{c-\dagger}_{m}(u),\psi^{c-}_{\ell}(v)]_{\mp} \frac{\overrightarrow{\delta}}{\delta \psi_{\ell}(v)}\Biggr) \mc{F}_i[\psi(x),\psi^\dagger(x)]: \\
=&:\mc{F}_i[\psi(x),\psi^\dagger(x)]\exp\Biggl(\pm \sum_{\ell m}\int d^4u d^4v \frac{\overleftarrow{\delta}}{\delta \psi^{\dagger}_{\ell}(v)} [\psi^+_{m}(u),\psi^{+\dagger}_{\ell}(v)]_{\mp} \frac{\overrightarrow{\delta}}{\delta \psi_m(u)} \\
&\qquad \qquad \qquad \qquad \pm\sum_{\ell m}\int d^4u d^4v \frac{\overleftarrow{\delta}}{\delta \psi_{\ell}(v)} [\psi^{c-\dagger}_{m}(u),\psi^{c-}_{\ell}(v)]_{\mp}\frac{\overrightarrow{\delta}}{\delta \psi^{\dagger}_m(u)} \Biggr) \mc{F}_j[\psi(y),\psi^\dagger(y)] :
\end{align*}
より(ここでフェルミオン的な汎関数微分同士は反可換になることを用いた),$\mc{F}_i(x)[\psi(x),\psi^\dagger(x)]$の中にある$\psi_\ell(x)$と$\mc{F}_j[\psi(y),\psi^\dagger(y)]$の中にある$\psi^\dagger_m(y)$との間から
\begin{align*}
\pm\theta(y-x)[\psi^{c-\dagger}_{m}(y),\psi^{c-}_{\ell}(x)]_{\mp}
\end{align*}
が現れる.これらを足し合わせれば,$\mc{H}_i(x)$に含まれる$\psi_\ell(x)$と,$\mc{H}_j(y)$に含まれる$\psi^\dagger_m(y)$との対から
\begin{align*}
&\theta(x-y)[\psi^{+}_{\ell}(x),\psi^{+\dagger}_{m}(y)]_{\mp} \pm \theta(y-x)[\psi^{c-\dagger}_{m}(y),\psi^{c-}_{\ell}(x)]_{\mp} \\
=&:-i\Delta_{\ell m}(x,y)
\end{align*}
が生じることがわかる.この結果をまとめれば
\begin{align*}
&T\{:\mc{F}_i[\psi(x),\psi^\dagger(x)]:,:\mc{F}_j[\psi(y),\psi^\dagger(y)]:\} \\
=&:\mc{F}_i[\psi(x),\psi^\dagger(x)]\exp\Biggl(\sum_{\ell m}\int d^4u d^4v \frac{\overleftarrow{\delta}}{\delta \psi_\ell(u)} (-i\Delta_{\ell m}(u,v)) \frac{\overrightarrow{\delta}}{\delta \psi^{\dagger}_{m}(v)} \\
&\qquad \qquad \qquad \qquad \pm\sum_{\ell m}\int d^4u d^4v \frac{\overleftarrow{\delta}}{\delta \psi^{\dagger}_\ell(u)} (-i\Delta_{\ell m}(v,u)) \frac{\overrightarrow{\delta}}{\delta \psi_{m}(v)}\Biggr) \mc{F}_j[\psi(y),\psi^\dagger(y)] :
\end{align*}
となる.($x>y$と$y<x$の場合で分ければ,プロパゲータがそれぞれ$[\psi^{+}_{\ell}(x),\psi^{+\dagger}_{m}(y)]_{\mp}$と$[\psi^{c-\dagger}_{m}(y),\psi^{c-}_{\ell}(x)]_{\mp}$になり,どちらの場合でも上の表式と一致する.)これをウィックの定理と呼ぶ.

\vskip\baselineskip

以上の公式を用いて$S$行列の計算を考えよう.3章で与えたように,$S$行列は次の形式で与えられる.
\begin{align*}
&S_{\beta\alpha} \\
=&(\Phi_{\beta},S\Phi_{\alpha}) \\
=&\sum_{N=0}^\infty \frac{(-1)^N}{N!} \int d^4x_1 \cdots d^x_N \\
&\times \left(a^\dagger(\mathbf{p}'_1,\sigma'_1,n'_1)a^\dagger(\mathbf{p}'_2,\sigma'_2,n'_2)\cdots \Phi_0 ,T\{\mc{H}_I(x_1)\cdots \mc{H}_I(x_N)\}a^\dagger(\mathbf{p}_1,\sigma_1,n_1)a^\dagger(\mathbf{p}_2,\sigma_2,n_2)\cdots \Phi_0\right) \\
=&\sum_{N=0}^\infty \frac{(-1)^N}{N!} \int d^4x_1 \cdots d^x_N \\
&\times \left(\Phi_0 ,\cdots a(\mathbf{p}_1',\sigma_1',n_1')a(\mathbf{p}'_2,\sigma_2',n_2')T\{\mc{H}_I(x_1)\cdots \mc{H}_I(x_N)\}a^\dagger(\mathbf{p}_1,\sigma_1,n_1)a^\dagger(\mathbf{p}_2,\sigma_2,n_2)\cdots \Phi_0\right)
\end{align*}




\end{document}