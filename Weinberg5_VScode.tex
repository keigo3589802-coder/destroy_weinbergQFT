\documentclass[dvipdfmx]{jsarticle}
\let\headfont=\gtfamily
\usepackage[dvips]{graphicx}
\usepackage{amsmath}
\usepackage{mathrsfs} % 花文字\mathscr{M}, 筆記体\mathcal{M}, 黒板文字\mathbb{M},ドイツ文字\mathfrak{M}
\usepackage{bm} %太文字
\usepackage{amssymb}
\usepackage{latexsym}
\usepackage{braket}
\usepackage{tikz}
\usepackage{tikz-feynhand}
\usepackage{ulem}
\usepackage{bigdelim}
\usepackage{multirow}
\usepackage{tcolorbox}
\usepackage{here}
\tcbuselibrary{theorems,skins}
\usetikzlibrary{decorations}
\usepackage{color}
\usepackage{tensor}

\usetikzlibrary{intersections, calc, arrows.meta}
 \usetikzlibrary{patterns}

\newfont{\bg}{cmr9 scaled\magstep4}
\newcommand{\bigzerol}{\smash{\lower1.0ex\hbox{\bg 0}}}
\newcommand{\bigzerou}{%
   \smash{\hbox{\bg 0}}}
\newcommand{\mcO}{\mathcal{O}}
\newcommand{\VAC}{\mathrm{VAC}}
\newcommand{\Slash}[1]{{\ooalign{\hfil/\hfil\crcr$#1$}}} %ファインマンのスラッシュ記号
\renewcommand{\mc}{\mathcal}
\newcommand{\mr}[1]{\mathrm{#1}}

% \textrm{Roman デフォルト}
% \textgt{Gothic 和文ゴシック体}*専門用語に
% \textbf{Boldface 太字}*専門用語(英語)に
% \textit{Italic 斜体}
% \textsl{Slanted ローマンを傾けただけ}
% \textsf{Sans Serif サンセリフ体}
% \texttt{Typewriter タイプライタ体、等幅}
% \textsc{Small Caps 小文字が大文字に}

\setlength{\textwidth}{\fullwidth}
\setlength{\textheight}{44\baselineskip}
\addtolength{\textheight}{\topskip}
\setlength{\voffset}{-0.6in}

\allowdisplaybreaks[4]

\makeatletter
  \renewcommand{\theequation}
  {\arabic{section}.\arabic{equation}}
  \@addtoreset{equation}{section}
 \makeatother

\title{\vspace{-1cm}\Huge{WeinbergQFT5}}
\author{Laplacyan}
\date{}
\begin{document}



\maketitle
\setcounter{part}{23}
\part{歴史的導入}
\setcounter{section}{24}
\subsection{型破りな対称性と禁止定理}
コールマン・マンデューラの定理の一部分について運動学的な証明をする.\par
4次元運動量$P_\mu$と可換な全ての対称性生成子を$B_\alpha$と書き,それらがリー代数を張るとする.すなわち
\begin{align*}
&B_\alpha P_\mu=P_\mu B_\alpha \\
&[B_\alpha,B_\beta]=iC_{\alpha\beta}^\gamma B_\gamma
\end{align*}
が成り立っているとする.($B_\alpha$は$SU(N)$群などの生成子以外にもローレンツ群の生成子も,すなわち(2.4.14)によって$P_\mu$も含まれており,添え字$\alpha$は$\mu,\nu$などの4次元添え字も含まれている.)これらの生成子に固有ローレンツ変換$x^\mu\to \Lambda^\mu_{\,\,\, \nu} x^\nu$がどのように働くかを考える.このローレンツ変換はヒルベルト空間上でユニタリー演算子$U(\Lambda)$で表されるとする.演算子$U(\Lambda)B_\alpha U^{-1}(\Lambda)$は$\Lambda_\mu^{\,\,\, \nu}P_\nu$と可換なエルミート対称性生成子となることは簡単に分かる.なぜなら(2.4.9)より
\begin{align*}
&U(\Lambda)B_{\alpha}P_\mu U^{-1}(\Lambda)=U(\Lambda)B_{\alpha}U^{-1}(\Lambda)U(\Lambda)P_\mu U^{-1}(\Lambda)=U(\Lambda)B_{\alpha}U^{-1}(\Lambda)\Lambda_\mu^{\,\,\, \nu}P_\nu \\
=&U(\Lambda)P_\mu B_{\alpha} U^{-1}(\Lambda)=\Lambda_\mu^{\,\,\, \nu}P_\nu U(\Lambda)B_{\alpha}U^{-1}(\Lambda)
\end{align*}
となるからだ.$\Lambda^{\,\,\, \nu}_\mu$は正則な行列であり,行列の成分はただの係数なのだから,つまり$P_\mu$の線型結合である$\Lambda_\mu^{\,\,\, \nu}P_\nu$と可換な演算子$U(\Lambda)B_\alpha U^{-1}(\Lambda)$は$P_\mu$と可換でなければならない.$P_\mu$と可換な全ての対称性生成子の集合が$B_\alpha$なのだったから,この演算子は$B_\alpha$の線型結合でなければならない.
\begin{align*}
U(\Lambda)B_\alpha U^{-1}(\Lambda)=\sum_\beta D^\beta_{\,\,\, \alpha}(\Lambda)B_\beta
\end{align*}
ただし,ここで$D^\beta_{\,\,\, \alpha}(\Lambda)$は斉次ローレンツ群の表現をなす実係数の集合であり,
\begin{align*}
&U(\Lambda_2)U(\Lambda_1)B_\alpha U^{-1}(\Lambda_1) U^{-1}(\Lambda_2)=\sum_{\beta\gamma}D^{\beta}_{\,\,\, \alpha}(\Lambda_1) D^{\gamma}_{\,\,\, \beta }(\Lambda_2) B_\gamma=\sum_\gamma \left[\sum_\beta D^{\gamma}_{\,\,\, \beta }(\Lambda_2) D^{\beta}_{\,\,\, \alpha}(\Lambda_1) \right] B_\gamma \\
=&U(\Lambda_2\Lambda_1)B_\alpha U^{-1}(\Lambda_2 \Lambda_1)=\sum_{\beta}D^\gamma_{\,\,\, \alpha}(\Lambda_2 \Lambda_1)B_\gamma \\
&\therefore \quad D(\Lambda_1)D(\Lambda_2)=D(\Lambda_1 \Lambda_2)
\end{align*}
を満たす.$U(\Lambda)B_\alpha U^{-1}(\Lambda)$は$B_\alpha$と同じ交換関係を満たさなければならない
\begin{align*}
&\left[ U(\Lambda)B_\alpha U^{-1}(\Lambda) , U(\Lambda)B_\beta U^{-1}(\Lambda)\right] \\
=&U(\Lambda)[B_\alpha ,B_\beta ]U^{-1}(\Lambda) =iC^\gamma_{\alpha\beta}U(\Lambda)B_\gamma U^{-1}(\Lambda)
\end{align*}
ので,このリー代数の構造定数$C^{\gamma}_{\alpha\beta}$は
\begin{align*}
\sum_{\gamma}iC^\gamma_{\alpha\beta}B_\gamma &=U(\Lambda^{-1})\left[ U(\Lambda)B_\alpha U^{-1}(\Lambda) , U(\Lambda)B_\beta U^{-1}(\Lambda)\right] U(\Lambda^{-1})\\
&=\sum_{\alpha'\beta'}D^{\alpha'}_{\,\,\, \alpha}(\Lambda) D^{\beta'}_{\,\,\, \beta}(\Lambda)U(\Lambda^{-1})[B_{\alpha'},B_{\beta'}]U^{-1}(\Lambda^{-1}) \\
&=\sum_{\alpha'\beta'\gamma'}D^{\alpha'}_{\,\,\, \alpha}(\Lambda) D^{\beta'}_{\,\,\, \beta}(\Lambda)iC^{\gamma'}_{\alpha' \beta'}U(\Lambda^{-1})B_{\gamma'}U^{-1}(\Lambda^{-1}) \\
&=\sum_{\alpha'\beta'\gamma'\gamma}D^{\alpha'}_{\,\,\, \alpha}(\Lambda) D^{\beta'}_{\,\,\, \beta}(\Lambda)D^\gamma_{\,\,\, \gamma'}(\Lambda^{-1})iC^{\gamma'}_{\alpha' \beta'}B_\gamma \\
\therefore &\quad C^{\gamma}_{\alpha\beta} =\sum_{\alpha'\beta'\gamma'}D^{\alpha'}_{\,\,\, \alpha}(\Lambda) D^{\beta'}_{\,\,\, \beta}(\Lambda)D^\gamma_{\,\,\, \gamma'}(\Lambda^{-1})C^{\gamma'}_{\alpha' \beta'}
\end{align*}
の意味で不変テンソルだ.これを,$C^{\alpha}_{\gamma\delta}$の対応する式と縮約すると以下を得る.
\begin{align*}
\sum_{\alpha\gamma}C^\gamma_{\alpha\beta}C^{\alpha}_{\gamma\delta}=&\sum_{\substack{\alpha'\beta'\gamma' \\ \alpha'' \gamma''\delta' \\ \alpha\gamma}}D^{\alpha'}_{\,\,\, \alpha}(\Lambda) D^{\beta'}_{\,\,\, \beta}(\Lambda)D^\gamma_{\,\,\, \gamma'}(\Lambda^{-1})C^{\gamma'}_{\alpha' \beta'} D^{\gamma''}_{\,\,\, \gamma}(\Lambda) D^{\delta'}_{\,\,\, \delta}(\Lambda)D^{\alpha}_{\,\,\, \alpha''}(\Lambda^{-1})C^{\alpha''}_{\gamma'' \delta'} \\
=&\sum_{\substack{\alpha'\beta'\gamma' \\ \alpha'' \gamma''\delta'}}\delta^{\alpha'}_{\alpha''}\delta^{\gamma''}_{\gamma'}D^{\beta'}_{\,\,\, \beta}(\Lambda)D^{\delta'}_{\,\,\, \delta}(\Lambda)C^{\gamma'}_{\alpha' \beta'}C^{\alpha''}_{\gamma'' \delta'} \\
=&\sum_{\alpha' \beta'\gamma'\delta'} D^{\beta'}_{\,\,\, \beta}(\Lambda)D^{\delta'}_{\,\,\, \delta}(\Lambda)C^{\gamma'}_{\alpha' \beta'}C^{\alpha'}_{\gamma' \delta'} \\
=&\sum_{ \beta'\delta'} D^{\beta'}_{\,\,\, \beta}(\Lambda)D^{\delta'}_{\,\,\, \delta}(\Lambda)\left[\sum_{\alpha'\gamma'}C^{\gamma'}_{\alpha' \beta'}C^{\alpha'}_{\gamma' \delta'}\right]
\end{align*}
ここで計量(15.A.10)
\begin{align*}
g_{\beta\gamma}\equiv -\sum_{\alpha\gamma}C^\gamma_{\alpha\beta}C^{\alpha}_{\gamma\delta}
\end{align*}
を定義すると
\begin{align*}
g_{\beta\gamma}=\sum_{ \beta'\delta'} D^{\beta'}_{\,\,\, \beta}(\Lambda)D^{\delta'}_{\,\,\, \delta}(\Lambda) g_{\beta'\delta'}
\end{align*}
となる.(すぐ使うので,これを行列表記にしておくと$g=D(\Lambda)^{\mathrm{T}}g D(\Lambda)$となる.)生成子$B_\alpha$は全て$P_\mu$と可換なのだから$C^\alpha_{\mu\beta}=-C^\alpha_{\beta\mu}=0$であり,$g_{\mu\alpha}=g_{\alpha\mu}=0$となる.\par
$P_\mu$以外の対称性生成子の添え字には$\alpha,\beta$などの代わりに$A,B$などを使って区別することにする.$C^A_{\mu B}=-C^A_{B\mu}=0$であることを(24.1.5)に使うと$g_{AB}=-\sum_{CD}C^D_{AC}C^C_{BD}$を得る.ここで生成子$B_A$はコンパクト半単純リー代数を張ると仮定すると,コンパクト性により行列$g_{AB}$は正定値となる.理由を一応示しておくと,コンパクト性により$C^A_{BC}$は完全反対称になり,ゼロでないベクトル$u_A$の二次形式は
\begin{align*}
u_Ag_{AB}u_B=-\sum_{ABCD}u_AC^D_{AC}C^C_{BD}u_B=\sum (u_AC^D_{AC})^2 \geq 0
\end{align*}
となって正であることが分かるからだ.したがって$g^{1/2}$が定義でき,(24.1.4)(24.1.2)より,行列$g^{1/2}D(\Lambda)g^{-1/2}$は斉次ローレンツ群の実直交,したがってユニタリーな有限次元表現を与える.実際
\begin{align*}
\left[g^{1/2}D(\Lambda) g^{-1/2}\right]^{\mathrm{T}}g^{1/2}D(\Lambda)g^{-1/2}=&g^{-1/2}D(\Lambda)^{\mathrm{T}}gD(\Lambda)g^{-1/2} \\
=&g^{-1/2}g g^{-1/2} =1
\end{align*}
となる.二番目の等号では(24.1.4)を用いた.しかし,ローレンツ群はコンパクト群ではないから,そのような表現は自明なものしか存在しない.したがって$g^{1/2}D(\Lambda)g^{-1/2}=1$,すなわち$D(\Lambda)=1$となる.(ここで相対論的な性質が効いている.ガリレイ群の半単純部分はコンパクト群$SU(2)$であり,これはもちろん無限個のユニタリーな有限次元表現を持つ.)$D(\Lambda)=1$と(24.1.1)より生成子$B_A$は全てのローレンツ変換$\Lambda^\mu_{\,\,\,\nu}$について$U(\Lambda)$と可換になる.\par
運動量が$p^\mu$でスピンと種類が離散的な添え字$n$で表される安定な一粒子状態$\ket{p,n}$に$B_A$が作用するとき,$P_\mu$と可換な$B_A$のような演算子は$\ket{p,n}$の線型結合しか作らない.なぜなら,状態$B_A\ket{p,n}$に$P_\mu$を作用させると
\begin{align*}
P_\mu B_A \ket{p,n}= B_A P_\mu \ket{p,n}=p_\mu B_A\ket{p,n}
\end{align*}
となって,$B_A\ket{p,n}$は運動量$p^\mu$の状態の線型結合で作られていると分かるからだ.したがって
\begin{align*}
B_A\ket{p,n}=\sum_m \Bigl( b_A (p)\Bigr)_{mn}\ket{p,m}
\end{align*}
と書けるが,$B_A$が$U(\Lambda)$と可換で,ブースト(2.5.5)と可換であるから,
\begin{align*}
B_A \ket{p,n}=N(p) U(L(p))B_A \ket{k,n}=\sum_{m} \Bigl( b_A (k)\Bigr)_{mn}N(p)U(L(p))\ket{k,m}=\sum_m \Bigl( b_A (k)\Bigr)_{mn}\ket{p,m}
\end{align*}
となって,$b_A(p)$は運動量に依らないことが言える.また$B_A$が回転とも可換であるから$J_3$を作用させて同様の議論を繰り返せば,$b_A(p)$がスピンを変化させず,スピンに対して単位行列として作用することがわかるので,$B_A$は通常の内部対称性の生成子であることがわかる.これが証明したいことであった.


\newpage

\subsection{超対称性の誕生}
コールマン・マンデューラ定理の適用範囲外が知りたい.そのために,この定理はボゾンをボゾンに変換しフェルミオンをフェルミオンに変換するために(反交換関係ではなく)交換関係を満たす演算子によって生成される変換のみを扱っている,という点に注目しよう.つまり,スピンに非自明な作用をしボゾンをフェルミオンに,あるいはフェルミオンをボゾンに変換し,交換関係ではなく反交換関係を満たすような対称性が相対論的理論で許されるかどうかが問題となる.\par
1960年代の終わり,各種のハドロンを「弦の異なった振動モード」と解釈する描像が登場した.パラメータ$\sigma$で表される弦の上の1点はある固定された時計の時刻$\tau$において$X^\mu(\sigma,\tau)$という時空の座標をもつ.したがって,$d$次元時空での弦の運動は$d$個のボソン場をもつ2次元の場の理論で記述される.その作用は
\begin{align*}
I[X]=\frac{T}{2}\int d\sigma \int d\tau \eta_{\mu\nu}\left[ \frac{\partial X^\mu }{\partial \tau }\frac{\partial X^\nu }{\partial \tau }-\frac{\partial X^\mu }{\partial \sigma }\frac{\partial X^\nu }{\partial \sigma } \right]
\end{align*}
ここで$\mu=0,\cdots , d-1$であり,$T$は弦の張力として知られる定数.$\sigma^\pm$は2次元の「光円錐」座標.$\tau \to \sigma^-=\tau-\sigma,\sigma\to \sigma^+=\tau+\sigma$と変数変換すればヤコビアンは2なので$d\sigma d\tau=\frac{1}{2}d\sigma^+d\sigma^-$となり
\begin{align*}
\frac{\partial}{\partial \tau}=&\frac{\partial \sigma^+}{\partial \tau}\frac{\partial}{\partial \sigma^+}+\frac{\partial \sigma^-}{\partial \tau}\frac{\partial}{\partial \sigma^-} \\
=&\frac{\partial}{\partial \sigma^+}+\frac{\partial}{\partial \sigma^-} \\
\frac{\partial}{\partial \sigma}=&\frac{\partial \sigma^+}{\partial \sigma}\frac{\partial}{\partial \sigma^+}+\frac{\partial \sigma^-}{\partial \sigma}\frac{\partial}{\partial \sigma^-} \\
=&\frac{\partial}{\partial \sigma^+}-\frac{\partial}{\partial \sigma^-}
\end{align*}
これにより作用は
\begin{align*}
I[X]=&\frac{T}{4}\int d\sigma^+ \int d\sigma^- \eta_{\mu\nu}\left[\left( \frac{\partial X^\mu}{\partial \sigma^+}+\frac{\partial X^\mu}{\partial \sigma^-} \right)\left(\frac{\partial X^\nu}{\partial \sigma^+}+\frac{\partial X^\nu}{\partial \sigma^-}\right)-\left( \frac{\partial X^\mu}{\partial \sigma^+}-\frac{\partial X^\mu}{\partial \sigma^-} \right)\left(\frac{\partial X^\nu}{\partial \sigma^+}-\frac{\partial X^\nu}{\partial \sigma^-}\right) \right] \\
=&\frac{T}{2}\int d\sigma^+ \int d\sigma^- \eta_{\mu\nu}\left[ \frac{\partial X^\mu}{\partial \sigma^+}  \frac{\partial X^\nu}{\partial \sigma^-}+ \frac{\partial X^\mu}{\partial \sigma^-} \frac{\partial X^\nu}{\partial \sigma^+}\right] \\
=&T\int d\sigma^+ \int d\sigma^- \eta_{\mu\nu}\frac{\partial X^\mu}{\partial \sigma^+}  \frac{\partial X^\nu}{\partial \sigma^-}
\end{align*}
を得る.この作用は,一対の世界面座標$\sigma^k=(\sigma,\tau)$の変換$\sigma^k\to \sigma'^k=f^k(\sigma,\tau)$のもとでの完全な不変性をもつより一般的な作用
\begin{align*}
I[X]=-\frac{T}{2}\int d^2\sigma \eta_{\mu\nu}\sqrt{\det \gamma} \gamma^{kl} \frac{\partial X^\mu }{\partial \sigma^k}\frac{\partial X^\nu}{\partial \sigma^l}
\end{align*}
から導くことができる.(不変性は明らか.相対論で$d^4x \sqrt{\det g}$が不変だったのを思い出そう.)これは,世界面の計量$\gamma^{kl}$が以下の条件を満たすような特別な座標系に移ればすぐわかる.
\begin{align*}
\sqrt{\det \gamma} \gamma^{kl}=\left(
\begin{matrix}
1 & 0 \\
0 & -1
\end{matrix}
\right)
\end{align*}
電磁理論において時間的な光子が,作用の中で負の符号をもつという問題があるが,これは理論のゲージ不変性によって回避される.これと同様に,(24.2.1)と(24.2.2)で$\mu=\nu=0$のときの$\eta_{\mu\nu}$の負符号の問題は,作用(24.2.2)の(境界条件を適切にとったときの)世界面の一般座標変換に対する不変性によって回避される.作用が(24.2.1)の形になる特別な座標系では,世界面の一般座標変換のもとでの不変性のなごりがある.これは一対の大域的共形変換
\begin{align*}
\sigma^+\to f^+(\sigma^+),\sigma^-\to f^-(\sigma^-)
\end{align*}
のもとでの不変性だ.(不変性は明らか.)\par
この弦理論で記述される粒子は現実の自然界でみられるものと一致しない.1971年にラモン,ヌヴォー,シュワルツが半整数スピン粒子とパイ中間子の量子数を持つ粒子を導入しようとし,$d$個のフェルミオン場2重項$\psi^\mu_1(\sigma,\tau),\psi_2^\mu(\sigma,\tau)$を加えることを提案し,ジェルベ,崎田はこの理論の作用として以下のものを提案した.
\begin{align*}
I[X,\psi]=\int d\sigma^+ \int d\sigma^- \left[ T\frac{\partial X^\mu}{\partial \sigma^+}\frac{\partial X_\mu}{\partial \sigma^-} +i\psi^\mu_2 \frac{\partial}{\partial \sigma^+}\psi_{2\mu}+i\psi^\mu_1\frac{\partial}{\partial \sigma^-}\psi_{1\mu} \right]
\end{align*}
(第一項目は(24.2.1)と同じもので,残りの項が追加した項である.)前と同じ共形変換$\sigma^\pm \to f^\pm(\sigma^\pm)$によって
\begin{align*}
& \int d\sigma^+ \int d\sigma^- \left[ T\frac{\partial X^\mu}{\partial \sigma^+}\frac{\partial X_\mu}{\partial \sigma^-} +i\psi^\mu_2 \frac{\partial}{\partial \sigma^+}\psi_{2\mu}+i\psi^\mu_1\frac{\partial}{\partial \sigma^-}\psi_{1\mu} \right] \\
\to &\int d\sigma^+\frac{d f^+}{d\sigma^+ } \int d\sigma^- \frac{d f^-}{d\sigma^-} \Biggl[\left(\frac{d f^+}{d\sigma^+ }\right)^{-1} \left(\frac{d f^-}{d\sigma^-}\right)^{-1} T\frac{\partial X^\mu}{\partial \sigma^+}\frac{\partial X_\mu}{\partial \sigma^-} \\
&\qquad \qquad \qquad +\left(\frac{d f^+}{d\sigma^+ }\right)^{-1} i\psi^\mu_2 \frac{\partial}{\partial \sigma^+}\psi_{2\mu}+\left(\frac{d f^-}{d\sigma^- }\right)^{-1} i\psi^\mu_1\frac{\partial}{\partial \sigma^-}\psi_{1\mu} \Biggr]
\end{align*}
となるから,さらに共形変換を一般化して(24.2.4)と同時にフェルミオン場を以下のように変換すると
\begin{align*}
\psi^{\mu}_1\to \left(\frac{d f^+}{d\sigma^+ }\right)^{-1/2} \psi_1^\mu,\quad \psi^{\mu}_2\to \left(\frac{d f^-}{d\sigma^- }\right)^{-1/2} \psi_2^\mu
\end{align*}
全体は共形不変性が保たれることがわかる.ジェルベ・崎田は,2次元共形不変性と$d$次元ローレンツ不変性を加えて適切な境界条件をとると,この理論はボゾン場$X^\mu$とフェルミオン場$\psi^\mu_r$を\uwave{交換する}以下の微小変換のもとで対称性をもつことに気付いた.
\begin{align*}
\delta \psi^\mu_2(\sigma^+,\sigma^-)=&iT \alpha_2(\sigma^-)\frac{\partial }{\partial \sigma^-}X^\mu (\sigma^+,\sigma^-) \\
\delta \psi^\mu_1(\sigma^+,\sigma^-)=&iT \alpha_1(\sigma^+)\frac{\partial}{\partial \sigma^+}X^\mu (\sigma^+,\sigma^-) \\
\delta X^\mu (\sigma^+,\sigma^-)=&\alpha_2(\sigma^-)\psi^\mu_2 (\sigma^+,\sigma^-)+\alpha_1(\sigma^+)\psi^\mu_1(\sigma^+,\sigma^-)
\end{align*}
ここで$\alpha_1,\alpha_2$はそれぞれ$\sigma^+,\sigma^-$のフェルミオン的微小関数であり,これらはグラスマン数のようなものだ.実際これらの変換により($\alpha_r$はグラスマン数なので$\psi_r^\mu$と交換すると符号が反転することに注意して)
\begin{align*}
\delta I[X,\psi]=&\delta \int d\sigma^+ \int d\sigma^- \left[ T\frac{\partial X^\mu}{\partial \sigma^+}\frac{\partial X_\mu}{\partial \sigma^-} +i\psi^\mu_2 \frac{\partial \psi_{2\mu}}{\partial \sigma^+}+i\psi^\mu_1\frac{\partial \psi_{1\mu}}{\partial \sigma^-} \right] \\
=&\int d\sigma^+ \int d\sigma^- \biggl[ T\alpha_1\frac{\partial \psi^\mu_1 }{\partial \sigma^+}\frac{\partial X_\mu}{\partial \sigma^-} +T\frac{\partial \alpha_1}{\partial \sigma^+}\psi^\mu_1 \frac{\partial X_\mu}{\partial \sigma^-} +T\alpha_2\frac{\partial \psi^\mu_2 }{\partial \sigma^+}\frac{\partial X_\mu}{\partial \sigma^-}  \\
&\qquad \qquad +T\alpha_1\frac{\partial X^\mu }{\partial \sigma^+}\frac{\partial \psi_{1\mu}}{\partial \sigma^-} +T\alpha_2\frac{\partial X^\mu }{\partial \sigma^+}\frac{\partial \psi_{2\mu}}{\partial \sigma^-}+T\frac{\partial X^\mu}{\partial \sigma^+}\frac{\partial \alpha_2}{\partial \sigma^-} \psi_{2\mu}\\
&\qquad \qquad -T\alpha_2\frac{\partial X^\mu}{\partial \sigma^-} \frac{\partial \psi_{2\mu}}{\partial \sigma^+} + T\alpha_2 \psi_2^\mu \frac{\partial^2 X_\mu}{\partial \sigma^+ \partial \sigma^-} \\
&\qquad \qquad -T\alpha_1 \frac{\partial X^\mu}{\partial \sigma^+}\frac{\partial \psi_{1\mu}}{\partial \sigma^-} +T\alpha_1\psi_1^\mu \frac{\partial^2 X_\mu}{\partial \sigma^+\partial \sigma^-} \biggr] \\
=&\int d\sigma^+ \int d\sigma^- \biggl[T\alpha_1\frac{\partial \psi^\mu_1 }{\partial \sigma^+}\frac{\partial X_\mu}{\partial \sigma^-} +T\frac{\partial \alpha_1}{\partial \sigma^+}\psi^\mu_1\frac{\partial X_\mu}{\partial \sigma^-}   \\
&\qquad \qquad +T\alpha_2\frac{\partial X^\mu }{\partial \sigma^+}\frac{\partial \psi_{2\mu}}{\partial \sigma^-}+T\frac{\partial X^\mu}{\partial \sigma^+}\frac{\partial \alpha_2}{\partial \sigma^-} \psi_{2\mu}\\
&\qquad \qquad + T\alpha_2 \psi_2^\mu \frac{\partial^2 X_\mu}{\partial \sigma^+ \partial \sigma^-} \\
&\qquad \qquad  +T\alpha_1\psi_1^\mu \frac{\partial^2 X_\mu}{\partial \sigma^+\partial \sigma^-} \biggr] \\
=&\int d\sigma^+ \int d\sigma^- T \biggl[\frac{\partial}{\partial \sigma^+}\left( \alpha_1 \psi_1^\mu \frac{\partial X_\mu}{\partial \sigma^-} \right)+\frac{\partial}{\partial \sigma^-}\left(\alpha_2 \psi_2^\mu \frac{\partial X_\mu}{\partial \sigma^+}\right)\biggr]
\end{align*}
これは全微分の形であるから,境界上で消えるように境界条件を課せば作用は確かに不変となる.これはやがて超対称性と呼ばれるようになったボゾンとフェルミオンをつなぐ対称性の例になっている.しかし,ここまででは,これは2次元場の理論の対称性に過ぎず,4次元時空の物理的理論の対称性ではない.\par
数年後,ヴェス,ズミノは超対称性模型を構成した.一番単純なものは,マヨラナ場(自己荷電共役なディラック場)$\psi$を一つ,実スカラーと実擬スカラーのボゾン場$A$と$B$の組,それと実スカラーと実擬スカラーのボゾン補助場(微分がラグランジアンに現れない場)$F$と$G$の組を含み,以下の微小変換のもとで不変なものだ.
\begin{align*}
\delta A=&\left(\bar{\alpha}\psi \right) ,\quad \delta B =-i(\bar{\alpha} \gamma_5 \psi) \\
\delta \psi =&\partial_\mu (A+i\gamma_5 B)\gamma^\mu \alpha +(F-i\gamma_5 G)\alpha \\
\delta F=&(\bar{\alpha}\gamma^\mu \partial_\mu \psi),\quad \delta G=-i(\bar{\alpha}\gamma_5 \gamma^\mu \partial_\mu \psi)
\end{align*}
ここで$\alpha$は任意の微小なマヨラナ・フェルミオンのc数の定数4成分パラメータだ.もしこれらの変換のもとでの不変性を作用に要求すると,これらの場$\psi,A,B,F,G$から作られる最も一般的な実・ローレンツ不変・パリティ保存・くりこみ可能なラグランジアン密度は
\begin{align*}
\mc{L}=&-\frac{1}{2}\partial_\mu A \partial^\mu A-\frac{1}{2}\partial _\mu B \partial^\mu B -\frac{1}{2}\bar{\psi}\gamma^\mu \partial_\mu \psi \\
&+\frac{1}{2}(F^2+G^2)+m[FA+GB-\frac{1}{2}\bar{\psi}\psi] \\
&+g\left[ F(A^2-B^2)+2GAB-\bar{\psi}(A+i\gamma_5 B)\psi \right]
\end{align*}
となる.($gFB^2$の項の符号は誤植.)\par
実際に変化分を計算する.$A=A^*$であるから$\delta A=\delta A^*=(\bar{\alpha}\psi)^*=\bar{\psi}\alpha$であり(グラスマン数の複素共役の性質),第一項目は
\begin{align*}
\delta \left(-\frac{1}{2}\partial_\mu A \partial^\mu A^*\right)=&-\frac{1}{2}\bar{\alpha}\partial_\mu \psi \partial^\mu A -\frac{1}{2}\partial_\mu A \partial^\mu \bar{\psi} \alpha 
\end{align*}
となる.同様に$B=B^*$より$\delta B=\delta B^*=-i(\bar{\psi}\gamma_5 \alpha)$と書けて,第二項目は
\begin{align*}
\delta \left(-\frac{1}{2}\partial_\mu B \partial^\mu B^*\right)=&+i\frac{1}{2}\bar{\alpha}\gamma_5 \partial_\mu \psi \partial^\mu B + i\frac{1}{2}\partial_\mu B \partial^\mu \bar{\psi} \gamma_5 \alpha
\end{align*}
となる.第三項目は
\begin{align*}
\delta\bar{\psi}=\delta \psi^\dagger \beta =&[\alpha^\dagger (\gamma^\mu)^\dagger \partial_\mu (A-i\gamma_5 B)+\alpha^\dagger(F+i\gamma_5 G)]\beta \\
=&-\bar{\alpha}  \gamma^\mu\partial_\mu(A+i\gamma_5 B) +\bar{\alpha}(F-i\gamma_5 G) \quad \because (5.4.27)(5.4.30) \\
\delta\left(-\frac{1}{2}\bar{\psi}\gamma^\mu \partial_\mu \psi\right)=&+\frac{1}{2}\bar{\alpha}\gamma^\nu\partial_\nu (A+i\gamma_5 B)\gamma^\mu \partial_\mu \psi -\frac{1}{2}\bar{\alpha}(F-i\gamma_5 G)\gamma^\mu \partial_\mu \psi \\
&-\frac{1}{2}\bar{\psi}\gamma^\mu \partial_\mu \partial_\nu (A+i\gamma_5 B)\gamma^\nu \alpha-\frac{1}{2}\bar{\psi}\gamma^\mu \partial_\mu (F-i\gamma_5 G)\alpha \\
=&+\frac{1}{2}\bar{\alpha}\gamma^\nu \partial_\nu A \gamma^\mu \partial_\mu \psi -\frac{1}{2}\bar{\psi} \gamma^\mu \partial_\mu \partial_\nu A \gamma^\nu \alpha \\
&+i\frac{1}{2}\bar{\alpha}\gamma^\nu \gamma_5 \partial_\nu B \gamma^\mu \partial_\mu \psi - i\frac{1}{2}\bar{\psi}\gamma^\mu\partial_\mu \partial_\nu B \gamma_5 \gamma^\nu \alpha \\
&-\frac{1}{2}\bar{\alpha} F \gamma^\mu \partial_\mu \psi -\frac{1}{2} \bar{\psi}\gamma^\mu \partial_\mu F \alpha \\
&+i\frac{1}{2}\bar{\alpha}\gamma_5 G \gamma^\mu \partial_\mu \psi +i\frac{1}{2} \bar{\psi} \gamma^\mu \partial_\mu G \gamma_5 \alpha
\end{align*}
ここで一つ目の項は
\begin{align*}
+\frac{1}{2}\bar{\alpha}\gamma^\nu \partial_\nu A \gamma^\mu \partial_\mu \psi=&-\frac{1}{2}\bar{\alpha}\gamma^\nu \partial_\mu \partial_\nu A \gamma^\mu \psi+(全微分) \\
=&-\frac{1}{2}\bar{\alpha} \partial_\mu \partial^\mu A \psi+ (全微分) \quad \because \{\gamma^\mu ,\gamma^\nu \}=2\eta^{\mu\nu} \\
=&\frac{1}{2}\bar{\alpha}\partial_\mu A \partial^\mu \psi +(全微分)
\end{align*}
と書ける.二つ目も同様の手順で
\begin{align*}
-\frac{1}{2}\bar{\psi} \gamma^\mu \partial_\mu \partial_\nu A \gamma^\nu \alpha=-\frac{1}{2}\partial_\mu A \partial^\mu \bar{\psi} \alpha 
\end{align*}
と書ける.三,四つ目も同様にして($\gamma_5$の交換による符号だけ気をつけて)
\begin{align*}
+i\frac{1}{2}\bar{\alpha}\gamma^\nu \gamma_5 \partial_\nu B \gamma^\mu \partial_\mu \psi =&-i\frac{1}{2}\bar{\alpha}\gamma_5 \partial_\mu \psi \partial^\mu B \\
- i\frac{1}{2}\bar{\psi}\gamma^\mu\partial_\mu \partial_\nu B \gamma_5 \gamma^\nu \alpha=&- i\frac{1}{2}\partial_\mu B \partial^\mu \bar{\psi} \gamma_5 \alpha
\end{align*}
となる.これらは先程の第一,二項目から生じる変化分とキャンセルする.残りの第四項目も$F=F^*,G=G^*$より$\delta F=\delta F^*=-(\partial_\mu \bar{\psi} \gamma^\mu \alpha)$と$\delta G=\delta G^*=i(\partial_\mu \bar{\psi} \gamma^\mu \gamma_5 \alpha)$
\begin{align*}
\delta \left(\frac{1}{2}F^2\right)=&\frac{1}{2}F\bar{\alpha}\gamma^\mu \partial_\mu \psi -\frac{1}{2}F\partial_\mu \bar{\psi} \gamma^\mu \alpha \\
=&\frac{1}{2}F\bar{\alpha}\gamma^\mu \partial_\mu \psi +\frac{1}{2}\bar{\psi}\partial_\mu F \gamma^\mu \alpha +(全微分)\\
\delta \left(\frac{1}{2}G^2\right)=&-i\frac{1}{2}G\bar{\alpha} \gamma_5 \gamma^\mu \partial_\mu \psi+i\frac{1}{2}G\partial_\mu \bar{\psi} \gamma^\mu \gamma_5 \alpha \\
=&-i\frac{1}{2}G\bar{\alpha} \gamma_5 \gamma^\mu \partial_\mu \psi-i\frac{1}{2} \bar{\psi}\partial_\mu G \gamma^\mu \gamma_5 \alpha+(全微分)
\end{align*}
となる.これらもキャンセルする.よってラグランジアンの第一項目から第四項目までの全て変化分はキャンセルする.$m$に比例する項に関しても計算すると
\begin{align*}
\delta (FA)=&\delta\left(\frac{1}{2}FA +\frac{1}{2}F^* A^*\right) \\
=&\frac{1}{2}F\bar{\alpha}\psi+\frac{1}{2}\bar{\alpha}\gamma^\mu \partial_\mu \psi A+\frac{1}{2}F\bar{\psi}\alpha -\frac{1}{2}\partial_\mu \bar{\psi}\gamma^\mu \alpha A \\
\delta (GB)=&\delta\left(\frac{1}{2}GB+\frac{1}{2}G^*B^*\right) \\
=&-i\frac{1}{2}G\bar{\alpha}\gamma_5 \psi -i\frac{1}{2}\bar{\alpha}\gamma_5 \gamma^\mu \partial_\mu \psi B -i\frac{1}{2}G\bar{\psi}\gamma_5 \alpha +i\frac{1}{2}\partial_\mu \bar{\psi} \gamma^\mu \gamma_5 \alpha B  \\
\delta\left(-\frac{1}{2}\bar{\psi}\psi\right)=&-\frac{1}{2}\bar{\psi}\partial_\mu (A+i\gamma_5 B)\gamma^\mu \alpha -\frac{1}{2}\bar{\psi}(F-i\gamma_5 G)\alpha \\
&+\frac{1}{2}\bar{\alpha}\gamma^\mu\partial_\mu (A+i\gamma_5 B)\psi-\frac{1}{2} \bar{\alpha} (F-i\gamma_5 G)\psi \\
=&-\frac{1}{2}\bar{\psi} \partial_\mu A \gamma^\mu \alpha +\frac{1}{2}\bar{\alpha}\gamma^\mu \partial_\mu A \psi \\
&-i\frac{1}{2}\bar{\psi}\gamma_5 \gamma^\mu \partial_\mu B \alpha+i\frac{1}{2}\bar{\alpha} \gamma^\mu \partial_\mu B \gamma_5 \psi \\
&-\frac{1}{2}\bar{\psi}\alpha F -\frac{1}{2}\bar{\alpha}\psi F \\
&+i\frac{1}{2}\bar{\psi}\gamma_5 \alpha G +i\frac{1}{2}\bar{\alpha}\gamma_5 \psi G
\end{align*}
これらも全てキャンセルする.残りの$g$に比例する項の中身も考える.第一項目は
\begin{align*}
\delta(FA^2)=&\delta\left( \frac{1}{2}FA^2 +\frac{1}{2}F^*A^{*2} \right) \\
=&\frac{1}{2}\bar{\alpha} \gamma^\mu \partial_\mu \psi A^2 +\bar{\alpha}\psi FA \\
&-\frac{1}{2}\partial_\mu \bar{\psi} \gamma^\mu \alpha A^2 +\bar{\psi}\alpha FA \\
=&-\bar{\alpha}\gamma^\mu \partial_\mu A A+\bar{\alpha}\psi FA \\
&+\bar{\psi} \gamma^\mu \alpha \partial_\mu AA +\bar{\psi}\alpha FA +(全微分)
\end{align*}
第二項目は
\begin{align*}
\delta(-FB^2)=&\delta\left(-\frac{1}{2}FB^2-\frac{1}{2}F^* B^{*2}\right) \\
=&-\frac{1}{2}\bar{\alpha} \gamma^\mu \partial_\mu \psi B^2 -i\bar{\alpha} \gamma_5 \psi FB \\
&+\frac{1}{2}\partial_\mu \bar{\psi} \gamma^\mu \alpha B^2 +i\bar{\psi}\gamma_5 \alpha FB \\
=&\bar{\alpha} \gamma^\mu \psi \partial_\mu BB -i\bar{\alpha} \gamma_5 \psi FB \\
&-\partial_\mu \bar{\psi} \gamma^\mu \alpha \partial_\mu B B +i\bar{\psi}\gamma_5 \alpha FB +(全微分)
\end{align*}
第三項目は
\begin{align*}
\delta(2GAB)=&\delta(GAB+G^* A^* B^*) \\
=&-i\bar{\alpha} \gamma_5 \gamma^\mu \partial_\mu \psi AB +\bar{\alpha} \psi GB -i\bar{\alpha} \gamma_5 \psi GA \\
&+i\partial_\mu \bar{\psi} \gamma^\mu \gamma_5 \alpha AB +\bar{\psi}\alpha GB -i\bar{\psi} \gamma_5 \alpha GA \\
=&i\bar{\alpha} \gamma_5 \gamma^\mu \psi \partial_\mu AB+i\bar{\alpha} \gamma_5 \gamma^\mu \psi A \partial_\mu B +\bar{\alpha} \psi GB -i\bar{\alpha} \gamma_5 \psi GA \\
&-i \bar{\psi} \gamma^\mu \gamma_5 \alpha \partial_\mu AB-i \bar{\psi} \gamma^\mu \gamma_5 \alpha A\partial_\mu B +\bar{\psi}\alpha GB -i\bar{\psi} \gamma_5 \alpha GA+(全微分)
\end{align*}
第四項目は
\begin{align*}
\delta(-\bar{\psi}(A+i\gamma_5 B)\psi)=&\bar{\alpha}\gamma^\mu [\partial_\mu (A+i\gamma_5 B)](A+i\gamma_5 B)\psi -\bar{\alpha} (F-i\gamma_5 G)(A+i\gamma_5 B)\psi \\
&-\bar{\psi}\Bigl((\bar{\alpha}\psi)+(\bar{\alpha}\gamma_5 \psi)\gamma_5\Bigr)\psi \\
&-\bar{\psi}(A+i\gamma_5 B)\partial_\mu (A+i\gamma_5 B) \gamma^\mu \alpha -\bar{\psi}(A+i\gamma_5 B)(F-i\gamma_5 G)\alpha \\
=&\bar{\alpha} \gamma^\mu \psi \partial_\mu A A +i\bar{\alpha} \gamma^\mu \gamma_5 \psi A\partial_\mu B+ i\bar{\alpha} \gamma^\mu \gamma_5 \psi \partial_\mu A B-\bar{\alpha}\gamma^\mu \psi \partial_\mu BB \\
&-\bar{\alpha}\psi FA-i\bar{\alpha}\gamma_5 \psi FB+i\bar{\alpha}\gamma_5 \psi GA-\bar{\alpha}\psi GB \\
&-(\bar{\psi}\psi)(\bar{\alpha}\psi)-(\bar{\psi}\gamma_5 \psi)(\bar{\alpha}\gamma_5 \psi) \\
&-\bar{\psi}\gamma^\mu \alpha \partial_\mu AA -i \bar{\psi} \gamma_5 \gamma^\mu \alpha \partial_\mu AB-i\bar{\psi}\gamma_5 \gamma^\mu \alpha A \partial_\mu B +\bar{\psi} \gamma^\mu \alpha \partial_\mu BB \\
&-\bar{\psi}\alpha FA - i\bar{\psi}\gamma_5 \alpha FB +i\bar{\psi}\gamma_5 \alpha AG -\bar{\psi}\alpha BG
\end{align*}
$\gamma_5 \gamma^\mu $の順による符号に気を付けて,これらは全て打ち消しあう.ただし$(\bar{\psi}\psi)(\bar{\alpha}\psi)$と$(\bar{\psi}\gamma_5 \psi )(\bar{\alpha}\gamma_5 \psi)$の打ち消しあいは,$\psi$がマヨラナ場であることを用いてフィルツ変換(26.A.16)を用いた.以上より,ラグランジアンは変換(24.2.8)のもとで不変であることが示された.\par
補助場$F,G$は二次で入っているから,それらを場の方程式
\begin{align*}
\partial_\mu \left( \frac{\partial \mc{L}}{\partial (\partial_\mu F)}\right)=\frac{\partial \mc{L}}{\partial F} \quad \Leftrightarrow& \quad F=-mA-g(A^2-B^2) \\
\partial_\mu \left( \frac{\partial \mc{L}}{\partial (\partial_\mu G)}\right)=\frac{\partial \mc{L}}{\partial G} \quad \Leftrightarrow& \quad G=-mB -2gAB
\end{align*}
で置き換えても同等のラグランジアンを得る.(補助場は物理的な場ではないから.)
\begin{align*}
mFA=& -m^2 A^2 -gmA (A^2-B^2) \\
mGB=& -m^2 B^2 -2gm AB^2 \\
\frac{1}{2}F^2=&\frac{1}{2}m^2 A^2 +\frac{1}{2}g^2(A^2-B^2)^2 +gmA(A^2-B^2) \\
\frac{1}{2}G^2=&\frac{1}{2}m^2 B^2 +2g^2 A^2 B^2 +2gm AB^2 \\
gF(A^2-B^2)=&-gm A(A^2-B^2 )-g^2(A^2-B^2)^2 \\
2gGAB=&-2gmAB^2 -4g^2 A^2B^2
\end{align*}
これらを合わせると
\begin{align*}
\mc{L}=&-\frac{1}{2}\partial_\mu A \partial^\mu A-\frac{1}{2}\partial _\mu B \partial^\mu B -\frac{1}{2}\bar{\psi}\gamma^\mu \partial_\mu \psi \\
&-\frac{1}{2}m^2(A^2+B^2)-\frac{1}{2}\bar{\psi}\psi \\
&-gmA(A^2+B^2)-\frac{1}{2}g^2(A^2+B^2)-g\bar{\psi}(A+i\gamma_5 B)\psi
\end{align*}
となる.(こっちは誤植なし.)このラグランジアン密度ではスカラー$A,B$とフェルミオン$\psi$の質量が等しく$m$になっており,さらに湯川相互作用とスカラー自己相互作用の関係がつく.これは超対称性理論に特徴的なことだ.ヴェス・ズミノはまた,ベクトル場を含む超対称性多重項についても,超対称性変換とラグランジアンを与えた.(26章でやる.)最後に,ヴェス・ズミノは別の論文でコールマン・マンデューラ定理を思い起こし,この定理が破られているのは,対称性生成子が交換関係ではなく反交換関係を満たすことによることを明らかにした.それから数年して初めて,グリオッツィ・シャーク・オリーブが弦理論において,ラモン・ヌヴォー・シュワルツ模型の場に適切な周期的境界条件を課すことによって世界面と時空の朗報で超対称性をもつ超弦理論を構成することが可能であることを示した.


\newpage

\subsection{コールマン・マンデューラ定理}
この補遺ではコールマン・マンデューラ定理の証明をする.この定理によれば,唯一の可能な(超対称代数ではない)リー代数は,並進の生成子$P_\mu$と斉次ローレンツ変換の生成子$J_{\mu\nu}$,それと内部対称性の生成子からなる.内部対称性の生成子とは,$P_\mu$とも$J_{\mu\nu}$とも可換であり,物理的状態に作用したとき,スピンと運動量に依存しないエルミート行列として働くものをいう.ここで「対称性の生成子」$B_\alpha$とは,$S$行列と可換であり
\begin{align*}
\bra{\beta}SB_\gamma\ket{\alpha}=\bra{\beta}B_\gamma S\ket{\alpha}
\end{align*}
その交換子もまた対称性の生成子となり
\begin{align*}
[B_\alpha ,B_\beta]=i\sum_{\gamma}C_{\alpha\beta}^\gamma B_\gamma
\end{align*}
1粒子状態を1粒子状態に変換し
\begin{align*}
B_\alpha \ket{p,n}=\sum_{n'}\Bigl(b_\alpha(p)\Bigr)_{n'n} \ket{p,n'}
\end{align*}
多粒子状態には(24.B.1)のように1粒子状態への作用の直和として作用する
\begin{align*}
B_\alpha \ket{pm,qn,\cdots}=&\sum_{m'}\Bigl(b_\alpha(p)\Bigr)_{m'm} \ket{pm',qn,\cdots} \\
&+\sum_{n'}\Bigl(b_\alpha(q)\Bigr)_{n'n}\ket{pm,qn',\cdots} + \cdots
\end{align*}
ような任意のエルミート演算子$B_\alpha$を意味する.他の技術的要件は必要になってから与える.2・3章で述べた相対論的量子力学の一般原理以外には,この証明に必要な仮定は以下のものだけだ.\par
\noindent \textbf{仮定1}:任意の$M$について,$M$より軽い質量の粒子の種類は有限.\par
\noindent \textbf{仮定2}:任意の2粒子状態はほぼ全てのエネルギー(つまり,例えば,孤立集合以外の全てのエネルギー)において何らかの反応をする.\par
\noindent \textbf{仮定3}:弾性2体散乱の散乱振幅はほぼすべてのエネルギーと角度で,散乱角の解析関数となっている.\par
ここで$S$行列が局所的量子場の理論から導かれることは必要ない.(つまり,(3.5.10)での相互作用項が場と場の微分で構成されていると仮定する必要はない.)\par
4元運動量演算子$P_\mu$と可換な対称性生成子$B_\alpha$のみからなる部分代数についてこの定理を証明する.上で述べたように,そのような対称生成生成子は,多粒子状態に以下のように作用する.
\begin{align*}
B_\alpha \ket{pm,qn,\cdots}=&\sum_{m'}\Bigl(b_\alpha(p)\Bigr)_{m'm} \ket{pm',qn,\cdots} \\
&+\sum_{n'}\Bigl(b_\alpha(q)\Bigr)_{n'n}\ket{pm,qn',\cdots} + \cdots
\end{align*}
ここで$m,n$などは質量が$m_n=\sqrt{-p^\mu p_\mu}$となっている粒子のスピンの$z$成分と種類を表す離散的な添え字.(例えば質量$m_e =\sqrt{-p^\mu p_\mu}$の粒子ならば,スピンは$\uparrow,\downarrow$であり,粒子の種類はさらに電子$e^-$と陽電子$e^+$の内部自由度がある.よってそのときの添え字$m$はその組み合わせの4種類をとる.質量ありの荷電ゲージボゾンならばスピンについて3種類と粒子反粒子で6の自由度がある.だから$b_\alpha(p)$と$b_\alpha(q)$は表記は似ているが,それぞれ$4\times 4$行列か$6\times 6$行列か,などという違いがある.)$b_\alpha(p)$は有限エルミート行列で$B_\alpha$の1粒子状態への作用を定義する.\par
(24.B.1)を用いると,ある固定された$p$に対して$B_\alpha$を$b_\alpha(p)$へと変換する写像は,
\begin{align*}
[B_\alpha ,B_\beta]=i\sum_{\gamma}C_{\alpha\beta}^\gamma B_\gamma
\end{align*}
より
\begin{align*}
&B_\alpha B_\beta \ket{pm,qn,\cdots} \\
=&B_\alpha \left\{\sum_{m'}\Bigl(b_\beta(p)\Bigr)_{m'm} \ket{pm',qn,\cdots}+\sum_{n'}\Bigl(b_\beta(q)\Bigr)_{n'n}\ket{pm,qn',\cdots} + \cdots \right\} \\
=&\sum_{m'm''}\Bigl(b_\alpha (p)\Bigr)_{m''m'}\Bigl(b_\beta(p)\Bigr)_{m'm} \ket{pm'',qn,\cdots}+\sum_{n'm'}\Bigl(b_\alpha(p)\Bigr)_{m'm}\Bigl(b_\beta(q)\Bigr)_{n'n}\ket{pm',qn',\cdots} + \cdots \\
&+\sum_{m'n'}\Bigl(b_\alpha (q)\Bigr)_{n'n}\Bigl(b_\beta(p)\Bigr)_{m'm} \ket{pm',qn',\cdots}+\sum_{n'n''}\Bigl(b_\alpha(q)\Bigr)_{n''n'}\Bigl(b_\beta(q)\Bigr)_{n'n}\ket{pm,qn'',\cdots} + \cdots \\
&+\cdots
\end{align*}
1行目2項目と2行目1項目は$\alpha,\beta$の入れ替えで同じであることがわかる.ほかの交差項も同様であり,$[B_\alpha,B_\beta]$の作用は
\begin{align*}
&[B_\alpha,B_\beta] \ket{pm,qn,\cdots} \\
=&\sum_{m'}\Bigl(b_\alpha (p) b_\beta(p)-b_\beta(p)b_\alpha(p)\Bigr)_{m'm} \ket{pm',qn,\cdots} \\
&+\sum_{n'}\Bigl(b_\alpha (q) b_\beta(q)-b_\beta(q)b_\alpha(q)\Bigr)_{n'n} \ket{pm,qn',\cdots}+\cdots \\
=&\sum_{m'}\Bigl(\left[b_\alpha (p), b_\beta(p)\right]\Bigr)_{m'm} \ket{pm',qn,\cdots} \\
&+\sum_{n'}\Bigl(\left[b_\alpha (q),b_\beta(q)\right]\Bigr)_{n'n} \ket{pm,qn',\cdots}+\cdots
\end{align*}
一方
\begin{align*}
[B_\alpha,B_\beta] \ket{pm,qn,\cdots}=&i\sum_\gamma C^\gamma_{\alpha\beta} B_\gamma \ket{pm,qn,\cdots} \\
=&i\sum_\gamma C^\gamma_{\alpha\beta}\Biggl[\sum_{m'}\Bigl(b_\gamma(p)\Bigr)_{m'm} \ket{pm',qn,\cdots} \\
&\qquad \qquad +\sum_{n'}\Bigl(b_\gamma(q)\Bigr)_{n'n}\ket{pm,qn',\cdots} + \cdots\Biggr]
\end{align*}
よって線形独立性から
\begin{align*}
[b_\alpha(p),b_\beta(p)]=i\sum_\gamma C^\gamma_{\alpha\beta} b_\gamma(p)
\end{align*}
が満たされていることがわかる.よって写像$B_\alpha \to b_\alpha(p)$はリー代数の構造を保つので,リー代数の準同型写像であることがわかる.15.2節の定理によれば,$b_\alpha(p)$のような有限エルミート行列が張る任意のリー代数は,コンパクト半単純リー代数と$U(1)$代数の直和でなければならない.それを$B_\alpha$についても同じことが言いたいが,そのまま適用することはできない.なぜなら,演算子$B_\alpha$と行列$b_\alpha(p)$の間の準同型写像が同型写像であるとは限らないからだ.同型であるためには,ある係数$c^\alpha$とある運動量$p$について$\sum_\alpha c^\alpha b_\alpha(p)=0$がなりたっているならば,いつでも全ての運動量$k$について$\sum_\alpha c^\alpha b_\alpha(k)=0$でなければならない.(単射性の条件より来る.ある運動量$p$を固定したとき,$f_p:B_\alpha \mapsto b_\alpha(p) $が単射であることと$\ker f_p=0$は同値.よって$\sum_\alpha c^\alpha b_\alpha(p)=0$ならば$\sum_\alpha c^\alpha B_\alpha=0$でなければならない.$\sum_\alpha c^\alpha B_\alpha=0$ならば(24.B.1)と線形独立性より全ての$k$で$\sum_\alpha c^\alpha b_\alpha(k)=0$である(これは逆もなりたつので同値).全射性については,像が$b_\alpha(p)$で生成されることからすでに分かっている.$b_\alpha(p)$は生成元であるから,像は全体を覆っており,全射であることは自明.よって上の条件を確かめれば同型性が確かめられる.)\par
$B_\alpha$を1粒子状態$b_\alpha(p)$に写像する準同型写像を考える代わりに,コールマン・マンデューラは$B_\alpha$から固定された運動量$p$と$q$をもつ2粒子状態への$B_\alpha$の作用を表す以下の行列$b_\alpha(p,q)_{m'n',mn}$へ写像する準同型写像を考えた.
\begin{align*}
B_\alpha \ket{pm,qn}=&\sum_{m'}\Bigl(b_\alpha(p)\Bigr)_{m'm} \ket{pm',qn} +\sum_{n'}\Bigl(b_\alpha(q)\Bigr)_{n'n}\ket{pm,qn'} \\
=&\sum_{m'n'}\left[\Bigl(b_\alpha(p)\Bigr)_{m'm}\delta_{n'n}+\Bigl(b_\alpha(q)\Bigr)_{n'n}\delta_{m'm}\right]\ket{pm',qn'} \\
\equiv &\sum_{m'n'}b_\alpha(p,q)_{m'n',mn}\ket{pm',qn'} \\
b_\alpha(p,q)_{m'n',mn}=&\Bigl(b_\alpha(p)\Bigr)_{m'm}\delta_{n'n}+\Bigl(b_\alpha(q)\Bigr)_{n'n}\delta_{m'm}
\end{align*}
4元運動慮$p$と$q$の2粒子状態から運動量$p'$と$q'$で,質量$\sqrt{-p'_\mu p'^\mu}=\sqrt{-p_\mu p^\mu}$と$\sqrt{-q'_\mu q'^\mu}=\sqrt{-q_\mu q^\mu}$の2粒子状態への弾性散乱か準弾性散乱の$S$行列の不変性(上で述べた対称性生成子の性質)から以下の条件が得られる.
\begin{align*}
\bra{p'm',q'n'} S B_\alpha \ket{pm,qn}=&\sum_{m''n''}\bra{p'm',q'n'} S\ket{pm'',qn''}b_\alpha(p,q)_{m''n'',mn} \\
=&\sum_{m''n''} S(pm'',qn''\to p'm',q'n')b_\alpha(p,q)_{m''n'',mn} \\
=&\sum_{m''n''} \delta^4(p'+q'-p-q)S(p',q';p,q)_{m'n',m''n''}b_\alpha(p,q)_{m''n'',mn} \\
=&\delta^4(p'+q'-p-q) \Bigl( S(p',q';p,q) b_\alpha(p,q) \Bigr)_{m'n',mn} \\
=\bra{p'm',q'n'} B_\alpha S \ket{pm,qn}=&\sum_{m''n''}b_\alpha(p',q')_{m'n',m''n''} \bra{p'm'',q'n''} S\ket{pm,qn} \\
=&\sum_{m''n''} b_\alpha(p',q')_{m'n',m''n''} S(pm,qn\to p'm'',q'n'') \\
=&\sum_{m''n''} b_\alpha(p',q')_{m'n',m''n''} \delta^4(p'+q'-p-q)S(p',q';p,q)_{m''n'',mn} \\
=&\delta^4(p'+q'-p-q) \Bigl( b_\alpha(p',q')S(p',q';p,q)  \Bigr)_{m'n',mn} \\
\therefore \quad b_\alpha(p',q')S(p',q';p,q)=&S(p',q';p,q) b_\alpha(p,q)
\end{align*}
ここで$S(p',q';p,q)$は連結$S$行列要素$S(pm,qn\to p'm',q'n')$を使って以下で定義される.
\begin{align*}
S(pm,qn\to p'm',q'n')=\delta^4(p'+q'-p-q)\Bigl( S(p',q';p,q) \Bigr)_{m'n',mn}
\end{align*}
(3.6.10)より前方散乱$\theta=0$で
\begin{align*}
\mathrm{Im} f(0)=\frac{k}{4\pi}\sigma_\alpha
\end{align*}
であり,$f(0)=0$ならば$\mathrm{Im}f(0)=0$より全断面積は$\sigma_\alpha=0$でなければならない.しかし仮定2より$p,q$をどのように選んでも必ず何らかの反応をするから,全散乱断面積は非ゼロである.したがって弾性散乱振幅は前方でゼロにはならないことがわかる.また仮定3より,行列$S(p',q';p,q)$は同じ質量殻上で保存則$p'+q'=p+q$を満たすほぼ全ての$p',q'$について正則(逆$S^{-1}=S^\dagger$が存在.ほぼ全てとは,ある領域でゼロとはならずに,零点が孤立している(測度ゼロ)ということ.ある領域でゼロならば正則性と一致の定理より全域でゼロになってしまう.スピンやパリティ対称性によりある散乱角やエネルギーでゼロになることはあるから「ほぼ全て」.)だから,ほぼ全てのそのような4元運動量について(24.B.5)は単に相似変換だ.\par
これにより,もしほぼ全ての固定された4元運動量$p,q$について$\sum_\alpha c^\alpha b_\alpha(p,q)=0$ならば,同じ質量殻上で保存則$p'+q'=p+q$を満たすほぼ全ての$p',q'$について
\begin{align*}
\sum_\alpha c^\alpha b_\alpha(p',q')=&S(p',q';p,q)\left(\sum_\alpha c^\alpha b_\alpha(p,q) \right)S^{-1}(p',q';p,q) \\
=&0
\end{align*}
となることがわかる.しかし(24.B.4)より
\begin{align*}
0=&\sum_\alpha c^\alpha b_\alpha(p',q') \\
=&\left(\sum_\alpha c^\alpha b_\alpha(p')\right)_{m'm}\delta_{n'n}+\delta_{m'm}\left(\sum_\alpha c^\alpha b_\alpha(q')\right)_{n'n} \\
\therefore \quad & \sum_\alpha c^\alpha b_\alpha(p')_{m'm}=\lambda \delta_{m'm},\quad \sum_\alpha c^\alpha b_\alpha(q')=-\lambda \delta_{n'n}
\end{align*}
となって,これからは$\sum_\alpha c^\alpha b_\alpha(p')$と$\sum_\alpha c^\alpha b_\alpha(q')$はゼロになるとは言えず,単位行列に(反対符号の係数で)比例しているとしか言えない.これを改善するために,$b_\alpha(p)$や$b_\alpha(p,q)$ではなく,それらのトレースレス部分を考える必要がある.

\vskip\baselineskip

(24.B.5)から
\begin{align*}
\mathrm{Tr} b_\alpha(p',q')=&\mathrm{Tr}\left(S(p',q';p,q)\left( b_\alpha(p,q) \right)S^{-1}(p',q';p,q)\right) \\
=&\mathrm{Tr}b_\alpha(p,q) \qquad \because トレースの巡回性
\end{align*}
である.ここで$\mathrm{Tr}A=\sum_{mn}A_{mn,mn}$である.ここで(24.B.4)も使うと
\begin{align*}
\mathrm{Tr} b_\alpha(p',q')=&\mathrm{tr}b_\alpha(p') \sum_n \delta_{nn}+\mathrm{tr}b_\alpha(q')\sum_m \delta_{mm} \\
=&N\left(\sqrt{-q'^\mu q'_\mu}\right) \mathrm{tr} b_\alpha(p')+N\left(\sqrt{-p'^\mu p'_\mu}\right)\mathrm{tr}b_\alpha(q') \\
=&N(\sqrt{-q^\mu q_\mu}) \mathrm{tr} b_\alpha(p')+N(\sqrt{-p^\mu p_\mu})\mathrm{tr}b_\alpha(q') \\
=\mathrm{Tr} b_\alpha(p',q')=&N(\sqrt{-q^\mu q_\mu}) \mathrm{tr} b_\alpha(p)+N(\sqrt{-p^\mu p_\mu})\mathrm{tr}b_\alpha(q)
\end{align*}
ここで$N(m)$は質量$m$の粒子の種類の多重度だ.($m,n$の添え字の自由度である.例えば,前の例でいう電子質量$m_e$の場合$N(m_e)=4$となる.)また弾性散乱であることを用いて,$\sqrt{-p'_\mu p'^\mu}=\sqrt{-p_\mu p^\mu}$と$\sqrt{-q'_\mu q'^\mu}=\sqrt{-q_\mu q^\mu}$を途中で用いた.$\mathrm{tr}$は$\mathrm{Tr}$とは異なり1粒子の指標について和をとり,例えば$\mathrm{tr}b_\alpha(p)=\sum_m (b_\alpha(p))_{mm}$である.両辺を$N(\sqrt{-q^\mu q_\mu})N(\sqrt{-p^\mu p_\mu})$で割れば
\begin{align*}
\frac{\mathrm{tr} b_\alpha(p')}{N(\sqrt{-p'^\mu p'_\mu})}+\frac{\mathrm{tr}b_\alpha(q')}{N(\sqrt{-q'^\mu q'_\mu})}=\frac{\mathrm{tr} b_\alpha(p)}{N(\sqrt{-p^\mu p_\mu})}+\frac{\mathrm{tr}b_\alpha(q)}{N(\sqrt{-q^\mu q_\mu})}
\end{align*}
を得る.$p'+q'=p+q$が成立するほぼ全ての質量殻上の4元運動量について,これが満たされているためには,関数$\mathrm{tr} b_\alpha(p)/N(\sqrt{-p^\mu p_\mu})$が$p$について線形でなければならない.
\begin{align*}
\frac{\mathrm{tr} b_\alpha(p)}{N(\sqrt{-p^\mu p_\mu})}=a^\mu_\alpha p_\mu
\end{align*}
ここで$a^\mu_\alpha$は$p$からも,また表示された$\mu,\alpha$指標以外の全てからも独立.(右辺に定数項が存在することは上の条件式からは禁止されない.しかしこれは$2\to 2$散乱だから(24.B.8)の両辺が2項と2項なので定数が打ち消し影響がないだけで,粒子数が変化するような過程,例えば$1\to 2,2\to 3$散乱などを考えると両辺が1項と2項や,2項と3項定数項が両辺で打ち消しあわない.どのような散乱過程でも成り立つためには定数項が禁止される.定数項を残しておいても物理的状態への内部対称性の作用に変化があるだけで本質的な問題はないらしい.)
運動量演算子$P^\mu$の1次の項を引き去って,新しい対称性生成子を以下のように定義することができる.
\begin{align*}
B^\sharp_\alpha \equiv B_\alpha-a^\mu_\alpha P_\mu
\end{align*}
これは(24.B.9)を用いれば,1粒子状態についてトレースレスの行列$b_\alpha^\sharp (p)$
\begin{align*}
B^\sharp_\alpha\ket{pm} =& \left[B_\alpha-a^\mu_\alpha P_\mu\right]\ket{pm} \\
=&\sum_{m'}\Bigl(b_\alpha (p)\Bigr)_{m'm}\ket{pm'}-a^\mu_\alpha p_\mu \ket{pm} \\
=&\sum_{m'}\left(\Bigl(b_\alpha(p)\Bigr)_{m'm}-\frac{\mathrm{tr} b_\alpha(p)}{N(\sqrt{-p^\mu p_\mu})} \delta_{m'm}\right)\ket{pm'} \quad \because (24.\mathrm{B}.9)\\
=&\sum_{m'm}\Bigl(b_\alpha^\sharp (p)\Bigr)_{m'm}\ket{pm'} \\
\Bigl(b_\alpha^\sharp (p)\Bigr)_{m'm} \equiv &\Bigl(b_\alpha(p)\Bigr)_{m'm}-\frac{\mathrm{tr} b_\alpha(p)}{N(\sqrt{-p^\mu p_\mu})} \delta_{m'm} \\
\mathrm{tr} b_\alpha^\sharp (p) =& \mathrm{tr} b_\alpha(p) -\frac{\mathrm{tr} b_\alpha(p)}{N(\sqrt{-p^\mu p_\mu})} N(\sqrt{-p^\mu p_\mu}) =0
\end{align*}
で表現される.最初の仮定より$P^\mu$は$B_\alpha$と可換であり,単位行列は全てと可換だから,$B^\sharp_\alpha$の交換子は$B_\alpha$の交換子と同じで,$b^\sharp_\alpha(p)$の交換子は$b_\alpha(p)$の交換子と同じだ.
\begin{align*}
[B_\alpha^\sharp,B_\beta^\sharp]=&[B_\alpha , B_\beta]-a^\mu_\beta [B_\alpha , P^\mu] -a^\mu_\alpha [P_\mu,B_\beta]-a^\mu_\alpha a^\nu_\beta[P^\mu ,P^\nu] \\
=&[B_\alpha,B_\beta]=i\sum_\gamma C^\gamma_{\alpha\beta}B_\gamma=i\sum_\gamma C^\gamma_{\alpha\beta}[B^\sharp_\gamma +a^\mu_\gamma P_\mu] \\
\left[b_\alpha^\sharp(p),b^\sharp_\beta(p)\right]=&\left[b_\alpha(p),b_\beta(p)\right] \\
&-\frac{\mathrm{tr} b_\beta(p)}{N(\sqrt{-p^\mu p_\mu})}[b_\alpha(p),I]-\frac{\mathrm{tr} b_\alpha(p)}{N(\sqrt{-p^\mu p_\mu})}[I,b_\beta(p)]+\frac{\mathrm{tr} b_\alpha(p)}{N(\sqrt{-p^\mu p_\mu})}\frac{\mathrm{tr} b_\alpha(p)}{N(\sqrt{-p^\mu p_\mu})} [I,I] \\
=&\left[b_\alpha(p),b_\beta(p)\right]=i\sum_\gamma C^\gamma_{\alpha\beta}b_\gamma(p)=i\sum_\gamma C^\gamma_{\alpha\beta}[b^\sharp_\gamma(p)+a^\mu_\gamma p_\mu]
\end{align*}
また,(24.B.13)の\uwave{交換関係}のトレースがゼロ
\begin{align*}
\mathrm{tr} \left[b_\alpha^\sharp(p),b^\sharp_\beta(p)\right]=&\mathrm{tr} (b_\alpha^\sharp(p)b^\sharp_\beta(p)-b_\beta^\sharp(p)b^\sharp_\alpha(p))=0 \quad \because トレースの巡回性 \\
=&\mathrm{tr}\left(i\sum_\gamma C^\gamma_{\alpha\beta}[b^\sharp_\gamma(p)+a^\mu_\gamma p_\mu]\right)=i\sum_{\gamma}C^\gamma_{\alpha\beta}a^\mu_\gamma p_\mu N(-\sqrt{p^\mu p_\mu})
\end{align*}
より$C^\gamma_{\alpha\beta}a^\mu_\gamma=0$を得る.ここで仮定1より粒子の種類が有限個であり$b_\alpha(p)$が有限次元行列になることを用いた.この仮定がなければ無限次元行列であることを許してしまい,この場合トレースの巡回性$\mathrm{tr}AB=\mathrm{tr}BA$がなりたつとは限らないからだ.この結果を(24.B.12)に用いると,$B^\sharp_\alpha$が$B_\alpha$と同じ交換関係を満たすことが示せる.
\begin{align*}
[B_\alpha^\sharp,B_\beta^\sharp]=&i\sum_\gamma C^\gamma_{\alpha\beta}B_\gamma^\sharp \\
[b_\alpha^\sharp(p),b_\beta^\sharp(p)]=&i\sum_\gamma C^\gamma_{\alpha\beta}b_\gamma^\sharp(p)
\end{align*}
を得る.$B^\sharp_\alpha$も対称性の生成子なのだから,(24.B.5)と同様再び散乱振幅は以下を満たす.
\begin{align*}
b^\sharp_\alpha(p',q')S(p',q';p,q)=S(p',q';p,q)b^\sharp_\alpha(p,q)
\end{align*}
ここで$b^\sharp_\alpha(p,q)$は$B^\sharp_\alpha$の2粒子状態への作用を表す行列だ.
\begin{align*}
\Bigl(b^\sharp_\alpha(p,q)\Bigr)_{m'n',mn}=\Bigl(b^\sharp_\alpha(p) \Bigr)_{m'm}\delta_{n'n}+\delta_{m'm}\Bigl(b^\sharp_\alpha(q)\Bigr)_{n'n}
\end{align*}
また,これは
\begin{align*}
b_\alpha^\sharp(p,q)b_\beta^\sharp(p,q)=&b_\alpha^\sharp(p)b_\beta^\sharp(p) \otimes I+ b_\alpha^\sharp(p)\otimes b_\beta^\sharp(q)+b_\alpha^\sharp(q)\otimes b_\beta^\sharp(p) +I\otimes b_\alpha^\sharp(q)b_\beta^\sharp(q) \\
\left[b_\alpha^\sharp(p,q) , b_\beta^\sharp(p,q)\right]=&(b_\alpha^\sharp(p)b_\beta^\sharp(p)-b_\beta^\sharp(p)b_\alpha^\sharp(p)) \otimes I+I\otimes \left(b_\alpha^\sharp(q)b_\beta^\sharp(q)-b_\beta^\sharp(q)b_\alpha^\sharp(q)\right) \\
=&i\sum_\gamma C^\gamma_{\alpha\beta}\left(b^\sharp_\gamma(p)\otimes I +I\otimes b^\sharp_\gamma(q)\right) \\
=&i\sum_\gamma C^\gamma_{\alpha\beta} b_\gamma^\sharp(p,q)
\end{align*}
となり,$B^\sharp_\alpha$と同じ交換関係を満たす.これらの2粒子行列を扱う利点は,$S(p',q';p,q)$が正則な行列であるから,もし固定された質量殻上の4元運動量$p,q$について$\sum_\alpha c^\alpha b^\sharp(p,q)=0$ならば,同じ質量殻上で$p'+q'=p+q$を満たすほぼ全ての$p',q'$について
\begin{align*}
\sum_\alpha c^\alpha b^\sharp_\alpha(p',q')=&S(p',q';p,q)\left(\sum_\alpha c^\alpha b^\sharp_\alpha(p,q) \right)S^{-1}(p',q';p,q) \\
=&0
\end{align*}
が示せる.ここまでは$b_\alpha(p',q')$について同様の式を導いたときと同様だが,今回はトレースがゼロの行列を扱っているから
\begin{align*}
0=&\sum_\alpha c^\alpha b^\sharp_\alpha(p',q') \\
=&\left(\sum_\alpha c^\alpha b^\sharp_\alpha(p')\right)_{m'm}\delta_{n'n}+\delta_{m'm}\left(\sum_\alpha c^\alpha b_\alpha^\sharp(q')\right)_{n'n} \\
\therefore \quad & \sum_\alpha c^\alpha b^\sharp_\alpha(p')_{m'm}=\lambda \delta_{m'm},\quad \sum_\alpha c^\alpha b^\sharp_\alpha(q')=-\lambda \delta_{n'n} \\
\therefore \quad &\sum_\alpha c^\alpha b^\sharp_\alpha(p')_{m'm}= \sum_\alpha c^\alpha b^\sharp_\alpha(q')=0 \quad \because トレースレス条件
\end{align*}
これより,\uwave{全ての}質量殻上の運動量$k$について$ \sum_\alpha c^\alpha b^\sharp_\alpha(k)=0$が成立すると言いたいが,ここまででは,ほぼ全ての質量殻上の$p'$について,$q'=p+q-p'$も質量殻上にあるなら$\sum_\alpha c^\alpha b^\sharp_\alpha(p')=0$であることを示したに過ぎない($q'$についても同様).この制限を回避するには,コールマン・マンデューラのトリックを使う.

\vskip\baselineskip

質量殻上の$p,q(-p^2=m_p^2,-q^2=m_q^2)$からスタートする.最初に$p,q\to p'(=p+q-q'),q'$の散乱過程を考える.これらが弾性散乱であるためには$-(p+q-q')^2=m_p^2,-q'^2=m_q^2$が要請される.もし$\sum_\alpha c^\alpha b^\sharp_\alpha(p,q)=0$ならば,(24.B.18)と(24.B.16)より
\begin{align*}
\sum_\alpha c^\alpha b^\sharp_\alpha(p,q)=&0 \Leftrightarrow \sum_\alpha c^\alpha b^\sharp_\alpha(p')_{m'm}= \sum_\alpha c^\alpha b^\sharp_\alpha(q')=0 \\
\sum_\alpha c^\alpha b^\sharp_\alpha(p,q')=&\left(\sum_\alpha c^\alpha b^\sharp_\alpha(p)\right)_{m'm}\delta_{n'n}+\delta_{m'm}\left(\sum_\alpha c^\alpha b_\alpha^\sharp(q')\right)_{n'n} \\
=&0 
\end{align*}
が示せる.次に$p,q'\to k,(p+q'-k)$の弾性散乱過程$(-k^2=m_p^2,-(p+q'-k)^2=m_q^2)$を考えると,(24.B.15)から
\begin{align*}
b^\sharp_\alpha(k,l)S(k,l;p,q')=&S(k,l;p,q')b^\sharp_\alpha(p,q') \\
b^\sharp_\alpha(k,p+q'-k)=&S(k,l;p,q')b^\sharp_\alpha(p,q')S^{-1}(k,l;p,q') \\
\sum_\alpha c^\alpha b_\alpha^\sharp (k,p+q'-k)=&S(k,l;p,q')\left(\sum_\alpha c^\alpha b^\sharp_\alpha(p,q')\right) S^{-1}(k,l;p,q') \\
=&0
\end{align*}
が得られる.これより,$p+q'-k$もまた質量殻上にあるような,ほぼ全ての4元運動量$k$について
\begin{align*}
\sum_\alpha c^\alpha b^\sharp_\alpha(k)=0
\end{align*}
が成立する.さて,一つ目の弾性散乱過程の質量殻条件$-(p+q-q')^2=m_p^2,-q'^2=m_q^2$により$q'$の取り得る自由度が2つ削減される.二つ目の弾性散乱過程の質量殻条件$-k^2=m^2_p$により$k$の取り得る自由度が1つ削減される.しかし,条件$-(p+q'-k)^2=m_q^2)$は$k$の自由度を削減しない.なぜならば,$q'$の自由度が2つ残っており,これだけの自由度があればどのような$k$を自由に選んでもこの条件式を満たすように$q'$を調整することが十分にできるからだ.(ただし$\mathbf{p},\mathbf{q}$を十分大きくとる必要がある.$\mathbf{p,q}\approx 0$だと一つ目の条件式での$q'$の取り得る範囲が非常に狭くなってしまい,与えられた$k$に対してうまく$q'$を選ぶことができなくなってしまう.)これにより$k$には自由度3が残り,空間成分$\mathbf{k}$を自由に選ぶだけの自由度が残る.\par
$\Rightarrow$ よって,まとめると,ある固定された質量殻上の4元運動量$p,q$について,$\sum_\alpha c^\alpha b^\sharp_\alpha(p,q)=0$ならば,\uwave{ほぼ全ての}質量殻上の4元運動量$k$について$\sum_\alpha c^\alpha b^\sharp_\alpha(k)=0$である!\par
もしある特定の質量殻上の4元運動量$k_0$について$\sum_\alpha c^\alpha b^\sharp_\alpha(k_0)\neq 0$となると仮定する.ここで,4元運動量$k_0,k$の粒子が$k',k''$に弾性散乱する過程$k_0,k\to k',k''$を考えると,$\sum_\alpha c^\alpha B_\alpha^\sharp$によって生成される対称性により
\begin{align*}
\sum_{\alpha}c^\alpha b^\sharp_\alpha(k',k'')S(k',k'';k_0,k) =S(k',k'';k_0,k) \sum_\alpha c^\alpha b^\sharp_\alpha(k_0,k)
\end{align*}
となる.一方,ここで$p,q\to k',k''$の弾性散乱を考えると
\begin{align*}
\sum_{\alpha}c^\alpha b^\sharp_\alpha(k',k'')S(k',k'';p,q) =S(k',k'';p,q) \sum_\alpha c^\alpha b^\sharp_\alpha(p,q)
\end{align*}
初期値$p,q$が$\sum_\alpha c^\alpha b^\sharp_\alpha(p,q)=0$を満たすことより,質量殻上にあるほぼ全ての$k',k''$について,ここから$\sum_{\alpha}c^\alpha b^\sharp_\alpha(k',k'')=0$を得る.上式左辺$\sum_{\alpha}c^\alpha b^\sharp_\alpha(k',k'')S(k',k'';k_0,k)$がゼロであるから右辺$S(k',k'';k_0,k) \sum_\alpha c^\alpha b^\sharp_\alpha(k_0,k)$もゼロである必要があるが,仮定より$S(k',k'';k_0,k)=0$が要請され,このような過程$k_0,k\to k',k''$は禁止される.ほぼ全ての$k,k',k''$についてこのような過程が禁止されるということは,散乱振幅の解析性についてのここでの仮定に矛盾する.(ほぼどのような$k,k',k''$をとってきても仮定が禁止されるということは,ほぼ全てのエネルギーで反応をしないということになる.よって仮定2に反する.)\par
$\Rightarrow$したがって,ある固定された質量殻上の4元運動量$p,q$について,$\sum_\alpha c^\alpha b^\sharp_\alpha(p,q)=0$ならば,\uwave{全ての}質量殻上の4元運動量$k$について$\sum_\alpha c^\alpha b^\sharp_\alpha(k)=0$である!\par
したがって,$\sum_{\alpha}c^\alpha B_\alpha^\sharp=0$である.p19の議論と同様にして,$B^\sharp_\alpha$を$b^\sharp_\alpha(p)$に移す写像は同型であることがわかる.当然$B_\alpha$と$B^\sharp_\alpha$は同型であるから,$B_\alpha$を$b^\sharp_\alpha(p)$に移す写像は同型となる.\par
これと,$b^\sharp_\alpha(p,q)$は(24.B.16)より独立成分は$N(\sqrt{-p^\mu p_\mu})^2+N(\sqrt{-q^\mu q_\mu})^2$を超えない.よって生成元$\{b_\alpha(p,q)\}$は有限個で,対称性生成子$B_\alpha$はこれと同型なので$B_\alpha$の独立な数も高々有限個だ.($N(\sqrt{-p^\mu p_\mu})N(\sqrt{-q^\mu q_\mu})$個を超えない理由はわからなかった.少なすぎる気がする.)コールマン・マンデューラの定理の証明に必要な最初の仮定の中に,対称性代数が有限次元であることを独立に入れる必要がなかった理由がこれだ.\par
15.2節の定理に従うと,固定された$p,q$についての$b^\sharp_\alpha(p,q)$のような有限エルミート行列のリー代数は高々,コンパクトな半単純リー代数といくつかの$U(1)$代数の直和だ.このリー代数は対称性生成子$B^\sharp_\alpha$のリー代数と同型であることはすでに見たので,$B^\sharp_\alpha$もまた,高々,コンパクトな半単純リー代数といくつかの$U(1)$代数の直和を張るだけだとわかる.

\vskip\baselineskip

まず最初に$U(1)$リー代数の可能性を消そう.\par
任意の一対の質量殻上の運動量$p,q$について,$p,q$を不変に保つようなローレンツ生成子$J$が存在する.もし$p,q$が光円錐上$p^2=q^2=0$にあり,平行ならば,
\begin{align*}
(p+q)^2=&p^2+q^2+2p\cdot q \\
=&2(-E_pE_q +|\mathbf{p}||\mathbf{q}|)=0 \quad \because E_p=|\mathbf{p}|, E_q=|\mathbf{q}|
\end{align*}
である.このときは$J$を$\mathbf{p}$と$\mathbf{q}$の共通の方向の周りの回転にとる.例えば$\mathbf{p,q}$がともに$z$軸方向を向いていれば,$z$軸まわりの回転ととれば$p,q$は変化しない.このときの生成子は$J_3=J_{12}$となる.もしそうでなければ,$p+q$は時間的
\begin{align*}
(p+q)^2=&p^2+q^2+2p\cdot q  \\
=&-m_p^2-m_q^2 +2(-E_pE_q +|\mathbf{p}||\mathbf{q}|\cos \theta) <0 \quad \because E_p=\sqrt{m^2+\mathbf{p}^2 }> |\mathbf{p}|
\end{align*}
になっているはずである.したがって連続ローレンツ変換が存在(空間的な場合は存在しないのだった)し,$\mathbf{p}=-\mathbf{q}$となる重心系に移行して,$\mathbf{p}$と$\mathbf{q}$の共通の方向の周りの回転をし,元の系に戻るようなローレンツ変換を考えればいい.このときのローレンツ変換の生成子が\par
2粒子状態の基底を選んで$J$を対角化することができるから,
\begin{align*}
J\ket{pm,qn}=\sigma(m,n)\ket{pm,qn}
\end{align*}
とできる.さて,$P_\mu$は全ての$B^\sharp_\alpha$と可換((24.B.10)参照)で,$[J,P_\mu]$は$P_\mu$の成分の線形結合((2.4.13)参照)だから,$P_\mu$は全ての$[J,B^\sharp_\alpha]$と可換なことがわかる.
\begin{align*}
0=&[P_\mu[J,B^\sharp_\alpha]]+[J,[B^\sharp_\alpha,P_\mu]]+[B^\sharp_\alpha,[P_\mu,J]] \quad \because ヤコビ恒等式\\
=&[P_\mu[J,B^\sharp_\alpha]]
\end{align*}
$P_\mu$と可換な対称性生成子全体を$B_\alpha$とするのだったから,対称性生成子$[J,B_\alpha^\sharp]$は$B_\alpha$の線形結合でなければならない.それは定義により,$P_\mu$と果敢な対称性生成子の完全系をなす.より詳しく述べると,対称性生成子の交換子を表現する行列は必ずトレースがゼロ
\begin{align*}
\mathrm{tr}[J,B^\sharp_\alpha]=&\int d^4p \sum_{m}\bra{pm}[J,B_\alpha^\sharp]\ket{pm} \\
=&\int d^4p d^4q \sum_{mn}\left(\bra{pm}J\ket{qn}\bra{qn}B_\alpha^\sharp \ket{pm} -\bra{pm}B^\sharp_\alpha \ket{qn}\bra{qn}J \ket{pm}\right)=0
\end{align*}
なので,トレースレス行列$b^\sharp_\beta$で表現される生成子$B^\sharp_\beta$の線形結合でなければならない.
\begin{align*}
[J,B^\sharp_\alpha]=\sum_\beta c^\beta_\alpha B^\sharp_\beta
\end{align*}
しかし,$B^\sharp_\beta$の代数の中の任意の$U(1)$生成子$B_i^\sharp$(エルミートとする)は,可換代数であるから全ての$B^\sharp_\beta$と可換でなければならない.特に$[J,B^\sharp_i]$と可換でなければならない.
\begin{align*}
[B_i^\sharp,[J,B^\sharp_i]]=[B_i^\sharp,\sum_\beta c^\beta_iB_\beta^\sharp]=0
\end{align*}
$J$が対角的になる基底$\ket{pm,qn}$で,この2重交換子の2粒子状態での期待値をとると,任意の$m,n$について以下を得る.
\begin{align*}
0=&\bra{pm,qn}[B_i^\sharp,[J,B^\sharp_i]] \ket{pm,qn} \\
=&\bra{pm,qn}(2B_i^\sharp J B_i^\sharp -J B^\sharp_i B^\sharp_i -B^\sharp_i B^\sharp_i J) \ket{pm,qn} \\
\bra{pm,qn} B_i^\sharp J B_i^\sharp \ket{pm,qn}=&\sum_{m',n'}\bra{pm,qn} B_i^\sharp J \ket{pm',qn'}\left(b^\sharp_i(p,q)\right)_{m'n',mn} \\
=&\sum_{m',n'}\sigma(m',n')\bra{pm,qn} B_i^\sharp \ket{pm',qn'}\left(b^\sharp_i(p,q)\right)_{m'n',mn} \\
=&\sum_{m',n'}\sigma(m',n')\left|\left(b^\sharp_i(p,q)\right)_{m'n',mn}\right|^2 \\
\bra{pm,qn}J B^\sharp_i B^\sharp_i\ket{pm,qn}=&\bra{pm,qn}B^\sharp_i B^\sharp_i J\ket{pm,qn} \\
=&\sigma(m,n)\bra{pm,qn}B^\sharp_i B^\sharp_i \ket{pm,qn} \\
=&\sigma(m,n)\sum_{m',n'}\left|\left(b^\sharp_i(p,q)\right)_{m'n',mn}\right|^2 \\
\therefore \quad 0=&\sum_{m',n'}\Bigl(\sigma(m',n')-\sigma(m,n)\Bigr)\left|\left(b^\sharp_i(p,q)\right)_{m'n',mn}\right|^2
\end{align*}
(ここで$J^\dagger=J$を用いた.)本文の通りの論理展開をしようとすると,$\sigma(m',n')\neq \sigma(m,n)$でなければ$\left(b^\sharp_i(p,q)\right)_{m'n',mn}=0$である必要がある…となるが,少し考えてみれば,$\sigma(1,1)=0,\sigma(1,2)=3,\sigma(2,1)=3,\sigma(2,2)=6$と設定し,$\left(b^\sharp_i(p,q)\right)_{m'n',12}$を全て1と設定してみれば,$m=1,n=2$の要素において
\begin{align*}
&\sum_{m',n'}\Bigl(\sigma(m',n')-\sigma(1,2)\Bigr)\left|\left(b^\sharp_i(p,q)\right)_{m'n',12}\right|^2 \\
=&(0-3)+(3-3)+(3-3)+(6-3)=0
\end{align*}
となり等式が成り立っていることがわかる.この場合$\sigma(1.1),\sigma(2,2)$が$\sigma(1,2)$と異なっているにもかかわらず全ての項で$\left(b^\sharp_i(p,q)\right)_{m'n',12}\neq 0$であるから,これは本文の主張の反例となってしまう.別の証明を考える必要がある.\par
ある固定された軸周りの回転変換はローレンツ変換の$SO(2)$部分群である.$SO(2)$はコンパクト群であり,その表現は整数スピン(角運動量)でラベル付けされる.したがって,$U(1)$代数の生成子がローレンツ変換を受けるならば,$SO(2)$の既約表現の成分の線形結合で書くことができるはずだ.
\begin{align*}
B^\sharp_i =&\sum_k B^\sharp_{ik} \\
B^\sharp_{ik} \to& e^{i\theta J} B^\sharp_{ik}e^{-i\theta J}=e^{ik\theta} B^\sharp_{ik}
\end{align*}
すると
\begin{align*}
[J,B^\sharp_{ik}]=kB^\sharp_{ik}
\end{align*}
を得る.(24.B.20)と合わせるとこれは昇降演算子と同じで
\begin{align*}
JB^\sharp_{ik}=&kB^\sharp_{ik}+B^\sharp_{ik}J \\
J[B^\sharp_{ik}\ket{pm,qn}]=&(k+\sigma(m,n))[B^\sharp_{ik}\ket{pm,qn}]
\end{align*}
となる.
\begin{align*}
\bra{p'm',q'n'}JB^\sharp_{ik}\ket{pm,qn}=&(k+\sigma(m,n))\bra{p'm',q'n'}B^\sharp_{ik}\ket{pm,qn} \\
=&(k+\sigma(m,n))\left(b^\sharp_{ik}(p,q)\right)_{m'n',mn}\delta^4(p-p')\delta^4(q-q') \\
=\sigma(m',n')\bra{p'm',q'n'}B^\sharp_{ik}\ket{pm,qn}=&\sigma(m',n')\left(b^\sharp_{ik}(p,q)\right)_{m'n',mn}\delta^4(p-p')\delta^4(q-q')
\end{align*}
であるから,$k=\sigma(m',n')-\sigma(m,n)$でなければ$\left(b^\sharp_{ik}(p,q)\right)_{m'n',mn}=0$がわかる.(24.B.21)の導出の$B^\sharp_i$を$B^\sharp_{ik}$に置き換えてもう一度条件式を出すと
\begin{align*}
0=&\sum_{m',n'}\Bigl(\sigma(m',n')-\sigma(m,n)\Bigr)\left|\left(b^\sharp_{ik}(p,q)\right)_{m'n',mn}\right|^2 \\
=&k\sum_{m',n'}\left|\left(b^\sharp_{ik}(p,q)\right)_{m'n',mn}\right|^2
\end{align*}
となるから,$k=0$でなければ左辺の和は正定値であり,矛盾が生じる.したがって$k=0$以外の$\left(b^\sharp_{ik}(p,q)\right)_{m'n',mn}$はゼロであり,$\left(b^\sharp_{ik}(p,q)\right)_{m'n',mn}$は$B^\sharp_{ik}$と同型だから,$k=0$以外の$B^\sharp_{ik}$は存在しない.これで$[J,B^\sharp_{i}]=[J,B^\sharp_{i0}]=0B^\sharp_{i0}=0$となり可換であることが示される.以上の議論は最初の$p,q$の選び方に任意性があり$p+q$が時間的でさえあればいいので,他の軸まわりのローレンツ変換に対しても同じ議論が適用できる.これですべてのローレンツ変換と可換であることが示される.
\begin{align*}
[B^\sharp_{i},J^{\mu\nu}]=0
\end{align*}
これらが2.5節でブーストと呼んだものと可換だという事実は,$\left(b^\sharp_i(p)\right)_{n'n}$が3元運動量と独立であることを意味し,さらにこれらが回転と可換であることは$\left(b^\sharp_i(p)\right)_{n'n}$がスピンには単位行列として作用することを意味する.
\begin{align*}
B^\sharp_i\ket{p\sigma n}=&\sum_{n'}\left(b^\sharp_i(p)\right)_{n'n}\delta_{\sigma'\sigma}\ket{p \sigma' n'} \\
=&\sum_{n'} (b^\sharp_i )_{n'n}\ket{p\sigma n'}
\end{align*}
(スピンに対しては単位行列として作用することを陽に書くために$\sigma$だけ$n$から分離して書いた.)したがって,これらの生成子は通常の内部対称性(3.3.29)の生成子だ!

\vskip\baselineskip

これまでの議論で残ったのは,半単純コンパクト・リー代数$B^\sharp_\alpha$だ.しかしこれについては既に24.1節で行っており,リー代数の半単純でコンパクトな部分代数生成子はローレンツ変換と可換であり,$U(1)$生成子のときに示したように,ここからそれらも内部対称性の生成子であることが示せるのだった.(単単純コンパクト性を用いて示しているから,$U(1)$のときは例外的にこの方法は使えなかった.)\par
以上より,$P_\mu$と可換な対称性生成子$B_\alpha$は内部対称性の生成子$B_A,B^\sharp_i$か,$P_\mu$自身の線形結合であることが示せた!

\vskip\baselineskip

次に,運動量演算子と可換\uwave{ではない}対称性生成子が存在する可能性を調べなければならない.一般の対称性生成子$A_\alpha$が4元運動量$p$を持つ1粒子状態$\ket{p,n}$に及ぼす作用は
\begin{align*}
A_\alpha \ket{p,n}=&\sum_{n'} \int d^4p' \ket{p',n'}\bra{p',n'}A_\alpha \ket{p,n} \\
=&\sum_{n'}\int d^4p' \Bigl(\mathcal{A}_\alpha(p',p)\Bigr)_{n'n} \ket{p',n'}
\end{align*}
となっている.ここで,$n,n'$は以前と同様にスピンの$z$成分と粒子の種類を意味する.核$\mc{A}_\alpha(p',p)$は$p,p'$が共に質量殻上になければゼロとする.これから,まず$\mc{A}_\alpha(p',p)$が任意の$p'\neq p$についてゼロとなることを示す.\par
この目的のために,もし$A_\alpha$が対称性生成子ならば
\begin{align*}
A^f_\alpha \equiv \int d^4x \exp(iP\cdot x)A_\alpha \exp(-iP\cdot x) f(x)
\end{align*}
も対称性生成子になることに注意する.ここで$P_\mu$は4元運動量演算子で,$f(x)$は自由に選べる関数だ.実際,1粒子状態に働くとこれは
\begin{align*}
A^f_\alpha \ket{p,n} =& \int d^4x \exp(iP\cdot x)A_\alpha \exp(-iP\cdot x) f(x)\ket{p,n} \\
=& \int d^4x \exp(iP\cdot x)A_\alpha  \ket{p,n} \exp(-ip\cdot x)f(x) \\
=&\int d^4x \exp(iP\cdot x)\sum_{n'}\int d^4p' \Bigl(\mathcal{A}_\alpha(p',p)\Bigr)_{n'n} \ket{p',n'} \exp(-ip\cdot x)f(x) \\
=&\sum_{n'} \int d^4x \int d^4p' \Bigl(\mathcal{A}_\alpha(p',p)\Bigr)_{n'n} \ket{p',n'} \exp(ip' \cdot x)\exp(-ip\cdot x)f(x) \\
=&\sum_{n'} \int d^4p' \int d^4x \exp(i(p'-p) \cdot x)f(x)  \Bigl(\mathcal{A}_\alpha(p',p)\Bigr)_{n'n} \ket{p',n'} \\
=&\sum_{n'} \int d^4p' \tilde{f}(p'-p)  \Bigl(\mathcal{A}_\alpha(p',p)\Bigr)_{n'n} \ket{p',n'}
\end{align*}
これは(24.B.22)と同じ形になっているから,実際に対称性生成子だ.ここで$\tilde{f}$はフーリエ変換
\begin{align*}
\tilde{f}(k)\equiv \int d^4x \exp(ik \cdot x)f(x)
\end{align*}
だ.ここで,どちらもある質量殻上にある一対の4元運動量$p,p+\Delta$が$\Delta\neq0$であり,$\mc{A}_\alpha(p+\Delta ,p)\neq 0$であるとする.これに対して$p'+q'=p+q$を満たす一般の質量殻上の4元運動量$q,p',q'$を考えると,$q+\Delta,p'+\Delta ,q'+\Delta $のいずれも一般には質量殻上にない.ここで,もし$\tilde{f}(k)$が$\Delta$の近傍の十分小さい領域のみ値をとり領域外ではゼロになるような関数に選ぶと,4元運動量$p$の1粒子状態は消滅させない
\begin{align*}
A^f_{\alpha}\ket{p,n}=&\sum_{n'} \int d^4p' \tilde{f}(p'-p)  \Bigl(\mathcal{A}_\alpha(p',p)\Bigr)_{n'n} \ket{p',n'} \\
\approx &\sum_{n'} \tilde{f}(\Delta)  \Bigl(\mathcal{A}_\alpha(p+\Delta ,p)\Bigr)_{n'n} \ket{p+\Delta,n'}\neq 0
\end{align*}
が,$q,p',q'$の1粒子状態全てを消滅させる.($\mc{A}_\alpha$の引数の4元運動量が質量殻外ならばゼロになるから)
\begin{align*}
A^f_{\alpha}\ket{q,n}\approx &\sum_{n'} \tilde{f}(\Delta)  \Bigl(\mathcal{A}_\alpha(q+\Delta ,q)\Bigr)_{n'n} \ket{q+\Delta,n'}=0 \\
A^f_{\alpha}\ket{p',n}\approx &\sum_{n'} \tilde{f}(\Delta)  \Bigl(\mathcal{A}_\alpha(p'+\Delta ,p')\Bigr)_{n'n} \ket{p'+\Delta,n'}=0 \\
A^f_{\alpha}\ket{q',n}\approx &\sum_{n'} \tilde{f}(\Delta)  \Bigl(\mathcal{A}_\alpha(q'+\Delta ,q')\Bigr)_{n'n} \ket{q'+\Delta,n'}=0
\end{align*}
したがって,そのような対称性により
\begin{align*}
\bra{p'm',q'n'}S A^f_\alpha \ket{pm,qn}=&\sum_{m''} \tilde{f}(\Delta)  \Bigl(\mathcal{A}_\alpha(p+\Delta ,p)\Bigr)_{m''m} \bra{p'm',q'n'}S \ket{(p+\Delta)m'',qn} \\
=\bra{p'm',q'n'}A^f_\alpha S \ket{pm,qn}=& 0 \\
\therefore \quad \bra{p'm',q'n'}S \ket{(p+\Delta)m'',qn}=&0
\end{align*}
となり,$(p+\Delta ),q,p',q'$の全てが質量殻上にあるにもかかわらず散乱$(p+\Delta),q \to p',q'$の振幅がゼロになる.これはほぼ全てのエネルギーと散乱角で何らかの散乱が起きることを許すという仮定2,3に矛盾する.よって$\Delta\neq0$ならば$\mathcal{A}_\alpha(p+\Delta ,p)=0$が必要となる.したがってともに質量殻上の$p',p$について$p'\neq p$ならば$\mathcal{A}_\alpha(p' ,p)$がゼロになることが示せた.\par
もし$A_\alpha$が$P_\mu$と交換するとすれば,
\begin{align*}
A_\alpha^f=\left(\int d^4x f(x)\right) A_\alpha
\end{align*}
となり単に$A_\alpha$にc数をかけたものとなり,上の矛盾は起きない.\par
この結果は,対称性生成子$A_\alpha$が$P_\mu$と\uwave{可換でなければならない}ということは意味しない.これは核$\mc{A}_\alpha(p',p)$が$\delta^4(p'-p)$自身に比例する項に加えて$\delta^4(p'-p)$の微分に比例する項も含むことがあるからだ.例えば
\begin{align*}
\Bigl(\mathcal{A}_\alpha(p' ,p)\Bigr)_{n'n}=\delta^4(p'-p)\left(a^0_\alpha(p',p) \right)_{n',n}
\end{align*}
という形をしているとすると
\begin{align*}
A_\alpha^f\ket{p,n}=&\sum_{n'} \int d^4p' \tilde{f}(p'-p)  \Bigl(\mathcal{A}_\alpha(p',p)\Bigr)_{n'n} \ket{p',n'} \\
=&\sum_{n'} \int d^4p' \tilde{f}(p'-p)  \delta^4(p'-p)\left(a^0_\alpha(p',p) \right)_{n',n} \ket{p',n'} \\
=&\sum_{n'} \tilde{f}(0) \left(a^0_\alpha(p) \right)_{n',n} \ket{p,n'}
\end{align*}
$\tilde{f}(0)$は,それが作用する状態には無関係だ.したがって$A^f_\alpha$が4元運動量$p$の状態を消滅させない場合,$q,p',q'$の状態も消滅させない.この場合,$A_\alpha$が$P_\mu$と可換でないと仮定しなくとも質量殻上のほとんど全ての運動量に対して散乱過程$p,q\to p',q'$が可能になる.1階微分までの項を入れてみると
\begin{align*}
\Bigl(\mathcal{A}_\alpha(p' ,p)\Bigr)_{n'n}=&\left(a^0_\alpha(p',p) \right)_{n',n}\delta^4(p'-p)+\left(a^1_\alpha(p',p)\right)^{\mu_1}_{n'n}\frac{\partial}{\partial p^{\mu_1}}\delta^4(p'-p) \\
A^f_\alpha \ket{p,n}=&\sum_{n'} \tilde{f}(0) \left(a^0_\alpha(p) \right)_{n',n} \ket{p,n'} \\
&+\sum_{n'} \int d^4p' \tilde{f}(p'-p)  \left(a^1_\alpha(p',p)\right)^{\mu_1}_{n'n}\frac{\partial}{\partial p^{\mu_1}}\delta^4(p'-p) \ket{p',n'} \\
=&\sum_{n'} \tilde{f}(0) \left(a^0_\alpha(p) \right)_{n',n} \ket{p,n'} \\
&-\sum_{n'}\frac{\partial}{\partial p^{\mu_1}} \left(\tilde{f}(p'-p)  \left(a^1_\alpha(p',p)\right)^{\mu_1}_{n'n}\ket{p',n'}\right)_{p'=p} \\
=&\sum_{n'} \tilde{f}(0) \left(a^0_\alpha(p) \right)_{n',n} \ket{p,n'} \\
&-\sum_{n'} \left(\frac{\partial \tilde{f}}{\partial p^{\mu_1}}\right)(0) \left(a^1_\alpha(p) \right)^{\mu_1}_{n',n} \ket{p,n'}-\sum_{n'}\tilde{f}(0)\left(a^1_\alpha(p)\right)^{\mu_1}_{n'n}\frac{\partial}{\partial p^{\mu_1}}\ket{p,n'} \\
& - \sum_{n'}\tilde{f}(0)\left(\left(\frac{\partial a^1_\alpha}{\partial p^{\mu_1}}\right)(p,p)\right)^{\mu_1}_{n'n}\ket{p,n'}
\end{align*}
ここで
\begin{align*}
\left(a'^0_\alpha(p)\right)_{n'n}=&\tilde{f}(0) \left(a^0_\alpha(p) \right)_{n',n} -\left(\frac{\partial \tilde{f}}{\partial p^{\mu_1}}\right)(0) \left(a^1_\alpha(p) \right)^{\mu_1}_{n',n}-\tilde{f}(0)\left(\left(\frac{\partial a^1_\alpha}{\partial p^{\mu_1}}\right)(p,p)\right)^{\mu_1}_{n'n} \\
\left(a'^1_\alpha(p)\right)_{n'n}^{\mu_1}=&-\tilde{f}(0)\left(a^1_\alpha(p)\right)^{\mu_1}_{n'n}
\end{align*}
と定義すれば
\begin{align*}
A^f_\alpha \ket{p,n}=\sum_{n'}\left(\left(a'^0_\alpha(p)\right)_{n'n} +\left(a'^1_\alpha(p)\right)_{n'n}^{\mu_1}\frac{\partial}{\partial p^{\mu_1}}\right)\ket{p,n'}
\end{align*}
となる.$\delta^4(p'-p)$の$D_\alpha$階微分までの項が入っていたとしても,同様の手順を踏むことで
\begin{align*}
A^f_\alpha \ket{p,n}=\sum_{n'}\left(\left(a'^0_\alpha(p)\right)_{n'n} +\left(a'^1_\alpha(p)\right)_{n'n}^{\mu_1}\frac{\partial}{\partial p^{\mu_1}}+\cdots +\left(a'^{D_\alpha}_\alpha(p)\right)_{n'n}^{\mu_1 \mu_2\cdots \mu_{D_\alpha}} \frac{\partial^{D_\alpha}}{\partial p^{\mu_1}\cdots \partial p^{\mu_{D_\alpha}}}\right)\ket{p,n'}
\end{align*}
という形に書くことができる.同様に$A_\alpha$でも同じ手順を踏めば,例えば1階微分までを含んでいる場合
\begin{align*}
\Bigl(\mathcal{A}_\alpha(p' ,p)\Bigr)_{n'n}=&\left(a^0_\alpha(p',p) \right)_{n',n}\delta^4(p'-p)+\left(a^1_\alpha(p',p)\right)^{\mu_1}_{n'n}\frac{\partial}{\partial p^{\mu_1}}\delta^4(p'-p) \\
A_\alpha \ket{p,n}=&\sum_{n'}\int d^4p' \Bigl(\mathcal{A}_\alpha(p',p)\Bigr)_{n'n} \ket{p',n'} \\
=&\sum_{n'} \left(a^0_\alpha(p,p) \right)_{n',n} \ket{p',n'} \\
&-\sum_{n'}\left(\left(\frac{\partial a^1_\alpha}{\partial p^{\mu_1}}\right)(p,p)\right)^{\mu_1}_{n'n}\ket{p,n'}-\sum_{n'}\left(a^1_\alpha(p,p)\right)^{\mu_1}_{n'n}\frac{\partial}{\partial p^{\mu_1}}\ket{p,n'}
\end{align*}
なので,
\begin{align*}
\left(a'^0_\alpha(p) \right)_{n'n}=&\left(a^0_\alpha(p,p) \right)_{n',n}-\left(\left(\frac{\partial a^1_\alpha}{\partial p^{\mu_1}}\right)(p,p)\right)^{\mu_1}_{n'n} \\
\left(a'^1_\alpha(p)\right)^{\mu_1}_{n'n}=&-\left(a^1_\alpha(p,p)\right)^{\mu_1}_{n'n}
\end{align*}
と定義すれば
\begin{align*}
A_\alpha \ket{p,n}=\sum_{n'}\left(\left(a'^0_\alpha(p)\right)_{n'n} +\left(a'^1_\alpha(p)\right)_{n'n}^{\mu_1}\frac{\partial}{\partial p^{\mu_1}}\right)\ket{p,n'}
\end{align*}
となる.$D_\alpha$階微分までの項が入っていても
\begin{align*}
A_\alpha \ket{p,n}=\sum_{n'}\left(\left(a'^0_\alpha(p)\right)_{n'n} +\left(a'^1_\alpha(p)\right)_{n'n}^{\mu_1}\frac{\partial}{\partial p^{\mu_1}}+\cdots +\left(a'^{D_\alpha}_\alpha(p)\right)_{n'n}^{\mu_1 \mu_2\cdots \mu_{D_\alpha}} \frac{\partial^{D_\alpha}}{\partial p^{\mu_1}\cdots \partial p^{\mu_{D_\alpha}}}\right)\ket{p,n'}
\end{align*}
という形に書くことができる.(記号を乱用しているけど,$A^f_\alpha$での係数$a'_\alpha(p)$と今回の$A_\alpha$での係数$a'_\alpha(p)$は違う.同じ手順をしたら同じ形に書けるというだけ.)この可能性を取り扱うために,コールマン・マンデューラは核$\mc{A}_\alpha(p',p)$が\textbf{ディストリビューション}(超関数)だという「汚い技術的仮定」をした.これは,それぞれが$\delta^4(p'-p)$の高々有限な$D_\alpha$階微分までしか含まない,ということだ.上の変形を用いて言い換えると,それぞれの対称性生成子$A_\alpha$は1粒子状態に対して微分$\partial /\partial p^\mu$の$D_\alpha$次の多項式として働き,その行列係数$a'_\alpha$はこの段階では運動量とスピンに依存してよい.運動量演算子と可換な対称性生成子についての今までの結果を適用するために,コールマン・マンデューラは運動量演算子の$A_\alpha$との$D_\alpha$重交換子
\begin{align*}
B^{\mu_1 \cdots \mu_{D_\alpha}}_{\alpha} \equiv \left[P^{\mu_1},[P^{\mu_2},\cdots ,[P^{\mu_{D_\alpha}},A_\alpha]\cdots ]\right]
\end{align*}
を考えた.$B^{\mu_1 \cdots \mu_{D_\alpha}}_{\alpha}$と$P_\mu$の交換子を運動量$p,p'$の状態で挟んだ行列要素は
\begin{align*}
&\bra{p',n'} [P^\mu,B^{\mu_1 \cdots \mu_{D_\alpha}}_{\alpha}]\ket{p,n} \\
=&\bra{p',n'}\left(P^\mu B^{\mu_1 \cdots \mu_{D_\alpha}}_{\alpha}-B^{\mu_1 \cdots \mu_{D_\alpha}}_{\alpha}P^\mu \right)\ket{p,n} \\
=&(p'-p)^\mu \bra{p',n'}B^{\mu_1 \cdots \mu_{D_\alpha}}_{\alpha}\ket{p,n} \\
=&(p'-p)^\mu \bra{p',n'}\left[P^{\mu_1},[P^{\mu_2},\cdots ,[P^{\mu_{D_\alpha}},A_\alpha]\cdots ]\right] \ket{p,n} \\
=&(p'-p)^{\mu} (p'-p)^{\mu_1}\cdots (p'-p)^{\mu_{D_\alpha}} \bra{p',n'}A_\alpha \ket{p,n} \\
=&(p'-p)^{\mu} (p'-p)^{\mu_1}\cdots (p'-p)^{\mu_{D_\alpha}} \\
&\quad \times\sum_{n''}\left(\left(a'^0_\alpha(p)\right)_{n''n} +\left(a'^1_\alpha(p)\right)_{n''n}^{\nu_1}\frac{\partial}{\partial p^{\nu_1}}+\cdots +\left(a'^{D_\alpha}_\alpha(p)\right)_{n''n}^{\nu_1 \nu_2\cdots \nu_{D_\alpha}} \frac{\partial^{D_\alpha}}{\partial p^{\nu_1}\cdots \partial p^{\nu_{D_\alpha}}}\right)\braket{p',n' |p,n''} \\
=&(p'-p)^{\mu} (p'-p)^{\mu_1}\cdots (p'-p)^{\mu_{D_\alpha}} \\
&\quad \times \left(\left(a'^0_\alpha(p)\right)_{n'n} +\left(a'^1_\alpha(p)\right)_{n'n}^{\nu_1}\frac{\partial}{\partial p^{\nu_1}}+\cdots +\left(a'^{D_\alpha}_\alpha(p)\right)_{n'n}^{\nu_1 \nu_2\cdots \nu_{D_\alpha}} \frac{\partial^{D_\alpha}}{\partial p^{\nu_1}\cdots \partial p^{\nu_{D_\alpha}}}\right) \delta^3 (\mathbf{p}'-\mathbf{p})
\end{align*}
(4個目の等式は一つずつ交換子を外して運動量演算子を運動量に変えていけばいい.$D_\alpha=2,3$くらいで実験すればすぐ法則がわかる.最後は(2.5.19)$\braket{p',n'|p,n}=\delta_{n'n}\delta^3(\mathbf{p}'-\mathbf{p})$を用いた.)$(p'-p)$について$D_\alpha +1$次になっており,デルタ関数にかかっている微分の数が最大で$D_\alpha$次であるから,デルタ関数の微分公式に従って変形しても$(p'-p)$を微分しきることはできず必ず$(p'-p)\delta^3(p'-p)$の形が残る.$x\delta(x)=0$になるのと同じ理屈で,これはゼロになる.$p',p$は任意であったから,結局
\begin{align*}
 [P^\mu,B^{\mu_1 \cdots \mu_{D_\alpha}}_{\alpha}]=0
\end{align*}
が得られる.生成子$B^{\mu_1 \cdots \mu_{D_\alpha}}_{\alpha}$は運動量演算子$P_\mu$と可換であることがわかったから,ここまでの結果(24.B.11)を適用することができて,これは1粒子状態に以下の行列として作用する.
\begin{align*}
B^{\mu_1 \cdots \mu_{D_\alpha}}_{\alpha} \ket{p,n}=&\sum_{n'}\Bigl(b^{\mu_1 \cdots \mu_{D_\alpha}}_{\alpha}(p) \Bigr)_{n'n} \ket{p,n'} \\
b^{\mu_1 \cdots \mu_{D_\alpha}}_{\alpha}(p)=&b^{\sharp \mu_1 \cdots \mu_{D_\alpha}}_{\alpha}+a^{\mu \mu_1 \cdots \mu_{D_\alpha}}_{\alpha} p_\mu 1
\end{align*}
ここで$b^{\sharp \mu_1 \cdots \mu_{D_\alpha}}_{\alpha}(p)$は運動量に依存しないトレースレスのエルミート行列で,通常の内部対称性代数を生成する.$a^{\mu \mu_1 \cdots \mu_{D_\alpha}}_{\alpha}$は運動量に依存しない数定数である.$b^{\sharp \mu_1 \cdots \mu_{D_\alpha}}_{\alpha}(p)$と$a^{\mu \mu_1 \cdots \mu_{D_\alpha}}_{\alpha}$はどちらも$\mu_1,\cdots \mu_{D_\alpha}$について対称だ.(上の変形を見れば明らか.)\par
また$A_\alpha$は$P_\mu$と可換ではない,つまり一般の可能性の場合を考えると
\begin{align*}
A_\alpha P_\mu \ket{p,n}=&\sum_{n'}p_\mu \left(\left(a'^0_\alpha(p)\right)_{n'n} +\left(a'^1_\alpha(p)\right)_{n'n}^{\mu_1}\frac{\partial}{\partial p^{\mu_1}}+\cdots +\left(a'^{D_\alpha}_\alpha(p)\right)_{n'n}^{\mu_1 \mu_2\cdots \mu_{D_\alpha}} \frac{\partial^{D_\alpha}}{\partial p^{\mu_1}\cdots \partial p^{\mu_{D_\alpha}}}\right)\ket{p,n'} \\
\neq &\sum_{n'} \left(\left(a'^0_\alpha(p)\right)_{n'n} +\left(a'^1_\alpha(p)\right)_{n'n}^{\mu_1}\frac{\partial}{\partial p^{\mu_1}}+\cdots +\left(a'^{D_\alpha}_\alpha(p)\right)_{n'n}^{\mu_1 \mu_2\cdots \mu_{D_\alpha}} \frac{\partial^{D_\alpha}}{\partial p^{\mu_1}\cdots \partial p^{\mu_{D_\alpha}}}\right)p_\mu \ket{p,n'} \\
=&P_\mu A_\alpha \ket{p,n} \\
\therefore \quad &[P_\mu ,A_\alpha]\neq 0
\end{align*}
であるが,$A_\alpha $は質量殻上から外すものではないから$-P^\mu P_\mu $とは可換である.
\begin{align*}
&A_\alpha (-P^\mu P_\mu)\ket{p,n} \\
=&m^2 A_\alpha\ket{p,n} \\
=&m^2 \sum_{n'} \left(\left(a'^0_\alpha(p)\right)_{n'n} +\left(a'^1_\alpha(p)\right)_{n'n}^{\mu_1}\frac{\partial}{\partial p^{\mu_1}}+\cdots +\left(a'^{D_\alpha}_\alpha(p)\right)_{n'n}^{\mu_1 \mu_2\cdots \mu_{D_\alpha}} \frac{\partial^{D_\alpha}}{\partial p^{\mu_1}\cdots \partial p^{\mu_{D_\alpha}}}\right)\ket{p,n'} \\
=&\sum_{n'} \left(\left(a'^0_\alpha(p)\right)_{n'n} +\left(a'^1_\alpha(p)\right)_{n'n}^{\mu_1}\frac{\partial}{\partial p^{\mu_1}}+\cdots +\left(a'^{D_\alpha}_\alpha(p)\right)_{n'n}^{\mu_1 \mu_2\cdots \mu_{D_\alpha}} \frac{\partial^{D_\alpha}}{\partial p^{\mu_1}\cdots \partial p^{\mu_{D_\alpha}}}\right)m^2 \ket{p,n'} \\
=&(-P^\mu P_\mu) A_\alpha \ket{p,n} \\
\therefore \quad & [-P^\mu P_\mu ,A_\alpha]=0
\end{align*}
これを用いれば
\begin{align*}
0=&[P_{\mu_1} P^{\mu_1}, [P^{\mu_2},\cdots ,[P^{\mu_{D_\alpha}},A_\alpha]\cdots ] ] \\
=&[P^{\mu_1} , [P^{\mu_2},\cdots ,[P^{\mu_{D_\alpha}},A_\alpha]\cdots ] ]P_{\mu_1}+P_{\mu_1} [P^{\mu_1}, [P^{\mu_2},\cdots ,[P^{\mu_{D_\alpha}},A_\alpha]\cdots ] ] \\
=&B^{\mu_1 \cdots \mu_{D_\alpha}}_{\alpha} P_{\mu_1}+P_{\mu_1}B^{\mu_1 \cdots \mu_{D_\alpha}}_{\alpha} \\
=&2P_{\mu_1}B^{\mu_1 \cdots \mu_{D_\alpha}}_{\alpha}
\end{align*}
となる.(最初の等式では,$P^\nu$も$A_\alpha$も全て$P^\mu P_\mu$と可換であることから交換子がゼロになることを用いている.$D_\alpha=2,3$などで実験してみればすぐわかる.最後の等式では$B^{\mu_1 \cdots \mu_{D_\alpha}}_{\alpha}$が$P^\mu$と可換であることを用いた.)よって
\begin{align*}
0=&p_{\mu_1} b_\alpha^{\mu_1 \cdots \mu_{D_\alpha}}(p) \\
=&p_{\mu_1}b_\alpha^{\sharp \mu_1 \cdots \mu_{D_\alpha}}+ p_\mu p_{\mu_1}a^{\mu \mu_1 \cdots \mu_{D_\alpha}}_{\alpha} 1 \\
=&p_{\mu_1}b_\alpha^{\sharp \mu_1 \cdots \mu_{D_\alpha}}+ \frac{1}{2}p_\mu p_{\mu_1}\left(a^{\mu \mu_1 \cdots \mu_{D_\alpha}}_{\alpha}+a^{\mu_1 \mu \cdots \mu_{D_\alpha}}_{\alpha}\right) 1
\end{align*}
を得る.理論が質量をもつ粒子を含む限り,これは任意の時間的な$p$について満たされなければならない.よって$D_\alpha \geq 1$について
\begin{align*}
b_\alpha^{\sharp \mu_1 \cdots \mu_{D_\alpha}}=0
\end{align*}
と
\begin{align*}
a^{\mu \mu_1 \cdots \mu_{D_\alpha}}_{\alpha}=-a^{\mu_1 \mu \cdots \mu_{D_\alpha}}_{\alpha}
\end{align*}
が満たされている必要がある.しかし$D_\alpha \geq 2$のとき
\begin{align*}
a^{\mu \mu_1\mu_2 \cdots \mu_{D_\alpha}}_{\alpha} =&a^{\mu \mu_2 \mu_1 \cdots \mu_{D_\alpha}}_{\alpha}  \\
=& -a^{\mu_2 \mu \mu_1 \cdots \mu_{D_\alpha}}_{\alpha} \\
=&-a^{\mu_2 \mu_1 \mu \cdots \mu_{D_\alpha}}_{\alpha} \\
=& a^{\mu_1 \mu_2 \mu \cdots \mu_{D_\alpha}}_{\alpha} \\
=&a^{\mu_1 \mu \mu_2 \cdots \mu_{D_\alpha}}_{\alpha} \\
=&-a^{\mu \mu_1\mu_2 \cdots \mu_{D_\alpha}}_{\alpha} \\
\therefore \quad a^{\mu \mu_1\mu_2 \cdots \mu_{D_\alpha}}_{\alpha}=&0
\end{align*}
となる.ここで符号の変化のない置換では$a^{\mu \mu_1\mu_2 \cdots \mu_{D_\alpha}}_{\alpha}$の$\mu_1\cdots \mu_{D_\alpha}$について対称なことを,符号が変化するときには(24.B.30)を使っている.よって残った可能性は,$D_\alpha =0,1 $の二つだ.$D_\alpha=0$については単に
\begin{align*}
A_\alpha \ket{p,n}=\sum_{n'}\left(A^0_{\alpha}(p)\right)_{n'n} \ket{p,n'}
\end{align*}
という形で作用するから,自明に$A_\alpha$と$P_\mu$は可換だ.よって前に証明した事実から,$A_\alpha$は内部対称性か$P_\mu$のある線形結合でなければならない.$D_\alpha=1$については(24.B.27)より
\begin{align*}
B^{\mu_1}_\alpha = [P^{\mu_1},A_\alpha]=& a^{\mu\mu_1}_\alpha P_\mu \\
\therefore \quad [P^\nu,A_\alpha]=&a^{\mu\nu}_\alpha P_\mu
\end{align*}
となる.$a^{\mu\nu}_\alpha$は(24.B.30)より$\mu,\nu$について反対称な数定数だ.一方,(2.4.13)より固有ローレンツ変換の生成子は
\begin{align*}
[P^\nu,J^{\rho\sigma}]=-i\eta^{\nu\rho}P^\sigma +i \eta^{\nu\sigma}P^\rho
\end{align*}
より
\begin{align*}
[P^\nu ,-\frac{1}{2}i a^{\rho\sigma}_\alpha J_{\rho\sigma}]=&-\frac{1}{2}ia^{\rho\sigma}_\alpha\left(-i\delta^\nu_\rho P_\sigma +i \delta^\nu_\sigma P_\rho\right) \\
=&-\frac{1}{2}a_\alpha^{\nu\sigma}P_\sigma + \frac{1}{2}a_\alpha^{\rho \nu}P_\rho \\
=&a_\alpha^{\rho \nu}P_\rho
\end{align*}
となり,$-\frac{1}{2}ia^{\mu\nu}_\alpha J_{\mu\nu}$は上の式を満たしている.さらに$P_\mu$と可換な演算子$B_\alpha$を付け加えても良いので
\begin{align*}
A_\alpha=-\frac{1}{2}ia^{\mu\nu}_\alpha J_{\mu\nu}+B_\alpha
\end{align*}
という形が要請される.$A_\alpha$と$J_{\mu\nu}$が対称性の生成子であるから,演算子$B_\alpha$は対称性の生成子であることがわかる.再び前に証明した事実から,$P_\mu$と可換な対称性生成子$B_\alpha$は内部対称性の生成子と$P_\mu$の線形結合でなければならない.よって$A_\alpha$は(24.B.32)より$P_\mu,J_{\mu\nu}$と内部対称性の生成子の線形結合であることがわかった.以上でコールマン・マンデューラ定理の証明が完了した!

\vskip\baselineskip

\uwave{質量がゼロの粒子のみ}を含む理論では,(24.B.30)を(24.B.28)から導くことは必ずしもできない.なぜなら,$p^\mu p_\mu=0$だから
\begin{align*}
p_\mu p_{\mu_1}\left(a^{\mu \mu_1 \cdots \mu_{D_\alpha}}_{\alpha}+a^{\mu_1 \mu \cdots \mu_{D_\alpha}}_{\alpha}\right) =0
\end{align*}
は
\begin{align*}
a^{\mu \mu_1 \cdots \mu_{D_\alpha}}_{\alpha}+a^{\mu_1 \mu \cdots \mu_{D_\alpha}}_{\alpha} \propto \eta^{\mu\mu_1}
\end{align*}
でも満たすことができるからだ.よって$D_\alpha= 2$のときも考慮する必要がある.ただし,$D_\alpha \geq 3$のときは必ずゼロになる.(これは後で示す.)このときは共形群の代数(ポアンカレ群の$J^{\mu\nu},P^\mu$にスケール変換と特殊共形変換の生成子$D,K^\mu$を加えたもの)
\begin{align*}
&\left[P^\mu , D \right]=iP^\mu ,\quad \left[K^\mu, D\right]=-iK^\mu \\
&\left[P^\mu, K^\nu \right]=2i\eta^{\mu\nu}D+2i J^{\mu\nu} ,\quad \left[K^\mu ,K^\nu\right]=0 \\
&\left[ J^{\rho\sigma},K^\mu \right]=i\eta^{\mu\rho}K^\sigma -i\eta^{\mu\sigma}K^\rho ,\quad \left[J^{\rho\sigma},D\right]=0 \\
&\left[J^{\mu\nu},J^{\rho\sigma}\right]=-i\eta^{\nu\rho}J^{\mu\sigma}+i\eta^{\mu\rho}J^{\nu\sigma}+i\eta^{\sigma\mu}J^{\rho\nu}-i\eta^{\sigma\nu}J^{\rho\mu} \\
&\left[P^\mu,J^{\rho\sigma}\right]=-i\eta^{\nu\rho}P^\sigma +i \eta^{\nu\sigma}P^\rho \\
&\left[P^\mu,P^\nu\right]=0
\end{align*}
を用いる.$D_\alpha=0$では前と同じで自明だ.$D_\alpha=1$では
\begin{align*}
[P^\nu,A_\alpha]=&a^{\mu\nu}_\alpha P_\mu
\end{align*}
となる.ここで比例定数を$C_\alpha$として
\begin{align*}
&a^{\mu\nu}_\alpha+a^{\nu\mu}_\alpha=C_\alpha \eta^{\mu\nu} \\
&C_\alpha =\frac{1}{4}\eta_{\mu\nu}\left(a^{\mu\nu}_\alpha+a^{\nu\mu}_\alpha\right)
\end{align*}
とおく.質量がある場合と同様に計算すると
\begin{align*}
\left[P^\nu ,-\frac{1}{2}ia_\alpha^{\rho\sigma}J_{\rho\sigma}\right]=&-\frac{1}{2}a^{\nu\rho}_\alpha P_\rho +\frac{1}{2}a^{\rho\nu}_\alpha P_\rho \\
=&a^{\rho\nu}_\alpha P_\rho -\frac{1}{2}C_\alpha P^\nu \\
\left[P^\nu,D\right]=&iP^\nu
\end{align*}
となるから
\begin{align*}
A_\alpha=-\frac{1}{2}ia_\alpha^{\mu\nu} J_{\mu\nu}-\frac{1}{2}iC_\alpha D+B_\alpha
\end{align*}
となる.$B_\alpha$は$P_\mu$と可換な演算子だ.$J^{\mu\nu},D$が対称性生成子だから,$B_\alpha$は対称性生成子となる.よって$A_\alpha$は$J^{\mu\nu},P^\mu,D$と内部対称性生成子の線形結合で書けることになる.$D_\alpha=2$のとき
\begin{align*}
\left[P^{\mu_1},\left[P^{\mu_2},A_\alpha \right]\right]=a^{\mu\mu_1 \mu_2 }_\alpha P_\mu
\end{align*}
となる$A_\alpha$を見つければよい.このために比例定数を$C^{\mu_2}_\alpha$として
\begin{align*}
&a^{\mu\mu_1\mu_2}_\alpha+a^{\mu_1\mu\mu_2}_\alpha=\eta^{\mu\mu_1}C^{\mu_2}_\alpha \\
&C_\alpha^{\mu}=\frac{1}{4}\eta_{\nu\sigma}\left(a_\alpha^{\nu\sigma\mu}+a_\alpha^{\sigma\nu\mu}\right)
\end{align*}
とおくと
\begin{align*}
a^{\mu \mu_1\mu_2}_{\alpha} =&a^{\mu \mu_2 \mu_1}_{\alpha} \\
=& -a^{\mu_2 \mu \mu_1}_{\alpha}+\eta^{\mu\mu_2}C^{\mu_1}_\alpha \\
=&-a^{\mu_2 \mu_1 \mu }_{\alpha} +\eta^{\mu\mu_2}C^{\mu_1}_\alpha \\
=& a^{\mu_1 \mu_2 \mu }_{\alpha}- \eta^{\mu_1\mu_2}C^{\mu}_\alpha +\eta^{\mu\mu_2}C^{\mu_1}_\alpha \\
=&a^{\mu_1 \mu \mu_2}_{\alpha}- \eta^{\mu_1\mu_2}C^{\mu}_\alpha +\eta^{\mu\mu_2}C^{\mu_1}_\alpha \\
=&-a^{\mu \mu_1\mu_2}_{\alpha}+\eta^{\mu\mu_1}C^{\mu_2}_\alpha- \eta^{\mu_1\mu_2}C^{\mu}_\alpha +\eta^{\mu\mu_2}C^{\mu_1}_\alpha \\
\therefore \quad a^{\mu \mu_1\mu_2}_{\alpha}=&\frac{1}{2}\eta^{\mu\mu_1}C^{\mu_2}_\alpha- \frac{1}{2}\eta^{\mu_1\mu_2}C^{\mu}_\alpha +\frac{1}{2}\eta^{\mu\mu_2}C^{\mu_1}_\alpha \\
a^{\mu\mu_1 \mu_2 }_\alpha P_\mu=&\frac{1}{2}C^{\mu_2}_\alpha P^{\mu_1}-\frac{1}{2} \eta^{\mu_1\mu_2}C_{\alpha\mu} P^\mu+\frac{1}{2}C^{\mu_1}_\alpha P^{\mu_2}
\end{align*}
であることを用いる.
\begin{align*}
\left[P^{\mu_1},\left[P^{\mu_2},K^\rho\right]\right]=&\left[P^{\mu_1},2i\eta^{\mu_2\rho} D+2iJ^{\mu_2\rho}\right] \\
=&2i\eta^{\mu_2\rho}iP^{\mu_1}+2i(-i\eta^{\mu_1\mu_2}P^\rho+i\eta^{\mu_1\rho}P^{\mu_2}) \\
=&-2\eta^{\mu_2\rho}P^{\mu_1}+2\eta^{\mu_1\mu_2}P^\rho -2\eta^{\mu_1\rho}P^{\mu_2} \\
\left[P^{\mu_1},\left[P^{\mu_2},-\frac{1}{4}C_{\alpha\rho} K^\rho\right]\right]=&\frac{1}{2}C_\alpha^{\mu_2}P^{\mu_1}-\frac{1}{2} \eta^{\mu_1\mu_2}C_{\alpha\mu} P^\mu+\frac{1}{2}C^{\mu_1}_\alpha P^{\mu_2} \\
=&a^{\mu\mu_1 \mu_2 }_\alpha P_\mu
\end{align*}
他の共形群生成子は$P_\mu$との二回交換子で消えてしまうので,任意の数係数$D_{\alpha\mu\nu},E_\alpha$を用いて
\begin{align*}
A_\alpha=-\frac{1}{4}C^\mu_\alpha K_\mu+D_{\alpha\mu\nu}J^{\mu\nu}+E_\alpha D +B_\alpha
\end{align*}
となる.$B_\alpha$は$P^\mu$と可換な演算子だ.再び,$A_\alpha,J^{\mu\nu},K^\mu,D$が対称性生成子だから同様に$B_\alpha$も対称性生成子で,$A_\alpha$は$J^{\mu\nu},P^\mu,K^\mu,D$と内部対称性生成子の線形結合で書けることになる.(もしかしたら$[P^{\mu_2},A_\alpha]$がローレンツ生成子$J^{\mu\nu}$の形で現れ,その結果(24.B.35)より$[P,J]\sim P$となる可能性が存在すると考えるかもしれない.しかしこの可能性を考えると
\begin{align*}
\left[P^\mu,A_\alpha \right]=c_{\alpha\nu} J^{\mu\nu}
\end{align*}
の形になるが,この形は$c_\nu\neq 0$とすると$A_\alpha$が$-P^\mu P_\mu$と可換であるという要請と矛盾する.
\begin{align*}
\left[-P_\mu P^\mu ,A_\alpha \right]=&-\left[P_\mu,A_\alpha \right]P^\mu -P_\mu \left[P^\mu,A_\alpha \right] \\
=&-c^\nu_\alpha J_{\mu\nu}P^\mu-P_\mu c_{\alpha\nu}J^{\mu\nu} \\
=&-c^\nu (J_{\mu\nu}P^\mu+P^\mu J_{\mu\nu})\neq 0
\end{align*}
よって$c_\nu=0$となり,上のような可能性は存在しない.)\par
$D_{\alpha}\geq 3$の場合は$a^{\mu\mu_1\mu_2\mu_3\cdots \mu_{D_\alpha}}_{\alpha}$はゼロになり,したがって共形代数がこの場合の最も大きい代数となる.実際,比例係数を$C^{\mu_2\cdots \mu_{D_\alpha}}$として
\begin{align*}
a^{\mu\mu_1\mu_2\mu_3\cdots \mu_{D_\alpha}}_{\alpha}+a^{\mu_1 \mu \mu_2\mu_3\cdots \mu_{D_\alpha}}_\alpha = \eta^{\mu \mu_1}C_\alpha^{\mu_2 \mu_3 \cdots \mu_{D_\alpha}}
\end{align*}
とすると,前と同じように
\begin{align*}
a^{\mu\mu_1\mu_2\mu_3\cdots \mu_{D_\alpha}}_{\alpha}=&-a^{\mu_1 \mu \mu_2\mu_3\cdots \mu_{D_\alpha}}_\alpha + \eta^{\mu \mu_1}C_\alpha^{\mu_2 \mu_3 \cdots \mu_{D_\alpha}} \\
=&+a^{\mu_2 \mu \mu_1\mu_3\cdots \mu_{D_\alpha}}_\alpha -\eta^{\mu_1\mu_2}C_\alpha^{\mu \mu_3 \cdots \mu_{D_\alpha}} + \eta^{\mu \mu_1}C_\alpha^{\mu_2 \mu_3 \cdots \mu_{D_\alpha}} \\
=&-a^{\mu \mu_1 \mu_2\mu_3\cdots \mu_{D_\alpha}}_\alpha + \eta^{\mu\mu_2}C_\alpha^{\mu_1 \mu_3 \cdots \mu_{D_\alpha}}-\eta^{\mu_1\mu_2}C_\alpha^{\mu \mu_3 \cdots \mu_{D_\alpha}} + \eta^{\mu \mu_1}C_\alpha^{\mu_2 \mu_3 \cdots \mu_{D_\alpha}} \\
a^{\mu\mu_1\mu_2\mu_3\cdots \mu_{D_\alpha}}_{\alpha}=&
\end{align*}



\newpage

\part{超対称代数}
\subsection{次数付きリー代数と次数付きパラメータ}
2.2節で,任意の連続対称性変換を,交換関係$[t_a,t_b]=i\sum_c C^c_{ab}t_c$を満たす線形独立な対称性生成子$t_a$のリー代数で表現する方法を見た.全く同様に超対称性は
\begin{align*}
t_a t_b -(-1)^{\eta_a \eta_b}t_b t_a= i\sum_c C^c_{ab}t_c
\end{align*}
の形の交換関係および反交換関係で表される\textbf{次数付き}リー代数を構成する対称性生成子$t_a$で表現される!ここで$\eta_a$は各々の$a$について$+1$か$0$のどちらかの値をとり,これを生成子$t_a$の次数と呼ばれている.また$C^c_{ab}$は構造定数の数値の組だ.$\eta_a=1$の生成子$t_a$はフェルミオン的と呼ばれ,そうでない生成子は$\eta_a=0$でボゾン的と呼ばれる.(25.1.1)はボゾン的演算子同士,およびボゾン的な演算子とフェルミオン的演算子の間では交換関係を与える
\begin{align*}
[t_a,t_b]=t_a t_b -t_b t_a= i\sum_c C^c_{ab}t_c
\end{align*}
が,フェルミオン的演算子同士の間では反交換関係
\begin{align*}
\{t_a,t_b \}=t_a t_b +t_b t_a= i\sum_c C^c_{ab}t_c
\end{align*}
を与える.\par
(25.1.1)より構造定数は
\begin{align*}
t_bt_a-(-1)^{\eta_a \eta_b}t_at_b =& i\sum_c C^c_{ba}t_c \\
=&-(-1)^{\eta_a \eta_b}\Bigl(t_a t_b -(-1)^{\eta_a \eta_b}t_b t_a\Bigr)= -(-1)^{\eta_a \eta_b} i\sum_c C^c_{ab}t_c \\
\quad \therefore C^c_{ba}=&-(-1)^{\eta_a \eta_{b}}C^c_{ab}
\end{align*}
の条件を満たす必要がある.それから,場の演算子の汎関数として構成された任意の演算子について,2個のボゾン的演算子または2個のフェルミオン的演算子の積はボゾン的であり,1個のボゾン的演算子と1個のフェルミオン的演算子の積はフェルミオン的である.よって$t_a,t_b$が共にボゾン的あるいはフェルミオン的ならば左辺の$t_c$はボゾン的でなければらなず,そうでなければ$t_c$はフェルミオン的でなければならない.これを表すと
\begin{align*}
\eta_c\equiv \eta_a+\eta_b \pmod 2 でなければ C^c_{ab}=0
\end{align*}
が成り立つ.また,このようにして構成された任意の演算子について,ボゾン的またはフェルミオン的演算子のエルミート共役は,それぞれボゾン的またはフェルミオン的だ.演算子$\{ t_a\}$がエルミート演算子なら,構造定数は
\begin{align*}
&\Bigl(t_a t_b -(-1)^{\eta_a \eta_b}t_b t_a\Bigr)^\dagger= -i\sum_c {C^{c}_{ab}}^*t_c^\dagger = -i\sum_c {C^{c}_{ab}}^*t_c \\
&=t_b^\dagger t_a^\dagger -(-1)^{\eta_a \eta_b}t_a^\dagger t_b^\dagger=t_b t_a -(-1)^{\eta_a \eta_b}t_a t_b=i\sum_c C^c_{ba}t_c \\
\therefore \quad & {C^{c}_{ab}}^*=-C^c_{ba}
\end{align*}
を満たす.\par
構造定数はまた,超ヤコビ恒等式
\begin{align*}
(-1)^{\eta_c \eta_a}\left[ \left[t_a,t_b\right\},t_c \right\}+(-1)^{\eta_a \eta_b}\left[ \left[t_b,t_c\right\},t_a \right\}+(-1)^{\eta_b \eta_c}\left[ \left[t_c,t_a\right\},t_b \right\}=0
\end{align*}
から導かれる非線型条件を満たす.ここで「$[\cdots\}$」は拡張された次数付きの交換子
\begin{align*}
\left[ \mc{O},\mc{O}' \right\} \equiv& \mc{O}\mc{O}'-(-1)^{\eta(\mc{O})\eta(\mc{O}')}\mc{O}'\mc{O} \\
=&(-1)^{\eta(\mc{O})\eta(\mc{O}')}\left[ \mc{O}' ,\mc{O} \right\}
\end{align*}
を表す.ここで生成子の任意の積$\mc{O}=t_a t_b t_c \cdots $の次数は$\eta(\mc{O})=\eta_a +\eta_b +\eta_c+\cdots \pmod 2$で与えられると理解する.\par
(25.1.5)を証明するには,左辺の$t_a t_b t_c$と$t_a t_c t_b$の係数がゼロになることを見れば十分.なぜなら,そうすれば(25.1.5)の左辺の循環置換$abc\to bca \to cab$の下での対称性から,生成子の他の全ての積($t_b t_c t_a, t_c t_b t_a$など)もゼロになることが保証されるからだ.(25.1.5)の$t_a t_b t_c$の係数は第一項目
\begin{align*}
\left[ \left[t_a,t_b\right\},t_c \right\}=&\left[ t_at_b ,t_c \right\}+\cdots=t_a t_b t_c +\cdots 
\end{align*}
と第二項目
\begin{align*}
\left[ \left[t_b,t_c\right\},t_a \right\}=\left[ t_b t_c ,t_a \right\}+\cdots =-(-1)^{\eta_a(\eta_b+ \eta_c)}t_a t_b t_c +\cdots
\end{align*}
から生じるので
\begin{align*}
(-1)^{\eta_c \eta_a}+(-1)^{\eta_a \eta_b} \left[-(-1)^{\eta_a (\eta_b+\eta_c)}\right]=&(-1)^{\eta_c \eta_a}-(-1)^{\eta_c \eta_a+2(\eta_a\eta_b)} \\
=&(-1)^{\eta_c \eta_a}-(-1)^{\eta_c \eta_a}=0
\end{align*}
となる.$t_a t_c t_b$の係数は,第二項目
\begin{align*}
\left[ \left[t_b,t_c\right\},t_a \right\}=&-(-1)^{\eta_b \eta_c}\left[ t_c t_b ,t_a \right\}+\cdots \\
=&+(-1)^{\eta_b \eta_c}(-1)^{\eta_a(\eta_b+\eta_c)}t_a t_c t_b +\cdots
\end{align*}
と第三項目
\begin{align*}
\left[ \left[t_c,t_a\right\},t_b \right\}=-(-1)^{\eta_c \eta_a}\left[t_a t_c, t_b\right\}+\cdots =-(-1)^{\eta_c \eta_a}t_a t_c t_b +\cdots
\end{align*}
より生じるので
\begin{align*}
&(-1)^{\eta_a \eta_b}(-1)^{\eta_b \eta_c}(-1)^{\eta_a(\eta_b+\eta_c)}+(-1)^{\eta_b \eta_c}\left[-(-1)^{\eta_c \eta_a}\right] \\
=&(-1)^{\eta_b\eta_c +\eta_a \eta_c+2(\eta_a \eta_b)}-(-1)^{\eta_b \eta_c +\eta_c \eta_a} \\
=&0
\end{align*}
となる.これで証明が終了した.\par
(25.1.1)を(25.1.5)に代入すると,条件式
\begin{align*}
&\left[ \left[t_a,t_b\right\},t_c \right\}=i\sum_d C^d_{ab}\left[t_d , t_c \right\}=-\sum_d C^d_{ab}C^e_{dc} \\
\therefore \quad &\sum_d (-1)^{\eta_c \eta_a} C^d_{ab}C^e_{dc}+\sum_d (-1)^{\eta_a \eta_b} C^d_{bc}C^e_{da}+\sum_d (-1)^{\eta_b \eta_c} C^d_{ca}C^e_{bc}=0
\end{align*}
を得る.全ての生成子がボゾン的な場合($\eta_a=\eta_b=\eta_c=0$)には,(25.1.5)は通常のヤコビ恒等式であり,(25.1.7)は構造定数に対する通常の非線型条件(2.2.22)になる.

\vskip\baselineskip


次数付きパラメータに依存する変換を考える.次数付き$c$数パラメータの組は,通常の数とグラスマン数パラメータ(9.5節で扱ったもの)とを含む「数」と考えることができる.この数は数論の結合則と分配則を満たすが,単純な交換則の代わりに
\begin{align*}
\alpha^a \beta^b=(-1)^{\eta_a \eta_b}\beta^b \alpha^a
\end{align*}
の関係を満たす.ここで記号$\alpha^a,\beta^a,\cdots$は$a$番目のパラメータの異なる値を区別するために使う.これはベクトル代数で$v^a$と$u^a$を使って,ある実ベクトルの異なる値の$a$成分を表すのと同じだ.ここで$a$番目の次数付きパラメータは$\eta_a$を持ち,これも$\alpha^a$がフェルミオン的かボゾン的かに応じて,それぞれ$+1$または$0$の値をとる.すなわち,これらのパラメータはどちらかがボゾン的なら交換
\begin{align*}
\alpha^a \beta^b=\beta^b \alpha^a
\end{align*}
し,両方がフェルミオン的な場合反交換
\begin{align*}
\alpha^a \beta^b=-\beta^b \alpha^a
\end{align*}
する.次数付きパラメータの組の積$\alpha^a \beta^b \gamma^c$は次数$\eta_a+\eta_b +\eta_c +\cdots \pmod 2$を持つ.つまり,そのような積は,フェルミオン的パラメータを奇数個含んでいればフェルミオン的で,それ以外の場合はボゾン的となる.この次数を使えば,次数付きパラメータの積が(25.1.8)のような交換則または反交換則を満たすことは容易にわかる.
\begin{align*}
\mc{O}=&\alpha^a \beta^b \gamma^c \cdots ,\quad \eta(\mc{O})=\eta_a +\eta_b +\eta_c +\cdots \pmod 2\\
\mc{O}'=&\alpha'^d \beta'^e \gamma'^f \cdots ,\quad \eta(\mc{O}')=\eta_d +\eta_e +\eta_f +\cdots \pmod 2\\
\mc{O}\mc{O}'=&(-1)^{\eta(\mc{O})\eta(\mc{O}')}\mc{O}'\mc{O}
\end{align*}
(証明は簡単.$\alpha^a$を一番右に動かすと$(-1)^{\eta_a(\eta_d+\eta_e+\cdots)}$が現れ,次に$\beta^b$を左に持っていって$\alpha_a$の左に移動させると$(-1)^{\eta_b(\eta_d+\eta_e+\cdots)}$が現れ…というのを繰り返して符号を全て足し合わせると$(-1)^{(\eta_a+\eta_b+\cdots)(\eta_d+\eta_e+\cdots )}=(-1)^{\eta(\mc{O})\eta(\mc{O}')}$となる.)\par
次数付きパラメータ$\{\alpha^a\}$の形式的ベキ級数
\begin{align*}
T(\alpha)=1+\sum_a \alpha^a t_a +\sum_{ab}\alpha^a \alpha^b t_{ab}+\cdots 
\end{align*}
で与えられる連続的な変換$T(\alpha)$を考える.(2.2節の$U(T(\theta))$と似たものだが,ここでの$T$は2.2節での$U$だ.群ではなく演算子.)ここで$t_a ,t_{ab}$などは$\alpha$に依らない係数演算子の組で,いまのところ(25.1.1)のような代数関係は全く仮定しない.パラメータ$\alpha^a$が(25.1.8)を満たすから,$t_{ab}$は
\begin{align*}
t_{ab}=(-1)^{\eta_a \eta_b}t_{ba}
\end{align*}
のような対称または反対称条件を満たす必要がある.また,変換$T(\alpha)$全体はボゾン的で,どの次数付きパラメータの任意の値$\alpha^a$とも交換すると仮定するのが便利だ.その場合,各項は$\alpha$と交換する必要があり,よって条件
\begin{align*}
\alpha^a \left(\sum_b\alpha^b t_b \right)=& \left(\sum_b\alpha^b t_b\right) \alpha^a \\
\therefore \quad \alpha^a t_b=&(-1)^{\eta_a \eta_b}t_b \alpha^a
\end{align*}
と
\begin{align*}
\alpha^a \left(\sum_{bc}\alpha^b \alpha^c t_{bc}\right)=& \left(\sum_{bc}\alpha^b \alpha^c t_{bc}\right)\alpha^a \\
\therefore \quad \alpha^a t_{bc}=&(-1)^{\eta_a (\eta_b+\eta_c)}t_{bc}\alpha^a
\end{align*}
を満たす.すなわち,$t_b$と$t_{bc}$はそれ自身があたかも次数付きパラメータであるかのように,次数付きパラメータと次数$\eta_b$および$\eta_b+\eta_c\pmod 2$で交換または反交換する.\par
それ以外に,これらの演算子の満たすべき条件は,$T(\alpha)$が半群(単位元と逆元のない群)を作る,すなわち,次数付きパラメータの異なる値$\alpha$と$\beta$を持つ演算子の積はそれ自身が$T$演算子だという要請だ.
\begin{align*}
T(\alpha)T(\beta)=T(f(\alpha,\beta))
\end{align*}
ここで$f^c(\alpha,\beta)$はそれ自身次数付きパラメータの形式的ベキ級数だ.$T(\beta)=T(0)T(\beta)=T(f(0,\beta))$および$T(\alpha)=T(\alpha)T(0)=T(f(\alpha,0))$より
\begin{align*}
f^c(0,\beta)=\beta^c ,\quad f^c(\alpha,0)=\alpha^c
\end{align*}
でなければならない.したがって$f(\alpha,\beta)$のベキ級数展開は
\begin{align*}
f^c(\alpha,\beta)=\alpha^c+\beta^c+\sum_{ab}f^c_{ab}\alpha^a \beta^b+\cdots
\end{align*}
の形をしている必要がある.ここで$f^c_{ab}$は通常の(ボゾン的な)定数の組で,「…」は次数付きパラメータ$\alpha,\beta$の3次以上の項を表す.$f^c(\alpha,\beta)$が次数付きパラメータであるためには,(25.1.15)の各項は同じ次数を持つ必要がある.つまり左辺$f^c(\alpha,\beta)$がボゾン的またはフェルミオン的であるならば,右辺全体もそうでなくてはならない.よって
\begin{align*}
\eta_c=\eta_a+\eta_b \pmod 2 でなければ f^c_{ab}=0
\end{align*}
となる.ベキ級数(25.1.9)と(25.1.15)を積の法則(25.1.13)に代入すると
\begin{align*}
\mathrm{LHS}=&\left[1+\sum_a \alpha^a t_a +\sum_{ab}\alpha^a \alpha^b t_{ab}+\cdots \right] \left[1+\sum_a \beta^a t_a +\sum_{ab}\beta^a \beta^b t_{ab}+\cdots\right] \\
=&1+\sum_a \left(\alpha^a+\beta^a \right)t_a+\sum_{ab}\left(\alpha^a \alpha^b +\beta^a \beta^b \right)t_{ab}+\sum_{bc}\alpha^b t_b \beta^c t_c+\cdots \\
=&1+\sum_a \left(\alpha^a+\beta^a \right)t_a+\sum_{ab}\left(\alpha^a \alpha^b +\beta^a \beta^b \right)t_{ab}+\sum_{bc}\alpha^b\beta^c (-1)^{\eta_b \eta_c}t_b  t_c+\cdots \\
=\mathrm{RHS}=&1+\sum_a f^a(\alpha,\beta) t_a +\sum_{ab}f^a(\alpha,\beta) f^b(\alpha,\beta) t_{ab}+\cdots \\
=&1+\sum_a\left[\alpha^a+\beta^a+\sum_{bc}f^a_{bc}\alpha^b \beta^c+\cdots \right] t_a  \\
&+\sum_{cd}\left[\alpha^c+\beta^c+\sum_{ab}f^c_{ab}\alpha^a \beta^b+\cdots \right]\left[\alpha^d+\beta^d+\sum_{ab}f^d_{ab}\alpha^a \beta^b+\cdots \right] t_{cd}+\cdots \\
=&1+\sum_a (\alpha^a+\beta^a)t_a+\sum_{ab}\left(\alpha^a\alpha^b +\beta^a \beta^b\right)t_{ab}+\sum_{bc}\alpha^b \beta^c \left(\sum_af^a_{bc}t_a+t_{bc}\right)+\sum_{bc}\beta^b \alpha^c t_{bc}+\cdots \\
=&1+\sum_a (\alpha^a+\beta^a)t_a+\sum_{ab}\left(\alpha^a\alpha^b +\beta^a \beta^b\right)t_{ab}+\sum_{bc}\alpha^b \beta^c \left(\sum_a f^a_{bc}t_a+t_{bc}+(-1)^{\eta_b\eta_c}t_{cb}\right)+\cdots
\end{align*}
が得られる.$1,\alpha^a,\beta^a,\alpha^a\alpha^b,\beta^a\beta^b$の係数は自動的に両辺が一致しているが,$\alpha^a\beta^b$の係数が等しいためには自明でない関係式
\begin{align*}
(-1)^{\eta_a \eta_b}t_a  t_b=&\sum_c f^c_{ab}t_c+t_{ab}+(-1)^{\eta_a\eta_b}t_{ba} \\
=&\sum_c f^c_{ab}t_c+2t_{ab} \quad \because(25.1.10)
\end{align*}
が得られる.これにより生成子$t_a$と群合成関数$f^a(\alpha,\beta)$の$f^c_{ab}$を知ってれば,演算子$t_{ab}$を知ることができる.これをさらに高次についても同様のことを繰り返せば,生成子$t_a$と群合成関数$f^a(\alpha,\beta)$を知っていれば,これにより関数(25.1.9)全体を計算することが可能となる!(この結論は2.2節と同じ.)ただし,これが可能であるためには,$t_a$は以下の条件を満たす必要がある.(25.1.10)を使うと,(25.1.17)および同じ方程式で$a,b$を入れ替えたものの差または和から
\begin{align*}
t_{ab}=&\frac{1}{2}(-1)^{\eta_a \eta_b}t_a  t_b-\frac{1}{2}\sum_c f^c_{ab}t_c \\
=(-1)^{\eta_a\eta_b}t_{ba}=&\frac{1}{2}t_b  t_a-\frac{1}{2}\sum_c (-1)^{\eta_b\eta_a} f^c_{ba}t_c \\
\therefore \quad \left[t_a,t_b \right\}=&t_a  t_b-(-1)^{\eta_a\eta_b}t_b  t_a=\sum_c \left((-1)^{\eta_a \eta_b} f^c_{ab} - f^c_{ba}\right)t_c \equiv i\sum_{c}C^c_{ab}t_c
\end{align*}
となりリー超対称代数(25.1.1)が得られる.その構造定数は
\begin{align*}
iC^c_{ab}=(-1)^{\eta_a \eta_b} f^c_{ab} - f^c_{ba}
\end{align*}
で与えられる.(25.1.3)も(25.1.16)とこの定義から直ちに得られる.

\vskip\baselineskip

反交換するc数$\alpha$の複素共役$\alpha^*$は,$\alpha$と任意の演算子$\mc{O}$との積のエルミート共役が
\begin{align*}
(\alpha \mc{O})^*=\mc{O}^* \alpha^*
\end{align*}
となるように定義する.したがって複素共役のもとでのc数の積の振る舞いは,エルミート共役のもとでの演算子のふるまいと同じく
\begin{align*}
(\alpha\beta)^*=\beta^* \alpha^*
\end{align*}
となり,$\alpha^*$は$\alpha$と同じ次数を持つ.(まぁこの性質は24.2節で超対称性を確かめるために既に用いたんだけど.)


\newpage

\subsection{超対称代数}
$S$行列と可換な対称性生成子の一般的な次数付きリー代数を考える.$Q$を任意のフェルミオン的な対称性生成子とすると,当然$U^{-1}(\Lambda)Q U(\Lambda)$もそうなる.(3.3節の議論より$S$演算子と$U(\Lambda)$は可換だったことを思い出そう.)ここで$U(\Lambda)$は任意の斉次ローレンツ変換$\tensor{\Lambda}{^\mu_\nu}$に対応する量子力学的演算子だ.したがって$U^{-1}(\Lambda)QU(\Lambda)$はフェルミオン的な対称性生成子の完全系の線形結合であり,よって,この生成子の完全系は斉次ローレンツ群の表現になっていなければならない.したがって,個々の生成子は,それが属する斉次ローレンツ群の既約表現に従って分類できる.\par
5.6節で説明したように,任意の演算子の組が持つ斉次ローレンツ群の表現は,
\begin{align*}
\mathbf{A}\equiv \frac{1}{2}\left(\mathbf{J}+i\mathbf{K}\right) ,\quad \mathbf{B}\equiv \frac{1}{2}\left(\mathbf{J}-i\mathbf{K}\right)
\end{align*}
で定義される生成子$\mathbf{A,B}$の交換関係を与えることで特定できる.ここで$\mathbf{J,K}$はそれぞれ回転とブーストのエルミート生成子だ.これらは交換関係
\begin{align*}
[A_i,A_j]=i\sum_k \epsilon_{ijk}A_k ,\quad [B_i,B_j ]=i\sum_k \epsilon_{ijk} B_k ,\quad [A_i ,B_j]=0
\end{align*}
を満たす.ここで$i,j,k$は$1,2,3$の値をとり,$\epsilon_{ijk}$は完全反対称で$\epsilon_{123}=+1$だ.$\mathbf{A,B}$はともに$SU(2)$代数を構成するから,斉次ローレンツ群の表現は2個の独立な$SU(2)$スピンをもつ状態のように,2個の整数または半整数$A,B$で指定され,その表現の要素は$-A\leq a \leq +A, -B\leq b \leq +B$を間隔1おきに値をとる2個の添え字$a,b$で指定される.より詳しく言えば,斉次ローレンツ群の$(A,B)$表現を形成する$(2A+1)(2B+1)$個の演算子$Q^{AB}_{ab}$は交換関係
\begin{align*}
U^{-1}(\Lambda)Q^{AB}_{ab}U(\Lambda)=&e^{-\frac{i}{2}\omega_{\mu\nu}J^{\mu\nu}}Q^{AB}_{ab}e^{\frac{i}{2}\omega_{\mu\nu}J^{\mu\nu}} \\
=&e^{-i\theta_i J_i+i\omega_i K_i}Q^{AB}_{ab}e^{i \theta_i J_i -i\omega_i K_i} \\
=&e^{-i\alpha_i A_i -i\beta_i B_i}Q^{AB}_{ab}e^{i\alpha_i A_i +i\beta_i B_i} \quad (\alpha_i=\theta_i+i\omega_i ,\beta_i=\theta-i\omega_i )\\
=&Q^{AB}_{ab}-i\alpha_i[A_i,Q^{AB}_{ab}]-i\beta_i [B_i ,Q^{AB}_{ab}]+\cdots \quad \because \mathrm{BCH}公式 \\
=\sum_{a'b'}D^{A0}_{aa'}(\Lambda)D^{0B}_{bb'}(\Lambda)Q^{AB}_{a'b'}=&Q^{AB}_{ab}+i\alpha_i \sum_{a'}(J^{(A)}_i)_{aa'}Q^{AB}_{a'b}+i\beta_i\sum_{b'}(J^{(B)}_i)_{bb'}Q^{AB}_{ab'}+\cdots  \quad \because (5.7.16)\\
\therefore \quad [\mathbf{A},Q^{AB}_{ab}]=&-\sum_{a'}\mathbf{J}^{(A)}_{aa'}Q^{AB}_{a'b},\quad [\mathbf{B},Q^{AB}_{ab}]=-\sum_{b'}\mathbf{J}^{(B)}_{bb'}Q^{AB}_{ab'}
\end{align*}
を満たす.ここで$\mathbf{J}^{(j)}_{\sigma\sigma'}$は角運動量$j$のスピン3元ベクトル行列で(5.6.16)(5.6.17)
\begin{align*}
\left(J_1^{(j)}\pm iJ^{(j)}_2\right)_{\sigma'\sigma}=&\delta_{\sigma', \sigma\pm 1}\sqrt{(j\mp \sigma)(j\pm \sigma+1)} \\
\left(J^{(j)}_3\right)_{\sigma'\sigma}=&\delta_{\sigma'\sigma}\sigma
\end{align*}
である.(25.2.4)から(5.7.5)
\begin{align*}
-\left(\mathbf{J}^{(j)}\right)^*_{\sigma'\sigma}=(-1)^{\sigma'-\sigma}\left(\mathbf{J}^{(j)}\right)_{-\sigma',-\sigma}
\end{align*}
を得る.実際
\begin{align*}
\left(J_1^{(j)}\right)_{\sigma'\sigma}=&\frac{1}{2}\left(\delta_{\sigma',\sigma+ 1}\sqrt{(j- \sigma)(j+ \sigma+1)}+\delta_{\sigma',\sigma-1}\sqrt{(j+ \sigma)(j- \sigma+1)}\right) \\
=&\frac{1}{2}\left( \delta_{-\sigma',-\sigma- 1}\sqrt{(j+( -\sigma))(j- (-\sigma) +1)}+\delta_{-\sigma',-\sigma+1}\sqrt{(j-(- \sigma))(j+(- \sigma)+1)}\right) \\
=&\frac{1}{2}\Bigl( (-1)^{\sigma'-(\sigma-1)}\delta_{-\sigma',-\sigma+1}\sqrt{(j-(- \sigma))(j+(- \sigma)+1)} \quad \because \sigma'=\sigma-1\\
&+(-1)^{(\sigma'-(\sigma+1))}\delta_{-\sigma',-\sigma- 1}\sqrt{(j+( -\sigma))(j- (-\sigma) +1)}\Bigr) \quad \because \sigma'=\sigma+1 \\
 &=-(-1)^{\sigma'-\sigma}\left(J_1^{(j)}\right)_{-\sigma',-\sigma}^* \\
 \left(J_2^{(j)}\right)_{\sigma'\sigma}=&\frac{1}{2i}\left(\delta_{\sigma',\sigma+ 1}\sqrt{(j- \sigma)(j+ \sigma+1)}-\delta_{\sigma',\sigma-1}\sqrt{(j+ \sigma)(j- \sigma+1)}\right) \\
=&\frac{1}{2i}\left( \delta_{-\sigma',-\sigma- 1}\sqrt{(j+( -\sigma))(j- (-\sigma) +1)}-\delta_{-\sigma',-\sigma+1}\sqrt{(j-(- \sigma))(j+(- \sigma)+1)}\right) \\
=&\frac{1}{2i}\Bigl( -(-1)^{\sigma'-(\sigma-1)}\delta_{-\sigma',-\sigma+1}\sqrt{(j-(- \sigma))(j+(- \sigma)+1)} \quad \because \sigma'=\sigma-1\\
&+(-1)^{(\sigma'-(\sigma+1))}\delta_{-\sigma',-\sigma- 1}\sqrt{(j+( -\sigma))(j- (-\sigma) +1)}\Bigr) \\
=&(-1)^{\sigma'-\sigma}\frac{1}{2i}\Bigl( \delta_{-\sigma',-\sigma+1}\sqrt{(j-(- \sigma))(j+(- \sigma)+1)} \\
&-\delta_{-\sigma',-\sigma- 1}\sqrt{(j+( -\sigma))(j- (-\sigma) +1)}\Bigr) \quad \because \sigma'=\sigma+1 \\
=&-(-1)^{\sigma'-\sigma}\left(J_2^{(j)}\right)_{-\sigma',-\sigma}^* \\
\left(J_3^{(j)}\right)_{\sigma'\sigma}=&\delta_{\sigma'\sigma}\sigma \\
=&-(-\sigma)\delta_{-\sigma',-\sigma}=-(-1)^{\sigma'-\sigma}(-\sigma)\delta_{-\sigma',-\sigma} \\
=&-(-1)^{\sigma'-\sigma}\left(J_3^{(j)}\right)_{-\sigma',-\sigma}^*
\end{align*}
となる.これにより,$Q^j_\sigma$が回転群のスピン$j$表現に従って変換する場
\begin{align*}
\delta Q^j_\sigma=i\theta_i \sum_{\sigma'}\left(J^{(j)}_i\right)_{\sigma\sigma'}Q^{j}_{\sigma'}
\end{align*}
ならば,$(-1)^{j-\sigma}Q^{j*}_{-\sigma}$もそうである.
\begin{align*}
\delta \left[(-1)^{j-\sigma}Q^{*j}_{-\sigma}\right]=&(-1)^{j-\sigma}(-i)\theta_i \sum_{\sigma'}\left(J^{(j)}_i\right)^*_{-\sigma,\sigma'}Q^{*j}_{\sigma'} \\
=&i(-1)^{j-\sigma}\theta_i \sum_{\sigma'}(-1)^{\sigma'-(-\sigma)}\left(J^{(j)}_i\right)_{\sigma,-\sigma'}Q^{*j}_{\sigma'} \\
=&i(-1)^{j-\sigma}\theta_i \sum_{\sigma'}(-1)^{-\sigma'-(-\sigma)}\left(J^{(j)}_i\right)_{\sigma,\sigma'}Q^{*j}_{-\sigma'} \\
=&i\theta_i \sum_{\sigma'}\left(J^{(j)}_i\right)_{\sigma,\sigma'}\left[(-1)^{j-\sigma'}Q^{*j}_{-\sigma'}\right]
\end{align*}
(ここで三個目の等号では,$\sigma'$は$-j$から$j$までの和であるから,$\sigma'\to -\sigma'$と変えても和の範囲に影響がないことを用いた.)また,さらに(25.2.1)より$\mathbf{A}^*=\mathbf{B}$がわかる.これを用いると,斉次ローレンツ群の$(A,B)$表現に従って変換する演算子$Q^{AB}_{ab}$のエルミート共役の変換性が
\begin{align*}
U^{-1}(\Lambda)\left(Q^{AB}_{ab}\right)^\dagger U(\Lambda)=&\left(U^{-1}(\Lambda)Q^{AB}_{ab}U(\Lambda)\right)^\dagger \\
=&\left(e^{-\frac{i}{2}\omega_{\mu\nu}J^{\mu\nu}}Q^{AB}_{ab}e^{\frac{i}{2}\omega_{\mu\nu}J^{\mu\nu}}\right)^\dagger \\
=&\left(e^{-i\alpha_i A_i -i\beta_i B_i}Q^{AB}_{ab}e^{i\alpha_i A_i +i\beta_i B_i}\right)^\dagger \\
=&\left(Q^{AB}_{ab}-i\alpha_i[A_i,Q^{AB}_{ab}]-i\beta_i [B_i ,Q^{AB}_{ab}]+\cdots \quad\right)^\dagger \because \mathrm{BCH}公式 \\
=&\left(Q^{AB}_{ab}\right)^\dagger+i\beta_i\left([A_i,Q^{AB}_{ab}]\right)^\dagger+i\alpha_i \left([B_i ,Q^{AB}_{ab}]\right)^\dagger+\cdots \quad \because \alpha^*_i=\beta_i \\
=&\left(Q^{AB}_{ab}\right)^\dagger-i\beta_i[B_i,\left(Q^{AB}_{ab}\right)^\dagger]-i\alpha_i [A_i ,\left(Q^{AB}_{ab}\right)^\dagger]+\cdots \\
=&\left(Q^{AB}_{ab}\right)^\dagger-i\beta_i \sum_{a'}\left(J^{(A)}_i\right)^*_{aa'}\left(Q^{AB}_{a'b}\right)^\dagger-i\alpha_i\sum_{b'}\left(J^{(B)}_i\right)^*_{bb'}\left(Q^{AB}_{ab'}\right)^\dagger+\cdots  \\
\therefore \quad [\mathbf{A},\left(Q^{AB}_{ab}\right)^\dagger]=&+\sum_{b'}\left(\mathbf{J}^{(B)}\right)^*_{bb'}\left(Q^{AB}_{ab'}\right)^\dagger,\quad [\mathbf{B},\left(Q^{AB}_{ab}\right)]=+\sum_{a'}\left(\mathbf{J}^{(A)}\right)^*_{aa'}\left(Q^{AB}_{a'b}\right)^\dagger
\end{align*}
となる.一方$(B,A)$表現に従って変換する演算子$\bar{Q}^{BA}_{ba}$からは
\begin{align*}
[\mathbf{A},(-1)^{A-a}(-1)^{B-b}\bar{Q}^{BA}_{-b,-a}]=&-(-1)^{A-a}(-1)^{B-b}\sum_{b'}\mathbf{J}^{(B)}_{-b,b'}\bar{Q}^{BA}_{b',-a} \\
=&+(-1)^{A-a}(-1)^{B-b}\sum_{b'}(-1)^{b'+b}\left(\mathbf{J}^{(B)}\right)^*_{b,-b'}\bar{Q}^{BA}_{b',-a} \\
=&+(-1)^{A-a}(-1)^{B-b}\sum_{a'}(-1)^{-b'+b}\left(\mathbf{J}^{(B)}\right)^*_{b,b'}\bar{Q}^{BA}_{-b',-a} \\
=&+\sum_{a'}\left(\mathbf{J}^{(B)}\right)^*_{b,b'}\left[(-1)^{A-a}(-1)^{B-b'}\bar{Q}^{BA}_{-b',-a}\right] \\
[\mathbf{B},(-1)^{A-a}(-1)^{B-b}\bar{Q}^{BA}_{-b,-a}]=&-(-1)^{A-a}(-1)^{B-b}\sum_{a'}\mathbf{J}^{(A)}_{-a,a'}\bar{Q}^{BA}_{-b,a'} \\
=&+(-1)^{A-a}(-1)^{B-b}\sum_{a'}(-1)^{a'+a}\left(\mathbf{J}^{(A)}\right)^*_{a,-a'}\bar{Q}^{BA}_{-b,a'} \\
=&+(-1)^{A-a}(-1)^{B-b}\sum_{a'}(-1)^{-a'+a}\left(\mathbf{J}^{(A)}\right)^*_{a,a'}\bar{Q}^{BA}_{-b,-a'} \\
=&+\sum_{a'}\left(\mathbf{J}^{(A)}\right)^*_{a,a'}\left[(-1)^{A-a'}(-1)^{B-b}\bar{Q}^{BA}_{-b,-a'}\right]
\end{align*}
という変換性を示す.したがって相似変換によって関係
\begin{align*}
\left(Q^{AB}_{ab}\right)^\dagger=(-1)^{A-a}(-1)^{B-b}\bar{Q}^{BA}_{-b,-a}
\end{align*}
がわかる.(ここから下では演算子のエルミート共役をダガーではなくアスタリスクで示すこともある.演算子に数値行列の係数がかかっていても,全体のエルミート共役をとったときに,行列の1要素についてエルミートをとっていれば係数はただの複素共役されるだけで行列のエルミート共役はされないことに注意すること!)

\vskip\baselineskip

これから証明するハーグ・ロプザンスキー・ゾーニウスの定理によれば,まず,「(1)フェルミオン的な対称性の演算子は$(0,1/2)$表現と$(1/2,0)$表現のみに属することができる」.既にみたように,$(1/2,0)$演算子あるいは$(0,1/2)$表現演算子のエルミート共役はそれぞれ
\begin{align*}
\left(Q^{\frac{1}{2}0}_{\frac{1}{2}0}\right)^\dagger=&\bar{Q}^{0\frac{1}{2}}_{0,-\frac{1}{2}},\quad \left(Q^{\frac{1}{2}0}_{-\frac{1}{2}0}\right)^\dagger=-\bar{Q}^{0\frac{1}{2}}_{0,\frac{1}{2}} \\
\therefore \quad \left(Q^{\frac{1}{2}0}_{a 0}\right)^\dagger=&i\sum_{b}(\sigma_2)_{ab}\bar{Q}^{0\frac{1}{2}}_{0,b} \quad ,\mathrm{where} \quad \sigma_2=\left(
\begin{matrix}
0 &-i \\
i & 0
\end{matrix}
\right) \\
\left(Q^{0\frac{1}{2}}_{0\frac{1}{2}}\right)^\dagger=&\bar{Q}^{\frac{1}{2}0}_{-\frac{1}{2}0},\quad \left(Q^{0\frac{1}{2}}_{0,-\frac{1}{2}}\right)^\dagger=-\bar{Q}^{\frac{1}{2}0}_{\frac{1}{2}0} \\
\therefore \quad \left(Q^{0\frac{1}{2}}_{0a}\right)^\dagger=&i\sum_{b}(\sigma_2)_{ab}\bar{Q}^{\frac{1}{2}0}_{b0}
\end{align*}
で$(0,1/2)$演算子あるいは$(1/2,0)$演算子の線形結合である.よってフェルミオン的な対称性演算子の完全系は,$(0,1/2)$生成子$\mr{Q}_{ar}\equiv \left(\mr{Q}_r\right)^{0\frac{1}{2}}_{0a}$とその$(1/2,0)$エルミート共役$\mr{Q}_{ar}^*$に分けることが可能だ.ここで$a$は値$\pm 1/2$をとるスピノル添え字で,$r$は同じローレンツ変換性を持つ異なる2成分生成子を区別するのに使う.さらにこの定理によれば,「(2)フェルミオン的生成子は反交換関係
\begin{align*}
\left\{\mr{Q}_{ar},\mr{Q}_{bs}^*\right\}=&2\delta_{rs}\sigma^\mu_{ab}P_\mu \\
\left\{\mr{Q}_{ar},\mr{Q}_{bs} \right\}=&e_{ab}Z_{rs}
\end{align*}
を満たすように定義できる」.ここで$P_\mu$は四元運動量,$Z_{rs}=-Z_{sr}$はボゾン的な対称性生成子,$\sigma_\mu$と$e$は以下の$2\times 2$行列だ(行と列は$+1/2,-1/2$の添え字をもつ.)
\begin{align*}
\sigma_1=&\left(
\begin{matrix}
0 & 1 \\
1 & 0
\end{matrix}
\right),\quad \sigma_2=\left(
\begin{matrix}
0 & -i \\
i & 0
\end{matrix}
\right) ,\quad \sigma_3=\left(
\begin{matrix}
1 & 0 \\
0 & -1
\end{matrix}
\right), \quad \sigma_0=\left(
\begin{matrix}
1 & 0 \\
0 & 1
\end{matrix}
\right) \\
e=&\left(
\begin{matrix}
0 & 1 \\
-1 & 0
\end{matrix}
\right)
\end{align*}
($e=i\sigma_2$なのを知っておくと便利かも.)最後に,フェルミオン的演算子はエネルギーおよび運動量と交換する.
\begin{align*}
[P_\mu,\mr{Q}_{ar}]=[P_\mu,\mr{Q}_{ar}^*]=0
\end{align*}
また,$Z_{rs}$と$Z^*_{rs}$は自分自身や$\mr{Q}$らと交換する
\begin{align*}
0=&[Z_{rs},\mr{Q}_{at}]=[Z_{rs},\mr{Q}_{at}^*]=[Z_{rs},Z_{tu}]=[Z_{rs},Z_{tu}^*] \\
=&[Z_{rs}^*,\mr{Q}_{at}]=[Z_{rs}^*,\mr{Q}_{at}^*]=[Z_{rs}^*,Z_{tu}^*] \\
=&[Z_{rs},P_\mu]=[Z_{rs}^*,P_\mu]
\end{align*}
の意味で,この代数の中心電荷だ.

\vskip\baselineskip

これらの結果を証明するために,斉次ローレンツ群のある$(A,B)$既約表現に属し,したがって$-A$から$+A$までと$-B$から$B$まで間隔1おきに値をとる添え字$a$と$b$を用いて$Q^{AB}_{ab}$と表されるような,ゼロでないフェルミオン的対称性生成子を考える.エルミート共役は$(B,A)$表現に属する演算子と(25.2.6)で関係がついているので,これらの演算子の反交換子は
\begin{align*}
&\left\{Q^{AB}_{ab},\bar{Q}^{BA}_{-b',-a'}\right\}=(-1)^{A-a'}(-1)^{B-b'}\left\{Q^{AB}_{ab},Q^{AB*}_{a'b'}\right\} \\
=&\left(\sum^{A+B}_{C=|A-B|} \sum^{C}_{c=-C} C_{AB}(C,c;a,-b')\right)\left(\sum^{B+A}_{D=|B-A|}\sum^D_{d=-D}C_{AB}(D,d;b,-a')\right)X^{CD}_{cd} \\
\therefore \quad &\left\{Q^{AB}_{ab},Q^{AB*}_{a'b'}\right\} \\
=&(-1)^{A-a'}(-1)^{B-b'} \sum^{A+B}_{C=|A-B|} \sum^{C}_{c=-C}\sum^{B+A}_{D=|B-A|}\sum^D_{d=-D}C_{AB}(C,c;a,-b')C_{AB}(D,d;b,-a')X^{CD}_{cd}
\end{align*}
の形をとらなければならない.($(A,B)$表現と$(B,A)$表現の積であるから,通常のスピンの合成と同様の考えで,$A+B \geq C \geq |A-B| , B+A\geq D \geq |B-A|$の範囲をとる$(C,D)$表現が現れる.$X^{CD}_{cd}$の$c,d$は$C\geq c \geq -C,D\geq d \geq -D$の範囲となる.$(A,B)$と$(B,A)$の前者を合成した$C$によってクレブシュゴルダン係数$C_{AB}(C,c;a,-b')$が現れ,後者を合成した$D$によって$C_{AB}(D,d;b,-a')$が現れる.)クレブシュゴルダン係数のユニタリー性
\begin{align*}
\sum_{J=|A-B|}^{A+B}\sum_{M=-J}^J C_{AB}(J,M;a,b)C_{AB}(J,M;a',b')=&\delta_{aa'}\delta_{bb'} \\
\sum_{a=-A}^A \sum_{b=-B}^B C_{AB}(J,M;a,b)C_{AB}(J',M';a,b)=&\delta_{JJ'}\delta_{MM'}
\end{align*}
を用いると$X^{CD}_{cd}$について書けて
\begin{align*}
X^{CD}_{cd}=\sum^{A}_{a=-A} \sum^{B}_{b=-B}\sum^{A}_{a'=-A}\sum^B_{b'=-B} (-1)^{A-a'}(-1)^{B-b'} C_{AB}(C,c;a,-b')C_{AB}(D,d;b,-a')\left\{Q^{AB}_{ab},Q^{AB*}_{a'b'}\right\}
\end{align*}
となる.これらの演算子$X^{CD}_{cd}$すべてが必ずゼロである必要はない.もし全てがゼロの場合は$Q^{AB}_{ab}=0$であることが得られてしまうことを以下でみる.\par
しかし,クレブシュゴルダン係数$C_{AB}(j\sigma;ab)$が$j=\sigma=A+B,C_{AB}(A+B,A+B;a,b)$のときにゼロでないのは$a=A,b=B$のときだけ(量子力学のとき学んだように$\sigma=a+b$という条件を満たさなければ$C_{AB}(j\sigma;ab)=0$なのだった.$-A\leq a \leq +A,-B\leq b \leq +B$なので,これを満たすのは$a=A,b=B$しかない)で,同様に$j=-\sigma=A+B,C_{AB}(A+B,-(A+B);a,b)$のときにゼロでないのは$a=-A,b=-B$のときだけだ.そしてその場合のクレブシュゴルダン係数の値はともに1だ.\par
全ての(25.2.13)で$C=D=c=-d=A+B$ととれば,$a=A,-b'=B,b=-B,-a'=-A$の項のみが生き残り
\begin{align*}
X^{A+B,A+B}_{A+B,-A-B}=&\sum^{A}_{a=-A} \sum^{B}_{b=-B}\sum^{A}_{a'=-A}\sum^B_{b'=-B} (-1)^{A-a'}(-1)^{B-b'} \\
&\quad \times C_{AB}(A+B,A+B;a,-b')C_{AB}(A+B,-(A+B);b,-a')\left\{Q^{AB}_{ab},\bar{Q}^{AB*}_{a'b'}\right\} \\
=&(-1)^{2B}\left\{Q^{AB}_{A,-B},Q^{AB*}_{A,-B}\right\}
\end{align*}
となる.これがゼロとなるためには$Q^{AB}_{A,-B}=0$でなければならない.($A\neq 0$で$AA^\dagger+A^\dagger A=0$となるものは存在しない.)これに下降演算子$A_1-iA_2$や上昇演算子$B_1+iB_2$を作用させていけば
\begin{align*}
0=\left[A_1-iA_2,Q^{AB}_{A,-B}\right]=&-\sum_{a'}\left(J_1^{(A)}+iJ_2^{(A)}\right)_{Aa'}Q^{AB}_{a',-B} \\
=&-\sqrt{2}Q^{AB}_{A-1,-B} \quad \therefore Q^{AB}_{A-1,-B}=0 \\
0=\left[B_1+iB_2,Q^{AB}_{A,-B}\right]=&-\sum_{a'}\left(J_1^{(B)}+iJ_2^{(B)}\right)_{-B,b'}Q^{AB}_{A,b'} \\
=&-\sqrt{2}Q^{AB}_{A,-B+1} \quad \therefore Q^{AB}_{A,-B+1}=0
\end{align*}
という操作を繰り返して,全ての$Q^{AB}_{ab}$がゼロであることを得る.したがって対偶をとれば,何らかのゼロでない$(A,B)$表現のフェルミオン的生成子$Q^{AB}_{ab}\neq 0$が存在すれば,それらとそれらの共役生成子による反交換子$\left\{Q^{AB}_{ab},Q^{AB*}_{a'b'}\right\}$は,少なくとも$(A+B,A+B)$表現に属するゼロでない\uwave{ボゾン的}な対称性生成子$X^{A+B,A+B}_{cd}$を含まなければならない.(フェルミオン的な生成子二つの積から作ったので,これはボゾン的だ.)\par
さて,コールマン・マンデューラの定理より,ボゾン的な対称性生成子$X$は,$(1/2,1/2)$の並進生成子$P^\mu$と$(1,0)\oplus (0,1)$の固有ローレンツ変換の生成子$J_{\mu\nu}$,そして多分$(0,0)$の様々な内部対称性生成子$T_A$からのみ構成されるのだった.(これらの演算子の変換性は5.5節で述べた.)これらのみをフェルミオン的な生成子から作り出すためには,それは$A+B \leq 1/2$を満たす表現$(A,B)$に属する場合だけが可能だ!これらの演算子はボゾンをフェルミオンに変え,またその逆も引き起こすので,$(0,0)$表現のスカラーではありえない.(整数スピンの場に作用して半整数スピンの場になるには,場の演算子と生成子との積で半整数スピンだけスピンの合成が起きていなければならない.これはスピン0演算子にはできない.)よって残されたのは$(1/2,0)$表現と$(0,1/2)$表現だけだ!これがハーグ・ロプザンスキー・ゾーニウスの定理の主張(1)だ.\par
線形独立な$(0,1/2)$表現のフェルミオン的生成子を$\mr{Q}_{ar}\equiv \left(\mr{Q}^{0\frac{1}{2}}_{0a}\right)_r$と表示すると,反交換子$\left\{\mr{Q}_{ar},\mr{Q}_{bs}^*\right\}$は表現$(0,1/2)\times (1/2,0)=(1/2,1/2)$に属するので,唯一の$(1/2,1/2)$のボゾン的対称性生成子である4元運動量ベクトル$P_\mu$に比例しなければならない.以下でみるように,ローレンツ不変性を用いるとこの関係式の形は
\begin{align*}
\left\{\mr{Q}_{ar},\mr{Q}_{bs}^*\right\}=2N_{rs}\sigma^\mu_{ab}P_\mu
\end{align*}
でなければならない.ここで$N_{rs}$は数値行列だ.\par
このことを見るために,2.7節で述べたローレンツ群(正確にはその被覆群)と2次元ユニモジュラ複素行列$\lambda$の群$SL(2,\mathbb{C})$との同型性を使う.ローレンツ変換$\tensor{\Lambda}{^\mu_\nu}$が$(0,1/2)$のフェルミオン的生成子$\mr{Q}_{ar}$に及ぼす効果は
\begin{align*}
U^{-1}(\Lambda)\mr{Q}_{ar}U(\Lambda)=\sum_b \lambda_{ab}\mr{Q}_{br}
\end{align*}
となる.ここで$\Lambda$は(2.7.40)
\begin{align*}
\lambda \sigma_\mu \lambda^\dagger =\tensor{\Lambda}{^\nu_\mu}\sigma_\nu
\end{align*}
で定義されるローレンツ変換だ.(25.2.16)が実際に$(0,1/2)$演算子について成り立っていることを確認するためには,微小ローレンツ変換$\tensor{\Lambda}{^\mu_\nu}=\delta^\mu_\nu+\tensor{\omega}{^\mu_\nu}$(ただし$\omega_{\mu\nu}=-\omega_{\nu\mu}$)に対して
\begin{align*}
\lambda=1+\frac{1}{2}\left[\frac{1}{2}i\epsilon_{ijk}\omega_{ij}+\omega_{k0}\right]\sigma_k
\end{align*}
ととればいい.($-\lambda$も同じローレンツ変換を再現するから,ひとつの$\Lambda$に対して$\lambda$の選び方は二種類ある(2価).しかしローレンツ変換が恒等変換のとき$\omega=0$,(2.5.16)の左辺は$U=1$となるから,矛盾が起きないように$\omega\to 0$で$\lambda\to 1$となるように選ぶ必要がある.したがって選び方は一つしかない.)このとき$\lambda$は
\begin{align*}
\lambda=&\left(
\begin{matrix}
1+\left[\frac{1}{2}i\epsilon_{ij3}\omega_{ij}+\omega_{30}\right] & \frac{1}{2}\left[\frac{1}{2}i\epsilon_{ij1}\omega_{ij}+\omega_{10}\right]-\frac{i}{2}\left[\frac{1}{2}i\epsilon_{ij2}\omega_{ij}+\omega_{20}\right] \\
 \frac{1}{2}\left[\frac{1}{2}i\epsilon_{ij1}\omega_{ij}+\omega_{10}\right]+\frac{i}{2}\left[\frac{1}{2}i\epsilon_{ij2}\omega_{ij}+\omega_{20}\right] & 1-\left[\frac{1}{2}i\epsilon_{ij3}\omega_{ij}+\omega_{30}\right]
\end{matrix}
\right) \\
\det \lambda =&1+\mc{O}(\omega^2)
\end{align*}
となって,$SL(2,\mathbb{C})$の元であることがわかる.また
\begin{align*}
\lambda \sigma_\mu \lambda^\dagger=&\left(1+\frac{1}{2}\left[\frac{1}{2}i\epsilon_{ijk}\omega_{ij}+\omega_{k0}\right]\sigma_k \right)\sigma_\mu \left(1+\frac{1}{2}\left[-\frac{1}{2}i\epsilon_{ijk}\omega_{ij}+\omega_{k0}\right]\sigma_k\right) \\
=&\sigma_\mu +\frac{1}{4}i\epsilon_{ijk}\omega_{ij}(\sigma_k \sigma_\mu-\sigma_\mu \sigma_k)+\frac{1}{2}\omega_{k0}(\sigma_\mu \sigma_k+\sigma_k \sigma_\mu) \\
&+\mc{O}(\omega^2)
\end{align*}
$\mu=0$のとき$\sigma_0=I$だから第二項目がゼロになり
\begin{align*}
\lambda \sigma_0 \lambda^\dagger=&\sigma_\mu +\omega_{k0}\sigma_k+\mc{O}(\omega^2) \\
=&\sigma_\mu +\tensor{\omega}{^k_0}\sigma_k+\tensor{\omega}{^0_0}\sigma_0+\mc{O}(\omega^2) \\
=&\sigma^0+\tensor{\omega}{^\nu_0}\sigma_\nu
\end{align*}
$\mu=1,2,3$のとき$[\sigma_i,\sigma_j]=2i\epsilon_{ijk}\sigma_k,\{\sigma_i,\sigma_j\}=2\delta_{ij}I$より
\begin{align*}
\lambda \sigma_l \lambda^\dagger=&\sigma_l -\frac{1}{2}\epsilon_{ijk}\omega_{ij}\epsilon_{klm}\sigma_m+\omega_{k0}\delta_{kl} \\
=&\sigma_l -\frac{1}{2}\omega_{ij}(\delta_{il}\delta_{jm}-\delta_{im}\delta_{lj})\sigma_m+\omega_{l0} \\
=&\sigma_l -\frac{1}{2}\omega_{lj}\sigma_j- \frac{1}{2}\omega_{il}\sigma_i-\omega_{0l} \\
=&\sigma_l+\tensor{\omega}{^i_l}\sigma_i+\tensor{\omega}{^0_l}\sigma_0 \\
=&\sigma_l+\tensor{\omega}{^\nu_l}\sigma_\nu
\end{align*}
よって(25.2.17)$\lambda \sigma_\mu \lambda^\dagger =\tensor{\Lambda}{^\nu_\mu}\sigma_\nu$が確かめられた.また$J_i=\frac{1}{2}\epsilon_{ijk}J^{jk},K_{i}=J_{i0}$を用いると
\begin{align*}
U(\Lambda)=1+\frac{1}{2}i\omega_{\mu\nu}J^{\mu\nu}=1+\frac{1}{2}i\epsilon_{ijk}\omega_{ij}J_k -i\omega_{i0}K_i
\end{align*}
となるから,(25.2.16)を満たす演算子$\mr{Q}_{ar}$は
\begin{align*}
U^{-1}(\Lambda)\mr{Q}_{ar}U(\Lambda)=&\left[1-\frac{1}{2}i\epsilon_{ijk}\omega_{ij}J_k +i\omega_{i0}K_i \right]\mr{Q}_{ar}\left[1+\frac{1}{2}i\epsilon_{ijk}\omega_{ij}J_k -i\omega_{i0}K_i \right] \\
=&\mr{Q}_{ar}-\frac{1}{2}i\epsilon_{ijk}\omega_{ij}\left[J_i,\mr{Q}_{ar}\right]-i\omega_{i0}\left[K_i,\mr{Q}_{ar}\right] \\
=\sum_b \lambda_{ab}\mr{Q}_{br}=&\sum_b \left[\delta_{ab}+\frac{1}{2}\left[\frac{1}{2}i\epsilon_{ijk}\omega_{ij}+\omega_{k0}\right](\sigma_k)_{ab}\right]\mr{Q}_{br} \\
=&\mr{Q}_{ar}+\frac{1}{4}i\epsilon_{ijk}\omega_{ij}\sum_b (\sigma_k)_{ab}\mr{Q}_{br}+\omega_{k0}\sum_b (\sigma_k)_{ab}\mr{Q}_{br}
\end{align*}
$\omega_{ij},\omega_{i0}$の係数比較をすれば
\begin{align*}
\left[\mathbf{J},\mr{Q}_{ar}\right]=-\frac{1}{2}\sum_b \bm{\sigma}_{ab}\mr{Q}_{br},\quad \left[\mathbf{K},\mr{Q}_{ar}\right]=-\frac{1}{2}i\sum_b \bm{\sigma}_{ab}\mr{Q}_{br}
\end{align*}
(25.2.1)より
\begin{align*}
\left[\mathbf{B},\mr{Q}_{ar}\right]=-\frac{1}{2}\sum_b \bm{\sigma}_{ab}\mr{Q}_{br} ,\quad \left[\mathbf{A},\mr{Q}_{ar}\right]=0
\end{align*}
が得られる.つまりこれは(25.2.16)を満たす演算子$\mr{Q}_{ar}$は$(0,1/2)$表現に属していることを示している!$\sigma_\mu$は$2\times 2$行列の完全系をなす(つまり,任意の$2\times 2$行列$A$を$A=s^\mu \sigma_\mu$の形で書くことができる)から,行列成分$a,b$の$2\times 2$演算子行列である反交換子$\{\mr{Q}_{ar},\mr{Q}_{bs}^*\}$を$N^\mu_{rs}(\sigma^\mu)_{ab}$の形で書くことができる.ここで$N^\mu$は演算子行列である.(25.2.16)と(25.2.17)より,
\begin{align*}
U^{-1}(\Lambda)\{\mr{Q}_{ar},\mr{Q}_{bs}^*\}U(\Lambda)=&U^{-1}(\Lambda)\left[\mr{Q}_{ar}\mr{Q}_{bs}^*+\mr{Q}_{bs}^* \mr{Q}_{ar}\right]U(\Lambda) \\
=&U^{-1}(\Lambda)\mr{Q}_{ar}U(\Lambda)U^{-1}(\Lambda)\mr{Q}_{bs}^* U(\Lambda) \\
&+U^{-1}(\Lambda)\mr{Q}_{bs}^*U(\Lambda) U^{-1}(\Lambda)\mr{Q}_{ar}U(\Lambda) \\
=&\sum_{cd}\left(\lambda_{ac}\mr{Q}_{cr}\lambda^*_{bd}\mr{Q}^*_{ds}+\lambda^*_{bd}\mr{Q}^*_{ds}\lambda_{ac}\mr{Q}_{cr}\right) \\
=&\sum_{cd}\lambda_{ac}\left(\mr{Q}_{cr}\mr{Q}^*_{ds}+\mr{Q}^*_{ds}\mr{Q}_{cr}\right)\lambda^\dagger_{db} \\
=&\sum_{cd}\lambda_{ac}\left\{\mr{Q}_{cr},\mr{Q}^*_{ds}\right\}\lambda^\dagger_{db} \\
=&\sum_{cd}\lambda_{ac}N^\mu_{rs}(\sigma_\mu)_{cd}\lambda^\dagger_{db} \\
=&N^\mu_{rs}(\lambda \sigma_\mu \lambda^\dagger)_{ab} \\
=&N^\mu_{rs}\tensor{\Lambda}{^\nu_\mu}(\sigma_\nu)_{ab} \\
=&(\tensor{\Lambda}{^\nu_\mu}N^\mu_{rs})(\sigma_\nu)_{ab} \\
=U^{-1}(\Lambda)N^\mu_{rs}(\sigma_\nu)_{ab} U(\Lambda)=&\left[U^{-1}(\Lambda)N^\mu_{rs} U(\Lambda)\right](\sigma_\nu)_{ab}
\end{align*}
$\sigma_\mu$はそれぞれ線形独立なので$U^{-1}(\Lambda)N^\mu_{rs}\sigma_\nu U(\Lambda)=\tensor{\Lambda}{^\nu_\mu}N^\mu_{rs}$となる.よってこれらの演算子は4元ベクトルである.したがってコールマン・マンデューラ定理によって,唯一のボゾン的な対称性生成子の4元ベクトルである$P^\mu$に比例しなければならない!そこで$N^\mu_{rs}=2P^\mu N_{rs}$とおくと,(25.2.15)
\begin{align*}
\{\mr{Q}_{ar},\mr{Q}_{bs}^*\}=2N_{rs}P_\mu \sigma^\mu_{ab}
\end{align*}
が得られる.\par
次に$\mr{Q}_{ar}$に線形変換を施して,それらの反交換子を(25.2.7)の形にする.そのためには,まず行列$N_{rs}$がエルミートかつ正定値であることを確かめておく必要がある.エルミート性は(25.2.15)のエルミート共役をとれば
\begin{align*}
\left(\{\mr{Q}_{ar},\mr{Q}_{bs}^*\}\right)^\dagger =&\left(\mr{Q}_{ar}\mr{Q}_{bs}^*+\mr{Q}_{bs}^*\mr{Q}_{ar}\right)^\dagger \\
=&\mr{Q}_{bs}\mr{Q}_{ar}^*+\mr{Q}_{ar}^*\mr{Q}_{bs} \\
=&\{\mr{Q}_{bs},\mr{Q}_{ar}^*\}=2N_{sr}P_\mu \sigma^\mu_{ba} \\
=&2\left(N^{*\dagger}\right)_{rs}P_\mu \left((\sigma^\mu)^{*\dagger}\right)_{ab} \\
=&2\left(N^{*\dagger}\right)_{rs}P_\mu \left(\sigma^{\mu*}\right)_{ab} \\
=\left(2N_{rs}P_\mu \sigma^\mu_{ab}\right)^\dagger=&2N^*_{rs}P_\mu (\sigma^{\mu*})_{ab} \\
\therefore \quad N^\dagger =N
\end{align*}
(要素$a,b,r,s$の成分についてエルミートをとっているので,ただの数値行列の成分である$N_{rs},\sigma_{ab}$はエルミートではなく複素共役であることに注意)これで$N_{rs}$がエルミート行列であることがわかった.正定値であることを見るには,$\mr{Q}_{ar}(r=1,2,\cdots)$が線形独立なようにとっていることに気付けばよい.演算子$A,B$が線形独立であるとは,任意の線形結合$\alpha A+\beta B$が作用しても$(\alpha A+\beta B)\ket{\Psi}=0$とならない$\ket{\Psi}$が(少なくとも一つ)存在することをいう(通常の線形独立性は,二つのベクトルが線形結合でゼロにならないことを指すが,演算子は状態に作用させないと議論ができないため.)よって任意のゼロでない線形結合($d_{a},c_r \neq 0,(a=1/2,-1/2,r=1,2,\cdots)$)
\begin{align*}
&\mr{Q}\equiv \sum_{ar}d_a c_r \mr{Q}_{ar}=d_{\frac{1}{2}}\sum_{r} c_r \mr{Q}_{\frac{1}{2},r}+d_{-\frac{1}{2}}\sum_{r}c_{r}\mr{Q}_{-\frac{1}{2},r}
\end{align*}
に対して,$\mr{Q}$でゼロにならない状態$\ket{\Psi}$が存在しなければならない.(25.2.15)の期待値をこの状態についてとると
\begin{align*}
&\sum_{ab,rs}d_{a}d_b^* c_r c_s^* \bra{\Psi}2N_{rs}P_\mu \sigma^\mu_{ab}\ket{\Psi}=2\bra{\Psi}\sum_{ab}P_\mu d_a (\sigma^\mu)_{ab}d^*_b\ket{\Psi}\sum_{rs}c_{s}N_{rs}c^*_s  \\
=&\sum_{ab,rs}d_{a}d_b^* c_r c_s^* \bra{\Psi}\{\mr{Q}_{ar},\mr{Q}_{bs}^*\}\ket{\Psi} \\
=&\sum_{ab,rs}d_{a}d_b^* c_r c_s^* \bra{\Psi}\left(\mr{Q}_{ar}\mr{Q}_{bs}^*+\mr{Q}_{bs}^*\mr{Q}_{ar}\right)\ket{\Psi} \\
=&\bra{\Psi}\left(\mr{Q}\mr{Q}^*+\mr{Q}^* \mr{Q}\right)\ket{\Psi}=\braket{\mr{Q}^* \Psi | Q^*\Psi}+\braket{\mr{Q} \Psi |Q\Psi} > 0 \\
=&\bra{\Psi}\left\{\mr{Q},\mr{Q}^* \right\}\ket{\Psi}
\end{align*}
となる.これにより,ゼロでない任意の$c_r$に対して二次形式$\sum_{rs}c_r N_{rs}c_s^*$がゼロでないことがわかる.よって$N_{rs}$は正定値か負定値だ.演算子$\sum_{ab}P^\mu d_a (\sigma_\mu)_{ab}d^*_b$は,$-P^\mu P_\mu \geq 0,P^0>0$を満たす物理的な状態$\ket{\Psi}$に対しては必ず正になる.実際
\begin{align*}
\sum_{ab}P^\mu d_a (\sigma_\mu)_{ab}d^*_b=P^0|d|^2+\sum_i P^i(d^\dagger \sigma_i d)^*
\end{align*}
と書ける.$P^0 >0$より第一項目は常に正である.よって第一項目と第二項目の大小関係を見てやればよい.ここでコーシー・シュワルツの不等式より
\begin{align*}
\left|\sum_i P^i(d^\dagger \sigma_i d)\right|\leq \sqrt{\sum_i (P^i)^2 } \sqrt{\sum_i (d^\dagger \sigma_i d)^2}
\end{align*}
であり,
\begin{align*}
d^\dagger \sigma_1 d=&(d_1^*, d_2^*)\left(
\begin{matrix}
0 & 1 \\
1 & 0
\end{matrix}
\right)\left(
\begin{matrix}
d_1 \\
d_2
\end{matrix}
\right)=2\mr{Re}(d_1^*d_2) \\
d^\dagger \sigma_2 d=&(d_1^*, d_2^*)\left(
\begin{matrix}
0 & -i \\
i & 0
\end{matrix}
\right)\left(
\begin{matrix}
d_1 \\
d_2
\end{matrix}
\right)=2\mr{Im}(d_1^* d_2) \\
d^\dagger \sigma_3 d=&(d_1^*, d_2^*)\left(
\begin{matrix}
1 & 0 \\
0 & -1
\end{matrix}
\right)\left(
\begin{matrix}
d_1 \\
d_2
\end{matrix}
\right)=|d_1|^2-|d_2|^2 \\
\sum_i (d^\dagger \sigma_i d)^2=&4(\mr{Re}(d_1^*d_2))^2+4(\mr{Im}(d^*_1d_2))^2+(|d_1|^2-|d_2|^2)^2 \\
=&4|d_1^*d_2|^2+(|d_1|^2-|d_2|^2)^2 \quad \because |z|^2=(\mr{Re}z)^2+(\mr{Im}z)^2 \\
=&4|d_1|^2|d_2|^2+(|d_1|^2-|d_2|^2)^2 \\
=&(|d_1|^2+|d_2|^2)^2=(|d|^2)^2 \\
\therefore \quad \sqrt{\sum_i (d^\dagger \sigma_i d)^2}=&|d|^2
\end{align*}
となる.また$-P^2\geq0$より$P^0 \geq \sqrt{\sum_i (P^i)^2}$であるから,第二項目の絶対値は必ず第一項目より小さいことがわかる.したがって$\sum_{ab}P^\mu d_a (\sigma_\mu)_{ab}d^*_b$は必ず正だ.以上より行列$N_{rs}$は正定値でなければならない.\par
行列$N_{rs}$がエルミートかつ正定値であることから,$N^{1/2}_{rs}$が存在して,その逆行列$N^{-1/2}_{rs}$も存在する.こうして,新しいフェルミオン的演算子を
\begin{align*}
\mr{Q}'_{ar} \equiv \sum_s N^{-1/2}_{rs}\mr{Q}_{as}
\end{align*}
と定めれば,この生成子についての反交換子が
\begin{align*}
\{\mr{Q}'_{ar},\mr{Q}'^*_{bs}\}=&\sum_{r's'}N^{-1/2}_{rr'}N^{-1/2*}_{ss'}\{\mr{Q}'_{ar},\mr{Q}'^*_{bs}\} \\
=&\sum_{r's'}N^{-1/2}_{rr'}N^{-1/2\dagger}_{s's}2N_{r's'}P_\mu \sigma^\mu_{ab} \\
=&2\sum_{r's'}N^{-1/2}_{rr'}N_{r's'}N^{-1/2}_{s's} P_\mu \sigma^\mu_{ab} \quad \because Nはエルミート \\
=&2\delta_{rs}P_\mu \sigma^\mu_{ab}
\end{align*}
となる.今後は,フェルミオン的生成子がこのように定義されていると仮定し,プライムを落とす.したがって(25.2.7)
\begin{align*}
\{\mr{Q}_{ar},\mr{Q}^*_{bs}\}=2\delta_{rs}\sigma^\mu_{ab} P_\mu
\end{align*}
がなりたっているものとする.\par

\vskip\baselineskip

次に,$\mr{Q}_{ar}$が運動量4元ベクトル$P_\mu$と交換することを示す必要がある.$P_\mu$のような$(1/2,1/2)$表現の演算子と,$\mr{Q}$のような$(0,1/2)$表現の演算子との交換子は$(1/2,1)$あるいは$(1/2,0)$表現の演算子だけが可能になる.しかしコールマン・マンデューラの定理より$(1/2,1)$の対称性生成子は存在しないので,$P_\mu$と$\mr{Q}$の交換子は$(1/2,0)$対称性生成子である$\mr{Q}^*$に比例することだけが可能だ.ローレンツ不変性の要求から
\begin{align*}
[\mc{M}_{ab},\mr{Q}_{cr}]=\sum_{s}e_{ac}K_{rs}\mr{Q}^*_{bs}
\end{align*}
ここで$K_{rs}$は数値行列で,$\mc{M}$は演算子の行列
\begin{align*}
\mc{M}\equiv \sigma_{\mu} P^\mu
\end{align*}
だ.(4元ベクトルを$(1/2,1/2)$表現にするために(2.7.36)でこのような形にする必要があるのだった.)行列$e_{ac}$は2個のスピン$1/2$を結合してゼロ・スピンを作るクレブシュ・ゴルダン係数(に$\sqrt{2}$をかけて簡潔にしたもの)
\begin{align*}
\ket{0,0}=&\frac{1}{\sqrt{2}}\Ket{\frac{1}{2},\frac{1}{2}}\Ket{\frac{1}{2},-\frac{1}{2}}-\frac{1}{2}\Ket{\frac{1}{2},-\frac{1}{2}}\Ket{\frac{1}{2},\frac{1}{2}} \\
=&\sum_{m_1 ,m_2}C_{\frac{1}{2}\frac{1}{2}}(0,0;m_1,m_2)\Ket{\frac{1}{2},m_1}\Ket{\frac{1}{2},m_2} \\
e_{\frac{1}{2},\frac{1}{2}}=&\sqrt{2}C_{\frac{1}{2}\frac{1}{2}}(0,0;\frac{1}{2},\frac{1}{2})=0 \\
e_{\frac{1}{2},-\frac{1}{2}}=&\sqrt{2}C_{\frac{1}{2}\frac{1}{2}}(0,0;\frac{1}{2},-\frac{1}{2})=1 \\
e_{-\frac{1}{2},\frac{1}{2}}=&\sqrt{2}C_{\frac{1}{2}\frac{1}{2}}(0,0;-\frac{1}{2},\frac{1}{2})=-1 \\
e_{-\frac{1}{2},-\frac{1}{2}}=&\sqrt{2}C_{\frac{1}{2}\frac{1}{2}}(0,0;-\frac{1}{2},-\frac{1}{2})=0 \\
\therefore \quad e=&\left(
\begin{matrix}
0 & 1 \\
-1 & 0 \\
\end{matrix}
\right)
\end{align*}
だ.これは$(1/2,1/2)$の後ろの$1/2$と$(0,1/2)$の後ろの$1/2$を合成してゼロスピン状態$(1/2,0)$を作っているため必要なものだ.この表式を用いると,まず
\begin{align*}
(\mc{M}^\dagger)_{ab}=&(\sigma_\mu)_{ab}^*P^\mu=(\sigma_\mu^T)_{ab}P^\mu \\
=&(\sigma_{\mu})_{ba}P^\mu=\mc{M}_{ba} \\
\left([\mc{M}_{ab},\mr{Q}_{cr}]\right)^\dagger =&-[\mc{M}_{ba},\mr{Q}^*_{cr}] \\
=&\left(\sum_{s}e_{ac}K_{rs}\mr{Q}^*_{bs}\right)^\dagger=\sum_{s}e_{ac}K^*_{rs}\mr{Q}_{bs}=\sum_{s}e_{ac}(K^\dagger)_{sr}\mr{Q}_{bs} \\
\therefore \quad [\mc{M}_{ab},\mr{Q}^*_{cr}]=&-\sum_{s}e_{bc}\mr{Q}_{as}(K^\dagger)_{sr}
\end{align*}
が得られるから
\begin{align*}
\left[\mc{M}_{-\frac{1}{2},-\frac{1}{2}},\left\{\mr{Q}_{\frac{1}{2}r},\mr{Q}^*_{\frac{1}{2}s} \right\} \right]=&\left[\mc{M}_{-\frac{1}{2},-\frac{1}{2}},\left(\mr{Q}_{\frac{1}{2}r}\mr{Q}^*_{\frac{1}{2}s}+\mr{Q}^*_{\frac{1}{2}s} \mr{Q}_{\frac{1}{2}r} \right) \right] \\
=&\left[\mc{M}_{-\frac{1}{2},-\frac{1}{2}},\mr{Q}_{\frac{1}{2}r} \right]\mr{Q}^*_{\frac{1}{2}s}+\mr{Q}_{\frac{1}{2}r}\left[\mc{M}_{-\frac{1}{2},-\frac{1}{2}},\mr{Q}^*_{\frac{1}{2}s} \right] \\
&+\left[\mc{M}_{-\frac{1}{2},-\frac{1}{2}},\mr{Q}^*_{\frac{1}{2}s} \right]\mr{Q}_{\frac{1}{2}r} +\mr{Q}^*_{\frac{1}{2}s} \left[\mc{M}_{-\frac{1}{2},-\frac{1}{2}},\mr{Q}_{\frac{1}{2}r} \right] \\
=&\sum_{r'}e_{-\frac{1}{2}\frac{1}{2}}K_{rr'}\mr{Q}^*_{-\frac{1}{2}r'}\mr{Q}^*_{\frac{1}{2}s}-\mr{Q}_{\frac{1}{2}r}\sum_{s'}e_{-\frac{1}{2}\frac{1}{2}}\mr{Q}_{-\frac{1}{2}s'}(K^\dagger)_{s's} \\
&-\sum_{s'}e_{-\frac{1}{2}\frac{1}{2}}\mr{Q}_{-\frac{1}{2}s'}(K^\dagger)_{s's}\mr{Q}_{\frac{1}{2}r}+\mr{Q}^*_{\frac{1}{2}s} \sum_{s}e_{-\frac{1}{2}\frac{1}{2}}K_{rr'}\mr{Q}^*_{-\frac{1}{2}r'} \\
=&-\sum_{r'}K_{rr'}\mr{Q}^*_{-\frac{1}{2}r'}\mr{Q}^*_{\frac{1}{2}s}+\mr{Q}_{\frac{1}{2}r}\sum_{s'}\mr{Q}_{-\frac{1}{2}s'}(K^\dagger)_{s's} \\
&+\sum_{s'}\mr{Q}_{-\frac{1}{2}s'}(K^\dagger)_{s's}\mr{Q}_{\frac{1}{2}r}-\mr{Q}^*_{\frac{1}{2}s} \sum_{r'}K_{rr'}\mr{Q}^*_{-\frac{1}{2}r'}
\end{align*}
さらに
\begin{align*}
&\left[\mc{M}_{-\frac{1}{2},-\frac{1}{2}}.\left[\mc{M}_{-\frac{1}{2},-\frac{1}{2}},\left\{\mr{Q}_{\frac{1}{2}r},\mr{Q}^*_{\frac{1}{2}s} \right\} \right]\right] \\
=&-\sum_{r'}K_{rr'}\left[\mc{M}_{-\frac{1}{2},-\frac{1}{2}},\mr{Q}^*_{-\frac{1}{2}r'}\right]\mr{Q}^*_{\frac{1}{2}s}-\sum_{r'}K_{rr'}\mr{Q}^*_{-\frac{1}{2}r'}\left[\mc{M}_{-\frac{1}{2},-\frac{1}{2}},\mr{Q}^*_{\frac{1}{2}s}\right] \\
&+\sum_{s'}\left[\mc{M}_{-\frac{1}{2},-\frac{1}{2}},\mr{Q}_{\frac{1}{2}r}\right] \mr{Q}_{-\frac{1}{2}s'}(K^\dagger)_{s's}+\sum_{s'}\mr{Q}_{\frac{1}{2}r} \left[\mc{M}_{-\frac{1}{2},-\frac{1}{2}},\mr{Q}_{-\frac{1}{2}s'}\right](K^\dagger)_{s's} \\
&+\sum_{s'}\left[\mc{M}_{-\frac{1}{2},-\frac{1}{2}},\mr{Q}_{-\frac{1}{2}s'}\right](K^\dagger)_{s's}\mr{Q}_{\frac{1}{2}r}+\sum_{s'}\mr{Q}_{-\frac{1}{2}s'}(K^\dagger)_{s's}\left[\mc{M}_{-\frac{1}{2},-\frac{1}{2}},\mr{Q}_{\frac{1}{2}r}\right] \\
&-\sum_{r'}\left[\mc{M}_{-\frac{1}{2},-\frac{1}{2}},\mr{Q}^*_{\frac{1}{2}s}\right]  K_{rr'}\mr{Q}^*_{-\frac{1}{2}r'}-\sum_{r'}\mr{Q}^*_{\frac{1}{2}s}  K_{rr'}\left[\mc{M}_{-\frac{1}{2},-\frac{1}{2}},\mr{Q}^*_{-\frac{1}{2}r'}\right] \\
=&\sum_{r'r''}K_{rr'}e_{-\frac{1}{2}-\frac{1}{2}}\mr{Q}_{-\frac{1}{2}r''}(K^\dagger)_{r''r'}\mr{Q}^*_{\frac{1}{2}s}+\sum_{r's'}K_{rr'}\mr{Q}^*_{-\frac{1}{2}r'} e_{-\frac{1}{2}\frac{1}{2}}\mr{Q}_{-\frac{1}{2}s'}(K^\dagger)_{s's} \\
&+\sum_{s'r'}e_{-\frac{1}{2}\frac{1}{2}}K_{rr'}\mr{Q}^*_{-\frac{1}{2}r'} \mr{Q}_{-\frac{1}{2}s'}(K^\dagger)_{s's}+\sum_{s's''}\mr{Q}_{\frac{1}{2}r} e_{-\frac{1}{2}-\frac{1}{2}}K_{s's''}\mr{Q}_{-\frac{1}{2}s''}(K^\dagger)_{s's} \\
&+\sum_{s's''}e_{-\frac{1}{2}-\frac{1}{2}}K_{s's''}\mr{Q}_{-\frac{1}{2}s''}^*(K^\dagger)_{s's}\mr{Q}_{\frac{1}{2}r}+\sum_{s'r'}\mr{Q}_{-\frac{1}{2}s'}(K^\dagger)_{s's}e_{-\frac{1}{2}\frac{1}{2}}K_{rr'}\mr{Q}^*_{-\frac{1}{2}r'} \\
&+\sum_{r's'}e_{-\frac{1}{2}\frac{1}{2}}\mr{Q}_{-\frac{1}{2}s'}(K^\dagger)_{s's}  K_{rr'}\mr{Q}^*_{-\frac{1}{2}r'}+\sum_{r'r''}\mr{Q}^*_{\frac{1}{2}s}  K_{rr'}e_{-\frac{1}{2}-\frac{1}{2}}\mr{Q}_{-\frac{1}{2}r''}(K^\dagger)_{r''r'} \\
=&-\sum_{r's'}K_{rr'}\mr{Q}^*_{-\frac{1}{2}r'} \mr{Q}_{-\frac{1}{2}s'}(K^\dagger)_{s's}-\sum_{s'r'}K_{rr'}\mr{Q}^*_{-\frac{1}{2}r'} \mr{Q}_{-\frac{1}{2}s'}(K^\dagger)_{s's} \\
&-\sum_{s'r'}\mr{Q}_{-\frac{1}{2}s'}(K^\dagger)_{s's}K_{rr'}\mr{Q}^*_{-\frac{1}{2}r'}-\sum_{r's'}\mr{Q}_{-\frac{1}{2}s'}(K^\dagger)_{s's}  K_{rr'}\mr{Q}^*_{-\frac{1}{2}r'} \\
=&-2\sum_{r's'}K_{rr'}\mr{Q}^*_{-\frac{1}{2}r'} \mr{Q}_{-\frac{1}{2}s'}(K^\dagger)_{s's}-2\sum_{s'r'}\mr{Q}_{-\frac{1}{2}s'}(K^\dagger)_{s's}K_{rr'}\mr{Q}^*_{-\frac{1}{2}r'} \\
=&-2\sum_{r's'}K_{rr'}\left\{\mr{Q}^*_{-\frac{1}{2}r'}, \mr{Q}_{-\frac{1}{2}s'}\right\}(K^\dagger)_{s's} \\
=&-4\sum_{r's'}K_{rr'}\delta_{r's'}(\sigma_\mu)_{-\frac{1}{2}-\frac{1}{2}}P^\mu(K^\dagger)_{s's} \\
=&-4\sum_{r's'}(\sigma_\mu)_{-\frac{1}{2}-\frac{1}{2}}P^\mu(KK^\dagger)_{rs} \\
=&-4(\mc{M})_{-\frac{1}{2}-\frac{1}{2}}(KK^\dagger)_{rs}
\end{align*}
が得られる.(25.2.7)を使うと,左辺$\left[\mc{M}_{-\frac{1}{2},-\frac{1}{2}}.\left[\mc{M}_{-\frac{1}{2},-\frac{1}{2}},\left\{\mr{Q}_{\frac{1}{2}r},\mr{Q}^*_{\frac{1}{2}s} \right\} \right]\right]$は多重交換子$[P_\mu,[P_\nu,P_\lambda]]$の線形結合となり,それはゼロとなる.しかし$(\mc{M})_{-1/2,-1/2}$は一般の運動量についてゼロでないので,$KK^\dagger=0$でなければならない.よって$K=0$が得られる.これと(25.2.18)から$[P_\mu,\mr{Q}_{ar}]=0$が得られる.これを複素共役すれば$[P_\mu,\mr{Q}^*_{ar}]=0$も得られる.こうして(25.2.10)の証明が完了した.

\vskip\baselineskip

これで,2個の$\mr{Q}$の反交換子について論じることができる.2個の$(0,1/2)$対称性生成子の反交換関係は,$(0,1)$と$(0,0)$の対称性生成子の線形結合でなければならない.再びコールマン・マンデューラ定理より,唯一の$(0,1)$対称性生成子は固有斉次ローレンツ変換の生成子$J_{\nu\lambda}$の線形結合だ.(生成子$B_i=J_i-iK_i$がそのまま$(0,1)$表現の演算子.実際わかりやすく$V_i=B_i$としたら
\begin{align*}
[A_i,V_j]=0,\quad [B_i,V_j]=i\epsilon_{ijk}V_k
\end{align*}
を満たす.これは$A$に対してはスピンゼロで$B$に対してはスピン1表現になっている.線形結合をもっとあらわに書けば$B_i=\frac{1}{2}\epsilon_{ijk}J^{jk}-iJ_{i0}$.)しかし$\mr{Q}$は$P_\mu$と交換することを見たので,それらの反交換子$\{\mr{Q},\mr{Q}\}$も$P_\mu$と交換しなければならない.一方(2.4.13)より$J_{\nu\lambda}$の線形結合は$P_\mu$と交換しない.このことから,$(0,0)$演算子だけが候補として残されており,それは$P_\mu$と$J_{\mu\nu}$の両方と交換する内部対称性生成子だ.よってローレンツ不変性より,$\mr{Q}$同士の反交換関係は(25.2.8)
\begin{align*}
\left\{\mr{Q}_{ar},\mr{Q}_{bs} \right\}=e_{ab}Z_{rs}
\end{align*}
の形をとる.(ここでスピンゼロ状態を作るためにクレブシュ・ゴルダン係数$e$が再び入っている)内部対称性生成子$Z_{rs}$は$r,s$について反対称だ.
\begin{align*}
Z_{rs}=-Z_{sr}
\end{align*}
なぜなら(25.2.8)の全体の表現は$r,a$を$s,b$と入れ替えたとき対称でなければならず,行列$e_{ab}$は$a,b$について反対称だからだ.
\begin{align*}
\left\{\mr{Q}_{bs},\mr{Q}_{ar} \right\}=&e_{ba}Z_{sr}=-e_{ab}Z_{sr} \\
=\left\{\mr{Q}_{ar},\mr{Q}_{bs} \right\}=&e_{ab}Z_{rs}
\end{align*}
これで(25.2.8)の証明も完了した.

\vskip\baselineskip

今や残されているのは(25.2.11),つまり$Z$が中心電荷だということを示すことだけだ.(25.2.8)と(25.2.10)からただちに
\begin{align*}
e_{ab} [P_\mu,Z_{rs}]=&[P_\mu ,\left\{\mr{Q}_{ar},\mr{Q}_{bs} \right\}]=0 \\
\therefore \quad [P_\mu,Z_{rs}]=&0
\end{align*}
がわかる.次に2個の$\mr{Q}$と1個の$\mr{Q}^*$を含む一般化されたヤコビ恒等式(25.1.5)
\begin{align*}
0=\left[\left\{\mr{Q}_{ar},\mr{Q}_{bs}\right\},\mr{Q}^*_{ct}\right]+\left[\left\{\mr{Q}_{bs},\mr{Q}^*_{ct}\right\},\mr{Q}^*_{ar}\right]+\left[\left\{\mr{Q}_{ct}^*,\mr{Q}_{ar}\right\},\mr{Q}_{bs}\right]
\end{align*}
を考える($\eta(\mr{Q})=\eta(\mr{Q}^*)=1$で,フェルミオン同士の交換子のみ反交換子になることに留意).(25.2.7)と(25.2.10)から,第二項と第三項がゼロになることがわかる.第一項目に(25.2.8)を使えば
\begin{align*}
[Z_{rs},\mr{Q}^*_{ct}]=0
\end{align*}
が得られる.最後に,1個の$Z$と1個の$\mr{Q}$と1個の$\mr{Q}^*$を含む一般化されたヤコビ恒等式(25.1.5)を考える.
\begin{align*}
0=-\left[Z_{rs},\left\{\mr{Q}_{at},\mr{Q}_{bu}^* \right\}\right]+\left\{\mr{Q}^*_{bu},\left[Z_{rs},\mr{Q}_{at}\right]\right\}-\left\{\mr{Q}_{at},\left[\mr{Q}^*_{bu},Z_{rs}\right]\right\}
\end{align*}
第一項目と第三項目はそれぞれ(25.2.21)と(25.2.22)を使えばゼロになる.よって第二項目だけが残り
\begin{align*}
\left\{\mr{Q}^*_{bu},\left[Z_{rs},\mr{Q}_{at}\right]\right\}=0
\end{align*}
となる.さて,$[Z_{rs},\mr{Q}_{at}]$は$(0,0)$と$(0,1/2)$の積で構成されているから当然$(0,1/2)$対称性生成子であり,よって$\mr{Q}$の線形結合で
\begin{align*}
[Z_{rs},\mr{Q}_{at}]=\sum_{u}M_{rstu}\mr{Q}_{au}
\end{align*}
と書ける($\mr{Q}_{ar}$の$r=1,2,\cdots $は線形独立にとったのだから,線形結合$\sum_{u}A_{u}\mr{Q}_{au}$で$(0,1/2)$の演算子が書ける.残りの添え字$rst$で係数が決まる).すると(25.2.23)から
\begin{align*}
0=&\left\{\mr{Q}^*_{bu},\left[Z_{rs},\mr{Q}_{at}\right]\right\} \\
=&\sum_{u}M_{rstu'}\left\{\mr{Q}^*_{bu},\mr{Q}_{au'}\right\}=\sum_{u}M_{rstu'}\left\{\mr{Q}_{au'},\mr{Q}^*_{bu}\right\} \\
=&\sum_{u}M_{rstu'}2\delta_{u'u}\sigma^\mu_{ab}P_\mu \\
=&2M_{rstu}\sigma^\mu_{ab}P_\mu \\
\therefore \quad &\sigma^\mu_{ab}P_\mu M_{rstu}=0
\end{align*}
が全ての$a,b,r,s,t,u$についてなりたつ.一般の運動量について$\sigma^\mu_{ab}P_\mu$はゼロでないから,$M_{rstu}=0$が結論され,したがって
\begin{align*}
[Z_{rs},\mr{Q}_{at}]=0
\end{align*}
を得る.反交換関係(25.2.8)とその共役式を交換関係(25.2.22)(25.2.25)およびそれらの共役式と合わせると
\begin{align*}
e_{ab} [Z_{tu},Z_{rs}]=&[Z_{tu} ,\left\{\mr{Q}_{ar},\mr{Q}_{bs} \right\}]=0 \\
\therefore \quad [Z_{rs},Z_{tu}]=&0 ,\quad [Z^*_{rs},Z^*_{tu}]=0 \\
e_{ab} [Z_{tu},Z_{rs}^*]=&[Z_{tu} ,\left\{\mr{Q}^*_{ar},\mr{Q}^*_{bs} \right\}]=0 \\
\therefore \quad [Z_{rs},Z^*_{tu}]=&0
\end{align*}
が得られる.これで(25.2.11)の証明が完了し,それを含めたハーグ・ロプザンスキー・ゾーニウスの定理の証明が完了した.

\vskip\baselineskip

もちろん,内部対称性生成子$Z_{rs}$が超対称代数の中心電荷だという事実は,他の\uwave{可換}と非可換な内部対称性が存在する可能性を排除するものではない.(もし$Z_{rs}$が例えば$SU(2)$の生成子ならば,他の$SU(2)$生成子とは可換でないからそれらは存在してはいけない.しかし$Z_{rs}$が$U(1)$であるならば可換代数だからそれらの可能性も残るということだ.これを以下でみる.)\par
$T_A$がボゾン的内部対称性のリー代数の完全系を張るとする(よって当然$Z_{rs}$も含む).すると$T_A$は$(0,0)$なので$[T_A,\mr{Q}_{ar}]$は$(0,1/2)$の対称性生成子であり,よって$\mr{Q}_{as}(s=1,2,\cdots)$の線形結合でなければらない.
\begin{align*}
[T_A,\mr{Q}_{ar}]=-\sum_{s}(t_A)_{rs}\mr{Q}_{as}
\end{align*}
ここで行列$t_A$は展開係数だが,2個の$T$と1個の$\mr{Q}$についてのヤコビ恒等式から
\begin{align*}
0=&\left[T_{A},\left[T_{B},\mr{Q}_{ar}\right]\right]+\left[T_{B},\left[\mr{Q}_{ar},T_A\right]\right]+\left[\mr{Q}_{ar},\left[T_{A},T_{B}\right]\right] \\
=&-\sum_{s}(t_B)_{rs}\left[T_{A},\mr{Q}_{as}\right]+\sum_s (t_A)_{rs}\left[T_{B},\mr{Q}_{as}\right]+i\sum_{C}C_{AB}^C\left[\mr{Q}_{ar},T_{C}\right] \\
=&\sum_{st}(t_B)_{rs}(t_A)_{st}\mr{Q}_{at}+\sum_{st} (t_A)_{rs}(t_B)_{st}\mr{Q}_{at}+i\sum_{t}\sum_{C}C_{AB}^C(t_C)_{rt}\mr{Q}_{at} \\
=&\sum_{t}\left[t_{B}t_{A}-t_A t_B+i\sum_C C^C_{AB} t_C\right]_{rt}\mr{Q}_{at}
\end{align*}
よって$\mr{Q}_{ar}$の線形独立性より
\begin{align*}
[t_A , t_B]=i\sum_C C^{C}_{AB}t_C
\end{align*}
が得られる.ここで係数$C^C_{AB}$は内部対称性代数
\begin{align*}
[T_A , T_B]=i\sum_C C^{C}_{AB}T_C
\end{align*}
の構造定数だ.よって$t_A$行列は内部対称性代数の表現になっていることがわかる.このとき$Z_{rs}$は,$\mr{Q},\mr{Q}^*,P_\mu,Z,Z^*$で構成される超対称代数の中心電荷(25.2.11)であるだけでなく,加えて全ての$T_A$を含むもっと大きな対称性の超対称代数の中心電荷であることが以下でわかる.\par
(25.2.27)と(25.2.8)より
\begin{align*}
[T_A,\left\{\mr{Q}_{ar},\mr{Q}_{bs}\right\}]=&e_{ab}[T_A,Z_{rs}] \\
=\left\{\left[T_{A} , \mr{Q}_{ar}\right],\mr{Q}_{bs}\right\}+\left\{ \mr{Q}_{ar},\left[T_{A},\mr{Q}_{bs}\right]\right\}=&-\sum_{r'}(t_A)_{rr'}\left\{\mr{Q}_{ar'},\mr{Q}_{bs}\right\} - \sum_{s'} (t_A)_{ss'}\left\{ \mr{Q}_{ar},\mr{Q}_{ss'}\right\} \\
=&-\sum_{r'}(t_A)_{rr'}e_{ab}Z_{r's}-\sum_{s'}(t_A)_{ss'}e_{ab}Z_{rs'} \\
\therefore \quad [T_A,Z_{rs}]=&-\sum_{r'}(t_A)_{rr'}Z_{r's}-\sum_{s'}(t_A)_{ss'}Z_{rs'}
\end{align*}
となる.よって$Z_{rs}$はボゾン的対称性代数$T_A$全体の\uwave{不変}可換部分代数を形作ることに気付ける.(対称性変換$Z_{rs}\to e^{-i\alpha_A T_A} Z_{rs} e^{i\alpha_A T_A}$で再び$Z_{rs}$の線形結合で書けるから不変部分空間であり,かつ$Z_{rs}$は自分と交換(25.2.11)するので可換部分代数である)しかしコールマン・マンデューラの定理を証明する際(p23の下から4行目),ボゾン的な内部対称性生成子は高々,コンパクトな半単純リー代数といくつかの$U(1)$代数の直和の直和と同型になっていることがわかったのだった.よって今回の$T_A$で張られているリー代数もそうだ.そのようなリー代数の唯一の不変可換部分代数は$U(1)$生成子で張られる.(半単純とは,中心がゼロであることなのだった.よって可換な不変部分代数は残りの$U(1)$代数しかない.)よって$Z_{rs}$は$U(1)$生成子でなければならない.したがって全ての$T_A$と交換しなければならない.\par
たとえ$Z$が全ての対称性生成子と交換するとしても,それは単なる数ではない.それは量子演算子であり,その値は状態毎に異なってよい.(例えば$U(1)_{em}$対称性生成子$Q$のように.)実際,$Z$は全ての超対称性生成子によって消される超対称性の真空状態では明らかに値ゼロをとらなければならないが,一般にはゼロである必要はないらしい.27.9節で$Z$を計算する方法をみるらしい.

\vskip\baselineskip

中心電荷がない場合は,超対称代数(25.2.7)(25.2.8)は
\begin{align*}
\left\{\mr{Q}_{ar},\mr{Q}_{bs}^*\right\}=2\delta_{rs}\sigma^\mu_{ab}P_{\mu} ,\quad \left\{\mr{Q}_{ar},\mr{Q}_{bs}\right\}=0
\end{align*}
となる.このとき,$V_{rs}$を$N\times N$のユニタリー行列(必ずユニモジュラ(行列式が$\pm 1$)である必要はない)として
\begin{align*}
\mr{Q}_{ar} \to \mr{Q}'_{ar}=\sum_{s}V_{rs}\mr{Q}_{as}
\end{align*}
と定義される内部対称性の群$U(N)$の下で不変だ.実際
\begin{align*}
\mr{Q}^*_{ar}\to \mr{Q}'^*_{ar}=&\sum_{s} V^*_{rs}\mr{Q}_{as}^* \\
=&\sum_{s} \mr{Q}_{as}^*V^\dagger_{sr}=\sum_{s} \mr{Q}_{as}^*V^{-1}_{sr} \\
\left\{\mr{Q}'_{ar},\mr{Q}'^*_{bs}\right\}=&\sum_{r's'}V_{rr'}\left\{\mr{Q}_{ar},\mr{Q}_{bs}^*\right\} V^{-1}_{s's} \\
=&\sum_{r's'}V_{rr'}2\delta_{r's'}\sigma^\mu_{ab}P_{\mu}V^{-1}_{s's} \\
=&2\delta_{r's'}\sigma^\mu_{ab}P_{\mu} \\
\left\{\mr{Q}'_{ar},\mr{Q}'_{bs}\right\}=&\sum_{r's'}V_{rr'}V_{ss'}\left\{\mr{Q}_{ar},\mr{Q}_{bs}\right\}=0
\end{align*}
となるからだ.これは\textbf{R対称性}と呼ばれている.この対称性は,(古典)作用の良い対称性であるかもしれないし,そうでないかもしれない.もし前者ならば,それは量子的にアノマリーによって破れているかもしれないし,自発的に破れているかもしれないし,あるいは自然界の良い対称性になっているかもしれない.

\vskip\baselineskip

$r,s$等が$N > 1$個の値をとる超対称性の代数は$\bm{N}$\textbf{次の拡張超対称性}と呼ばれる.$N=1$で$\mr{Q}$が1個だけ存在する場合は,条件$Z_{rs}=-Z_{sr}$より$Z$はゼロであり,より簡単な形の反交換関係
\begin{align*}
\left\{\mr{Q}_{a},\mr{Q}_{b}^*\right\}=&2\sigma^\mu_{ab}P_{\mu} \\
\left\{\mr{Q}_{ar},\mr{Q}_{bs}\right\}=&0
\end{align*}
が得られる.これは\textbf{単純超対称性}あるいは$N=1$超対称性と呼ばれる.この場合のR対称性変換は$U(1)$,つまり位相変換
\begin{align*}
\mr{Q}_a \to \exp(i\varphi )\mr{Q}_a
\end{align*}
となり,$\varphi$は実位相だ.

\vskip\baselineskip

ハーグ・ロプザンスキー・ゾーニウスの定理を証明する前に示したこととして,$(0,1/2)$演算子$\mr{Q}_{ar}$を用いて作った
\begin{align*}
\left(Q^{0\frac{1}{2}}_{0a}\right)^\dagger=&i\sum_{b}(\sigma_2)_{ab}\bar{Q}^{\frac{1}{2}0}_{b0} \\
\therefore \quad \bar{Q}^{\frac{1}{2}0}_{b0}=&-i\sum_b\left(\sigma_{2}\right)_{ab}\mr{Q}_{br}^*
\end{align*}
は$(1/2,0)$表現になるのだった.よって$e=i\sigma_2$を用いると,$e_{ab}\mr{Q}_{br}^*$が$(1/2,0)$として振舞ってくれる.よってそれらを組み合わせて
\begin{align*}
Q_r \equiv \left(
\begin{matrix}
e\mr{Q}^*_r \\
\mr{Q}_r
\end{matrix}
\right)
\end{align*}
あるいはもっと陽に
\begin{align*}
Q_{1r}=\mr{Q}^*_{-\frac{1}{2}r},\quad Q_{2r}=-\mr{Q}^*_{\frac{1}{2}r},\quad Q_{3r}=\mr{Q}_{\frac{1}{2}r},\quad Q_{4r}=\mr{Q}_{-\frac{1}{2}r}
\end{align*}
を満たす4成分マヨナラ・スピノル生成子$Q_{\alpha r}$を作るほうが便利となることがある.これは実際(5.5.48)と同じ形の(マイナスだけ違うが)
\begin{align*}
Q_{r}=-\beta \epsilon \gamma_5 Q_{r}^*
\end{align*}
を満たす意味でマヨラナ・スピノルだ.ここで$\beta,\epsilon,\gamma_5$は$4\times 4$行列だが,5章で見た通り$2\times 2$ブロック行列
\begin{align*}
\beta=\left(
\begin{matrix}
0 & 1 \\
1 & 0
\end{matrix}
\right), \quad \epsilon=\left(
\begin{matrix}
e & 0 \\
0 & e
\end{matrix}
\right), \quad \gamma_5 =\left(
\begin{matrix}
1 & 0 \\
0 & -1
\end{matrix}
\right)
\end{align*}
と書ける行列だ.実際左辺を計算してみると
\begin{align*}
-\beta \epsilon \gamma_5 Q_{r}^*=-\left(
\begin{matrix}
0 & -e \\
e & 0
\end{matrix}
\right)\left(
\begin{matrix}
e\mr{Q}_r \\
\mr{Q}_r^*
\end{matrix}
\right)=\left(
\begin{matrix}
e\mr{Q}^*_r \\
\mr{Q}_r
\end{matrix}
\right)=Q_{r}
\end{align*}
となっている.(マヨラナ・スピノルについては26章の補遺で概説されている.5章のときとは若干マヨラナ条件が違い,$\mc{C}=- \epsilon \gamma_5$に注意.)(25.2.34)の形は斉次ローレンツ群の4成分ディラック表示に対する通常の記法(5.4節)に従って選んである.その表示では(5.4.4)にしたがって回転とブーストの生成子は(5.4.19)と(5.4.20)の通りで($K_=J_{i0}=-J^{i0}$になっていることに注意!)
\begin{align*}
\mc{J}_i=\frac{1}{2}\left[
\begin{matrix}
\sigma_i & 0 \\
0 & \sigma_i
\end{matrix}
\right], \quad \mc{K}_i=-\frac{i}{2}\left[
\begin{matrix}
\sigma_i & 0 \\
0 & -\sigma_i
\end{matrix}
\right]
\end{align*}
と表される.(25.2.1)より
\begin{align*}
\mc{A}_i=&\frac{1}{2}(\mc{J}_i+ i\mc{K}_i)=\frac{1}{2}\left[
\begin{matrix}
\sigma_i & 0 \\
0 & 0
\end{matrix}
\right] \\
\mc{B}_i=&\frac{1}{2}(\mc{J}_i- i\mc{K}_i)=\frac{1}{2}\left[
\begin{matrix}
0 & 0 \\
0 & \sigma_i
\end{matrix}
\right]
\end{align*}
これは演算子$\mathbf{A},\mathbf{B}$がそれぞれディラックスピノルの上の2成分と下の2成分にのみに作用することを示している.
\begin{align*}
[A_i,Q_r]=\left(
\begin{matrix}
[A_i ,e\mr{Q}_r^*] \\
[A_i , \mr{Q}_r]
\end{matrix}
\right)=&\left(
\begin{matrix}
-\frac{1}{2}\sum_b (\bm{\sigma}_i)_{ab}(e\mr{Q}_r^*)_{br} \\
0
\end{matrix}
\right) \\
=&-\frac{1}{2}\left(
\begin{matrix}
\sigma_i & 0 \\
0 & 0
\end{matrix}
\right)\left(
\begin{matrix}
e\mr{Q}^*_r \\
\mr{Q}_r
\end{matrix}
\right) \\
&=-\mc{A}_iQ_{r} \\
[A_i,Q_{\alpha r}]=&-\sum_\beta (\mc{A}_i)_{\alpha\beta}Q_{\beta r} \\
[B_i ,Q_r]=\left(
\begin{matrix}
[B_i ,e\mr{Q}_r^*] \\
[B_i , \mr{Q}_r]
\end{matrix}
\right)=&\left(
\begin{matrix}
0 \\
-\frac{1}{2}\sum_b (\bm{\sigma}_i)_{ab}\mr{Q}_r
\end{matrix}
\right) \\
=&-\frac{1}{2}\left(
\begin{matrix}
0 & 0 \\
0 & \sigma_i
\end{matrix}
\right)\left(
\begin{matrix}
e\mr{Q}^*_r \\
\mr{Q}_r
\end{matrix}
\right) \\
&=-\mc{B}_iQ_{r} \\
[B_i,Q_{\alpha r}]=&-\sum_\beta (\mc{B}_i)_{\alpha\beta}Q_{\beta r}
\end{align*}
これが$(0,1/2)$演算子$\mr{Q}_{ar}$を(25.2.34)の上2成分ではなく下2成分として扱う理由だ.\par
この4成分表記では,単純超対称性の場合の基本的な反交換関係(25.2.31)(25.2.32)は
\begin{align*}
\left\{\mr{Q},\mr{Q}^*\right\}=&\mr{Q}\otimes \mr{Q}^\dagger+(\mr{Q}^*\otimes \mr{Q}^T)^T =2\sigma_\mu P^\mu \\
\left\{\mr{Q},\mr{Q}\right\}=&\mr{Q}\otimes \mr{Q}^T+(\mr{Q} \otimes \mr{Q}^T)^T=0 \\
\left\{\mr{Q}^*,\mr{Q}^*\right\}=&\mr{Q}^*\otimes \mr{Q}^\dagger+(\mr{Q}^* \otimes \mr{Q}^\dagger)^T=0
\end{align*}
となるので
\begin{align*}
Q=&\left(
\begin{matrix}
e\mr{Q}^* \\
\mr{Q}
\end{matrix}
\right),\quad \bar{Q}=Q^\dagger\beta =(\mr{Q}^T e^T,\mr{Q}^\dagger) \beta =(\mr{Q}^\dagger,\mr{Q}^T e^T ) \\
\left\{Q,\bar{Q}\right\}=&Q\otimes \bar{Q}+(\bar{Q}^T \otimes  Q^T)^T \\
=&\left(
\begin{matrix}
e\mr{Q}^* \\
\mr{Q}
\end{matrix}
\right)\otimes (\mr{Q}^\dagger,\mr{Q}^T e^T)+ \left( \left(
\begin{matrix}
\mr{Q}^* \\
e\mr{Q}
\end{matrix}
\right)\otimes \left( \mr{Q}^\dagger e^T ,\mr{Q}^T \right)\right)^T \\
=&\left(
\begin{matrix}
e\mr{Q}^*\otimes \mr{Q}^\dagger & e\mr{Q}^*\otimes \mr{Q}^Te^T \\
\mr{Q}\otimes \mr{Q}^\dagger & \mr{Q}\otimes \mr{Q}^T e^T
\end{matrix}
\right)+\left(\left(
\begin{matrix}
\mr{Q}^* \otimes \mr{Q}^\dagger e^T & \mr{Q}^* \otimes \mr{Q}^T \\
e\mr{Q} \otimes \mr{Q}^\dagger e^T & e \mr{Q} \otimes \mr{Q}^T
\end{matrix}
\right)\right)^T \\
=&\left(
\begin{matrix}
e\mr{Q}^* \otimes \mr{Q}^\dagger & e\mr{Q}^* \otimes \mr{Q}^Te^T \\
\mr{Q}\otimes \mr{Q}^\dagger & \mr{Q}\otimes \mr{Q}^T e^T
\end{matrix}
\right)+\left(
\begin{matrix}
(\mr{Q}^* \otimes \mr{Q}^\dagger e^T)^T & (e\mr{Q} \otimes \mr{Q}^\dagger e^T)^T\\
 (\mr{Q}^*\otimes \mr{Q}^T)^T & (e \mr{Q}\otimes \mr{Q}^T)^T
\end{matrix}
\right) \\
=&\left(
\begin{matrix}
e\mr{Q}^* \otimes \mr{Q}^\dagger & e\mr{Q}^* \otimes \mr{Q}^Te^T \\
\mr{Q}\otimes \mr{Q}^\dagger & \mr{Q}\otimes \mr{Q}^T e^T
\end{matrix}
\right)+\left(
\begin{matrix}
e(\mr{Q}^* \otimes \mr{Q}^\dagger )^T & e(\mr{Q} \otimes \mr{Q}^\dagger )^Te^T\\
 (\mr{Q}^*\otimes \mr{Q}^T)^T & (\mr{Q}\otimes \mr{Q}^T)^Te^T
\end{matrix}
\right) \\
=&\left(
\begin{matrix}
e\left[\mr{Q}^* \otimes \mr{Q}^\dagger +(\mr{Q}^* \otimes \mr{Q}^\dagger)^T \right] & e\left[\mr{Q}^* \otimes \mr{Q}^T +(\mr{Q} \otimes \mr{Q}^\dagger )^T \right]e^T \\
\mr{Q}\otimes \mr{Q}^\dagger+ (\mr{Q}^*\otimes \mr{Q}^T)^T & \left[\mr{Q}\otimes \mr{Q}^T+ (\mr{Q}\otimes \mr{Q}^T)^T\right] e^T
\end{matrix}
\right) \\
=&\left(
\begin{matrix}
0 & e \left[2\sigma_\mu P^\mu \right]^T e^T \\
2 \sigma_\mu P^\mu & 0
\end{matrix}
\right) \\
=&\left(
\begin{matrix}
0 & -e \left[2\sigma_\mu P^\mu \right]^T e \\
2 \sigma_\mu P^\mu & 0
\end{matrix}
\right) \\
=&\left(
\begin{matrix}
0 & -e \left[2\sigma_0 P^0 \right]^T e \\
2 \sigma_0 P^0 & 0
\end{matrix}
\right)+\left(
\begin{matrix}
0 & -e \left[2\sigma_i P^i \right]^T e \\
2 \sigma_i P^i & 0
\end{matrix}
\right) \\
=&\left(
\begin{matrix}
0 & 2\sigma_0 P^0 \\
2 \sigma_0 P^0 & 0
\end{matrix}
\right)+\left(
\begin{matrix}
0 & -2\sigma_i P^i \\
2 \sigma_i P^i & 0
\end{matrix}
\right) \\
 &\quad \because e=i\sigma_2, \quad \sigma_2 (\sigma_i)^T \sigma_2 =-\sigma_i(i=1,2,3),\quad \sigma_2 \sigma_0 \sigma_2=\sigma_0 \\
=&+2i\gamma^0 P^0-2i\gamma^i P^i \\
=&-2iP_\mu \gamma^\mu
\end{align*}
とまとめられる.ここでディラック行列が5.4節で定義した
\begin{align*}
\gamma^0 =-i\beta =-i\left(
\begin{matrix}
0 & \sigma_0 \\
\sigma_0 & 0
\end{matrix}
\right),\quad \gamma^i=-i\left(
\begin{matrix}
0 & \sigma_i \\
-\sigma_i & 0
\end{matrix}
\right)
\end{align*}
を用いた.(反交換関係を行列表示にするのが若干苦戦した.原因は第二項目に現れる転置.これは,例えば$2$成分ベクトル$f=(f_1,f_2)^T,g=(g_1,g_2)^T$のテンソル積を使うと
\begin{align*}
\{f_a, g_b\}=&f_a g_b +g_b f_a
\end{align*}
が行列表記で
\begin{align*}
\{f,g\}=\left(
\begin{matrix}
f_1 g_1 +g_1 f_1 & f_1 g_2 + g_2 f_1 \\
f_2 g_1 +g_1 f_2 & f_2 g_2 + g_2 f_2 
\end{matrix}
\right)&=\left(
\begin{matrix}
f_1 g_1 & f_1 g_2 \\
f_2 g_1 & f_2 g_2 
\end{matrix}
\right)+\left(
\begin{matrix}
g_1 f_1 &  g_2 f_1 \\
g_1 f_2 &  g_2 f_2 
\end{matrix}
\right) \\
=&\left(
\begin{matrix}
f_1 g_1 & f_1 g_2 \\
f_2 g_1 & f_2 g_2 
\end{matrix}
\right)+\left(
\begin{matrix}
g_1 f_1 &  g_1 f_2 \\
g_2 f_1 &  g_2 f_2 
\end{matrix}
\right)^T \\
=&\left(
\begin{matrix}
f_1 \\
f_2 
\end{matrix}
\right)\otimes (g_1 ,g_2 )+ \left[\left(
\begin{matrix}
g_1 \\
g_2
\end{matrix}
\right)\otimes (f_1 , f_2)\right]^T \\
=&f\otimes g^T +[g\otimes f^T]^T
\end{align*}
となることからくる.特に今回の場合は$\bar{Q}$が列ベクトルでなく行ベクトルなので,第二項目のテンソル積に使うためには転置をかけた$\bar{Q}^T$を使う必要があった.)拡張超対称性の場合には中心電荷の存在によりこの式は変更され,
\begin{align*}
\left\{\mr{Q}_r,\mr{Q}_s^*\right\}=&\mr{Q}_r\otimes \mr{Q}_s^\dagger+(\mr{Q}_s^*\otimes \mr{Q}_r^T)^T =2\sigma_\mu P^\mu\delta_{rs} \\
\left\{\mr{Q}_r,\mr{Q}_s\right\}=&\mr{Q}_r\otimes \mr{Q}_s^T+(\mr{Q}_s \otimes \mr{Q}_r^T)^T=e Z_{rs} \\
\left\{\mr{Q}_r^*,\mr{Q}_s^*\right\}=&\mr{Q}_r^*\otimes \mr{Q}_s^\dagger+(\mr{Q}_s^* \otimes \mr{Q}_r^\dagger)^T=e Z_{rs}^*=-eZ_{sr}^*
\end{align*}
を用いて(同様の計算を繰り返す)
\begin{align*}
\left\{Q_r,\bar{Q}_s\right\}=&\left(
\begin{matrix}
e\left[\mr{Q}_r^* \otimes \mr{Q}_s^\dagger +(\mr{Q}_s^* \otimes \mr{Q}_r^\dagger)^T \right] & e\left[\mr{Q}_r^* \otimes \mr{Q}_s^T +(\mr{Q}_s \otimes \mr{Q}_r^\dagger )^T \right]e^T \\
\mr{Q}_r\otimes \mr{Q}_s^\dagger+ (\mr{Q}_s^*\otimes \mr{Q}_r^T)^T & \left[\mr{Q}_r\otimes \mr{Q}_s^T+ (\mr{Q}_s\otimes \mr{Q}_r^T)^T\right] e^T
\end{matrix}
\right) \\
=&\left(
\begin{matrix}
Z_{sr}^* & -e(2\sigma_\mu P^\mu \delta_{rs})^T e \\
2\sigma_\mu P^\mu & Z_{rs}
\end{matrix}
\right) \\
=&-2i\gamma^\mu P_\mu \delta_{rs}+\left(
\begin{matrix}
Z_{sr}^* &0 \\
0 & Z_{rs}
\end{matrix}
\right) \\
=&-2i\gamma^\mu P_\mu \delta_{rs}+\left(\frac{1+\gamma_5}{2}\right)Z_{sr}^*+\left(\frac{1-\gamma_5}{2}\right)Z_{rs}
\end{align*}
となる.ここでディラック行列
\begin{align*}
\gamma_5=\left(
\begin{matrix}
1 & 0 \\
0 & -1
\end{matrix}
\right)
\end{align*}
を用いた.\par
ここで与えた4次元時空の場合の解析は,32章で一般の$D$次元時空の場合について繰り返されるらしい.

\vskip\baselineskip

共形対称代数(24.B.34)(24.B.35)の下で不変な,質量ゼロ粒子の理論では,さらに2つのボゾン的対称性生成子$D,K_\mu$が存在し,それが超対称性の反交換関係の右辺に現れることができる.これらの新しい生成子はそれぞれ$Z_{rs}$および$P_\mu$と同様にスカラー$(0,0)$と4元ベクトル$(1/2,1/2)$のローレンツ変換性
\begin{align*}
[\mathbf{A},D]=&[\mathbf{B},D]=0 \\
[A_i,(\sigma_\mu K^\mu)_{ab}]=&+\frac{1}{2}\sum_{a'}(\sigma_i)^*_{aa'}(\sigma_\mu K^\mu)_{a'b},\quad [B_i,(\sigma_\mu K^\mu)_{ab}]=-\frac{1}{2}\sum_{b'}(\sigma_i)_{bb'}(\sigma_\mu K^\mu)_{ab'}
\end{align*}
をもつ.($K^\mu$の変換性は$P^\mu$と同じなのが(25.B.34)の5つ目と(25.B.35)の2つ目を見比べればすぐわかる.$A_i$との交換関係で$-\sigma^*_\mu/2$が出てくる理由は九後ゲージ1章などを見ればわかりやすいかも.)よってp43の議論を繰り返して,フェルミオン的生成子はこの場合も,ローレンツ代数の基本スピノル表現$(1/2,0)$およびそのエルミート共役である$(0,1/2)$表現に属さなければならない.\par
すべての演算子をそれぞれディラトン生成子$D$との交換子に応じて分類しておくのが便利らしい.演算子$X$が
\begin{align*}
[X,D]=iaX
\end{align*}
のとき,次元$a$をもつという.(24.B.34)からわかる通り,ボゾン的対称性生成子$J^{\mu\nu},P^\mu,K^\mu,D$は
\begin{align*}
[J^{\mu\nu},D]=0,\quad [P^\mu,D]=iP^\mu,\quad [K^\mu,D]=-iK^\mu,\quad[D,D]=0
\end{align*}
よりそれぞれ次元$0,+1,-1,0$をもつ.(なぜ次元というかというと,$D=-ix^\mu \partial_\mu$(オイラー作用素)と書くから,それぞれ$m,n$個の$x^\rho,\partial_\sigma$の積は$[\partial_\mu,x^\nu]=\delta^{\mu}_\nu$より
\begin{align*}
\left[x^{\rho_1}\cdots x^{\rho_m}\frac{\partial}{\partial x^{\sigma_1}}\cdots \frac{\partial}{\partial x^{\sigma_n}},D\right]=&-i\left[x^{\rho_1}\cdots x^{\rho_m}\frac{\partial}{\partial x^{\sigma_1}}\cdots \frac{\partial}{\partial x^{\sigma_n}},x^\mu \frac{\partial}{\partial x^\mu}\right] \\
=&-ix^{\rho_1}\cdots x^{\rho_m}\left[\frac{\partial}{\partial x^{\sigma_1}}\cdots \frac{\partial}{\partial x^{\sigma_n}},x^\mu \frac{\partial}{\partial x^\mu}\right] \\
&-i\left[x^{\rho_1}\cdots x^{\rho_m},x^\mu \frac{\partial}{\partial x^\mu}\right]\frac{\partial}{\partial x^{\sigma_1}}\cdots \frac{\partial}{\partial x^{\sigma_n}} \\
=&-ix^{\rho_1}\cdots x^{\rho_m}\left[\frac{\partial}{\partial x^{\sigma_1}}\cdots \frac{\partial}{\partial x^{\sigma_n}},x^\mu \right] \frac{\partial}{\partial x^\mu} \\
&-ix^\mu\left[x^{\rho_1}\cdots x^{\rho_m},\frac{\partial}{\partial x^\mu}\right]\frac{\partial}{\partial x^{\sigma_1}}\cdots \frac{\partial}{\partial x^{\sigma_n}} \\
=&i(n-m)x^{\rho_1}\cdots x^{\rho_m}\frac{\partial}{\partial x^{\sigma_1}}\cdots \frac{\partial}{\partial x^{\sigma_n}}
\end{align*}
となり,次元は$n-m$であることがわかる.$x^\mu$の数だけ次元が下がり,$\partial_\mu$の数だけ次元が上がっている.運動量$P^\mu$は微分演算子で書けば$-i\partial^\mu$だったのを思い出せば,実際に次元が+1になっていることも理解できる.他も同様.)任意の内部対称性のリー群の生成子は次元0をもつ.次元$a$のフェルミオン的生成子
\begin{align*}
[\mr{Q},D]=ia\mr{Q}
\end{align*}
とその共役生成子との反交換子は
\begin{align*}
\left[\mr{Q}^*,D\right]=ia\mr{Q}^* \\
[\left\{\mr{Q},\mr{Q}^*\right\},D]=&\left\{[\mr{Q},D],\mr{Q}^*\right\}+\left\{\mr{Q},[D,\mr{Q}^*]\right\} \\
=&i(2a)\{\mr{Q},\mr{Q}^*\}
\end{align*}
次元$2a$で,かつp45の議論より正定値の$(1/2,1/2)$表現に属するボゾン的対称性生成子であることがわかる.そのようなボゾン的対称性生成子は$P_\mu$と$K_\mu$の線形結合だから,$2a=\pm 1$より唯一のフェルミオン的対称性生成子は次元$+1/2$と$-1/2$をもつことがわかる.そこで次元$+1/2$のフェルミオン的対称性生成子$\mr{Q}_{ar}$とその共役生成子$\mr{Q}_{ar}^*$を考えることができて,今までの議論を同様に進めることができて,合わせて4成分マヨラナ・スピノル$Q_{r\alpha}$を形成し
\begin{align*}
\{Q_{r\alpha},\bar{Q}_{s\beta}\}=&-2iP_\mu (\gamma^\mu)_{\alpha\beta}\delta_{rs} \\
[P_\mu,Q_{r\alpha}]=&0 \\
[D,Q_{r\alpha}]=&-\frac{1}{2}iQ_{r\alpha}
\end{align*}
を満たすようにできる.一つ目の式は(25.2.38)に対応する.ただし,ここでは中心電荷$Z_{rs}$は存在しない.なぜなら反交換子$\{\mr{Q}_{ar},\mr{Q}_{bs}\}$は次元$+1$を持つが,中心電荷は内部対称性生成子であるから次元0となり(25.2.8)のように現れることができない.二つ目の式は(25.2.10)に対応する.三つ目の式は$Q_{r\alpha}$の次元が$+1/2$であることからくる.\par
$K_\mu$と$Q_{r\alpha}$との交換子は,次元が$1/2-1=-1/2$となり,$(0,1/2)$表現に属するフェルミオン的生成子だから,(次元$+1/2$の$Q_{r\alpha}$とは別の)次元$-1/2$のフェルミオン的対称性生成子$Q^\#_{r\alpha}$の線形結合になる.ローレンツ不変性から
\begin{align*}
[K^\mu,Q_{r\alpha}]=i(\gamma^\mu)_{\alpha\beta}Q^\#_{r\beta}
\end{align*}
と書いてよい.右辺の係数因子と位相は$Q^\#_{r\alpha}$の定義に吸収させることで自由に選べる.このように選ぶことで
\begin{align*}
\left([K^\mu,Q_{r}]\right)^*=&-[K^\mu,Q^*_{r}] \\
=&+\beta \epsilon \gamma_5 [K^\mu,Q_{r}] \quad \because Q_r のマヨラナ性 \\
=&+i\beta \epsilon \gamma_5\gamma^\mu Q_{r}^\# \\
=&+i(\gamma^\mu)^* \beta \epsilon \gamma_5 Q_{r}^\# \quad \because \mc{C}=-\epsilon \gamma_5 と(5.4.40)\\
=(i\gamma^\mu Q^\#_{r})^*=&-i(\gamma^\mu)^*(Q^\#_{r})^* \\
\therefore (Q^\#_{r})^*=-\beta \epsilon \gamma_5 Q_{r}^\#
\end{align*}
と$Q^\#_{r\alpha}$がマヨラナ性を示すようにできる.$Q^\#_{r\alpha}$は次元$-1/2$だから,
\begin{align*}
[D,Q^\#_{r\alpha}]=+\frac{1}{2}iQ^\#_{r\alpha}
\end{align*}
となる.(25.2.43)と$P^\nu$との交換子をとり,$K^\mu,P^\nu$の交換子(24.B.34)を使うと
\begin{align*}
[P^\nu,[K^\mu,Q_{r\alpha}]]=&[[P^\nu,K^\mu],Q_{r\alpha}]+[K^\mu,[P^\nu,Q_{r\alpha}]] \\
=&2i\eta^{\mu\nu}[D,Q_{r\alpha}]+2i[J^{\nu\mu},Q_{r\alpha}] \quad\ \because(24.B.34) \\
=&\eta^{\mu\nu}Q_{r\alpha}+2i(\mc{J}^{\mu\nu})_{\alpha\beta}Q_{r\beta} \quad \because [J^{\nu\mu},Q_{r\alpha}]=-(\mc{J}^{\mu\nu})_{\alpha\beta}Q_{r\beta} \\
=&\eta^{\mu\nu}Q_{r\alpha}+\frac{1}{2}[\gamma^\mu,\gamma^\nu]_{\alpha\beta}Q_{r\beta} \\
=&\eta^{\mu\nu}Q_{r\alpha}+\frac{1}{2}(\gamma^\mu \gamma^\nu-\gamma^\nu \gamma^\mu)_{\alpha\beta}Q_{r\beta} \\
=&\eta^{\mu\nu}Q_{r\alpha}+\frac{1}{2}(2\gamma^\mu \gamma^\nu-2\eta^{\mu\nu}1)_{\alpha\beta}Q_{r\beta} \quad \because \{\gamma^\mu,\gamma^\nu\}=2\eta^{\mu\nu}\\
=&(\gamma^\mu)_{\alpha\beta}(\gamma^\nu)_{\beta\gamma}Q_{r\gamma} \\
=i(\gamma^\mu)_{\alpha\beta}[P^\nu,Q^\#_{r\beta}] \\
\therefore \quad [P^\nu,Q^\#_{r\alpha}]=&-i(\gamma^\nu)_{\alpha\beta}Q_{r\beta}
\end{align*}
が得られる.これと(25.2.43)を見比べると,$Q\leftrightarrow Q^\# ,P^\mu \leftrightarrow K^\mu$で対応していることがわかる.\par
さらに反交換関係(25.2.40)と$K_\mu$との交換子をとると
\begin{align*}
[K^\mu,\bar{Q}_{r}]=&[K^\mu,Q_{r}^\dagger]\beta \\
=&+i(Q^\#_r)^\dagger (\gamma^\mu)^\dagger \beta \\
=&-i(Q^\#_r)^\dagger \beta \gamma^\mu \quad \because (5.4.30)\\
=& -i\bar{Q}^\#_r \gamma^\mu
\end{align*}
を用いて
\begin{align*}
[K^\nu,\{Q_{r\alpha},\bar{Q}_{s\beta}\}]=&\{[K^\nu,Q_{r\alpha}],\bar{Q}_{s\beta}\}+\{Q_{r\alpha},[K^\nu,\bar{Q}_{s\beta}]\} \\
=&i(\gamma^\nu)_{\alpha\alpha'}\{Q^\#_{r\alpha'},\bar{Q}_{s\beta}\}-i\{Q_{r\alpha},\bar{Q}^\#_{s\beta'}\}(\gamma^\nu)_{\beta'\beta} \\
=-2i(\gamma^\mu)_{\alpha\beta}[K^\nu,P_\mu]\delta_{rs}=&-4(\gamma^\mu)_{\alpha\beta}\delta^{\nu}_{\mu}D\delta_{rs}-4(\gamma^\mu)_{\alpha\beta}\tensor{J}{_\mu^\nu}\delta_{rs} \\
=&-4(\gamma^\nu)_{\alpha\beta}D\delta_{rs}-4(\gamma^\mu)_{\alpha\beta}\tensor{J}{_\mu^\nu}\delta_{rs}
\end{align*}
両辺に$\gamma^\mu$をかけて,その後縮約することで
\begin{align*}
i(\gamma^\mu \gamma^\nu)_{\alpha\alpha'}\{Q^\#_{r\alpha'},\bar{Q}_{s\beta}\}-i(\gamma^\mu)_{\alpha\alpha'}\{Q_{r\alpha'},\bar{Q}^\#_{s\beta'}\}(\gamma^\nu)_{\beta'\beta} =&-4(\gamma^\mu\gamma^\nu)_{\alpha\beta}\delta_{rs}D-4(\gamma^\mu\gamma^\rho)_{\alpha\beta}\tensor{J}{_\rho^\nu}\delta_{rs} \\
4i\{Q^\#_{r\alpha},\bar{Q}_{s\beta}\}-i(\gamma^\mu)_{\alpha\alpha'}\{Q_{r\alpha'},\bar{Q}^\#_{s\beta'}\}(\gamma_\mu)_{\beta'\beta}=&-16\delta_{\alpha\beta}\delta_{rs}D-4(\gamma^\mu\gamma^\rho)_{\alpha\beta}\tensor{J}{_\rho_\mu}\delta_{rs} \\
=&-16\delta_{\alpha\beta}\delta_{rs}D+2(\gamma^\mu\gamma^\rho-\gamma^\rho \gamma^\mu)_{\alpha\beta}\tensor{J}{_\mu_\rho}\delta_{rs}  \\
=&-16\delta_{\alpha\beta}\delta_{rs}D+8i(\mc{J}^{\mu\rho})_{\alpha\beta}\tensor{J}{_\mu_\rho}\delta_{rs}
\end{align*}
整理して
\begin{align*}
\{Q^\#_{r\alpha},\bar{Q}_{s\beta}\}=4i\delta_{\alpha\beta}\delta_{rs}D+2(\mc{J}^{\mu\rho})_{\alpha\beta}\tensor{J}{_\mu_\rho}\delta_{rs}+\frac{1}{4}(\gamma^\mu)_{\alpha\alpha'}\{Q_{r\alpha'},\bar{Q}^\#_{s\beta'}\}(\gamma_\mu)_{\beta'\beta}
\end{align*}
という形になる.第三項目を処理する.ローレンツ不変性とディラック行列の線形独立性を用いると
\begin{align*}
\{Q_{r\alpha'},\bar{Q}^\#_{s\beta'}\}=&A_S \delta_{\alpha\beta}+A_P (\gamma_5)_{\alpha\beta} +A^\mu_{V}(\gamma_\mu)_{\alpha\beta} +A^\mu_A (\gamma_\mu \gamma_5)_{\alpha\beta} +A^{\mu\nu}_T [\gamma^\mu,\gamma^\nu]_{\alpha\beta} \\
=&A_S \delta_{\alpha\beta}+A_P (\gamma_5)_{\alpha\beta}+A^{\mu\nu}_T [\gamma^\mu,\gamma^\nu]_{\alpha\beta}
\end{align*}
と書ける.ここで$A_S,A_P,A^\mu_V,A^\mu_A ,A^{\mu\nu}_T$はそれぞれスカラー,擬スカラー,ベクトル,擬ベクトル,テンソルに対応したボゾン的対称性生成子だ.これらの生成子は次元が0である必要があり,そのようなベクトル・擬ベクトル演算子は存在しないのだったから$A_V^\mu,A_P^\mu=0$となっている.両側から$\gamma^\mu$をかけてみれば(ガンマ行列の公式を用いて)
\begin{align*}
\gamma^\mu A_{S}1 \gamma_\mu=&4A_S \\
\gamma^\mu A_P \gamma_5 \gamma_\mu =&-4A_P\gamma_5 \\
\gamma^\mu [\gamma^\rho ,\gamma^\sigma]\gamma_\mu=&\gamma^\mu \gamma^\rho \gamma^\sigma \gamma_\mu -\gamma^\mu \gamma^\sigma \gamma^\rho \gamma_\mu \\
=&4g^{\rho\sigma}-4g^{\sigma\rho}=0
\end{align*}
となり,
\begin{align*}
\frac{1}{4}(\gamma^\mu)_{\alpha\alpha'}\{Q_{r\alpha'},\bar{Q}^\#_{s\beta'}\}(\gamma_\mu)_{\beta'\beta}=&A^S_{rs}\delta_{\alpha\beta}-A^P_{rs}(\gamma_5)_{\alpha\beta} \\
=&O^S_{rs}\delta_{\alpha\beta}+O^P_{rs}(\gamma_5)_{\alpha\beta}
\end{align*}
という形に書くことができる.よって
\begin{align*}
\{Q^\#_{r\alpha},\bar{Q}_{s\beta}\}=4i\delta_{\alpha\beta}\delta_{rs}D+2(\mc{J}^{\rho\sigma})_{\alpha\beta}\tensor{J}{_\rho_\sigma}\delta_{rs}+O^S_{rs}\delta_{\alpha\beta}+O^P_{rs}(\gamma_5)_{\alpha\beta}
\end{align*}
という形になる.ここで$O^S,O^P$に関する条件を抽出するために,$Q_r,Q^\#_r$のマヨラナ性と$\mc{C}^T=\mc{C}^{-1}=-\mc{C}$を用いて
\begin{align*}
\{Q_{r\alpha},\bar{Q}^\#_{s\beta}\}=&\{Q_{r\alpha},Q^{\#*}_{s\beta'}\}\beta_{\beta'\beta} \\
=&\{Q_{r\alpha},Q^{\#}_{s\beta''}\}(-\beta \epsilon\gamma_5)^T_{\beta''\beta'}\beta_{\beta'\beta} \\
=&\{Q_{r\alpha},Q^{\#}_{s\beta'}\}(+\epsilon\gamma_5)_{\beta'\beta} \\
=&-\{Q_{r\alpha},Q^{\#}_{s\beta'}\}\mc{C}_{\beta'\beta} \\
=&-\{Q^\#_{s\beta'},Q_{r\alpha}\}\mc{C}_{\beta'\beta} \\
=&+(\{Q^\#_{s\beta'},Q_{r\alpha''}\}\mc{C}_{\alpha''\alpha'}\mc{C}_{\alpha'\alpha})\mc{C}_{\beta'\beta} \\
=&-\mc{C}_{\alpha\alpha'}(\{Q^\#_{s\beta'},Q_{r\alpha''}\}\mc{C}_{\alpha''\alpha'})\mc{C}_{\beta'\beta} \\
=&+\mc{C}_{\alpha\alpha'}\{Q^\#_{s\beta'},\bar{Q}_{r\alpha'}\}\mc{C}_{\beta'\beta}
\end{align*}
最後の等号は,最初の4行の計算を$Q_r,Q_r^\#$を入れ替えて同じ計算をして導ける.これを用いて
\begin{align*}
&\frac{1}{4}(\gamma^\mu)_{\alpha\alpha'}\{Q_{r\alpha'},\bar{Q}^\#_{s\beta'}\}(\gamma_\mu)_{\beta'\beta}=O^S_{rs}\delta_{\alpha\beta}+O^P_{rs}(\gamma_5)_{\alpha\beta} \\
=&\frac{1}{4}(\gamma^\mu \mc{C})_{\alpha\alpha'}\{Q^\#_{s\beta'},\bar{Q}_{r\alpha'}\}(\mc{C}\gamma_\mu)_{\beta'\beta} \\
=&\frac{1}{4}(\gamma^\mu \mc{C})_{\alpha\alpha'}\left[4i\delta_{\beta'\alpha'}\delta_{sr}D+2(\mc{J}^{\rho\sigma})_{\beta'\alpha'}\tensor{J}{_\rho_\sigma}\delta_{sr}+O^S_{sr}\delta_{\beta'\alpha'}+O^P_{sr}(\gamma_5)_{\beta'\alpha'} \right](\mc{C}\gamma_\mu)_{\beta'\beta} \\
=&-4i\delta_{\alpha\beta}\delta_{rs}D+2(\gamma^\mu \mc{C} (\mc{J}^{\rho\sigma})^T\mc{C}\gamma_\mu)_{\alpha\beta}J_{\rho\sigma}\delta_{rs}-O^S_{sr}\delta_{\alpha\beta}+O^P_{sr}(\gamma_5)_{\alpha\beta} \\
=&-4i\delta_{\alpha\beta}\delta_{rs}D-O^S_{sr}\delta_{\alpha\beta}+O^P_{sr}(\gamma_5)_{\alpha\beta} \\
\therefore \quad &(O^S_{rs}+O^S_{sr}+4i\delta_{rs}D)\delta_{\alpha\beta}+(O^P_{rs}-O^P_{sr})(\gamma_5)_{\alpha\beta}=0 \\
& O^S_{rs}+O^S_{sr}+4i\delta_{rs}D=0 ,\quad O^P_{rs}=O^P_{sr}
\end{align*}
ここで
\begin{align*}
\gamma^\mu \mc{C} (\mc{J}^{\rho\sigma})^T\mc{C}\gamma_\mu =&\gamma^\mu \mc{J}^{\rho\sigma} \gamma_\mu \quad \because (5.4.37) \\
=&0
\end{align*}
となることを用いた.$r=s$とおくと全ての$r$で
\begin{align*}
2O^S_{rr}+4iD=0 \\
O^S_{rr}=-2iD
\end{align*}
が得られる.よって対角成分を差し引いて,対角成分がゼロの新しい生成子を$O^S_{rs}=O'^S_{rs}-2iD\delta_{rs}$と定められる.これを用いると上の条件から$O'^S_{rs}=-O'^S_{sr}$が得られ,また
\begin{align*}
\{Q^\#_{r\alpha},\bar{Q}_{s\beta}\}=&4i\delta_{\alpha\beta}\delta_{rs}D+2(\mc{J}^{\rho\sigma})_{\alpha\beta}\tensor{J}{_\rho_\sigma}\delta_{rs}+O^S_{rs}\delta_{\alpha\beta}+O^P_{rs}(\gamma_5)_{\alpha\beta} \\
=&2i\delta_{\alpha\beta}\delta_{rs}D+2(\mc{J}^{\rho\sigma})_{\alpha\beta}\tensor{J}{_\rho_\sigma}\delta_{rs}+O'^S_{rs}\delta_{\alpha\beta}+O^P_{rs}(\gamma_5)_{\alpha\beta}
\end{align*}
が得られる.プライムを落とせば
\begin{align*}
\{Q^\#_{r\alpha},\bar{Q}_{s\beta}\}=2i\delta_{\alpha\beta}\delta_{rs}D+2(\mc{J}^{\rho\sigma})_{\alpha\beta}\tensor{J}{_\rho_\sigma}\delta_{rs}+O^S_{rs}\delta_{\alpha\beta}+O^P_{rs}(\gamma_5)_{\alpha\beta}
\end{align*}
が求まる.ここで以上より,$O^S_{rs},O^P_{rs}$は次数0のローレンツ不変な生成子であり
\begin{align*}
O^S_{rs}=-O^S_{sr},\quad O^P_{rs}=+O^P_{sr}
\end{align*}
を満たす.\par
(25.2.43)と$K_\nu$の交換子をとり,$[K_\nu,K_\mu]=0$を使うと
\begin{align*}
[K^\nu,[K^\mu,Q_{r\alpha}]]=&[[K^\nu,K^\mu],Q_{r\alpha}]+[K^\mu,[K^\nu,Q_{r\alpha}]] \\
=&[K^\mu,[K^\nu,Q_{r\alpha}]]=i(\gamma^\nu)_{\alpha\beta}[K^\mu,Q^\#_{r\beta}] \\
=i(\gamma^\mu)_{\alpha\beta}[K^\nu,Q^\#_{r\beta}]
\end{align*}
となり,$(\gamma^\mu)_{\alpha\beta}[K^\nu,Q^\#_{r\beta}]$は$\mu,\nu$について対称であることがわかる.代数計算により
\begin{align*}
4[K^\nu,Q^\#_r]=&\gamma_\mu \gamma^\mu[K^\nu,Q^\#_r] \\
=&\gamma_\mu \gamma^\nu[K^\mu,Q^\#_r] \\
=&2\delta_\mu^\nu[K^\mu,Q^\#_r]-\gamma^\nu \gamma_\mu [K^\mu,Q^\#_r] \\
=&2[K^\nu,Q^\#_r]-\gamma^\nu \gamma_\mu [K^\mu,Q^\#_r] \\
\therefore \quad  [K^\nu,Q^\#_r]=&-\frac{1}{2}\gamma^\nu \gamma_\mu [K^\mu,Q^\#_r] \\
=&+\frac{1}{4}\gamma^\nu \gamma_\mu(\gamma^\mu \gamma_\rho [K^\rho,Q^\#_r]) \\
=&+\gamma^\nu \gamma_\rho [K^\rho,Q^\#_r]=-2 [K^\nu,Q^\#_r] \\
\therefore \quad [K^\mu,Q^\#_{r\alpha}]=&0
\end{align*}
が得られる.(左辺は次元$-3/2$となっており,そのような対称性生成子が存在しないことを表している.)\par
(25.2.46)(を使うと$O^S,O^P$との交換子を考える必要があるので代わりに
\begin{align*}
\{Q^\#_{r\alpha},\bar{Q}_{s\beta}\}=4i\delta_{\alpha\beta}\delta_{rs}D+2(\mc{J}^{\mu\rho})_{\alpha\beta}\tensor{J}{_\mu_\rho}\delta_{rs}+\frac{1}{4}(\gamma^\mu)_{\alpha\alpha'}\{Q_{r\alpha'},\bar{Q}^\#_{s\beta'}\}(\gamma_\mu)_{\beta'\beta}
\end{align*}
を使う)と$K_\nu$の交換子から
\begin{align*}
&[K^\nu,\{Q^\#_{r\alpha},\bar{Q}_{s\beta}\}]=\{[K^\nu,Q^\#_{r\alpha}],\bar{Q}_{s\beta}\}+\{Q^\#_{r\alpha},[K^\nu,\bar{Q}_{s\beta}]\} =-i\{Q^\#_{r\alpha},\bar{Q}^\#_{s\beta'}\}(\gamma^\nu)_{\beta'\beta} \\
=&4i\delta_{\alpha\beta}\delta_{rs}[K^\nu,D]+2(\mc{J}_{\rho\sigma})_{\alpha\beta}\delta_{rs}[K^\nu,J^{\rho\sigma}] +i\frac{1}{4}(\gamma^\mu\gamma^\nu)_{\alpha\alpha'}\{Q^\#_{r\alpha'},\bar{Q}^\#_{s\beta'}\}(\gamma_\mu)_{\beta'\beta} \\
=&4\delta_{\alpha\beta}\delta_{rs}K^\nu-2i(\mc{J}_{\rho\sigma})_{\alpha\beta}\delta_{rs}\eta^{\nu\rho}K^\sigma+2i(\mc{J}_{\rho\sigma})_{\alpha\beta}\delta_{rs}\eta^{\nu\sigma}K^\rho +i\frac{1}{4}(\gamma^\mu\gamma^\nu)_{\alpha\alpha'}\{Q^\#_{r\alpha'},\bar{Q}^\#_{s\beta'}\}(\gamma_\mu)_{\beta'\beta} \\
=&4\delta_{\alpha\beta}\delta_{rs}K^\nu-2i(\mc{J}^{\nu\sigma})_{\alpha\beta}\delta_{rs}K_\sigma+2i(\mc{J}^{\rho\nu})_{\alpha\beta}\delta_{rs}K_\rho+i\frac{1}{4}(\gamma^\mu\gamma^\nu)_{\alpha\alpha'}\{Q^\#_{r\alpha'},\bar{Q}^\#_{s\beta'}\}(\gamma_\mu)_{\beta'\beta} \\
=&4\delta_{\alpha\beta}\delta_{rs}K^\nu-4i(\mc{J}^{\nu\sigma})_{\alpha\beta}\delta_{rs}K_\sigma +i\frac{1}{4}(\gamma^\mu\gamma^\nu)_{\alpha\alpha'}\{Q^\#_{r\alpha'},\bar{Q}^\#_{s\beta'}\}(\gamma_\mu)_{\beta'\beta} \\
=&4\delta_{\alpha\beta}\delta_{rs}K^\nu-(\gamma^\nu \gamma^\sigma-\gamma^\sigma \gamma^\nu)_{\alpha\beta}\delta_{rs}K_\sigma +i\frac{1}{4}(\gamma^\mu\gamma^\nu)_{\alpha\alpha'}\{Q^\#_{r\alpha'},\bar{Q}^\#_{s\beta'}\}(\gamma_\mu)_{\beta'\beta} \\
=&4\delta_{\alpha\beta}\delta_{rs}K^\nu-(-2\gamma^\sigma \gamma^\nu+2\eta^{\nu\sigma}1)_{\alpha\beta}\delta_{rs}K_\sigma+i\frac{1}{4}(\gamma^\mu\gamma^\nu)_{\alpha\alpha'}\{Q^\#_{r\alpha'},\bar{Q}^\#_{s\beta'}\}(\gamma_\mu)_{\beta'\beta} \\
=&2\delta_{\alpha\beta}\delta_{rs}K^\nu+2(\gamma^\mu \gamma^\nu)_{\alpha\beta} \delta_{rs}K_\mu+i\frac{1}{4}(\gamma^\mu\gamma^\nu)_{\alpha\alpha'}\{Q^\#_{r\alpha'},\bar{Q}^\#_{s\beta'}\}(\gamma_\mu)_{\beta'\beta}
\end{align*}
よって
\begin{align*}
-i\{Q^\#_{r\alpha},\bar{Q}^\#_{s\beta'}\}(\gamma^\nu)_{\beta'\beta} =&2\delta_{\alpha\beta}\delta_{rs}K^\nu+2(\gamma^\mu \gamma^\nu)_{\alpha\beta} \delta_{rs}K_\mu+i\frac{1}{4}(\gamma^\mu\gamma^\nu)_{\alpha\alpha'}\{Q^\#_{r\alpha'},\bar{Q}^\#_{s\beta'}\}(\gamma_\mu)_{\beta'\beta} \\
-4i\{Q^\#_{r\alpha},\bar{Q}^\#_{s\beta}\}=&2(\gamma^\nu)_{\alpha\beta}\delta_{rs}K_\nu +8(\gamma^\mu)_{\alpha\beta}\delta_{rs}K_\mu +\frac{i}{4}(\gamma^\mu\gamma^\nu)_{\alpha\alpha'}\{Q^\#_{r\alpha'},\bar{Q}^\#_{s\beta'}\}(\gamma_\mu\gamma_\nu)_{\beta'\beta}
\end{align*}
ここで,$\{Q^\#_{r\alpha},\bar{Q}^\#_{s\beta}\}$は次元-1のボゾン的対称性生成子だから,ローレンツ不変性より$(\gamma^\mu)_{\alpha\beta}K_\mu$に比例していなければならない.上の式を満たすためには
\begin{align*}
\{Q^\#_{r\alpha},\bar{Q}^\#_{s\beta}\}=+2i(\gamma^\mu)_{\alpha\beta}\delta_{rs}K_\mu
\end{align*}
となる.\par
最後に,(25.2.46)と$Q_{t\gamma}$との交換子から,まず右辺が
\begin{align*}
[Q_{t\gamma},\{Q^\#_{r\alpha},\bar{Q}_{s\beta}\}]=&[\{Q_{t\gamma},Q^\#_{r\alpha}\},\bar{Q}_{s\beta}]+[\{Q_{t\gamma},\bar{Q}_{s\beta}\},Q^\#_{r\alpha}] \\
=&[\{Q^\#_{r\alpha},\bar{Q}_{t\gamma'}\},\bar{Q}_{s\beta}]\mc{C}_{\gamma' \gamma}+[\{Q_{t\gamma},\bar{Q}_{s\beta}\},Q^\#_{r\alpha}] \\
=&2i\mc{C}_{\alpha\gamma}\delta_{rt}[D,\bar{Q}_{s\beta}]+2(\mc{J}^{\rho\sigma}\mc{C})_{\alpha\gamma}\delta_{rt}[J_{\rho\sigma},\bar{Q}_{s\beta}]+[O^S_{rt}, \bar{Q}_{s\beta}]\mc{C}_{\alpha\gamma}+[O^P_{rt},\bar{Q}_{s\beta}](\gamma_5 \mc{C})_{\alpha\gamma} \\
&-2i(\gamma^\mu)_{\gamma\beta}\delta_{ts}[P_\mu,Q^\#_{r\alpha}] \\
=&+\mc{C}_{\alpha\gamma}\delta_{rt}\bar{Q}_{s\beta}+2(\mc{J}^{\rho\sigma}\mc{C})_{\alpha\gamma}\delta_{rt}(\mc{J}_{\rho\sigma})_{\beta'\beta}\bar{Q}_{s\beta'}+[O^S_{rt}, \bar{Q}_{s\beta}]\mc{C}_{\alpha\gamma}+[O^P_{rt},\bar{Q}_{s\beta}](\gamma_5 \mc{C})_{\alpha\gamma} \\
&-2(\gamma^\mu)_{\gamma\beta}(\gamma_{\mu})_{\alpha\alpha'}Q_{r\alpha'}\delta_{ts}
\end{align*}
となり(ここで$[J^{\rho\sigma},\bar{Q}_{r\alpha}]=+\bar{Q}_{r\beta}(\mc{J}^{\rho\sigma})_{\beta\alpha}$を用いた),左辺は
\begin{align*}
[Q_{t\gamma},\{Q^\#_{r\alpha},\bar{Q}_{s\beta}\}]=&2i\delta_{\alpha\beta}\delta_{rs}[Q_{t\gamma},D]+2(\mc{J}^{\rho\sigma})_{\alpha\beta}\delta_{rs}[Q_{t\gamma},J_{\rho\sigma}]+[Q_{t\gamma},O^S_{rs}]\delta_{\alpha\beta}+[Q_{t\gamma},O^P_{rs}](\gamma_5)_{\alpha\beta} \\
=&-\delta_{\alpha\beta}\delta_{rs}Q_{t\gamma}+2(\mc{J}^{\rho\sigma})_{\alpha\beta}\delta_{rs}(\mc{J}_{\rho\sigma})_{\gamma\gamma'}Q_{t\gamma'}+[Q_{t\gamma},O^S_{rs}]\delta_{\alpha\beta}+[Q_{t\gamma},O^P_{rs}](\gamma_5)_{\alpha\beta}
\end{align*}
$\alpha,\gamma$を縮約して
\begin{align*}
\mathrm{tr}{\mc{C}}=&\mathrm{tr}\gamma_5=0 \\
\mathrm{tr}{(\mc{J}^{\rho\sigma}\mc{C})}=&\frac{1}{4}\mr{tr}(\gamma^\rho \gamma^\sigma \gamma^2 \gamma^0)-\frac{1}{4}\mr{tr}(\gamma^\sigma \gamma^\rho \gamma^2 \gamma^0) \\
&=2(\eta^{\rho 0}\eta^{\sigma 2}-\eta^{\rho 2}\eta^{\sigma 0}) \\
\mr{tr}(\gamma_5 \mc{C})=&i\mr{tr}(\gamma_5 \gamma^2 \gamma^0)=0 \\
(\gamma^\mu)^T \gamma_\mu=&\mc{C}\gamma^\mu \mc{C} \gamma_\mu \\
=&-\gamma^2 \gamma^0 \gamma^\mu \gamma^2 \gamma^0 \gamma_\mu \\
=&-4\gamma^2 \gamma^0 (4\eta^{20})=0 \\
(\mc{J}^{\rho\sigma})^T \mc{J}_{\rho\sigma}=&\mc{C} \mc{J}^{\rho\sigma} \mc{C} \mc{J}_{\rho\sigma} \\
=&-\gamma^2 \gamma^0 \mc{J}^{\rho\sigma} \gamma^2 \gamma^0 \mc{J}_{\rho\sigma} \\
=&\frac{1}{16}\gamma^2 \gamma^0 (\gamma^\rho \gamma^\sigma -\gamma^\sigma \gamma^\rho )\gamma^2 \gamma^0 (\gamma_\rho \gamma_\sigma -\gamma_\sigma \gamma_\rho) \\
=&\frac{1}{8}\gamma^2 \gamma^0 (\gamma^\rho \gamma^\sigma \gamma^2 \gamma^0 \gamma_\rho \gamma_\sigma - \gamma^\rho \gamma^\sigma \gamma^2 \gamma^0 \gamma_\rho \gamma_\sigma) \\
=&\frac{1}{8}\gamma^2 \gamma^0 (-2\gamma^\rho \gamma_\rho \gamma^0 \gamma^2 - 4 \gamma^\rho \eta^{20} \gamma_\rho) \\
=&+1
\end{align*}
を用いると,まず右辺が
\begin{align*}
&+\mc{C}_{\alpha\alpha}\delta_{rt}\bar{Q}_{s\beta}+2(\mc{J}^{\rho\sigma}\mc{C})_{\alpha\alpha}\delta_{rt}(\mc{J}_{\rho\sigma})_{\beta'\beta}\bar{Q}_{s\beta'}+[O^S_{rt}, \bar{Q}_{s\beta}]\mc{C}_{\alpha\alpha}+[O^P_{rt},\bar{Q}_{s\beta}](\gamma_5 \mc{C})_{\alpha\alpha} \\
=&4\eta^{\rho 0}\eta^{\sigma2}\delta_{rt}(\mc{J}_{\rho\sigma})_{\beta'\beta}\bar{Q}_{s\beta'}-4\eta^{\rho2}\eta^{\sigma0}\delta_{rt} (\mc{J}_{\rho\sigma})_{\beta'\beta}\bar{Q}_{s\beta'} \\
&-2(\gamma^\mu)_{\alpha\beta}(\gamma_{\mu})_{\alpha\alpha'}Q_{r\alpha'}\delta_{ts} \\
=&+4(\mc{J}^{02})_{\beta'\beta}\delta_{rt} \bar{Q}_{s\beta'}-4(\mc{J}^{20})_{\beta'\beta}\delta_{rt} \bar{Q}_{s\beta'} \\
=&+8(\mc{J}^{02})_{\beta'\beta}\delta_{rt} \bar{Q}_{s\beta'} \\
=&+4i(\gamma^2 \gamma^0)_{\beta'\beta}\delta_{rt}\bar{Q}_{s\beta'} \\
=&+4\delta_{rt}\bar{Q}_{s\beta'}\mc{C}_{\beta'\beta}=+4\delta_{rt} Q_{s\beta} \quad \because \bar{Q}_{r\alpha}=-Q_{r\beta}\mc{C}_{\beta\alpha}
\end{align*}
左辺が
\begin{align*}
&-\delta_{\alpha\beta}\delta_{rs}Q_{t\alpha}+2(\mc{J}^{\rho\sigma})_{\alpha\beta}\delta_{rs}(\mc{J}_{\rho\sigma})_{\alpha\gamma'}Q_{t\gamma'}+[Q_{t\alpha},O^S_{rs}]\delta_{\alpha\beta}+[Q_{t\alpha},O^P_{rs}](\gamma_5)_{\alpha\beta} \\
=&-\delta_{rs}Q_{t\beta}+2\delta_{rs} Q_{t\beta}+[Q_{t\beta},O^S_{rs}]+[(\gamma_5 Q)_{t\beta},O^P_{rs}] \\
=&+\delta_{rs}Q_{t\beta}-[O^S_{rs},Q_{t\beta}]-[O^P_{rs},(\gamma_5 Q)_{t\beta}]
\end{align*}
よって
\begin{align*}
[O^S_{rs},Q_{t\alpha}]+[O^P_{rs},(\gamma_5 Q)_{t\alpha}]=-4\delta_{rt}Q_{s\alpha}+\delta_{rs}Q_{t\alpha}
\end{align*}
となる.$\alpha,\beta$を縮約すると
\begin{align*}
&+\mc{C}_{\alpha\gamma}\delta_{rt}\bar{Q}_{s\alpha}+2(\mc{J}^{\rho\sigma}\mc{C})_{\alpha\gamma}\delta_{rt}(\mc{J}_{\rho\sigma})_{\beta'\alpha}\bar{Q}_{s\beta'}+[O^S_{rt}, \bar{Q}_{s\alpha}]\mc{C}_{\alpha\gamma}+[O^P_{rt},\bar{Q}_{s\alpha}](\gamma_5 \mc{C})_{\alpha\gamma} \\
&-2(\gamma^\mu)_{\gamma\alpha}(\gamma_{\mu})_{\alpha\alpha'}Q_{r\alpha'}\delta_{ts} \\
=&+\delta_{rt}Q_{s\gamma}+2\delta_{rt}(\mc{J}^{\rho\sigma}\mc{J}_{\rho\sigma}\mc{C})_{\beta'\gamma}\bar{Q}_{s\beta'}+[O^S_{rt}, Q_{s\gamma}]+[O^P_{rt},(\gamma_5 Q)_{s\gamma}]-8\delta_{st}Q_{r\gamma}  \\
=&\delta_{rt}Q_{s\gamma}+6\delta_{rt}Q_{s\gamma}+[O^S_{rt}, Q_{s\gamma}]+[O^P_{rt},(\gamma_5 Q)_{s\gamma}]-8\delta_{st}Q_{r\gamma} \\
=&8\delta_{rt}Q_{s\gamma}-4\delta_{rs}Q_{t\gamma}-8\delta_{st}Q_{r\gamma}
\end{align*}
左辺は
\begin{align*}
&-\delta_{\alpha\alpha}\delta_{rs}Q_{t\gamma}+2(\mc{J}^{\rho\sigma})_{\alpha\alpha}\delta_{rs}(\mc{J}_{\rho\sigma})_{\gamma\gamma'}Q_{t\gamma'}+[Q_{t\gamma},O^S_{rs}]\delta_{\alpha\alpha}+[Q_{t\gamma},O^P_{rs}](\gamma_5)_{\alpha\alpha} \\
=&-4\delta_{rs}Q_{t\gamma}+4[Q_{t\gamma},O^S_{rs}]
\end{align*}
よって
\begin{align*}
[O^S_{rs},Q_{t\alpha}]=&-2\delta_{rt}Q_{s\alpha}+2\delta_{st}Q_{r\alpha} \\
[O^P_{rs},Q_{t\alpha}]=&\delta_{rs}(\gamma_5 Q)_{t\alpha}-2\delta_{rt}(\gamma_5 Q)_{s\alpha}-2\delta_{st}(\gamma_5 Q)_{r\alpha}
\end{align*}
がわかる.($\beta,\gamma$の縮約からはこれ以上の情報は得られない.)したがって
\begin{align*}
\left[O^S_{rs},\left(\frac{1+\gamma_5}{2} Q\right)_{t\alpha}\right]=&-2\delta_{rt}\left(\frac{1+\gamma_5}{2}Q\right)_{s\alpha}+2\delta_{st}\left(\frac{1+\gamma_5}{2}Q\right)_{r\alpha} \\
=&-\left(2\delta_{rt}\delta_{su}-2\delta_{st}\delta_{ru}\right)\left(\frac{1+\gamma_5}{2}Q\right)_{u\alpha} \\
\left[O^S_{rs},\left(\frac{1-\gamma_5}{2} Q\right)_{t\alpha}\right]=&-2\delta_{rt}\left(\frac{1+\gamma_5}{2}Q\right)_{s\alpha}+2\delta_{st}\left(\frac{1+\gamma_5}{2}Q\right)_{r\alpha} \\
=&-\left(2\delta_{rt}\delta_{su}-2\delta_{st}\delta_{ru}\right)\left(\frac{1+\gamma_5}{2}Q\right)_{u\alpha}
\end{align*}
となる.この左辺の表現行列はエルミートでないから,新しく$O'^S_{rs}=iO^{S}_{rs}$で生成子を定めれば(プライムを落として)
\begin{align*}
\left[O^S_{rs},\left(\frac{1+\gamma_5}{2} Q\right)_{t\alpha}\right]=&-i\left(2\delta_{rt}\delta_{su}-2\delta_{st}\delta_{ru}\right)\left(\frac{1+\gamma_5}{2}Q\right)_{u\alpha} \\
=&-(\mc{O}^S_{rs})_{tu}\left(\frac{1+\gamma_5}{2}Q\right)_{u\alpha} \\
\left[O^S_{rs},\left(\frac{1-\gamma_5}{2} Q\right)_{t\alpha}\right]=&-(\mc{O}^S_{rs})_{tu}\left(\frac{1-\gamma_5}{2}Q\right)_{u\alpha} \\
=&-(\mc{O}^S_{rs})_{tu}^\dagger \left(\frac{1-\gamma_5}{2}Q\right)_{u\alpha} \\
(\mc{O}^S_{rs})_{tu} \equiv &i\left(2\delta_{rt}\delta_{su}-2\delta_{st}\delta_{ru}\right) 
\end{align*}
となる.このエルミートな表現行列$(\mc{O}^S_{rs})_{tu}$は,
\begin{align*}
(\mc{O}^S_{rs} \mc{O}^S_{xy})_{tv}=&(\mc{O}^S_{rs})_{tu}(\mc{O}^S_{xy})_{uv} \\
=&-4(\delta_{rt}\delta_{su}-\delta_{st}\delta_{ru})(\delta_{xu}\delta_{yv}-\delta_{yu}\delta_{xv}) \\
=&-4(\delta_{rt}\delta_{sx}\delta_{yv}+\delta_{st}\delta_{ry}\delta_{xv}-\delta_{rt}\delta_{sy}\delta_{xv}-\delta_{st}\delta_{rx}\delta_{yv}) \\
(\mc{O}^S_{xy}\mc{O}^S_{rs})_{tv}=&-4(\delta_{xt}\delta_{yr}\delta_{sv}+\delta_{yt}\delta_{xs}\delta_{rv}-\delta_{xt}\delta_{ys}\delta_{rv}-\delta_{yt}\delta_{xr}\delta_{sv})  \\
[\mc{O}^S_{rs},\mc{O}^S_{xy}]_{tv}=&(\mc{O}^S_{rs} \mc{O}^S_{xy})_{tv}-(\mc{O}^S_{xy}\mc{O}^S_{rs})_{tv} \\
=&-2(\delta_{rw}\delta_{sx}\delta_{yz}+\delta_{sw}\delta_{ry}\delta_{xz}-\delta_{rw}\delta_{sy}\delta_{xz}-\delta_{sw}\delta_{rx}\delta_{yz})(2\delta_{wt}\delta_{zv}-2\delta_{zt}\delta_{wv}) \\
=&i2(\delta_{rw}\delta_{sx}\delta_{yz}+\delta_{sw}\delta_{ry}\delta_{xz}-\delta_{rw}\delta_{sy}\delta_{xz}-\delta_{sw}\delta_{rx}\delta_{yz})(\mc{O}^S_{wz})_{tv} \\
=&iC^{wz}_{rsxy}(\mc{O}^S_{wz})_{tv}
\end{align*}
と書けるから,閉じておりリー代数となっている.(25.2.28)の逆の手順から
\begin{align*}
[O^S_{rs},O^S_{xy}]=iC^{wz}_{rsxy}O^S_{wz}
\end{align*}
もわかる.以上より
\begin{align*}
e^{-i\alpha_{rs}O^S_{rs}} (P_L Q)_{t\alpha} e^{i\alpha_{rs} O_{rs}^S}=&(P_LQ)_{t\alpha}-i\alpha_{rs}[O^S_{rs},(P_LQ)_{t\alpha}] +\cdots \\
=&(P_LQ)_{t\alpha}+i\alpha_{rs}(\mc{O}^S_{rs})_{tu} (P_LQ)_{u\alpha} \cdots \\
=&\left[e^{i\alpha_{rs}(\mc{O}^S_{rs})}\right]_{tu}(P_LQ)_{u\alpha} \\
e^{-i\alpha_{rs}O^S_{rs}} (P_R Q)_{t\alpha} e^{i\alpha_{rs} O_{rs}^S}=&(P_RQ)_{t\alpha}-i\alpha_{rs}[O^S_{rs},(P_RQ)_{t\alpha}] +\cdots \\
=&(P_RQ)_{t\alpha}+i\alpha_{rs} (\mc{O}^S_{rs})_{tu}^\dagger (P_RQ)_{u\alpha} \cdots \\
=&(P_RQ)_{t\alpha}-i\alpha_{rs} (P_RQ)_{u\alpha}(\mc{O}^S_{rs})_{ut}^\dagger \cdots \\
=&(P_RQ)_{u\alpha}\left[e^{i\alpha_{rs}(\mc{O}^S_{rs})}\right]_{ut}^\dagger
\end{align*}
となり,$O^S_{rs}$はR対称性の$U(N)$の生成子であり,左手成分$P_LQ_r$と右手成分$P_R Q_r$はそれぞれ$\mathbf{N}$表現と$\bar{\mathbf{N}}$表現として変換されることがわかる.$O^P$についても同様で
\begin{align*}
\left[O^P_{rs},\left(\frac{1+\gamma_5}{2} Q\right)_{t\alpha}\right]=&\delta_{rs}\left(\frac{1+\gamma_5}{2} Q\right)_{t\alpha}-2\delta_{rt}\left(\frac{1+\gamma_5}{2} Q\right)_{s\alpha}-2\delta_{st}\left(\frac{1+\gamma_5}{2} Q\right)_{r\alpha} \\
=&-\left(-\delta_{rs}\delta_{tu}+2\delta_{rt}\delta_{su}+2\delta_{st}\delta_{ru}\right)\left(\frac{1+\gamma_5}{2} Q\right)_{u\alpha} \\
=&-(\mc{O}^P_{rs})_{tu}\left(\frac{1+\gamma_5}{2} Q\right)_{u\alpha} \\
\left[O^P_{rs},\left(\frac{1-\gamma_5}{2} Q\right)_{t\alpha}\right]=&-\delta_{rs}\left(\frac{1-\gamma_5}{2} Q\right)_{t\alpha}+2\delta_{rt}\left(\frac{1-\gamma_5}{2} Q\right)_{s\alpha}+2\delta_{st}\left(\frac{1-\gamma_5}{2} Q\right)_{r\alpha} \\
=&\left(-\delta_{rs}\delta_{tu}+2\delta_{rt}\delta_{su}+2\delta_{st}\delta_{ru}\right)\left(\frac{1-\gamma_5}{2} Q\right)_{r\alpha} \\
=&+(\mc{O}^P_{rs})_{tu}\left(\frac{1-\gamma_5}{2} Q\right)_{u\alpha} \\
=&+(\mc{O}^P_{rs})^\dagger_{tu}\left(\frac{1-\gamma_5}{2} Q\right)_{u\alpha}
\end{align*}
となる.(今回の行列$\mc{O}^P$はエルミートなので生成子の再定義は必要ない.)ここから
\begin{align*}
e^{-i\alpha_{rs}O^P_{rs}} (P_L Q)_{t\alpha} e^{i\alpha_{rs} O_{rs}^P}=&(P_LQ)_{t\alpha}-i\alpha_{rs}[O^P_{rs},(P_LQ)_{t\alpha}] +\cdots \\
=&(P_LQ)_{t\alpha}+i\alpha_{rs}(\mc{O}^P_{rs})_{tu} (P_LQ)_{u\alpha} \cdots \\
=&\left[e^{i\alpha_{rs}(\mc{O}^S_{rs})}\right]_{tu}(P_LQ)_{u\alpha} \\
e^{-i\alpha_{rs}O^S_{rs}} (P_R Q)_{t\alpha} e^{i\alpha_{rs} O_{rs}^S}=&(P_RQ)_{t\alpha}-i\alpha_{rs}[O^S_{rs},(P_RQ)_{t\alpha}] +\cdots \\
=&(P_RQ)_{t\alpha}-i\alpha_{rs} (\mc{O}^P_{rs})_{tu}^\dagger (P_RQ)_{u\alpha} \cdots \\
=&(P_RQ)_{t\alpha}-i\alpha_{rs} (P_RQ)_{u\alpha}(\mc{O}^P_{rs})_{ut}^\dagger \cdots \\
=&(P_RQ)_{u\alpha}\left[e^{i\alpha_{rs}(\mc{O}^S_{rs})}\right]_{ut}^\dagger
\end{align*}
となって,$O^P_{rs}$はR対称性の$U(N)$の生成子であり,左手成分$P_LQ_r$と右手成分$P_R Q_r$はそれぞれ$\mathbf{N}$表現と$\bar{\mathbf{N}}$表現として変換されることがわかる.(もっと言えば$O^S_{rs}$はトレースレスだからユニモジュラであり,$SU(N)$の生成子だ.$O^P_{rs}$による変換を見ると$\delta_{rs}$に比例する$U(1)$の項が存在するから,こちらは$U(N)$の生成子だ.)\par
$P_\mu,K_\mu,D$は$U(N)$不変であることがわかるらしい.面倒だから示さないけど.これらの生成子同士およびほかの生成子との$U(N)$交換関係
\begin{align*}
&\left[P^\mu , D \right]=iP^\mu ,\quad \left[K^\mu, D\right]=-iK^\mu \\
&\left[P^\mu, K^\nu \right]=2i\eta^{\mu\nu}D+2i J^{\mu\nu} ,\quad \left[K^\mu ,K^\nu\right]=0 \\
&\left[ J^{\rho\sigma},K^\mu \right]=i\eta^{\mu\rho}K^\sigma -i\eta^{\mu\sigma}K^\rho ,\quad \left[J^{\rho\sigma},D\right]=0 \\
&\left[J^{\mu\nu},J^{\rho\sigma}\right]=-i\eta^{\nu\rho}J^{\mu\sigma}+i\eta^{\mu\rho}J^{\nu\sigma}+i\eta^{\sigma\mu}J^{\rho\nu}-i\eta^{\sigma\nu}J^{\rho\mu} \\
&\left[P^\mu,J^{\rho\sigma}\right]=-i\eta^{\nu\rho}P^\sigma +i \eta^{\nu\sigma}P^\rho \\
&\left[P^\mu,P^\nu\right]=0 \\
&\{Q_{r\alpha},\bar{Q}_{s\beta}\}=-2iP_\mu (\gamma^\mu)_{\alpha\beta}\delta_{rs} \\
&[P_\mu,Q_{r\alpha}]=0 \\
&[D,Q_{r\alpha}]=-\frac{1}{2}iQ_{r\alpha} \\
&[K^\mu,Q_{r\alpha}]=i(\gamma^\mu)_{\alpha\beta}Q^\#_{r\beta} \\
&[D,Q^\#_{r\alpha}]=+\frac{1}{2}iQ^\#_{r\alpha} \\
&[P^\nu,Q^\#_{r\alpha}]=-i(\gamma^\nu)_{\alpha\beta}Q_{r\beta} \\
&\{Q^\#_{r\alpha},\bar{Q}_{s\beta}\}=2i\delta_{\alpha\beta}\delta_{rs}D+2(\mc{J}^{\rho\sigma})_{\alpha\beta}\tensor{J}{_\rho_\sigma}\delta_{rs}-iO^S_{rs}\delta_{\alpha\beta}+O^P_{rs}(\gamma_5)_{\alpha\beta} \\
&[K^\mu,Q^\#_{r\alpha}]=0 \\
&[P_\mu,O^S_{rs}]=[P_\mu,O^P_{rs}]=[J_{\mu\nu},O^S_{rs}]=[J_{\mu\nu},O^P_{rs}] \\
&=[K_\mu,,O^S_{rs}] =[K_\mu,O^P_{rs}]=[D,O^S_{rs}]=[D,O^P_{rs}]=0 \\
&[O^S_{rs},Q_{t\alpha}]=-2\delta_{rt}Q_{s\alpha}+2\delta_{st}Q_{r\alpha} \\
&[O^P_{rs},Q_{t\alpha}]=\delta_{rs}(\gamma_5 Q)_{t\alpha}-2\delta_{rt}(\gamma_5 Q)_{s\alpha}-2\delta_{st}(\gamma_5 Q)_{r\alpha}
\end{align*}
は全体で\textbf{超共形代数}を構成する.この代数と,通常の単純超対称性あるいは$N$次拡張超対称性との大きな違いの一つは,$U(N)$対称性が「作用の対称性であってもなくても構わない単なる外部自己同型」でなく,それが超共形代数の一部であり,したがって超対称性と共形不変性をもつ任意の作用にはこの$U(N)$対称性もなければならないというところだ.

\newpage

\subsection{超対称性生成子の空間反転性}
パリティ保存則を満たす理論では,フェルミオン的生成子$\mr{Q}_{ar}$にパリティ演算子$\mathsf{P}$を作用させた結果$\mathsf{P}^{-1}\mr{Q}_{ar}\mathsf{P}$は再びフェルミオン的対称性生成子でなければならない.$J_i ,K_i$は空間反転の元でそれぞれ偶(2.6.7)と奇(2.6.8)なのだったから,(25.2.1)から$A_i$にパリティ演算子を作用させると
\begin{align*}
\mathsf{P}^{-1}A_i \mathsf{P}=B_i
\end{align*}
になることがわかる.(25.2.3)に従って,$(0,1/2)$演算子としての$\mr{Q}_{ar}$は
\begin{align*}
[B_i,\mr{Q}_{ar}]=-\frac{1}{2}\sum_b \Bigl(\sigma_ i\Bigr)_{ab} \mr{Q}_{br} ,\quad [A_i ,\mr{Q}_{ar}]=0
\end{align*}
と変換される.パリティ演算子を両辺に作用させると
\begin{align*}
&\mathsf{P}^{-1}[B_i,\mr{Q}_{ar}]\mathsf{P}=-\frac{1}{2}\sum_b \Bigl(\sigma_ i\Bigr)_{ab} \mathsf{P}^{-1}\mr{Q}_{br}\mathsf{P} \\
=&[\mathsf{P}^{-1}\mr{Q}_{ar}\mathsf{P},\mathsf{P}^{-1}\mr{Q}_{ar}\mathsf{P}] \\
=&[A_i,\mathsf{P}^{-1}\mr{Q}_{ar}\mathsf{P}] \\
\therefore \quad & [A_i,\mathsf{P}^{-1}\mr{Q}_{ar}\mathsf{P}]=-\frac{1}{2}\sum_b \Bigl(\sigma_ i\Bigr)_{ab} \mathsf{P}^{-1}\mr{Q}_{br}\mathsf{P}
\end{align*}
と
\begin{align*}
& \mathsf{P}^{-1}[A_i,\mr{Q}_{ar}]\mathsf{P}=0 \\
\therefore \quad &[B_i,\mathsf{P}^{-1}\mr{Q}_{ar}\mathsf{P}]=0
\end{align*}
が得られる.よって$\mathsf{P}^{-1}\mr{Q}_{ar}\mathsf{P}$は$(1/2,0)$表現の対称性生成子として変換されることがわかり,そのような生成子は$e_{ab}\mr{Q}_{br}^*$の線形結合でなければならないのだった.よってローレンツ不変性よりこの関係は
\begin{align*}
\mathsf{P}^{-1}\mr{Q}_{ar}\mathsf{P}=\sum_{bs}\mc{P}_{rs}e_{ab}\mr{Q}_{bs}^*
\end{align*}
の形をとる必要がある.ここで$\mc{P}$は数値行列.\par
基本的な交換関係(25.2.7)を使えば,行列$\mc{P}$の性質についていくつかわかる.(25.3.4)とその共役式から
\begin{align*}
\mathsf{P}^{-1}\{\mr{Q}_{ar},\mr{Q}_{bs}^*\}\mathsf{P}=&\{\mathsf{P}^{-1}\mr{Q}_{ar}\mathsf{P},\mathsf{P}^{-1}\mr{Q}_{bs}^*\mathsf{P}\} \\
=&\sum_{cdtu}\mc{P}_{rt}e_{ac}\mc{P}_{su}^* e_{bd}\{\mr{Q}_{ct}^*,\mr{Q}_{du}\}
\end{align*}
さらに(25.2.7)を代入すると
\begin{align*}
\delta_{rs}\mathsf{P} \sigma^\mu_{ab}P_\mu \mathsf{P}=&\sum_{cdtu}\mc{P}_{rt}e_{ac}\mc{P}_{su}^* e_{bd} \delta_{tu} \sigma^\mu_{dc} P_\mu \\
=&(\mc{P}\mc{P}^\dagger)_{rs} (e (\sigma^\mu)^T e^{-1})_{ab}P_\mu
\end{align*}
となる.ここで$e\sigma^T_i e^{-1} =-\sigma_i$と$e \sigma^T_0 e^{-1}=+\sigma_0$であり$\mathsf{P}^{-1} P_i \mathsf{P}=-P_i$と$\mathsf{P}^{-1} P^0 \mathsf{P}=+P^0$より,$\mc{P}$がユニタリー
\begin{align*}
\mc{P}\mc{P}^\dagger=1
\end{align*}
であることがわかる.\par
行列$\mc{P}$にはある程度の任意性がある.(25.3.2)と(25.2.7)を満たす任意のフェルミオン的生成子$\mr{Q}_{ar}$の組についてユニタリー変換
\begin{align*}
\mr{Q}_{ar}'=\sum_{s}\mc{U}_{rs}\mr{Q}_{as} ,\quad \mc{U}^{\dagger}=\mc{U}^{-1}
\end{align*}
は再び(25.3.2)と(25.2.7)を満たし,
\begin{align*}
\{\mr{Q}'_{ar},\mr{Q}'^*_{bs} \}=&\sum_{r's'}\mc{U}_{rr'}\mc{U}_{ss'}^*\{\mr{Q}_{ar'},\mr{Q}_{bs'}^*\} \\
=&\sum_{r's'}\mc{U}_{rr'}\mc{U}_{s's}^\dagger\delta_{r's'} \sigma^\mu_{ab}P_\mu \\
=&(\mc{U}\mc{U}^\dagger)_{rs}\sigma^\mu_{ab}P_\mu=\delta_{rs}\sigma^\mu_{ab}P_\mu \\
[B_i,\mr{Q}'_{ar}]=&\sum_{s}\mc{U}_{rs}[B_i,\mr{Q}_{as}] \\
=&-\frac{1}{2}\sum_{sb}\Bigl(\sigma_i \Bigr)_{ab}\mc{U}_{rs} \mr{Q}_{bs} \\
=&-\frac{1}{2}\sum_{b}\Bigl(\sigma_i \Bigr)_{ab}\mr{Q}'_{br} \\
[A_i,\mr{Q}'_{ar}]=&0
\end{align*}
この時パリティ変換則(25.3.4)は
\begin{align*}
\mathsf{P}^{-1}\mr{Q}_{ar}'\mathsf{P}=\sum_{bs}\mc{P}'_{rs}e_{ab} \mr{Q}'^*_{bs} ,\quad \mc{P}'=\mc{U}\mc{P}\mc{U}^{-1*}=\mc{U}\mc{P}\mc{U}^T
\end{align*}
となるからだ.\par
単純超対称性の場合には,$\mc{P}$は単に$1\times 1$の位相因子で,(25.3.4)は
\begin{align*}
\mathsf{P}^{-1} \mr{Q}_a \mathsf{P}=\mc{P} \sum_{b} e_{ab} \mr{Q}_{b}^*
\end{align*}
となる.これとその共役式を組み合わせると
\begin{align*}
\mathsf{P}^{-2} \mr{Q}_a \mathsf{P}^2 =&\mc{P} \sum_{b} e_{ab} \mathsf{P}^{-1}\mr{Q}_{b}^*\mathsf{P} \\
=&|\mc{P}|^2 \sum_{bc} e_{ab} e_{bc}\mr{Q}_{c} \\
=&-\mr{Q}_a
\end{align*}
が得られる.このことから,もし粒子の超対称多重項のボゾンが実数の固有パリティを持てば,このボゾン状態に$\mr{Q}_a$を作用させて得られるフェルミオンは\uwave{虚数}の固有パリティを持つという結果が得られる.
\begin{align*}
\mathsf{P} \ket{\psi}=+\ket{\psi} \\
\mathsf{P}^2 \mr{Q}_a \ket{\psi}=- \ket{\psi}
\end{align*}
単純超対称性の場合に$\mc{U},\mc{P}$は単に位相因子だから,$\mc{U}$を適切に選べば(25.3.8)から位相因子$\mc{P}'$は任意の値に選ぶことができる.よって$\mc{P}'=i$とできて
\begin{align*}
\mc{P}'=\mc{U}\mc{P}\mc{U}=e^{i\phi }e^{i\theta}e^{i\phi }=e^{i(\theta+2\phi)}=e^{i\pi}=i \quad (\phi=(\pi -\theta)/2 と選ぶ )
\end{align*}
(25.3.7)が
\begin{align*}
\mathsf{P}^{-1}\mr{Q}_{ar}'\mathsf{P}=i\sum_{b}e_{ab}\mr{Q}_{b}^*
\end{align*}
と簡単な形にするのが便利だ.(25.2.34)で定義した4成分ディラックスピノル生成子$Q_{a}$を作れば空間反転の表現はより簡単になって
\begin{align*}
\mathsf{P}^{-1} Q \mathsf{P}=\left(
\begin{matrix}
\mathsf{P}^{-1} e\mr{Q}^* \mathsf{P} \\
\mathsf{P}^{-1} \mr{Q} \mathsf{P}
\end{matrix}
\right)=\left(
\begin{matrix}
i\mr{Q} \\
ie\mr{Q}^*
\end{matrix}
\right)=i\beta Q
\end{align*}
という形になる.\par
拡張超対称性の場合には,必ずしも$\mc{P}'$が対角になるように$\mc{U}$を選ぶことはできない.($\mc{U}\mc{P}\mc{U}^{\dagger}$ではなく$\mc{U}\mc{P}\mc{U}^{T}$だから)しかし,(2章補遺Cで証明した)行列代数の定理を用いると,$\mc{U}$を適切に選べば$\mc{P}'$がブロック対角で
\begin{align*}
\mc{P}'=\left(
\begin{matrix}
P_{1} & 0 & \cdots &           &      \\
0 & P_2 & \cdots &            &       \\
\vdots & \vdots &\ddots   &       \\
          &           &             &P_i & \\
           &         &             &    &\ddots 
\end{matrix}
\right)
\end{align*}
一般に対角ブロックのいくつかは$1\times 1$部分行列で$i$(または他の任意の位相因子)に等しくとることができ
\begin{align*}
P_i=e^{i\theta}
\end{align*}
対角ブロックの他の部分行列は$2\times 2$行列で
\begin{align*}
P_i=\left(
\begin{matrix}
0 & \exp(i\phi) \\
\exp(-i\phi ) & 0
\end{matrix}
\right)
\end{align*}
の形を持つように採ることができる.ここで$\phi$は様々な位相だ.この$\mc{U}$の選択に対応して,(プライムを落として)2成分の$\mr{Q}$にはふたつの種類があることがわかる.$1\times 1$部分行列$P_r$からは
\begin{align*}
\mathsf{P}^{-1}\mr{Q}_{ar}\mathsf{P}=& i\sum_b e_{ab} \mr{Q}^*_{br} \\
\mathsf{P}^{-2}\mr{Q}_{ar}\mathsf{P}^2=&- \mr{Q}_{ar}
\end{align*}
が出て,(25.3.11)と同じ形ができる($i$以外の位相になる場合は$\mr{Q}$の位相を調節してやればいい).$2\times 2$部分行列$P_s$からは
\begin{align*}
\mathsf{P}^{-1}\left(
\begin{matrix}
\mr{Q}_{a1} \\
\mr{Q}_{a2}
\end{matrix}
\right)\mathsf{P} =&\left(
\begin{matrix}
0 & \exp(i\phi_s) \\
\exp(-i\phi_s) & 0
\end{matrix}
\right) \left(
\begin{matrix}
e_{ab}\mr{Q}^*_{b1} \\
e_{ab}\mr{Q}^*_{b2}
\end{matrix}
\right) \\
\therefore \quad \mathsf{P}^{-1} \mr{Q}_{as1}\mathsf{P}=&e^{i\phi_s}\sum_{b}e_{ab}\mr{Q}^*_{bs2} \\
 \mathsf{P}^{-1} \mr{Q}_{as2}\mathsf{P}=&e^{-i\phi_s}\sum_{b}e_{ab}\mr{Q}^*_{bs1}
\end{align*}
という二つの$\mr{Q}$の対が現れる.特に
\begin{align*}
\mathsf{P}^{-2}\mr{Q}_{as1}\mathsf{P}^2=&-e^{2i\phi_s} \mr{Q}_{as1} \\
\mathsf{P}^{-2}\mr{Q}_{as2}\mathsf{P}^2=&-e^{-2i\phi_s} \mr{Q}_{as2}
\end{align*}
となる.このことより,
\begin{align*}
\mathsf{P}^{-2}\left(A\mr{Q}_{as1}+B\mr{Q}_{as1}\right)\mathsf{P}^2=&-e^{2i\phi_s} A\mr{Q}_{as1}-e^{-2i\phi_s} B\mr{Q}_{as2} \\
=&- (A\mr{Q}_{as1}+ B\mr{Q}_{as2} ) \quad(\phi_s=0 \pmod \pi)
\end{align*}
となって,$\phi_s=0 \pmod \pi$でない限り,2番目の種類の拡張超対称性生成子の線形結合から1番目の種類の超対称性生成子を構成するのは不可能だ.\par
4成分スピノル(25.2.34)を用いると,パリティ演算子が1番目の種類の拡張対称性の生成子へ及ぼす効果は
\begin{align*}
\mathsf{P}^{-1} Q_r \mathsf{P}=i\beta Q_r
\end{align*}
と再び表され,2番目の種類の生成子については
\begin{align*}
\mathsf{P}^{-1} Q_{s1} \mathsf{P}=&\left(
\begin{matrix}
\mathsf{P}^{-1} e\mr{Q}^*_{s1} \mathsf{P} \\
\mathsf{P}^{-1} \mr{Q}_{s1} \mathsf{P}
\end{matrix}
\right)=\left(
\begin{matrix}
-e^{-i\phi_s}\mr{Q}_{s2} \\
e^{i\phi_s}e\mr{Q}^*_{s2}
\end{matrix}
\right) \\
=&\left(
\begin{matrix}
0 & 1 \\
1 & 0
\end{matrix}
\right)\left(
\begin{matrix}
1 & 0 \\
0 & -1
\end{matrix}
\right) \left(
\begin{matrix}
e^{i\phi_s}e\mr{Q}^*_{s2} \\
e^{-i\phi_s}\mr{Q}_{s2}
\end{matrix}
\right) \\
=&\left(
\begin{matrix}
0 & 1 \\
1 & 0
\end{matrix}
\right)\left(
\begin{matrix}
1 & 0 \\
0 & -1
\end{matrix}
\right) e^{i\gamma_5 \phi_2 }\left(
\begin{matrix}
e\mr{Q}^*_{s2} \\
\mr{Q}_{s2}
\end{matrix}
\right) \\
=&\beta \gamma_5 \exp(i\gamma_5 \phi_s )Q_{s2}
\end{align*}
(4つ目の式変形は,$Q_{s}$の上成分と下成分が,それぞれ$\gamma_5$が作用すると$\pm 1$になることを考えれば簡単に逆算する形で導ける.愚直に$\exp$を展開してもいいかもしれん)同様に
\begin{align*}
\mathsf{P}^{-1} Q_{s2} \mathsf{P}=\beta \gamma_5 \exp(-i\gamma_5 \phi_s )Q_{s1}
\end{align*}
となる.

\newpage

\subsection{質量ゼロ粒子の超対称多重項}
超対称性により,基地の粒子には超対称代数の既約表現の「s粒子(Sparticle)」が伴っていることが要求される.それは,クォークとレプトンにともなうボゾンの「スクォーク」と「スレプトン」,そしてゲージ・ボゾンに伴う,フェルミオンの「ゲージーノ」だ.もし超対称性があり,破れていないのであれば,既知の粒子とその超対称性パートナーは同じ質量をもつのだから既に見つかっていなければならない.しかし見つかっていないので,超対称性は確実に破れており,s粒子の質量は電弱$SU(2)\times U(1)$群の自発的破れによって生じたクォーク,レプトン,ゲージボゾンの質量よりはるかに大きいことはほぼ確実だ.よって超対称多重項内の質量分裂と同じ程度の大きさだ.超対称性の破れとこれらの質量の分裂を無視できるくらいの大きなエネルギースケールの理論では,既知のクォーク,レプトン,ゲージボゾンだけでなくその超対称パートナーを質量ゼロとして取り扱うことができる可能性が極めて高い.したがって,質量ゼロ粒子の超対称多重項超対称には特に興味が持たれる.

\vskip\baselineskip

ある超対称多重項に属する質量ゼロ粒子を1個だけ含むヘリシティ$\lambda$の状態$\ket{p,\lambda}$を考える.同じ超対称多重項の残りの状態は,演算子$\mr{Q}_{ar}$および(または)$\mr{Q}^*_{ar}$をこの状態に作用させて得られる.$\mr{Q}_{ar}$および$\mr{Q}_{ar}^*$は$P_\mu$と交換するのだったから,これらすべての状態は同じ4元運動量の値をもつ.
\begin{align*}
P_\mu \ket{p,\lambda}=&p_\mu \ket{p,\lambda} \\
P_\mu \mr{Q}_{ar}\ket{p,\lambda}=&p_\mu \mr{Q}_{ar}\ket{p,\lambda},\quad P_\mu \mr{Q}^*_{ar}\ket{p,\lambda}=p_\mu \mr{Q}^*_{ar}\ket{p,\lambda}
\end{align*}
これらの状態の4元運動量が(質量ゼロなので)$p^1=p^2=0,p^3=p^0=E$となるローレンツ系で考える.4元運動量をこのように選ぶと
\begin{align*}
\sigma_\mu p^\mu =E(\sigma_0 +\sigma_3)=2E \left(
\begin{matrix}
1 & 0 \\
0 & 0
\end{matrix}
\right)
\end{align*}
となり,これは因子$2E$を除いてヘリシティ$+1/2$の部分空間への射影行列だ.反交換関係(25.2.7)より,この運動量をもつ超対称多重項の任意の状態に作用させたとき
\begin{align*}
\{\mr{Q}_{-\frac{1}{2}r},\mr{Q}^*_{-\frac{1}{2}r}\}\ket{p,\lambda}=2(\sigma_\mu P^\mu)_{-\frac{1}{2},-\frac{1}{2}} \ket{p,\lambda}=2(\sigma_\mu p^\mu)_{-\frac{1}{2},-\frac{1}{2}} \ket{p,\lambda}=0
\end{align*}
となる.よって
\begin{align*}
\bra{p,\lambda} \{\mr{Q}_{-\frac{1}{2}r},\mr{Q}^*_{-\frac{1}{2}r}\}\ket{p,\lambda}=&\left|\mr{Q}_{-\frac{1}{2}r} \ket{p,\lambda}\right|^2+\left|\mr{Q}_{-\frac{1}{2}r}^{*} \ket{p,\lambda}\right|^2=0 \\
\therefore \quad \mr{Q}_{-\frac{1}{2}r}\ket{p,\lambda}=&\mr{Q}_{-\frac{1}{2}r}^*\ket{p,\lambda}
\end{align*}
がわかり,$\mr{Q}_{-\frac{1}{2}r}$と$\mr{Q}^*_{-\frac{1}{2}r}$もこの状態に作用するとゼロになることがわかる.したがって$\mr{Q}_{\frac{1}{2}r}$と$\mr{Q}^*_{\frac{1}{2}r}$のみを作用させて超対称多重項の状態を構成しなければならない.さらに以下のように$\mr{Q}$の添え字には$\mr{Q}_{ar}$の添え字$a$は,$J_3$の値
\begin{align*}
[J_3,\mr{Q}_{ar}]=-a \mr{Q}_{ar}
\end{align*}
をとっているのだった.したがって(質量ゼロ状態のヘリシティ$\lambda$は$J_3$の固有値$J_3\ket{p,\lambda}=\lambda \ket{p,\lambda}$だという2章の結果を思い出して)
\begin{align*}
J_3 \left(\mr{Q}_{\frac{1}{2}r}\ket{p,\lambda}\right)=\left[\mr{Q}_{\frac{1}{2}r}J_3 -\frac{1}{2}\mr{Q}_{\frac{1}{2}r}\right]\ket{p,\lambda} =\left(\lambda -\frac{1}{2}\right)\left(\mr{Q}_{\frac{1}{2}r}\ket{p,\lambda}\right)
\end{align*}
同様に
\begin{align*}
[J_3,\mr{Q}_{ar}^*]=&+a\mr{Q}_{ar}^* \\
J_3 \left(\mr{Q}^*_{\frac{1}{2}r}\ket{p,\lambda}\right)=\left(\lambda +\frac{1}{2}\right)\left(\mr{Q}^*_{\frac{1}{2}r}\ket{p,\lambda}\right)
\end{align*}
となり,$\mr{Q}_{\frac{1}{2}r}$と$\mr{Q}^*_{\frac{1}{2}r}$はそれぞれヘリシティを$1/2$だけ下げ上げするということがわかる.

\vskip\baselineskip


最初に単純超対称性の場合を考える.超対称多重項を考えて,その中で最大のヘリシティが$\lambda_{\mr{max}}$であるとする.(例えば天下りになるけど,ヘリシティ$+1/2$の電子とそのパートナであるヘリシティ$0$のスエレクトロンで超対称性多重項を組むと,$\lambda_{\max}=+1/2$になる.代わりに電子がヘリシティ$-1/2$の状態なら$\lambda_{\max}=0$になる.)$\ket{p,\lambda_{\max}}$をこのヘリシティと4元運動量$p^\mu$をもつ任意の1粒子状態とする.するとこれに$\mr{Q}^*_{\frac{1}{2}}$を作用させたものは最大ヘリシティ以上になってしまうから
\begin{align*}
\mr{Q}^*_{\frac{1}{2}}\ket{p,\lambda_{\max}}=0
\end{align*}
となる.一方この状態に$\mr{Q}_{\frac{1}{2}}$を作用させるとヘリシティ$\lambda_{\max}-1/2$の状態$\ket{p,\lambda_{\max}-1/2}$が得られる.この状態を
\begin{align*}
\ket{p,\lambda_{\max}-1/2}\equiv (4E)^{-1/2}\mr{Q}_{\frac{1}{2}}\ket{p,\lambda_{\max}}
\end{align*}
と定義する.基本的な反交換関係(25.2.7)と(25.4.1),(25.4.3)から
\begin{align*}
\braket{\lambda_{\max}-1/2|\lambda_{\max}-1/2}=&(4E)^{-1}\bra{p,\lambda_{\max}}\mr{Q}_{\frac{1}{2}}^*\mr{Q}_{\frac{1}{2}}\ket{p,\lambda_{\max}} \\
=&(4E)^{-1}\bra{p,\lambda_{\max}}2\cdot 2E-\mr{Q}_{\frac{1}{2}}\mr{Q}_{\frac{1}{2}}^*\ket{p,\lambda_{\max}} \\
=&\braket{p,\lambda_{\max}|p,\lambda_{\max}} \quad \because (25.4.3)
\end{align*}
となるので,$\ket{p,\lambda_{\max}}$と同じく規格化されている.よって特にこの状態は$\ket{p,\lambda_{\max}}$がゼロでないならゼロにはなれない.(もしゼロなら$\braket{p,\lambda_{\max}|p,\lambda_{\max}}$は非ゼロなのに上の関係式から矛盾が起きる.)単純超対称性だから(25.2.32)から$\mr{Q}_{\frac{1}{2}}^2=0$であり,よって$\mr{Q}_{\frac{1}{2}}$を$\ket{p,\lambda_{\max}-1/2}$に作用させるとゼロになる.
\begin{align*}
\mr{Q}_{\frac{1}{2}}\ket{p,\lambda_{\max}-1/2}=(4E)^{-1/2}\mr{Q}_{\frac{1}{2}}^2\ket{p,\lambda_{\max}}=0
\end{align*}
一方,$\mr{Q}_{\frac{1}{2}}^*$をこの状態に作用させると
\begin{align*}
\mr{Q}_{\frac{1}{2}}^*\ket{p,\lambda_{\max}-1/2}=&(4E)^{-1/2}\mr{Q}_{\frac{1}{2}}^*\mr{Q}_{\frac{1}{2}}\ket{p,\lambda_{\max}} \\
=&(4E)^{-1/2}\left\{\mr{Q}_{\frac{1}{2}}^*,\mr{Q}_{\frac{1}{2}}\right\}\ket{p,\lambda_{\max}} \quad \because (25.4.3) \\
=&(4E)^{-1/2}(\sigma_\mu P^\mu)_{\frac{1}{2}\frac{1}{2}}\ket{p,\lambda_{\max}}=(4E)^{1/2}\ket{p,\lambda_{\max}}
\end{align*}
となる.よって$\mr{Q}_{\frac{1}{2}}^*/\sqrt{4E},\mr{Q}_{\frac{1}{2}}/\sqrt{4E}$はそれぞれフェルミオン的な消滅・生成演算子として振舞うことがわかる.そしてこのように超対称多重項は,ヘリシティ$\lambda_{\max}$と$\lambda_{\max}-1/2$の2個の状態だけからなる.これら2個の状態からなる基底では,演算子$\mr{Q}_{\frac{1}{2}},\mr{Q}_{\frac{1}{2}}^*$はそれぞれ
\begin{align*}
\ket{p,\lambda_{\max}}=&\left(
\begin{matrix}
1 \\
0
\end{matrix}
\right) ,\quad \ket{p,\lambda_{\max}-1/2}=\left(
\begin{matrix}
0 \\
1
\end{matrix}
\right) \\
q_{\frac{1}{2}}=&\sqrt{4E}\left(
\begin{matrix}
0 & 0 \\
1 & 0
\end{matrix}
\right) ,\quad q^\dagger_{\frac{1}{2}}=\sqrt{4E}\left(
\begin{matrix}
0 & 1 \\
0 & 0
\end{matrix}
\right)
\end{align*}
で表現され,演算子$\mr{Q}_{-\frac{1}{2}},\mr{Q}^*_{-\frac{1}{2}}$はゼロで表現される. \\
これが単純超対称性を持つ理論において\uwave{唯一}の種類の質量ゼロ超対称多重項である.超対称パートナーをもたない質量ゼロ粒子は存在せず,2個以上の超対称パートナーをもつ質量ゼロ粒子も存在しない.もちろん$\mathsf{CPT}$不変性から,ヘリシティ$\lambda$と$\lambda-1/2$の質量ゼロ超対称多重項には,必ずヘリシティ$-\lambda+1/2$と$-\lambda$の反粒子からなる超対称多重項も伴って存在しなければならない.特に,ヘリシティ$+1/2$と$-1/2$の,質量ゼロ粒子とその反粒子には,ヘリシティ$+1$と$-1$か,またはヘリシティが共にゼロの質量ゼロ粒子と反粒子が伴っていなければならない.($\lambda=1/2$の場合には$\lambda-1/2=0$のパートナーが伴って,ヘリシティ$-1/2$の反粒子にもヘリシティ$0$のパートナーが伴うので,この場合は後者.$\lambda-1/2$の方が$1/2$なら$\lambda=+1$がパートナーとして伴い,これは前者になる.)\par
既知のクォーク,レプトン,ゲージ・ボゾンはこの描像にどのように当てはまるか?超対称性生成子は$SU(3)\times SU(2)\times U(1)$ゲージ群の生成子とは交換すると仮定する.クォークとレプトンは$SU(3)$と$SU(2)$の基本表現に属しており,随伴表現に属するゲージ・ボゾンとはゲージ群の別の表現に属する.よってクォークとレプトンの超対称多重項も同様に基本表現に属している必要がある.(仮定より$\mr{Q}$とゲージ群生成子とは交換するから,$(e,\nu_e)$が$SU(2)$基本表現として変換されるなら,$\mr{Q}$を作用させて作った電子とニュートリノの超対称多重項の集まり$(\tilde{e},\tilde{\nu}_e)$も$SU(2)$基本表現として変換される必要がある.)それらの超対称パートナーはボゾンである必要があるが,随伴表現として変換されてはいけない.したがって$SU(2)\times U(1)$対称性の破れが無視できる高エネルギー理論では,それぞれのカラーとフレーバーをもった質量ゼロのレプトンとクォークは,\uwave{ゼロ・ヘリシティ}かつ同じカラーおよびフレーバーをもった質量ゼロのスレプトンとスクォークと対になって超対称多重項を組む(ヘリシティ$\pm 1$のベクトルボゾンと組むと,それらは質量ゼロであるからゲージ・ボゾンでなければならないため,随伴表現に属することになってしまう.).一方質量ゼロのゲージ・ボゾンはヘリシティ$\pm 1/2$のゲージーノを伴って$SU(3)\times SU(2)\times U(1)$の随伴表現を構成すると結論づけなければならない.\par
重力が存在するから,標準模型の粒子に加えてヘリシティ$\pm 2$の質量ゼロ粒子である重力子も存在しなければならないことがわかる.$|\lambda|\leq 1/2$のヘリシティ$\lambda$をもつ質量ゼロ粒子は,低運動量では保存量と結合しなければならない(12章).ヘリシティ$\pm 1$の低エネルギー質量ゼロ粒子は様々な内部対称性の生成子と結合でき(例えば,光子は$U(1)_{\mr{em}}$電荷と結合するのだった.),ヘリシティ$\pm 3/2$の低エネルギー質量ゼロ粒子は超対称性の生成子$\mr{Q}_a$と結合できるらしい.ヘリシティ$\pm 2$の低エネルギー質量ゼロ粒子は単一の保存量である4元運動量ベクトル$P_\mu$と結合できる.しかし$|\lambda|>2$の低エネルギー質量ゼロ粒子が結合できる保存量はない.これより重力子はヘリシティ$\pm 5/2$の粒子と超対称性多重項を組むことができず,したがってヘリシティ$\pm3/2$の質量ゼロ粒子と超対称多重項を組むことが期待される.この粒子はグラヴィティーノと呼ばれ,超対称性生成子自身と結合する.この超対称多重項の場の理論は超重力理論と呼ばれる.(31章)\par

\vskip\baselineskip

次に,$N$個の超対称性生成子をもつ拡張超対称性の場合を考える.まず最初に議論した一般論より$\mr{Q}_{-\frac{1}{2}r}$は超対称性多重項の状態に作用すると($\mr{Q}_{\frac{1}{2}s}$を多重項の任意の状態に作用させて得られる状態を含めて)全てゼロになるので
\begin{align*}
0=&\mr{Q}_{-\frac{1}{2}r}\mr{Q}_{\frac{1}{2}s}\ket{p,\lambda}+\mr{Q}_{\frac{1}{2}s}\mr{Q}_{-\frac{1}{2}r}\ket{p,\lambda} \\
=&e_{-\frac{1}{2} \frac{1}{2}}Z_{rs} \ket{p,\lambda}=-Z_{rs} \ket{p,\lambda}
\end{align*}
よって中心電荷$Z_{rs}$もまた多重項の任意の状態$Z_{rs} \ket{p,\lambda}=0$で消さなければならない.中心電荷がなければ,超対称性生成子$\mr{Q}_{\frac{1}{2}r}(r=1,2,\cdots ,N)$たちは質量ゼロ粒子の超対称多重項に作用するときには全て反交換し,もちろん$\mr{Q}^2_{\frac{1}{2}r}=0$となる.よって$N$個の$\mr{Q}_{\frac{1}{2}r}$の中から,$n$個を重複なく選んで$\ket{p,\lambda_{\max}}$に作用させると
\begin{align*}
\ket{p,\lambda_{\max}-n/2;r_1,r_2,\cdots r_n}\equiv &\mr{Q}_{\frac{1}{2}r_1}\mr{Q}_{\frac{1}{2}r_2}\cdots \mr{Q}_{\frac{1}{2}r_n}\ket{p,\lambda_{\max}} \\
J_3\ket{p,\lambda_{\max}-n/2;r_1,r_2,\cdots r_n}=&(\lambda_{\max}-n/2)\ket{p,\lambda_{\max}-n/2;r_1,r_2,\cdots r_n}
\end{align*}
これは同じ4元運動量$p$でヘリシティ$\lambda_{\max}-n/2$をもつ状態になる(それぞれの$\mr{Q}$は反交換するから,順序は関係なくなる).順序を考えず重複のない$n$個の選び方は$\binom{N}{n}=N!/n!(N-n)!$通り存在する.よって4元運動量$p$でヘリシティ$\lambda_{\max}-n/2$の要素の数は$\binom{N}{n}=N!/n!(N-n)!$個となり,これは$SU(N)$R対称性(25.2.30)の$n$階反対称テンソル表現を作る.
\begin{align*}
\mr{Q}_{\frac{1}{2}r_1}\mr{Q}_{\frac{1}{2}r_2}\cdots \mr{Q}_{\frac{1}{2}r_n}\ket{p,\lambda_{\max}} \to& \sum_{s_1 \cdots s_n}V_{r_1 s_1}V_{r_2 s_2}\cdots V_{r_n s_n}\mr{Q}_{\frac{1}{2}s_1}\mr{Q}_{\frac{1}{2}s_2}\cdots \mr{Q}_{\frac{1}{2}s_n}\ket{p,\lambda_{\max}} \\
\ket{p,\lambda_{\max}-n/2;r_1,r_2,\cdots r_n}\to& \sum_{s_1 \cdots s_n}V_{r_1 s_1}V_{r_2 s_2}\cdots V_{r_n s_n} \ket{p,\lambda_{\max}-n/2;s_1,s_2,\cdots s_n} \\
\ket{p,\lambda_{\max}-n/2;\cdots, r_i, \cdots, r_j, \cdots }=&-\ket{p,\lambda_{\max}-n/2;\cdots, r_j,\cdots r_i, \cdots }
\end{align*}
ゼロでない状態を与える$n$の最大値は$n=N$であり(それ以上だと重複が起きて$\mr{Q}^2=0$のベキ零性から消えてしまう),したがって超対称性多重項の最小ヘリシティは
\begin{align*}
\lambda_{\min}=\lambda_{\max}-N/2
\end{align*}
で与えられる.質量ゼロ粒子のヘリシティ$\lambda$が$|\lambda|>2$となることを排除したければ,$\lambda_{\max}-\lambda_{\min}\leq 2-(-2)=4$でなければならず,よって拡張超対称性は$N\leq 8$のものしか許されない.(もちろんこれは4次元時空の場合であり,高次元では別となる.例えばM理論は11次元時空であるが,このときは$N=1$しか許されない.)

\vskip\baselineskip


$N=8$の場合には,$|\lambda|>2$のヘリシティが許されないのだから,可能な超対称多重項はただ一つとなる.それは,\par
ヘリシティ$\pm 2$の1個の重力子\par
ヘリシティ$\pm 3/2$の$8!/1!(8-1)!=8$個のグラヴィティーノ\par
ヘリシティ$\pm 1$の$8!/2!(8-2)!=28$個のゲージ・ボゾン\par
ヘリシティ$\pm 1/2$の$8!/3!(8-3)!=56$個のフェルミオン\par
ヘリシティ$0$の$8!/4!(8-4)!=70$個のボゾン\par
\noindent からなる.\par
これを$N=7$の場合と比較しよう.まずヘリシティ$+2$から下って行って\par
ヘリシティ$+ 2$の1個の重力子\par
ヘリシティ$+ 3/2$の$7!/1!(7-1)!=7$個のグラヴィティーノ\par
ヘリシティ$+ 1$の$7!/2!(7-2)!=21$個のゲージ・ボゾン\par
ヘリシティ$+ 1/2$の$7!/3!(7-3)!=35$個のフェルミオン\par
ヘリシティ$ 0$の$7!/4!(7-4)!=35$個のボゾン\par
ヘリシティ$- 1/2$の$7!/5!(7-5)!=21$個のフェルミオン\par
ヘリシティ$- 1$の$7!/6!(7-6)!=7$個のゲージ・ボゾン\par
ヘリシティ$-3/2 $の$7!/7!(7-7)!=1$個のグラヴィティーノ\par
\noindent が得られる.これより下げることは不可能だ.重力子はヘリシティ$\pm 2$であるがヘリシティ$-2$が足りないので,全てのヘリシティを反転した$\mathsf{CPT}$共役な超対称多重項を作り加えなければならない.するとここに\par
ヘリシティ$- 2$の1個の重力子\par
ヘリシティ$- 3/2$の$7$個のグラヴィティーノ\par
ヘリシティ$- 1$の$21$個のゲージ・ボゾン\par
ヘリシティ$- 1/2$の$35$個のフェルミオン\par
ヘリシティ$ 0$の$35$個のボゾン\par
ヘリシティ$+ 1/2$の$21$個のフェルミオン\par
ヘリシティ$+ 1$の$7$個のゲージ・ボゾン\par
ヘリシティ$+ 3/2 $の$1$個のグラヴィティーノ\par
\noindent を加えることになり,よって合計で\par
ヘリシティ$\pm 2$の1個の重力子\par
ヘリシティ$\pm 3/2$の$8$個のグラヴィティーノ\par
ヘリシティ$\pm 1$の$28$個のゲージ・ボゾン\par
ヘリシティ$\pm 1/2$の$56$個のフェルミオン\par
ヘリシティ$0$の$70$個のボゾン\par
\noindent となる.これは$N=8$のときの粒子内容と同じだ.このように,$N=8$と$N=7$の拡張超対称重力理論は正確に同じ粒子内容をもち,実際同等となるらしい.\par
他方,$N\leq 6$の拡張超対称重力理論は各々のヘリシティが$\pm3/2$のグラヴィティーノをちょうど$N$個持つことになり,したがって全て異なる理論となる.(上の操作を同様にやってやればすぐわかる.)\par
ただし,演習問題にあるようにヘリシティ$\pm 3/2$を超える粒子を含まない場合,$N=6$と$N=5$($\mathsf{CPT}$含む)をやってみると,\par
ヘリシティ$\pm 3/2$の$1$個のグラヴィティーノ\par
ヘリシティ$\pm 1$の$6$個のゲージ・ボゾン\par
ヘリシティ$\pm 1/2$の$15$個のフェルミオン\par
ヘリシティ$0$の$20$個のボゾン\par
\noindent となって,$N=6$と$N=5$は同等になる.

\vskip\baselineskip


$N\leq 4$の場合には大域的超対称性理論,すなわち重力子やグラヴィティーノを含まない超対称多重項を含む理論の可能性がある($\lambda =+1$から上の操作をスタートしても$\lambda=-1$までに終わるから).$N=4$超対称性の場合には,超対称多重項が1種類だけ存在し,それには\par
ヘリシティ$\pm 1$の1個のゲージ・ボゾン\par
ヘリシティ$\pm 1/2$の$4!/1!(4-1)!=4$個のフェルミオン\par
ヘリシティ$0$の$4!/2!(4-2)!=6$個のボゾン \par
\noindent が含まれる.これは$N=3$の大域的超対称性理論と同等になっている.実際$N=3$では\par
ヘリシティ$+ 1$の1個のゲージ・ボゾン\par
ヘリシティ$+ 1/2$の$3!/1!(3-1)!=3$個のフェルミオン\par
ヘリシティ$0$の$3!/2!(3-2)!=3$個のボゾン \par
ヘリシティ$-1/2$の$3!/3!(3-3)!=1$個のフェルミオン\par
\noindent と,全てのヘリシティを反転した$\mathsf{CPT}$共役な超対称多重項\par
ヘリシティ$- 1$の1個のゲージ・ボゾン\par
ヘリシティ$- 1/2$の$3$個のフェルミオン\par
ヘリシティ$0$の$3$個のボゾン \par
ヘリシティ$+1/2$の$1$個のフェルミオン\par
\noindent を加えることで\par
ヘリシティ$\pm 1$の1個のゲージ・ボゾン\par
ヘリシティ$\pm 1/2$の$4$個のフェルミオン\par
ヘリシティ$0$の$6$個のボゾン \par
\noindent となって,粒子内容が同じだとわかる.$N=4$については27.9節.

\vskip\baselineskip

$N=2$大域的拡張超対称性の場合は,$\mathsf{CPT}$によって関連している超対称多重項とは別に,二つの異なる種類の超対称多重項が存在する.一つ目は\textbf{ゲージ超対称多重項}で,その各々は\par
ヘリシティ$+1$の1個のゲージ・ボゾン\par
ヘリシティ$+1/2$の$2!/1!(2-1)!=2$個のフェルミオン($SU(2)$R対称性で二重項を作る)\par
ヘリシティ$0$の$2!/2!(2-2)!=1$個のボゾン\par
\noindent を含む.またこの超対称多重項にはヘリシティを逆にした$\mathsf{CPT}$共役な超対称多重項が伴っている.これは\par
ヘリシティ$-1$の1個のゲージ・ボゾン\par
ヘリシティ$-1/2$の$2$個のフェルミオン($SU(2)$R対称性で二重項を作る)\par
ヘリシティ$0$の$1$個のボゾン\par
であるから,これを合わせると
ヘリシティ$\pm 1$の1個のゲージ・ボゾン\par
ヘリシティ$\pm 1/2$の$2$個のフェルミオン($SU(2)$R対称性で二重項を作る)\par
ヘリシティ$0$の$2$個のボゾン($SU(2)$1重項)\par
\noindent となる.\par
もう一個の種類は,\textbf{ハイパー多重項}で\par
ヘリシティ$\pm 1/2$の1個のフェルミオン\par
ヘリシティ$0$の$2!/1!(2-1)!=2$個のボゾン($SU(2)$R対称性で二重項を作る)\par
\noindent を含み,さらに\uwave{$\mathsf{CPT}$多重項を伴っている}.(そうでないとすると,自分自身が$\mathsf{CPT}$で同じ,すなわち自分自身が反粒子となり,このようなヘリシティ$0$のボゾンは実スカラー場で表現されるのだった(5章).しかし$SU(2)$の2次元実表現は存在せず,これは不可能だからだ.)\par
もちろん,現実の世界では重力子も存在するだろうから\par
ヘリシティ$+2$の$1$個の重力子\par
ヘリシティ$+3/2$の$2$個のグラヴィティーノ($SU(2)$R対称性で二重項を作る)\par
ヘリシティ$+1$の$1$個のゲージ・ボゾン\par
\noindent を含む重力子超対称多重項,およびその逆ヘリシティを持った$\mathsf{CPT}$共役多重項も存在しなければならないだろう.$N=2$のゲージ理論を27.9節で構成し,29.5節で非摂動論的に調べる.

\vskip\baselineskip


これらの超対称多重項の粒子内容は,拡張超対称性を到達可能なエネルギーでの粒子の現実的な理論に組み込むことが如何に困難であるかを浮き彫りにする.一つの例外($N=2$ハイパー多重項)を除いて,ヘリシティ$+1/2$のフェルミオンはヘリシティ$+1$のゲージ・ボゾンと必ず超対称多重項に属する.ゲージ・ボゾンはゲージ群の随伴表現($SU(N)$や$SO(N)$の場合,これは実表現)に属するから,超対称性生成子がゲージ群のもとで不変ならヘリシティ$+1/2$のフェルミオンもまた随伴表現に属さなければならず,よって実表現に属することになる.これは既知のクォークとレプトンが属する$SU(3)\times SU(2)\times U(1)$の表現が\textbf{カイラル}であるという事実と矛盾する.カイラルであるとは,ヘリシティ$+1/2$のフェルミオンはその複素表現に属しているということであり,その表現は必然的に$\mathsf{CPT}$共役であるヘリシティ$-1/2$のフェルミオンが持っている表現とは異なる.(例えばヘリシティ$+1/2$レプトンは$SU(2)$の$\mathbf{2}$表現に属していて,その共役は共役表現$\bar{\mathbf{2}}$に属している.)\par
ひとつの例外となる$N=2$ハイパー多重項については,ヘリシティ$+1/2$のフェルミオンはゲージ・ボゾンと同じ超対称多重項に属さない.しかしこの場合にはヘリシティ$+1/2$と$-1/2$の両方の粒子は同じ超対称多重項に属し,したがって超対称性生成子を不変に保つ任意のゲージ変換の下で\uwave{同じ変換性}を持つ必要がある.この場合,今度はゲージ群の複素表現に属していてもよいが,このハイパー多重項の$\mathsf{CPT}$共役多重項は複素共役表現に属す.よってハイパー多重項のヘリシティ$+1/2$のフェルミオンがゲージ群の複素表現に属しているとすると,その$\mathsf{CPT}$共役多重項のヘリシティ$+1/2$のフェルミオンは共役表現に属することとなり,この二つの粒子の和は実表現となり(?),再び既存のクォークとレプトンのカイラルな性質とは矛盾する.\par
これに対して,単純超対称性の場合にはヘリシティ$+1/2$とヘリシティ$0$\uwave{のみ}を含む超対称多重項が存在し,それは(前に言ったように)$\mathsf{CPT}$共役な超対称多重項が持つ表現とは異なるゲージ群の複素表現に属することが可能だ!この場合にはカイラルな性質との矛盾はない.到達可能なエネルギースケールにおいて破れずに残っている対称性としての超対称性の議論が,拡張超対称性ではなく単純超対称性に集中しているのはこれが理由らしい.

\newpage

\subsection{質量をもつ粒子の超対称多重項}
既知のクォーク,レプトン,ゲージ・ボゾンとそれらの超対称パートナーは超対称性の破れが無視できるエネルギースケールでは多分質量ゼロとして扱えるかもしれないが,これは強い相互作用と電弱相互作用との大統一理論で要請される大きな質量をもった余分なゲージ・ボゾンを含めて,他の粒子については必ずしも正しくない.Wess・Zumino模型以降,質量をもつ粒子の理論は超対称性理論を研究するための有用な試験の場となってきた.そこで,質量をもつ粒子の場合に破れていない超対称性がもつ意味を簡潔に考察するのは有益となる.\par
前節の序盤と同様に,超対称多重項の様々な1粒子状態は,それらの任意の一つに演算子$\mr{Q}_{ar}$と$\mr{Q}^*_{ar}$を作用させて得られて,それらの状態は全て同じ4元運動量をもつ.今度は質量ゼロの場合とは異なり,$M>0$の質量の場合には,これを静止した粒子の4元運動量に採ることができる.それは$i=1,2,3$については$p^i=0$で,$p^0=M$となる.この座標系では
\begin{align*}
\sigma_\mu p^\mu =M\sigma^0=M\left(
\begin{matrix}
1 & 0 \\
0 & 1
\end{matrix}
\right)
\end{align*}
となる.したがって,この4元運動量をもつ超対称多重項の任意の状態$\ket{p}$に反交換関係(25.2.7)を作用させると
\begin{align*}
\{\mr{Q}_{ar},\mr{Q}_{bs}^*\}\ket{p}=2(\sigma_\mu P^\mu)_{ab}\ket{p}=2M\delta_{ab}\delta_{rs} \ket{p}\neq 0
\end{align*}
を得る.よってゼロ質量の場合とは違い,この場合は$\mr{Q}_{ar}$あるいは$\mr{Q}_{ar}^*$が多重項全体を消すことはできず,よって2組の昇降演算子が存在する.つまり,$\mr{Q}_{\frac{1}{2}r}$と$\mr{Q}^*_{-\frac{1}{2}r}$の両方ともがスピンの第3成分を$1/2$だけ下げ,$\mr{Q}_{-\frac{1}{2}r}$と$\mr{Q}^*_{\frac{1}{2}r}$の両方ともがスピンの第3成分を$1/2$だけ上げる.
\begin{align*}
J_3\ket{p,\sigma}=&\sigma \ket{p,\sigma} \\
J_3 \mr{Q}_{\frac{1}{2}r} \ket{p,\sigma}=&\left(\sigma-\frac{1}{2}\right)\mr{Q}_{\frac{1}{2}r} \ket{p,\sigma} ,\quad J_3 \mr{Q}_{-\frac{1}{2}r}^* \ket{p,\sigma}=\left(\sigma-\frac{1}{2}\right)\mr{Q}_{-\frac{1}{2}r}^* \ket{p,\sigma} \\
J_3 \mr{Q}_{-\frac{1}{2}r} \ket{p,\sigma}=&\left(\sigma+\frac{1}{2}\right)\mr{Q}_{-\frac{1}{2}r} \ket{p,\sigma} ,\quad  J_3 \mr{Q}_{\frac{1}{2}r}^* \ket{p,\sigma}=\left(\sigma+\frac{1}{2}\right)\mr{Q}_{\frac{1}{2}r}^* \ket{p,\sigma}
\end{align*}
しかし,以下でみるように,拡張超対称性の場合には$Q$と$Q^*$のある線形結合をとるとゼロになることは可能らしい.

\vskip\baselineskip

最初に単純超対称性を考える.超対称代数(25.2.31)と(25.2.32)を使って,質量がゼロでない一般的な超対称多重項はスピン$j+1/2$の粒子1個と,スピン$j$の粒子の\uwave{対}と,スピン$j-1/2$の粒子1個からなることを示す.パリティが保存する場合には,スピン$j\pm 1/2$の粒子はある位相$\eta$で与えられる同一の固有パリティをもち,スピン$j$の対となる粒子はそれぞれパリティ$+i\eta$と$-i\eta$をもつ.ここで$j$はゼロより大きな整数または半整数だ.$2$個のスピンゼロ粒子と1個のスピン$1/2$粒子からなるつぶれた超対称多重項も存在し(つまり$j=0$なので$j-1/2$がない),パリティが保存するとき,対となるスピンゼロ粒子はそれぞれパリティ$i\eta$と$-i\eta$をもつ.ここで$\eta$はスピン$1/2$粒子の固有パリティだ.\par
証明は以下の通り.任意の超対称多重項は,状態$\ket{j,\sigma}$のスピン多重項で,スピンの第3成分$\sigma$は$-j$から$j$まで間隔1おきに値をとり,そのような全ての$\sigma$と$a=\pm 1/2$について
\begin{align*}
\mr{Q}_{a}\ket{j,\sigma}=0
\end{align*}
を満たすようなスピン$j$の状態を少なくとも1個含む,という特徴をもつことを最初に示そう.まず超対称多重項のゼロでない任意の状態$\ket{\psi}$(運動量は先程の通り静止しているとして)に対して,ゼロでない状態
\begin{align*}
\ket{\psi'}\equiv \left\{
\begin{array}{ll}
(2M)^{-1/2}\mr{Q}_{\frac{1}{2}}\ket{\psi} & (\mr{if} \quad \mr{Q}_{\frac{1}{2}}\ket{\psi}\neq 0) \\
\ket{\psi} & (\mr{if} \quad \mr{Q}_{\frac{1}{2}}\ket{\psi}=0)
\end{array}
 \right.
\end{align*}
を定義できて,さらにそれに対してゼロでない状態
\begin{align*}
\ket{\psi''}\equiv \left\{
\begin{array}{ll}
(2M)^{-1/2}\mr{Q}_{-\frac{1}{2}}\ket{\psi'} & (\mr{if} \quad \mr{Q}_{-\frac{1}{2}}\ket{\psi'} \neq 0) \\
\ket{\psi'} & (\mr{if} \quad \mr{Q}_{-\frac{1}{2}}\ket{\psi'}=0)
\end{array}
 \right.
\end{align*}
を定義できる.(25.2.32)より$\mr{Q}_a$は反交換するから,
\begin{align*}
\mr{Q}_{\frac{1}{2}}\ket{\psi'} =& \left\{
\begin{array}{ll}
(2M)^{-1/2}\mr{Q}_{\frac{1}{2}}^2\ket{\psi} & (\mr{if} \quad \mr{Q}_{\frac{1}{2}}\ket{\psi}\neq 0) \\
\mr{Q}_{\frac{1}{2}}\ket{\psi} & (\mr{if} \quad \mr{Q}_{\frac{1}{2}}\ket{\psi}=0)
\end{array}
 \right. \\
 =&0
\end{align*}
となってどちらにせよ消える.これを用いると,まず
\begin{align*}
\mr{Q}_{\frac{1}{2}}\ket{\psi''}=& \left\{
\begin{array}{ll}
(2M)^{-1/2}\mr{Q}_{\frac{1}{2}}\mr{Q}_{-\frac{1}{2}}\ket{\psi'} & (\mr{if} \quad \mr{Q}_{-\frac{1}{2}}\ket{\psi'} \neq 0) \\
\mr{Q}_{\frac{1}{2}} \ket{\psi'} & (\mr{if} \quad \mr{Q}_{-\frac{1}{2}}\ket{\psi'}=0)
\end{array}
 \right. \\
 =&\left\{
\begin{array}{ll}
-(2M)^{-1/2}\mr{Q}_{-\frac{1}{2}}\mr{Q}_{\frac{1}{2}}\ket{\psi'} & (\mr{if} \quad \mr{Q}_{-\frac{1}{2}}\ket{\psi'} \neq 0) \\
\mr{Q}_{\frac{1}{2}} \ket{\psi'} & (\mr{if} \quad \mr{Q}_{-\frac{1}{2}}\ket{\psi'}=0)
\end{array}
 \right. \\
 =& 0
\end{align*}
かつ
\begin{align*}
\mr{Q}_{-\frac{1}{2}}\ket{\psi''}=& \left\{
\begin{array}{ll}
(2M)^{-1/2}\mr{Q}_{-\frac{1}{2}}^2\ket{\psi'} & (\mr{if} \quad \mr{Q}_{-\frac{1}{2}}\ket{\psi'} \neq 0) \\
\mr{Q}_{-\frac{1}{2}} \ket{\psi'} & (\mr{if} \quad \mr{Q}_{-\frac{1}{2}}\ket{\psi'}=0)
\end{array}
 \right. \\
 =&0
\end{align*}
となり,よって$a=\pm 1/2$について$\mr{Q}_{a}\ket{\psi''}=0$がなりたつ.任意の状態$\ket{\psi''}$が条件$\mr{Q}_{a}\ket{\psi''}=0$を満たすならば,$U(R)\ket{\psi''}$も(p44の$[J_i,\mr{Q}_a]=-\frac{1}{2}(\sigma_i)_{ab}\mr{Q}_b$から)
\begin{align*}
\mr{Q}_{a}\left[U(R)\ket{\psi''}\right]=&U(R) U^{-1}(R)\mr{Q}_{a}U(R)\ket{\psi''} \\
=&e^{i\theta_i J_i } e^{-i\theta_i J_i} \mr{Q}_{a} e^{i\theta_i J_i} \ket{\psi''} \\
=&e^{i\theta_i J_i } \left[e^{i\theta_i \frac{\sigma_i}{2}}\right]_{ab}\mr{Q}_{b} \ket{\psi''} \\
=&0
\end{align*}
となり,この条件を満たす.したがってこの状態部分空間$\{\ket{\psi''} :\mr{Q}_{a}\ket{\psi''}=0\}$は$SU(2)$回転群のもとで閉じており,完全系$\{\ket{j,\sigma}\}$で展開することができる.(回転$SU(2)$群の不変部分空間であるから,peter-weylの定理「コンパクト位相群の連続なユニタリ表現は有限次元既約表現の直和に分かれる」より,既約表現の有限個の直和として書けるので回転$SU(2)$の既約表現$\ket{j,\sigma}$の線形結合で書ける.)
\begin{align*}
\ket{\psi''}=\sum_{j,\sigma} c_{j,\sigma}\ket{j,\sigma}
\end{align*}
$\ket{\psi''}$はゼロでないから,少なくともひとつの係数$c_{j,\sigma}$は非ゼロであり,$\mr{Q}_{a}\ket{\psi''}=0$より
\begin{align*}
\mr{Q}_{a}\ket{j,\sigma}=0
\end{align*}
を満たす$j,\sigma$の組を少なくとも1つ含む.上と同じことだけど
\begin{align*}
\mr{Q}_a(J_{1}\pm i J_2)\ket{j,\sigma} =&+\frac{1}{2}(\sigma_1\pm i\sigma_2)_{ab}\mr{Q}_{b} \ket{j,\sigma}+(J_1\pm iJ_2)\mr{Q}_a \ket{j,\sigma}=0 \\
\propto &\mr{Q}_a \ket{j,\sigma \pm 1} \\
\therefore \quad &\mr{Q}_a \ket{j,\sigma \pm 1}=0
\end{align*}
これを繰り返して全ての$\sigma$について考えてやれば,これで全ての$-j\leq \sigma \leq +j$で条件式(25.5.3)を満たす状態$\ket{j,\sigma}$が存在するようなスピン$j$が少なくとも一つ存在することが分かった.これがまず証明したいことだった.\par
(25.5.3)を満たすこれらのスピン多重項の任意の一つで
\begin{align*}
\braket{j,\sigma'|j,\sigma}=\delta_{\sigma\sigma'}
\end{align*}
と規格化されたものに着目する.$j>0$の場合,スピン$1/2$演算子$\mr{Q}_{a}^*$をこれらの状態に作用させると,スピン$j+a=j\pm 1/2$状態($\mr{Q}_a^*$はスピンの第三成分を$a=\pm 1/2$だけ上げるのだったから)
\begin{align*}
\ket{j\pm 1/2,\sigma}=\frac{1}{\sqrt{2M}}\sum_a C_{\frac{1}{2}j }\left(j\pm 1/2 ,\sigma ;a ,\sigma- a\right)\mr{Q}_{a}^*\ket{j,\sigma-a}
\end{align*}
を構成できる.ここで$C_{jj'}(j'',\sigma'';\sigma,\sigma')$はスピン第三成分$\sigma,\sigma'$の$j,j'$を合成して第三成分$\sigma''=\sigma+\sigma'$のスピン$j''$を作る通常のグレブシュ・ゴルダン係数だ.(25.5.2)~(25.5.5)とクレブシュ・ゴルダン係数の正規直交性
\begin{align*}
\sum_{a=-A}^A \sum_{b=-B}^B C_{AB}(J,M;a,b)C_{AB}(J',M';a,b)=&\delta_{JJ'}\delta_{MM'} \\
\therefore \quad \sum_{a=-A}^A C_{AB}(J,M;a,M-a)C_{AB}(J',M;a,M-a)=&\delta_{JJ'} \quad \because a+b=M
\end{align*}
を使うと
\begin{align*}
\braket{j\pm 1/2 ,\sigma|j\pm 1/2, \sigma'} =&\frac{1}{2M}\sum_{a}\sum_{b}C_{\frac{1}{2}j }\left(j\pm 1/2 ,\sigma ;a ,\sigma- a\right)C_{\frac{1}{2}j }\left(j\pm 1/2 ,\sigma' ;b ,\sigma- b\right)  \\
&\qquad \qquad \qquad \times \bra{j,\sigma-a}\mr{Q}_{a}\mr{Q}_{b}^* \ket{j,\sigma'-b} \\
=&\frac{1}{2M}\sum_{a}\sum_{b}C_{\frac{1}{2}j }\left(j\pm 1/2 ,\sigma ;a ,\sigma- a\right)C_{\frac{1}{2}j }\left(j\pm 1/2 ,\sigma' ;b ,\sigma'- b\right) \\
&\qquad \qquad \qquad \times \bra{j,\sigma-a}\{\mr{Q}_{a},\mr{Q}_{b}^*\} \ket{j,\sigma'-b}\quad \because(25.5.3) \\
=&\frac{1}{2M}\sum_{a}\sum_{b}C_{\frac{1}{2}j }\left(j\pm 1/2 ,\sigma ;a ,\sigma- a\right)C_{\frac{1}{2}j }\left(j\pm 1/2 ,\sigma' ;b ,\sigma'- b\right) \\
&\qquad \qquad \qquad \times 2M \delta_{ab}\braket{j,\sigma-a|j,\sigma'-b} \\
=&\frac{1}{2M}\sum_{a}\sum_{b}C_{\frac{1}{2}j }\left(j\pm 1/2 ,\sigma ;a ,\sigma- a\right)C_{\frac{1}{2}j }\left(j\pm 1/2 ,\sigma' ;b ,\sigma'- b\right) \\
&\qquad \qquad \qquad \times 2M \delta_{ab}\braket{j,\sigma-a|j,\sigma'-a} \\
=&\frac{1}{2M}\sum_{a}\sum_{b}C_{\frac{1}{2}j }\left(j\pm 1/2 ,\sigma ;a ,\sigma- a\right)C_{\frac{1}{2}j }\left(j\pm 1/2 ,\sigma' ;b ,\sigma'- b\right) \\
&\qquad \qquad \qquad \times 2M \delta_{\sigma\sigma'}\delta_{ab} \\
=&\delta_{\sigma\sigma'}\sum_{a}C_{\frac{1}{2}j }\left(j\pm 1/2 ,\sigma ;a ,\sigma- a\right)C_{\frac{1}{2}j }\left(j\pm 1/2 ,\sigma ;a ,\sigma- a\right) \\
=&\delta_{\sigma\sigma'} \\
\braket{j\pm 1/2 ,\sigma|j\mp 1/2, \sigma'}=&0
\end{align*}
と適切に規格化されていることがわかる.したがって状態$\ket{j\pm 1/2 ,\sigma}$はどれもゼロになれない.ただし例外は$j=0$のときで,その場合はスピン$j-1/2$がないから状態$\ket{j-1/2,\sigma}$は存在しない.また\uwave{2個}の$\mr{Q}^*$を$\ket{j,\sigma}$に作用させて別の状態を構成できる.各々の$\mr{Q}_{a}^*$は自分自身と反交換するから$\mr{Q}_{a}^{*2}=0$となり,唯一のゼロでないそのような状態は演算子$\mr{Q}^*_{\frac{1}{2}}\mr{Q}^*_{-\frac{1}{2}}(=-\mr{Q}^*_{-\frac{1}{2}}\mr{Q}^*_{\frac{1}{2}})$を作用させて作ることのできるものだ.この演算子は
\begin{align*}
\mr{Q}^*_{\frac{1}{2}}\mr{Q}^*_{-\frac{1}{2}}=\frac{1}{2}\mr{Q}^*_{\frac{1}{2}}\mr{Q}^*_{-\frac{1}{2}}-\frac{1}{2}\mr{Q}^*_{-\frac{1}{2}}\mr{Q}^*_{\frac{1}{2}}=\frac{1}{2}e_{ab}\mr{Q}^*_{a}\mr{Q}^*_{b}
\end{align*}
と書くことができ,したがって回転不変(直接出しておくと,$e\mr{Q}^*$は$(1/2,0)$表現なので
\begin{align*}
[A_i,(e\mr{Q}^*)_a]=&-\frac{1}{2}\sum_{b}(\sigma_i)_{ab}(e\mr{Q}^*)_b,\quad [B_i,(e\mr{Q}^*)_a]=0 \\
\therefore \quad  [J_i,(e\mr{Q}^*)_a]=&-\frac{1}{2}\sum_{b}(\sigma_i)_{ab}(e\mr{Q}^*)_b \\
[A_i,\mr{Q}_{a}^*]=&0,\quad [B_i,\mr{Q}_{a}^*]=+\frac{1}{2}\sum_{b}(\sigma_i)_{ba}\mr{Q}^*_b \\
\therefore \quad [J_i,\mr{Q}_{a}^*]=&+\frac{1}{2}\sum_{b}(\sigma_i)_{ba}\mr{Q}^*_b \\
[J_i,\mr{Q}^*_a e_{ab}\mr{Q}^*_b]=&[J_i,\mr{Q}^*_{a}](e\mr{Q}^*)_a+\mr{Q}^*_{a}[J_i,(e\mr{Q}^*)_a] \\
=&+\frac{1}{2}\sum_{c}(\sigma_i)_{ca}\mr{Q}_{c}^*(e\mr{Q}^*)_a-\frac{1}{2}\sum_{c}\mr{Q}_{a}^* (\sigma_i)_{ac}(e\mr{Q}^*)_{c} \\
=&+\frac{1}{2}\sum_{c}\mr{Q}_{a}^* (\sigma_i)_{ac}(e\mr{Q}^*)_{c}-\frac{1}{2}\sum_{c}\mr{Q}_{a}^* (\sigma_i)_{ac}(e\mr{Q}^*)_{c} \\
=&0 \\
U^{-1}(R)\mr{Q}^*_a e_{ab}\mr{Q}^*_b U(R)=&\mr{Q}^*_a e_{ab}\mr{Q}^*_b
\end{align*}
となって確かに回転不変になる.)これはスピン$j$の\uwave{別の}スピン多重項を与える.
\begin{align*}
\ket{j,\sigma}^\flat=\frac{1}{2M}\mr{Q}^*_{\frac{1}{2}}\mr{Q}^*_{-\frac{1}{2}}\ket{j,\sigma}
\end{align*}
(実際に$J_3$を作用させたら$\mr{Q}^*_{\frac{1}{2}}\mr{Q}^*_{-\frac{1}{2}}$を通りぬけて$\ket{j,\sigma}$に作用して$\sigma$が出てくるし,$J^2$をかけたら$j(j+1)$が出てくる.)これは(25.5.3)の代わりに
\begin{align*}
\mr{Q}^*_{a}\ket{j,\sigma}^\flat=0
\end{align*}
を満たすという点が$\ket{j,\sigma}$と異なる.再び(25.5.2)~(25.5.4)から
\begin{align*}
{}^\flat\braket{j,\sigma'|j,\sigma}^\flat=&\frac{1}{4M^2}\bra{j,\sigma'} \mr{Q}_{-\frac{1}{2}}\mr{Q}_{\frac{1}{2}}\mr{Q}^*_{\frac{1}{2}}\mr{Q}^*_{-\frac{1}{2}}\ket{j,\sigma} \\
=&\frac{1}{2M}\bra{j,\sigma'} \mr{Q}_{-\frac{1}{2}}\mr{Q}^*_{-\frac{1}{2}}\ket{j,\sigma} \\
&-\frac{1}{4M^2}\bra{j,\sigma'} \mr{Q}_{-\frac{1}{2}}\mr{Q}^*_{\frac{1}{2}} \mr{Q}_{\frac{1}{2}} \mr{Q}^*_{-\frac{1}{2}}\ket{j,\sigma}\\
=&\braket{j,\sigma'|j,\sigma}-\frac{1}{2M}\bra{j,\sigma'} \mr{Q}^*_{-\frac{1}{2}} \mr{Q}_{-\frac{1}{2}} \ket{j,\sigma}  \\
&+\frac{1}{4M^2}\bra{j,\sigma'} \mr{Q}_{-\frac{1}{2}}\mr{Q}^*_{\frac{1}{2}} \mr{Q}^*_{-\frac{1}{2}} \mr{Q}_{\frac{1}{2}}\ket{j,\sigma} \\
=&\delta_{\sigma'\sigma} \quad \because (25.5.3)から最初の項だけ残り,(25.5.4) \\
\braket{j,\sigma'|j,\sigma}^\flat =&\frac{1}{2M}\bra{j,\sigma'}\mr{Q}^*_{\frac{1}{2}}\mr{Q}^*_{-\frac{1}{2}}\ket{j,\sigma} \\
=&0 \quad \because (25.5.3)
\end{align*}
と規格化された状態であることがわかる.次に,ここまでに構成されてきた状態が超対称代数の完全な表現を構成することを示す.クレブシュ・ゴルダン係数の正規直交性から($\pm$についての和は$J=j+1/2,j-1/2$についての和と解釈する)
\begin{align*}
\ket{j+1/2,\sigma+1/2}=&\frac{1}{\sqrt{2M}}C_{\frac{1}{2}j}(j+1/2,\sigma+1/2;1/2,\sigma)\mr{Q}_{\frac{1}{2}}^*\ket{j,\sigma} \\
&+\frac{1}{\sqrt{2M}}C_{\frac{1}{2}j}(j+1/2,\sigma+1/2;-1/2,\sigma+1)\mr{Q}_{-\frac{1}{2}}^*\ket{j,\sigma+1} \\
=&\frac{1}{\sqrt{2M}}\sqrt{\frac{j+\sigma+1}{2j+1}}\mr{Q}_{\frac{1}{2}}^*\ket{j,\sigma} \\
&+\frac{1}{\sqrt{2M}}\sqrt{\frac{j-\sigma}{2j+1}}\mr{Q}_{-\frac{1}{2}}^*\ket{j,\sigma+1}  \\
\ket{j-1/2,\sigma+1/2}=&\frac{1}{\sqrt{2M}}C_{\frac{1}{2}j}(j-1/2,\sigma+1/2;1/2,\sigma)\mr{Q}_{\frac{1}{2}}^*\ket{j,\sigma} \\
&+\frac{1}{\sqrt{2M}}C_{\frac{1}{2}j}(j+1/2,\sigma+1/2;-1/2,\sigma+1)\mr{Q}_{-\frac{1}{2}}^*\ket{j,\sigma+1} \\
=&-\frac{1}{\sqrt{2M}}\sqrt{\frac{j-\sigma}{2j+1}}\mr{Q}_{\frac{1}{2}}^*\ket{j,\sigma} \\
&+\frac{1}{\sqrt{2M}}\sqrt{\frac{j+\sigma+1}{2j+1}}\mr{Q}_{-\frac{1}{2}}^*\ket{j,\sigma+1}  \\
\therefore \quad \mr{Q}^*_{\frac{1}{2}}\ket{j,\sigma}=& \sqrt{2M}C_{\frac{1}{2}j}(j+1/2,\sigma+1/2;1/2,\sigma)\ket{j+1/2,\sigma+1/2} \\
&+\sqrt{2M}C_{\frac{1}{2}j}(j-1/2,\sigma+1/2;1/2,\sigma)\ket{j-1/2,\sigma+1/2} \\
=&\sqrt{2M}\sum_{\pm}C_{\frac{1}{2}j}(j \pm 1/2,\sigma+1/2;1/2,\sigma)\ket{j\pm 1/2,\sigma+1/2}
\end{align*}
同様に
\begin{align*}
\mr{Q}^*_{-\frac{1}{2}}\ket{j,\sigma}=\sqrt{2M}\sum_{\pm}C_{\frac{1}{2}j}(j \pm 1/2,\sigma-1/2;-1/2,\sigma)\ket{j\pm 1/2,\sigma-1/2}
\end{align*}
もわかって
\begin{align*}
\mr{Q}^*_{a}\ket{j,\sigma}=\sqrt{2M}\sum_{\pm}C_{\frac{1}{2}j}(j \pm 1/2,\sigma+a;a,\sigma)\ket{j\pm 1/2,\sigma +a}
\end{align*}
がわかる.(直交関係式がダイレクトに使えない気がしたので,具体的なクレブシュゴルダン係数の表示を用いて導いた.うまくいかないかなぁ.一応クレブシュゴルダン係数の具体形がこうなることを証明しておく.数学的帰納法を使う.$j$スピンと$1/2$スピンの合成が
\begin{align*}
\Ket{j+\frac{1}{2},\sigma+\frac{1}{2}}=&C_{\frac{1}{2}j}\left(j+\frac{1}{2},\sigma+\frac{1}{2};\frac{1}{2},\sigma \right)\Ket{\frac{1}{2},\frac{1}{2}}\ket{j,\sigma} \\
&+C_{\frac{1}{2}j}\left(j+\frac{1}{2},\sigma+\frac{1}{2};-\frac{1}{2},\sigma+1 \right)\Ket{\frac{1}{2},-\frac{1}{2}}\ket{j,\sigma+1}
\end{align*}
となるが,ここである$\sigma$で
\begin{align*}
\Ket{j+\frac{1}{2},\sigma+\frac{1}{2}}=&\sqrt{\frac{j+\sigma+1}{2j+1}}\Ket{\frac{1}{2},\frac{1}{2}}\ket{j,\sigma} \\
&+\sqrt{\frac{j-\sigma}{2j+1}}\Ket{\frac{1}{2},-\frac{1}{2}}\ket{j,\sigma+1}
\end{align*}
となると仮定する.$\sigma=j,-j-1$では自明になりたつ.両辺に下降演算子(あるいは上昇演算子)を作用させると
\begin{align*}
J_-\Ket{j+\frac{1}{2},\sigma+\frac{1}{2}}=&\sqrt{\left(j+\frac{1}{2}\right)\left(j+\frac{3}{2}\right)-\left(\sigma+\frac{1}{2}\right)^2+\left(\sigma+\frac{1}{2}\right)}\Ket{j+\frac{1}{2},\sigma-\frac{1}{2}} \\
=&\sqrt{(j-\sigma+1)(j+\sigma+1)}\Ket{j+\frac{1}{2},\sigma-\frac{1}{2}} \\
=&+\sqrt{\frac{j+\sigma+1}{2j+1}}\sqrt{1}\Ket{\frac{1}{2},-\frac{1}{2}}\ket{j,\sigma} \\
&+\sqrt{\frac{j+\sigma+1}{2j+1}}\sqrt{j(j+1)-\sigma^2+\sigma}\Ket{\frac{1}{2},\frac{1}{2}}\ket{j,\sigma-1} \\
&+\sqrt{\frac{j-\sigma}{2j+1}}\sqrt{j(j+1)-(\sigma+1)^2+(\sigma+1)}\Ket{\frac{1}{2},-\frac{1}{2}}\ket{j,\sigma} \\
=&+\left[\sqrt{\frac{j+\sigma+1}{2j+1}}+\sqrt{\frac{j-\sigma}{2j+1}}\sqrt{j^2+j-\sigma^2-\sigma}\right]\Ket{\frac{1}{2},-\frac{1}{2}}\ket{j,\sigma} \\
&+\sqrt{\frac{j-\sigma}{2j+1}}\sqrt{j^2+j-\sigma^2+\sigma}\Ket{\frac{1}{2},\frac{1}{2}}\ket{j,\sigma-1} \\
=&+\sqrt{\frac{j+\sigma+1}{2j+1}}(j-\sigma+1)\Ket{\frac{1}{2},-\frac{1}{2}}\ket{j,\sigma} \\
&+\sqrt{\frac{(j+\sigma+1)(j+\sigma)(j-\sigma+1)}{2j+1}}\Ket{\frac{1}{2},\frac{1}{2}}\ket{j,\sigma-1} \\
\therefore \quad \Ket{j+\frac{1}{2},(\sigma-1)+\frac{1}{2}}=&\sqrt{\frac{j+(\sigma-1)+1}{2j+1}}\Ket{\frac{1}{2},\frac{1}{2}}\ket{j,\sigma-1} \\
&+\sqrt{\frac{j-(\sigma-1)}{2j+1}}\Ket{\frac{1}{2},-\frac{1}{2}}\ket{j,(\sigma-1)+1}
\end{align*}
となって,証明が完了する.まぁここで使いたいだけだったので,もっと一般的な結果があるはず.)また(25.5.2)から超対称多重項の任意の状態$\ket{p}$は
\begin{align*}
\left[\mr{Q}_a ,\mr{Q}_{\frac{1}{2}}^* \mr{Q}_{-\frac{1}{2}}^*\right]=&\left\{\mr{Q}_{a},\mr{Q}_{\frac{1}{2}}^*\right\}\mr{Q}_{-\frac{1}{2}}^*-\mr{Q}_{\frac{1}{2}}^* \left\{\mr{Q}_{a},\mr{Q}_{-\frac{1}{2}}^*\right\} \\
=&2(\sigma_\mu P^\mu )_{a\frac{1}{2}} \mr{Q}_{-\frac{1}{2}}^*- \mr{Q}_{\frac{1}{2}}^* 2(\sigma_\mu P^\mu)_{a-\frac{1}{2}} \\
=&\mr{Q}_{-\frac{1}{2}}^*2(\sigma_\mu P^\mu )_{a\frac{1}{2}} - \mr{Q}_{\frac{1}{2}}^* 2(\sigma_\mu P^\mu)_{a-\frac{1}{2}} \quad \because [\mr{Q}_a ,P_\mu]=0 \\
\left[\mr{Q}_a ,\mr{Q}_{\frac{1}{2}}^* \mr{Q}_{-\frac{1}{2}}^*\right]\ket{p}=&\left[\mr{Q}_{-\frac{1}{2}}^*2M\delta_{a\frac{1}{2}} -\mr{Q}_{\frac{1}{2}}^* 2M\delta_{a-\frac{1}{2}}\right]\ket{p} \\
=&2M\sum_b e_{ab} \mr{Q}_{b}^* \ket{p}
\end{align*}
となる.(あるいはエレガントに
\begin{align*}
\left[\mr{Q}_a ,\mr{Q}_{\frac{1}{2}}^* \mr{Q}_{-\frac{1}{2}}^*\right]=&\left[\mr{Q}_a ,\frac{1}{2}e_{bc}\mr{Q}_{b}^* \mr{Q}_{c}^*\right] \\
=&\frac{1}{2}e_{bc}\left\{\mr{Q}_{a},\mr{Q}_{b}^*\right\}\mr{Q}_{c}^*-\frac{1}{2}e_{bc} \mr{Q}_{b}^* \left\{\mr{Q}_{a},\mr{Q}_{c}^*\right\} \\
=&e_{bc} (\sigma_\mu P^\mu)_{ab}\mr{Q}_{c}^*- e_{bc}\mr{Q}_{b}^* (\sigma_\mu P^\mu)_{ac} \\
=&\mr{Q}_{c}^*e_{bc} (\sigma_\mu P^\mu)_{ab}-e_{bc}\mr{Q}_{b}^* (\sigma_\mu P^\mu)_{ac} \quad \because [\mr{Q}_a ,P_\mu]=0 \\
\left[\mr{Q}_a ,\mr{Q}_{\frac{1}{2}}^* \mr{Q}_{-\frac{1}{2}}^*\right]\ket{p}=&M \mr{Q}_{c}^*e_{bc} \delta_{ab}\ket{p}-Me_{bc}\mr{Q}_{b}^* \delta_{ac}\ket{p} \\
=&2M\sum_b e_{ab}\mr{Q}^*_{b}\ket{p}
\end{align*}
とする.別に計算量減らなかったな)したがって,(25.5.7)(25.5.3)から
\begin{align*}
\mr{Q}_a \ket{j,\sigma}^\flat=&\frac{1}{2M} \mr{Q}_a \mr{Q}_{\frac{1}{2}}^* \mr{Q}_{-\frac{1}{2}}^* \ket{j,\sigma} \\
=&\frac{1}{2M}\left[\mr{Q}_a ,\mr{Q}_{\frac{1}{2}}^* \mr{Q}_{-\frac{1}{2}}^*\right]\ket{j,\sigma}+\frac{1}{2M} \mr{Q}_{\frac{1}{2}}^* \mr{Q}_{-\frac{1}{2}}^* \mr{Q}_a \ket{j,\sigma} \\
=&\sum_{b}e_{ab}\mr{Q}_{b}^* \ket{j,\sigma} \quad \because (25.5.3) \\
=&\sqrt{2M}\sum_b e_{ab}\sum_{\pm}C_{\frac{1}{2}j}(j \pm 1/2,\sigma+b;b,\sigma)\ket{j\pm 1/2,\sigma +b} \quad \because (25.5.10)
\end{align*}
となる.(25.5.2)(25.5.3)(25.5.5)から
\begin{align*}
\mr{Q}_{a}\ket{j\pm 1/2 ,\sigma}=&\frac{1}{\sqrt{2M}}\sum_b C_{\frac{1}{2}j }\left(j\pm 1/2 ,\sigma ;b ,\sigma- b\right)\mr{Q}_a \mr{Q}_{b}^*\ket{j,\sigma-b} \\
=&\frac{1}{\sqrt{2M}}\sum_b C_{\frac{1}{2}j }\left(j\pm 1/2 ,\sigma ;b ,\sigma- b\right)\{\mr{Q}_a ,\mr{Q}_{b}^*\}\ket{j,\sigma-b} \quad \because (25.5.3)\\
=&\sqrt{2M}C_{\frac{1}{2}j }\left(j\pm 1/2 ,\sigma ;a ,\sigma- a\right) \ket{j,\sigma-a}
\end{align*}
が得られる.一方(25.5.5)(25.2.31)(25.5.7)から
\begin{align*}
\mr{Q}_{\frac{1}{2}}^*\ket{j\pm 1/2 ,\sigma}=&\frac{1}{\sqrt{2M}}\sum_b C_{\frac{1}{2}j }\left(j\pm 1/2 ,\sigma ;b ,\sigma- b\right)\mr{Q}^*_{\frac{1}{2}} \mr{Q}_{b}^*\ket{j,\sigma-b} \\
=&\frac{1}{\sqrt{2M}} C_{\frac{1}{2}j }\left(j\pm 1/2 ,\sigma ;-1/2 ,\sigma+ 1/2 \right)\mr{Q}^*_{\frac{1}{2}} \mr{Q}_{-\frac{1}{2}}^*\ket{j,\sigma+1/2} \\
=&\sqrt{2M} C_{\frac{1}{2}j }\left(j\pm 1/2 ,\sigma ;-1/2 ,\sigma+ 1/2 \right) \ket{j,\sigma+1/2}^\flat \\
=&\sqrt{2M} \sum_{b}e_{\frac{1}{2}b}C_{\frac{1}{2}j }\left(j\pm 1/2 ,\sigma ;b ,\sigma- b \right) \ket{j,\sigma-b}^\flat \\
\mr{Q}_{-\frac{1}{2}^*}\ket{j\pm 1/2 ,\sigma}=&\frac{1}{\sqrt{2M}}\sum_b C_{\frac{1}{2}j }\left(j\pm 1/2 ,\sigma ;b ,\sigma- b\right)\mr{Q}^*_{-\frac{1}{2}} \mr{Q}_{b}^*\ket{j,\sigma-b} \\
=&\frac{1}{\sqrt{2M}} C_{\frac{1}{2}j }\left(j\pm 1/2 ,\sigma ;1/2 ,\sigma- 1/2 \right)\mr{Q}^*_{-\frac{1}{2}} \mr{Q}_{\frac{1}{2}}^*\ket{j,\sigma-1/2} \\
=&-\sqrt{2M} C_{\frac{1}{2}j }\left(j\pm 1/2 ,\sigma ;1/2 ,\sigma- 1/2 \right) \ket{j,\sigma-1/2}^\flat \\
=&\sqrt{2M} \sum_{b}e_{-\frac{1}{2}b}C_{\frac{1}{2}j }\left(j\pm 1/2 ,\sigma ;b ,\sigma- b \right) \ket{j,\sigma-b}^\flat \\
\therefore \quad \mr{Q}_{a}^*\ket{j\pm 1/2 ,\sigma}=&\sqrt{2M} \sum_{b}e_{ab}C_{\frac{1}{2}j }\left(j\pm 1/2 ,\sigma ;b ,\sigma- b \right) \ket{j,\sigma-b}^\flat
\end{align*}
が得られる.ここまでで,超対称多重項$\ket{j,\sigma},\ket{j\pm1/2,\sigma},\ket{j,\sigma}^\flat$全てへの$\mr{Q}_{a},\mr{Q}^*_a$の作用が全て求まった.\par
$j=0$の場合はつぶれた超対称性多重項が得られる.すなわち,(25.5.3)(25.5.8)(25.5.12)~(25.5.14)は
\begin{align*}
\mr{Q}_{a}\ket{0,0}=&0 ,\quad \mr{Q}^*_a \ket{0,0}^\flat =0 \\
\mr{Q}^*_a \ket{0,0}=&\sqrt{2M}\ket{1/2,a } ,\quad \mr{Q}_a \ket{0,0}^\flat=\sqrt{2M}\sum_b e_{ab}\ket{1/2,b} \\
\mr{Q}_a \ket{1/2,b}=&\sqrt{2M}\delta_{ab} \ket{0,0} ,\quad \mr{Q}_a^*\ket{1/2,b}=\sqrt{2M}e_{ab}\ket{0,0}^\flat
\end{align*}
となる.

\vskip\baselineskip


ここでパリティが保存されていると仮定する.単純超対称性生成子の位相は,これらの演算子へのパリティ演算子が(25.3.11)となるように選ばれていることを思い出そう.すると$\mr{Q}^*_{a}$を$\mathsf{P}\ket{j,\sigma}$に作用させたものは
\begin{align*}
\mr{Q}^*_a \mathsf{P} \ket{j,\sigma}=-i\sum_b e_{ab}\mathsf{P}\mr{Q}_b \ket{j,\sigma}
\end{align*}
となり$\mathsf{P} \mr{Q}_a \ket{j,\sigma}$の線形結合となり,それ(25.5.3)よりゼロだ.よって(25.5.8)と比べて,同じ性質をもち,かつ
\begin{align*}
\exp(i\theta_i J_i)\mathsf{P}\ket{j,\sigma}=&\mathsf{P}\exp(i\theta_i J_i)\ket{j,\sigma} \\
=&\sum_{\sigma'}D_{\sigma \sigma'}^{(j)} \mathsf{P}\ket{j,\sigma'} \\
\exp(i\theta_i J_i)\ket{j,\sigma}^\flat=&\sum_{\sigma'}D_{\sigma \sigma'}^{(j)} \ket{j,\sigma'}^\flat
\end{align*}
となって回転のもとで同じ特性をもっているので,両者は単に比例する.
\begin{align*}
\mathsf{P}\ket{j,\sigma}=-\eta \ket{j,\sigma}^\flat
\end{align*}
$\mathsf{P}$はユニタリーだから規格化条件を満たすように$\eta$は$|\eta|=1$を満たす位相因子だ.同様の議論から$\mathsf{P}\ket{j,\sigma}^\flat$は$\ket{j,\sigma}$に比例することがわかる.その比例係数を知るには,
\begin{align*}
\mathsf{P} \ket{j,\sigma}^\flat =&\frac{1}{2M}\mathsf{P} \mr{Q}^*_{\frac{1}{2}} \mr{Q}^*_{-\frac{1}{2}}\ket{j,\sigma} \\
=&\frac{1}{2M}\mathsf{P} \mr{Q}^*_{\frac{1}{2}}\mathsf{P}^{-1} \mathsf{P} \mr{Q}^*_{-\frac{1}{2}} \mathsf{P}^{-1} \mathsf{P}\ket{j,\sigma} \\
=&\frac{1}{2M}\left(+i e_{\frac{1}{2}b}\mr{Q}_b\right)\left(+i e_{-\frac{1}{2}c}\mr{Q}_c\right)\mathsf{P} \ket{j,\sigma} \\
=&\frac{1}{2M}\mr{Q}_{-\frac{1}{2}}\mr{Q}_{\frac{1}{2}} \mathsf{P} \ket{j,\sigma} \\
=&-\frac{1}{2M}\mr{Q}_{-\frac{1}{2}}\mr{Q}_{\frac{1}{2}} \eta \ket{j,\sigma}^\flat \\
=&-\eta \frac{1}{(2M)^2}\mr{Q}_{-\frac{1}{2}}\mr{Q}_{\frac{1}{2}} \mr{Q}^*_{\frac{1}{2}}\mr{Q}^*_{-\frac{1}{2}}\ket{j,\sigma} \\
=&-\eta\ket{j,\sigma}
\end{align*}
からわかる.最後の変形は(25.5.9)の導出計算を繰り返したらよい.するとスピン$j$で第三成分$\sigma$の状態
\begin{align*}
\ket{j,\sigma}^\pm \equiv \frac{1}{\sqrt{2}}\left(\ket{j,\sigma}\mp \ket{j,\sigma}^\flat\right)
\end{align*}
が作れて,これは
\begin{align*}
\mathsf{P}\ket{j,\sigma}^\pm =& \frac{1}{\sqrt{2}}\left(-\eta \ket{j,\sigma}^\flat \pm \eta \ket{j,\sigma}\right) \\
=&\pm \eta \frac{1}{\sqrt{2}}\left(\ket{j,\sigma}\mp \ket{j,\sigma}^\flat\right) \\
=&\pm \eta \ket{j,\sigma}^\pm
\end{align*}
となってパリティ固有状態になっている.最後にパリティ演算子を(25.5.5)に作用させ(25.3.13)(25.5.16)を使うと
\begin{align*}
\mathsf{P} \ket{j\pm 1/2,\sigma} =&\frac{1}{\sqrt{2M}}\sum_a C_{\frac{1}{2}j }\left(j\pm 1/2 ,\sigma ;a ,\sigma- a\right)\mathsf{P} \mr{Q}_{a}^*\ket{j,\sigma-a} \\
=& \frac{1}{\sqrt{2M}}\sum_a C_{\frac{1}{2}j }\left(j\pm 1/2 ,\sigma ;a ,\sigma- a\right)i\sum_b e_{ab}\mr{Q}_{b}\mathsf{P}\ket{j,\sigma-a} \\
=&-\frac{i\eta}{\sqrt{2M}}\sum_a C_{\frac{1}{2}j }\left(j\pm 1/2 ,\sigma ;a ,\sigma- a\right)\sum_b e_{ab}\ket{j,\sigma-a}^\flat
\end{align*}
となる.(25.5.12)から
\begin{align*}
\mathsf{P} \ket{j\pm 1/2,\sigma} =&-\frac{i\eta}{\sqrt{2M}}\sum_a C_{\frac{1}{2}j }\left(j\pm 1/2 ,\sigma ;a ,\sigma- a\right)\\
&\quad \times \sum_b e_{ab} \sqrt{2M}\sum_c e_{bc} \sum_{\pm}C_{\frac{1}{2}j}\left(j\pm 1/2,\sigma-a+c;c ,\sigma-a \right)\ket{j\pm 1/2 ,\sigma-a+c} \\
=&i\eta\sum_a C_{\frac{1}{2}j }\left(j\pm 1/2 ,\sigma ;a ,\sigma- a\right) \\
& \quad \times \sum_{\pm}C_{\frac{1}{2}j}\left(j\pm 1/2,\sigma;a ,\sigma-a \right)\ket{j\pm 1/2 ,\sigma} 
\end{align*}
(記号の乱用になっているが,右辺最後の$\pm$についての和は左辺の$\pm$とは無関係であることに注意.)クレブシュ・ゴルダン係数の正規直交性
\begin{align*}
\sum_{a=-A}^A C_{AB}(J,M;a,M-a)C_{AB}(J',M;a,M-a)=&\delta_{JJ'}
\end{align*}
を使えば(混乱するので$j+1/2$のときと$j-1/2$で場合分けして)
\begin{align*}
\mathsf{P} \ket{j+ 1/2,\sigma} =&i\eta  \sum_{\pm} \left[ \sum_a C_{\frac{1}{2}j }\left(j+ 1/2 ,\sigma ;a ,\sigma- a\right)C_{\frac{1}{2}j}\left(j\pm 1/2,\sigma;a ,\sigma-a \right)\right]\ket{j\pm 1/2 ,\sigma} \\
=&i\eta \ket{j+ 1/2 ,\sigma} \\
\mathsf{P} \ket{j- 1/2,\sigma} =&i\eta  \sum_{\pm} \left[ \sum_a C_{\frac{1}{2}j }\left(j- 1/2 ,\sigma ;a ,\sigma- a\right)C_{\frac{1}{2}j}\left(j\pm 1/2,\sigma;a ,\sigma-a \right)\right]\ket{j\pm 1/2 ,\sigma} \\
=&i\eta \ket{j- 1/2 ,\sigma}
\end{align*}
よって
\begin{align*}
\mathsf{P} \ket{j\pm 1/2,\sigma} =i\eta \ket{j\pm 1/2,\sigma} 
\end{align*}
が得られる.(どっちの結果も本文と違うけど,誤植か?)

\vskip\baselineskip

次に$N$個の超対称性生成子をもつ拡張超対称性の場合を簡単に見る.前節で述べたように,質量ゼロ状態に中心電荷が作用すると必ず消えてしまい,したがって中心電荷のどれかについて固有値がゼロでない質量ゼロ粒子は存在できないのだった.ここからさらに進んで,中心電荷演算子の固有値は,任意の超対称多重項の\uwave{質量に下限を与える}ことが示せる.(25.2.10)(25.2.11)より,中心電荷$Z_{rs}$と$Z_{rs}^*$は互いに交換し$P_\mu$とも交換するので,1粒子状態は全ての中心電荷および$P_\mu$の固有状態にとることができる.また中心電荷は$\mr{Q}_{ar},\mr{Q}_{ar}^*$と交換するので,超対称多重項の全ての状態は同じ固有値をもつ.\par
超対称多重項の質量$M$と,この超対称多重項の中心電荷の固有値を関係付ける不等式を導くために,反交換関係(25.2.7)(25.2.8)を使って
\begin{align*}
&\sum_{ar}\left\{ \left( \mr{Q}_{ar}-\sum_{bs}e_{ab} U_{rs} \mr{Q}^*_{bs} \right) , \left( \mr{Q}^*_{ar}-\sum_{ct}e_{ac} U^*_{rt} \mr{Q}_{ct} \right) \right\} \\
=&\sum_{ar}\left\{\mr{Q}_{ar}, \mr{Q}^*_{ar}\right\}-\sum_{abrs}e_{ab}U_{rs}\left\{ \mr{Q}_{bs}^*,\mr{Q}_{ar}^* \right\}-\sum_{acrt}e_{ac}U_{rt}^*\left\{ \mr{Q}_{ar},\mr{Q}_{ct} \right\}  \\
&+\sum_{ar}\sum_{bs}\sum_{ct}e_{ab}e_{ac}U_{rs}U^\dagger_{tr}\left\{\mr{Q}_{bs}^*, \mr{Q}_{ct}\right\} \\
=&\sum_{ar}\left\{\mr{Q}_{ar}, \mr{Q}^*_{ar}\right\}-\sum_{abrs}e_{ab}U_{rs}\left\{ \mr{Q}_{bs}^*,\mr{Q}_{ar}^* \right\}-\sum_{acrt}e_{ac}U_{rt}^*\left\{ \mr{Q}_{ar},\mr{Q}_{ct} \right\} \\
&+\sum_{bs}\sum_{ct}\delta_{ac}\delta_{st}\left\{\mr{Q}_{bs}^*, \mr{Q}_{ct}\right\} \\
=&\sum_{ar}\left\{\mr{Q}_{ar}, \mr{Q}^*_{ar}\right\}-\sum_{abrs}e_{ab}U_{rs}\left\{ \mr{Q}^*_{bs},\mr{Q}_{ar}^* \right\}-\sum_{acrt}e_{ac}U_{rt}^*\left\{ \mr{Q}_{ar},\mr{Q}_{ct} \right\} \\
&+\sum_{ar}\left\{\mr{Q}_{ar}^*, \mr{Q}^*_{ar}\right\} \\
=&2\sum_{ar}\left\{\mr{Q}_{ar}, \mr{Q}^*_{ar}\right\}-\sum_{abrs}e_{ab}U_{rs}\left\{ \mr{Q}^*_{bs},\mr{Q}_{ar}^* \right\}-\sum_{acrt}e_{ac}U_{rt}^*\left\{ \mr{Q}_{ar},\mr{Q}_{ct} \right\} \\
=&4\sum_{ar}(\sigma_\mu P^\mu)_{aa}\delta_{rr}-\sum_{abrs}e_{ab}U_{rs}e_{ba}Z_{sr}^*-\sum_{acrt}e_{ac}U_{rt}^*e_{ac}Z_{rt} \\
=&8NP^0-2\sum_{rs}U_{rs}Z_{rs}^*-2\sum_{rs}U_{tr}^\dagger Z_{rt} \\
=&8NP^0 -2\mr{Tr}(ZU^\dagger + UZ^\dagger)
\end{align*}
と書ける.ここで$U_{rs}$は任意の$N\times N$ユニタリー行列だ.(演算子のダガーはアスタリスクの意味と同じだったが,ここでは$Z_{rs}$の$rs$に関する転置を含めるとした.)左辺は$\{A,A^*\}$の形の式だから,必ず正定値な演算子であり,静止した超対称多重項の状態$\ket{p}$にこれを作用させて
\begin{align*}
0<&\bra{p}\sum_{ar}\left\{ \left( \mr{Q}_{ar}+\sum_{bs}e_{ab} U_{rs} \mr{Q}^*_{bs} \right) , \left( \mr{Q}^*_{ar}+\sum_{ct}e_{ac} U^*_{rt} \mr{Q}_{ct} \right) \right\}\ket{p} \\
=&\bra{p}\left[8NP^0 -2\mr{Tr}(ZU^\dagger + UZ^\dagger)\right]\ket{p} \\
=&8NM-2\mr{Tr}(ZU^\dagger + UZ^\dagger) \\
\therefore \quad M \geq& \frac{1}{4N}\mr{Tr}(ZU^\dagger + UZ^\dagger)
\end{align*}
を得る.ここで最後の$Z_{rs}$は質量$M$の超対称多重項の中心電荷$Z_{rs}$の\uwave{値}を表す.つまり記号が被らないように演算子である中心電荷の方を$\mc{Z}_{rs}$と書くと
\begin{align*}
\mc{Z}_{rs}\ket{p}=&Z_{rs}\ket{p} \\
\mc{Z}_{rs}^\dagger\ket{p}=&\mc{Z}_{sr}^*\ket{p}=Z^*_{sr}\ket{p}=Z^\dagger_{rs}\ket{p}
\end{align*}
としたときの行列値$Z_{rs}$だ.極分解定理より任意の正方行列$Z$は正エルミート行列$H$とユニタリー行列$V$で$Z=HV$と書ける.(2.7節参照)任意の行列$U$を$U=V$とすれば,
\begin{align*}
M\geq& \frac{1}{4N}\mr{Tr}(HVV^\dagger +V V^\dagger H^\dagger ) \\
=&\frac{1}{2N}\mr{Tr}H \\
=&\frac{1}{2N}\mr{Tr}\sqrt{Z^\dagger Z}
\end{align*}
と書ける.$M$がこの不等式で許される最小値に等しい状態は,23.3節で議論したボゴモルニ・プラサド・ゾンマーフェルトの磁気単極子との類推から\textbf{BPS状態}と呼ばれる.\par
(25.5.22)のこの導出からわかるように,BPS超対称多重項の場合には演算子$\mr{Q}_{ar}-\sum_{bs}e_{ab}U_{rs}\mr{Q}^*_{bs}$はこの超対称多重項の任意の状態に作用したときにゼロになる.(不等式が等式になるときであるから,$\bra{p}\{A,A^*\}\ket{p}=0$となる場合であり,よって$A=0$.)したがって$N$個の独立なスピン第三成分を下げる演算子$\mr{Q}_{\frac{1}{2}r}$と,$N$個の独立なスピン第三成分を上げる演算子$\mr{Q}_{-\frac{1}{2}r}$が存在する.($\mr{Q}_{-\frac{1}{2}r}^*$や$\mr{Q}_{\frac{1}{2}}^*$はスピン第三成分を下げ上げするが,線形結合$\mr{Q}_{ar}-\sum_{bs}e_{ab}U_{rs}\mr{Q}^*_{bs}$が任意の状態に対してゼロになってしまうから,これらと独立な演算子ではない.)この結果,一般の場合に見られるより小さな超対称多重項が導かれる.(もしBPS状態でなければ$\mr{Q}_{-\frac{1}{2}r}^*$や$\mr{Q}_{\frac{1}{2}}^*$も独立に存在でき,スピン第三成分をもっと下げることができるため)\par
例えば,$N=2$超対称性の場合には,中心電荷の行列固有値は
\begin{align*}
Z=\left(
\begin{matrix}
0 & Z_{12} \\
-Z_{12} & 0 
\end{matrix}
\right)
\end{align*}
となるが,これは一個の複素数$Z_{12}$で決まる.極分解は
\begin{align*}
V=\left(
\begin{matrix}
0 & \frac{Z_{12}}{|Z_{12}|} \\
-\frac{Z_{12}}{|Z_{12}|} & 0
\end{matrix}
\right) ,\quad H= |Z_{12}|I
\end{align*}
となる.不等式(25.5.22)はこの場合
\begin{align*}
\sqrt{Z^\dagger Z}=&\sqrt{Z_{12}^2I}=|Z_{12}|I \\
M\geq & \frac{1}{4}\mr{Tr}(|Z_{12}|I)=\frac{|Z_{12}|}{2}
\end{align*}
となる.BPS状態の任意の超対称多重項に作用するとゼロになる演算子は
\begin{align*}
\mr{Q}_{a1}-\sum_{bs}e_{ab}V_{1s}\mr{Q}_{bs}^* =&\mr{Q}_{a1}-\frac{Z_{12}}{|Z_{12}|}(e\mr{Q}^*)_{a2} \\
\mr{Q}_{a2}-\sum_{bs}e_{ab}V_{2s}\mr{Q}_{bs}^* =&\mr{Q}_{a2}+\frac{Z_{12}}{|Z_{12}|}(e\mr{Q}^*)_{a1}
\end{align*}
となる.$M=|Z_{12}|/2$のBPS状態のとき,質量がゼロでない超対称多重項のヘリシティ成分は質量ゼロ超対称多重項の成分と同じだ.すなわち,(作り方は質量ゼロのときと同じで,スピンが最大のところからだんだん$\mr{Q}_{\frac{1}{2}r}$を作用させていく)\par
スピン1粒子が1個\par
スピン$1/2$粒子が2個(1個の$SU(2)$R対称性2重項を作る)\par
スピン$0$粒子が1個\par
\noindent からなるゲージ超対称多重項(質量ゼロの方ではヘリシティ0粒子が2個だったので1個少なく見えるが,質量のあるスピン1粒子はヘリシティ$+1,0,-1$が存在するため,ここのヘリシティ0と合わせて勘定が合う.)および\par
スピン1/2粒子が1個(第三成分$+1/2$と,$-1/2$の分を合わせて) \par
スピン0粒子が2個(1個の$SU(2)$R対称性2重項を作る)\par
\noindent からなるハイパー多重項が存在する.これらは$M> |Z_{12}|/2$のときに現れるもっと大きな超対称多重項と区別するために「小さい」超対称多重項と呼ばれることがある.


\newpage

\part{超対称場の理論}
\setcounter{section}{25}
\setcounter{subsection}{0}
\subsection{場の超対称多重項の直接的構成}
場の超対称多重項の直接的な構成法を見るために,25.5節で述べた最も単純な超対称多重項($j=0$から作られる潰れたもの)に属する任意の質量の粒子を消滅させる場を考えよう.この多重項は2つのスピン・ゼロ粒子と1つのスピン$1/2$粒子からなっているのだった.(25.5.15)でスピン・ゼロの1粒子状態$\ket{0,0}$は超対称性の生成子$\mr{Q}_a$で消滅させられるが,$\mr{Q}_{a}^*$では消滅させられないことを見たのだった.したがって,この粒子を真空$\ket{v}$(この真空は全ての超対称性の生成子で消滅させられるとする)から作るスカラー場$\phi(x)$は$\mr{Q}_a$と可換だが$\mr{Q}_a^*$とは可換ではないと思われる.すなわち
\begin{align*}
\mr{Q}_a \phi(x)\ket{v}\propto & \mr{Q}_{a} \ket{0,0}=0,\quad \phi(x)\mr{Q}_a \ket{v}=0 \\
\therefore \quad  [\mr{Q}_{a}, \phi(x)]=&0 \\
\mr{Q}_a^* \phi(x) \ket{v}\propto& \mr{Q}_{a}^* \ket{0,0} \neq 0 ,\quad \phi(x)\mr{Q}_a^* \ket{v}=0 \\
\therefore \quad -i\sum_b e_{ab}[\mr{Q}^*_a ,\phi(x)] \equiv &\zeta_a (x) \neq 0
\end{align*}
が成立している.ここで反対称な$2\times 2$行列$e_{ab}$を導入したが,これは斉次ローレンツ群のもとで$(1/2,0)$表現として変換されるのが$\sum_b e_{ab} \mr{Q}_{b}^*$だからだ.これにより$\zeta_a(x)$も斉次ローレンツ群の$(1/2,0)$表現に属する\uwave{フェルミオン的な}2成分スピノル場だとわかる.\par
(26.1.1)(26.1.2)と反交換関係(25.2.31)から以下が分かる.
\begin{align*}
\left\{\mr{Q}_b ,\zeta_a\right\}=&-i\sum_c e_{ac} \{ \mr{Q}_b ,[\mr{Q}_{c}^* ,\phi] \} \\
=&+i\sum_c e_{ac}[\phi,\{\mr{Q}_b ,\mr{Q}_c^*\}]-i \sum_c e_{ac}\{\mr{Q}_{c}^*,[\phi,\mr{Q}_b]\} \quad \because ヤコビ恒等式 \\
=&-i\sum_c e_{ac}[\{\mr{Q}_b ,\mr{Q}_c^*\},\phi] \quad \because (26.1.1) \\
=& -2i\sum_c e_{ac}[ (\sigma_\mu P^\mu)_{bc},\phi] \quad \because (25.2.31) \\
=& 2i(\sigma_\mu e )_{ba}[P^\mu ,\phi]
\end{align*}
したがって(10.1.1)より
\begin{align*}
\left\{\mr{Q}_b ,\zeta_a(x)\right\}=-2(\sigma^\mu e )_{ba}\partial_\mu \phi(x)
\end{align*}
となる.一方(26.1.2)と反交換関係(25.2.32)から
\begin{align*}
-i\sum_c e_{ac} \left\{\mr{Q}^*_b , \zeta_c \right\}=& -\sum_{cd} e_{ac} e_{cd}\left\{\mr{Q}_b^* ,[\mr{Q}_{d}^*,\phi] \right\} \\
=&\left\{\mr{Q}_b^* ,[\mr{Q}_{a}^*,\phi] \right\} \\
=&-\left\{\mr{Q}_a^* ,[\mr{Q}_{b}^*,\phi] \right\}+\left[\phi ,\{\mr{Q}_{b}^*,\mr{Q}_{a}^*\} \right] \quad \because ヤコビ恒等式 \\
=&-\left\{\mr{Q}_a^* ,[\mr{Q}_{b}^*,\phi] \right\} \quad \because (25.2.32) \\
=&+i\sum_c e_{bc} \left\{\mr{Q}^*_a , \zeta_c \right\} \quad (最初の計算を逆にするだけ)
\end{align*}
だから,$\sum_c e_{ac}\{\mr{Q}_b^*,\zeta_c \}$は$a,b$に関して反対称であり,これより反対称$2\times 2$行列$e_{ab}$に比例することが分かる.
\begin{align*}
i \sum_c e_{ac}\{\mr{Q}_b^*,\zeta_c \}=&2e_{ab}\mc{F}\\
\therefore \quad i\{\mr{Q}^*_b ,\zeta_a(x)\}=&2\delta_{ab} \mc{F}(x)
\end{align*}
また,ローレンツ不変性から係数$\mc{F}(x)$はスカラー場だとわかる.\par
ここで更に,超対称性生成子と$\mc{F}(x)$の交換子も計算する必要がある.(26.1.4)(26.1.2)(25.2.32)を使うと
\begin{align*}
\delta_{ab} \left[\mr{Q}^*_c ,\mc{F}\right]=&\frac{1}{2}i \left[\mr{Q}^*_{c} ,\{\mr{Q}^*_b ,\zeta_a \}\right] \\
=&-\frac{1}{2}i \left[\zeta_a, \{\mr{Q}_c^*,\mr{Q}_b^*\}\right]-\frac{1}{2}i \left[\mr{Q}_b^*,\{\zeta_a ,\mr{Q}_c^*\}\right] \quad \because ヤコビ恒等式\\
=&-\frac{1}{2}i \left[\mr{Q}_b^*,\{\zeta_a ,\mr{Q}_c^*\}\right] \\
=&-\delta_{ac}[\mr{Q}_{b}^*,\mc{F}]
\end{align*}
を得る.$a=b\neq c$とおけば,この交換子がゼロになることがわかる.
\begin{align*}
[\mr{Q}_{c}^*,\mc{F}(x)]=0
\end{align*}
最後に(26.1.4)(25.2.31)(26.1.3)を使うと
\begin{align*}
\delta_{ab}[\mr{Q}_c ,\mc{F}]=&\frac{1}{2}i\left[\mr{Q}_c ,\left\{\mr{Q}^*_{b},\zeta_a\right\}\right] \\
=&-\frac{1}{2}i\left[\zeta_a ,\left\{\mr{Q}_{c},\mr{Q}^*_{b}\right\}\right]-\frac{1}{2}i\left[\mr{Q}_b^* ,\left\{\zeta_a, \mr{Q}_{c}\right\}\right] \quad \because ヤコビ恒等式 \\
=&-i(\sigma_\mu )_{cb}\left[\zeta_a ,P^\mu\right] +i(\sigma^\mu e)_{ca}[\mr{Q}_{b}^* ,\partial_\mu \phi] \quad \because (25.2.31)(26.1.3) \\
=&-(\sigma^\mu )_{cb} \partial_\mu \zeta_a+i(\sigma^\mu e)_{ca}\partial_\mu [\mr{Q}_{b}^* , \phi] \\
=&-(\sigma^\mu )_{cb} \partial_\mu \zeta_a+\sum_d e_{bd} (\sigma^\mu e)_{ca}\partial_\mu \zeta_d \quad \because(26.1.2)
\end{align*}
となる.両辺に$\delta_{ab}$と縮約させて
\begin{align*}
2[\mr{Q}_c ,\mc{F}]=&-\sum_a (\sigma^\mu )_{ca} \partial_\mu \zeta_a+\sum_d (\sigma^\mu e e)_{cd}\partial_\mu \zeta_d \\
=&-2\sum_a (\sigma^\mu )_{ca} \partial_\mu \zeta_a \\
\therefore \quad [\mr{Q}_c ,\mc{F}(x)]=&-\sum_a (\sigma^\mu )_{ca} \partial_\mu \zeta_a(x)
\end{align*}
となる.式(26.1.1)~(26.1.6)は,場$\phi(x),\zeta_a(x),\mc{F}(x)$が超対称代数の完全な表現をなすことを意味する.
\begin{align*}
[\mr{Q}_{a}, \phi(x)]=&0 \\
\left\{\mr{Q}_b ,\zeta_a(x)\right\}=&-2(\sigma^\mu e )_{ba}\partial_\mu \phi(x) \\
[\mr{Q}_c ,\mc{F}(x)]=&-\sum_a (\sigma^\mu )_{ca} \partial_\mu \zeta_a(x) \\
[\mr{Q}_{a}^*, \phi(x)]=&-i\sum_b e_{ab}\zeta_b(x) \\
\{\mr{Q}^*_b ,\zeta_a(x)\}=&-2i\delta_{ab} \mc{F}(x) \\
[\mr{Q}_{c}^*,\mc{F}(x)]=0
\end{align*}
これらの場はエルミートではないので,それらの複素共役は別の超対称多重項を与える.
\begin{align*}
[\mr{Q}^*_{a}, \phi^*(x)]=&0 \\
\left\{\mr{Q}^*_b ,\zeta_a^*(x)\right\}=&2(e \sigma^\mu )_{ab}\partial_\mu \phi^*(x) \\
[\mr{Q}^*_c ,\mc{F}^*(x)]=&\sum_a (\sigma^\mu )_{ac} \partial_\mu \zeta_a^*(x) \\
[\mr{Q}_{a}, \phi^*(x)]=&-i\sum_b e_{ab}\zeta_b^*(x) \\
\{\mr{Q}_b ,\zeta_a^*(x)\}=&2i\delta_{ab} \mc{F}^*(x) \\
[\mr{Q}_{c},\mc{F}^*(x)]=0
\end{align*}
これらの交換・反交換関係は,(超対称性生成子を文字通り超対称性変換の生成子として)超対称性変換のもとでの変換則として表すことができる.この超対称性変換は,任意のボゾン場またはフェルミオン場の演算子$\mc{O}(x)$を微小量
\begin{align*}
\delta \mc{O}(x)\equiv \left[\sum_a (\epsilon_a^* \mr{Q}_a +\epsilon_a \mr{Q}^*_a),\mc{O}(x)\right]
\end{align*}
だけずらす.ここで$\epsilon_a$は微小な\uwave{フェルミオン的}c数スピノルだ.よって$\epsilon_a ,\epsilon^*_a$は$\mr{Q}_a,\mr{Q}_{a}^*$と反可換だから,$\epsilon^*_a \mr{Q}_{a}+\epsilon_a \mr{Q}^*_a$という量は
\begin{align*}
\left(\epsilon^*_a \mr{Q}_{a}+\epsilon_a \mr{Q}^*_a\right)^*=&\mr{Q}_a^*\epsilon_a +\mr{Q}_a \epsilon_a^* \quad \because(25.1.19)\\
=&-(\epsilon^*_a \mr{Q}_{a}+\epsilon_a \mr{Q}^*_a)
\end{align*}
だから反エルミートとなる.したがって
\begin{align*}
\left(\delta\mc{O}\right)^*=&\left[\sum_a (\epsilon_a^* \mr{Q}_a +\epsilon_a \mr{Q}^*_a),\mc{O}\right]^* \\
=&-\left[\sum_a (\epsilon_a^* \mr{Q}_a +\epsilon_a \mr{Q}^*_a)^*,\mc{O}^*\right] \\
=&\left[\sum_a (\epsilon_a^* \mr{Q}_a +\epsilon_a \mr{Q}^*_a),\mc{O}^*\right]=\delta (\mc{O}^*)
\end{align*}
となる.エルミートと超対称性変換は可換となってくれて嬉しい.この定義により
\begin{align*}
\delta \phi(x)=&\left[\sum_a (\epsilon_a^* \mr{Q}_a +\epsilon_a \mr{Q}^*_a),\phi(x)\right] \\
=&\sum_a\Bigl(\epsilon_a^* [\mr{Q}_a,\phi(x)] +\epsilon_a [\mr{Q}^*_a,\phi(x)]\Bigr) \\
=&-i\sum_{ab}\epsilon_a e_{ab} \zeta_b(x) \\
\delta \zeta_a (x)=&\left[\sum_b (\epsilon_b^* \mr{Q}_b +\epsilon_b \mr{Q}^*_b),\zeta_a(x)\right] \\
=&\sum_b\Bigl(\epsilon_b^* \{\mr{Q}_b,\zeta_a(x)\} +\epsilon_b \{\mr{Q}^*_b,\zeta_a(x)\}\Bigr) \\
=&-2\sum_b \epsilon^*_b (\sigma_\mu e)_{ba}\partial^\mu \phi(x)-2i\epsilon_a \mc{F}(x) \\
=&-2\sum_b (\sigma_\mu e)^T_{ab} \epsilon^*_b \partial^\mu \phi(x)-2i\epsilon_a \mc{F}(x) \\
=&+2\sum_b (e\sigma_\mu^T)_{ab} \epsilon^*_b \partial^\mu \phi(x)-2i\epsilon_a \mc{F}(x) \\
\delta\mc{F}(x) =&\left[\sum_a (\epsilon_a^* \mr{Q}_a +\epsilon_a \mr{Q}^*_a),\mc{F}(x)\right] \\
=&\sum_a\Bigl(\epsilon_a^* [\mr{Q}_a,\mc{F}(x)] +\epsilon_a [\mr{Q}^*_a,\mc{F}(x)]\Bigr) \\
=&-\sum_{ab}\epsilon_b^* (\sigma_\mu)_{ba}\partial^\mu \zeta_a(x)
\end{align*}
となる.\par
これは微小なマヨラナ4成分スピノル変換パラメータ
\begin{align*}
\alpha \equiv -i \left(
\begin{matrix}
\epsilon \\
e\epsilon^*
\end{matrix}
\right)
\end{align*}
を導入するとこれは実際にマヨラナ性
\begin{align*}
\alpha^*=+i\left(
\begin{matrix}
\epsilon^* \\
e\epsilon
\end{matrix}
\right)=+i\left(
\begin{matrix}
0 & -e \\
e & 0
\end{matrix}
\right)\left(
\begin{matrix}
\epsilon \\
e\epsilon^*
\end{matrix}
\right)=-\beta \epsilon \gamma_5 \alpha
\end{align*}
を満たしていて,
\begin{align*}
\bar{\alpha}Q=+i(\epsilon^{*T} ,-\epsilon^T e)\left(
\begin{matrix}
0 & 1 \\
1 & 0
\end{matrix}
\right)\left(
\begin{matrix}
e\mr{Q}^* \\
\mr{Q}
\end{matrix}
\right) =i\sum_a (\epsilon_a^* \mr{Q}_a +\epsilon_a \mr{Q}^*_a)
\end{align*}
となるから
\begin{align*}
i\delta \mc{O}(x) \equiv [\bar{\alpha} Q ,\mc{O}(x)]
\end{align*}
と4成分ディラック記法で書くことができる.\par
変換則(26.1.14)~(26.1.16)とそれらの複素共役は,実のボゾン場$A,B,F,G$と4成分マヨラナ・スピノル場を導入すると便利な共変形で書くことができる.これらの実ボゾン場は
\begin{align*}
\frac{A+iB}{\sqrt{2}} \equiv \phi,\quad \frac{F-iG}{\sqrt{2}}\equiv \mc{F} \\
A=\frac{\phi +\phi^*}{\sqrt{2}},\quad B=-i\frac{\phi -\phi^*}{\sqrt{2}} \\
F=\frac{\mc{F} +\mc{F}^*}{\sqrt{2}},\quad G=i\frac{\mc{F} -\mc{F}^*}{\sqrt{2}}
\end{align*}
で定義され(つまり$\phi,\mc{F}$はエルミートでなかったから,その実部と虚部を別々に実場として定義する),マヨラナ・スピノル場は
\begin{align*}
\psi \equiv \frac{1}{\sqrt{2}}\left(
\begin{matrix}
 \zeta \\
 -e\zeta^*
\end{matrix}
\right)
\end{align*}
で定義される.$4\times 4$ディラック行列と$2\times 2$行列$\sigma_\mu$の関係
\begin{align*}
\gamma_\mu =\left(
\begin{matrix}
0 & -ie \sigma^T_\mu e \\
i\sigma_\mu & 0
\end{matrix}
\right)
\end{align*}
を思い出そう.(これは(25.2.36)の導出のときに一応使った性質.まぁ確かめるにしても$\mu=0\sim3$でそれぞれ確かめればすぐわかる.$\gamma^\mu$ではなく下付きなので$\mu=0$のときだけマイナスつきなので注意.)エルミート場の変換則
\begin{align*}
\delta \phi^*(x)=&+i\sum_{ab}e_{ab} \zeta_b^*(x)\epsilon_a^* \\
=&-i\sum_{ab}\epsilon_a^*e_{ab}\zeta_b^*(x) \\
\delta \zeta_a^* (x)=&+2\sum_b (e\sigma_\mu^T)_{ab}^* \epsilon_b \partial^\mu \phi^*(x)+2i\epsilon^*_a \mc{F}^*(x) \\
=&+2\sum_b (e \sigma_\mu)_{ab}\epsilon_b\partial^\mu \phi^*(x)+2i\epsilon^*_a \mc{F}^*(x) \\
\delta\mc{F}^*(x) =&-\sum_{ab}(\sigma^\mu)^*_{ba}\partial_\mu \zeta_a^*(x)\epsilon_b \\
=&+\sum_{ab}\epsilon_b (\sigma^\mu)^T_{ba}\partial_\mu \zeta_a^*(x)
\end{align*}
と
\begin{align*}
\gamma^\mu \alpha =&-i\left(
\begin{matrix}
0 & -ie \sigma^T_\mu e \\
i\sigma_\mu & 0
\end{matrix}
\right)\left(
\begin{matrix}
\epsilon \\
e\epsilon^*
\end{matrix}
\right)=\left(
\begin{matrix}
e\sigma_\mu^T\epsilon^* \\
\sigma_\mu\epsilon
\end{matrix}
\right) \\
\bar{\alpha}\gamma^\mu =&+i(\epsilon^{*T},-\epsilon^T e)\left(
\begin{matrix}
0 & 1 \\
1 & 0
\end{matrix}
\right)\left(
\begin{matrix}
0 & -ie \sigma^T_\mu e \\
i\sigma_\mu & 0
\end{matrix}
\right) =\left(-\epsilon^{*T} \sigma_\mu ,+\epsilon^T\sigma^T_\mu e \right)
\end{align*}
も使って,それぞれの変換則が
\begin{align*}
\delta A=&\frac{1}{\sqrt{2}}\delta \phi + \frac{1}{\sqrt{2}}\delta\phi^* \\
=&-i\frac{1}{\sqrt{2}}\sum_{ab} \epsilon_a e_{ab}\zeta_b -i \frac{1}{\sqrt{2}}\sum_{ab}\epsilon^*_a e_{ab} \zeta_b^* \\
=& +i(\epsilon^{*T} ,-\epsilon^T e)\left(
\begin{matrix}
0 & 1 \\
1 & 0
\end{matrix}
\right)\frac{1}{\sqrt{2}}\left(
\begin{matrix}
\zeta \\
-e\zeta^*
\end{matrix}
\right) \\
=&\bar{\alpha}\psi \\
\delta B=&-i \frac{1}{\sqrt{2}}\delta \phi+i\frac{1}{\sqrt{2}}\delta \phi^* \\
=&-\frac{1}{\sqrt{2}}\sum_{ab} \epsilon_a e_{ab}\zeta_b + \frac{1}{\sqrt{2}}\sum_{ab}\epsilon^*_a e_{ab} \zeta_b^* \\
=& (-i)i(\epsilon^{*T} ,-\epsilon^T e)\left(
\begin{matrix}
0 & 1 \\
1 & 0
\end{matrix}
\right)\left(
\begin{matrix}
1 & 0 \\
0 & -1
\end{matrix}
\right)\frac{1}{\sqrt{2}}\left(
\begin{matrix}
\zeta \\
-e\zeta^*
\end{matrix}
\right) \\
=&-i \bar{\alpha}\gamma_5 \psi \\
\delta \psi=&\frac{1}{\sqrt{2}}\left(
\begin{matrix}
\delta \zeta \\
 -e\delta\zeta^*
\end{matrix}
\right) \\
=&\frac{1}{\sqrt{2}}\left(
\begin{matrix}
 +2 e\sigma_\mu^T\epsilon^* \partial^\mu \phi -2i\epsilon \mc{F} \\
 +2\sigma_\mu \epsilon \partial^\mu \phi^*-2ie\epsilon^* \mc{F}^*
\end{matrix}
\right) \\
=&\left(
\begin{matrix}
 + e \sigma_\mu^T\epsilon^* \partial^\mu A  \\
 +\sigma_\mu \epsilon \partial^\mu A
\end{matrix}
\right)+\left(
\begin{matrix}
 + ie \sigma_\mu^T\epsilon^* \partial^\mu B  \\
 -i\sigma_\mu \epsilon \partial^\mu B
\end{matrix}
\right) \\
&+\left(
\begin{matrix}
-i\epsilon F \\
-ie\epsilon^* F
\end{matrix}
\right)+\left(
\begin{matrix}
-\epsilon G \\
+e\epsilon^* G
\end{matrix}
\right) \\
=&\partial^\mu \gamma_\mu\alpha +\partial^\mu B \gamma_5 \gamma_\mu \alpha +F \alpha -iG\gamma_5 \alpha \\
=&\partial_\mu(A+i\gamma_5 B)\gamma^\mu \alpha +(F-i\gamma_5 G)\alpha \\
\delta F=& \frac{1}{\sqrt{2}}\delta \mc{F}+\frac{1}{\sqrt{2}}\delta \mc{F}^* \\
=&-\frac{1}{\sqrt{2}}\sum_{ab}\epsilon^*_a (\sigma_\mu)_{ab}\partial^\mu \zeta_a +\frac{1}{\sqrt{2}}\sum_{ab}\epsilon_a(\sigma_\mu)^T_{ab}\partial^\mu \zeta^*_b \\
=&\left( -\epsilon^{*T} \sigma_\mu ,+\epsilon^T\sigma^T_\mu e \right) \frac{1}{\sqrt{2}}\left(
\begin{matrix}
 \partial^\mu \zeta \\
 -e\partial^\mu \zeta^*
\end{matrix}
\right) \\
=&\bar{\alpha}\gamma^\mu \partial_\mu \psi \\
\delta G=&i\frac{1}{\sqrt{2}}\delta \mc{F}-\frac{1}{\sqrt{2}}\delta \mc{F}^* \\
=&-i\frac{1}{\sqrt{2}}\sum_{ab}\epsilon^*_a (\sigma_\mu)_{ab}\partial^\mu \zeta_a -i\frac{1}{\sqrt{2}}\sum_{ab}\epsilon_a(\sigma_\mu)^T_{ab}\partial^\mu \zeta^*_b \\
=&i\left( -\epsilon^{*T} \sigma_\mu ,+\epsilon^T\sigma^T_\mu e \right) \left(
\begin{matrix}
1 & 0 \\
0 & -1
\end{matrix}
\right)\frac{1}{\sqrt{2}}\left(
\begin{matrix}
 \partial^\mu \zeta \\
 -e\partial^\mu \zeta^*
\end{matrix}
\right) \\
=&i\bar{\alpha} \gamma^\mu \gamma_5 \partial_\mu \psi \\
=&-i\bar{\alpha} \gamma_5 \gamma^\mu \partial_\mu \psi
\end{align*}
となる.前に示した通り,この変換は作用
\begin{align*}
I=&\int d^4x \Bigl\{-\frac{1}{2}\partial_\mu A \partial^\mu A-\frac{1}{2}\partial _\mu B \partial^\mu B -\frac{1}{2}\bar{\psi}\gamma^\mu \partial_\mu \psi \\
&+\frac{1}{2}(F^2+G^2)+m[FA+GB-\frac{1}{2}\bar{\psi}\psi] \\
&+g\left[ F(A^2-B^2)+2GAB-\bar{\psi}(A+i\gamma_5 B)\psi \right]\Bigr\}
\end{align*}
を不変に保つことがわかる.以下の三つの節では,さらに一般的な超対称作用を導くのに便利な手法を調べる.\par
フェルミオン場$\psi(x)$が自由場のディラック方程式$(\gamma^\mu \partial_\mu +m)\psi=0$を満たす場合には,これらの変換則から
\begin{align*}
\delta(F+mA)=\bar{\alpha} (\gamma^\mu \partial_\mu +m)\psi=0 ,\quad \delta(G+mB)=-i\bar{\alpha} \gamma_5 (\gamma^\mu \partial_\mu +m)\psi=0
\end{align*}
となって$F+mA,G+mB$が不変となり,したがって$\mr{Q}_a,\mr{Q}_a^*$と可換であり,そのようなものは(25.2.31)からまた$P_\mu$とも可換であることが分かる.これが$F=-mA,G=-mB$を示すわけではないが,しかし交換・反交換関係(26.1.1)~(26.1.6)も変換則(26.1.21)も変えずに,場$F,G$からそれぞれ定数$F+mA,B+mG$を差し引いて再定義することができ,新しい場$F,G$が$F=-mA,G=-mB$で与えられ,したがって$\mc{F}=-m\phi^*$となるようにすることが可能となる.これは相互作用がある場合には正しい操作にならないが,相互作用がある場合にも$\mc{F}(x),F(x),G(x)$は通常,補助場(微分がラグランジアンにない場)であり,(26.1.22)の作用の場合のように超対称多重項の他の場を使って表すことが可能となる.

\newpage

\subsection{一般的な補助場}
前の節で述べたような直接的な手法で場の超対称多重項を構成するのは労力がえぐい.サラムとストラスディーによって発明された超対称多重項の場を単一の超場にまとめる理論形式を使うと労力が非常に節約できる.\par
4元運動量演算子$P_\mu$が通常の時空座標$x^\mu$の並進の生成子として定義されたのと全く同じように,4元超対称性生成子$(\mr{Q}_a,\mr{Q}^*_a)$は4つのフェルミオン的c数の\uwave{超空間}座標の並進の生成子とみなすことができる.これらの超空間座標はお互いに反可換で,フェルミオン場と反可換,しかし$x^\mu$および全てのボゾン場と可換だとする.ローレンツ不変なラグランジアン密度を構成するのが目的なので,25.2節の4成分ディラック形式を採用すると便利だ.超対称性生成子は4成分マヨラナ・スピノル$\mr{Q}_a$にまとめられるので,それに対応して超空間座標は別の4成分マヨラナ・スピノル$\theta_\alpha$にまとめられる.(運動量演算子は$\partial/\partial x^\mu$に比例しており$[P^\mu,P^\nu]=0$だったが,)超対称性生成子の反交換子はゼロとはならないので,超座標の並進演算子$\partial/\partial \theta_\alpha$に単に比例するとはとれない.その代わりに,サムラとストラスディーは,もし超対称性生成子$Q$とボゾン的またはフェルミオン的な超場$S(x,\theta)$(4次元座標と超空間座標に依存する場)との交換子または反交換子が
\begin{align*}
[Q,S\}=i\mc{Q}S
\end{align*}
ただし,$\mc{Q}$は超空間の微分演算子
\begin{align*}
\mc{Q}\equiv -\frac{\partial }{\partial \bar{\theta}}+\gamma^\mu \theta \frac{\partial}{\partial x^\mu}
\end{align*}
だとすると,超対称代数が満たされることを発見した.(ここでいつもの通り$\bar{\theta}=\theta^\dagger \beta$.フェルミオン的c数変数についての微分は全て左微分,つまりある変数について微分する際にはその変数を項の最も左に動かして計算
\begin{align*}
\frac{\partial}{\partial \theta}\theta' \theta =-\frac{\partial}{\partial \theta} \theta \theta'=-\theta'
\end{align*}
するものとする.)$\epsilon$を(26.A.3)で与えられる$4\times 4$行列として,マヨラナ・スピノルについては(25.2節で散々使ったように)$\bar{\theta}=\theta^T \gamma_5 \epsilon$なので,(26.2.2)をより明確に書き出すと
\begin{align*}
\bar{\theta}_{\alpha}=&\theta_\gamma (\gamma_5 \epsilon)_{\gamma\alpha}\\
 \therefore \quad \theta_\alpha=&-\bar{\theta}_\gamma (\gamma_5\epsilon)_{\gamma\alpha} \\
\frac{\partial}{\partial \bar{\theta}_\alpha}=&\frac{\partial}{\partial \theta_\gamma}
\frac{\partial}{\partial \bar{\theta}_\alpha}=\frac{\partial \theta_\beta}{\partial \bar{\theta}_\alpha}\frac{\partial}{\partial \theta_\beta} =-(\gamma_5\epsilon)_{\alpha\beta}\frac{\partial}{\partial \theta_\beta} \\
\mc{Q}_\alpha =&-\frac{\partial}{\partial \bar{\theta}_\alpha} +\gamma^\gamma_{\alpha\beta} \theta_\beta \frac{\partial}{\partial x^\mu} \\
=&\sum_\beta (\gamma_5\epsilon)_{\alpha\beta}\frac{\partial}{\partial \theta_\beta}+\sum_\beta \gamma^\mu_{\alpha\beta} \theta_\beta \frac{\partial}{\partial x^\mu}
\end{align*}
と書ける.また$\bar{Q}_\alpha=Q_\gamma(\gamma_5 \epsilon)_{\gamma\alpha}$なのだったから
\begin{align*}
[\bar{Q}_\alpha,S\}=&i\bar{\mc{Q}}_\alpha S \\
=&[Q_\gamma,S\}(\gamma_5 \epsilon)_{\gamma\alpha}=i\mc{Q}_\gamma S (\gamma_5 \epsilon)_{\gamma\alpha} \\
\therefore \quad \bar{\mc{Q}}_\gamma=&\sum_{\alpha}\mc{Q}_\alpha(\gamma_5 \epsilon)_{\alpha\gamma} \\
=&\sum_{\alpha\beta} (\gamma_5\epsilon)_{\alpha\beta}\frac{\partial}{\partial \theta_\beta}(\gamma_5 \epsilon)_{\alpha\gamma}+\sum_{\alpha\beta} \gamma^\mu_{\alpha\beta} \theta_\beta (\gamma_5 \epsilon)_{\alpha\gamma} \frac{\partial}{\partial x^\mu} \\
=&\sum_\beta ((\gamma_5 \epsilon)^T\gamma_5\epsilon)_{\gamma\beta}\frac{\partial}{\partial \theta_\beta}+\sum_\beta ((\gamma_5 \epsilon)^T \gamma^\mu)_{\gamma\beta} \theta_\beta \frac{\partial}{\partial x^\mu} \\
=&-\sum_\beta (\epsilon \gamma_5 \gamma_5\epsilon)_{\gamma\beta}\frac{\partial}{\partial \theta_\beta}-\sum_\beta (\epsilon \gamma_5 \gamma^\mu)_{\gamma\beta} \theta_\beta \frac{\partial}{\partial x^\mu} \\
=&\frac{\partial}{\partial \theta_\gamma}-\sum_\beta (\gamma_5 \epsilon \gamma^\mu)_{\gamma\beta} \theta_\beta \frac{\partial}{\partial x^\mu}
\end{align*}
と書ける.さらにこれらの交換関係は
\begin{align*}
\{\mc{Q}_\alpha ,\bar{\mc{Q}}_\beta\}=&\left\{(\gamma_5\epsilon)_{\alpha\gamma}\frac{\partial}{\partial \theta_\gamma}+ \gamma^\mu_{\alpha\gamma} \theta_\gamma \frac{\partial}{\partial x^\mu}, \frac{\partial}{\partial \theta_\beta}- (\gamma_5 \epsilon \gamma^\nu)_{\beta\delta} \theta_\delta \frac{\partial}{\partial x^\nu}\right\} \\
=&(\gamma_5 \epsilon)_{\alpha\gamma}\left\{\frac{\partial}{\partial \theta_\gamma},\frac{\partial}{\partial \theta_\beta}\right\}+\gamma^\mu_{\alpha\gamma}\frac{\partial}{\partial x^\mu}\left\{\theta_\gamma ,\frac{\partial}{\partial \theta_\beta }\right\} \\
&-(\gamma_5 \epsilon)_{\alpha\gamma}(\gamma_5 \epsilon \gamma^\nu)_{\beta\delta}\frac{\partial}{\partial x^\nu}\left\{\frac{\partial}{\partial \theta_\gamma},\theta_{\delta}\right\} -\gamma^\mu_{\alpha\gamma}(\gamma_5 \epsilon \gamma^\nu)_{\beta\delta}\frac{\partial^2}{\partial x^\mu x^\nu}\left\{\theta_\gamma ,\theta_\delta \right\}\\
=&\gamma^\mu_{\alpha\beta}\frac{\partial}{\partial x^\mu}-(\gamma_5 \epsilon)_{\alpha\gamma}(\gamma_5 \epsilon \gamma^\mu)_{\beta\gamma}\frac{\partial}{\partial x^\mu} \\
=&\gamma^\mu_{\alpha\beta}\frac{\partial}{\partial x^\mu}-[(-\epsilon\gamma_5 )(\gamma^\mu)^T(\gamma_5 \epsilon )]_{\alpha\beta}\frac{\partial}{\partial x^\mu} \\
=&\gamma^\mu_{\alpha\beta}\frac{\partial}{\partial x^\mu}+\gamma^\mu_{\alpha\beta}\frac{\partial}{\partial x^\mu} \quad  \because (5.4.35)\gamma_\mu^T=-\mc{C}\gamma_\mu \mc{C}^{-1},\mc{C}=-\epsilon \gamma_5 \\
=&2\gamma^\mu_{\alpha\beta}\frac{\partial}{\partial x^\mu}
\end{align*}
となる.ここで
\begin{align*}
\left\{\frac{\partial}{\partial \theta_\alpha},\frac{\partial}{\partial \theta_\beta}\right\}=&\frac{\partial}{\partial \theta_\alpha}\frac{\partial}{\partial \theta_\beta} +\frac{\partial}{\partial \theta_\beta}\frac{\partial}{\partial \theta_\alpha}=\frac{\partial}{\partial \theta_\alpha}\frac{\partial}{\partial \theta_\beta}-\frac{\partial}{\partial \theta_\alpha}\frac{\partial}{\partial \theta_\beta}=0 \\
\left\{\frac{\partial}{\partial \theta_\alpha},\theta_{\beta}\right\}=& \frac{\partial}{\partial \theta_\alpha}\theta_\beta+\theta_\beta\frac{\partial}{\partial \theta_\alpha}=\delta_{\alpha\beta}-\theta_\beta\frac{\partial}{\partial \theta_\alpha}+\theta_\beta\frac{\partial}{\partial \theta_\alpha}=\delta_{\alpha\beta} \\
\{\theta_\alpha ,\theta_\beta\}=&\theta_\alpha \theta_\beta +\theta_\beta \theta_\alpha =0
\end{align*}
となることを使った.(26.2.6)(26.2.1)およびヤコビ恒等式(25.1.5)から
\begin{align*}
\left[\left\{Q_\alpha ,\bar{Q}_\beta \right\},S\right]=&-\left[\left[S,Q_{\alpha}\right\},\bar{Q}_\beta\right\}+(-1)^{\eta_S}\left[\left[\bar{Q}_\beta ,S\right\},Q_\alpha\right\} \\
=&+(-1)^{\eta_S}\left[\left[Q_{\alpha},S\right\},\bar{Q}_\beta\right\}+(-1)^{\eta_S}\left[\left[\bar{Q}_\beta ,S\right\},Q_\alpha\right\} \quad \because (25.1.6) \\
=&(-1)^{\eta_S}i\left[\mc{Q}_\alpha S,\bar{Q}_\beta\right\}+(-1)^{\eta_S}i\left[\bar{\mc{Q}}_\beta S,Q_\alpha\right\} \\
=&-i\left[\bar{Q}_\beta ,\mc{Q}_\alpha S\right\}-i\left[Q_\alpha,\bar{\mc{Q}}_\beta S\right\} \\
=&\bar{\mc{Q}}_\beta (\mc{Q}_\alpha S)+ \mc{Q}_\alpha (\bar{\mc{Q}}_\beta S) \\
=&\left\{ \mc{Q}_\alpha ,\bar{\mc{Q}}_\beta \right\} S \\
=&2\gamma^\mu_{\alpha\beta}\partial_\mu S =-2i \gamma^\mu_{\alpha\beta}[P_\mu, S] \quad \because (10.1.1)
\end{align*}
となり,これは実際に反交換関係(25.2.36)と一致する.

\vskip\baselineskip


交換・反交換関係(26.2.1)を微小超対称性変換のもとでの変換則して表すと,より都合がよいことが多い.前に述べた通り,(5.1.6)よりスカラー場の時空並進が$\phi(x+a)=U(a)\phi(x)U^{-1}(a)=e^{-ia^\mu P_\mu}\phi(x)e^{+ia^\mu P_\mu}$であることの類推として$Q_\alpha$を$\theta_\alpha$の並進演算子として表すと,スカラー超場$S(x,\theta)$に対して
\begin{align*}
S(x+\delta x ,\theta+ \delta \theta)=&e^{-i\bar{\epsilon} Q -ia^\mu P_\mu}S(x,\theta)e^{+i\bar{\epsilon} Q+ia^\mu P_\mu } \\
=&e^{-i\bar{\epsilon} Q -ia^\mu P_\mu}e^{-i\bar{\theta}Q-ix^\mu P_\mu}S(0,0)e^{+i\bar{\theta} Q+ix^\mu P_\mu}e^{+i\bar{\epsilon} Q+ia^\mu P_\mu }
\end{align*}
と書く.BCH公式
\begin{align*}
e^{A} e^B=\exp\left( A+B +\frac{1}{2}[A,B]+\cdots \right)
\end{align*}
より
\begin{align*}
&\exp(i\bar{\theta} Q+ix^\mu P_\mu)\exp( i\bar{\epsilon} Q+ia^\mu P_\mu)  \\
=&\exp\left(i (\bar{\theta}+\bar{\epsilon})Q+i(x^\mu+a^\mu)P_\mu-\frac{1}{2}[\bar{\theta}Q+x^\mu P_\mu ,\bar{\epsilon}Q+a^\mu P_\mu]\right) \\
=&\exp\left(i (\bar{\theta}+\bar{\epsilon})Q+i(x^\mu+a^\mu)P_\mu- \frac{1}{2}[\bar{\theta}_\alpha Q_\alpha,\bar{\epsilon}_\beta Q_\beta]\right) \\
=&\exp\left(i (\bar{\theta}+\bar{\epsilon})Q+i(x^\mu+a^\mu)P_\mu- \frac{1}{2}[ \bar{Q}_\alpha \theta_\alpha,\bar{\epsilon}_\beta Q_\beta]\right) \\
=&\exp\left(i (\bar{\theta}+\bar{\epsilon})Q+i(x^\mu+a^\mu)P_\mu- \frac{1}{2}\theta_\alpha \bar{\epsilon}_\beta\{ \bar{Q}_\alpha, Q_\beta\}\right) \\
=&\exp\left(i (\bar{\theta}+\bar{\epsilon})Q+i(x^\mu+a^\mu)P_\mu+ \frac{1}{2}\theta_\alpha(2i\gamma^\mu P_\mu)_{\beta\alpha} \bar{\epsilon}_\beta \right) \\
=&\exp\left(i (\bar{\theta}+\bar{\epsilon})Q+i(x^\mu+a^\mu-\bar{\epsilon}\gamma^\mu \theta)P_\mu \right)
\end{align*}
途中で
\begin{align*}
\theta_\alpha \bar{Q}_\alpha=\theta_\alpha Q_\beta (\gamma_5 \epsilon )_{\beta\alpha}=-\theta_\alpha(\gamma_5 \epsilon)Q_{\beta}=-\bar{\theta}_\alpha Q_\alpha=Q_\alpha \bar{\theta}_\alpha
\end{align*}
を用いた.したがって$\delta x^\mu=a^\mu-\bar{\epsilon}\gamma^\mu \theta ,\delta \theta=\epsilon$がわかる.並進$a$をゼロにとれば,$e^{-i\bar{\alpha}Q}$による微小な並進は,$[P^\mu,\phi]=i\partial^\mu \phi$から類推して$[Q_\alpha,S\}=i\mc{Q}_\alpha S$という微分演算子を導入し
\begin{align*}
S(x+\delta x,\theta+\delta \epsilon)=&e^{-i\bar{\alpha}Q}S(x,\theta )e^{i\bar{\alpha}Q}=S(x,\theta)-i\bar{\alpha}_\beta[Q_\beta,S(x,\theta)\}=S(x,\theta)+\bar{\alpha}\mc{Q}S \\
=&S(x^\mu-\bar{\alpha}\gamma^\mu \theta,\theta+\alpha) \\
\therefore \quad \delta S=&S(x+\delta x,\theta+\delta \epsilon)-S(x,\theta)= \bar{\alpha}\mc{Q}S \\
=& \delta \theta_\alpha \frac{\partial S}{\partial \theta_\alpha} +\delta x^\mu \frac{\partial S}{\partial x^\mu} \\
=&\alpha_\beta \frac{\partial S}{\partial \theta_\beta} -(\bar{\alpha}\gamma^\mu \theta) \frac{\partial S}{\partial x^\mu} \\
=&\bar{\alpha} \frac{\partial S}{\partial \bar{\theta}} -(\bar{\alpha}\gamma^\mu \theta) \frac{\partial S}{\partial x^\mu} \quad \because \frac{\partial }{\partial \bar{\theta}_\alpha}=-(\gamma_5 \epsilon)_{\alpha\gamma}\frac{\partial}{\partial \theta_\gamma} より \bar{\alpha}\frac{\partial }{\partial \bar{\theta}}=\alpha \frac{\partial }{\partial \theta}
\end{align*}
だけ変化させることがわかる.(符号が完全に逆だが大丈夫か?まぁ直接的な問題はないと今のところ思う.わざわざ全部符号を直すのも面倒なので,以下では教科書に合わせることとする.)ここでは$\partial / \partial \bar{\theta}$は任意の表式の左から働くことを思い出す.特に$M$が$1,\gamma_5 \gamma_\mu ,\gamma_5$の任意の1次結合で$\bar{\theta}M\theta$がゼロにならないとき,
\begin{align*}
\bar{\theta''} M \theta'=&-\theta''_\alpha(\gamma_5 \epsilon)_{\alpha\beta} M_{\beta\gamma} (\gamma_5 \epsilon)^T_{\gamma\delta} \bar{\theta'}_{\delta} \\
=&+\bar{\theta'}_\delta [(\gamma_5 \epsilon) M (\gamma_5 \epsilon)^T]^T \theta''_\alpha \\
=&\bar{\theta'}_\delta [\mc{C} M \mc{C}^{-1}]^T_{\delta\alpha} \theta''_\alpha \\
=&\bar{\theta'}M\theta'' \quad \because (5.4.35)~(5.4.39)
\end{align*}
だから
\begin{align*}
\frac{\partial}{\partial \bar{\theta}}(\bar{\theta}M\theta)=&\left[\frac{\partial}{\partial \bar{\theta'}}(\bar{\theta'}M\theta'')+\frac{\partial}{\partial \bar{\theta''}}(\bar{\theta'}M\theta'')\right]_{\theta'=\theta''=\theta} \\
=&\left[\frac{\partial}{\partial \bar{\theta'}}(\bar{\theta'}M\theta'')+\frac{\partial}{\partial \bar{\theta''}}(\bar{\theta''}M\theta')\right]_{\theta'=\theta''=\theta} \\
=&2M\theta
\end{align*}
となる.ここで積の微分が
\begin{align*}
\frac{d}{dx}f(x)g(x)=\left[\frac{\partial}{\partial x}f(x)g(y)+\frac{\partial}{\partial y}f(x)g(y)\right]_{x=y}
\end{align*}
と書けることを用いた.\par

\vskip\baselineskip

$\theta$の成分は反可換なので,積において同じ成分が二つあれば,その積はゼロになる.($\theta=(\theta_1,\theta_2,\theta_3,\theta_4)$なので$\theta_1\theta_2\neq 0$だが$\theta_2\theta_2=0$という感じ.)しかし$\theta$には4つの成分しかないので,$\theta$の任意の4次の項で終わるベキ級数となる.(5次以上だと必ず成分の被りがでる.)さらに,この章の補遺の(26.A.10)で示してあるように,2つの$\theta$の積は$(\bar{\theta}\theta),(\bar{\theta}\gamma_\mu \gamma_5 \theta),(\bar{\theta}\gamma_5 \theta)$の一次結合に比例する.また3つの積は(26.A.13)で示してあるように,$(\bar{\theta}\gamma_5 \theta)\theta$に比例し,4つの積は(26.A.15)で示してあるように$(\bar{\theta}\gamma_5 \theta)^2$に比例する.したがって,$x^\mu$と$\theta$の最も一般的な関数は以下のように表すことができる.
\begin{align*}
S(x,\theta)=&C(x)-i\left(\bar{\theta}\gamma_5 \omega(x)\right)-\frac{i}{2}\left(\bar{\theta}\gamma_5 \theta \right)M(x)-\frac{1}{2}\left(\bar{\theta}\theta \right)N(x) \\
&+\frac{i}{2}\left(\bar{\theta}\gamma_5 \gamma_\mu \theta \right)V^\mu(x)-i\left(\bar{\theta}\gamma_5 \theta\right)\left( \bar{\theta} \left[\lambda(x)+\frac{1}{2}\Slash{\partial} \omega(x)\right] \right) \\
&-\frac{1}{4}\left( \bar{\theta}\gamma_5 \theta \right)^2 \left( D(x)+\frac{1}{2}\Box C(x) \right)
\end{align*}
($\frac{1}{2}\Slash{\partial}\omega$と$\frac{1}{2}\Box C(x)$の項は,後で便利なようにそれぞれ$\lambda(x),D(x)$の項から分離してあるらしい.(26.216)(26.2.17)参照)もし$S(x,\theta)$がスカラー場ならば,$C(x),M(x),N(x),D(x)$はスカラー(もしくは擬スカラー)場,$\omega(x),\lambda(x)$は4成分スピノル場,$V^\mu(x)$はベクトル場となる.また,この章の補遺で与えるマヨナラ場の双線形積の実条件の性質(26.A.21)を使うと,もし$S(x,\theta)$が実ならば,$C(x),M(x),N(x),V^\mu(x),D(x)$は全て実となり,また$\omega(x),\lambda(x)$は位相が$s^*=-\beta \epsilon \gamma_5 s$に従うマヨラナ・スピノルとなる.以下ではしばしば,これら$C\sim D$の場はそれぞれ超場$S$の成分場と呼ぶ.\par
次に,(26.2.10)の成分場の超対称性変換則を求めなければならない.
\begin{align*}
\delta S =\left(-\bar{\alpha}\frac{\partial}{\partial \bar{\theta}}+(\bar{\alpha}\gamma^\mu \theta) \frac{\partial}{\partial x^\mu}\right)S(x,\theta)
\end{align*}
$S(x,\theta)$を展開していくと,まず1項目は
\begin{align*}
&\left(-\bar{\alpha}\frac{\partial}{\partial \bar{\theta}}+(\bar{\alpha}\gamma^\mu \theta) \frac{\partial}{\partial x^\mu}\right)C \\
&= (\bar{\alpha}\gamma^\mu \theta)\frac{\partial C}{\partial x^\mu}
\end{align*}
第2項目は
\begin{align*}
&\left(-\bar{\alpha}\frac{\partial}{\partial \bar{\theta}}+(\bar{\alpha}\gamma^\mu \theta) \frac{\partial}{\partial x^\mu}\right)[-i(\bar{\theta}\gamma_5 \omega)] \\
&=+i(\bar{\alpha}\gamma_5 \omega )-i(\bar{\alpha}\gamma^\mu \theta) \left(\bar{\theta}\gamma_5 \frac{\partial \omega }{\partial x^\mu}\right)
\end{align*}
第3項目は
\begin{align*}
&\left(-\bar{\alpha}\frac{\partial}{\partial \bar{\theta}}+(\bar{\alpha}\gamma^\mu \theta) \frac{\partial}{\partial x^\mu}\right)\left[-\frac{i}{2}\left(\bar{\theta}\gamma_5 \theta \right)M\right] \\
=&+i(\bar{\alpha}\gamma_5 \theta)M-\frac{i}{2}(\bar{\alpha}\gamma^\mu \theta )(\bar{\theta}\gamma_5 \theta)\frac{\partial M}{\partial x^\mu}
\end{align*}
第4項目は
\begin{align*}
&\left(-\bar{\alpha}\frac{\partial}{\partial \bar{\theta}}+(\bar{\alpha}\gamma^\mu \theta) \frac{\partial}{\partial x^\mu}\right)\left[-\frac{1}{2}\left(\bar{\theta}\theta \right)N  \right] \\
&=+(\bar{\alpha}\theta)N-\frac{1}{2}(\bar{\alpha}\gamma^\mu \theta)(\bar{\theta}\theta)\frac{\partial N}{\partial x^\mu}
\end{align*}
第5項目は
\begin{align*}
&\left(-\bar{\alpha}\frac{\partial}{\partial \bar{\theta}}+(\bar{\alpha}\gamma^\mu \theta) \frac{\partial}{\partial x^\mu}\right)\left[+\frac{i}{2}\left(\bar{\theta}\gamma_5 \gamma_\nu \theta \right)V^\nu(x)\right] \\
=&-i(\bar{\alpha}\gamma_5 \gamma_\nu \theta )V^\nu+\frac{i}{2}(\bar{\alpha}\gamma^\mu \theta)(\bar{\theta}\gamma_5 \gamma_\nu \theta)\frac{\partial V^\nu}{\partial x^\mu}
\end{align*}
第6項目は
\begin{align*}
&\left(-\bar{\alpha}\frac{\partial}{\partial \bar{\theta}}+(\bar{\alpha}\gamma^\mu \theta) \frac{\partial}{\partial x^\mu}\right)\left[ -i\left(\bar{\theta}\gamma_5 \theta\right)\left( \bar{\theta} \left[\lambda+\frac{1}{2}\Slash{\partial} \omega \right] \right) \right] \\
=&+2i(\bar{\alpha} \gamma_5 \theta)\left(\bar{\theta}\left[\left(\lambda +\frac{1}{2}\Slash{\partial}\omega\right)\right]\right)+i(\bar{\theta}\gamma_5 \theta )\left(\bar{\alpha}\left[\left(\lambda +\frac{1}{2}\Slash{\partial}\omega\right)\right]\right) \\
&-i(\bar{\alpha}\gamma^\mu \theta)(\bar{\theta}\gamma_5 \theta)\left(\bar{\theta}\partial_\mu \left[\left(\lambda +\frac{1}{2}\Slash{\partial}\omega\right)\right]\right)
\end{align*}
第7項目は
\begin{align*}
&\left(-\bar{\alpha}\frac{\partial}{\partial \bar{\theta}}+(\bar{\alpha}\gamma^\mu \theta) \frac{\partial}{\partial x^\mu}\right)\left[-\frac{1}{4}\left( \bar{\theta}\gamma_5 \theta \right)^2 \left( D+\frac{1}{2}\Box C \right)\right] \\
=&+\frac{1}{2}(\bar{\alpha}\gamma_5 \theta )(\bar{\theta}\gamma_5 \theta)\left[ D+\frac{1}{2}\Box C \right]+\frac{1}{2}(\bar{\theta}\gamma_5 \theta)(\bar{\alpha}\gamma_5 \theta )\left[ D+\frac{1}{2}\Box C \right] \\
&-\frac{1}{4}(\bar{\alpha}\gamma^\mu \theta)(\bar{\theta}\gamma_5 \theta)^2 \partial_\mu \left[ D+\frac{1}{2}\Box C \right] \\
=&+(\bar{\theta}\gamma_5 \theta)(\bar{\alpha}\gamma_5 \theta )\left[ D+\frac{1}{2}\Box C \right] 
\end{align*}
最後は$\theta$について5次の項は必ずゼロになることを使った.これをまとめれば
\begin{align*}
\delta S=& (\bar{\alpha}\gamma^\mu \theta)\frac{\partial C}{\partial x^\mu} \\
&+i(\bar{\alpha}\gamma_5 \omega )-i(\bar{\alpha}\gamma^\mu \theta) \left(\bar{\theta}\gamma_5 \frac{\partial \omega }{\partial x^\mu}\right) \\
&+i(\bar{\alpha}\gamma_5 \theta)M-\frac{i}{2}(\bar{\alpha}\gamma^\mu \theta )(\bar{\theta}\gamma_5 \theta)\frac{\partial M}{\partial x^\mu} \\
&+(\bar{\alpha}\theta)N-\frac{1}{2}(\bar{\alpha}\gamma^\mu \theta)(\bar{\theta}\theta)\frac{\partial N}{\partial x^\mu} \\
&-i(\bar{\alpha}\gamma_5 \gamma_\nu \theta )V^\nu+\frac{i}{2}(\bar{\alpha}\gamma^\mu \theta)(\bar{\theta}\gamma_5 \gamma_\nu \theta)\frac{\partial V^\nu}{\partial x^\mu}  \\
&+2i(\bar{\alpha} \gamma_5 \theta)\left(\bar{\theta}\left[\left(\lambda +\frac{1}{2}\Slash{\partial}\omega\right)\right]\right)+i(\bar{\theta}\gamma_5 \theta )\left(\bar{\alpha}\left[\left(\lambda +\frac{1}{2}\Slash{\partial}\omega\right)\right]\right) \\
&-i(\bar{\alpha}\gamma^\mu \theta)(\bar{\theta}\gamma_5 \theta)\left(\bar{\theta}\partial_\mu \left[\left(\lambda +\frac{1}{2}\Slash{\partial}\omega\right)\right]\right) \\
&+(\bar{\theta}\gamma_5 \theta)(\bar{\alpha}\gamma_5 \theta )\left[ D+\frac{1}{2}\Box C \right] 
\end{align*}
となる.各項を(26.2.10)のように標準的な形にまとめることで変換則がわかる.まず(26.A.9)から
\begin{align*}
&(\bar{\alpha}\gamma^\mu \theta)(\bar{\theta} \gamma_5 \partial_\mu \omega)=(\bar{\alpha}\gamma^\mu)_{\alpha}\theta_\alpha \bar{\theta}_\beta ( \gamma_5 \partial_\mu \omega)_{\beta} \\
=&-\frac{1}{4}(\bar{\alpha}\gamma^\mu)_{\alpha}\delta_{\alpha\beta}(\bar{\theta}\theta) ( \gamma_5 \partial_\mu \omega)_{\beta}+\frac{1}{4}(\bar{\alpha}\gamma^\mu)_{\alpha} (\gamma_5\gamma_\nu)_{\alpha\beta}(\bar{\theta}\gamma_5 \gamma^\nu \theta) ( \gamma_5 \partial_\mu \omega)_{\beta} \\
&-\frac{1}{4}(\bar{\alpha}\gamma^\mu)_{\alpha} (\gamma_5)_{\alpha\beta} (\bar{\theta}\gamma_5 \theta) ( \gamma_5 \partial_\mu \omega)_{\beta} \\
=&-\frac{1}{4}(\bar{\theta}\theta)(\bar{\alpha}\gamma^\mu \gamma_5 \partial_\mu \omega)+\frac{1}{4} (\bar{\theta}\gamma_5 \gamma^\nu \theta)(\bar{\alpha}\gamma^\mu \gamma_5 \gamma_\nu \gamma_5 \partial_\mu \omega)-\frac{1}{4}(\bar{\theta}\gamma_5 \theta)(\bar{\alpha}\gamma^\mu \gamma_5 \gamma_5 \partial_\mu \omega) \\
=&-\frac{1}{4}(\bar{\theta}\theta)(\bar{\alpha} \Slash{\partial} \gamma_5 \omega)-\frac{1}{4} (\bar{\theta}\gamma_5 \gamma^\nu \theta)(\bar{\alpha} \Slash{\partial}\gamma_\nu \omega)-\frac{1}{4}(\bar{\theta}\gamma_5 \theta)(\bar{\alpha}\Slash{\partial} \omega)
\end{align*}
がわかる.これは第3項目に使える.また(26.A.16)から
\begin{align*}
&(\bar{\alpha}\gamma^\mu \theta)(\bar{\theta}\theta)=(\bar{\alpha}\gamma^\mu)_{\alpha}\theta_\alpha (\bar{\theta}\theta) \\
=&-(\bar{\alpha}\gamma^\mu)_{\alpha}(\gamma_5 \theta)_\alpha (\bar{\theta}\gamma_5 \theta) \\
=&-(\bar{\alpha}\gamma^\mu \gamma_5 \theta) (\bar{\theta}\gamma_5\theta)
\end{align*}
となる.これは第7項目に使える.(26.A.17)は
\begin{align*}
&(\bar{\alpha}\gamma^\mu \theta)(\bar{\theta}\gamma_5 \gamma_\nu \theta)=(\bar{\alpha}\gamma^\mu)_\alpha  \theta_\alpha(\bar{\theta}\gamma_5 \gamma_\nu \theta) \\
=&-(\bar{\alpha}\gamma^\mu)_\alpha (\gamma_\nu \theta)_\alpha(\bar{\theta}\gamma_5 \theta) \\
=&-(\bar{\alpha}\gamma^\mu \gamma_\nu \theta)(\bar{\theta}\gamma_5 \theta)
\end{align*}
となる.これは第9項目に使える.(26.A.9)から
\begin{align*}
&(\bar{\alpha}\gamma_5 \theta )\left(\bar{\theta}\left[\lambda +\frac{1}{2}\Slash{\partial}\omega\right]\right) =(\bar{\alpha}\gamma_5)_\alpha \theta_\alpha \bar{\theta}_\beta \left[\lambda +\frac{1}{2}\Slash{\partial}\omega\right]_\beta \\
=&-\frac{1}{4}(\bar{\alpha}\gamma_5)_\alpha \delta_{\alpha\beta}(\bar{\theta}\theta)\left[\lambda+\frac{1}{2}\Slash{\partial}\omega\right]_\beta+\frac{1}{4}(\bar{\alpha}\gamma_5)_\alpha (\gamma_5 \gamma_\mu)_{\alpha\beta}(\bar{\theta}\gamma_5 \gamma^\mu\theta) \left[\lambda+\frac{1}{2}\Slash{\partial}\omega\right]_\beta \\
&-\frac{1}{4}(\bar{\alpha}\gamma_5)_\alpha (\gamma_5)_{\alpha\beta}(\bar{\theta}\gamma_5 \theta)\left[\lambda+\frac{1}{2}\Slash{\partial}\omega\right]_\beta \\
=&-\frac{1}{4}(\bar{\theta}\theta) \left(\bar{\alpha}\gamma_5 \left[\lambda+\frac{1}{2}\Slash{\partial}\omega\right]\right)+\frac{1}{4}(\bar{\theta}\gamma_5 \gamma^\mu\theta) \left(\bar{\alpha}\gamma_\mu \left[\lambda+\frac{1}{2}\Slash{\partial}\omega\right]\right)-\frac{1}{4}(\bar{\theta}\gamma_5 \theta) \left(\bar{\alpha}\left[\lambda+\frac{1}{2}\Slash{\partial}\omega\right]\right)
\end{align*}
となる.これは第10項目に使える.(26.A.19)から
\begin{align*}
&(\bar{\alpha}\gamma^\mu \theta)(\bar{\theta}\gamma_5 \theta)\left(\bar{\theta}\partial_\mu \left[\lambda+\frac{1}{2}\Slash{\partial}\omega \right]\right)=(\bar{\alpha}\gamma^\mu )_\alpha (\bar{\theta}\gamma_5 \theta)\theta_\alpha \bar{\theta}_\beta \left(\partial_\mu \left[\lambda+\frac{1}{2}\Slash{\partial}\omega \right]\right)_\beta \\
=&-\frac{1}{4}(\bar{\alpha}\gamma^\mu )_\alpha (\gamma_5)_{\alpha\beta}(\bar{\theta}\gamma_5 \theta)^2 \left(\partial_\mu \left[\lambda+\frac{1}{2}\Slash{\partial}\omega \right]\right)_\beta \\
=&-\frac{1}{4}\left(\bar{\alpha}\gamma^\mu \gamma_5 \partial_\mu \left[\lambda+\frac{1}{2}\Slash{\partial}\omega \right]\right)(\bar{\theta}\gamma_5 \theta)^2 \\
=&-\frac{1}{4}\left(\bar{\alpha} \Slash{\partial}\gamma_5 \left[\lambda+\frac{1}{2}\Slash{\partial}\omega \right]\right)(\bar{\theta}\gamma_5 \theta)^2 
\end{align*}
となる.これは第12項目に使える.これらを使っていくと
\begin{align*}
\delta S=& (\bar{\alpha}\gamma^\mu \theta)\frac{\partial C}{\partial x^\mu} \\
&+i(\bar{\alpha}\gamma_5 \omega )
+i\frac{1}{4}(\bar{\theta}\theta)(\bar{\alpha} \Slash{\partial} \gamma_5 \omega)+i\frac{1}{4} (\bar{\theta}\gamma_5 \gamma^\nu \theta)(\bar{\alpha} \Slash{\partial}\gamma_\nu \omega)+i\frac{1}{4}(\bar{\theta}\gamma_5 \theta)(\bar{\alpha}\Slash{\partial} \omega) \\
&+i(\bar{\alpha}\gamma_5 \theta)M-\frac{i}{2}(\bar{\alpha}\gamma^\mu \theta )(\bar{\theta}\gamma_5 \theta)\frac{\partial M}{\partial x^\mu} \\
&+(\bar{\alpha}\theta)N+\frac{1}{2}(\bar{\alpha}\gamma^\mu \gamma_5 \theta) (\bar{\theta}\gamma_5\theta)\frac{\partial N}{\partial x^\mu} \\
&-i(\bar{\alpha}\gamma_5 \gamma_\nu \theta )V^\nu-\frac{i}{2}(\bar{\alpha}\gamma^\mu \gamma_\nu \theta)(\bar{\theta}\gamma_5 \theta) \frac{\partial V^\nu}{\partial x^\mu}  \\
&-\frac{i}{2}(\bar{\theta}\theta) \left(\bar{\alpha}\gamma_5 \left[\lambda+\frac{1}{2}\Slash{\partial}\omega\right]\right)+\frac{i}{2}(\bar{\theta}\gamma_5 \gamma^\mu\theta) \left(\bar{\alpha}\gamma_\mu \left[\lambda+\frac{1}{2}\Slash{\partial}\omega\right]\right)-\frac{i}{2}(\bar{\theta}\gamma_5 \theta) \left(\bar{\alpha}\left[\lambda+\frac{1}{2}\Slash{\partial}\omega\right]\right) \\
&+i(\bar{\theta}\gamma_5 \theta )\left(\bar{\alpha}\left[\left(\lambda +\frac{1}{2}\Slash{\partial}\omega\right)\right]\right) \\
&+i\frac{1}{4}\left(\bar{\alpha} \Slash{\partial}\gamma_5 \left[\lambda+\frac{1}{2}\Slash{\partial}\omega \right]\right)(\bar{\theta}\gamma_5 \theta)^2  +(\bar{\theta}\gamma_5 \theta)(\bar{\alpha}\gamma_5 \theta )\left[ D+\frac{1}{2}\Box C \right]
\end{align*}
$\theta$の因子が増える順序に項を並び替えてまとめると,$\theta$のゼロ次は2項目のみ,$\theta$の項は1,6,7,10項目,$(\bar{\theta}\theta)$の項は3,12項目,$(\bar{\theta}\gamma_5 \theta)$の項は5,14,15項目,$(\bar{\theta}\gamma_5 \gamma^\mu \theta)$の項は4,13項目,$(\bar{\theta}\gamma_5 \theta)\theta$の項は7,9,11,17項目,$(\bar{\theta}\gamma_5 \theta)^2$の項は16項目なので,
\begin{align*}
\delta S=& i(\bar{\alpha}\gamma_5 \omega ) \\
&+(\bar{\alpha}(\Slash{\partial}C+i\gamma_5 M +N -i\gamma_5 \Slash{V} )\theta) \\
&-\frac{1}{2}i(\bar{\theta}\theta)(\bar{\alpha} \gamma_5 [\lambda+\Slash{\partial}\omega])+\frac{i}{2}(\bar{\theta}\gamma_5 \theta)(\bar{\alpha}[\lambda+\Slash{\partial}\omega]) \\
&+\frac{i}{2}(\bar{\theta}\gamma_5 \gamma^\mu \theta)(\bar{\alpha}\gamma^\mu \lambda)+\frac{i}{4}(\bar{\theta}\gamma_5 \gamma^\mu\theta)(\bar{\alpha} \gamma_\mu\Slash{\partial} \omega)+\frac{i}{4}(\bar{\theta}\gamma_5 \gamma^\mu\theta)(\bar{\alpha}\Slash{\partial}\gamma_\mu \omega) \\
&+\frac{1}{2}(\bar{\theta}\gamma_5 \theta)(\bar{\alpha}\left[-i\Slash{\partial} M-\gamma_5 \Slash{\partial}N -i\Slash{\partial}\Slash{V}+\gamma_5(D+\Box C)\right]\theta) \\
&-\frac{i}{4}(\bar{\theta}\gamma_5 \theta)^2 \left(\bar{\alpha} \gamma_5 \left[\Slash{\partial}\lambda+\frac{1}{2}\Slash{\partial}\Slash{\partial}\omega \right]\right) \\
=& i(\bar{\alpha}\gamma_5 \omega ) \\
&+(\bar{\alpha}(\Slash{\partial}C+i\gamma_5 M +N -i\gamma_5 \Slash{V} )\theta) \\
&-\frac{1}{2}i(\bar{\theta}\theta)(\bar{\alpha} \gamma_5 [\lambda+\Slash{\partial}\omega])+\frac{i}{2}(\bar{\theta}\gamma_5 \theta)(\bar{\alpha}[\lambda+\Slash{\partial}\omega]) \\
&+\frac{i}{2}(\bar{\theta}\gamma_5 \gamma^\mu \theta)(\bar{\alpha}\gamma^\mu \lambda)+\frac{i}{4}(\bar{\theta}\gamma_5 \gamma^\mu\theta)(\bar{\alpha} \partial_\mu \omega) \\
&+\frac{1}{2}(\bar{\theta}\gamma_5 \theta)(\bar{\alpha}\left[-i\Slash{\partial} M-\gamma_5 \Slash{\partial}N -i\Slash{\partial}\Slash{V}+\gamma_5(D+\Box C)\right]\theta) \\
&-\frac{i}{4}(\bar{\theta}\gamma_5 \theta)^2 \left(\bar{\alpha} \gamma_5 \left[\Slash{\partial}\lambda+\frac{1}{2}\Box \omega \right]\right)
\end{align*}
となる.(二つ目の式は6,7項目で$\gamma_\mu \Slash{\partial}=-\Slash{\partial}\gamma_\mu+2\partial_\mu$を用いてまとめ,最後の項で$\Slash{\partial}\Slash{\partial}=\Box$を用いて変更した.)これは対称性(26.A.7)を用いると2項目と7項目の$\bar{\alpha},\theta$をひっくり返して$\bar{\theta},\alpha$にできる.ここでそれらの間にある$\gamma_\mu$について一次の項と$[\gamma_\mu ,\gamma_\nu]$の項の符号が反転する.$\Slash{\partial}\Slash{V}=\frac{1}{2}\partial_\mu V_\nu[\gamma_\mu,\gamma_\nu]+\partial_\mu V^\mu$が反転して$-\frac{1}{2}\partial_\mu V_\nu[\gamma_\mu,\gamma_\nu]+\partial_\mu V^\mu=\partial_\mu \Slash{V}\gamma^\nu$になることに注意すれば
\begin{align*}
\delta S =&i(\bar{\alpha}\gamma_5 \omega ) \\
&+(\bar{\theta}(-\Slash{\partial}C+i\gamma_5 M +N -i\gamma_5 \Slash{V} )\alpha) \\
&-\frac{1}{2}i(\bar{\theta}\theta)(\bar{\alpha} \gamma_5 [\lambda+\Slash{\partial}\omega])+\frac{i}{2}(\bar{\theta}\gamma_5 \theta)(\bar{\alpha}[\lambda+\Slash{\partial}\omega]) \\
&+\frac{i}{2}(\bar{\theta}\gamma_5 \gamma^\mu \theta)(\bar{\alpha}\gamma^\mu \lambda)+\frac{i}{4}(\bar{\theta}\gamma_5 \gamma^\mu\theta)(\bar{\alpha} \partial_\mu \omega) \\
&+\frac{1}{2}(\bar{\theta}\gamma_5 \theta)(\bar{\theta}\left[+i\Slash{\partial} M-\gamma_5 \Slash{\partial}N -i\partial_\mu \Slash{V}\gamma^\mu+\gamma_5(D+\Box C)\right]\alpha) \\
&-\frac{i}{4}(\bar{\theta}\gamma_5 \theta)^2 \left(\bar{\alpha} \gamma_5 \left[\Slash{\partial}\lambda+\frac{1}{2}\Box \omega \right]\right) 
\end{align*}
とも書ける.これを(26.2.10)の展開の$\theta$の2次までの項と見比べると,以下の変換則が対応していることがわかる.
\begin{align*}
\delta C=&i(\bar{\alpha} \gamma_5 \omega) \\
\delta \omega=&(-i\gamma_5 \Slash{\partial}C- M +i\gamma_5 N + \Slash{V} )\alpha \\
\delta M=&-(\bar{\alpha}[\lambda+\Slash{\partial}\omega]) \\
\delta N=&i(\bar{\alpha}\gamma_5[\lambda+\Slash{\partial}\omega]) \\
\delta V_\mu=& (\bar{\alpha}\gamma_\mu \lambda)+(\bar{\alpha}\partial_\mu \omega)
\end{align*}
$\theta$について3次,4次の項からは
\begin{align*}
\delta \left[\lambda+\frac{1}{2}\Slash{\partial} \omega\right]=&\frac{1}{2}\left[-\Slash{\partial}M-i\gamma_5 \Slash{\partial}N+\partial_\mu \Slash{V}\gamma^\mu +i \gamma_5 \left(D+\frac{1}{2}\Box C\right)\right]\alpha \\
\delta \left[D+\frac{1}{2}\Box C\right]=&i\left(\bar{\alpha}\gamma_5 \left[\Slash{\partial}\lambda +\frac{1}{2}\Box \omega\right]\right)
\end{align*}
を得る.最後の二つの変換則と$C,\omega$の変換則を使うと複雑な部分がキャンセルして
\begin{align*}
\delta \left[\lambda+\frac{1}{2}\Slash{\partial} \omega\right]=&\delta \lambda +\frac{1}{2}\Slash{\partial} \delta \omega \\
=&\delta\lambda +\frac{1}{2}\left[+i\gamma_5 \Slash{\partial}\Slash{\partial}C- \Slash{\partial}M -i\gamma_5 \Slash{\partial} N + \Slash{\partial}\Slash{V} \right]\alpha \\
=&\delta \lambda +\frac{1}{2}\left[+i\gamma_5 \Box C-\Slash{\partial}M-i\gamma_5 \Slash{\partial}N+\gamma^\mu \partial_\mu \Slash{V}\right] \\
\therefore \quad \delta \lambda=&\left(\frac{1}{2}\left[\partial_\mu \Slash{V},\gamma^\mu\right]+i\gamma_5 D\right)\alpha \\
\delta \left[D+\frac{1}{2}\Box C\right]=&\delta D+ \frac{1}{2}\Box \delta C \\
=& \delta D +\frac{1}{2}i\bar{\alpha}\gamma_5 \Box \omega \\
\therefore \quad \delta D=&i(\bar{\alpha}\gamma_5 \Slash{\partial}\lambda)
\end{align*}
を得る.(26.2.10)の展開で$\frac{1}{2}\Slash{\partial}\omega$と$\frac{1}{2}\Box C$の項を$\lambda,D$から分離しておいたのは,この単純な変換則を得るためだった.(これらの場は超場ではないので,単純にそれぞれの場に$\bar{\alpha}\mc{Q}$を作用させて得られるわけではない.)



\vskip\baselineskip

超場形式を使うのは,超対称多重項を他の超対称多重項から作る仕事を簡単にするためだ.二つの超場$S_1$と$S_2$がともに変換則(26.2.8)を満たすとする.このとき,それらの積$S\equiv S_1 S_2$は
\begin{align*}
\delta S=&\left[\bar{\alpha}Q,S_1 S_2\right] \\
=&\left[\bar{\alpha}Q,S_1\right]S_2 +S_1 \left[\bar{\alpha}Q,S_2\right] \\
=&(\bar{\alpha} \mc{Q} S_1)S_2 +S_1 \bar{\alpha}\mc{Q} S_2 \\
=&\left(\bar{\alpha}\mc{Q} \right)S
\end{align*}
を満たすから,やはり超場となっている.二つの(26.2.10)を積にして,$\theta$について4次までを残せば
\begin{align*}
S=&S_1 S_2 \\
=&C_1C_2-i(\bar{\theta}\gamma_5 \omega_1C_2) -i(\bar{\theta}\gamma_5 C_1 \omega_2)-\frac{i}{2}(\bar{\theta}\gamma_5 \theta )M_1 C_2 -\frac{i}{2}(\bar{\theta}\gamma_5 \theta)C_1 M_2 \\
&-\frac{1}{2}(\bar{\theta}\theta)N_1 C_2 -\frac{1}{2}(\bar{\theta}\theta)C_1 N_2 \\
&-(\bar{\theta}\gamma_5 \omega_1)(\bar{\theta}\gamma_5 \omega_2) +\frac{i}{2}(\bar{\theta}\gamma_5 \gamma_\mu \theta)C_1 V^\mu_2+\frac{i}{2}(\bar{\theta}\gamma_5 \gamma_\mu \theta)V^\mu_1 C_2 \\
&- i(\bar{\theta}\gamma_5 \theta)\left(\bar{\theta} \left[C_1 \lambda_2+ \frac{1}{2}C_1 \Slash{\partial}\omega_2\right]\right) -i(\bar{\theta}\gamma_5 \theta)\left(\bar{\theta}\left[ \lambda_1 C_2+\frac{1}{2}(\Slash{\partial} \omega_1)C_2  \right]\right) \\
&-\frac{1}{2}(\bar{\theta}\gamma_5 \theta)(\bar{\theta}\gamma_5 M_1\omega_2)-\frac{1}{2}(\bar{\theta}\gamma_5 \theta)(\bar{\theta}\gamma_5 M_2 \omega_1) +\frac{i}{2}(\bar{\theta}\theta)(\bar{\theta}\gamma_5 N_1 \omega_2)+\frac{i}{2}(\bar{\theta}\theta)(\bar{\theta}\gamma_5 \omega_1 N_2) \\
&+\frac{1}{2}(\bar{\theta}\gamma_5 \gamma_\mu \theta)(\bar{\theta} \gamma_5 \omega_2 V^\mu_1)+\frac{1}{2}(\bar{\theta}\gamma_5 \gamma_\mu \theta)(\bar{\theta}\gamma_5 \omega_1 V^\mu_2) \\
&-\frac{1}{4}(\bar{\theta}\gamma_5 \theta )(\bar{\theta}\gamma_5 \theta)M_1 M_2 +\frac{1}{4}(\bar{\theta} \theta )(\bar{\theta} \theta)N_1 N_2 -\frac{1}{4}(\bar{\theta}\gamma_5 \gamma_\mu \theta)(\bar{\theta}\gamma_5 \gamma_\nu \theta) V^\mu_1 V^\nu_2 \\
&+\frac{i}{4}(\bar{\theta}\theta)(\bar{\theta}\gamma_5 \theta)M_1 N_2+\frac{i}{4}(\bar{\theta}\theta)(\bar{\theta}\gamma_5 \theta)N_1 M_2 \\
&+\frac{1}{4}(\bar{\theta}\gamma_5 \theta)(\bar{\theta}\gamma_5 \gamma_\mu \theta)V^\mu_1 M_2 +\frac{1}{4}(\bar{\theta}\gamma_5 \theta)(\bar{\theta}\gamma_5 \gamma_\mu \theta)M_1 V^\mu_2 \\
&-\frac{i}{4}(\bar{\theta} \theta)(\bar{\theta}\gamma_5 \gamma_\mu \theta)V^\mu_1 N_2-\frac{i}{4}(\bar{\theta} \theta)(\bar{\theta}\gamma_5 \gamma_\mu \theta)N_1 V^\mu_2 \\
&-(\bar{\theta}\gamma_5 \theta)\left(\bar{\theta} \left[\lambda_1 +\frac{1}{2}\Slash{\partial}\omega_1\right]\right)(\bar{\theta}\gamma_5 \omega_2)-(\bar{\theta}\gamma_5 \theta)\left(\bar{\theta} \left[\lambda_2 +\frac{1}{2}\Slash{\partial}\omega_2 \right]\right)(\bar{\theta}\gamma_5 \omega_1) \\
&-\frac{1}{4}(\bar{\theta}\gamma_5 \theta)^2 \left(C_1 D_2+\frac{1}{2}C_1 \Box C_2\right)-\frac{1}{4}(\bar{\theta}\gamma_5 \theta)^2 \left(D_1C_2+\frac{1}{2} (\Box C_1) C_2\right)
\end{align*}
となる.これを再び(26.2.10)と見比べてどのような積に対応しているかを調べる.発狂しそうなくらい項が多いが,一つずつ見ていく.まず$\theta$についてゼロ次の項と1次の項からは単に
\begin{align*}
C=&C_1 C_2 \\
\omega=& \omega_1 C_2 +C_1 \omega_2
\end{align*}
がわかる.$\theta$について2次の項は
\begin{align*}
&-\frac{i}{2}(\bar{\theta}\gamma_5 \theta )M_1 C_2 -\frac{i}{2}(\bar{\theta}\gamma_5 \theta)C_1 M_2 \\
&-\frac{1}{2}(\bar{\theta}\theta)N_1 C_2 -\frac{1}{2}(\bar{\theta}\theta)C_1 N_2 \\
&-(\bar{\theta}\gamma_5 \omega_1)(\bar{\theta}\gamma_5 \omega_2) +\frac{i}{2}(\bar{\theta}\gamma_5 \gamma_\mu \theta)C_1 V^\mu_2+\frac{i}{2}(\bar{\theta}\gamma_5 \gamma_\mu \theta)V^\mu_1 C_2 \\
=&-\frac{i}{2}(\bar{\theta}\gamma_5 \theta)(C_1 M_2 +M_1 C_2 ) -\frac{1}{2}(\bar{\theta}\theta)(C_1 N_2 +N_1 C_2) \\
 &+\frac{i}{2}(\bar{\theta}\gamma_5 \gamma_\mu \theta)(C_1 V_2^\mu +V_1^\mu C_2) \\
 &-(\bar{\omega}_1\gamma_5 \theta )(\bar{\theta}\gamma_5 \omega_2)
\end{align*}
この最後の項を変形すると
\begin{align*}
 &-(\bar{\omega}_1\gamma_5)_\alpha \theta_\alpha \bar{\theta}_\beta(\gamma_5 \omega_2)_\beta \\
 =&+\frac{1}{4}(\bar{\omega}_1\gamma_5)_\alpha\delta_{\alpha\beta}(\bar{\theta}\theta)(\gamma_5 \omega_2)_\beta-\frac{1}{4}(\bar{\omega}_1\gamma_5)_\alpha (\gamma_5 \gamma_\mu)_{\alpha\beta}(\bar{\theta}\gamma_5 \gamma^\mu \theta)(\gamma_5 \omega_2)_\beta +\frac{1}{4}(\bar{\omega}_1\gamma_5)_\alpha (\gamma_5)_{\alpha\beta}(\bar{\theta}\gamma_5 \theta)(\gamma_5 \omega_2)_\beta\\
=&+\frac{1}{4}(\bar{\theta}\theta)(\bar{\omega}_1 \omega_2)+\frac{1}{4}(\bar{\theta}\gamma_5 \gamma_\mu\theta)(\bar{\omega}_1\gamma_5 \gamma^\mu \omega_2)+\frac{1}{4}(\bar{\theta}\gamma_5 \theta)(\bar{\omega}_1\gamma_5\omega_2)
\end{align*}
となるから
\begin{align*}
&-\frac{i}{2}(\bar{\theta}\gamma_5 \theta)\left(C_1 M_2 +M_1 C_2 +\frac{i}{2}(\bar{\omega}_1 \gamma_5 \omega_2)\right) -\frac{1}{2}(\bar{\theta}\theta)\left(C_1 N_2 +N_1 C_2-\frac{1}{2}(\bar{\omega}_1 \omega_2)\right) \\
 &+\frac{i}{2}(\bar{\theta}\gamma_5 \gamma_\mu \theta)\left(C_1 V_2^\mu +V_1^\mu C_2-\frac{i}{2}(\bar{\omega}_1 \gamma_5 \gamma^\mu \omega_2)\right)
\end{align*}
よって
\begin{align*}
M=&C_1 M_2 +M_1 C_2 +\frac{i}{2}(\bar{\omega}_1 \gamma_5 \omega_2) \\
N=&C_1 N_2 +N_1 C_2 -\frac{1}{2}(\bar{\omega}_1\omega_2) \\
V^\mu=& C_1 V_2^\mu +V_1^\mu C_2-\frac{i}{2}(\bar{\omega}_1 \gamma_5 \gamma^\mu \omega_2)
\end{align*}
がわかる.$\theta$について3次の項は
\begin{align*}
&- i(\bar{\theta}\gamma_5 \theta)\left(\bar{\theta} \left[C_1 \lambda_2+ \frac{1}{2}C_1 \Slash{\partial}\omega_2\right]\right) -i(\bar{\theta}\gamma_5 \theta)\left(\bar{\theta}\left[ \lambda_1 C_2+\frac{1}{2}(\Slash{\partial} \omega_1)C_2  \right]\right) \\
&-\frac{1}{2}(\bar{\theta}\gamma_5 \theta)(\bar{\theta}\gamma_5 M_1\omega_2)-\frac{1}{2}(\bar{\theta}\gamma_5 \theta)(\bar{\theta}\gamma_5 M_2 \omega_1) +\frac{i}{2}(\bar{\theta}\theta)(\bar{\theta}\gamma_5 N_1 \omega_2)+\frac{i}{2}(\bar{\theta}\theta)(\bar{\theta}\gamma_5 \omega_1 N_2) \\
&+\frac{1}{2}(\bar{\theta}\gamma_5 \gamma_\mu \theta)(\bar{\theta} \gamma_5 \omega_2 V^\mu_1)+\frac{1}{2}(\bar{\theta}\gamma_5 \gamma_\mu \theta)(\bar{\theta}\gamma_5 \omega_1 V^\mu_2) \\
=&-i(\bar{\theta}\gamma_5 \theta)\left(\bar{\theta}\left[C_1 \lambda_2 +\lambda_1 C_2 -\frac{i}{2}\gamma_5 M_1 \omega_2-\frac{i}{2}\gamma_5 \omega_1M_2 +\frac{1}{2}C_1 \Slash{\partial}\omega_2 +\frac{1}{2}(\Slash{\partial}\omega_1)C_2 \right]\right) \\
&-\frac{i}{2}(\bar{\theta}\gamma_5\theta)(\bar{\theta} N_1 \omega_2)-\frac{i}{2}(\bar{\theta}\gamma_5 \theta)(\bar{\theta} \omega_1 N_2) \\
&-\frac{1}{2}(\bar{\theta}\gamma_5 \theta)(\bar{\theta}\gamma_5 \gamma_\mu \omega_2 V^\mu_1)-\frac{1}{2}(\bar{\theta}\gamma_5 \theta)(\bar{\theta}\gamma_5 \gamma_\mu \omega_1 V^\mu_2) \\
=&-i(\bar{\theta}\gamma_5 \theta)\biggl(\bar{\theta}\biggl[C_1 \lambda_2 +\lambda_1 C_2 -\frac{i}{2}\gamma_5 \gamma_\mu \omega_2 V^\mu_1 -\frac{i}{2}\gamma_5 \gamma_\mu \omega_1 V^\mu_2 \\
&+\frac{1}{2}(N_1-\gamma_5 M_1) \omega_2+\frac{1}{2}(N_2-\gamma_5 M_2)\omega_1 +\frac{1}{2}C_1 \Slash{\partial}\omega_2 +\frac{1}{2}(\Slash{\partial}\omega_1)C_2 \biggr]\biggr) \\
=&-i(\bar{\theta}\gamma_5 \theta)\biggl(\bar{\theta}\biggl[C_1 \lambda_2 +\lambda_1 C_2 +\frac{i}{2}\Slash{V}_1\gamma_5 \omega_2 +\frac{i}{2}\Slash{V}_2\gamma_5 \omega_1 \\
&+\frac{1}{2}(N_1-\gamma_5 M_1) \omega_2+\frac{1}{2}(N_2-\gamma_5 M_2)\omega_1 -\frac{1}{2}\gamma^\mu \omega_1 \partial_\mu C_2 -\frac{1}{2}\gamma^\mu \omega_2 \partial_\mu C_1\\
&+\frac{1}{2}\Slash{\partial}(C_1\omega_2 +\omega_1 C_2) \biggr]\biggr)
\end{align*}
ここで(26.A.16)(26.A.17)から
\begin{align*}
(\bar{\theta}\theta)(\bar{\theta}\gamma_5 \omega)=&(\bar{\theta}\theta)(\bar{\omega}\gamma_5 \theta)=-(\bar{\theta}\gamma_5 \theta)(\bar{\omega} \theta)=-(\bar{\theta}\gamma_5 \theta)(\bar{\theta} \omega) \\
(\bar{\theta}\gamma_5 \gamma_\mu \theta)(\bar{\theta} \gamma_5 \omega_2)=&(\bar{\theta}\gamma_5 \gamma_\mu \theta)(\bar{\omega} \gamma_5 \theta)=-(\bar{\theta}\gamma_5 \theta)(\bar{\omega}\gamma_5 \gamma_\mu \theta)=-(\bar{\theta}\gamma_5 \theta)(\bar{\theta}\gamma_5 \gamma_\mu \omega)
\end{align*}
となることを用いた.これより
\begin{align*}
\lambda=& C_1 \lambda_2 +\lambda_1 C_2 -\frac{1}{2}\gamma^\mu \omega_1 \partial_\mu C_2 -\frac{1}{2}\gamma^\mu \omega_2 \partial_\mu C_1 \\
&+\frac{1}{2}i \Slash{V}_1 \gamma_5 \omega_2 +\frac{1}{2}i \Slash{V}_2 \gamma_5 \omega_1 \\
&+\frac{1}{2}(N_1-i\gamma_5 M_1)\omega_2 +\frac{1}{2}(N_2 -i\gamma_5 M_2) \omega_1
\end{align*}
がわかる.$\theta$についての4次の項は
\begin{align*}
&-\frac{1}{4}(\bar{\theta}\gamma_5 \theta )(\bar{\theta}\gamma_5 \theta)M_1 M_2 +\frac{1}{4}(\bar{\theta} \theta )(\bar{\theta} \theta)N_1 N_2 -\frac{1}{4}(\bar{\theta}\gamma_5 \gamma_\mu \theta)(\bar{\theta}\gamma_5 \gamma_\nu \theta) V^\mu_1 V^\nu_2 \\
&+\frac{i}{4}(\bar{\theta}\theta)(\bar{\theta}\gamma_5 \theta)M_1 N_2+\frac{i}{4}(\bar{\theta}\theta)(\bar{\theta}\gamma_5 \theta)N_1 M_2 \\
&+\frac{1}{4}(\bar{\theta}\gamma_5 \theta)(\bar{\theta}\gamma_5 \gamma_\mu \theta)V^\mu_1 M_2 +\frac{1}{4}(\bar{\theta}\gamma_5 \theta)(\bar{\theta}\gamma_5 \gamma_\mu \theta)M_1 V^\mu_2 \\
&-\frac{i}{4}(\bar{\theta} \theta)(\bar{\theta}\gamma_5 \gamma_\mu \theta)V^\mu_1 N_2-\frac{i}{4}(\bar{\theta} \theta)(\bar{\theta}\gamma_5 \gamma_\mu \theta)N_1 V^\mu_2 \\
&-(\bar{\theta}\gamma_5 \theta)\left(\bar{\theta} \left[\lambda_1 +\frac{1}{2}\Slash{\partial}\omega_1\right]\right)(\bar{\theta}\gamma_5 \omega_2)-(\bar{\theta}\gamma_5 \theta)\left(\bar{\theta} \left[\lambda_2 +\frac{1}{2}\Slash{\partial}\omega_2 \right]\right)(\bar{\theta}\gamma_5 \omega_1) \\
&-\frac{1}{4}(\bar{\theta}\gamma_5 \theta)^2 \left(C_1 D_2+\frac{1}{2}C_1 \Box C_2\right)-\frac{1}{4}(\bar{\theta}\gamma_5 \theta)^2 \left(D_1C_2+\frac{1}{2} (\Box C_1) C_2\right) \\
=&-\frac{1}{4}(\bar{\theta}\gamma_5 \theta)^2\left(C_1 D_2 +D_1 C_2 + M_1 M_2 +N_1 N_2 -V_1^\mu V_{2\mu} +\frac{1}{2}C_1 \Box C_2 +\frac{1}{2}C_2\Box C_1 \right) \\
&-(\bar{\theta}\gamma_5 \theta)(\bar{\omega}_2\gamma_5 \theta) \left(\bar{\theta} \left[\lambda_1 +\frac{1}{2}\Slash{\partial}\omega_1\right]\right)-(\bar{\theta}\gamma_5 \theta)(\bar{\omega}_1\gamma_5 \theta)\left(\bar{\theta} \left[\lambda_2 +\frac{1}{2}\Slash{\partial}\omega_2 \right]\right)
\end{align*}
となる.ここで(26.A.16)(26.A.17)(26.A.19)から
\begin{align*}
(\bar{\theta}\gamma_5 \theta)(\bar{\theta}\theta)=&-(\bar{\theta}\theta)(\bar{\theta}\gamma_5 \theta) \quad \therefore (\bar{\theta}\gamma_5 \theta)(\bar{\theta}\theta)=0 \\
(\bar{\theta}\gamma_5 \theta )(\bar{\theta}\gamma_5 \gamma_\mu \theta)=&-(\bar{\theta}\gamma_5 \gamma_\mu \theta)(\bar{\theta}\gamma_5 \theta) \quad \therefore (\bar{\theta}\gamma_5 \theta )(\bar{\theta}\gamma_5 \gamma_\mu \theta)=0 \\
(\bar{\theta}\theta)(\bar{\theta}\gamma_5 \gamma_\mu \theta)=&-(\bar{\theta}\gamma_\mu \theta)(\bar{\theta}\gamma_5 \theta)=0 \quad \because (26.A.8)
\end{align*}
なので2,3,4行目が全てゼロとなることを用いた.さらにこの公式より
\begin{align*}
&-(\bar{\theta}\gamma_5 \theta)(\bar{\omega}_2\gamma_5 \theta) \left(\bar{\theta} \left[\lambda_1 +\frac{1}{2}\Slash{\partial}\omega_1\right]\right) \\
=&-(\bar{\theta}\gamma_5 \theta)(\bar{\omega}_2 \gamma_5)_\alpha \theta_\alpha \bar{\theta}_\beta \left[\lambda_1 +\frac{1}{2}\Slash{\partial}\omega_1\right]_\beta \\
=&+\frac{1}{4}(\bar{\theta}\gamma_5 \theta)(\bar{\theta}\theta)\left(\bar{\omega}_2 \gamma_5 \left[\lambda_1 +\frac{1}{2}\Slash{\partial}\omega_1\right]\right)-\frac{1}{4}(\bar{\theta}\gamma_5 \theta)(\bar{\theta}\gamma_5 \gamma_\mu \theta)\left(\bar{\omega}_1\gamma_5 \gamma^\mu \left[\lambda_1 +\frac{1}{2}\Slash{\partial}\omega_1\right]\right) \\
& +\frac{1}{4}(\bar{\theta}\gamma_5 \theta)^2 \left(\bar{\omega}_1\gamma_5 \gamma_5 \left[\lambda_1 +\frac{1}{2}\Slash{\partial}\omega_1\right]\right) \\
=& +\frac{1}{4}(\bar{\theta}\gamma_5 \theta)^2 \left(\bar{\omega}_1 \left[\lambda_1 +\frac{1}{2}\Slash{\partial}\omega_1\right]\right)
\end{align*}
となる.以上より$\theta$の4次の項は
\begin{align*}
&-\frac{1}{4}(\bar{\theta}\gamma_5 \theta)^2\biggl(C_1 D_2 +D_1 C_2 + M_1 M_2 +N_1 N_2 -V_1^\mu V_{2\mu} +\frac{1}{2}C_1 \Box C_2 +\frac{1}{2}C_2\Box C_1 \\
&-\left(\bar{\omega}_1\left[\lambda_2 +\frac{1}{2}\Slash{\partial}\omega_2\right]\right)-\left(\bar{\omega}_2 \left[\lambda_1 +\frac{1}{2}\Slash{\partial}\omega_1\right]\right)\biggr) \\
=&-\frac{1}{4}(\bar{\theta}\gamma_5 \theta)^2\biggl(C_1 D_2 +D_1 C_2 + M_1 M_2 +N_1 N_2 -V_1^\mu V_{2\mu} -\partial_\mu C_1 \partial^\mu C_2 \\
&-\left(\bar{\omega}_1\left[\lambda_2 +\frac{1}{2}\Slash{\partial}\omega_2\right]\right)-\left(\bar{\omega}_2 \left[\lambda_1 +\frac{1}{2}\Slash{\partial}\omega_1\right]\right)+\frac{1}{2}\Box(C_1 C_2)\biggr)
\end{align*}
となり,よって
\begin{align*}
D=&-\partial_\mu C_1 \partial^\mu C_2 +C_1 D_2+D_1 C_2+M_1 M_2 +N_1 N_2 \\
&-\left(\bar{\omega}_1\left[\lambda_2 +\frac{1}{2}\Slash{\partial}\omega_2\right]\right)-\left(\bar{\omega}_2 \left[\lambda_1 +\frac{1}{2}\Slash{\partial}\omega_1\right]\right)-V_{1\mu}V^{\mu}_2
\end{align*}
となる.これで全て求まった!\par
超場の1次結合は当然超場となり,超場の時空微分と複素共役も超場となっていることは自明だ.しかし,超場に$\theta$のある関数をかけたり,それを$\theta$で微分すると,一般には超場にはならない.例えば,$\theta$自身は明らかに超場ではない.なぜなら$\theta$はフェルミオン的c数であるために$Q$と可換であるが,微分$\mc{Q}$ではゼロとはならず,
\begin{align*}
[ Q_\alpha ,\theta]=&0 ,\quad \mc{Q}_\alpha \theta_\beta=(\gamma_5 \epsilon)_{\alpha\beta}\neq 0 \\
\therefore \quad \delta \theta_\beta \neq& (\bar{\alpha}\mc{Q})\theta_\beta
\end{align*}
となり超場の変換則を満たさないからだ.しかし,超場を$\theta$で微分して因子$\theta$をかけることで別の超場を得る方法は存在する!それを以下で見よう.\par
以下で定義される超空間の微分演算子$\mc{D}_\alpha$を考える.
\begin{align*}
\mc{D} \equiv -\frac{\partial}{\partial \bar{\theta}}-\gamma^\mu \theta \frac{\partial}{\partial x^\mu}
\end{align*}
より陽に書くと(26.2.3)と同様
\begin{align*}
\mc{D}_\alpha= \sum_\gamma (\gamma_5 \epsilon)_{\alpha\gamma}\frac{\partial}{\partial \theta_\gamma}-\sum_\gamma \gamma^\mu_{\alpha\gamma} \theta_\gamma \frac{\partial}{\partial x^\mu}
\end{align*}
となる.$\mc{D}$と$\mc{Q}$の定義の間の唯一の違いは,時空の微分を含む第二項目の符号の違いだ.この符号の違いにより,$\mc{D}_\beta$と$\mc{Q}_\alpha$の反交換子は
\begin{align*}
\{\mc{Q}_\alpha,\mc{D}_\beta \}=&\left\{(\gamma_5\epsilon)_{\alpha\gamma}\frac{\partial}{\partial \theta_\gamma}+ \gamma^\mu_{\alpha\gamma} \theta_\gamma \frac{\partial}{\partial x^\mu}, (\gamma_5\epsilon)_{\beta\delta}\frac{\partial}{\partial \theta_\delta}- \gamma^\nu_{\beta\delta} \theta_\delta \frac{\partial}{\partial x^\nu}\right\} \\
=&(\gamma_5 \epsilon)_{\alpha\gamma}(\gamma_5 \epsilon)_{\beta\delta}\left\{\frac{\partial}{\partial \theta_\gamma},\frac{\partial}{\partial \theta_\delta}\right\}+\gamma^\mu_{\alpha\gamma}(\gamma_5 \epsilon)_{\beta\delta}\frac{\partial}{\partial x^\mu}\left\{\theta_\gamma ,\frac{\partial}{\partial \theta_\delta }\right\} \\
&-(\gamma_5 \epsilon)_{\alpha\gamma}(\gamma^\nu)_{\beta\delta}\frac{\partial}{\partial x^\nu}\left\{\frac{\partial}{\partial \theta_\gamma},\theta_{\delta}\right\} -\gamma^\mu_{\alpha\gamma}\gamma^\nu_{\beta\delta}\frac{\partial^2}{\partial x^\mu x^\nu}\left\{\theta_\gamma ,\theta_\delta \right\}\\
=&\gamma^\mu_{\alpha\gamma}(\gamma_5 \epsilon)^T_{\beta\gamma}\frac{\partial}{\partial x^\mu}-(\gamma_5 \epsilon)_{\alpha\gamma}\gamma^\mu_{\beta\gamma}\frac{\partial}{\partial x^\mu} \\
=&(\gamma^\mu (-\epsilon\gamma_5))_{\alpha\beta}\frac{\partial}{\partial x^\mu}+[(-\epsilon\gamma_5 )(\gamma^\mu)^T]_{\alpha\beta}\frac{\partial}{\partial x^\mu} \\
=&(\gamma^\mu (-\epsilon \gamma_5))_{\alpha\beta}\frac{\partial}{\partial x^\mu}+(\gamma^\mu\gamma_5 \epsilon)_{\alpha\beta}\frac{\partial}{\partial x^\mu} \quad  \because (5.4.35)\gamma_\mu^T=-\mc{C}\gamma_\mu \mc{C}^{-1},\mc{C}=-\epsilon \gamma_5 \\
=&0
\end{align*}
となり消える.$\alpha$はフェルミオン的なパラメータだから,$(\bar{\alpha}\mc{Q})$は$\mc{D}_\beta$と可換$[(\bar{\alpha}\mc{Q},\mc{D}_\beta)]=0$となる.よってもし$S(x,\theta)$が超場ならば$\delta S=-i[\bar{\alpha}Q,S]=(\bar{\alpha}\mc{Q})S$を満たしていたのを思い出すと
\begin{align*}
[(\bar{\alpha}Q),\mc{D}_\beta S]=&\mc{D}_\beta [(\bar{\alpha}Q),S] =\mc{D}_\beta \left(i\bar{\alpha}\mc{Q} S\right)=i(\bar{\alpha}\mc{Q})\mc{D}_\beta S \\
\delta (\mc{D}_\beta S)\equiv &-i[(\bar{\alpha}Q),\mc{D}_\beta S]=(\bar{\alpha}\mc{Q})\mc{D}_\beta S
\end{align*}
となり,$\mc{D}_\beta S$も超場の変換則を満たし,超場となることがわかる.したがって,超場$S$の任意の多項式の関数と,その超微分$\mc{D}_\beta S,\mc{D}_\beta \mc{D}_\gamma S$等もまた超場だ!\par
二つの超微分から時空微分を作ることが可能となる.実際
\begin{align*}
\{\mc{D}_\alpha ,\bar{\mc{D}}_\beta\}=&\left\{(\gamma_5\epsilon)_{\alpha\gamma}\frac{\partial}{\partial \theta_\gamma}- \gamma^\mu_{\alpha\gamma} \theta_\gamma \frac{\partial}{\partial x^\mu}, \frac{\partial}{\partial \theta_\beta}+ (\gamma_5 \epsilon \gamma^\nu)_{\beta\delta} \theta_\delta \frac{\partial}{\partial x^\nu}\right\} \\
=&(\gamma_5 \epsilon)_{\alpha\gamma}\left\{\frac{\partial}{\partial \theta_\gamma},\frac{\partial}{\partial \theta_\beta}\right\}-\gamma^\mu_{\alpha\gamma}\frac{\partial}{\partial x^\mu}\left\{\theta_\gamma ,\frac{\partial}{\partial \theta_\beta }\right\} \\
&+(\gamma_5 \epsilon)_{\alpha\gamma}(\gamma_5 \epsilon \gamma^\nu)_{\beta\delta}\frac{\partial}{\partial x^\nu}\left\{\frac{\partial}{\partial \theta_\gamma},\theta_{\delta}\right\} -\gamma^\mu_{\alpha\gamma}(\gamma_5 \epsilon \gamma^\nu)_{\beta\delta}\frac{\partial^2}{\partial x^\mu x^\nu}\left\{\theta_\gamma ,\theta_\delta \right\}\\
=&-\gamma^\mu_{\alpha\beta}\frac{\partial}{\partial x^\mu}+(\gamma_5 \epsilon)_{\alpha\gamma}(\gamma_5 \epsilon \gamma^\mu)_{\beta\gamma}\frac{\partial}{\partial x^\mu} \\
=&-2\gamma^\mu_{\alpha\beta}\frac{\partial}{\partial x^\mu}
\end{align*}
となる.(途中から(26.2.5)の途中計算の符号を変えただけだと気付けばすぐ終わる)よってある超場から別の超場を作る際には,超場の時空微分を入れることも可能となる.
\begin{align*}
-\frac{1}{2}\left[\{\mc{D}_\alpha,\bar{\mc{D}}_\gamma \}\gamma^\nu_{\gamma\beta}+\gamma^\nu_{\alpha\gamma}\{\mc{D}_\gamma,\bar{\mc{D}}_\beta\}\right]=&(\gamma^\mu\gamma^\nu)_{\alpha\beta}\frac{\partial}{\partial x^\mu}+(\gamma^\nu \gamma^\mu)_{\alpha\beta}\frac{\partial}{\partial x^\mu}\\
=&\frac{\partial}{\partial x^\nu}
\end{align*}
左辺が超微分から構成されており$\bar{\alpha}\mc{Q}$と可換だから,当然右辺も$\bar{\alpha}\mc{Q}$と可換となる.(まあシンプルに,(26.2.3)を見れば$\mc{Q}$と$\partial/\partial x^\mu$は可換なことはすぐわかるので,時空微分した超場もまた超場になることは一目瞭然かな.時空微分も超微分の線形結合として書けるということの方が大事かも.)

\vskip\baselineskip

超場から超対称性作用を構築する方法を考察しよう.超対称ラグランジアン密度なるものは\uwave{存在しない}!なぜなら,(26.2.6)の反交換関係から,もし任意の$\alpha$に対して$\delta \mc{L}=0$なら
\begin{align*}
0=&\delta \mc{L}=-i[\bar{\alpha}Q,\mc{L}]=(\bar{\alpha}\mc{Q})\mc{L} \quad \therefore \mc{Q}_\alpha \mc{L}=0\\
0=&\{\mc{Q}_\alpha ,\bar{\mc{Q}}_\beta\}\mc{L}=2\gamma^\mu_{\alpha\beta} \frac{\partial}{\partial x^\mu}\mc{L} \\
0=&\frac{1}{2}\left[\{\mc{Q}_\alpha ,\bar{\mc{Q}}_\gamma\}\gamma^\nu_{\gamma\beta}+\gamma^\nu_{\alpha\gamma}\{\mc{Q}_\gamma,\bar{\mc{Q}}_\beta\}\right]=\frac{\partial}{\partial x^\nu}\mc{L}
\end{align*}
となり,$\mc{L}$は定数でなくてはならないからだ.たとえ\uwave{ラグランジアン密度が}超対称でなくても,もし$\delta \mc{L}(x)$がなんらかの全微分ならば$\delta \int \mc{L} d^4x $に寄与はなく,\uwave{作用は}超対称となる.一般にラグランジアン密度$\mc{L}$は,基本的な超場とそれらの超微分(時空微分含め)とから構成されたある超場となり,よって(26.2.10)のように$C,\omega,M,N,V^\mu,\lambda,D$成分に展開することができる.各成分の変換則(26.2.11)~(26.2.17)から,ある一般的な超場に何も特別な条件が課されていなければ,変分が何かしらの微分になっているような超場の成分は$D$成分だけであることがわかる.(他を何かしらの微分だけの項にするためには,$D=0$とか$\lambda=0$などの条件が必要.この話は次の節で.)また,任意の超場の$D$成分がスカラーであるためには,その超場自身がスカラーである必要がある.したがって,ラグランジアン密度(スカラー)を構成する個々の超場に何も特別な条件が課されていなければ,超対称作用はスカラー超場$\Lambda$の$D$項の積分でなければならない.
\begin{align*}
I=\int d^4 x [\Lambda]_D
\end{align*}
これならば確かに変分は必ず微分形にしかならない.しかし実際には,この類の作用は,それを構成する超場$\Lambda$に特別な条件が課されないと物理的に妥当なものとならない.ある一般的な超場$S(x,\theta)$について,$S$と$S^*$について双線形で,成分場について3階以上の微分を含まない超対称な運動を持つ作用$I_0$の唯一の形は以下となる.
\begin{align*}
I_0 \propto \int d^4x \Bigl[S^*S \Bigr]_D
\end{align*}
(26.2.25)を用いれば,$S^*S$は$D$成分
\begin{align*}
\Bigl[S^* S\Bigr]_D=&-\partial_\mu C^* \partial^\mu C-\frac{1}{2}\left(\bar{\omega}\gamma^\mu \partial_\mu \omega\right) +\frac{1}{2}\left(\left(\partial_\mu \bar{\omega}\right)\gamma^\mu \omega\right) \\
&+C^*D +D^* C -(\bar{\omega}\lambda)-(\bar{\lambda}\omega) \\
&+M^* M +N^*N-V^*_\mu V^\mu
\end{align*}
を持つことがわかる.(作用全体を実にするために$\omega,\lambda$はマヨラナ・スピノルであることは仮定した.しかし$C$が実場だと第一項目を運動項とみなした時に係数$1/2$がついていないことが不自然だから,$C,D,M,N,V^\mu$はそれぞれ複素場とした.作用全体が実であることにこれらが実である必要はない.)$C$か$\omega$について2次の項は質量ゼロの場のラグランジアンの運動項として期待できるように見える.((7.5.34)などを参照.複素スカラー場は係数$1/2$なし.)末尾3項に特に問題はない.しかし,$D$と$\lambda$を含む項は1次の項しかないため,経路積分を評価すると発散してしまい,それを防ぐためには$C,\omega$をゼロに拘束する必要があるという困難が出る.幸運にも,次の節で見るように,拘束された超場を使って物理的に意味のある作用を構成することが可能となる!それらの拘束された超場を導入すると,超場の関数の$D$成分となっていない超対称項を作用に含めることも可能となる!

\vskip\baselineskip

パリティが保存されるなら,超場の成分場の空間反転性は超対称性によって関係づけられる.この関係を調べるには,パリティ演算子$\mathsf{P}$を(26.2.1)の交換・反交換子に施して,超対称性生成子の変換則(25.3.12)を使えば
\begin{align*}
\mathsf{P}^{-1}\left[Q,S(x,\theta)\right\}\mathsf{P}=&\left[\mathsf{P}^{-1} Q \mathsf{P},\mathsf{P}^{-1} S(x,\theta)\mathsf{P}\right\} \\
=&i\beta \left[ Q, \mathsf{P}^{-1}S(x,\theta)\mathsf{P}\right\} \quad \because (25.3.12) \\
=&i\mc{Q}\mathsf{P}^{-1}S(x,\theta)\mathsf{P} ,\quad \mc{Q}=-\frac{\partial}{\partial \bar{\theta}}+\gamma^\mu \theta \frac{\partial}{\partial x^\mu}
\end{align*}
となる.これのスカラー超場スカラー超場についての解は
\begin{align*}
\mathsf{P}^{-1}S(x,\theta)\mathsf{P} =\eta S(\Lambda_Px,-i\beta \theta)
\end{align*}
である.ここで$\eta$はある位相(超場の内部パリティ)で,$\Lambda_P x\equiv (-\mathbf{x},+x^0)$だ.実際これを代入して左辺を計算すると
\begin{align*}
\frac{\partial}{\partial \theta_\gamma}=&\frac{\partial (-i\beta \theta)_\delta}{\partial \theta_\gamma}\frac{\partial}{\partial (-i\beta \theta)_\delta}=-i\beta_{\delta\gamma}\frac{\partial}{\partial (-i\beta \theta)_\delta} \\
\frac{\partial}{\partial (-i\beta \theta)_\gamma}=&i\beta_{\gamma\delta}\frac{\partial}{\partial \theta_\delta} \\
\frac{\partial}{\partial (\Lambda_P x)^\mu}=&\tensor{\mc{P}}{_\mu^\nu}\frac{\partial}{\partial x^\nu} ,\quad \mc{P}=\mathrm{diag}(-1,-1,-1,+1)\\
i\beta_{\alpha\beta} \left[Q_\beta,\eta S(\Lambda_P x,-i\beta \theta)\right\}=&i\eta \beta _{\alpha\beta}i\left((\gamma_5 \epsilon)_{\beta\gamma}\frac{\partial}{\partial (-i\beta \theta)_\gamma}+\gamma^\mu_{\beta\gamma}(-i\beta \theta)_{\gamma}\frac{\partial}{\partial (\Lambda_P x)^\mu}\right) S(\Lambda_P x,-i\beta \theta)\\
=&-\eta \left(i(\beta\gamma_5 \epsilon \beta)_{\alpha\beta}\frac{\partial}{\partial \theta_\beta}-i(\beta\gamma^\mu\beta \theta)_{\alpha}\tensor{\mc{P}}{_\mu^\nu}\frac{\partial}{\partial x^\nu}\right)S(\Lambda_P x,-i\beta \theta) \\
=&-\eta \left(-i(\gamma_5 \epsilon)_{\alpha\beta}\frac{\partial}{\partial \theta_\beta}-i(\gamma^\mu \theta)_{\alpha}\frac{\partial}{\partial x^\mu}\right)S(\Lambda_P x,-i\beta \theta) \\
=&i\eta \left((\gamma_5 \epsilon)_{\alpha\beta}\frac{\partial}{\partial \theta_\beta}+(\gamma^\mu \theta)_{\alpha}\frac{\partial}{\partial x^\mu}\right)S(\Lambda_P x,-i\beta \theta) \\
=&i\eta \mc{Q}_{\alpha}S(\Lambda_P x,-i\beta \theta)
\end{align*}
となり解であることが分かる.最初の等号では,同じ生成子でも引数によって(例えば運動量演算子では$[P_\mu,\phi(y) ]=i\partial \phi(y)/\partial y^\mu$のように)微分や変数が対応して変化することに注意する.最後から3つ目の等号では(5.4.35)と$\beta=i\gamma^0,\mc{C}=-\epsilon \gamma_5$より$\beta (\epsilon\gamma_5 )\beta=-\epsilon\gamma_5$となること(直接行列計算したほうが正直早い)と,$\mu=0$のときだけ$\gamma^\mu$は$\beta$と交換しそれ以外は反交換することから$\beta \gamma^\mu\beta=\tensor{\mc{P}}{^\mu_\nu}\gamma^\nu$を用いた.(26.2.35)に(26.2.10)の展開を適用すると
\begin{align*}
(-i\beta\theta)^\dagger \beta=&+i\theta^\dagger=i\bar{\theta}\beta \\
S(\Lambda_P x,-i\beta\theta)=&C(\Lambda_P x)-i(\bar{\theta}i\beta \gamma_5 \omega(\Lambda_P x))-\frac{i}{2}(\bar{\theta}i\beta \gamma_5(-i\beta)\theta )M(\Lambda_P x)-\frac{1}{2}(\bar{\theta}i\beta (-i\beta)\theta)N(\Lambda_P x) \\
&+\frac{i}{2}(\bar{\theta}i\beta \gamma_5 \gamma_\mu (-i\beta)\theta)V^\mu(\Lambda_Px)-i(\bar{\theta}i\beta \gamma_5 (-i\beta)\theta)\left(\bar{\theta}i\beta \left[\lambda(\Lambda_P x)+\frac{1}{2}\gamma^\mu \left(\frac{\partial \omega}{\partial x^\mu}\right)(\Lambda_Px)\right]\right) \\
&-\frac{1}{4}(\bar{\theta}i\beta\gamma_5 (-i\beta)\theta)^2\left(D(\Lambda_Px)+\frac{1}{2}\left(\eta^{\mu\nu}\frac{\partial^2 C}{\partial x^\mu \partial x^\nu}\right)(\Lambda_Px)\right) \\
=&C(\Lambda_P x)-i(\bar{\theta} \gamma_5 \left[-i\beta \omega(\Lambda_P x)\right])-\frac{i}{2}(\bar{\theta}\gamma_5 \theta )\left[-M(\Lambda_P x)\right]-\frac{1}{2}(\bar{\theta}\theta)N(\Lambda_P x) \\
&+\frac{i}{2}(\bar{\theta}\gamma_5 \gamma_\nu \theta)\left[-\tensor{\mc{P}}{^\nu_\mu}V^\mu(\Lambda_Px)\right]-i(\bar{\theta} \gamma_5 \theta)\left(\bar{\theta} \left[-i\beta\lambda(\Lambda_P x)-\frac{1}{2}i\beta \gamma^\mu \tensor{\mc{P}}{_\mu^\nu}\frac{\partial}{\partial x^\nu}\omega(\Lambda_Px)\right]\right) \\
&-\frac{1}{4}(\bar{\theta}\gamma_5 \theta)^2\left(D(\Lambda_Px)+\frac{1}{2}\eta^{\mu\nu}\frac{\partial^2}{\partial x^\mu \partial x^\nu} C(\Lambda_Px)\right) \\
=&C(\Lambda_P x)-i(\bar{\theta} \gamma_5 \left[-i\beta \omega(\Lambda_P x)\right])-\frac{i}{2}(\bar{\theta}\gamma_5 \theta )\left[-M(\Lambda_P x)\right]-\frac{1}{2}(\bar{\theta}\theta)N(\Lambda_P x) \\
&+\frac{i}{2}(\bar{\theta}\gamma_5 \gamma_\nu \theta)\left[-\tensor{\mc{P}}{^\nu_\mu}V^\mu(\Lambda_Px)\right]-i(\bar{\theta} \gamma_5 \theta)\left(\bar{\theta} \left[-i\beta\lambda(\Lambda_P x)+\frac{1}{2}\gamma^\mu \frac{\partial}{\partial x^\mu}\left[-i\beta \omega(\Lambda_Px)\right]\right]\right) \\
&-\frac{1}{4}(\bar{\theta}\gamma_5 \theta)^2\left(D(\Lambda_Px)+\frac{1}{2}\eta^{\mu\nu}\frac{\partial^2}{\partial x^\mu \partial x^\nu} C(\Lambda_Px)\right)
\end{align*}
となる.(微分の項では$(df/dx)(y(x))\neq (d/dx)f(y(x))$であることに注意.)よって成分場の空間反転性は
\begin{align*}
\mathsf{P}^{-1}C(x)\mathsf{P}=&\eta C(\Lambda_P x) \\
\mathsf{P}^{-1}\omega(x)\mathsf{P}=&-i\eta \beta \omega(\Lambda_P x) \\
\mathsf{P}^{-1}M(x) \mathsf{P} =&-\eta M(\Lambda_P x) \\
\mathsf{P}^{-1}N(x)\mathsf{P}=&\eta N(\Lambda_Px) \\
\mathsf{P}^{-1}V^\mu(x) \mathsf{P}=&-\eta \tensor{(\Lambda_P)}{^\mu_\nu}V^\nu(\Lambda_Px) \\
\mathsf{P}^{-1}\lambda(x) \mathsf{P}=&-i\eta \beta \lambda(\Lambda_Px) \\
\mathsf{P}^{-1} D(x)\mathsf{P} =&\eta D(\Lambda_Px)
\end{align*}
となっていることがわかる.このパリティ変換則を見れば,実際に$C,N$はスカラー場であり,$M$は擬スカラー場,$V^\mu$は擬ベクトル場,$\omega,\lambda$は4成分スピノル場であることがわかる.

\vskip\baselineskip

一般の実スカラー超場$S$は4つの実スカラー(擬スカラー)場$C,M,N,D$と,1つの4元実擬ベクトル場$V_\mu$を持ち,合計8個の独立なボゾン場の成分を持つ.比較しておくと,フェルミオン場も$\omega$と$\lambda$という二つの4成分マヨラナ・スピノル場があり,合計8個の独立な場の成分がある.独立なボゾン場とフェルミオン場の成分の数が等しくなっている!これはこの節で述べた拘束されない一般の超場のみならず,次の節で論じるカイラル超場や他の拘束された超場のように超対称な拘束条件を課して得られる一般の超場について成立する性質だ!\par
これを一般的に見るために,$N_B$個の線形独立な実ボゾン場演算子$b_n(x)$と$N_F$個の線形独立なフェルミオン場演算子$f_k(x)$がなす超対称性代数の表現があるとしよう.これらの場の非自明な場の方程式のみを満たし,ゼロでない係数での$b_n$や$f_k$のどのような線形結合も斉次線形な場の方程式(自由Dirac方程式や自由Maxwell方程式のような$L(\partial)\phi=0$という形)を満たすことができないと仮定する.(つまり他の場との相互作用項や自己相互作用項があって,ラグランジアンから導かれる場の方程式は全て他の場や自己場の2次以上の項を含んでいたりして$\Box \phi=-\lambda \phi^3,(\Slash{\partial}+m)\psi=-ie\Slash{A}$のような非斉次方程式になっているとする.そしてどのような線形結合をしても上のような単純な形の微分方程式にはならないとする.)
\begin{align*}
Q(u)\equiv \Bigl(\bar{u} Q\Bigr)=\Bigl(\bar{Q}u\Bigr)
\end{align*}
ここで$u$はある通常の数値マヨラナ・スピノル(反可換c数では\uwave{ない},ディラック場の係数スピノルのようなもの)だ.(拡張超対称性では$Q_\alpha$の代わりに$Q_{r\alpha}$のどれか,例えば$Q_{1\alpha}$を使えばよい.)$b_n$と$f_k$が超対称性代数の表現になるには,ある行列微分演算子$q(\partial),p(\partial)$を用いて
\begin{align*}
\left[Q(u),b_n\right]=&i\sum_k q_{nk}(\partial) f_k \\
\left\{Q(u),f_k \right\}=&\sum_n p_{kn}(\partial)b_n
\end{align*}
という形になっていなければならない.例えば上の式の左辺はフェルミオン的だから左辺もフェルミオン的な形になっていなければならず,下の式も左辺はボゾン的になっているから右辺もそうでなくてはならないからこの形になる.係数$i$は便宜上だと思う.また,さらに$Q(u)$との交換子,反交換子をとると
\begin{align*}
\left[\{Q(u)\}^2,b_n\right]=&Q(u)\left[Q(u),b_n\right]+\left[Q(u),b_n\right]Q(u) \\
=&i\sum_k q_{nk}(\partial)\{Q(u),f_k\} \\
=&i\sum_{m}\Bigl(q(\partial)p(\partial)\Bigr)_{nm} b_m \\
\left[\{Q(u)\}^2,f_k\right]=&Q(u)\left\{Q(u),f_k\right\}-\left\{Q(u),f_k\right\}Q(u) \\
=&\sum_n p_{kn}(\partial)\left[Q(u),b_n\right] \\
=&i\sum_{l} \Bigl(p(\partial)q(\partial)\Bigr)_{kl}f_l
\end{align*}
を得る.さらに(25.2.36)の反交換関係から
\begin{align*}
\{Q(u)\}^2=&\frac{1}{2}\{Q(u)\}^2+\frac{1}{2}\{Q(u)\}^2 \\
=&\frac{1}{2}\bar{u}_\alpha Q_\alpha \bar{Q}_\beta u_\beta +\frac{1}{2}\bar{Q}_\beta u_\beta \bar{u}\beta Q_\beta \\
=&\frac{1}{2}\bar{u}_\alpha \{Q_\alpha,\bar{Q}_\beta \}u_\beta=-iP_\mu \bar{u}_\alpha \gamma^\mu_{\alpha\beta}u_\beta \\
=&-iP_\mu (\bar{u}\gamma^\mu u)
\end{align*}
(拡張超対称性の場合は(25.2.38)の反交換関係を使うが,$Q(u)$の定義でひとつの$r$を決めてしまっているので$r=s$であり,$Z_{rr}=0=Z^*_{rr}$から余計な項は消えて,全く同じ計算で同様の結果になる.)よって
\begin{align*}
(\bar{u}\gamma^\mu u)\partial_\mu b_n=&i\sum_m \Bigl(q(\partial)p(\partial)\Bigr)_{nm} b_m \\
(\bar{u}\gamma^\mu u)\partial_\mu f_k=&i\sum_l \Bigl(p(\partial)q(\partial)\Bigr)_{kl} f_l
\end{align*}
と書ける.以下で,正方行列$q(\partial)p(\partial)$と$p(\partial)q(\partial)$は両方とも正則であることがわかる.正則行列は固有値0の固有ベクトルを持たない行列であったのだから,任意のゼロでない係数$c_n(\partial),d_k(\partial)$について,$\sum_n c_n(\partial)(q(\partial)p(\partial))_{nm}=0$か$\sum_k d_k (\partial)(p(\partial)q(\partial))_{kl}=0$と仮定して矛盾を導けばいい.実際こう仮定すれば,上の関係式から
\begin{align*}
(\bar{u}\gamma^\mu u)\partial_\mu \sum_n c_n(\partial) b_n =0
\end{align*}
または
\begin{align*}
(\bar{u}\gamma^\mu u)\partial_\mu \sum_k d_k(\partial) f_k =0
\end{align*}
が導かれる.これは斉次線形な場の方程式だ.よって線形結合によって斉次線形な場の方程式を満たすようにはできないという仮定と矛盾が起き,証明完了.ここからは線形代数を使う.$N_B\times N_F$行列$q$と$N_F\times N_B$行列$p$の積$qp$($N_B\times N_B$正方行列)が正則であるならば,$\mathrm{rank}(qp)=N_B$であり,
\begin{align*}
\mathrm{rank}(qp)=N_B \leq \mathrm{rank}(p) \leq N_F
\end{align*}
となり$N_B \leq N_F$が得られる.($n\times m$行列$A$のrankは$m$以下かつ$n$以下となることを使った.)一方$N_F\times N_F$正方行列$pq$も正則であることから
\begin{align*}
\mathrm{rank}(pq)=N_F\leq \mathrm{rank}(q)\leq N_B
\end{align*}
もわかる.したがって$N_B=N_F$が結論される.(このままではフェルミオン場の成分の独立な数がボゾン場の数と同じとはまだ言えない.ボゾン場は実だと仮定したが,フェルミオン場は実場ではありえず,それぞれのフェルミオン場の複素共役が独立ならば合計のフェルミオン場の成分の数は$2N_F$となり$N_B$より多くなってしまう.)さて,このとき$p,q$はともに正方行列となり,二つの正方行列が正則であることから$p,q$が共に正則行列である.よって(26.2.38)の複素共役から
\begin{align*}
[Q(u),b^*_n]=&i\sum_k q_{nk}^*(\partial )f_k^* \\
=[Q(u),b_n]=&i\sum_k q_{nk}(\partial)f_k \quad \because b_n は実場 b^*_n=b_n\\
\therefore \quad f^*=q^{*-1}q f 
\end{align*}
がわかる.したがって複素共役$f^*$は全て$f$とは独立ではありえず,独立なフェルミオン場の数は$2N_F$ではなく$N_F$であり,独立なボゾン場の数$N_B$に等しい!これが証明したかったことだ.



\newpage



\subsection{カイラル線形超場}
前の節では,一般の超場には$D$と$\lambda$成分があるために,そのような超場を使って物理的に満足できるラグランジアン密度を構成することが困難になることを見た.そこで
\begin{align*}
\lambda=D=0
\end{align*}
となる超場を考えたらどうなるだろうか?このような条件は超対称性変換で保存されるだろうか?(26.2.17)(26.2.16)に従うと,$D=0$という条件は,もし$\lambda=0$ならば不変だ.さらに$\lambda=0$条件は,もし
\begin{align*}
[\partial_\mu \Slash{V},\gamma^\nu]=\partial_\mu V_\nu \gamma^\nu \gamma^\mu -\partial_\mu V_\nu \gamma^\mu \gamma^\nu =(\partial_\mu V_\nu-\partial_\nu V_\mu)\gamma^\mu\gamma^\nu=0
\end{align*}
となるとき,つまり$\partial_\mu V_\nu-\partial_\nu V_\mu=0$という条件も課したときにのみ不変となる.これは$V_\mu$が
\begin{align*}
V_\mu(x)=\partial_\mu Z(x)
\end{align*}
と純ゲージであることを意味する.$\lambda=0$と共に(26.2.15)を使うと,この条件が超対称性変換で
\begin{align*}
\delta V_\mu =(\bar{\alpha}\partial_\mu \omega)=\partial_\mu(\bar{\alpha} \omega)
\end{align*}
となり,$V_\mu$の変分も微分で書けるから,純ゲージ条件も保たれていることがわかる.こうして,(26.3.1)と(26.3.2)の拘束条件を満たす縮小された超場が得られる.この超場の成分場は(26.2.11)~(26.2.15)から,以下の変換則を満たす.
\begin{align*}
\delta C=&i(\bar{\alpha}\gamma_5 \omega) \\
\delta \omega=& \left(-i\gamma_5 \Slash{\partial} C-M+i\gamma_5 N+\Slash{\partial}Z\right)\alpha \\
\delta M=& -(\bar{\alpha}\Slash{\partial}\omega) \\
\delta N=& i(\bar{\alpha}\gamma_5 \Slash{\partial}\omega) \\
\delta Z =& (\bar{\alpha}\omega)
\end{align*}
これを(26.1.21)と比較してみよう.するとこれは26.1節で直接的な方法で構成した超対称多重項と同じになっていることがわかる.
\begin{align*}
\delta A=&\bar{\alpha}\psi \\
\delta B=&-i \bar{\alpha}\gamma_5 \psi \\
\delta \psi =&(\Slash{\partial}A+i\gamma_5 \Slash{\partial}B+F-i\gamma_5 G)\alpha \\
\delta F=&\bar{\alpha}\gamma^\mu \partial_\mu \psi \\
\delta G=&-i\bar{\alpha} \gamma_5 \gamma^\mu \partial_\mu \psi
\end{align*}
対応関係は以下の通りだ.
\begin{align*}
C=A,\quad \omega=-i\gamma_5 \psi,\quad M=G,\quad N=-F,\quad Z=B
\end{align*}
(ここで$C=-B,\omega=\psi,M=-F,N=-G,Z=A$と採っても対応が付くし,こっちの方が綺麗だと思うかもしれない.このようにとったのは,すぐに見るようにスカラー超場ではこれが通常の,$A,F$がスカラーで$B,G$が擬スカラーだという通常の決まりと矛盾しないからだ.)(26.3.1)と(26.3.2)の条件を満たす超場は\textbf{カイラル}であるという.超場の一般的な形(26.2.10)にカイラル条件(26.3.1)(26.3.2)と対応(26.3.8)を使うと,一般のカイラル超場の形が
\begin{align*}
X(x,\theta)=&A(x)-\Bigl(\bar{\theta}\psi(x)\Bigr)+\frac{1}{2}\Bigl(\bar{\theta}\theta\Bigr)F(x)-\frac{i}{2}\Bigl(\bar{\theta}\gamma_5 \theta\Bigr)G(x)\\
&+\frac{i}{2}\Bigl(\bar{\theta}\gamma_5 \gamma_\mu \theta\Bigr)\partial^\mu B(x)+\frac{1}{2}\Bigl(\bar{\theta}\gamma_5 \theta\Bigr)\Bigl(\bar{\theta}\gamma_5 \Slash{\partial}\psi(x)\Bigr) \\
&-\frac{1}{8}\Bigl(\bar{\theta}\gamma_5 \theta\Bigr)^2 \Box A(x)
\end{align*}
となることがわかる.\par
(26.3.9)のカイラル超場はさらに以下のように分解することができる.
\begin{align*}
X(x,\theta)=\frac{1}{\sqrt{2}}\left[\Phi(x,\theta)+\tilde{\Phi}(x,\theta)\right]
\end{align*}
ここで
\begin{align*}
\Phi(x,\theta)=&\phi(x)-\sqrt{2}\Bigl(\bar{\theta}\psi_L(x)\Bigr)+\mc{F}(x)\left(\bar{\theta}\left(\frac{1+\gamma_5}{2}\right)\theta\right) \\
&+\frac{1}{2}\Bigl(\bar{\theta}\gamma_5 \gamma_\mu \theta\Bigr)\partial^\mu \phi(x)-\frac{1}{\sqrt{2}}\Bigl(\bar{\theta}\gamma_5 \theta\Bigr)\Bigl(\bar{\theta}\Slash{\partial}\psi_L(x)\Bigr) \\
&-\frac{1}{8}\Bigl(\bar{\theta}\gamma_5 \theta\Bigr)^2 \Box \phi(x) \\
\tilde{\Phi}(x,\theta)=&\tilde{\phi}(x)-\sqrt{2}\Bigl(\bar{\theta}\psi_R(x)\Bigr)+\tilde{\mc{F}}(x)\left(\bar{\theta}\left(\frac{1+\gamma_5}{2}\right)\theta\right) \\
&-\frac{1}{2}\Bigl(\bar{\theta}\gamma_5 \gamma_\mu \theta\Bigr)\partial^\mu \tilde{\phi}(x)+\frac{1}{\sqrt{2}}\Bigl(\bar{\theta}\gamma_5 \theta\Bigr)\Bigl(\bar{\theta}\Slash{\partial}\psi_R(x)\Bigr) \\
&-\frac{1}{8}\Bigl(\bar{\theta}\gamma_5 \theta\Bigr)^2 \Box \tilde{\phi}(x)
\end{align*}
であり,成分場は
\begin{align*}
&\phi\equiv \frac{A+iB}{\sqrt{2}},\quad \psi_L\equiv \left(\frac{1+\gamma_5}{2}\right)\psi ,\quad \mc{F}\equiv \frac{F-iG}{\sqrt{2}} \\
&\tilde{\phi}\equiv \frac{A-iB}{\sqrt{2}},\quad \psi_R\equiv \left(\frac{1-\gamma_5}{2}\right)\psi ,\quad \tilde{\mc{F}}\equiv \frac{F+iG}{\sqrt{2}}
\end{align*}
で定義される.実際にこれらは
\begin{align*}
&\frac{1}{\sqrt{2}}\left[\Phi+\tilde{\Phi}\right] \\
=&\frac{1}{\sqrt{2}}(\phi+\tilde{\phi})-\Bigl(\bar{\theta}(\psi_L+\psi_R)\Bigr)+\frac{1}{\sqrt{2}}(\mc{F}+\tilde{\mc{F}})\frac{1}{2}\Bigl(\bar{\theta}\theta\Bigr)+\frac{1}{\sqrt{2}}(\mc{F}-\tilde{\mc{F}})\frac{1}{2}\Bigl(\bar{\theta}\gamma_5 \theta\Bigr) \\
&+\frac{1}{2\sqrt{2}}\Bigl(\bar{\theta}\gamma_5 \gamma_\mu \theta\Bigr)\partial^\mu (\phi-\tilde{\phi})-\frac{1}{2}\Bigl(\bar{\theta}\gamma_5 \theta\Bigr)\Bigl(\bar{\theta}\Slash{\partial}(\psi_L-\psi_R)\Bigr) \\
&-\frac{1}{8\sqrt{2}}\Bigl(\bar{\theta}\gamma_5 \theta\Bigr)^2 \Box (\phi+\tilde{\phi}) \\
=&A-\Bigl(\bar{\theta}\psi\Bigr)+\frac{1}{2}\Bigl(\bar{\theta}\theta\Bigr)F-\frac{i}{2}\Bigl(\bar{\theta}\gamma_5 \theta\Bigr)G \\
&+\frac{i}{2}\Bigl(\bar{\theta}\gamma_5 \gamma_\mu \theta\Bigr)\partial^\mu B-\frac{1}{2}\Bigl(\bar{\theta}\gamma_5 \theta\Bigr)\Bigl(\bar{\theta}\Slash{\partial}\gamma_5 \psi\Bigr)  \\
&-\frac{1}{8}\Bigl(\bar{\theta}\gamma_5 \theta\Bigr)^2 \Box \phi \\
=&A-\Bigl(\bar{\theta}\psi\Bigr)+\frac{1}{2}\Bigl(\bar{\theta}\theta\Bigr)F-\frac{i}{2}\Bigl(\bar{\theta}\gamma_5 \theta\Bigr)G \\
&+\frac{i}{2}\Bigl(\bar{\theta}\gamma_5 \gamma_\mu \theta\Bigr)\partial^\mu B+\frac{1}{2}\Bigl(\bar{\theta}\gamma_5 \theta\Bigr)\Bigl(\bar{\theta}\gamma_5\Slash{\partial} \psi\Bigr)  \\
&-\frac{1}{8}\Bigl(\bar{\theta}\gamma_5 \theta\Bigr)^2 \Box \phi \\
=&X
\end{align*}
となって,分解が正しくできていることが確認できる.$\Phi$か$\tilde{\Phi}$の成分場は
\begin{align*}
\delta \psi_L=& \left(\frac{1+\gamma_5}{2}\right)\delta \psi \\
=&\left(\frac{1+\gamma_5}{2}\right)(\Slash{\partial}A+i\gamma_5 \Slash{\partial}B+F-i\gamma_5 G)\alpha \\
=&(\Slash{\partial}A+i\Slash{\partial}B)\left(\frac{1-\gamma_5}{2}\right)\alpha+(F-i G)\left(\frac{1+\gamma_5}{2}\right)\alpha \quad \because \gamma_5 \left(\frac{1\pm \gamma_5}{2}\right)=\pm\left(\frac{1\pm\gamma_5}{2}\right) \\
=&\sqrt{2} \partial_\mu \phi \gamma^\mu \alpha_R+ \sqrt{2}\mc{F} \alpha_L \\
\delta \mc{F}=&\frac{1}{\sqrt{2}}\delta F -\frac{i}{\sqrt{2}}\delta G \\
=&\frac{1}{\sqrt{2}}\bar{\alpha} \gamma^\mu \partial_\mu \psi -\frac{1}{\sqrt{2}}\bar{\alpha}\gamma_5 \gamma^\mu \partial_\mu \psi \\
=&\sqrt{2}\bar{\alpha}\gamma^\mu \left(\frac{1+\gamma_5}{2}\right)\partial_\mu \psi \\
=&\sqrt{2}\Bigl(\overline{\alpha_L} \Slash{\partial}\psi_L\Bigr) \quad \because \overline{\alpha_L}\gamma^\mu=\alpha^\dagger \left(\frac{1+\gamma_5}{2}\right)\beta\gamma^\mu=\bar{\alpha}\left(\frac{1-\gamma_5}{2}\right)\gamma^\mu =\bar{\alpha}\gamma^\mu\left(\frac{1+\gamma_5}{2}\right) \\
\delta \phi=& \frac{1}{\sqrt{2}}\delta A+\frac{i}{\sqrt{2}}\delta B \\
=&\frac{1}{\sqrt{2}}\bar{\alpha}\psi+\frac{1}{\sqrt{2}}\bar{\alpha}\gamma_5 \psi \\
=&\sqrt{2}\bar{\alpha} \left(\frac{1+\gamma_5}{2}\right)\psi \\
=&\sqrt{2}\Bigl(\overline{\alpha_R}\psi_L\Bigr) \\
\delta \psi_R=&  \left(\frac{1-\gamma_5}{2}\right)\delta \psi \\
=&\left(\frac{1-\gamma_5}{2}\right)(\Slash{\partial}A+i\gamma_5 \Slash{\partial}B+F-i\gamma_5 G)\alpha \\
=&(\Slash{\partial}A-i\Slash{\partial}B)\left(\frac{1+\gamma_5}{2}\right)\alpha+(F+i G)\left(\frac{1-\gamma_5}{2}\right)\alpha \quad \because \gamma_5 \left(\frac{1\pm \gamma_5}{2}\right)=\pm\left(\frac{1\pm\gamma_5}{2}\right) \\
=&\sqrt{2} \partial_\mu \tilde{\phi} \gamma^\mu \alpha_L+ \sqrt{2}\tilde{\mc{F}} \alpha_R \\
\delta \tilde{\mc{F}}=&\frac{1}{\sqrt{2}}\delta F +\frac{i}{\sqrt{2}}\delta G \\
=&\frac{1}{\sqrt{2}}\bar{\alpha} \gamma^\mu \partial_\mu \psi +\frac{1}{\sqrt{2}}\bar{\alpha}\gamma_5 \gamma^\mu \partial_\mu \psi \\
=&\sqrt{2}\bar{\alpha}\gamma^\mu \left(\frac{1-\gamma_5}{2}\right)\partial_\mu \psi \\
=&\sqrt{2}\Bigl(\overline{\alpha_R} \Slash{\partial}\psi_R\Bigr) \\
\delta \tilde{\phi}=& \frac{1}{\sqrt{2}}\delta A-\frac{i}{\sqrt{2}}\delta B \\
=&\frac{1}{\sqrt{2}}\bar{\alpha}\psi-\frac{1}{\sqrt{2}}\bar{\alpha}\gamma_5 \psi \\
=&\sqrt{2}\bar{\alpha} \left(\frac{1-\gamma_5}{2}\right)\psi \\
=&\sqrt{2}\Bigl(\overline{\alpha_L}\psi_R\Bigr) 
\end{align*}
となり,超対称性代数の完全な表現を与える.ここでいつものように
\begin{align*}
\alpha_L=\left(\frac{1+\gamma_5}{2}\right)\alpha ,\quad \alpha_R=\left(\frac{1-\gamma_5}{2}\right)\alpha
\end{align*}
である.(26.3.11)の$\Phi$,もしくは(26.3.12)の$\tilde{\Phi}$の形の超場は,それぞれ\textbf{左カイラル},もしくは\textbf{右カイラル}と呼ばれる.カイラル超場$X(x,\theta)$が実という特別な場合には,その左カイラル部分$\Phi$と右カイラル部分$\tilde{\Phi}$は互いに複素共役であり,$\tilde{\phi}=\phi^*,\tilde{\mc{F}}=\mc{F}^*$となっていて,$\psi$はマヨラナ場となる.しかし,もし$X(x,\theta)$が実であることを要求しなければ,一般には$\Phi$と$\tilde{\Phi}$には関係はない.この二つのうち一方がゼロとなることさえ可能となる.\par
超場$\Phi$の成分場には,$\phi$と$\mc{F}$の二つの複素ボゾン成分,あるいは4つの独立な実ボゾン成分$A,B,F,G$,それと4成分を持つ1つのマヨラナ・フェルミオン場$\psi$が含まれる.これは前の節の最後に導いた,超対称性代数の表現をなす場の組は同じ数だけの独立なボゾン成分とフェルミオン成分をもつという一般的な結果の別の例となっている. \par
(ここで少し$\overline{\alpha_L}$や$\overline{\alpha_R}$についての注意.$\bar{\alpha} \frac{1+\gamma_5}{2}=\overline{\alpha_R},\bar{\alpha}\frac{1-\gamma_5}{2}=\overline{\alpha_L}$は正しいが,$\alpha_L,\alpha_R$を勝手にマヨラナフェルミオンとみなした$\overline{\alpha_L}=\alpha^T_L \epsilon\gamma_5$などの関係式は正しくない.正しい関係式は
\begin{align*}
\overline{\alpha_L}=\alpha_R^T \epsilon \gamma_5,\quad \overline{\alpha_R}=\alpha^T_L \epsilon\gamma_5
\end{align*}
だ.これはマヨラナフェルミオンの関係式$\bar{\alpha}=\alpha^T \epsilon\gamma_5$の両辺右側から$(1\pm \gamma_5)/2$を作用させて$\gamma_5\epsilon=\epsilon \gamma_5$と交換できることを使えばすぐわかる.4成分フェルミオンの上2成分だけあるいは下2成分だけを抜き出す$(1\pm\gamma_5)/2$によって,$\alpha_L,\alpha_R$はもはやマヨラナフェルミオンの性質が満たされなくなるからだ.)

\vskip\baselineskip

なぜ「左カイラル」とか「右カイラル」と言うのか,を以下でみる.(26.A.5)(26.A.17)(26.A.18)を使って,(26.3.11)(26.3.12)を書きなおすと
\begin{align*}
x^\mu_+=&x^\mu+\frac{1}{2}(\bar{\theta}\gamma_5 \gamma^\mu \theta) \\
\phi(x)=&\phi(x_+)-\frac{1}{2}(\bar{\theta}\gamma_5 \gamma^\mu \theta)\partial_\mu \phi(x_+)+\frac{1}{8}(\bar{\theta}\gamma_5 \gamma^\mu \theta)(\bar{\theta}\gamma_5 \gamma^\nu \theta) \partial_\mu \partial_\nu \phi(x_+) \\
=&\phi(x_+)-\frac{1}{2}(\bar{\theta}\gamma_5 \gamma^\mu \theta)\partial_\mu \phi(x_+)-\frac{1}{8}(\bar{\theta}\gamma_5 \theta)^2 \Box \phi(x_+) \quad \because (26.A.18)\\
\psi(x)=&\psi(x_+)-\frac{1}{2}(\bar{\theta}\gamma_5 \gamma^\mu \theta)\partial_\mu \psi(x_+) +\mc{O}(\theta^4) \\
\mc{F}(x)=&\mc{F}(x_+)-\frac{1}{2}(\bar{\theta}\gamma_5 \gamma^\mu \theta)\partial_\mu\mc{F}(x_+)+\mc{O}(\theta^4)
\end{align*}
を用いて
\begin{align*}
\Phi(x,\theta)=&\phi(x)+\frac{1}{2}(\bar{\theta}\gamma_5 \gamma^\mu \theta)\partial_\mu\phi(x)-\frac{1}{8}(\bar{\theta}\gamma_5 \theta)^2 \Box \phi(x) \\
&-\sqrt{2}(\bar{\theta}\psi_L(x))-\frac{1}{\sqrt{2}}(\bar{\theta}\gamma_5 \theta)(\bar{\theta}\Slash{\partial}\psi_L(x)) \\
&+\mc{F}(x)\left(\bar{\theta}\left(\frac{1+\gamma_5}{2}\right)\theta\right) \\
=&\phi(x_+)-\frac{1}{2}(\bar{\theta}\gamma_5 \gamma^\mu \theta)\partial_\mu \phi(x_+)-\frac{1}{8}(\bar{\theta}\gamma_5 \theta)^2 \Box \phi(x_+) \\
&+\frac{1}{2}(\bar{\theta}\gamma_5 \gamma^\mu \theta)\partial_\mu \phi(x_+)-\frac{1}{4}(\bar{\theta}\gamma_5\gamma^\mu \theta)(\bar{\theta}\gamma_5\gamma^\nu \theta) \partial_\nu\partial_\mu \phi(x_+) \\
&-\frac{1}{8}(\bar{\theta}\gamma_5 \theta)^2 \Box \phi(x_+) \\
&-\sqrt{2}(\bar{\theta}\psi_L(x_+))+\frac{1}{\sqrt{2}}(\bar{\theta}\gamma_5 \gamma^\mu \theta)(\bar{\theta}\partial_\mu \psi_L(x_+)) \\
&-\frac{1}{\sqrt{2}}(\bar{\theta}\gamma_5 \theta)(\bar{\theta}\gamma^\mu \partial_\mu \psi_L(x_+)) \\
&+\mc{F}(x_+)\left(\bar{\theta}\left(\frac{1+\gamma_5}{2}\right)\theta\right) -\frac{1}{2}(\bar{\theta}\gamma_5 \gamma^\mu \theta)\left(\bar{\theta}\left(\frac{1+\gamma_5}{2}\right)\theta\right) \partial_\mu\mc{F}(x_+) \\
=&\phi(x_+)-\sqrt{2}(\bar{\theta}\psi_L(x_+))+\mc{F}(x_+)\left(\bar{\theta}\left(\frac{1+\gamma_5}{2}\right)\theta\right) \\
=&\phi(x_+)-\sqrt{2}(\theta^T_L\epsilon \psi_L(x_+))+\mc{F}(x_+)\left(\theta^T_L \epsilon \theta_L\right)
\end{align*}
最初の等号では上の展開を代入し,$\theta$について4次以上の項は消えることを用いた.次の等号では,$\phi$についての項はマヨラナスピノルの性質(26.A.18)によりほとんどがキャンセルし,$\psi$についての項は(26.A.17)(26.A.7)より$(\bar{\theta}\gamma_5 \gamma^\mu \theta)(\bar{\theta} \partial_\mu \psi_L)=+(\bar{\theta}\gamma_5 \theta)(\bar{\theta}\gamma^\mu \partial_\mu \psi_L)$となることを用いて,$\mc{F}$についての項では(26.2.25)で用いた性質
\begin{align*}
(\bar{\theta}\gamma_5 \theta )(\bar{\theta}\gamma_5 \gamma_\mu \theta)=&0 \\
(\bar{\theta}\theta)(\bar{\theta}\gamma_5 \gamma_\mu \theta)=&-(\bar{\theta}\gamma_\mu \theta)(\bar{\theta}\gamma_5 \theta)=0 
\end{align*}
を使って最後の項がゼロとなることを用いた.最後の等号ではマヨラナスピノルの性質$\bar{\theta}=\theta^T\epsilon\gamma_5$と$\gamma_5 \theta_L=+\theta_L$を用いた.かなりきれいになるなぁ.$\tilde{\Phi}$についても同様に書き換えていく.
\begin{align*}
x^\mu_-=&x^\mu-\frac{1}{2}(\bar{\theta}\gamma_5 \gamma^\mu \theta) \\
\tilde{\phi}(x)=&\tilde{\phi}(x_-)+\frac{1}{2}(\bar{\theta}\gamma_5 \gamma^\mu \theta)\partial_\mu \tilde{\phi}(x_-)+\frac{1}{8}(\bar{\theta}\gamma_5 \gamma^\mu \theta)(\bar{\theta}\gamma_5 \gamma^\nu \theta) \partial_\mu \partial_\nu \tilde{\phi}(x_-) \\
=&\tilde{\phi}(x_-)+\frac{1}{2}(\bar{\theta}\gamma_5 \gamma^\mu \theta)\partial_\mu \tilde{\phi}(x_-)-\frac{1}{8}(\bar{\theta}\gamma_5 \theta)^2 \Box \tilde{\phi}(x_-) \\
\psi(x)=&\psi(x_-)+\frac{1}{2}(\bar{\theta}\gamma_5 \gamma^\mu \theta)\partial_\mu \psi(x_-) +\cdots \\
\mc{F}(x)=&\mc{F}(x_-)+\frac{1}{2}(\bar{\theta}\gamma_5 \gamma^\mu \theta)\partial_\mu\mc{F}(x_-)+\cdots 
\end{align*}
を用いて
\begin{align*}
\tilde{\Phi}(x,\theta)=&\tilde{\phi}(x)-\frac{1}{2}(\bar{\theta}\gamma_5 \gamma^\mu \theta)\partial_\mu\tilde{\phi}(x)-\frac{1}{8}(\bar{\theta}\gamma_5 \theta)^2 \Box \tilde{\phi}(x) \\
&-\sqrt{2}(\bar{\theta}\psi_R(x))+\frac{1}{\sqrt{2}}(\bar{\theta}\gamma_5 \theta)(\bar{\theta}\Slash{\partial}\psi_R(x)) \\
&+\tilde{\mc{F}}(x)\left(\bar{\theta}\left(\frac{1-\gamma_5}{2}\right)\theta\right) \\
=&\phi(x_-)+\frac{1}{2}(\bar{\theta}\gamma_5 \gamma^\mu \theta)\partial_\mu \tilde{\phi}(x_-)-\frac{1}{8}(\bar{\theta}\gamma_5 \theta)^2 \Box \tilde{\phi}(x_-) \\
&-\frac{1}{2}(\bar{\theta}\gamma_5 \gamma^\mu \theta)\partial_\mu \tilde{\phi}(x_-)-\frac{1}{4}(\bar{\theta}\gamma_5\gamma^\mu \theta)(\bar{\theta}\gamma_5\gamma^\nu \theta) \partial_\nu\partial_\mu \tilde{\phi}(x_-) \\
&-\frac{1}{8}(\bar{\theta}\gamma_5 \theta)^2 \Box \tilde{\phi}(x_+) \\
&-\sqrt{2}(\bar{\theta}\psi_R(x_-))-\frac{1}{\sqrt{2}}(\bar{\theta}\gamma_5 \gamma^\mu \theta)(\bar{\theta}\partial_\mu \psi_R(x_-)) \\
&+\frac{1}{\sqrt{2}}(\bar{\theta}\gamma_5 \theta)(\bar{\theta}\gamma^\mu \partial_\mu \psi_R(x_-)) \\
&+\tilde{\mc{F}}(x_-)\left(\bar{\theta}\left(\frac{1-\gamma_5}{2}\right)\theta\right) -\frac{1}{2}(\bar{\theta}\gamma_5 \gamma^\mu \theta)\left(\bar{\theta}\left(\frac{1-\gamma_5}{2}\right)\theta\right) \partial_\mu\tilde{\mc{F}}(x_-) \\
=&\tilde{\phi}(x_-)-\sqrt{2}(\bar{\theta}\psi_R(x_-))+\tilde{\mc{F}}(x_-)\left(\bar{\theta}\left(\frac{1-\gamma_5}{2}\right)\theta\right) \\
=&\tilde{\phi}(x_-)+\sqrt{2}(\theta^T_R\epsilon \psi_R(x_-))-\tilde{\mc{F}}(x_-)\left(\theta^T_R \epsilon \theta_R\right)
\end{align*}
となる.きれいだねぇ.一般的な形から特に条件を設けずに変形できたから,$\theta_L$と$x^\mu_+$のみに依存し$\theta_R$には依存しない超場は必ず(26.3.21)の形にならなければならず,$\theta_R$と$x^\mu_-$にのみ依存し$\theta_L$に依存しない超場は(26.3.22)の形にならなければならない!\par
超微分の右巻き部分と左巻き部分を
\begin{align*}
\mc{D}_{R\alpha} \equiv& \left[\left(\frac{1-\gamma_5}{2}\right) \mc{D}\right]_{\alpha} \\
=&\left(\frac{1-\gamma_5}{2}\right)_{\alpha\beta} (\gamma_5 \epsilon)_{\beta\gamma}\frac{\partial}{\partial \theta_\gamma}-\left(\frac{1-\gamma_5}{2}\right)_{\alpha\beta}\gamma^\mu_{\beta\gamma}\theta_\gamma\frac{\partial}{\partial x^\mu} \\
=&-\epsilon_{\alpha \beta}\left(\frac{1-\gamma_5}{2}\right)_{\beta\gamma} \frac{\partial}{\partial \theta_\gamma}-\gamma^\mu_{\alpha\beta} \left(\frac{1+\gamma_5}{2}\right)_{\beta\gamma}\theta_\gamma \frac{\partial}{\partial x^\mu} \quad \because \epsilon\gamma_5 =\gamma_5 \epsilon \\
=&-\sum_\beta \epsilon_{\alpha\beta}\frac{\partial}{\partial \theta_{R\beta}}-(\gamma^\mu \theta_L)_\alpha \frac{\partial}{\partial x^\mu} \\
\mc{D}_{L\alpha}\equiv &\left[\left(\frac{1+\gamma_5}{2}\right) \mc{D}\right]_{\alpha} \\
=&\left(\frac{1+\gamma_5}{2}\right)_{\alpha\beta} (\gamma_5 \epsilon)_{\beta\gamma}\frac{\partial}{\partial \theta_\gamma}-\left(\frac{1+\gamma_5}{2}\right)_{\alpha\beta}\gamma^\mu_{\beta\gamma}\theta_\gamma\frac{\partial}{\partial x^\mu} \\
=&+\epsilon_{\alpha \beta}\left(\frac{1+\gamma_5}{2}\right)_{\beta\gamma} \frac{\partial}{\partial \theta_\gamma}-\gamma^\mu_{\alpha\beta} \left(\frac{1-\gamma_5}{2}\right)_{\beta\gamma}\theta_\gamma \frac{\partial}{\partial x^\mu} \\
=&+\sum_\beta \epsilon_{\alpha\beta}\frac{\partial}{\partial \theta_{L\beta}}-(\gamma^\mu \theta_R)_\alpha \frac{\partial}{\partial x^\mu}
\end{align*}
と書く.これを用いると,まず$x^\mu_+,x^\mu_-$の書き換えをして
\begin{align*}
x^\mu_\pm=&x^\mu\pm\frac{1}{2}(\bar{\theta}\gamma_5 \gamma^\mu \theta) \\
=&x^\mu \pm \frac{1}{2}(\theta^T \epsilon \gamma^\mu \theta) \\
=&x^\mu \pm \frac{1}{2}((\theta_L+\theta_R)^T \epsilon \gamma^\mu (\theta_L+\theta_R)) \\
=&x^\mu \pm \frac{1}{2}(\theta^T_L \epsilon \gamma^\mu \theta_R)\pm \frac{1}{2}(\theta^T_R \epsilon \gamma^\mu \theta_L) \quad \because (\theta^T_L \epsilon \gamma^\mu \theta_L) =(\theta^T_R \epsilon \gamma^\mu \theta_R)=0\\
=&x^\mu\pm (\theta^T_R \epsilon \gamma^\mu \theta_L)=x^\mu\pm (\theta^T_L \epsilon \gamma^\mu \theta_R)
\end{align*}
最後の行への変形は(26.A.7)より
\begin{align*}
(\theta^T_L \epsilon \gamma^\mu \theta_R)=&\left(\theta^T \left(\frac{1+\gamma_5}{2}\right) \epsilon \gamma_\mu \left(\frac{1-\gamma_5}{2}\right) \theta \right)=\left(\theta^T \epsilon \gamma_\mu \left(\frac{1-\gamma_5}{2}\right) \theta \right) \\
=&\left(\bar{\theta}\gamma_5 \gamma^\mu \left(\frac{1-\gamma_5}{2}\right) \theta \right)=\frac{1}{2}(\bar{\theta}\gamma_5 \gamma^\mu \theta)+\frac{1}{2}(\bar{\theta}\gamma^\mu \theta) \\
=&\frac{1}{2}(\bar{\theta}\gamma_5 \gamma^\mu \theta) \quad \because (26.A.7)\\
=&\left(\bar{\theta}\gamma_5 \gamma^\mu \left(\frac{1+\gamma_5}{2}\right) \theta \right)=(\theta^T_R \epsilon \gamma^\mu \theta_L)
\end{align*}
を用いた.これを用いると
\begin{align*}
\mc{D}_{R\alpha} x^\mu_+=&-\sum_\beta \epsilon_{\alpha\beta}\frac{\partial}{\partial \theta_{R\beta}} (\theta^T_R \epsilon \gamma^\mu \theta_L)-(\gamma^\nu \theta_L)_\alpha \frac{\partial}{\partial x^\nu} x^\mu\\
=&+(\gamma^\mu \theta_L)_{\alpha}-(\gamma^\mu\theta_L)_\alpha \\
=&0 \\
\mc{D}_{L\alpha}x^\mu_- =&+\sum_\beta \epsilon_{\alpha\beta}\frac{\partial}{\partial \theta_{L\beta}} (-\theta^T_L \epsilon \gamma^\mu \theta_R) -(\gamma^\nu \theta_R)_\alpha \frac{\partial}{\partial x^\nu}x^\mu \\
=&+(\gamma^\mu \theta_R)_{\alpha}-(\gamma^\mu \theta_R)_{\alpha} \\
=&0
\end{align*}
となる.さらに当然$\mc{D}_{R\alpha}\theta_{L\beta}=0,\mc{D}_{L\alpha} \theta_{R\beta}=0$だ.右カイラル超場$\Phi(x,\theta)$の表式(26.3.21)を見れば,これは$x^\mu_+,\theta_L$にのみ依存しており,したがって$\mc{D}_{R\alpha}$によって必ずゼロとなる.
\begin{align*}
\mc{D}_{R\alpha}\Phi=0
\end{align*}
同様に$\tilde{\Phi}(x,\theta)$は$x^\mu_-,\theta_R$にのみ依存しており,したがって$\mc{D}_{L\alpha}$によって必ずゼロとなる.
\begin{align*}
\mc{D}_{L\alpha}\tilde{\Phi}=0
\end{align*}
すぐ前に述べた通り,$\theta_L,x^\mu_+$にのみ依存し$\theta_R$には依存しない超場は必ず(26.3.21)の形になり,$\theta_R,x^\mu$にのみ依存し$\theta_L$には(26.3.22)の形になるのだったから,この議論から逆にもし超場$\Phi$が$\mc{D}_{R}\Phi=0$を満たすならばそれは左カイラルであり,もし$\mc{D}_{L}\Phi=0$を満たすならばそれは右カイラルである!右カイラル超微分によってゼロになるような超場は左カイラルなのであり,左カイラル超微分によってゼロになるような超場が右カイラルなのである.さらに,超場$\Phi_n$が全て$\mc{D}_R\Phi_n=0$を満たすか,もしくは$\mc{D}_L\Phi_n=0$を満たすとき,それらの任意の関数$f(\Phi)$は$\mc{D}_Rf(\Phi)=0$もしくは$\mc{D}_Lf(\Phi)=0$を満たす.したがって,左カイラル超場から作られる関数は再び左カイラル,そして右カイラル超場から作られる関数も再び右カイラルだ.しかし左カイラル超場と右カイラル超場\uwave{両方}から作られる関数は一般にはカイラルですらない.\par
以上の話をまとめると,左カイラル超場(もしくは右カイラル超場)の任意の関数で,その複素共役や時空微分を含まないものは再び左(もしくは右)カイラル超場だ,となる.左カイラル超場の複素共役は右カイラル超場で,右カイラル超場の複素共役は左カイラル超場となることも示すことができる.これは関係式(26.3.24)を複素共役し,超微分はマヨラナであるから
\begin{align*}
\bar{\mc{D}}_{L}=&\mc{D}^T_R \epsilon \gamma_5 =\epsilon \mc{D}_{R} \quad \therefore \mc{D}_{L\alpha}^*=(\beta \epsilon)_{\alpha\beta} \mc{D}_{R\beta}\\
\bar{\mc{D}}_{R}=&\mc{D}^T_L \epsilon \gamma_5 =-\epsilon \mc{D}_{L} \quad \therefore \mc{D}_{R\alpha}^*=-(\beta \epsilon)_{\alpha\beta} \mc{D}_{L\beta}
\end{align*}
を用いれば,
\begin{align*}
0=&\left(\mc{D}_{R\alpha} \Phi\right)^*=-(\beta\epsilon)_{\alpha\beta}\mc{D}_{L\beta} \Phi^* \quad \therefore \mc{D}_{L\alpha} \Phi^*=0 \\
0=&\left(\mc{D}_{L\alpha} \tilde{\Phi}\right)^*=(\beta\epsilon)_{\alpha\beta}\mc{D}_{R\beta} \tilde{\Phi}^* \quad \therefore \mc{D}_{R\alpha} \tilde{\Phi}^*=0
\end{align*}
がわかるからだ.また,左カイラル超場の超微分されたものは左カイラル超場にはならない.
左カイラル超場の表現(26.3.21)を使うと,容易にそれらの積の性質を調べることができる.たとえば,もし$\Phi_1$と$\Phi_2$が二つの左カイラル超場ならば,それらの積$\Phi=\Phi_1\Phi_2$は再び左カイラル超場で
\begin{align*}
\Phi=&\left[\phi_1-\sqrt{2}(\theta^T_L \epsilon \psi_{1L})+\mc{F}_1\left(\theta_L^T \epsilon \theta_L\right)\right] \left[\phi_2-\sqrt{2}(\theta^T_L \epsilon \psi_{2L})+\mc{F}_2\left(\theta_L^T \epsilon \theta_L\right)\right] \\
=&\phi_1\phi_2 -\sqrt{2}(\theta^T_L\epsilon \left[\phi_1 \psi_{2L}+\phi_2 \psi_{1L}\right])+2(\theta^T_L \epsilon \psi_{1L})(\theta^T_L\epsilon \psi_{2L}) \\
&+(\phi_1 \mc{F}_2+\mc{F}_1 \phi_2)\left(\theta_L^T\epsilon \theta_L\right) -\sqrt{2}\mc{F}_1(\theta^T_L \epsilon \psi_{2L})\left(\theta_L^T\epsilon \theta_L\right)-\sqrt{2}\mc{F}_2 (\theta^T_L \epsilon \psi_{1L})\left(\theta_L^T\epsilon \theta_L\right) \\
+&\mc{F}_1 \mc{F}_2 \left(\theta_L^T\epsilon \theta_L\right)^2 \\
=&\phi_1\phi_2 -\sqrt{2}(\theta^T_L\epsilon \left[\phi_1 \psi_{2L}+\phi_2 \psi_{1L}\right])+2(\theta^T_L \epsilon \psi_{1L})(\theta^T_L\epsilon \psi_{2L}) \\
&+(\phi_1 \mc{F}_2+\mc{F}_1 \phi_2)\left(\theta_L^T\epsilon \theta_L\right) \\
=&\phi_1\phi_2 -\sqrt{2}(\theta^T_L\epsilon \left[\phi_1 \psi_{2L}+\phi_2 \psi_{1L}\right]) \\
&+(\phi_1 \mc{F}_2+\mc{F}_1 \phi_2-(\psi_{1L}^T \epsilon \psi_{2L}))\left(\theta_L^T\epsilon \theta_L\right) 
\end{align*}
ここで$\theta_L$について3次以上の項は必ず消えること(p146参照)を用いて,さらに(26.A.11)から
\begin{align*}
(\theta^T_L \epsilon \psi_{1L})(\theta^T_L\epsilon \psi_{2L}) =&(\theta^T \epsilon \psi_{1L})(\theta^T\epsilon \psi_{2L}) \\
=&-\theta_{\alpha} \theta_{\beta}\left(\epsilon \psi_{1L}\right)_\alpha \left(\epsilon \psi_{2L}\right)_\beta \quad (フェルミオン入れ替えでマイナス) \\
=&-\frac{1}{4}\left(\epsilon \gamma_5 \right)_{\alpha\beta}(\theta^T \epsilon \gamma_5 \theta)\left(\epsilon \psi_{1L}\right)_\alpha \left(\epsilon \psi_{2L}\right)_\beta -\frac{1}{4} (\gamma_\mu \epsilon)_{\alpha\beta}(\theta^T\epsilon \gamma^\mu \theta)\left(\epsilon \psi_{1L}\right)_\alpha \left(\epsilon \psi_{2L}\right)_\beta \\
&-\frac{1}{4}\epsilon_{\alpha\beta}(\theta^T \epsilon \theta)\left(\epsilon \psi_{1L}\right)_\alpha \left(\epsilon \psi_{2L}\right)_\beta \\
=&-\frac{1}{4}(\psi_{1L}^T\epsilon^T \epsilon \gamma_5 \epsilon \psi_{2L})(\theta^T \epsilon \gamma_5 \theta) -\frac{1}{4}(\psi^T_{1L} \epsilon^T \gamma_\mu \epsilon \epsilon \psi_{2L})(\theta^T \epsilon \gamma^\mu \theta) \\
&-\frac{1}{4}(\psi_{1L}\epsilon^T \epsilon \epsilon \psi_{2L})(\theta^T \epsilon \theta) \\
=&-\frac{1}{4}(\psi_{1L}^T \epsilon \psi_{2L})(\theta^T \epsilon \gamma_5 \theta) -\frac{1}{4}(\psi^T_{1L} \epsilon \gamma_\mu \psi_{2L})(\theta^T \epsilon \gamma^\mu \theta) \\
&-\frac{1}{4}(\psi_{1L} \epsilon \psi_{2L})(\theta^T \epsilon \theta) \\
=&-\frac{1}{2}(\psi_{1L}^T \epsilon \psi_{2L})\left(\theta^T \epsilon \left(\frac{1+\gamma_5}{2}\right) \theta\right) \quad \because (\psi^T_{1L} \epsilon \gamma_\mu \psi_{2L})=0 \\
=&-\frac{1}{2}(\psi_{1L}^T \epsilon \psi_{2L})\left(\theta^T_L \epsilon \theta_L\right)
\end{align*}
であることを用いた.これにより成分場が
\begin{align*}
\phi=&\phi_1 \phi_2 \\
\psi_L=& \phi_1 \psi_{2L} +\phi_2 \psi_{1L} \\
\mc{F}=& \phi_1 \mc{F}_2 +\phi_2 \mc{F}_1 -(\psi_{1L}^T \epsilon \psi_{2L})
\end{align*}
となっていることがわかる.

\vskip\baselineskip


理論にカイラル超場があると,超対称作用を構成するのにより広い可能性が開ける.(26.3.16)の変換則を調べてみると,超対称性変換は左カイラル超場の$\mc{F}$項を微分だけ変化させることがわかる.したがって,任意の左カイラル超場の$\mc{F}$項の時空積分は超対称だ!よって超対称な作用を右カイラル超場の$\mc{F}$項だけ抜き出すことによって構成できる.さらに前の議論と同様に,別の一般的な超場の$D$項も付け加えることができる.これより,超対称作用を
\begin{align*}
I=\int d^4x \Bigl[f \Bigr]_{\mc{F}}+\int d^4x \Bigl[f \Bigr]_{\mc{F}}^* +\frac{1}{2}\int d^4 x \Bigl[K \Bigr]_{D}
\end{align*}
と構成することができる.ここで$f$は基本超場から作られる任意の左カイラル超場で,$K$は基本超場から作られる一般的な実超場だ.(基本超場が左カイラルだからといってそこから作られる一般的な超場が左カイラルとなるわけではない.その話を以下でする.)第二項目は全体が実になるようにしている.\par
$f,K$は何に依存できるだろうか?関数$f$は,基本的な左カイラル超場$\Phi_n$にのみ依存し,それらの右カイラルな複素共役$\Psi_{n}^*$には依存しなければ,左カイラルなのだった.一方,カイラル超場の超微分はカイラルではない.実際,$\Psi$を左カイラル超場として
\begin{align*}
\{\mc{D}_\alpha ,\mc{D}_\beta\}=&+2(\gamma^\mu\gamma_5 \epsilon)_{\alpha\beta}\frac{\partial}{\partial x^\mu} \quad \because (26.2.30),\bar{\mc{D}}=\mc{D}^T(\gamma_5 \epsilon)\\
\mc{D}_{R\beta}\left(\mc{D}_{\alpha}\Psi\right)=&\left(\frac{1+\gamma_5}{2}\right)_{\beta\gamma}\mc{D}_{\gamma}\left(\mc{D}_{\alpha}\Psi\right) \\
&=-\left(\frac{1+\gamma_5}{2}\right)_{\beta\gamma}\mc{D}_{\alpha}\left(\mc{D}_{\gamma}\Psi\right)+2\left(\frac{1+\gamma_5}{2}\right)_{\beta\gamma}(\gamma^\mu \gamma_5 \epsilon)_{\alpha\gamma}\frac{\partial}{\partial x^\mu} \Psi \\
=&-\mc{D}_{\alpha}\left(\mc{D}_{R\beta}\Psi\right)+2\left(\frac{1+\gamma_5}{2}\right)_{\beta\gamma}(\gamma^\mu \gamma_5 \epsilon)_{\alpha\gamma}\frac{\partial}{\partial x^\mu}\Psi \\
=&+2(\gamma^\mu \epsilon)_{\alpha\beta}\frac{\partial}{\partial x^\mu}\Psi \neq 0
\end{align*}
となり,左カイラル超場としての条件式を満たさない.同様に右カイラル超場の超微分は右カイラルとならない.したがって$\Phi_n$の超微分を自由に$f$に含めることはできない.しかし,左カイラルではない超場$S$(たとえば左カイラル超場の複素共役を含んでいる場合)に右超微分の対を作用させると,左カイラル超場を得る.これは
\begin{align*}
\{\mc{D}_{R\alpha} ,\mc{D}_{R\beta}\}=&+2\left(\left(\frac{1+\gamma_5}{2}\right)\gamma^\mu\gamma_5 \epsilon\left(\frac{1+\gamma_5}{2}\right)\right)_{\alpha\beta}\frac{\partial}{\partial x^\mu} \\
=&+2\left(\gamma^\mu\gamma_5 \epsilon\left(\frac{1-\gamma_5}{2}\right)\left(\frac{1+\gamma_5}{2}\right)\right)_{\alpha\beta}\frac{\partial}{\partial x^\mu} \\
=&0
\end{align*}
となり,右超微分同士は通常のフェルミオン的c数のように反可換であり,よってp146下の議論と同様に,そのような三つの積は必ずゼロとなるから
\begin{align*}
\mc{D}_{R\alpha}\left(\mc{D}_{R\beta}\mc{D}_{R\gamma}S\right)=0
\end{align*}
となり,実際に$\mc{D}_{R\beta}\mc{D}_{R\gamma}S$は左カイラル超場の条件を満たしている.しかし,このようにして2階の右超微分で構成された任意の関数$f$の$\mc{F}$項は,ある他の複合超場の$D$項と同等の寄与を作用に与えることが以下でわかる.$\mc{D}_R$は反可換だから,p147の議論が同様に行えて$\mc{D}_{R\alpha}\mc{D}_{R\beta}=\frac{1}{4}[\epsilon(1-\gamma_5)]_{\alpha\beta}(\mc{D}^T_{R}\epsilon \mc{D}_{R})$とできる.よって一般の超場$S$に$\mc{D}_R$を二つ作用させてできる最も一般の左カイラル超場は$(\mc{D}_R^T \epsilon \mc{D}_R)S$を使って表すことができる.$f$が依存する基本的左カイラル超場がこの形ならば,個々の$\mc{D}_R$が超ポテンシャルの他の全ての超場を消去するために
\begin{align*}
f=&f\Bigl(\Phi_n ,(\mc{D}_R\epsilon \mc{D}_R )S\Bigr) \\
=&(\mc{D}_R^T \epsilon \mc{D}_R)h(\Phi_n,S)
\end{align*}
と書くことができる.ここで$h$はある他の超場で,もちろん左カイラルとは全く限らない一般的な形をしている.例えば$f=\Phi_n [(\mc{D}_R^T \epsilon \mc{D}_R)S_1]+\Phi_m \Phi_k [(\mc{D}_R^T \epsilon \mc{D}_R)S_2]$の形をしているとすると,$\mc{D}_R \Phi_n=0$だから$\mc{D}_R^T \epsilon \mc{D}_R$で括りだすことができて$f=(\mc{D}_R^T \epsilon \mc{D}_R)(\Phi_n S_1+\Phi_m \Phi_k S_2)$と書くことができる.$h=\Phi_n S_1+\Phi_m \Phi_k S_2$とおけば,これは左カイラルでもない一般的超場で$f=(\mc{D}_R^T \epsilon \mc{D}_R)h$と書ける.(仮定より全ての項は$(\mc{D}^T_{R}\epsilon \mc{D}_R)S$から作られる関数になっている必要がある.もしそうでなければ,例えば$f=\Phi_n [(\mc{D}_R^T \epsilon \mc{D}_R)S]+\Phi_m$は全体で左カイラル超場だが上のように書けない.この関数の第二項目は$(\mc{D}_R^T \epsilon \mc{D}_R)S$から作られるものではなく,基本左カイラル超場のみから作られるものだから,今回の話の適用外だ.)さて
\begin{align*}
\frac{\partial}{\partial \theta_{R\alpha}}\theta_{R\beta}=&\left(\frac{1-\gamma_5}{2}\right)_{\alpha\gamma}\frac{\partial}{\partial \theta_\gamma} \left(\frac{1-\gamma_5}{2}\right)_{\beta\delta}\theta_{\delta} \\
=&\left(\frac{1-\gamma_5}{2}\right)_{\alpha\beta} \\
\mc{D}_{R\alpha} (\theta_{R}^T \epsilon \theta_R)=&-\epsilon_{\alpha \beta}\frac{\partial}{\partial \theta_{R\beta}}(\theta_{R\gamma} \epsilon_{\gamma\delta} \theta_{R\delta}) \\
=&-\epsilon_{\alpha\beta}\left(\frac{1-\gamma_5}{2}\right)_{\beta\gamma}\epsilon_{\gamma\delta}\theta_{R\delta} + \epsilon_{\alpha\beta}\theta_{R\gamma}\epsilon_{\gamma\delta}\left(\frac{1-\gamma_5}{2}\right)_{\beta\delta} \\
=&2\theta_{R\alpha} \\
(\mc{D}_R^T \epsilon \mc{D}_R) (\theta^T_R \epsilon \theta_R)=&(\mc{D}_R^T \epsilon)_\alpha \mc{D}_{R\alpha} (\theta_{R}^T \epsilon \theta_R) \\
=&2(\mc{D}_R^T \epsilon \theta_R)  \\
=&2\left(-\epsilon_{\alpha\beta}\frac{\partial}{\partial \theta_{R\beta}}\epsilon_{\alpha\gamma}\theta_{R\gamma} \right) =2\left(-\epsilon_{\alpha\beta}\epsilon_{\alpha\gamma}\left(\frac{1-\gamma_5}{2}\right)_{\beta\gamma}\right)\\
=&-2\left(\frac{1-\gamma_5}{2}\right)_{\alpha\alpha} \\
=&-4
\end{align*}
だから,作用に寄与しない時空微分($\mc{D}$の第二項目からくる)を除いて,$(\mc{D}_R^T \epsilon \mc{D}_R)h$は$h$の$-\frac{1}{4}(\theta^T_R \epsilon \theta_R)$の係数を抜き出すことに対応していることがわかる.($h$の中のこれより$\theta_R$について低い次数の項は2階$\theta_R$微分により消えるか,時空微分の項になってしまう.これより高次の項は,以下で存在しないことがわかる.)さらに,
\begin{align*}
\mc{F}(x_+)=&\Bigl[f\Bigr]_{\mc{F}}=\Bigl[\cdots +\mc{F}(x_+)(\theta_L^T \epsilon \theta_L )\Bigr]_{\mc{F}} \\
=&\Bigl[(\mc{D}_R^T \epsilon \mc{D}_R)h\Bigr]_{\mc{F}}
\end{align*}
を見れば,$h$の$-\frac{1}{4}(\theta^T_L \epsilon \theta_L)(\theta_R^T \epsilon \theta_R)$の係数が$\mc{F}(x_+)$になっていることがわかる.実際逆算してみれば
\begin{align*}
h=&\cdots -\frac{1}{4}\mc{F}(x_+)(\theta^T_L \epsilon \theta_L)(\theta_R^T \epsilon \theta_R) \\
(\mc{D}_R^T \epsilon \mc{D}_R)h=&\mc{F}(x_+) (\theta^T_L \epsilon \theta_L) \quad \because \mc{D}_R x_+=\mc{D}_R \theta_L=0
\end{align*}
となることがわかる.ここで
\begin{align*}
-\frac{1}{4}(\theta^T_L \epsilon \theta_L)(\theta_R^T \epsilon \theta_R)=&+\frac{1}{4}\left(\bar{\theta}\left(\frac{1+\gamma_5}{2}\right)\theta \right)\left(\bar{\theta}\left(\frac{1-\gamma_5}{2}\right)\theta\right) \\
=&\frac{1}{16}\left[(\bar{\theta}\theta)^2-(\bar{\theta}\gamma_5 \theta)^2\right] \\
=&-\frac{1}{8}(\bar{\theta}\gamma_5 \theta)^2 \quad \because (26.A.18)
\end{align*}
であるから,$\mc{F}$は$h$の$-\frac{1}{8}(\bar{\theta}\gamma_5 \theta)^2$の係数であり,$D$項は$-\frac{1}{4}(\bar{\theta}\gamma_5 \theta)^2$の係数であるのだったから,
\begin{align*}
D=&\Bigl[h\Bigr]_{D}=\Bigl[\cdots -\frac{1}{4}(\bar{\theta}\gamma_5 \theta)^2 D\Bigr]_{D} =\Bigl[\cdots -\frac{1}{8}(\bar{\theta}\gamma_5 \theta)^2 2D\Bigr]_{D} \\
\therefore \quad \mc{F}=&2\Bigl[h\Bigr]_D
\end{align*}
となることがわかる.($D$項は$\theta$について最大次数の項なのだった.だから上の議論で$(\theta_R^T \epsilon \theta_R)$以上の項は考える必要がない.これより高次の項は5次以上に対応し,ゼロとなる.)以上をまとめれば
\begin{align*}
\int d^4x \Bigl[(\mc{D}_R^T \epsilon \mc{D}_R)h\Bigr]_{\mc{F}}=2\int d^4 x \Bigl[h\Bigr]_D
\end{align*}
がわかる!これが欲しかった結果だ.したがって$\mc{D}_{R\beta}\mc{D}_{R\gamma}S$の形の左カイラル超場に依存する項は新しく$f$に含める必要がない.そのような項はどれも,全ての可能な$D$項のリストに既に含まれているからだ.$f$が基本的な左カイラル超場のみの関数で表されて,それらの超微分や時空微分を含まないとき,そのような$f$は超ポテンシャルと呼ばれる.

\vskip\baselineskip

一方,関数$K$は一般に左カイラル超場$\Phi_n$とそれらの右カイラル複素共役$\Phi_n^*$の両方,さらにそれらの超微分と時空微分の実スカラー関数であり,ケーラー・ポテンシャルと呼ばれる.(どんな右カイラル超場も,ある左カイラル超場の複素共役で関係しているのだった.実際右カイラル超場$\tilde{\Phi}$の複素共役$\tilde{\Phi}^*$は左カイラルになるから,左カイラル超場を$\Phi\equiv \tilde{\Phi}^*$とおけば$\Phi^*=\tilde{\Phi}$と書ける.よって$K$が左カイラル超場とそれらの複素共役のみに依存しているとしても一般性は失われない.)しかし,そのような全ての$K$が異なる作用を与えるわけではない.例えば,カイラル超場は$D$項を持たないので,二つの$K,K'$がカイラル超場だけ異なるときは$[K]_D=[K']_D$となって同じ作用として寄与する.\par
$K$の形を超空間の部分積分だけ変化させて,作用を同じに保つことも可能だ.任意の超場の超微分$\mc{D}_\alpha S$の$D$項は
\begin{align*}
\int d^4x \left[\mc{D}_\alpha S\right]=0
\end{align*}
となって作用に寄与しない.これを見るには,
\begin{align*}
\mc{D}_\alpha S=\sum_\beta (\gamma_5 \epsilon)_{\alpha\beta}\frac{\partial S}{\partial \theta_\beta}-(\gamma^\mu \theta)_{\alpha} \frac{\partial S}{\partial x^\mu}
\end{align*}
を思い出す.$S$は$\theta$について高々4次の多項式だから,$\mc{D}_\alpha S$の第一項目は$\theta$について高々3次の多項式となり,$D$項に対応する項が存在しない.第二項目はまた時空微分だから,その$D$項も時空微分の形となる.したがって$\mc{D}_\alpha S$の第一項目と第二項目も(26.3.32)の積分に寄与せず,上の式が確かめられる.また,超微分は分配則に従って働くので,(26.3.32)より,超空間で部分積分できる.つまり,任意の二つのボゾン的超場$S_1$と$S_2$について
\begin{align*}
\int d^4x \left[S_1 \mc{D}_\alpha S_2 \right]_D=& \int d^4x \left[\mc{D}_{\alpha}(S_1S_2)\right]_D -\int d^4x \left[S_2 \mc{D}_\alpha S_1 \right]_D \\
=&-\int d^4x \left[S_2 \mc{D}_\alpha S_1 \right]_D \quad \because (26.3.32)
\end{align*}
が成立する.26.4節と26.8節では,$f,K$が基本超場のみに依存してそれらの超微分や通常の微分には依存しない場合を考察する.

\vskip\baselineskip


前の節では,パリティが保存する理論において,一般のスカラー超場への時空反転演算子$\mathsf{P}$の効果は,その引数を$x^\mu \to \tensor{(\Lambda_P )}{^\mu_\nu}x^\nu,\theta \to -i\beta \theta$へと変換し,また超場に位相$\eta$をかけて(26.2.35)となることを見た.これらの変換のもとでは(26.3.21)(26.3.22)の引数$x^\mu_{\pm}$は
\begin{align*}
x^\mu_\pm=&x^\mu \pm \frac{1}{2}(\bar{\theta}\gamma_5 \gamma^\mu \theta) \\
\to & (\Lambda_P x)^\mu \pm \frac{1}{2}(\bar{\theta} (-i\beta) \gamma_5 \gamma^\mu i\beta\theta) \\
=&(\Lambda_P x)^\mu \pm \tensor{(\Lambda_P)}{^\mu_\nu}\frac{1}{2}(\bar{\theta} \gamma_5 \gamma^\nu \theta) \quad \because \beta \gamma^\mu \beta =\tensor{(\Lambda_P)}{^\mu_\nu} \gamma^\nu \\
=&(\Lambda_P x_\mp)^\mu
\end{align*}
となり,また
\begin{align*}
\theta_L=\left(\frac{1+\gamma_5}{2}\right) \theta \to \left(\frac{1+\gamma_5}{2}\right)(-i\beta \theta) =-i\beta \left(\frac{1-\gamma_5}{2}\right)\theta =-i\beta \theta_R \\
\theta_R=\left(\frac{1-\gamma_5}{2}\right)\theta \to \left(\frac{1-\gamma_5}{2}\right)(-i\beta \theta) =-i\beta \left(\frac{1+\gamma_5}{2}\right)\theta =-i\beta \theta_L
\end{align*}
と変換される.したがって,時空反転により左カイラル超場は右カイラル超場に,またその逆に右カイラル超場は左カイラル超場に変換される.(左カイラル超場は$x_+,\theta_L$依存であったが,この変換で$x_-,\theta_R$依存になるから.)よって$X(x,\theta)$をスカラー超場とすると,前節の議論により
\begin{align*}
\mathsf{P}^{-1} X(x,\theta)\mathsf{P}=&\mathsf{P}^{-1}\Phi(x,\theta)\mathsf{P}+\mathsf{P}^{-1} \tilde{\Phi}(x,\theta)\mathsf{P} \\
=\eta X(\Lambda_P x,-i\beta \theta)=&\eta \Phi(\Lambda_P x,-i\beta \theta)+\eta\tilde{\Phi}(\Lambda_P x,-i\beta \theta) \\
\therefore \quad \mathsf{P}^{-1}\Phi(x,\theta)\mathsf{P}=&\eta\tilde{\Phi}(\Lambda_P x,-i\beta\theta) \\
\mathsf{P}^{-1}\tilde{\Phi}(x,\theta)\mathsf{P}=&\eta\Phi(\Lambda_P x,-i\beta\theta)
\end{align*}
となる.(この対応でなければ,左カイラルの反転が右カイラルであるようにならない.)4.2節の(4.2.14)より,左カイラル超場$\Phi$に含まれる成分場が生成・消滅する粒子は,空間反転された右カイラル超場$\mathsf{P}^{-1}\Phi\mathsf{P}$に含まれる成分場によっても生成・消滅されなければならない.そのような右カイラル・スカラー超場は複素共役$\Phi^*$だけだ.よって$\mathsf{P}^{-1}\Phi\mathsf{P}\propto \tilde{\Phi}$は$\Phi^*(\Lambda_P x,-i\beta \theta)$に比例しなければならない.よって$\tilde{\Phi}$と$\Phi^*$は比例しており,その比例係数を$\alpha$とおき$\tilde{\Phi}=\alpha\Phi^*$とすると
\begin{align*}
\mathsf{P}^{-1}\Phi(x,\theta) \mathsf{P}=&\eta \tilde{\Phi}(\Lambda_P x,-i\beta\theta)=\eta\alpha \Phi^*(\Lambda_P x,-i\beta\theta) \\
=\alpha^*\mathsf{P}^{-1}\tilde{\Phi}^*(x,\theta) \mathsf{P}=&\alpha^*\eta^* \Phi^*(\Lambda_P x,-i\beta\theta) \\
\therefore \quad \alpha\eta=\alpha^*\eta^*
\end{align*}
となり,$\alpha\eta$は実数であることがわかる.さらに
\begin{align*}
\mathsf{P}^{-1} \tilde{\Phi}(x,\theta)\mathsf{P} =&\alpha \mathsf{P}^{-1} \Phi^*(x,\theta)\mathsf{P}= \alpha \eta^* \tilde{\Phi}^*(\Lambda_P x,-i\beta\theta) \\
=\eta \Phi(\Lambda_P,-i\beta\theta)=&\frac{\eta}{\alpha^*} \tilde{\Phi}^*(\Lambda_Px,-i\beta\theta) \\
\therefore \quad \alpha \alpha^*\eta^* =&\eta
\end{align*}
ということもわかる.後者の条件と$|\eta|=1$を使うと$\alpha$も位相因子であることがわかり,$\Phi$の位相を適切に選ぶと
\begin{align*}
\mathsf{P}^{-1}\Phi(x,\theta)\mathsf{P}=&\Phi^*(\Lambda_P x,-i\beta\theta)
\end{align*}
とできる.($\alpha\eta=e^{i\theta}$とすれば,$\Phi'\equiv e^{-i\theta/2}\Phi$として位相因子が$\mathsf{P}^{-1}\Phi(x,\theta) \mathsf{P}=\eta\alpha \Phi^*(\Lambda_P x,-i\beta\theta)$の両辺で消えるようにできる.)成分場で書くと,この変換則は
\begin{align*}
\Phi^*(\Lambda_P x,-i\beta \theta)=&\phi^*(\Lambda_P x_-)-\sqrt{2}\left[(-i\beta \theta_R)^T \epsilon \psi_L(\Lambda_P x_-)\right]^*+\mc{F}^*(\Lambda_P x_-)((-i\beta \theta_R)^T \epsilon (-i\beta \theta_R))^* \\
=&\phi^*(\Lambda_P x_-)-\sqrt{2}\left[-i\theta_R^T \beta \epsilon \psi_L(\Lambda_P x_-)\right]^*-\mc{F}^*(\Lambda_P x_-)(\theta_R^T \beta \epsilon \beta \theta_R)^* \\
=&\phi^*(\Lambda_P x_-)-\sqrt{2}\left[-i\theta_R^\dagger \beta \epsilon \psi_L^*(\Lambda_P x_-)\right]+\mc{F}^*(\Lambda_P x_-)(\theta_R^\dagger \beta \epsilon \beta \theta_R^*) \quad \because (25.1.20) \\
=&\phi^*(\Lambda_P x_-)-\sqrt{2}\left[-i\overline{\theta_R} \epsilon \psi_L^*(\Lambda_P x_-)\right]+\mc{F}^*(\Lambda_P x_-)(\overline{\theta_R} \epsilon \beta \theta_R^*) \\
=&\phi^*(\Lambda_P x_-)-\sqrt{2}\left[-i\theta_L^T \epsilon \epsilon \psi_L^*(\Lambda_P x_-)\right]-\mc{F}^*(\Lambda_P x_-)(\theta_L^T \epsilon \theta_L) \quad \because \overline{\theta_R}=\theta^T_L \epsilon \gamma_5\\
=&\phi^*(\Lambda_P x_-)-\sqrt{2}\left[\theta_L^T\epsilon(-i\epsilon \psi_L^*(\Lambda_P x_-))\right]-\mc{F}^*(\Lambda_P x_-)(\theta_L^T \epsilon \theta_L)
\end{align*}
から
\begin{align*}
\mathsf{P}^{-1} \phi(x)\mathsf{P}=&\phi^*(\Lambda_P x) \\
\mathsf{P}^{-1} \psi_L(x)\mathsf{P}=& -i\epsilon \gamma_5 \psi_L^*(\Lambda_P x) \\
\mathsf{P}^{-1} \mc{F}(x)\mathsf{P}=& \mc{F}^*(\Lambda_P x)
\end{align*}
となる.($\psi_L$についての変換則は誤植)\par


\vskip\baselineskip


他の種類の対称性も可能で,これは$R$対称性と呼ばれる.この対称性は26.5節で述べる超対称性が自発的に破れる模型のいくつかで重要となり,27.6節で非くりこみ定理を証明する際にも使われるらしい.25.2節で触れたように,$N=1$の単純超対称性の理論では,$R$対称性は$U(1)$変換(25.2.33)のもとでの不変性だ.この変換のもとでは,(25.2節で$\mr{Q}_a$と呼んだ)超対称性生成子右巻き成分はゼロではない量子数,例えば$+1$を持ち,さらにその場合にはその共役,つまり超対称性演算子の左巻き成分($\mr{Q}_a^*$)は反対の量子数$-1$を持つ.($\mr{Q}_a$は(0,1/2)表現なので右巻き成分だ.本文は誤植.)
\begin{align*}
[R,\mr{Q}_a]=&-(+1)\mr{Q}_a ,\quad e^{-iR\varphi}\mr{Q}_a e^{+iR\varphi}=e^{i\varphi}\mr{Q}_a\\
[R,\mr{Q}^*_a]=&-(-1)\mr{Q}_a^* ,\quad e^{-iR\varphi}\mr{Q}_a^* e^{+iR\varphi}=e^{-i\varphi}\mr{Q}_a^*
\end{align*}
(26.2.2)から超空間座標$\theta$は$R$対称性変換のもとで非自明な変換性を持つことがわかる.具体的には,(25.2.34)より
\begin{align*}
\left[R,Q\right]=&\left(
\begin{matrix}
+e\mr{Q}^* \\
-\mr{Q}
\end{matrix}
\right) \\
\therefore \quad \left[R,\left(\frac{1+\gamma_5}{2}\right)Q\right]=&+\left(\frac{1+\gamma_5}{2}\right)Q ,\quad \left[R,\left(\frac{1-\gamma_5}{2}\right)Q\right]=-\left(\frac{1-\gamma_5}{2}\right)Q \\
\left[R,Q_L\right]=&+Q_L ,\quad \left[R,Q_R\right]=-Q_R \\
e^{-iR\varphi}Q_Le^{iR\varphi}=& e^{-i\varphi} Q_L ,\quad e^{-iR\varphi}Q_Re^{iR\varphi}= e^{+i\varphi} Q_R
\end{align*}
だから,(26.2.1)より
\begin{align*}
[Q_L, S \}=&i\mc{Q}_L S , \quad [Q_R, S \}=i\mc{Q}_R S \\
e^{-iR\varphi} [Q_L, S \} e^{iR\varphi}=& \left[ e^{-iR\varphi} Q_L e^{iR\varphi} ,e^{-iR\varphi}Se^{iR\varphi} \right\} \\
=&e^{-i\varphi} \left[Q_L, e^{-iR\varphi}Se^{iR\varphi} \right\} \\
=ie^{-iR\varphi}\mc{Q}_L S e^{iR\varphi}=&ie^{-iR\varphi} \mc{Q}_L e^{iR\varphi} \Bigl[e^{-iR\varphi}Se^{iR\varphi}\Bigr] 
\end{align*}
ここで
\begin{align*}
\mc{Q}_{L\alpha}=+\epsilon_{\alpha\beta}\frac{\partial}{\partial \theta_{L\beta}}+(\gamma^\mu \theta_R)_\alpha \frac{\partial}{\partial x^\mu}
\end{align*}
だ.($\mc{D}$と$\mc{Q}$は第二項目の符号だけが違うので,(26.3.26)からすぐわかる.)$e^{-iR\varphi}S(x,\theta)e^{iR\varphi}=S'(x,\theta)$とすると,($S'(x,\theta')$としてもいいが,$\theta'$は$\theta$の関数となるから$S''(x,\theta)=S'(x,\theta'(\theta))$と再定義できるのでこれでいい)
\begin{align*}
\left[Q_{L\alpha}, e^{-iR\varphi}S(x,\theta)e^{iR\varphi} \right\}=\left[Q_{L\alpha}, S'(x,\theta) \right\}=i\mc{Q}_{L\alpha} S'(x,\theta)
\end{align*}
となる.よって
\begin{align*}
e^{-iR\varphi}\mc{Q}_L e^{iR\varphi}=&e^{-i\varphi} \mc{Q}_L \\
e^{-iR\varphi}\mc{Q}_{L\alpha} e^{iR\varphi}=&+\epsilon_{\alpha\beta}\left[e^{-iR\varphi}\frac{\partial}{\partial \theta_{L\beta}}e^{iR\varphi}\right]+(\gamma^\mu \left[e^{-iR\varphi}\theta_Re^{iR\varphi}\right])_{\alpha}\frac{\partial}{\partial x^\mu} \\
=&e^{-i\varphi}\left(+\epsilon_{\alpha\beta}\frac{\partial}{\partial \theta_{L\beta}}+(\gamma^\mu \theta_R)_\alpha \frac{\partial}{\partial x^\mu}\right) \\
\therefore \quad e^{-iR\varphi}\frac{\partial}{\partial \theta_{L\beta}}e^{iR\varphi}=&e^{-i\varphi}\frac{\partial}{\partial \theta_{L\beta}} ,\quad e^{-iR\varphi}\theta_Re^{iR\varphi}=e^{-i\varphi}\theta_R
\end{align*}
ここから
\begin{align*}
e^{-iR\varphi}\frac{\partial}{\partial \theta_{L\alpha}} \theta_{L\beta}e^{iR\varphi}=&\left(\frac{1+\gamma_5}{2}\right)_{\alpha\beta} \\
=&e^{-i\varphi}\frac{\partial}{\partial \theta_{L\alpha}}e^{-iR\varphi}\theta_{L\beta} e^{iR\varphi} \\
\therefore \quad e^{-iR\varphi}\theta_{L} e^{iR\varphi}=&e^{+i\varphi} \theta_{L}
\end{align*}
が得られる.よって$\theta_L$は$R$対称性変換のもとで量子数$+1$を持ち,$\theta_R$は$R$量子数$-1$を持つことがわかる.\par
さらに,超場全体がある特定の$R$量子数を持つこともできる.もしカラル超場$\Phi(x,\theta)$が$R$量子数$R_{\Phi}$を持てば,
\begin{align*}
&e^{-iR \varphi}\Phi(x,\theta)e^{iR\varphi}=e^{iR_{\Phi}\varphi}\Phi(x,\theta) \\
=&e^{-iR \varphi}\phi(x_+)e^{iR\varphi}-e^{+i\varphi}\sqrt{2}\left(\theta^T_L \epsilon \left[e^{-iR \varphi}\psi_L(x_+)e^{iR\varphi}\right]\right)+e^{+2i\varphi}\left[e^{-iR \varphi}\mc{F}(x_+)e^{iR\varphi}\right](\theta^T_L \epsilon \theta_L) \\
\therefore \quad & e^{-iR \varphi}\phi e^{iR\varphi}=e^{iR_\Phi\varphi} \phi \\
&e^{-iR \varphi}\psi_Le^{iR\varphi}=e^{i(R_\Phi-1)\varphi} \psi_L \\
&e^{-iR \varphi}\mc{F}e^{iR\varphi}=e^{i(R_\Phi-2)\varphi}\mc{F}
\end{align*}
と変換され,スカラー成分$\phi$は同じ$R$量子数$R_\Phi$を持ち,左スピノル成分は$\psi_L$は$R_\psi=R_\Phi-1$,補助場$\mc{F}$は$R_{\mc{F}}=R_{\Phi}-2$を持つ.特に,超ポテンシャル項$\int d^4x [f]_\mc{F}$が$R$対称性を保存するならば,($R_\mc{F}=0$となるように)超ポテンシャル$f$自身は$R_f=+2$を持たなければならない.したがって,もし$f$が一つの左カイラル超場$\Phi$のみに依存する$f=f(\Phi)$ならば,$\Phi^{2/R_\Phi}$に比例していなければならない.実際
\begin{align*}
e^{-iR\varphi} \Phi^{2/R_\Phi}e^{iR\varphi}=(e^{iR_\Phi\varphi})^{2/R_\Phi}\Phi^{2/R_\Phi}=e^{2i\varphi}\Phi^{2/R_\Phi}
\end{align*}
となって$R_f=2$であり,これ以外の作り方ではそのようにできない.言い換えると,もし$f(\Phi)$が$\Phi^2$に比例する純粋な質量項ならば,$R_\Phi=+1$となる基準超場を選ばなければならないし,もし$f(\Phi)$が$\Phi^3$に比例する純粋な相互作用項ならば$R_\Phi=2/3$となる基準超場を選ばなければならない.一方,(26.2.10)を調べると
\begin{align*}
(\bar{\theta}\gamma_5\theta)^2\propto& (\theta^T_L \epsilon\theta_L)(\theta^T_R \epsilon \theta_R) \\
e^{-iR\varphi}(\bar{\theta}\gamma_5\theta)^2 e^{iR\varphi} \propto& e^{-iR\varphi}(\theta^T_L \epsilon\theta_L)(\theta^T_R \epsilon \theta_R) e^{iR\varphi} =(\theta^T_L \epsilon\theta_L)(\theta^T_R \epsilon \theta_R) \\
e^{-iR\varphi}S(x,\theta) e^{iR\varphi}=&e^{iR_\Phi\varphi}S(x,\theta) \\
=&\cdots -\frac{1}{4}e^{-iR\varphi}(\bar{\theta}\gamma_5\theta)^2 D(x) e^{iR\varphi} \\
=&\cdots -\frac{1}{4} (\bar{\theta}\gamma_5\theta)^2e^{-iR\varphi}D(x) e^{iR\varphi} \\
\therefore \quad e^{-iR\varphi}D(x) e^{iR\varphi}=e^{iR_\Phi \varphi}D(x)
\end{align*}
となって,$D$はその超場と同じ$R$量子数を持つことがわかる.よって作用の$\int d^4x [K]_D$項が$R$対称性を保存するならば,$K$が$R=0$でありすれば良い.$K$が$\Phi$のみに依存する場合,$\Phi$がどのような$R$量子数を持とうとも$K$の各項が$\Phi$の因子と$\Phi^*$の因子を同じ数だけ持つようになっていればよく,逆にそのときのみ許される.\par
もちろん,作用が$R$不変性を持たなければならない一般的な理由はないし,それが自発的に破れてはいけない理由もない.

\vskip\baselineskip

今回は(26.3.1)(26.3.2)で超場を拘束したが,他の方法もある.それを使えば場の別種の超対称性多重項が得られる.そのうち,比較的知られているものの一つは,\textbf{線形}の超場だ.この種の超場を定義する条件を調べるためには,ある一般の超場$S$から
\begin{align*}
S'\equiv \frac{1}{4}(\bar{\mc{D}}\mc{D})S
\end{align*}
とカイラル超場を作ることができることに着目する.これは
\begin{align*}
\frac{1}{4}(\bar{\mc{D}}\mc{D})S=&\frac{1}{4}(\overline{\mc{D}_R}\mc{D}_L)S+\frac{1}{4}(\overline{\mc{D}_L}\mc{D}_R)S \\
=&\frac{1}{4}(\mc{D}_L^T \epsilon \mc{D}_L)S+\frac{1}{4}(\mc{D}^T_R \epsilon \mc{D}_R)S
\end{align*}
と書けて,第一項目はp98の議論により右カイラルで,第二項目は同様に左カイラルであるから,全体として(26.3.10)の形になっており,$S'$はカイラル超場だ.さて
\begin{align*}
(\bar{\mc{D}}\mc{D})=&\left[ \frac{\partial}{\partial \theta_\alpha}+(\gamma_5 \epsilon \gamma^\mu)_{\alpha\beta}\theta_\beta \frac{\partial}{\partial x^\mu} \right]\left[(\gamma_5 \epsilon)_{\alpha\gamma}\frac{\partial}{\partial \theta_\gamma}-\gamma^\nu_{\alpha\gamma}\theta_\gamma \frac{\partial}{\partial x^\nu}\right] \\
=&(\gamma_5 \epsilon)_{\alpha\beta}\frac{\partial}{\partial \theta_\alpha} \frac{\partial}{\partial \theta_\beta}-\gamma^\mu_{\alpha\beta}\frac{\partial}{\partial \theta_\alpha}\left(\theta_\beta \frac{\partial}{\partial x^\mu}\right) \\
&+(\gamma_5 \epsilon \gamma^\mu)_{\alpha\beta}\theta_\beta (\gamma_5 \epsilon)_{\alpha\gamma}\frac{\partial}{\partial \theta_\gamma} \frac{\partial}{\partial x^\mu}-(\gamma_5 \epsilon \gamma^\mu)_{\alpha\beta}\theta_\beta \gamma^\nu_{\alpha\gamma}\theta_\gamma \frac{\partial}{\partial x^\mu}\frac{\partial}{\partial x^\nu} \\
=&(\gamma_5 \epsilon)_{\alpha\beta}\frac{\partial}{\partial \theta_\alpha} \frac{\partial}{\partial \theta_\beta}-\gamma^\mu_{\alpha\alpha} \frac{\partial}{\partial x^\mu} +\gamma^\mu_{\alpha\beta}\theta_\beta \frac{\partial}{\partial \theta_\alpha}\frac{\partial}{\partial x^\mu}\\
&-(\gamma_5 \epsilon \gamma_5 \epsilon \gamma^\mu)_{\gamma\beta}\theta_\beta \frac{\partial}{\partial \theta_\gamma} \frac{\partial}{\partial x^\mu}-((\gamma^\nu)^T\gamma_5 \epsilon \gamma^\mu)_{\gamma\beta}\theta_\beta \theta_\gamma \frac{\partial}{\partial x^\mu}\frac{\partial}{\partial x^\nu} \\
=&(\gamma_5 \epsilon)_{\alpha\beta}\frac{\partial}{\partial \theta_\alpha} \frac{\partial}{\partial \theta_\beta}+\gamma^\mu_{\alpha\beta}\theta_\beta \frac{\partial}{\partial \theta_\alpha}\frac{\partial}{\partial x^\mu} \quad \because \mr{tr}\gamma^\mu=0\\
&+\gamma^\mu_{\gamma\beta}\theta_\beta \frac{\partial}{\partial \theta_\gamma} \frac{\partial}{\partial x^\mu}+(\epsilon \gamma_5 \gamma^\nu \gamma^\mu)_{\gamma\beta}\theta_\beta \theta_\gamma \frac{\partial}{\partial x^\mu}\frac{\partial}{\partial x^\nu} \quad \because (\gamma^\mu)^T=(-\epsilon \gamma_5)\gamma^\mu (-\epsilon \gamma_5 )\\
=&(\gamma_5 \epsilon)_{\alpha\beta}\frac{\partial}{\partial \theta_\alpha} \frac{\partial}{\partial \theta_\beta}+2(\gamma^\mu)^T_{\alpha\beta}\theta_\alpha \frac{\partial}{\partial \theta_\beta}\frac{\partial}{\partial x^\mu} +(\epsilon \gamma_5)_{\gamma\beta}\theta_\beta \theta_\gamma \frac{\partial}{\partial x^\mu}\frac{\partial}{\partial x_\mu} \\
=&-\frac{\partial}{\partial \theta_\alpha} \frac{\partial}{\partial \bar{\theta}_\alpha}+2\theta_\alpha (\epsilon \gamma_5 )_{\alpha\gamma}\gamma^\mu_{\gamma\delta} (\epsilon \gamma_5)_{\delta \beta}\frac{\partial}{\partial \theta_\beta}\frac{\partial}{\partial x^\mu} -(\bar{\theta} \theta) \Box \\
=&-\frac{\partial}{\partial \theta_\alpha} \frac{\partial}{\partial \bar{\theta}_\alpha}-2\bar{\theta}_\alpha \gamma^\mu_{\alpha\beta} \frac{\partial}{\partial \bar{\theta}_\beta}\frac{\partial}{\partial x^\mu} -(\bar{\theta} \theta) \Box \\
\therefore \quad \frac{1}{4}(\bar{\mc{D}}\mc{D})=&-\frac{1}{4}\frac{\partial}{\partial \theta_\alpha} \frac{\partial}{\partial \bar{\theta}_\alpha}-\frac{1}{2}\bar{\theta}_\alpha \gamma^\mu_{\alpha\beta} \frac{\partial}{\partial \bar{\theta}_\beta}\frac{\partial}{\partial x^\mu} -\frac{1}{4}(\bar{\theta} \theta) \Box 
\end{align*}
これを(26.2.10)に作用させる.各項で見てみると,(26.2.9)とその類似式
\begin{align*}
\frac{\partial}{\partial \theta_\alpha}(\bar{\theta}M\theta)=&-2(\bar{\theta}M)_\alpha \\
\frac{\partial}{\partial \bar{\theta}_\alpha}(\bar{\theta}M\theta)=&2(M\theta)_\alpha
\end{align*}
と,$\theta$について3次の項以降の微分ではフェルミオン的な微分とフェルミオン的変数の入れ替えでマイナスが出ることに気を付けて,
\begin{align*}
\frac{1}{4}(\bar{\mc{D}}\mc{D})C=&-\frac{1}{4}(\bar{\theta}\theta)\Box C \\
\frac{1}{4}(\bar{\mc{D}}\mc{D})\left[-i(\bar{\theta}\gamma_5 \omega)\right]=&+\frac{i}{2}\left(\bar{\theta}\gamma^\mu \gamma_5 \frac{\partial \omega}{\partial x^\mu}\right)+\frac{i}{4}(\bar{\theta}\theta)(\bar{\theta}\gamma_5 \Box \omega) \\
=&+\frac{i}{2}\left(\bar{\theta}\gamma^\mu \gamma_5 \frac{\partial \omega}{\partial x^\mu}\right)-\frac{i}{4}(\bar{\theta}\gamma_5\theta)(\bar{\theta} \Box \omega) \quad \because (26.A.16)\\
=&-\frac{i}{2}\left(\bar{\theta}\gamma_5 \Slash{\partial}\omega\right)-\frac{i}{2}(\bar{\theta}\gamma_5\theta)\left(\bar{\theta} \frac{1}{2}\Slash{\partial}\Slash{\partial} \omega\right) \\
\frac{1}{4}(\bar{\mc{D}}\mc{D})\left[-\frac{i}{2}\left(\bar{\theta}\gamma_5 \theta\right)M\right]=&+\frac{i}{4}(\gamma_5)_{\alpha\alpha}M+\frac{i}{2}(\bar{\theta}\gamma^\mu \gamma_5 \theta)\frac{\partial M}{\partial x^\mu}+\frac{i}{8}(\bar{\theta}\theta)(\bar{\theta}\gamma_5 \theta)\Box M \\
=&-\frac{i}{2}(\bar{\theta}\gamma_5 \gamma_\mu \theta)\partial^\mu M \quad \because \mr{tr}\gamma_5=0,(26.A.16) \\
\frac{1}{4}(\bar{\mc{D}}\mc{D})\left[-\frac{1}{2}(\bar{\theta}\theta)N\right]=&+\frac{1}{4}\delta_{\alpha\alpha}N+\frac{1}{2}(\bar{\theta}\gamma^\mu \theta)\frac{\partial N}{\partial x^\mu}+\frac{1}{8}(\bar{\theta}\theta)^2 \Box N \\
=&N-\frac{1}{8}(\bar{\theta}\gamma_5 \theta)^2\Box N \quad \because (26.A.8)\\
\frac{1}{4}(\bar{\mc{D}}\mc{D})\left[+\frac{i}{2}(\bar{\theta}\gamma_5 \gamma^\mu \theta)V^\mu\right]=&-\frac{i}{4}(\gamma_5 \gamma_\mu )_{\alpha\alpha}V^\mu -\frac{i}{2}(\bar{\theta}\gamma^\nu \gamma_5 \gamma_\mu \theta)\frac{\partial V^\mu}{\partial x^\nu}-\frac{i}{8}(\bar{\theta}\theta)(\bar{\theta}\gamma_5 \gamma_\mu \theta)\Box V^\mu \\
=&+\frac{i}{2}(\bar{\theta}\gamma^\nu \gamma^\mu \gamma_5 \theta)\partial_\nu V_\mu \\
=&+\frac{i}{2}(\bar{\theta} \gamma_5 \theta)\partial_\mu V^\mu \\
\frac{1}{4}(\bar{\mc{D}}\mc{D})\left[-i(\bar{\theta}\gamma_5 \theta)\left(\bar{\theta}\lambda\right)\right]=&+\frac{i}{2}(\gamma_5)_{\alpha\alpha}(\bar{\theta}\lambda)-\frac{i}{2}(\bar{\theta}\gamma_5 \lambda)+\frac{i}{2}(\gamma_5 \theta)_\alpha \bar{\lambda}_\alpha \\
&+i(\bar{\theta}\gamma^\mu\gamma_5 \theta)\left(\bar{\theta}\frac{\partial \lambda}{\partial x^\mu}\right)+\frac{i}{2}(\bar{\theta}\gamma_5 \theta)\left(\bar{\theta}\gamma^\mu \frac{\partial \lambda}{\partial x^\mu}\right) \\
=&-i(\bar{\theta}\gamma_5 \lambda)-\frac{i}{2}(\bar{\theta}\gamma_5 \theta)(\bar{\theta}\gamma^\mu \partial_\mu \lambda) \quad \because (26.A.17),(26.A.8)\\
=&-i(\bar{\theta}\gamma_5 \lambda)-i(\bar{\theta}\gamma_5 \theta)\left(\bar{\theta}\frac{1}{2}\Slash{\partial} \lambda\right) \\
\frac{1}{4}(\bar{\mc{D}}\mc{D})\left[-i(\bar{\theta}\gamma_5 \theta)\left(\bar{\theta}\frac{1}{2}\Slash{\partial}\omega\right)\right]=&+\frac{i}{2}(\gamma_5)_{\alpha\alpha}\left(\bar{\theta}\frac{1}{2}\Slash{\partial}\omega\right)-\frac{i}{2}\left(\bar{\theta}\gamma_5 \frac{1}{2}\Slash{\partial}\omega\right)-\frac{i}{2}(\gamma_5 \theta)_\alpha \left(\frac{1}{2}\partial_\mu \bar{\omega}\gamma^\mu \right)_\alpha \\
&+i(\bar{\theta}\gamma^\mu \gamma_5\theta)\left(\bar{\theta}\partial_\mu\frac{1}{2} \Slash{\partial}\omega\right)+\frac{i}{2}(\bar{\theta}\gamma_5 \theta)\left(\bar{\theta} \gamma^\mu \partial_\mu \frac{1}{2}\Slash{\partial}\omega\right) \\
=&-\frac{i}{2}\left(\bar{\theta}\gamma_5 \Slash{\partial}\omega\right)-\frac{i}{2}(\bar{\theta}\gamma_5 \theta)\left(\bar{\theta}\frac{1}{2}\Slash{\partial}\Slash{\partial}\omega\right) \\
\frac{1}{4}(\bar{\mc{D}}\mc{D})\left[-\frac{1}{4}(\bar{\theta}\gamma_5 \theta)^2 D\right]=&-\frac{1}{4}\frac{\partial}{\partial \theta_\alpha}\left[-(\bar{\theta}\gamma_5 \theta)(\gamma_5 \theta)_\alpha D\right]+\frac{1}{2}(\bar{\theta}\gamma_5 \theta)(\bar{\theta}\gamma^\mu \gamma_5 \theta)D \\
=&-\frac{1}{2}(\bar{\theta}\gamma_5 \gamma_5 \theta)D +\frac{1}{4}(\bar{\theta}\gamma_5 \theta)(\gamma_5)_{\alpha\alpha}D \\
=&-\frac{1}{2}(\bar{\theta}\theta)D \\
\frac{1}{4}(\bar{\mc{D}}\mc{D})\left[-\frac{1}{4}(\bar{\theta}\gamma_5 \theta)^2 \frac{1}{2}\Box C\right]=&-\frac{1}{4}(\bar{\theta}\theta)\Box C
\end{align*}
ここで
\begin{align*}
(\bar{\theta}\gamma_5 \theta )(\bar{\theta}\gamma_5 \gamma_\mu \theta)=&0 \\
(\bar{\theta}\theta)(\bar{\theta}\gamma_5 \gamma_\mu \theta)=&-(\bar{\theta}\gamma_\mu \theta)(\bar{\theta}\gamma_5 \theta)=0 
\end{align*}
であることを用いた.さらに$V^\mu$の項では
\begin{align*}
\gamma^\mu \gamma^\nu \gamma_5 =&\eta^{\mu\nu}\gamma_5+\frac{1}{2}[\gamma^\mu,\gamma^\nu]\gamma_5 \\
=&\eta^{\mu\nu}\gamma_5 +\frac{1}{2}i\epsilon^{\mu\nu\rho\sigma}[\gamma_\rho,\gamma_\sigma]
\end{align*}
となることと(26.A.8)を用いた.二行目から三行目は$\mu=0,\nu=1$として$\epsilon^{0123}=+1$と$\gamma_5=-i\gamma^0\gamma^1 \gamma^2 \gamma^3$を用いればすぐ確かめられる.$\Slash{\partial}\omega$の項では
\begin{align*}
(\bar{\theta}\gamma^\mu \gamma_5 \theta)\left(\bar{\theta} \gamma^\nu \frac{1}{2}\partial_\mu\partial_\nu \omega\right)=&(\bar{\theta}\gamma_5 \gamma^\mu \theta)\left(\frac{1}{2}\partial_\mu\partial_\nu \bar{\omega} \gamma^\nu \theta\right) \quad \because (26.A.7)\\
=&-(\bar{\theta}\gamma_5 \theta)\left(\frac{1}{2}\partial_\mu\partial_\nu \bar{\omega}\gamma^\mu \gamma^\nu \theta\right) \quad \because (26.A.17) \\
=&-(\bar{\theta}\gamma_5 \theta)\left(\frac{1}{2}\Box \bar{\omega} \theta\right)=-(\bar{\theta}\gamma_5 \theta)\left(\bar{\theta}\frac{1}{2}\Box \omega\right) \\
=&-(\bar{\theta}\gamma_5 \theta)\left(\bar{\theta}\frac{1}{2}\Slash{\partial}\Slash{\partial}\omega\right) 
\end{align*}
を用いた.これらの項を全て足し合わせると
\begin{align*}
S'=&\frac{1}{4}(\bar{\mc{D}}\mc{D})S \\
=&N-i(\bar{\theta}\gamma_5 [\lambda+\Slash{\partial}\omega])-\frac{i}{2}(\bar{\theta}\gamma_5\theta)[-\partial_\mu V^\mu]-\frac{1}{2}(\bar{\theta}\theta)[D+\Box C] \\
&+\frac{i}{2}(\bar{\theta}\gamma_5 \gamma_\mu \theta)[-\partial^\mu M]-i(\bar{\theta}\gamma_5 \theta)\left(\bar{\theta}\left[+\frac{1}{2}\Slash{\partial}(\lambda+\Slash{\partial}\omega)\right]\right) \\
&-\frac{1}{4}(\bar{\theta}\gamma_5 \theta)^2 \left(+\frac{1}{2}\Box N\right)
\end{align*}
となる.よって$S'$の成分は$S$の成分を用いて
\begin{align*}
C'=&N \\
\omega'=&\lambda+\Slash{\partial}\omega \\
M'=& -\partial_\mu V^\mu \\
N'=& D+\Box C \\
V'_\mu =&-\partial_\mu M \\
\lambda'=&D'=0
\end{align*}
となる.もし,このように定義された超場$S'$がゼロ,つまり
\begin{align*}
(\bar{\mc{D}}\mc{D})S=0
\end{align*}
であるならば,多重項$S$は\textbf{線形}だという.これは$S'$の成分場が全てゼロであることに対応するから,
\begin{align*}
N=M=\partial_\mu V^\mu=0,\quad \lambda=-\Slash{\partial}\omega ,\quad D=-\Box C
\end{align*}
となる.これにより$C$と,条件$\partial_\mu V^\mu=0$を満たす$V_\mu$の独立な3成分(条件が一つだから一つは独立でない)のあわせて4つのボゾン場と,マヨラナ4スピノル$\omega$の4つの独立なフェルミオンが残る.再び前節の最後に示したように,実際にボゾンとフェルミオンが同数あることがわかる.26.6節では,その$V_\mu$が対称性変換に伴う保存カレントであるようなカレント超場は線形超場であることを見る.


\newpage

\subsection{カイラル超場のくりこみ可能な理論}
さて,スカラー・カイラル超場の一般的なくりこみ可能な理論の詳細を調べる.これにより超対称性の帰結について見通しが得られ,ここで得る理論は28章で論じる超対称標準模型の一部を構成するらしい.\par
12.2節で述べた通り,くりこみ可能な理論のラグランジアン密度は($\hbar=c=1$としてエネルギーか運動量の次数で)次元が4以下の演算子のみを含むことができるのだった.(26.2.6)を見れば,二つの$\mc{Q}_\alpha$で次元1の時空微分となるのだから,$\mc{Q}_\alpha$は次元$+1/2$を持ち,
\begin{align*}
\frac{1}{2}=d(\mc{Q}_\alpha)=d\left((\gamma_5\epsilon)\frac{\partial}{\partial \theta_\gamma}+\gamma^\mu_{\alpha\gamma}\theta_\gamma \frac{\partial}{\partial x^\mu}\right)
\end{align*}
および$\partial /\partial \theta_\alpha$は次元$+1/2$を持つことがわかる.これにより$\mc{D}_\alpha$は次元$+1/2$,$\theta_\alpha$は次元$-1/2$を持つこともわかる.
\begin{align*}
d(\mc{D}_\alpha)=&d\left(\frac{\partial}{\partial \theta_\alpha}\right)=\frac{1}{2} \\
d(\theta_\alpha)=&-\frac{1}{2}
\end{align*}
さて,超場$S$の$\mc{F}$項と$D$項はそれぞれ$\theta$の因子を2つと4つ持つ項の係数なのだった.したがって,その超場が次元$d(S)$を持つならば,その$\mc{F}$項と$D$項は
\begin{align*}
d(S)=&d(\cdots +\mc{F}^S(\theta_L^T \epsilon \theta_L))=d(\mc{F}^S)-\frac{1}{2}\times 2=d(\mc{F}^S)-1 \quad \therefore d(\mc{F}^S)=d(S)+1 \\
d(S)=&d\left(\cdots -\frac{1}{4}(\bar{\theta}\gamma_5 \theta)D^S\right)=d(D^S)-\frac{1}{2}\times 4 =d(D^S)-2 \quad \therefore d(D^S)=d(S)+2
\end{align*}
から$d(\mc{F}^S)=d(S)+1$と$d(D^S)=d(S)+2$を持つ.これにより,くりこみ可能な理論を構成するためには$d(\mc{F}^f)=4=d(D^K)$となるように,(26.3.30)の関数$f$と$K$はそれぞれ,
\begin{align*}
4=&d(\mc{F}^f)=d(f)+1 \quad \therefore d(f)=3 \\
4=&d(D^K)=d(K)+2 \quad \therefore d(K)=2
\end{align*}
であるから,演算子次元が高々$3$と$2$の項からなることがわかる.\par
(26.2.10)より$\Phi_n=\phi+\cdots $で,基本的なスカラー超場$\Phi_n$の次元は基本的スカラー場の次元($+1$)と同じとなる必要があるから次元$+1$だ.
\begin{align*}
d(\Phi_n)=+1
\end{align*}
したがって演算子次元3以下である左カイラルな関数$f$の項はどれも,$\Phi_n$やその微分$\partial/\partial x^\mu$,もしくはスピノル超微分$\mc{D}_\alpha$の対の因子を高々3つしか含むことができない.
\begin{align*}
f=\{\partial_\mu \partial_\nu \Phi_n,\quad \partial_\mu \mc{D}_\alpha \Phi_n ,\quad \Phi_n\mc{D}_\alpha \Phi_m,\quad \mc{D}_\alpha \mc{D}_\beta \Phi_n ,\quad \Phi_n \Phi_m \Phi_k ,\quad \cdots \}
\end{align*}
さらに前の節で論じたように,超微分から構成される$f$のどんな左カイラル項も$K$の項で置き換えることができるので,$f$において超微分は落とすことができる.(26.2.30)から,時空微分は超微分を使って表されることがわかるから,それらも同様の議論から省略することができる.(いずれにせよローレンツ不変性を考えれば時空微分一つだけの項はスカラー超場$\Phi_n$とスピノル超微分$\mc{D}_\alpha$ではローレンツ添え字を縮約できず,排除される.二つの時空微分の項はローレンツ不変性を満たすが,くりこみ可能な理論では次元が3以下だから,そこに$\Phi_n$の因子をひとつ追加することしかできず,$\partial_\mu \partial^\mu \Phi_n$の形となり,結局積分の中に入れると全微分であるから作用に寄与することができない.)したがって,$f(\Phi)$は$\Phi_n$について高々3次の多項式であり,時空微分も超微分も含まないことが結論される!
\begin{align*}
f(\Phi)=\sum_{nml}g_{nml}\Phi_n \Phi_m \Phi_l
\end{align*}
関数$f$は全体が左カイラルである必要があるから,右カイラル的な因子となる$\Phi^*_n$は含めることができない.\par
$K$についても同様の次元解析をしてやる.くりこみ可能な理論では$K$は高々次元2なのだったから,$\Phi_n$の因子1つに時空微分かスピノル超微分$\mc{D}_\alpha$がかかったものになっているか,$\Phi_n\Phi_m$か$\Phi_n \Phi^*_m$の形になっているという候補が考えられる.
\begin{align*}
K=\{\partial_\mu \Phi_n,\quad \mc{D}_\alpha \Phi_n , \quad \mc{D}_\alpha \Phi_m^*, \quad \Phi_n \Phi_m, \quad \Phi_n \Phi_m^* , \quad \Phi_n^* \Phi_m^* \cdots \}
\end{align*}
しかし先程と同様にローレンツ不変性から1つだけの時空微分はありえず,スピノル超微分に関しては(26.3.32)から作用に寄与しない.したがってくりこみ可能な理論においては$K$は$\Phi_n$と$\Phi^*_n$の高々の高々2次の関数であり,微分を含まないことがわかる.
\begin{align*}
K=\{\Phi_n \Phi_m ,\quad \Phi_n \Phi_m^* , \quad \Phi_n^* \Phi_m^*\}
\end{align*}
さらに,$\Phi_n$のみか$\Phi_n^*$のみを含む$K(\Phi,\Phi^*)$の項は,必ず左カイラルか右カイラル超場となり,定義(26.3.1)からカイラル超場は$D$項を持たない.よって$\Phi_n$と$\Phi^*_n$の\uwave{両方}を含む$K(\Phi,\Phi^*)$の項のみが$[K(\Phi,\Phi^*)]_D$に寄与する.したがって考えるべき$K(\Phi,\Phi^*)$は以下の形でなければならない.
\begin{align*}
K(\Phi,\Phi^*)=\sum_{nm}g_{nm}\Phi^*_n \Phi_m
\end{align*}
ここで$g_{nm}$は,(全体が実となるように)エルミート行列をなす定数係数だ.

\vskip\baselineskip

さて,$f(\Phi)$の$\mc{F}$項と$K(\Phi,\Phi^*)$の$D$項を計算しなければならない.$K(\Phi,\Phi^*)$の$D$項を求めるために,二つの左カイラル超場の積$\Phi^*_n \Phi_m$の中の$\theta$について4次の項を求める必要がある.$\Phi_n,\Phi_m$は左カイラル超場だから条件(26.3.1)を満たしており,$D_n=\lambda_n=D_m =\lambda_m=0$であり,かつ$V_{n\mu}=\partial_\mu Z _n,V_{m\mu}=\partial_\mu Z_m$と書ける.したがって二つの超場の積(26.2.25)より
\begin{align*}
[\Phi^*_n\Phi_m]_D=&-\partial_\mu C^*_n \partial^\mu C_m +M_n^* M_m +N^*_n N_m \\
&-\frac{1}{2}(\bar{\omega}_n \gamma^\mu \partial_\mu \omega_m)-\frac{1}{2}(\bar{\omega}_m \gamma^\mu \partial_\mu \omega_n)-\partial_\mu Z_n^* \partial^\mu Z_m
\end{align*}
さらに(26.3.8)の対応を代入する.(26.3.10)より,カイラル超場$X$が左カイラル超場$\Phi$であるためには$\tilde{\Phi}$がゼロ,つまり成分場(26.3.14)$\tilde{\phi},\psi_R,\tilde{\mc{F}}$が全てゼロとなり$\phi=A=iB,\sqrt{2}\psi_L=\psi,\mc{F}=F=iG$であることと対応している((26.3.13)を見て$\phi=\sqrt{2}A=i\sqrt{2}B,\psi_L=\psi,\mc{F}=\sqrt{2}F=i\sqrt{2}G$と思うかもしれないが,(26.3.10)で$X=\Phi$とするには分母の$\sqrt{2}$と打ち消しあうように(26.3.11)の各成分場を$\sqrt{2}$倍しなければならないことを考慮するとこの対応になる)から
\begin{align*}
[\Phi^*_n\Phi_m]_D=&-\partial_\mu A^*_n \partial^\mu A_m -\partial_\mu B_n^* \partial^\mu B_m +G_n^* G_m +F^*_n F_m \\
&-\frac{1}{2}(\bar{\psi}_n \gamma^\mu \partial_\mu \psi_m)-\frac{1}{2}(\bar{\psi}_m \gamma^\mu \partial_\mu \psi_n) \\
=&-2\partial_\mu \phi^*_n \partial^\mu \phi_m +2\mc{F}^*_n \mc{F}_m \\
&-(\overline{\psi_{nL}} \gamma^\mu \partial_\mu \psi_{mL})-(\overline{\psi_{mL}} \gamma^\mu \partial_\mu \psi_{nL}) \\
=&-2\partial_\mu \phi^*_n \partial^\mu \phi_m +2\mc{F}^*_n \mc{F}_m \\
&-(\overline{\psi_{nL}} \gamma^\mu \partial_\mu \psi_{mL})+(\partial_\mu (\overline{\psi_{nL}}) \gamma^\mu \psi_{mL})
\end{align*}
よって
\begin{align*}
\frac{1}{2}\Bigl[K(\Phi,\Phi^*)\Bigr]_D =&\sum_{nm}g_{nm}\Bigl[-\partial_\mu \phi^*_n \partial^\mu \phi_m +\mc{F}^*_n \mc{F}_m \\
&-\frac{1}{2}(\overline{\psi_{nL}} \gamma^\mu \partial_\mu \psi_{mL})+\frac{1}{2}(\partial_\mu (\overline{\psi_{nL}}) \gamma^\mu \psi_{mL})\Bigr]
\end{align*}
となる.(本文通り展開してもよいが,せっかく成分場の積を計算したのだから利用したいと思った.本文でやっていることは成分場の積を計算するときと全く同じ手順なので行間計算は省略する.)もし$\Phi_n$を新しい超場$\Phi'_m$の線形結合$\Phi_n=\sum_{m}N_{nm}\Phi'_m$で書いたならば,$K(\Phi,\Phi^*)$は新しい超場を使って
\begin{align*}
K(\Phi,\Phi^*)=&\sum_{nm}\Phi^*_n g_{nm} \Phi_m \\
=&\sum_{nm}\Phi'^*_n (N^\dagger g N)_{nm}\Phi'_m=\sum_{nm}g'_{nm}\Phi'^*_n \Phi'_m
\end{align*}
と書けるから,これは$g_{nm}$を$g'_{nm}=(N^\dagger g N)_{nm}$で置き換えたのと同じ表式だ.スカラー場とスピノル場の運動項が量子交換・反交換関係と矛盾しない符号を持つためには,エルミート行列$g_{nm}$が正定値でなければならない.(これは,例えばスカラー場の運動項が$-A_{nm}\partial_\mu \phi^*_n \partial^\mu \phi_m$となるとする.$A$はエルミートだから$\phi_n$の再定義で対角化できる.$\pi_n=\partial \mc{L}/\partial (\partial_0 \phi^*_n)=\sum_mA_{nm}\partial_0\phi_m$が得られ,$A$が対角とすれば$\pi_n=a_n \partial_0 \phi_n$という形になる.$[\phi_n,\pi_m]=i\delta_{nm}\delta^3$を要請すると,各対角要素$a_n$は正の値にならなければならない.もし負だと2巻7章p14の脚注で述べている通り,生成消滅演算子で自由ハミルトニアンを展開したときに正のエネルギーが出てこない.フェルミオンについても同様)12.5節で示しているように,これは$g'_{nm}=\delta_{nm}$となるように$N$を選べることを意味する.プライムを落として書くと,項(26.4.2)はいまや
\begin{align*}
\frac{1}{2}\Bigl[K(\Phi,\Phi^*)\Bigr]_D =&\sum_n \Bigl[-\partial_\mu \phi^*_n \partial^\mu \phi_n +\mc{F}^*_n \mc{F}_n \\
&-\frac{1}{2}(\overline{\psi_{nL}} \gamma^\mu \partial_\mu \psi_{nL})+\frac{1}{2}(\partial_\mu (\overline{\psi_{nL}}) \gamma^\mu \psi_{nL})\Bigr]
\end{align*}
となる.この(26.4.3)の形を変えずに超場をユニタリー変換して再定義することはまだ可能だ.($K=\sum_n \Phi^*_n \Phi_n$という形になっているから,ユニタリー行列によって$\Phi'=N\Phi$と線形変換しても$K$は変わらない)この自由度は以下ですぐに必要になるらしい.\par
(26.4.3)で$\phi_n$と$\psi_{nL}$を含む項は,通常のように規格化された複素スカラー場とマヨラナ・スピノル場のラグランジアンの正しい運動項になっている!質量項を調べた後に,このフェルミオン項をより馴染みのある形に書き換える.


\vskip\baselineskip


$f(\Phi)$の$\mc{F}$項を計算するには,(26.3.21)の超場の表現を使い,$\theta_L$について二次の項を拾い上げるのが一番便利だ.
\begin{align*}
&\Bigl[f(\Phi(x,\theta))\Bigr]_{\theta_L^2} \\
&=\Bigl[f\left(\phi(x_+)-\sqrt{2}(\theta^T_L \epsilon \psi_L(x_+))+\mc{F}(x_+)(\theta^T_L \epsilon \theta_L)\right)\Bigr]_{\theta_L^2}\\
=&\biggl[f(\phi(x_+))+\sum_n\frac{\partial f}{\partial \phi_n}(\phi(x_+))\left\{-\sqrt{2}(\theta_L^T \epsilon \psi_{nL}(x_+))+\mc{F}_n(x_+)(\theta_L^T \epsilon \theta_L)\right\} \\
&+\frac{1}{2}\sum_{nm}\frac{\partial^2 f}{\partial \phi_n \partial \phi_m}(\phi(x_+))\left\{-\sqrt{2}(\theta_L^T \epsilon \psi_{nL}(x_+))+\mc{F}_n(x_+)(\theta_L^T \epsilon \theta_L)\right\}  \\
&\qquad \qquad \qquad \qquad \qquad \times \left\{-\sqrt{2}(\theta_L^T \epsilon \psi_{mL}(x_+))+\mc{F}_m(x_+)(\theta_L^T \epsilon \theta_L)\right\}\biggr]_{\theta^2_L} \\
=&\sum_{nm}(\theta_L^T \epsilon \psi_{nL}(x_+))(\theta^T_L \epsilon \psi_{mL}(x_+))\sum_{nm}\frac{\partial^2 f}{\partial \phi_n \partial \phi_m}(\phi(x_+)) +\sum_n \mc{F}(x_+)\frac{\partial f}{\partial \phi_n}(\phi(x_+)) \\
=&\sum_{nm}(\theta_L^T \epsilon \psi_{nL}(x))(\theta^T_L \epsilon \psi_{mL}(x))\sum_{nm}\frac{\partial^2 f}{\partial \phi_n \partial \phi_m}(\phi(x)) +\sum_n \mc{F}(x_+)\frac{\partial f}{\partial \phi_n}(\phi(x)) 
\end{align*}
最後の等号では,$x_+$を$x$で置き換えた.$x$まわりで展開すると(26.3.23)より,さらに$\theta_L$の高次の項が出るが,$\theta_L$について3次以上の項はゼロとなることからこれが許される.右辺の第一項の$\theta$依存性は(26.A.11)を使って標準形にかける.
\begin{align*}
(\theta_L^T \epsilon \psi_{nL})(\theta^T_L \epsilon \psi_{mL})=&\left(\theta^T \epsilon \left(\frac{1+\gamma_5}{2}\right)\psi_{n}\right)\left(\theta^T \epsilon \left(\frac{1+\gamma_5}{2}\right)\psi_{m}\right) \\
=&\left(\bar{\theta}\left(\frac{1+\gamma_5}{2}\right)\psi_{n}\right)\left(\bar{\theta} \left(\frac{1+\gamma_5}{2}\right)\psi_{m}\right) \\
=&\left(\bar{\psi}_n\left(\frac{1+\gamma_5}{2}\right)\theta\right)\left(\bar{\theta} \left(\frac{1+\gamma_5}{2}\right)\psi_{m}\right) \\
=&-\frac{1}{4}\left(\bar{\psi}_n\left(\frac{1+\gamma_5}{2}\right)\left(\frac{1+\gamma_5}{2}\right)\psi_m\right)(\bar{\theta}\theta) \\
&+\frac{1}{4}(\bar{\psi}_n \left(\frac{1+\gamma_5}{2}\right)\gamma_5 \gamma_\mu \left(\frac{1+\gamma_5}{2}\right)\psi_m)(\bar{\theta}\gamma_5 \gamma_\mu \theta) \\
&-\frac{1}{4}\left(\bar{\psi}_n \left(\frac{1+\gamma_5}{2}\right)\gamma_5 \left(\frac{1+\gamma_5}{2}\right) \psi_{m}\right)(\bar{\theta}\gamma_5 \theta) \\
=&-\frac{1}{4}(\bar{\psi}_{nL}\psi_{mL})[(\bar{\theta}(1+\gamma_5) \theta)]\\
=&-\frac{1}{2}(\bar{\psi}_{nL}\psi_{mL})(\theta^T_L \epsilon \theta_L)
\end{align*}
となる.ここで$\bar{\psi}_{nL}$は$\overline{\psi_{nL}}$ではなく,$\bar{\psi}_n$の左巻き成分$\overline{\psi_{nR}}$のことだ.任意の左カイラル超場の$\mc{F}$項は$(\theta_L^T \epsilon \theta_L)$の係数だったのだから
\begin{align*}
\Bigl[f(\Phi)\Bigr]_{\mc{F}}=-\frac{1}{2}\sum_{nm}\frac{\partial^2 f(\phi)}{\partial \phi_n \partial \phi_m}(\bar{\psi}_{nL}\psi_{mL}) +\sum_n \mc{F}_n \frac{\partial f(\phi)}{\partial \phi_n}
\end{align*}
となる.完全なラグランジアン密度は(26.3.30)より,(26.4.3)(26.4.4)の項と(26.4.4)の複素共役の項の和で与えられる.
\begin{align*}
\mc{L}=&\sum_n \biggl[-\partial_\mu \phi^*\partial^\mu \phi_n +\mc{F}^*_n \mc{F}_n -\frac{1}{2}(\overline{\psi_{nL}} \gamma^\mu \partial_\mu \psi_{nL})+\frac{1}{2}(\partial_\mu (\overline{\psi_{nL}}) \gamma^\mu \psi_{nL}) \biggr] \\
&-\frac{1}{2}\sum_{nm}\frac{\partial^2 f(\phi)}{\partial \phi_n \partial \phi_m}(\bar{\psi}_{nL}\psi_{mL}) -\frac{1}{2}\sum_{nm}\left(\frac{\partial^2 f(\phi)}{\partial \phi_n \partial \phi_m}\right)^*(\bar{\psi}_{nL}\psi_{mL})^*  \\
&+\sum_n \mc{F}_n \frac{\partial f(\phi)}{\partial \phi_n}  +\sum_n \mc{F}^*_n \left(\frac{\partial f(\phi)}{\partial \phi_n} \right)^*
\end{align*}
ここで$\psi_{nL}$についての二次の項は
\begin{align*}
(\bar{\psi}_{nL}\psi_{mL})=&(\overline{\psi_{nR}}\psi_{mL}) \\
(\bar{\psi}_{nL}\psi_{mL})^*=&\left(\bar{\psi}_n\frac{1+\gamma_5}{2}\psi_{m}\right)^* \\
=&\left(\bar{\psi}_n\frac{1-\gamma_5}{2}\psi_{m}\right) \quad \because (26.A.21) \\
=&(\overline{\psi_{nL}}\psi_{mR})
\end{align*}
と書き換えられることに注意しておく.$\mc{F}_n$は作用に2次で入り時間微分を伴っていないから,これは補助場だ.さらにその係数は定数だから,$\mc{F}_n$をラグランジアン密度(26.4.5)が$\mc{F}_n,\mc{F}^*_n$について停留的になる値
\begin{align*}
\frac{\partial \mc{L}}{\partial \mc{F}_n} =&0 \\
\therefore \quad \mc{F}_n =&-\left(\frac{\partial f(\phi)}{\partial \phi_n }\right)^*
\end{align*}
にとることで,それらを消すことができる.これを(26.4.5)に代入して
\begin{align*}
\mc{L}=&\sum_n \biggl[-\partial_\mu \phi^*\partial^\mu \phi_n -\frac{1}{2}(\overline{\psi_{nL}} \gamma^\mu \partial_\mu \psi_{nL})+\frac{1}{2}(\partial_\mu (\overline{\psi_{nL}}) \gamma^\mu \psi_{nL}) \biggr] \\
&-\frac{1}{2}\sum_{nm}\frac{\partial^2 f(\phi)}{\partial \phi_n \partial \phi_m}(\bar{\psi}_{nL}\psi_{mL}) -\frac{1}{2}\sum_{nm}\left(\frac{\partial^2 f(\phi)}{\partial \phi_n \partial \phi_m}\right)^*(\bar{\psi}_{nL}\psi_{mL})^*  \\
&-\sum_n \left(\frac{\partial f(\phi)}{\partial \phi_n }\right)^* \frac{\partial f(\phi)}{\partial \phi_n}
\end{align*}
を得る.したがってスカラー場のポテンシャルは($\mc{L}=\mc{L}_0-V$なので)
\begin{align*}
V(\phi)=\sum_n \left|\frac{\partial f(\phi)}{\partial \phi_n }\right|^2
\end{align*}
となる.\par
このように補助場を消去すると,残りの場$\psi_{nL}$と$\phi_n$についての超対称性変換(26.3.15)(26.3.17)
\begin{align*}
\delta \psi_{nL}=&\sqrt{2}\partial_\mu \phi_n \gamma^\mu \alpha_R -\sqrt{2} \left(\frac{\partial f(\phi)}{\partial \phi_n }\right)^* \alpha_L \\
\delta \phi_n=&\sqrt{2} \left(\overline{\alpha_R}\psi_{nL}\right)
\end{align*}
のもとで,作用はもはや不変でなくなる.これは,表式(26.4.6)が$\mc{F}_n$について(26.3.16)で与えられる変換則$\delta \mc{F}_n=\sqrt{2} (\overline{\alpha_L}\Slash{\partial}\psi_{nL})$にもはや従わないからだ.実際(26.4.6)の左辺を変換してみると
\begin{align*}
\delta\left(-\frac{\partial f}{\partial \phi_n}(\phi)\right)^* =&-\sum_{m} \left(\frac{\partial^2 f}{\partial \phi_n \partial \phi_m}(\phi)\right)^*\delta \phi_m^* \\
=&-\sqrt{2} \sum_m \left(\frac{\partial^2 f(\phi)}{\partial \phi_n \partial \phi_m}\right)^*(\overline{\alpha_L}\psi_{mR})  \\
&\neq \sqrt{2} (\overline{\alpha_L}\Slash{\partial}\psi_{nL})
\end{align*}
となる.さらに,補助場を消去する前では(26.2.8)より
\begin{align*}
\delta_\beta (\delta_\alpha S)-\delta_\alpha (\delta_\beta S)=&\Bigl[-i(\bar{\beta}Q),\Bigl[-i(\bar{\alpha}Q),S\Bigr]\Bigr]-\Bigl[-i(\bar{\alpha}Q),\Bigl[-i(\bar{\beta}Q),S\Bigr]\Bigr] \\
=&\Bigl[\Bigl[-i(\bar{\beta}Q),-i(\bar{\alpha}Q)\Bigr],S\Bigr] \quad \because ヤコビ恒等式\\
=&-\Bigl[\bar{\beta}_\gamma\alpha_\delta \Bigl\{Q_{\gamma},\bar{Q}_{\delta}\Bigr\},S\Bigr]  \quad \because (\bar{\alpha}Q)=(\bar{Q}\alpha)\\
=&+2i(\bar{\beta}\gamma^\mu \alpha)\Bigl[P_\mu ,S\Bigr] \\
=&-2(\bar{\beta}\gamma^\mu \alpha)\partial_\mu S
\end{align*}
となることから,各成分場の超対称性変換(26.2.12)~(26.2.17)は$[\delta_\beta,\delta_\alpha]\chi=-2(\bar{\beta}\gamma^\mu \alpha)\partial_\mu \chi$となる.ここで添え字の$\alpha,\beta$は4成分スピノル添え字ではなく,パラメータが$\alpha,\beta$だということを表している.(実際(26.3.16)~(26.3.17)などで二回超対称性変換したものの交換子をとってみると簡単に具体例が確かめられる.)これは明らかに閉じた代数になっている.補助場を消去する前では,成分場の超対称性変換の交換子$[\delta_\beta ,\delta_\alpha]$は超対称反交換子$\{Q,\bar{Q}\}$で与えられる.しかし,補助場を消去した後は,$\phi_n$と$\psi_{nL}$の超対称性変換の交換子は超対称反交換関係で与えられず,閉じたリー超代数を構成しない.しかし,これは超対称性の反交換関係を満たす量子力学的演算子$Q_\alpha$の存在とは\uwave{矛盾しない}.(この「量子力学的演算子」とは7章で議論した,今まで議論してきた古典場の対称性変換(7.3.36)の古典的生成子ではなく,正準交換関係を課して正準量子化した量子場を変換する演算子(7.3.40)だ.つまり今までの$Q$とは違う.)これらの演算子は,$-i(\bar{\alpha}Q)$と\uwave{ハイゼンベルグ表示}での任意の量子場$\phi_n$または$\psi_{nL}$の交換子が,微小パラメータ$\alpha$の超対称性変換のもとでのその場の変化に等しい
\begin{align*}
\delta \chi=\left[-i(\bar{\alpha}Q),\chi\right]
\end{align*}
という意味で超対称性変換を生成する.例えば$\delta \psi_{nL}$についての対称性変換を考えてやると
\begin{align*}
\delta\psi_{nL\alpha}=&\sqrt{2}\partial_\nu \phi_n \gamma^\nu \alpha_R -\sqrt{2} \left(\frac{\partial f(\phi)}{\partial \phi_n }\right)^* \alpha_L \\
=&\left[\sqrt{2}\partial_\nu \phi_n \gamma^\nu \frac{1-\gamma_5}{2} -\sqrt{2} \left(\frac{\partial f(\phi)}{\partial \phi_n }\right)^* \frac{1+\gamma_5}{2}\right]\alpha \\
=&i\left[-i\sqrt{2}\partial_\nu \phi_n \gamma^\nu \frac{1-\gamma_5}{2} +i\sqrt{2} \left(\frac{\partial f(\phi)}{\partial \phi_n }\right)^* \frac{1+\gamma_5}{2}\right]\alpha \\
=&i\mc{G}_{\alpha \beta} \alpha_\beta
\end{align*}
だからネーターカレントおよびネーターチャージは,$\psi_{nL},\phi$が場の方程式を満たすものと置くことによって
\begin{align*}
J^\mu_\alpha =&-i\left(\frac{\partial \mc{L}}{\partial (\partial_\mu \psi_{nL})}\right)_\beta \mc{G}_{\beta\alpha}\\
=&\frac{i}{2}\overline{\psi_{nL}}\gamma^\mu \left[-i\sqrt{2}\partial_\nu \phi_n \gamma^\nu \frac{1-\gamma_5}{2} +i\sqrt{2} \left(\frac{\partial f(\phi)}{\partial \phi_n }\right)^* \frac{1+\gamma_5}{2} \right] \\
\Pi_{nL}=& \frac{\partial \mc{L}}{\partial (\partial_0 \psi_{nL})}=-\frac{1}{2}\overline{\psi_{nL}} \gamma^0 \\
\bar{Q}_\alpha=&\int d^3x J^0_\alpha \\
=&\int d^3x\frac{i}{2}\overline{\psi_{nL}}\gamma^0\left[-i\sqrt{2}\partial_\nu \phi_n \gamma^\nu \frac{1-\gamma_5}{2}+i\sqrt{2} \left(\frac{\partial f(\phi)}{\partial \phi_n }\right)^* \frac{1+\gamma_5}{2} \right] \\
=&-i\int d^3x\Pi_{nL\gamma} \left[-i\sqrt{2}\partial_\nu \phi_n \gamma^\nu \frac{1-\gamma_5}{2}+i\sqrt{2} \left(\frac{\partial f(\phi)}{\partial \phi_n }\right)^* \frac{1+\gamma_5}{2} \right]_{\gamma\alpha}
\end{align*}
で与えられる.($Q$ではなく$\bar{Q}$なのは,$\delta\psi$が$\bar{\alpha}$ではなく$\alpha$の形で書いたことに合わせた.)逆に,このネーターチャージは,正準反交換関係を課して量子化した後で,量子場$\psi$の超対称性変換の生成子となっている.
\begin{align*}
\{\psi_{nL}(\mathbf{x},t),\Pi_{nL}(\mathbf{y},t)\}=&i\delta^3(\mathbf{x}-\mathbf{y})\mathbf{1} \\
\{\bar{Q}_\alpha,\psi_{nL\beta}\}=&\left[-i\sqrt{2}\partial_\nu \phi_n \gamma^\nu \frac{1-\gamma_5}{2}+i\sqrt{2} \left(\frac{\partial f(\phi)}{\partial \phi_n }\right)^* \frac{1+\gamma_5}{2} \right]_{\beta\alpha} \\
[-i(\bar{\alpha}Q),\psi_{nL\beta}]=&[-i(\bar{Q}\alpha),\psi_{nL\beta}] \\
=&+i \{\bar{Q}_\alpha,\psi_{nL\beta}\} \alpha_\alpha \\
=&+i \left[-i\sqrt{2}\partial_\nu \phi_n \gamma^\nu \frac{1-\gamma_5}{2}+i\sqrt{2} \left(\frac{\partial f(\phi)}{\partial \phi_n }\right)^* \frac{1+\gamma_5}{2} \right]\alpha \\
=&i\mc{G}_{\beta\alpha}\alpha_{\alpha}=\delta \psi_{nL\beta}
\end{align*}
この$Q_\alpha$がここでいう量子力学的演算子だ!(本当は$\phi$についての項もカレントおよびチャージに追加してやらないと$[-i(\bar{\alpha}Q),\phi_n]=\delta \phi_n$が出せないが,面倒なので省略した.)構成より,ハイゼンベルグ表示の量子場$\psi_{nL},\phi_n$は場の方程式を満たす.この場合ならば,$\mc{F}_n$が(26.4.6)で
\begin{align*}
\mc{F}_n \equiv -\left(\frac{\partial f(\phi)}{\partial \phi_n }\right)^*
\end{align*}
と与えられるとき,$-i(\bar{\alpha} Q)$という量子力学的演算子と$\mc{F}_n$の交換子は$\delta\mc{F}_n=\sqrt{2}(\overline{\alpha_L}\Slash{\partial}\psi_{nL})$で与えられる!これは,ハイゼンベルグ表示では量子場$\psi_{nL}$はラグランジアン(26.4.7)から導かれる場の方程式
\begin{align*}
&\partial_\mu \frac{\partial \mc{L}}{\partial (\partial_\mu \overline{\psi_{nL}})}=\frac{\partial \mc{L}}{\partial \overline{\psi_{nL}}} \\
&\Slash{\partial} \psi_{nL}=-\sum_{n}\left(\frac{\partial^2 f(\phi)}{\partial \phi_n \partial \phi_m}\right)^* \psi_{mR}
\end{align*}
を満たし,
\begin{align*}
\delta \mc{F}_n \equiv& [-i(\bar{\alpha}Q),\mc{F}]=\left[-i(\bar{\alpha}Q),-\left(\frac{\partial f(\phi)}{\partial \phi_n }\right)^*\right] \\
=&-\sum_{m} \left(\frac{\partial^2 f(\phi)}{\partial \phi_n \partial \phi_m}\right)^*\delta \phi_m^* \\
=&-\sqrt{2} \sum_m \left(\frac{\partial^2 f(\phi)}{\partial \phi_n \partial \phi_m}\right)^*(\overline{\alpha_L}\psi_{mR})  \\
=& \sqrt{2} (\overline{\alpha_L}\Slash{\partial}\psi_{nL}) \quad \because 場の方程式
\end{align*}
となるからだ.同様に,量子場$\phi_n$と$\psi_{nL}$の超対称性変換は,場の方程式を考慮に入れると閉じたリー超代数を構成する.(わざわざ確かめなくても,場の方程式を入れたら既存の閉じた変換則(26.3.15)~(26.3.17)を再現することを考えれば,超対称性変換則は自明に閉じた代数$[\delta_\alpha ,\delta_\beta]=2(\bar{\beta}\gamma^\mu \alpha)\partial_\mu $となっていることが成り立つ.)このような代数はしばしば\textbf{殻上}(on-shell)だと言われる.つまり,条件$\delta I/\delta\chi=0$を満たす代数だということだ.15.9節のp56で指摘しているような,「代数は場の方程式が満たされているときにのみ閉じている」というものはこれのことを指している!

\vskip\baselineskip


スカラー場$\phi_n$のゼロ次の期待値$\phi_{n0}$は(26.4.7)の最後の項を最大(ポテンシャルを最低値にするものだから,負符号によって最大値になる)にするものでなければならない.ポテンシャルは絶対値の二乗の形となっており,この項は常に負かゼロだ.
\begin{align*}
V(\phi)=\sum_{n} \left|\frac{\partial f(\phi)}{\partial \phi_n}\right|^2\geq 0
\end{align*}
よって最大値は時空座標に依らない場の値$\phi_{n0}$で実現し,そのとき,この項はゼロとなっている.(真空の並進不変性より$\bra{0}\phi(x)\ket{0}=\bra{0}\phi(0)\ket{0}$なので真空期待値は時空非依存.)したがって
\begin{align*}
\left.\frac{\partial f(\phi)}{\partial \phi_n}\right|_{\phi=\phi_0}=0
\end{align*}
となる.(もちろんこれは,この方程式の解$\phi_{n0}$が存在するとしての話.)この式は,(26.4.7)の最後の項を最大化するだけではなく,超対称性が破れないための条件でもある.超対称性変換のもとで真空が不変であるためには,超対称性変換のもとでのどの場の変化分の真空期待値もゼロでなければならない.(19章参照.)ボゾン場の変化分はフェルミオン場であり,その真空期待値はローレンツ不変性によりもちろんゼロとなる.しかし(26.3.15)から$\delta\psi_{nL}$の真空期待値は補助場$\mc{F}_n$の真空期待値に比例することが分かる.($\partial_\mu \phi$は真空期待値の時間非依存性によりゼロ.)したがって,超対称性が破れていないならばこの項はゼロにならなければならない.(26.4.6)によれば,摂動論のゼロ次でこの条件は($\bra{0}\phi_n\ket{0}=\phi_{n0}$なので)
\begin{align*}
0=\bra{0}\mc{F}_n \ket{0}=-\bra{0}\left(\frac{\partial f(\phi)}{\partial \phi_n}\right)\ket{0}=-\left.\frac{\partial f(\phi)}{\partial \phi_n}\right|_{\phi=\phi_0}
\end{align*}
となり,(26.4.8)が満たされていることを意味する.27.6節で,もし(26.4.8)が満たされているならば,超対称性は摂動論の全次数で破れていないことを見る.\par
左カイラル超場$\Phi$が一つだけの場合については,代数の基本的定理(任意$n$次複素多項式$P(z)=a_0+\cdots +a_nz^n(a_n\neq 0)$は複素数平面$\mathbb{C}$上で$n$個の根を持つ)から,多項式$\partial f(\phi)/\partial \phi$は常にゼロ点を複素空間のどこかに最低1つは持つ.これは超場が二つ以上ある場合には必ずしも正しくない.もし(26.4.8)に解$\phi_{n0}$があると\uwave{仮定}すると
\begin{align*}
\phi_n(x)=\phi_{n0}+\varphi_{n}(x)
\end{align*}
として$\varphi_n$のベキ展開をすることで理論にある物理的自由度を調べることができる.この理論の粒子の質量は,$\varphi$と$\psi$について2次の項を調べることで計算できる.このために対称複素行列
\begin{align*}
\mc{M}_{nm}\equiv \left.\frac{\partial^2 f(\phi)}{\partial\phi_n \phi_m}\right|_{\phi=\phi_0}
\end{align*}
を使って
\begin{align*}
\frac{\partial f}{\partial \phi_n}(\phi_0+\varphi)=&\left.\frac{\partial f(\phi)}{\partial \phi_n}\right|_{\phi=\phi_0}+\sum_m \left.\frac{\partial^2 f(\phi)}{\partial\phi_n \phi_m}\right|_{\phi=\phi_0}\varphi_m+\mc{O}(\varphi^2) \\
=&\sum_m \mc{M}_{nm}\varphi_m+\mc{O}(\varphi^2) \quad \because (26.4.8)\\
\frac{\partial^2 f}{\partial\phi_n \phi_m}(\phi_{0}+\varphi )=&\left.\frac{\partial^2 f(\phi)}{\partial\phi_n \phi_m}\right|_{\phi=\phi_0}+\mc{O}(\varphi) \\
=&\mc{M}_{nm}+\mc{O}(\varphi)
\end{align*}
であることを用いると
\begin{align*}
\mc{L}_0=&\sum_n \biggl[-\partial_\mu \varphi^*\partial^\mu \varphi_n -\frac{1}{2}(\overline{\psi_{nL}} \gamma^\mu \partial_\mu \psi_{nL})+\frac{1}{2}(\partial_\mu (\overline{\psi_{nL}}) \gamma^\mu \psi_{nL}) \biggr] \\
&-\frac{1}{2}\sum_{nm}\mc{M}_{nm}(\bar{\psi}_{nL}\psi_{mL}) -\frac{1}{2}\sum_{nm}\mc{M}^*_{nm}(\bar{\psi}_{nL}\psi_{mL})^*  \\
&-\sum_{nml} \mc{M}^*_{nm}\varphi_m^* M_{nl}\varphi_l \\
=&\sum_n \biggl[-\partial_\mu \varphi^*\partial^\mu \varphi_n -\frac{1}{2}(\overline{\psi_{nL}} \gamma^\mu \partial_\mu \psi_{nL})+\frac{1}{2}(\partial_\mu (\overline{\psi_{nL}}) \gamma^\mu \psi_{nL}) \biggr] \\
&-\frac{1}{2}\sum_{nm}\mc{M}_{nm}(\bar{\psi}_{nL}\psi_{mL}) -\frac{1}{2}\sum_{nm}\mc{M}^*_{nm}(\bar{\psi}_{nL}\psi_{mL})^*  \\
&-\sum_{nm} (\mc{M}^\dagger \mc{M})_{mn}\varphi_m^* \varphi_n
\end{align*}
となる.さて,もし場をユニタリー変換
\begin{align*}
\varphi_n=\sum_{m}\mc{U}_{nm}\varphi'_m , \quad \psi_{n}=\sum_{m} \mc{U}_{nm}\psi'_{m}
\end{align*}
で再定義すると,自由場のラグランジアン(26.4.10)は$\mc{M}$を
\begin{align*}
\mc{M}'=\mc{U}^T \mc{M}\mc{U}
\end{align*}
で置き換えたのと同じ形になる.行列代数の定理を


















\end{document}