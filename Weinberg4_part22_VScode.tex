\documentclass[dvipdfmx]{jsarticle}
\let\headfont=\gtfamily
\usepackage[dvips]{graphicx}
\usepackage{amsmath}
\usepackage{mathrsfs} % 花文字\mathscr{M}, 筆記体\mathcal{M}, 黒板文字\mathbb{M},ドイツ文字\mathfrak{M}
\usepackage{bm} %太文字
\usepackage{amssymb}
\usepackage{latexsym}
\usepackage{braket}
\usepackage{tikz}
\usepackage{tikz-feynhand}
\usepackage{ulem}
\usepackage{tensor}
\usepackage{bigdelim}
\usepackage{multirow}
\usepackage{tcolorbox}
\usepackage{here}
\tcbuselibrary{theorems,skins}
\usetikzlibrary{decorations}
\usepackage{color}

\usetikzlibrary{intersections, calc, arrows.meta}
 \usetikzlibrary{patterns}

\newfont{\bg}{cmr9 scaled\magstep4}
\newcommand{\bigzerol}{\smash{\lower1.0ex\hbox{\bg 0}}}
\newcommand{\bigzerou}{%
   \smash{\hbox{\bg 0}}}
\newcommand{\mcO}{\mathcal{O}}
\newcommand{\VAC}{\mathrm{VAC}}
\newcommand{\Slash}[1]{{\ooalign{\hfil/\hfil\crcr$#1$}}} %ファインマンのスラッシュ記号
\renewcommand{\mc}{\mathcal}
\newcommand{\mr}[1]{\mathrm{#1}}

% \textrm{Roman デフォルト}
% \textgt{Gothic 和文ゴシック体}*専門用語に
% \textbf{Boldface 太字}*専門用語(英語)に
% \textit{Italic 斜体}
% \textsl{Slanted ローマンを傾けただけ}
% \textsf{Sans Serif サンセリフ体}
% \texttt{Typewriter タイプライタ体、等幅}
% \textsc{Small Caps 小文字が大文字に}

\setlength{\textwidth}{\fullwidth}
\setlength{\textheight}{44\baselineskip}
\addtolength{\textheight}{\topskip}
\setlength{\voffset}{-0.6in}

\allowdisplaybreaks[4]

\makeatletter
  \renewcommand{\theequation}
  {\arabic{section}.\arabic{equation}}
  \@addtoreset{equation}{section}
 \makeatother

\title{\vspace{-1cm}\Huge{WeinbergQFT Part22}}
\author{坂井 啓悟(Sakai Keigo)}
\date{}
\begin{document}



\maketitle
\setcounter{part}{21}
\part{アノマリー}
\setcounter{section}{22}
\setcounter{subsection}{0}
\subsection{$\pi^0$崩壊の問題}
重い束縛された粒子を全て積分した後には,$\pi^0\to 2\gamma$の相互作用は以下のラグランジアンで与えられると考えられる.
\begin{align*}
\mc{L}_{\pi\gamma\gamma}=g\pi^0 \epsilon^{\mu\nu\rho\lambda}F_{\mu\nu}F_{\rho\lambda}
\end{align*}
19章で述べた通り,$\pi^0$は微分を必ず一つ持って表れるため,ヤンミルズ項$F_{\mu\nu}F^{\mu\nu}$との積では微分を3つ以上含んでしまう.したがって微分を3つ以上含まない有効ラグランジアンはこの形に限定され,部分積分によって$\pi^0$場の微分を外したものだと理解できる.西島和彦著「Fields and particles」あるいは「場の理論」の計算によると,これより
\begin{align*}
\Gamma(\pi^0\to2\gamma)=\frac{m^3_\pi g^2}{\pi}
\end{align*}
となるらしい.この反応は($\epsilon^{\mu\nu\rho\lambda}$によりカイラルを保存しないので)1ループを必ず含む.したがって単純に考えれば$g$は
\begin{align*}
g\approx \frac{\alpha}{2\pi F_\pi}=\frac{e^2}{8\pi^2F_\pi}
\end{align*}
となる.\par
カイラル不変性を課すと,崩壊率は(22.17)のように小さくなるが,実際の崩壊率は(22.1.3)から得られる値に近いと分かっている.すなわち,カイラル対称性を破る異常な何かがあると結論付けられる.これがアノマリーである.

\newpage

\subsection{測度の変換:可換アノマリー}
ここではミンコフスキー空間における経路積分を用いる.まず質量ゼロのスピン1/2複素フェルミオン場の列ベクトル$\psi_n(x)$についての任意の局所行列変換$\psi(x)\to U(x)\psi(x)$のもとでの測度のアノマリーを計算する.これらはフェルミオン変数で$\psi$と$\bar{\psi}$は独立な変数であるから,これらの変換は(連続的な添え字の縮約は積分であることを思い出して)
\begin{align*}
\psi_n(x)&\to U(x)_{nm}\psi_m(x)=\int d^4y \, U(x)_{nm}\delta^4(x-y)\cdot \psi_m(y)\equiv \mc{U}_{xn,ym}\psi_m(y) \\
\bar{\psi}_n(x)&\to\left[U(x)\psi(x)\right]^\dagger \gamma_4 \\
&=\psi^\dagger(x)\gamma_4 \gamma_4U(x)^\dagger \gamma_4 =\int d^4 \, y \bar{\psi}(y)\cdot \gamma_4 U(y)^\dagger \gamma_4 \delta^4(y-x) \\
&\equiv \bar{\psi}_m(y)\bar{\mc{U}}_{ym,xn}
\end{align*}
となって,測度は(9.5.38)の通り変換して
\begin{align*}
[d\psi][d\bar{\psi}]\to (\mr{Det}\mc{U}\mr{Det}\bar{\mc{U}})^{-1}[d\psi][d\bar{\psi}]
\end{align*}
となる.ここで$\gamma_4=i\gamma_0$で$\beta$と同じ行列だ.添え字$n,m$はフレーバーと,ディラックのスピンの添え字についてとる.(つまり,ガンマ行列の添え字も含めていることに注意.)\par
$\mr{Det}AB=\mr{Det}A\mr{Det}B$だから,$\mr{Det}\gamma_4=1$より
\begin{align*}
\mr{Det}\bar{\mc{U}}=\mr{Det}\gamma_4 \mr{Det}\mc{U}^\dagger \mr{Det}\gamma_4 =\mr{Det}\mc{U}^\dagger
\end{align*}
となるから,$\gamma_4$は行列式に影響しない.それにもかかわらずこれを計算に含めるのは,プロパゲータや行列式においてフェルミオンのモードについての和を正則化する必要があり,正則化された行列式には$\gamma_4$が実際に影響することがあるからだ.

\vskip\baselineskip

まず,$U(x)$がユニタリーな非カイラル的変換
\begin{align*}
U(x)=\exp[i\alpha(x)t]
\end{align*}
である場合を考える.ここで$t$は($\gamma_5$を含まないが必ずしもトレースレスでなくてもよい)エルミート行列だ.また$\alpha(x)$は任意の実関数だ.この場合は$U(x)$は$\gamma_4$と可換で$\gamma_4 U(x)^\dagger \gamma_4 =U(x)^\dagger$となり,かつ$U(x)^\dagger=U(x)^{-1}$だから
\begin{align*}
\bar{\mc{U}}\mc{U}&=\bar{\mc{U}}_{xn,ym}\mc{U}_{ym,z\ell}=\int d^4y\, U(x)^\dagger_{nm}\delta^4(x-y)U(y)_{m\ell}\delta^4(y-z) \\
&=\delta_{n\ell}\delta^4(x-z)
\end{align*}
となる.したがってこの変換では$\mr{Det}\mc{U}\mr{Det}\bar{\mc{U}}=1$なので,測度は不変だ.\par
$\Rightarrow$特に,ゲージ群自身の対称性において,$t$は非カイラル生成子の一つ$t_\alpha$になっていて,この対称性がアノマリーによって破られることはない.\par
次に,以下の局所カイラル変換を考える.
\begin{align*}
U(x)=\exp[i\gamma_5\alpha(x)t]
\end{align*}
この場合,$\mc{U}$は
\begin{align*}
\gamma_4 U(x)^\dagger \gamma_4=&\gamma_4 \exp[-i\gamma_5^\dagger \alpha(x)t]\gamma_4 \\
=&\exp[i\gamma_5 \alpha(x)t]=U(x)\quad \because (5.4.33), \gamma_4\gamma_5^\dagger \gamma_4=-\gamma_5
\end{align*}
より擬エルミート
\begin{align*}
\bar{\mc{U}}=\mc{U}
\end{align*}
がわかる.よって測度はカイラル変換のもとで不変ではなく,
\begin{align*}
[d\psi][d\bar{\psi}]\to (\mr{Det}\mc{U})^{-2}[d\psi][d\bar{\psi}]
\end{align*}
となる.以下ではこれを解析していく.\par
さて,微小局所カイラル変換の場合に特定する.(22.2.6)において$\alpha(x)$が微小とすると
\begin{align*}
U(x)&=1+i\gamma_5\alpha(x)t \\
\Rightarrow\quad \mc{U}_{xn,ym}&=\delta_{nm}\delta^4(x-y)+i\alpha(x)[\gamma_5t]_{nm}\delta^4(x-y) \\
[\mc{U}-1]_{xn,ym}&=i\alpha(x)[\gamma_5t]_{nm}\delta^4(x-y)
\end{align*}
となる.ここで恒等式$\mr{Det}M=\exp\mr{Tr}\ln M$を用いる.(これを証明するには,行列$A$のジョルダン標準形$J$とすると$A=PJP^{-1}$であるから,行列式が固有値の積と同値であることを用いれば,$A$の固有値を$\lambda_1,\lambda_2,\cdots \lambda_n$として
\begin{align*}
\mr{Det}e^A&=\mr{Det} e^{(PJ P^{-1})}=\mr{Det}(Pe^J P^{-1})=\mr{Det}e^J \\
&=e^{\lambda_1}e^{\lambda_2}\cdots e^{\lambda_n}=e^{\lambda_1+\lambda_2+\cdots \lambda_n} \\
&=\exp \mr{Tr} A
\end{align*}
となって,$\exp A=M$とおけば恒等式が示せる.)$\alpha(x)$が微小であるから,$x\to0$で$\ln(1+x)\to x$であることを用いることができて
\begin{align*}
[d\psi][d\bar{\psi}]\to& (\mr{Det}\mc{U})^{-2}[d\psi][d\bar{\psi}] \\
&=\exp \left\{-2 \mr{Tr} \ln [1+(\mc{U}-1)]\right\} [d\psi][d\bar{\psi}] \\
&=\exp \left\{ -2\mr{Tr}\left[i\alpha(x)(\gamma_5t)\delta^4(x-y)\right] \right\} \\
&=\exp \left\{ -2i \int d^4x \, \alpha(x) \mr{Tr}[\gamma_5 t] \delta^4(x-x) \right\} \\
&=\exp \left\{ i\int d^4x \, \alpha(x)\mc{A}(x) \right\}
\end{align*}
となる.ここで$\mc{A}$はアノマリー関数
\begin{align*}
\mc{A}(x)=-2\mr{Tr}[\gamma_5 t]\delta^4(x-x)
\end{align*}
だ.ここで$\mr{Tr}$はディラック添え字と種類添え字の両方についてトレースをとることを意味する.測度$[d\psi][d\bar{\psi}]$は経路積分において因子$\exp\left\{i\int d^4 x \mc{L}(x)\right\}$の重み付きで表れるのだったから,測度の変換則(22.2.10)の因子$\exp \left\{ i\int d^4x \, \alpha(x)\mc{A}(x) \right\}$は,あたかもラグランジアン密度$\mc{L}(x)$がこれらの変換で不変でなく,その代わりに$\mc{L}(x)\to \mc{L}(x)+\alpha (x)\mc{A}(x)$となるのと同じ影響をする.したがって,フェルミオンを積分した後の有効ラグランジアンを扱うときには,アノマリーも考慮に入れて,元の不変な項に加えて不変でない項を
\begin{align*}
\mc{L}_{eff}(x)\to \mc{L}_{eff}(x)+\alpha (x)\mc{A}(x)
\end{align*}
となるように入れておかなければならない.あとは,このアノマリー関数の計算だ.

\vskip\baselineskip

アノマリー関数を観察してみると,デルタ関数$\delta^4(x-x)$は無限大だがトレース$\mr{Tr}[\gamma_5 t]$は
\begin{align*}
\gamma_5=\left(
\begin{array}{cc}
1 & 0 \\
0 & -1
\end{array}
\right)
\end{align*}
によりゼロとなる.($t$は$\gamma_5$を含まないので$\gamma_5$のみでディラック添え字のトレースがとれる.)このままでは扱えないため,これらの量に意味を持たせるために正則化因子を導入する.これは,デルタ関数の引数がゼロとなる前に微分演算子$f(-\Slash{D}^2_x/M^2)$をデルタ関数に働かせることにょりゲージ不変に実行できる.(熱核による正則化)
\begin{align*}
\mc{A}(x)=-2[\mr{Tr}\left\{ \gamma_5 t f(-\Slash{D}^2_x/M^2) \right\} \delta^4(x-y)]_{y\to x}
\end{align*}
ここで$D_x$はゲージ場$A^\mu_\alpha(x)$があるもとでのディラック演算子だ.
\begin{align*}
(D_x)_\mu=\frac{\partial}{\partial x^\mu}-it_\alpha A_{\alpha \mu}(x)
\end{align*}
また$M$はある大きな質量で,後に無限大に極限をとる.また,$f(s)$は滑らかな関数で,$s$が$0から\infty$に行くにつれて$f(s)$は$1$から$0$に滑らかに落ちる,という条件さえ満たしていれば良い.(藤川の方法では$\exp(-[\Slash{D}_x/M)]^2$ととっていた.)
\begin{align*}
&f(0)=1,\quad f(\infty)=0 \\
&sf'(s)=0 \quad (s=0,s=\infty において)
\end{align*}
正則化関$f$はゲージ不変性を保つために$\Slash{\partial}$の関数にはとらず,また行列式だけでなくフェルミオン・プロパゲータ$\Slash{D}^{-1}$(このプロパゲータは通常のものではなく,完全なプロパゲータ(14.2.1)であることに注意.)も正則化する必要がある.したがって$D^\mu D_\mu$の関数にもとらないことに注意.(22.2.22)の通り,$\Slash{D}^2_x=D^2_x$ではないことにも注意だ.\par
(22.2.13)の表式を計算するために,デルタ関数のフーリエ積分表示を用いて,アノマリー関数を
\begin{align*}
\mc{A}(x)=&-2\int \frac{d^4 k}{(2\pi)^4}\left[\mr{Tr}\left\{\gamma_5 tf(-\Slash{D}^2_x/M^2) \right\}e^{ik(x-y)}\right]_{y\to x} \\
=&-2\int \frac{d^4 k}{(2\pi)^4}\mr{Tr}\left\{\gamma_5 tf(-[i\Slash{k}+\Slash{D}_x]^2/M^2) \right\}
\end{align*}
と書く.ここで
\begin{align*}
\Slash{D}_x^2 e^{ik(x-y)}=&\left\{ \Slash{\partial}_x-it_\alpha \Slash{A}_{\alpha}(x) \right\}\left\{ \Slash{\partial}_x-it_\alpha \Slash{A}_{\alpha}(x) \right\}e^{ik(x-y)} \\
=&\left\{ \Slash{\partial}_x-it_\alpha \Slash{A}_{\alpha}(x) \right\}\left\{i\Slash{k}-it_\alpha \Slash{A}_\alpha(x)\right\}e^{ik(x-y)} \\
=&\left\{ \Slash{\partial}_x-it_\alpha \Slash{A}_{\alpha}(x) \right\}i\Slash{k}e^{ik(x-y)}+\left\{ \Slash{\partial}_x-it_\alpha \Slash{A}_{\alpha}(x) \right\}(-it_\alpha \Slash{A}_\alpha(x))e^{ik(x-y)} \\
=&\left\{ i\Slash{k}-it_\alpha \Slash{A}_{\alpha}(x) \right\}i\Slash{k}e^{ik(x-y)} \\
&+i\Slash{k}(-it_\alpha \Slash{A}_\alpha(x))e^{ik(x-y)}+\left[ \Slash{\partial}(-it_\alpha \Slash{A}_\alpha(x))\right]e^{ik(x-y)}+(-it_\alpha \Slash{A}_\alpha(x))^2e^{ik(x-y)} \\
\overset{y\to x}{\longrightarrow} \quad &(i\Slash{k})^2+(-it_\alpha \Slash{A}_\alpha(x))i\Slash{k}+i\Slash{k}(-it_\alpha \Slash{A}_\alpha(x))+[\Slash{\partial}(-it_\alpha \Slash{A}_\alpha(x))]+(-it_\alpha \Slash{A}_\alpha(x))^2 \\
=&(i\Slash{k})^2+(\Slash{\partial}-it_\alpha \Slash{A}_\alpha(x))i\Slash{k}+i\Slash{k}(\Slash{\partial}-it_\alpha \Slash{A}_\alpha(x))+(\Slash{\partial}-it_\alpha \Slash{A}_\alpha(x))^2 \\
=& [i\Slash{k}+\Slash{D}_x]^2
\end{align*}
となることを用いた.ここでは微分演算子が作用する関数を省略しているのではなく$1$に作用していることに注意.極限をとった後の等号はどれも複雑だが,逆算してみれば成り立つことが確認できる.運動量$k^\mu$をスケール変換$k^\mu \to Mk^\mu$すると
\begin{align*}
\mc{A}(x)=&-2\int \frac{d(Mk)^4}{(2\pi)^4}\mr{Tr}\left\{\gamma_5 tf(-[iM\Slash{k}+\Slash{D}_x]^2/M^2) \right\} \\
=&-2M^4 \int \frac{d^4k}{(2\pi)^4}\mr{Tr}\left\{\gamma_5 tf(-[i\Slash{k}+\Slash{D}_x/M]^2) \right\}
\end{align*}
となる.切断関数の変数は
\begin{align*}
-\left[ i\Slash{k} +\frac{\Slash{D}}{M} \right]^2=&-(i\Slash{k})^2-i\Slash{k}=-(i\Slash{k})^2-i\Slash{k}\frac{\Slash{D}_x}{M}-\frac{\Slash{D}_x}{M}i\Slash{k}-\left(\frac{\Slash{D}_x}{M}\right)^2 \\
=&\frac{k^\mu k^\nu}{2} \left\{ \gamma_\mu ,\gamma_\nu \right\}-\frac{ik^\mu D_x^\nu}{M}\left\{ \gamma_\mu ,\gamma_\nu \right\}+\left(\frac{\Slash{D}_x}{M}\right)^2 \\
=&k^2-\frac{2ik\cdot D_x}{M}-\left(\frac{\Slash{D}_x}{M}\right)^2 \quad \because \left\{ \gamma_\mu ,\gamma_\nu \right\}=2\eta_{\mu\nu}
\end{align*}
と書ける.$M\to \infty$の極限では,(22.2.17)は$f(-[i\Slash{k}+\Slash{D}_x/M]^2)$の展開のなかで$1/M$の因子を5個以上持たない項と,(8.A.10)(8.A.11)(8.A.12)よりディラックのガンマ行列を最低4個持つ項のみから寄与を受ける.これにより,$\Slash{D}_x^2$について2次の項のみが残る.テイラー展開すれば
\begin{align*}
f\left(k^2-\frac{2ik\cdot D_x}{M}-\left(\frac{\Slash{D}_x}{M}\right)^2\right)=&f(k^2)+f'(k^2)\left(-\frac{2ik\cdot D_x}{M}-\left(\frac{\Slash{D}_x}{M}\right)^2\right) \\
&+f''(k^2)\left(-\frac{2ik\cdot D_x}{M}-\left(\frac{\Slash{D}_x}{M}\right)^2\right)^2+\cdots \\
=&f''(k^2)\frac{\Slash{D}^2_x}{M^4} +\cdots
\end{align*}
となるから
\begin{align*}
\mc{A}(x)=-\int \frac{d^4k}{(2\pi)^4}f''(k^2)\mr{Tr}\left\{\gamma_5 t \Slash{D}^4_x\right\}
\end{align*}
が残る.これは正則化質量$M$に独立だ.\par
$k$積分を計算するには,(11.2.10)で用いた方法を再び用いれば良い.すなわち,$k^0$積分経路をファインマンダイアグラムを計算する際と同様に回転させ,$k^0$を$ik^4$に置き換え,$k^4$は$-\infty$から$+\infty$まで動くとする.そして(11.2.10)の方法を用いて,4次元球面の面積は$2\pi^2$であるから$\kappa=\sqrt{k^2}$とすれば4次元極座標表示にできる.
\begin{align*}
\int d^4k\, f''(k^2)=i\int (d^4k)_E \, f''(k^2)=i\int^\infty_0 2\pi^2 \kappa^3 d\kappa f''(\kappa^2)
\end{align*}
(22.2.16)(22.2.15)を用いて繰り返し部分積分すると
\begin{align*}
\int d^4 k \, f''(k^2)=&i\int^\infty_0 2\pi^2 \kappa^3 d\kappa f''(\kappa^2) \\
=&i\pi^2 \int^\infty_0 ds\,s f''(s) \quad \because \kappa^2=s ,\,ds=2\kappa d\kappa \\
=&\left[i\pi^2 sf'(s)\right]^\infty_0-i\pi^2\int^\infty_0 dsf'(s)=-i\pi^2\int^\infty_0 dsf'(s) \quad \because (22.2.16) \\
=&-i\pi^2[f(\infty)-f(0)]=i\pi^2 \quad\because (22.2.15)
\end{align*}
となる.トレースを計算するには
\begin{align*}
\Slash{D}^2_x=&\frac{1}{4}\left\{ (D_x)^\mu,(D_x)^\nu \right\}\left\{ \gamma_\mu,\gamma_\nu \right\}+\frac{1}{4}[(D_x)^\mu,(D_x)^\nu][\gamma_\mu,\gamma_\nu] \\
=&\frac{1}{2}\left\{(D_x)^\mu,(D_x)^\nu\right\}\eta_{\mu\nu}-\frac{1}{4}it_\alpha F^{\mu\nu}_\alpha [\gamma_\mu,\gamma_\nu] \quad \because (15.1.12)\\
=&D_x^2-\frac{1}{4}it_\alpha F^{\mu\nu}_\alpha [\gamma_\mu,\gamma_\nu]
\end{align*}
と書けることを用いる.最初の等号に関しては
\begin{align*}
&\frac{1}{4}\left\{ (D_x)^\mu,(D_x)^\nu \right\}\left\{ \gamma_\mu,\gamma_\nu \right\}+\frac{1}{4}[(D_x)^\mu,(D_x)^\nu][\gamma_\mu,\gamma_\nu] \\
=&\frac{1}{4}((D_x)^\mu (D_x)^\nu+(D_x)^\nu(D_x)^\mu)(\gamma_\mu \gamma_\nu+\gamma_\nu \gamma_\mu) \\
&+\frac{1}{4}((D_x)^\mu (D_x)^\nu-(D_x)^\nu(D_x)^\mu)(\gamma_\mu \gamma_\nu-\gamma_\nu \gamma_\mu) \\
=&\frac{1}{4}(D_x)^\mu(D_x)^\nu\gamma_\mu\gamma_\nu +\frac{1}{4}(D_x)^\mu(D_x)^\nu\gamma_\nu\gamma_\mu +\frac{1}{4}(D_x)^\nu(D_x)^\mu\gamma_\mu\gamma_\nu+\frac{1}{4}(D_x)^\nu(D_x)^\mu\gamma_\nu\gamma_\mu \\
&+\frac{1}{4}(D_x)^\mu(D_x)^\nu\gamma_\mu\gamma_\nu -\frac{1}{4}(D_x)^\mu(D_x)^\nu\gamma_\nu\gamma_\mu -\frac{1}{4}(D_x)^\nu(D_x)^\mu\gamma_\mu\gamma_\nu+\frac{1}{4}(D_x)^\nu(D_x)^\mu\gamma_\nu\gamma_\mu \\
=& \frac{1}{2}(D_x)^\mu(D_x)^\nu\gamma_\mu\gamma_\nu+\frac{1}{2}(D_x)^\nu(D_x)^\mu\gamma_\nu\gamma_\mu=(D_x)^\mu(D_x)^\nu\gamma_\mu\gamma_\nu =\Slash{D}^2_x
\end{align*}
として確認できる.$(\Slash{D}^2_x)^2$でガンマ行列を4つ含むものは第二項の二次のみであり,これからは(8.A.12)より
\begin{align*}
\mr{tr_D}\left\{ \gamma_5[\gamma_\mu,\gamma_\nu][\gamma_\rho,\gamma_\sigma] \right\}=&\mr{tr_D}\left\{ \gamma_5(\gamma_\mu\gamma_\nu-\gamma_\nu\gamma_\mu)(\gamma_\rho\gamma_\sigma-\gamma_\sigma\gamma_\rho) \right\}  \\
=&\mr{tr_D}\left\{\gamma_5 \gamma_\mu\gamma_\nu \gamma_\rho \gamma_\sigma\right\}-\mr{tr_D}\left\{\gamma_5 \gamma_\mu\gamma_\nu\gamma_\sigma \gamma_\rho\right\}\\
&-\mr{tr_D}\left\{\gamma_5 \gamma_\nu\gamma_\mu \gamma_\rho \gamma_\sigma\right\}+\mr{tr_D}\left\{\gamma_5 \gamma_\mu\gamma_\nu \gamma_\sigma \gamma_\rho\right\} \\
=&4i\epsilon_{\mu\nu\rho\sigma}-4i\epsilon_{\mu\nu\sigma\rho}-4i\epsilon_{\nu\mu\rho\sigma}+4i\epsilon_{\nu\mu\sigma\rho}=16i\epsilon_{\mu\nu\rho\sigma}
\end{align*}
が生じる.ここで$\mr{tr_D}$はディラック添え字についてのみトレースをとることを意味する.したがってアノマリー関数は(22.2.19)に(22.2.21)~(22.2.23)を用いると
\begin{align*}
\mc{A}(x)=&-\frac{i\pi^2}{(2\pi)^4}\left(-\frac{i}{4}\right)^216i\epsilon_{\mu\nu\rho\sigma}F^{\mu\nu}_\alpha(x) F^{\rho\sigma}_\beta(x) \mr{tr}\left\{t_\alpha t_\beta t \right\} \\
=&-\frac{1}{16\pi^2}\epsilon_{\mu\nu\rho\sigma}F^{\mu\nu}_\alpha(x) F^{\rho\sigma}_\beta(x) \mr{tr}\left\{t_\alpha t_\beta t \right\}
\end{align*}
が得られる!ここで$\mr{tr}$は各種フェルミオンの種類添え字についてのみトレースをとることを意味する.$t$が単位行列だという特別な場合には,量(22.2.24)はチャーン・ポントリャーギン密度として知られている.\par
この結果はアノマリーをもつ対称性にともなうカレントを用いて表すことができる.簡単のため,定数の微小パラメータ$\alpha$の対称性変換$\psi(x)\to \psi(x)+it\gamma_5 \alpha \psi(x)$のもとで作用自身が不変だとする.その場合,(7.3.14)で示したように時空に\uwave{依存する}パラメータ$\alpha(x)$で変換を行うと,作用には$\delta I=\int d^4x J^\mu_5 (x)\partial_\mu \alpha(x)$だけの変化が生じる.(この項は作用の$\int d^4x J^\mu_\alpha (x)A_{\alpha\mu}(x)$から生じる.$\delta A_{\alpha\mu}=\partial_\mu \epsilon_\alpha+\cdots $だったのを思い出す.)ここで$J^\mu_5(x)$は,場についての任意の変化のもとで作用が停留するように場の演算子が動的な方程式を満たす時には保存されるカレントだ.すなわち,古典的作用の変化が停留する$\delta I=0$と仮定すると部分積分よりカレントは
\begin{align*}
\partial_\mu J^\mu_5(x)=0
\end{align*}
と保存されることがわかる.しかし,ここで$\delta \psi(x)=it\gamma_5 \alpha (x)\psi(x)$という変数変換を行うと(22.2.10)より
\begin{align*}
\delta \int [d\psi][d\bar{\psi}]e^{iI}=&\int [d\psi][d\bar{\psi}]\exp\left(i\int d^4x\,\left\{\alpha(x)\mc{A}(x)+J^\mu_5(x)\partial_\mu\alpha(x)\right\}+iI\right)-\int [d\psi][d\bar{\psi}]e^{iI} \\
=&i\int d^4x \int [d\psi][d\bar{\psi}][\mc{A}(x)\alpha(x)+J^\mu_5(x)\partial_\mu\alpha(x)]e^{iI}
\end{align*}
となる.(通常通りテイラー展開すれば導ける.)これは単に変数変換であるから,任意の$\alpha(x)$について経路積分に影響はない.したがってこの変化分が全微分であることが要請され,任意のゲージ場について($\mc{A}$は$\psi,\bar{\psi}$に依存しないので外に出せて)
\begin{align*}
&\int [d\psi][d\bar{\psi}]e^{iI}\partial_\mu J^\mu_5 =\mc{A}\int [d\psi][d\bar{\psi}]e^{iI}  \\
&\braket{\partial_\mu J^\mu_5}_A=\frac{\int[d\psi][d\bar{\psi}]e^{iI}\partial_\mu J^\mu_5}{\int[d\psi][d\bar{\psi}]e^{iI}}=\mc{A}=-\frac{1}{16\pi^2}\epsilon_{\mu\nu\rho\sigma}F^{\mu\nu}_\alpha F^{\rho\sigma}_\beta \mr{tr}\left\{t_\alpha t_\beta t \right\}
\end{align*}
ここで任意の演算子$\mc{O}$について,$\braket{\mc{O}}_A$は固定された背景場$A^\mu_\alpha(x)$のもとでの$\mc{O}$の量子平均
\begin{align*}
\braket{\mc{O}}_A=\frac{\int[d\psi][d\bar{\psi}]e^{iI}\mc{O}}{\int[d\psi][d\bar{\psi}]e^{iI}}
\end{align*}
だ.\par
表式(22.2.26)を保存条件に書き換えることもできる.ここで$\mr{tr}\{t_\alpha t_\beta t\}$が$\delta_{\alpha\beta}$に比例する特別の場合を考える.
\begin{align*}
\mr{tr}\{t_\alpha t_\beta t\}=N\delta_{\alpha\beta}
\end{align*}
つまりこの場合においては
\begin{align*}
\braket{\partial_\mu J^\mu_5}_A=-\frac{N}{16\pi^2}\epsilon_{\mu\nu\rho\sigma}F^{\mu\nu}_\alpha F^{\rho\sigma}_\alpha
\end{align*}
となる.ここでチャーン・サイモンズ類として知られるカレントを定義する.
\begin{align*}
G^\mu&\equiv 2\epsilon^{\mu\nu\lambda\rho}\left[A_{\gamma\nu}\partial_\lambda A_{\gamma\rho}+\frac{1}{3}C_{\alpha\beta\gamma}A_{\alpha\nu}A_{\beta\lambda}A_{\gamma\rho}\right] \\
&=\epsilon^{\mu\nu\lambda\rho}\left[A_{\gamma\nu}F_{\gamma\lambda\rho}-\frac{1}{3}C_{\alpha\beta\gamma}A_{\alpha\nu}A_{\beta\lambda}A_{\gamma\rho}\right]
\end{align*}
(二番目の等号は
\begin{align*}
G^\mu=&\epsilon^{\mu\nu\lambda\rho}\left[A_{\gamma\nu}F_{\gamma\lambda\rho}-\frac{1}{3}C_{\alpha\beta\gamma}A_{\alpha\nu}A_{\beta\lambda}A_{\gamma\rho}\right] \\
=&\epsilon^{\mu\nu\lambda\rho}\left[A_{\gamma\nu}\left( \partial_\lambda A_{\gamma\rho}-\partial_\rho A_{\gamma\lambda} +C_{\alpha\beta\gamma}A_{\alpha\lambda}A_{\beta\rho} \right)-\frac{1}{3}C_{\alpha\beta\gamma}A_{\alpha\nu}A_{\beta\lambda}A_{\gamma\rho}\right] \\
=&\epsilon^{\mu\nu\lambda\rho}\left[A_{\gamma\nu}\partial_\lambda A_{\gamma\rho}-A_{\gamma\nu}\partial_\rho A_{\gamma\lambda} +C_{\alpha\beta\gamma}A_{\alpha\lambda}A_{\beta\rho}A_{\gamma\nu}-\frac{1}{3}C_{\alpha\beta\gamma}A_{\alpha\nu}A_{\beta\lambda}A_{\gamma\rho}\right] \\
=&\epsilon^{\mu\nu\lambda\rho}\left[A_{\gamma\nu}\partial_\lambda A_{\gamma\rho}+A_{\gamma\nu}\partial_\lambda A_{\gamma\rho} +C_{\alpha\beta\gamma}A_{\alpha\nu}A_{\beta\lambda}A_{\gamma\rho}-\frac{1}{3}C_{\alpha\beta\gamma}A_{\alpha\nu}A_{\beta\lambda}A_{\gamma\rho}\right] \quad \because \epsilon^{\mu\nu\lambda\rho} の反対称性 \\
=& 2\epsilon^{\mu\nu\lambda\rho}\left[A_{\gamma\nu}\partial_\lambda A_{\gamma\rho}+\frac{1}{3}C_{\alpha\beta\gamma}A_{\alpha\nu}A_{\beta\lambda}A_{\gamma\rho}\right]
\end{align*}
として確認できる.)これを微分すると
\begin{align*}
\partial_\mu G^\mu=&2\epsilon^{\mu\nu\lambda\rho}\biggl[ (\partial_\mu A_{\gamma\nu} )(\partial_\lambda A_{\gamma\rho})+ A_{\gamma\nu}(\partial_\mu \partial_\lambda A_{\gamma\rho}) \\
&\quad +\frac{1}{3}C_{\alpha\beta\gamma}\left\{(\partial_\mu A_{\alpha\nu})A_{\beta\lambda}A_{\gamma\rho} +A_{\alpha\nu}(\partial_\mu A_{\beta\lambda})A_{\gamma\rho} +A_{\alpha\nu}A_{\beta\lambda}(\partial_\mu A_{\gamma\rho}) \right\} \biggr] \\
=&2\epsilon^{\mu\nu\lambda\rho}\biggl[ (\partial_\mu A_{\gamma\nu} )(\partial_\lambda A_{\gamma\rho}) \quad \because \epsilon^{\mu\nu\lambda\rho}の反対称性と微分の可換性 \\
&\quad +\frac{1}{3}C_{\alpha\beta\gamma}\left\{ (\partial_\mu A_{\alpha\nu})A_{\beta\lambda}A_{\gamma\rho} -(\partial_\mu A_{\alpha\lambda})A_{\beta\nu}A_{\gamma\rho} +(\partial_\mu A_{\alpha\rho})A_{\beta\lambda}A_{\gamma\nu}\right\}\biggr] \quad\because C_{\alpha\beta\gamma}の反対称性 \\
=&2\epsilon^{\mu\nu\lambda\rho}\biggl[ (\partial_\mu A_{\gamma\nu} )(\partial_\lambda A_{\gamma\rho}) \\
&\quad +\frac{1}{3}C_{\alpha\beta\gamma}\left\{ (\partial_\mu A_{\alpha\nu})A_{\beta\lambda}A_{\gamma\rho} +(\partial_\mu A_{\alpha\nu})A_{\beta\lambda}A_{\gamma\rho} +(\partial_\mu A_{\alpha\nu})A_{\beta\lambda}A_{\gamma\rho}\right\}\biggr] \quad\because \epsilon^{\mu\nu\lambda\rho}の反対称性 \\
=&2\epsilon^{\mu\nu\lambda\rho}\left[ (\partial_\mu A_{\alpha\nu} )(\partial_\lambda A_{\alpha\rho})+(\partial_\mu A_{\alpha\nu})C_{\alpha\beta\gamma}A_{\beta\lambda}A_{\gamma\rho}\right]
\end{align*}
となるが,一方
\begin{align*}
\epsilon^{\mu\nu\lambda\rho}F_{\alpha\mu\nu}F_{\alpha\lambda\rho}=&\epsilon^{\mu\nu\lambda\rho}(\partial_\mu A_{\alpha \nu}-\partial_\nu A_{\alpha \mu}+C_{\alpha\beta\gamma}A_{\beta\mu}A_{\gamma\nu})(\partial_\lambda A_{\alpha\rho}-\partial_\rho A_{\alpha\lambda}+C_{\alpha\beta\gamma}A_{\beta\lambda}A_{\gamma\rho}) \\
=&\epsilon^{\mu\nu\lambda\rho}(2\partial_\mu A_{\alpha\nu} +C_{\alpha\beta\gamma}A_{\beta\mu}A_{\gamma\nu})(2\partial_\lambda A_{\alpha\rho}+C_{\alpha\beta\gamma}A_{\beta\lambda}A_{\gamma\rho}) \quad \because \epsilon^{\mu\nu\lambda\rho}の反対称性 \\
=&\epsilon^{\mu\nu\lambda\rho}[4(\partial_\mu A_{\alpha\nu})(\partial_\lambda A_{\alpha\rho})+2(\partial_\mu A_{\alpha\nu})C_{\alpha\beta\gamma}A_{\beta\lambda}A_{\gamma\rho}+2(\partial_\lambda A_{\alpha\rho})C_{\alpha\beta\gamma}A_{\beta\mu}A_{\gamma\nu} \\
&+C_{\alpha\beta\gamma}C_{\alpha\delta\epsilon}A_{\beta\mu}A_{\gamma\nu}A_{\delta\lambda}A_{\epsilon\rho}] \\
=&\epsilon^{\mu\nu\lambda\rho}[4(\partial_\mu A_{\alpha\nu})(\partial_\lambda A_{\alpha\rho})+4(\partial_\mu A_{\alpha\nu})C_{\alpha\beta\gamma}A_{\beta\lambda}A_{\gamma\rho} ]
\end{align*}
となる.最後の等号ではヤコビの恒等式より
\begin{align*}
0=&\epsilon^{\mu\nu\lambda\rho}(C_{\alpha\beta\gamma}C_{\alpha\delta\epsilon}+C_{\alpha\delta\beta}C_{\alpha\gamma\epsilon}+C_{\alpha\gamma\delta}C_{\alpha\beta\epsilon})A_{\beta\mu}A_{\gamma\nu}A_{\delta\lambda}A_{\epsilon\rho} \\
=&\epsilon^{\mu\nu\lambda\rho}C_{\alpha\beta\gamma}C_{\alpha\delta\epsilon}A_{\beta\mu}A_{\gamma\nu}A_{\delta\lambda}A_{\epsilon\rho} \\
&+\epsilon^{\mu\nu\lambda\rho}C_{\alpha\delta\beta}C_{\alpha\gamma\epsilon}A_{\beta\mu}A_{\gamma\nu}A_{\delta\lambda}A_{\epsilon\rho}+\epsilon^{\mu\nu\lambda\rho}C_{\alpha\gamma\delta}C_{\alpha\beta\epsilon}A_{\beta\mu}A_{\gamma\nu}A_{\delta\lambda}A_{\epsilon\rho} \\
=&\epsilon^{\mu\nu\lambda\rho}C_{\alpha\beta\gamma}C_{\alpha\delta\epsilon}A_{\beta\mu}A_{\gamma\nu}A_{\delta\lambda}A_{\epsilon\rho}\\
&+\epsilon^{\mu\nu\lambda\rho}C_{\alpha\beta\gamma}C_{\alpha\delta\epsilon}A_{\gamma\mu}A_{\delta\nu}A_{\beta\lambda}A_{\epsilon\rho}+ \epsilon^{\mu\nu\lambda\rho}C_{\alpha\beta\gamma}C_{\alpha\delta\epsilon}A_{\delta\mu}A_{\beta\nu}A_{\gamma\lambda}A_{\epsilon\rho} \quad(\beta\gamma\delta 添え字の再定義)\\
=&\epsilon^{\mu\nu\lambda\rho}C_{\alpha\beta\gamma}C_{\alpha\delta\epsilon}A_{\beta\mu}A_{\gamma\nu}A_{\delta\lambda}A_{\epsilon\rho}\\
&+\epsilon^{\mu\nu\lambda\rho}C_{\alpha\beta\gamma}C_{\alpha\delta\epsilon}A_{\beta\lambda}A_{\gamma\mu}A_{\delta\nu}A_{\epsilon\rho}+ \epsilon^{\mu\nu\lambda\rho}C_{\alpha\beta\gamma}C_{\alpha\delta\epsilon}A_{\beta\nu}A_{\gamma\lambda}A_{\delta\mu}A_{\epsilon\rho}\quad (Aの並び替え) \\
=&\epsilon^{\mu\nu\lambda\rho}C_{\alpha\beta\gamma}C_{\alpha\delta\epsilon}A_{\beta\mu}A_{\gamma\nu}A_{\delta\lambda}A_{\epsilon\rho} \\
&+\epsilon^{\mu\nu\lambda\rho}C_{\alpha\beta\gamma}C_{\alpha\delta\epsilon}A_{\beta\mu}A_{\gamma\nu}A_{\delta\lambda}A_{\epsilon\rho}+\epsilon^{\mu\nu\lambda\rho}C_{\alpha\beta\gamma}C_{\alpha\delta\epsilon}A_{\beta\mu}A_{\gamma\nu}A_{\delta\lambda}A_{\epsilon\rho} \quad (\mu\nu\lambda 添え字の再定義) \\
=&3\epsilon^{\mu\nu\lambda\rho}C_{\alpha\beta\gamma}C_{\alpha\delta\epsilon}A_{\beta\mu}A_{\gamma\nu}A_{\delta\lambda}A_{\epsilon\rho}
\end{align*}
であるから最後の項がゼロとなることを用いた.したがって
\begin{align*}
\partial_\mu G^\mu=\frac{1}{2}\epsilon^{\mu\nu\lambda\rho}F_{\alpha\mu\nu}F_{\alpha\lambda\rho}
\end{align*}
となることがわかる.したがって(22.2.26)は
\begin{align*}
\braket{\partial_\mu J^\mu_5}_A&=-\frac{N}{8\pi^2}\partial_\mu G^\mu \\
\partial_\mu \left[\braket{J^\mu_5}_A +\frac{N}{8\pi^2}G^\mu\right]&=0 \\
\partial_\mu K^\mu &=0
\end{align*}
という保存条件に書き換えることができる.しかし,カレント$K^\mu$の保存則を用いて$\pi^0\to2\gamma$過程が抑えられると論じることはできない.$K^\mu$は保存されているが,(22.2.29)からわかるようにゲージ不変でないからだ.\par
アノマリーの表式(22.2.24)の導出を見れば,(22.2.13)において$f(-\Slash{D}^2_x/M^2)$の代わりに微分演算子$f(-\Slash{\partial}^2_x/M^2)$を使って計算すれば
\begin{align*}
\mc{A}(x)=&-2\int \frac{d^4 k}{(2\pi)^4}\left[\mr{Tr}\left\{\gamma_5 tf(-\Slash{\partial}^2_x/M^2) \right\}e^{ik(x-y)}\right]_{y\to x} \\
=&-2\int \frac{d^4 k}{(2\pi)^4}\mr{Tr}\left\{\gamma_5 tf(-(i\Slash{k})^2/M^2) \right\}\\
=&-2\int \frac{d^4 k}{(2\pi)^4}\mr{Tr}\left\{\gamma_5 tf(+k^2/M^2) \right\}=-2\int \frac{d^4 k}{(2\pi)^4}f(+k^2/M^2)\mr{Tr}\left\{\gamma_5 t \right\} \\
=&0
\end{align*}
となってアノマリー関数がゼロになることがわかる.このやり方の問題点は,前に述べた通り,正則化演算子がゲージ不変でなく,そのあめに$K^\mu$にゲージ不変でない項が生じることだ.フェルミオン・プロパゲータと行列式の正則化において,ゲージ不変かつカイラル不変な方法は存在しない!

\vskip\baselineskip

さて,アノマリーの話の原点となった問題に戻り,上の結果を使い$\pi^0\to 2\gamma$過程の実際の反応率を計算しよう.興味のある対称性は狩りクォーク場の中性カイラル変換(22.1.5)によって生成される.
\begin{align*}
&\left(
\begin{array}{cc}
u\\
d
\end{array}
\right)\to \exp(i\alpha\gamma_5\tau_3)\left(
\begin{array}{cc}
u\\
d
\end{array}
\right) \\
&\delta \left(
\begin{array}{cc}
u\\
d
\end{array}
\right)=i\alpha \gamma_5 \tau_3 \left(
\begin{array}{cc}
u\\
d
\end{array}
\right)=i\alpha \gamma_5 \left(
\begin{array}{cc}
1 & 0 \\
0 & -1
\end{array}
\right)\left(
\begin{array}{cc}
u\\
d
\end{array}
\right) \\
&\delta u=i\alpha \gamma_5 u ,\quad \delta d=-i\alpha \gamma_5 d
\end{align*}
純粋な量子色力学(クォークとグルーオンの相互作用のみについての理論)ではこの対称性はアノマリーを持たない.なぜなら$u,d$が色ゲージ群の同じ表現に属し,(22.2.33)の対称性のアノマリーへのグルーオン・グルーオン項へのそれらの寄与は相殺するからだ.つまり$t_\alpha$をゲージ群の生成子,$G^{\mu\nu}_\alpha$をグルーオン場のテンソルとして
\begin{align*}
\mc{A}(x)&=\mc{A}_u(x)+\mc{A}_d(x) \\
&=-\frac{1}{16\pi^2}\epsilon_{\mu\nu\rho\sigma}G^{\mu\nu}_\alpha G^{\rho\sigma}_\beta \mr{tr}\left\{t_\alpha t_\beta (+1)\right\}-\frac{1}{16\pi^2}\epsilon_{\mu\nu\rho\sigma}G^{\mu\nu}_\alpha G^{\rho\sigma}_\beta\mr{tr}\left\{t_\alpha t_\beta (-1)\right\} \\
&=0
\end{align*}
となって相殺する.($t_\alpha$は)一方,電磁場$A^\mu(x)$が存在すれば,この対称性はアノマリー
\begin{align*}
\mc{A}(x)=-\frac{1}{16\pi^2}\epsilon_{\mu\nu\rho\sigma}F^{\mu\nu} F^{\rho\sigma} \mr{tr}\left\{q^2 \tau_3\right\}
\end{align*}
をもつ.ここで$q$はクォークの電荷行列だ.もし,通常のように$N_c$個の電荷$2e/3$をもつ$u$クォークと,同じ個数の電荷$-e/3$をもつ$d$クォークがあるとすれば(つまりカラーの数が$N_c$個あるとすれば),トレースは固有値の和であるから
\begin{align*}
\mr{tr}\left\{q^2 \tau_3 \right\}=N_c \left(\frac{2e}{3}\right)(+1)+N_c\left(\frac{-e}{3}\right)(-1)=\frac{N_c e^2}{3}
\end{align*}
となる.したがってアノマリー関数は
\begin{align*}
\mc{A}(x)=-\frac{N_ce^2}{48\pi^2}\epsilon_{\mu\nu\rho\sigma}F^{\mu\nu}(x)F^{\rho\sigma}(x)
\end{align*}
だ.そのため有効ラグランジアンにはカイラル変換(22.2.33)のもとで(22.2.12)を与えるような項を含まなければならない.それは以下で与えられる.
\begin{align*}
\delta \mc{L}_{eff}(x)=\alpha \mc{A}(x)=-\frac{N_ce^2}{48\pi^2}\epsilon_{\mu\nu\rho\sigma}F^{\mu\nu}(x)F^{\rho\sigma}(x)\alpha
\end{align*}
さて,変換(22.2.33)の対称性がアノマリーによって完全に破れることで,(19.7節参照)$\tau_3$に結合するNGボゾンである$\pi^0$場は
\begin{align*}
g\gamma(\xi)&=\gamma(\xi')h(\xi,g) \\
\exp[i\gamma_5\tau_3\alpha]\exp[i\gamma_5 \tau_3 \xi_3(x)]&=\exp[i\gamma_5 \tau_3 \xi'_3(x)] \\
\xi'_3&=\xi_3+\alpha \\
{\pi^0}'&=\pi^0+\alpha F_\pi \quad \because \pi^0=F_\pi \xi_3 \\
\delta \pi^0&=\alpha F_\pi
\end{align*}
と変換する.ここで$F_\pi=184\mr{MeV}$は19章で導入した定数だ.これにより,有効ラグランジアンは以下の項を含まなければならない.
\begin{align*}
\frac{\pi^0\mc{A}(x)}{F_\pi}=-\frac{N_ce^2}{48\pi^2F_\pi}\epsilon_{\mu\nu\rho\sigma}F^{\mu\nu}(x)F^{\rho\sigma}(x)\pi^0(x)
\end{align*}
これは変換(22.2.33)のもとで不変であることがすぐに分かる.これを$\pi^0\to2\gamma$崩壊の有効ラグランジアンの一般的な表式(22.1.1)と比較すると,(22.1.1)の定数$g$は
\begin{align*}
g=\frac{N_ce^2}{48\pi^2F_\pi}
\end{align*}
という値をとらなければならないことがわかる.こうして,(22.1.2)のパイ中間子崩壊率は
\begin{align*}
\Gamma(\pi^0\to2\gamma)=\frac{N_c^2\alpha^2 m^3_\pi}{144\pi^3 F^2_\pi}=\left(\frac{N_c}{3}\right)^2\times 1.11\times 10^{16}\,\mr{s}^{-1}
\end{align*}
と予言される.観測された値は$\Gamma(\pi^0\to 2\gamma)=(1.19\pm 0.08)\times 10^{16}\,\mr{s}^{-1}$であり,これは$N_c=3$のときに限って理論値(22.2.39)とよく一致する.この計算の成功は,クォークに3つの色があるという証拠のもっとも初期のものの一つだ.

\vskip\baselineskip

アノマリーのより厳密な導出は,ウィック回転したユークリッド時空での経路積分を使って得られる.ユークリッド第4座標を$x_4=ix^0=-ix_0$で導入する.そしてそれに対応して,$\partial_4=\partial/\partial x_4=-i\partial/\partial x^0=-i\partial_0, \gamma_4\equiv i\gamma^0 , A_{\alpha 4}=iA_{\alpha}^0$とする.($x_4,A_{\alpha 4}$は実であるようにウィック回転していることに注意.また$\gamma_4^\dagger=\gamma_4$かつ$\gamma_i^\dagger=\gamma_i$よりガンマ行列はエルミートとなる.)また時空の微小体積は$(d^4x)_E$をユークリッド体積要素$(d^4x)_E=dx_1 dx_2 dx_3 dx_4$として$d^4x=-i(d^4x)_E$となる.ユークリッド時空では場$\psi(x)$とそのディラック共役場$\bar{\psi}(x)$は完全に独立として扱わなければならない.それらの局所カイラル変換は,これらが独立でないときに
\begin{align*}
\delta\psi(x)&=i\alpha(x)t \gamma_5 \psi(x) \\
\delta\bar{\psi}(x)&=[\delta \psi(x)]^\dagger \gamma_4=[i\alpha(x)t \gamma_5 \psi(x)]\gamma_4 \\
&=-i\alpha(x)\psi(x)^\dagger t\gamma_5^\dagger \gamma_4 \\
&=i\alpha(x)\bar{\psi}(x)t\gamma_5 \quad \because (5.4.33)
\end{align*}
と変換されるのだったから,$\delta\psi(x)=i\alpha(x)t \gamma_5 \psi(x),\delta\bar{\psi}(x)=i\alpha(x)\bar{\psi}(x)t\gamma_5$で定義される.そうすると,測度の変換は,アノマリー関数$\mc{A}(x)$を(22.2.11)として,前と同じく(22.2.10)で与えられる.
\begin{align*}
&[d\psi][d\bar\psi]\to\exp\left\{\int (d^4x)_E \alpha(x)\mc{A}(x)\right\}[d\psi][d\bar\psi] \\
&\mc{A}(x)=-2\mr{Tr}\left\{\gamma_5 t\right\}\delta^4(x-x)
\end{align*}
前と同様に正則化関数を導入すると,これは$\mc{A}(x)$について(22.2.13)と同様の表式を与える.
\begin{align*}
\mc{A}(x)=-2\lim_{M\to \infty}\left[\mr{Tr}\left\{\gamma_5 t f(-\Slash{D}^2/M^2)\right\}\delta^4(x-y)\right]_{y\to x}
\end{align*}
ユークリッド化した取り扱いの非常な有利な点は,$x_4$と$A_{\alpha 4}$が実としているために,表式(22.2.13)のディラック演算子$i\Slash{D}$がエルミートなことだ.
\begin{align*}
i\Slash{D}=&[i\partial_\mu+t_\alpha A_{\mu\alpha}]\gamma^\mu \\
=&[i\partial_0 +t_\alpha A_{0\alpha}]\gamma^0+[i\partial_i +t_\alpha A_{i \alpha}]\gamma^i \quad (i=1\sim 3 )\\
=&[-\partial_4+it_\alpha A_{4\alpha}](-i\gamma_4)+[i\partial_i +t_\alpha A_{i \alpha}]\gamma_i \quad (i=1\sim 3) \\
=&[i\partial_i +t_\alpha A_{i \alpha}]\gamma_i \quad (i=1\sim 4)\\
(i\Slash{D})^\dagger=&[(i\partial_i)^\dagger +t_\alpha^\dagger A_{i\alpha}^\dagger]\gamma^\dagger_i \\
=&[i\partial_i +t_\alpha A_{i\alpha}]\gamma_i \quad \because \gamma_i^\dagger =\gamma_i ,(i\partial_i)^\dagger =i\partial_i \\
=& i\Slash{D}
\end{align*}
ここで$i\partial_i$がエルミートであることは,運動量演算子であることを思い出せば理解できる.あるいは部分積分より
\begin{align*}
\int d^4x \psi^\dagger(x) \left(\frac{\partial}{\partial x_i} \psi(x)\right)=\int d^4x \left(-\frac{\partial}{\partial x_i} \psi(x)\right)^\dagger \psi(x)
\end{align*}
で,$\partial_i$のエルミート共役は$-\partial_i$だとわかるから,$i\partial_i$はエルミートだとわかる.したがって,ディラック演算子$i\Slash{D}$は正規直交スピノル固有関数$\varphi_\kappa(x)$をもつ.
\begin{align*}
&i\Slash{D}\varphi_\kappa=\lambda_\kappa \varphi_\kappa \\
&\int (d^4x)_E \varphi_\kappa(x)^\dagger \varphi_{\kappa'}(x)=\delta_{\kappa\kappa'}
\end{align*}
エルミート演算子の固有値は実であるから,$\lambda_\kappa$は実だ.また,この節でずっとそうしているように$t$は$i\Slash{D}$と可換だとしているから,
\begin{align*}
i\Slash{D}(t\varphi_\kappa)=t(i\Slash{D}\varphi_\kappa)=\lambda_\kappa (t\varphi_\kappa)
\end{align*}
となって$t\varphi_\kappa$も固有値$\lambda_\kappa$をもつ固有ベクトルであり$t\varphi_\kappa\propto \varphi_\kappa$とできる.この比例定数を$t_\kappa$とすれば$t\varphi_\kappa=t_\kappa \varphi_\kappa$を満たすように$\varphi_\kappa$を選ぶことができる.これらの固有関数は完全性条件
\begin{align*}
\sum_\kappa \varphi_\kappa(x)\varphi_\kappa^\dagger(y)=\sum_\kappa \braket{x|\varphi_\kappa}\braket{\varphi_\kappa|y}=\delta^4(x-y)1
\end{align*}
を満たす.ここで「1」は$4\times 4$単位行列だ.したがって,アノマリー関数はいまや明白に収束する和の極限として
\begin{align*}
\mc{A}(x)=&-2\lim_{M\to \infty}\mr{Tr}\left\{\gamma_5 t f(-\Slash{D}^2/M^2)\right\}\delta^4(x-x) \\
=&-2\lim_{M\to\infty}\mr{Tr}\left\{ \gamma_5 tf(-\Slash{D}^2/M^2)\sum_\kappa\varphi_\kappa(x)\varphi_\kappa^\dagger(x) \right\} \\
=&-2\lim_{M\to\infty}\sum_\kappa  t_\kappa f(\lambda_\kappa^2/M^2) \mr{Tr}\left\{ \gamma_5 \varphi_\kappa(x)\varphi_\kappa^\dagger(x) \right\} \\
=&-2\lim_{M\to\infty}\sum_\kappa  t_\kappa f(\lambda_\kappa^2/M^2) \left( \varphi_\kappa^\dagger(x) \gamma_5 \varphi_\kappa(x) \right)
\end{align*}
と書くことができる.\par
アノマリー関数について(22.2.24)の表式を導いたのと同様に再び導く.
\begin{align*}
\mc{A}(x)=&-2\lim_{M\to \infty}\left[\mr{Tr}\left\{\gamma_5 t f(-\Slash{D}^2/M^2)\right\}\delta^4(x-y)\right]_{y\to x} \\
=&-2\lim_{M\to \infty}\int \frac{(d^4 k)_E}{(2\pi)^4}\left[\mr{Tr}\left\{\gamma_5 t f(-\Slash{D}^2/M^2)\right\}e^{ik(x-y)}\right]_{y\to x} \quad (指数の肩の内積はユークリッド内積) \\
=&-2\lim_{M\to \infty}\int \frac{(d^4 k)_E}{(2\pi)^4}\mr{Tr}\left\{\gamma_5 t f(-[i\Slash{k} +\Slash{D}]^2/M^2)\right\} \\
=&-2\lim_{M\to \infty}M^4 \int \frac{(d^4 k)_E}{(2\pi)^4}\mr{Tr}\left\{\gamma_5 t f(-[i\Slash{k} +\Slash{D}/M]^2)\right\} \\
=&-2\int \frac{(d^4 k)_E}{(2\pi)^4}f''(k^2)\mr{Tr}\left\{\gamma_5 t \Slash{D}^4\right\} \quad \because (22.2.18)
\end{align*}
ここで$k$は既にユークリッド座標であるから,ウィック回転を介さずに(22.2.20)の計算を行うことができる.
\begin{align*}
\int (d^4 k)_Ef''(k^2)=\int^\infty_0 2\pi^2 \kappa^3 d\kappa f''(\kappa^2)
\end{align*}
(ウィック回転を介していないために虚数が表れていないことに注意せよ.)したがって(22.2.21)は
\begin{align*}
\int (d^4k)_E f''(k^2)=\pi^2
\end{align*}
となる.トレースを計算するには(22.2.22)と同様に
\begin{align*}
\Slash{D}^2=D^2-\frac{1}{4}it_\alpha F^{ij}_\alpha[\gamma_i,\gamma_j]
\end{align*}
と書き
\begin{align*}
\mr{tr_D}\left\{\gamma_5 [\gamma_i,\gamma_j][\gamma_k,\gamma_\ell]\right\}=16\epsilon^E_{ijk\ell}
\end{align*}
を用いると((22.2.23)において$\epsilon_{\mu\nu\rho\sigma}$により$\mu\sim\sigma$のどれのガンマ行列は必ずゼロ成分であるから,$\gamma_4=i\gamma^0=-i\gamma_0$を用いて全体に$-i$をかければこの等式が得られる.)
\begin{align*}
\mc{A}(x)=&-2\frac{\pi^2}{(2\pi)^4}\left(-\frac{i}{4}\right)^216\epsilon^E_{ijk\ell}F_{ij\alpha}F_{k\ell \beta}\mr{tr}\left\{t_\alpha t_\beta t\right\} \\
=&\frac{1}{16\pi^2}\epsilon^E_{ijk\ell}F_{ij\alpha}F_{k\ell \beta}\mr{tr}\left\{t_\alpha t_\beta t\right\}
\end{align*}
を示すことができる.(ユークリッド化による影響で,符号が(22.2.24)とは異なることに注意せよ.)\par
さて,固有値が$\lambda_\kappa\neq 0$である$i\Slash{D}$と$t$の任意の固有関数$\varphi_\kappa(x)$が与えられたとき,$\varphi_{\kappa^-}(x)\equiv \gamma_5 \varphi_\kappa(x)$で与えられる別の規格化された固有関数$\varphi_{\kappa^-}(x)$があって,$i\Slash{D}$の固有値は
\begin{align*}
i\Slash{D}\varphi_{\kappa^-}(x)&=i\Slash{D}\gamma_5\varphi_\kappa(x) \\
&=-\gamma_5 i\Slash{D}\varphi_\kappa(x) \\
&=-\lambda_\kappa \gamma_5 \varphi_\kappa(x)=-\lambda_\kappa \varphi_{\kappa^-}(x)
\end{align*}
であるから$-\lambda_\kappa$となり,$t$の固有値は相変わらず$t_\kappa$だ.さらに
\begin{align*}
\left(\varphi_\kappa^\dagger(x)\varphi_\kappa(x)\right)=&\left(\varphi_\kappa^\dagger(x)\gamma_5 \gamma_5\varphi_\kappa(x)\right) \\
=&\left((\gamma_5\varphi_\kappa(x))^\dagger \gamma_5 \varphi_\kappa(x)\right) \\
=&\left(\varphi_{\kappa^-}^\dagger(x)\varphi_{\kappa^-}(x)\right)
\end{align*}
だから,$\varphi_\kappa(x)$は規格直交基底であって異なる固有値に属するものは直交するので$\lambda_\kappa\neq 0$のとき
\begin{align*}
\left( \varphi_\kappa^\dagger(x) \gamma_5 \varphi_\kappa(x) \right)=\left( \varphi_\kappa^\dagger(x) \varphi_{\kappa^-}(x) \right)=0
\end{align*}
となる.したがって(22.2.44)の和において$\lambda_\kappa=0$の固有関数についての和のみが残る.これらは一般に対になっていない.むしろ,$\gamma_5$は$i\Slash{D}$と反可換だから,$i\Slash{D}$の固有値がゼロで\uwave{同時に}それぞれ$\gamma_5$の固有値(カイラリティ)$+1$と$-1$をもつ同時正規直交固有関数$\varphi_u$と$\varphi_v$
\begin{align*}
&i\Slash{D}\varphi_u=0 ,\quad \gamma_5 \varphi_u=+\varphi_u \\
&i\Slash{D}\varphi_u=0 ,\quad \gamma_5 \varphi_v=-\varphi_v
\end{align*}
に選ぶことができる.$M\to \infty$を実行して(22.2.15)$f(0)=1$を用いると
\begin{align*}
\mc{A}(x)=&-2\left[\sum_u t_u \left( \varphi_u^\dagger(x) \gamma_5 \varphi_u(x) \right)+\sum_v t_v \left( \varphi_v^\dagger(x) \gamma_5 \varphi_v(x) \right) \right] \\
=&-2\left[\sum_u t_u \left( \varphi_u^\dagger(x) \varphi_u(x) \right)-\sum_v t_v \left( \varphi_v^\dagger(x) \varphi_v(x) \right) \right]
\end{align*}
となる.さて,$\varphi_u$と$\varphi_v$は(22.2.42)のように規格化されているから,(22.2.47)の積分は以下を与える.
\begin{align*}
\int (d^4x)_E \mc{A}(x)=-2\left[\sum_u t_u -\sum_v t_v \right]
\end{align*}
ここで$u$と$v$についての和はそれぞれ,演算子$i\Slash{D}$の左手と右手成分のゼロ・モードについてとる.特に,$t$が単位行列の場合は,(22.2.45)を使って,これをゲージ場の汎関数と,ディラック演算子のカイラリティが決まったゼロ・モードの数との関係式として表すことができる!
\begin{align*}
-\frac{1}{32\pi^2}\int (d^4x)_E \epsilon^E_{ijk\ell}F_{ij\alpha}F_{k\ell \beta}\mr{tr}\left[t_\alpha t_\beta\right]=n_+-n_-
\end{align*}
ここで$n_\pm$は$i\Slash{D}$のゼロ・モードで,$\gamma_5$の固有値$\pm 1$をもつものの数だ.($n_++n_-=n_0$で,$n_0$がゼロ・モードの全体の数だ.)これが有名なアティヤ・シンガーの指数定理だ.(中原幹夫著「理論物理学のための幾何学とトポロジーII」参照)それは色々なことを示しているが,その中でも特に,ゲージ場の変化のもとで(22.2.49)の左辺の積分が滑らかに変化することができないことを示している!特に,変化分は整数に限られているので,この積分はゲージ場のトポロジーにのみ依存できる.

\newpage

\subsection{一般的な場合のアノマリーの直接計算}
一般のアノマリーを扱うには,全ての左手フェルミオン場を(区別がある場合には,反フェルミオン場も)一つの列ベクトル$\chi$に統一して記す.たとえば,もし$\psi$が全てのクォークとレプトンを含む(反クォークと反レプトンは含まない)列ベクトルとすると,
\begin{align*}
\chi \equiv \left[
\begin{array}{cc}
\frac{1}{2}(1+\gamma_5)\psi \\
\frac{1}{2}[\mc{C}(1-\gamma_5)\psi]^*
\end{array}
\right]= \left[
\begin{array}{cc}
\frac{1}{2}(1+\gamma_5)\psi \\
\frac{1}{2}(1+\gamma_5)\mc{C}\psi^*
\end{array}
\right] \quad\because \mc{C}^*=\mc{C},\gamma_5^*=\gamma_5, \mc{C}\gamma_5 =-\gamma_5 \mc{C}
\end{align*}
ここで,$\mc{C}$は
\begin{align*}
\mc{C}\gamma^T_\mu \mc{C}^{-1}=-\gamma_\mu ,\gamma ,\quad \mc{C}=\gamma_2\gamma_4 
\end{align*}
で定義された行列だ.(すなわち,(5.5.47)より,下成分は右手フェルミオンの荷電共役場を表しており,これで全ての左手フェルミオンを統一して記すことができている.)フェルミオン数(バリオン数か,バリオン数引くレプトン数)を保存する微小ゲージ変換
\begin{align*}
\delta\psi=i\theta_\alpha\left[\frac{1}{2}(1+\gamma_5)t^L_\alpha +\frac{1}{2}(1-\gamma_5)t^R_\alpha \right]\psi
\end{align*}
のもとで,この列ベクトルは
\begin{align*}
\delta\chi=&\left[
\begin{array}{cc}
\frac{1}{2}(1+\gamma_5)\delta\psi \\
\frac{1}{2}[\mc{C}(1-\gamma_5)\delta\psi]^*
\end{array}
\right] \\
=&\left[
\begin{array}{cc}
\frac{1}{2}(1+\gamma_5)i\theta_\alpha t^L_\alpha \psi \\
\frac{1}{2}[\mc{C}(1-\gamma_5)i\theta_\alpha t^R_\alpha \psi]^*
\end{array}
\right] \\
=&\left[
\begin{array}{cc}
i\theta_\alpha t^L_\alpha\frac{1}{2}(1+\gamma_5)\psi \\
-i\theta_\alpha t^{R*}_\alpha \frac{1}{2}[\mc{C}(1-\gamma_5)\psi]^*
\end{array}
\right] \\
=&i\theta_\alpha \left[
\begin{array}{cc}
t^L_\alpha & 0 \\
0 & -t^{R*}_\alpha
\end{array}
\right]\left[
\begin{array}{cc}
\frac{1}{2}(1+\gamma_5)\delta\psi \\
\frac{1}{2}[\mc{C}(1-\gamma_5)\delta\psi]^*
\end{array}
\right]=i\theta_\alpha T_\alpha \chi
\end{align*}
という変換を受ける.ここで
\begin{align*}
T_\alpha =\left[
\begin{array}{cc}
t^L_\alpha & 0 \\
0 & -t^{R*}_\alpha
\end{array}
\right]=\left[
\begin{array}{cc}
t^L_\alpha & 0 \\
0 & -(t^{R}_\alpha)^T
\end{array}
\right] \quad \because t^R_\alpha はエルミート
\end{align*}
だ.ここではフェルミオン数を保存する理論に話を限らない.そのため,$T_\alpha$はゲージ代数の任意のエルミート表現であり,必ずしも(22.3.4)のブロック対角形をしていない.まず質量ゼロのフェルミオンのみを考察し,後でフェルミオンの質量の影響を論じる.

\vskip\baselineskip

ここで問題となるのは1ループ3点関数
\begin{align*}
\Gamma^{\mu\nu\rho}_{\alpha\beta\gamma}(x,y,z)\equiv \braket{T\left\{ j^\mu_\alpha (x),j^\nu_\beta(y),j^\rho_\gamma(z) \right\} }_{\VAC}
\end{align*}
だ.ここで$j^\mu_\alpha$はフェルミオン的カレントで自由場
\begin{align*}
j^\mu_\alpha =-i\bar{\chi}T_\alpha \gamma^\mu \chi
\end{align*}
を使って計算される.

\begin{figure}[H]
\centering
\begin{tikzpicture}[decoration={markings, 
mark= at position -1cm with {\arrow[line width=0.5mm]{Stealth}}}, scale=0.5]

\coordinate (a1) at (-4,0){};
\coordinate (a2) at (-2,0){};
\coordinate (b1) at ({2},{2*sqrt(3)}){};
\coordinate (c1) at ({2},{-2*sqrt(3)}){};
\coordinate (b2) at ({1},{sqrt(3)}){};
\coordinate (c2) at ({1},{-sqrt(3)}){};

\draw[thick,postaction={decorate}](b2)--(a2);
\draw[thick,postaction={decorate}](c2)--(b2);
\draw[thick,postaction={decorate}](a2)--(c2);

\begin{feynhand}
\propag[photon,thick](a2)--(a1);
\propag[photon,thick](b2)--(b1);
\propag[photon,thick](c2)--(c1);
\end{feynhand}

\draw(a1)node[above right]{$j^\mu_\alpha(x)$};
\draw(b1)node[right]{$j^\nu_\beta(y)$};
\draw(c1)node[right]{$j^\rho_\gamma(z)$};

\end{tikzpicture}
\begin{tikzpicture}[decoration={markings, 
mark= at position -1cm with {\arrow[line width=0.5mm]{Stealth}}}, scale=0.5]
\coordinate (a1) at (-4,0){};
\coordinate (a2) at (-2,0){};
\coordinate (b1) at ({2},{2*sqrt(3)}){};
\coordinate (c1) at ({2},{-2*sqrt(3)}){};
\coordinate (b2) at ({1},{sqrt(3)}){};
\coordinate (c2) at ({1},{-sqrt(3)}){};

\draw[thick,postaction={decorate}](b2)--(a2);
\draw[thick,postaction={decorate}](c2)--(b2);
\draw[thick,postaction={decorate}](a2)--(c2);

\begin{feynhand}
\propag[photon,thick](a2)--(a1);
\propag[photon,thick](b2)--(b1);
\propag[photon,thick](c2)--(c1);
\end{feynhand}

\draw(a1)node[above right]{$j^\mu_\alpha(x)$};
\draw(b1)node[right]{$j^\rho_\gamma(z)$};
\draw(c1)node[right]{$j^\nu_\beta(y)$};

\end{tikzpicture}
\end{figure}

\noindent 図の二つのファインマン・ダイアグラムは以下を与える.
\begin{align*}
\Gamma^{\mu\nu\rho}_{\alpha\beta\gamma}(x,y,z)=&-(-i)^3\mr{Tr}[S(x-y)T_\beta \gamma^\nu P_L S(y-z)T_\gamma \gamma^\rho P_L S(z-x)T_\alpha \gamma^\mu P_L] \\
&-(-i)^3\mr{Tr}[S(x-z)T_\gamma \gamma^\rho P_L S(z-y)T_\beta \gamma^\nu P_L S(z-x)T_\alpha \gamma^\mu P_L] \\
=&-i\mr{Tr}[S(x-y)T_\beta \gamma^\nu P_L S(y-z)T_\gamma \gamma^\rho P_L S(z-x)T_\alpha \gamma^\mu P_L] \\
&-i\mr{Tr}[S(x-z)T_\gamma \gamma^\rho P_L S(z-y)T_\beta \gamma^\nu P_L S(z-x)T_\alpha \gamma^\mu P_L]
\end{align*}
$(-i)$の因子はカレントから生じる.マイナスはフェルミオンループの存在から生じる.(1巻368pを参照.)ここで$P_L$は左手フェルミオン場への射影演算子
\begin{align*}
P_L=\left(\frac{1+\gamma_5}{2}\right)
\end{align*}
で,また$S(x-y)$は質量ゼロのフェルミオン場のプロパゲータ
\begin{align*}
S(x-y)=\frac{-i}{(2\pi)^4}\int d^4p\,\left(\frac{-i\Slash{p}}{p^2-i\epsilon}\right)e^{ip(x-y)}
\end{align*}
だ.(これは$\braket{T\left\{\chi(x),\bar{\chi}(y)\right\}}$ではなく,22.3節の最後の議論における$\braket{T\left\{\Psi(x),\bar{\Psi}(y)\right\}}$であることに注意.)因子を集めると,(22.3.7)は
\begin{align*}
&\Gamma^{\mu\nu\rho}_{\alpha\beta\gamma}(x,y,z)= \\
&-i\mr{Tr}\biggl[\frac{-i}{(2\pi)^4}\int d^4 q_1 \left(\frac{-i\Slash{q}_1}{q^2_1-i\epsilon} \right)e^{iq_1\cdot (x-y)}T_\beta \gamma^\nu \frac{1+\gamma_5}{2} \\
&\qquad \times \frac{-i}{(2\pi)^4}\int d^4 q_2 \left(\frac{-i\Slash{q}_2}{q^2_2-i\epsilon} \right)e^{iq_2\cdot (y-z)}T_\gamma \gamma^\rho \frac{1+\gamma_5}{2} \\
&\qquad \times \frac{-i}{(2\pi)^4}\int d^4 q_3 \left(\frac{-i\Slash{q}_3}{q^2_3-i\epsilon} \right)e^{iq_3\cdot (z-x)}T_\alpha \gamma^\mu \frac{1+\gamma_5}{2} \biggr] \\
&-i\mr{Tr}\biggl[\frac{-i}{(2\pi)^4}\int d^4 q_1 \left(\frac{-i\Slash{q}_1}{q^2_1-i\epsilon} \right)e^{iq_1\cdot (x-z)}T_\beta \gamma^\nu \frac{1+\gamma_5}{2} \\
&\qquad \times \frac{-i}{(2\pi)^4}\int d^4 q_2 \left(\frac{-i\Slash{q}_2}{q^2_2-i\epsilon} \right)e^{iq_2\cdot (z-y)}T_\gamma \gamma^\rho \frac{1+\gamma_5}{2} \\
&\qquad \times \frac{-i}{(2\pi)^4}\int d^4 q_3 \left(\frac{-i\Slash{q}_3}{q^2_3-i\epsilon} \right)e^{iq_3\cdot (y-x)}T_\alpha \gamma^\mu \frac{1+\gamma_5}{2} \biggr] \\
=&\frac{i}{(2\pi)^{12}}\biggl\{\int d^4 q_1 d^4 q_2 d^4 q_3 e^{i(q_1-q_3)\cdot x}e^{i(q_2-q_1)\cdot y}e^{i(q_3-q_2)\cdot z} \\
& \qquad \times \mr{tr}\left[\frac{\Slash{q}_1}{q^2_1-i\epsilon} \gamma^\nu \frac{\Slash{q}_2}{q^2_2-i\epsilon}\gamma^\rho\frac{\Slash{q}_3}{q^2_3-i\epsilon}\gamma^\mu \frac{1+\gamma_5}{2}\right]\mr{tr}[T_\beta T_\gamma T_\alpha ] \\
&+ \int d^4 q_1 d^4 q_2 d^4 q_3e^{i(q_1-q_3)\cdot x}e^{i(q_2-q_1)\cdot y}e^{i(q_3-q_2)\cdot z} \\
& \qquad \times \mr{tr}\left[\frac{\Slash{q}_1}{q^2_1-i\epsilon} \gamma^\rho \frac{\Slash{q}_2}{q^2_2-i\epsilon}\gamma^\nu \frac{\Slash{q}_3}{q^2_3-i\epsilon}\gamma^\mu \frac{1+\gamma_5}{2}\right]\mr{tr}[T_\gamma T_\beta T_\alpha ]\biggr\}
\end{align*}
ここで,第一項目では変数変換
\begin{align*}
q_1^\mu \to k_1^\mu=p^\mu-q_1^\mu+a^\mu,\quad q_2^\mu \to p^\mu=q_2^\mu-a^\mu ,\quad q^\mu_3\to k_2=-p^\mu+q_3^\mu-a^\mu
\end{align*}
を,第二項目では
\begin{align*}
q_1^\mu \to k_2^\mu=p^\mu-q_1^\mu+b^\mu ,\quad q_2^\mu \to p^\mu=q_2^\mu-b^\mu ,\quad q^\mu_3\to k_1^\mu=-p^\mu+q_3^\mu-b^\mu
\end{align*}
を行うと(体積要素から出るマイナスは積分領域のマイナスと打ち消し合って)
\begin{align*}
\Gamma^{\mu\nu\rho}_{\alpha\beta\gamma}(x,y,z)&=\frac{i}{(2\pi)^{12}}\int d^4 k_1 d^4k_2 e^{-i(k_1+k_2)\cdot x}e^{ik_1\cdot y}e^{ik_2\cdot z}\int d^4p \\
&\times \biggl\{\mr{tr}\left[\frac{\Slash{p}-\Slash{k}_1+\Slash{a}}{(p-k_1+a)^2-i\epsilon}\gamma^\nu \frac{\Slash{p}+\Slash{a}}{(p+a)^2-i\epsilon}\gamma^\rho \frac{\Slash{p}+\Slash{k}_2+\Slash{a}}{(p+k_2+a)^2-i\epsilon}\gamma^\mu \frac{1+\gamma_5}{2}\right]\mr{tr}[T_\beta T_\gamma T_\alpha] \\
&+\mr{tr}\left[\frac{\Slash{p}-\Slash{k}_2+\Slash{b}}{(p-k_2+b)^2-i\epsilon}\gamma^\rho \frac{\Slash{p}+\Slash{b}}{(p+b)^2-i\epsilon}\gamma^\nu \frac{\Slash{p}+\Slash{k}_1+\Slash{b}}{(p+k_1+b)^2-i\epsilon}\gamma^\mu \frac{1+\gamma_5}{2}\right]\mr{tr}[T_\gamma T_\beta T_\alpha]  \biggr\}
\end{align*}
となる.
\begin{align*}
\Slash{k}_1+\Slash{k}_2=(\Slash{p}+\Slash{k}_2+\Slash{a})-(\Slash{p}-\Slash{k}_1+\Slash{a})=(\Slash{p}+\Slash{k}_1+\Slash{b})-(\Slash{p}-\Slash{k}_2+\Slash{b})
\end{align*}
を用いると,トレースの巡回性と$\gamma_5\gamma^\mu=-\gamma^\mu \gamma_5$と$\Slash{k}\Slash{k}=k^2$より
\begin{align*}
&\frac{\partial}{\partial x^\mu}\Gamma^{\mu\nu\rho}_{\alpha\beta\gamma}(x,y,z) \\
=&\frac{1}{(2\pi)^{12}}\int d^4k_1 d^4k_2(k_{1\mu} +k_{2\mu})e^{-i(k_1+k_2)\cdot x}e^{ik_1\cdot y}e^{ik_2\cdot z}\int d^4p \\
&\times \biggl\{\mr{tr}\left[\frac{\Slash{p}-\Slash{k}_1+\Slash{a}}{(p-k_1+a)^2-i\epsilon}\gamma^\nu \frac{\Slash{p}+\Slash{a}}{(p+a)^2-i\epsilon}\gamma^\rho \frac{\Slash{p}+\Slash{k}_2+\Slash{a}}{(p+k_2+a)^2-i\epsilon}\gamma^\mu \frac{1+\gamma_5}{2}\right]\mr{tr}[T_\beta T_\gamma T_\alpha] \\
&+\mr{tr}\left[\frac{\Slash{p}-\Slash{k}_2+\Slash{b}}{(p-k_2+b)^2-i\epsilon}\gamma^\rho \frac{\Slash{p}+\Slash{b}}{(p+b)^2-i\epsilon}\gamma^\nu \frac{\Slash{p}+\Slash{k}_1+\Slash{b}}{(p+k_1+b)^2-i\epsilon}\gamma^\mu \frac{1+\gamma_5}{2}\right]\mr{tr}[T_\gamma T_\beta T_\alpha]  \biggr\} \\
=&\frac{1}{(2\pi)^{12}}\int d^4k_1 d^4k_2e^{-i(k_1+k_2)\cdot x}e^{ik_1\cdot y}e^{ik_2\cdot z}\int d^4p \\
&\times \biggl\{\mr{tr}\left[\frac{\Slash{p}-\Slash{k}_1+\Slash{a}}{(p-k_1+a)^2-i\epsilon}\gamma^\nu \frac{\Slash{p}+\Slash{a}}{(p+a)^2-i\epsilon}\gamma^\rho \frac{\Slash{p}+\Slash{k}_2+\Slash{a}}{(p+k_2+a)^2-i\epsilon}(\Slash{k}_1+\Slash{k}_2) \frac{1+\gamma_5}{2}\right]\mr{tr}[T_\beta T_\gamma T_\alpha] \\
&+\mr{tr}\left[\frac{\Slash{p}-\Slash{k}_2+\Slash{b}}{(p-k_2+b)^2-i\epsilon}\gamma^\rho \frac{\Slash{p}+\Slash{b}}{(p+b)^2-i\epsilon}\gamma^\nu \frac{\Slash{p}+\Slash{k}_1+\Slash{b}}{(p+k_1+b)^2-i\epsilon}(\Slash{k}_1+\Slash{k}_2) \frac{1+\gamma_5}{2}\right]\mr{tr}[T_\gamma T_\beta T_\alpha]  \biggr\} \\
=&\frac{1}{(2\pi)^{12}}\int d^4k_1 d^4k_2e^{-i(k_1+k_2)\cdot x}e^{ik_1\cdot y}e^{ik_2\cdot z}\int d^4p \\
&\times \biggl\{\mr{tr}\left[\frac{\Slash{p}-\Slash{k}_1+\Slash{a}}{(p-k_1+a)^2-i\epsilon}\gamma^\nu \frac{\Slash{p}+\Slash{a}}{(p+a)^2-i\epsilon}\gamma^\rho \frac{\Slash{p}+\Slash{k}_2+\Slash{a}}{(p+k_2+a)^2-i\epsilon}(\Slash{p}+\Slash{k}_2+\Slash{a}) \frac{1+\gamma_5}{2}\right]\mr{tr}[T_\beta T_\gamma T_\alpha] \\
&-\mr{tr}\left[\frac{\Slash{p}-\Slash{k}_1+\Slash{a}}{(p-k_1+a)^2-i\epsilon}\gamma^\nu \frac{\Slash{p}+\Slash{a}}{(p+a)^2-i\epsilon}\gamma^\rho \frac{\Slash{p}+\Slash{k}_2+\Slash{a}}{(p+k_2+a)^2-i\epsilon}(\Slash{p}-\Slash{k}_1+\Slash{a}) \frac{1+\gamma_5}{2}\right]\mr{tr}[T_\beta T_\gamma T_\alpha] \\
&+\mr{tr}\left[\frac{\Slash{p}-\Slash{k}_2+\Slash{b}}{(p-k_2+b)^2-i\epsilon}\gamma^\rho \frac{\Slash{p}+\Slash{b}}{(p+b)^2-i\epsilon}\gamma^\nu \frac{\Slash{p}+\Slash{k}_1+\Slash{b}}{(p+k_1+b)^2-i\epsilon}(\Slash{p}+\Slash{k}_1+\Slash{b}) \frac{1+\gamma_5}{2}\right]\mr{tr}[T_\gamma T_\beta T_\alpha] \\
&-\mr{tr}\left[\frac{\Slash{p}-\Slash{k}_2+\Slash{b}}{(p-k_2+b)^2-i\epsilon}\gamma^\rho \frac{\Slash{p}+\Slash{b}}{(p+b)^2-i\epsilon}\gamma^\nu \frac{\Slash{p}+\Slash{k}_1+\Slash{b}}{(p+k_1+b)^2-i\epsilon}(\Slash{p}-\Slash{k}_2+\Slash{b}) \frac{1+\gamma_5}{2}\right]\mr{tr}[T_\gamma T_\beta T_\alpha]  \biggr\} \\
=&\frac{1}{(2\pi)^{12}}\int d^4k_1 d^4k_2e^{-i(k_1+k_2)\cdot x}e^{ik_1\cdot y}e^{ik_2\cdot z}\int d^4p \\
&\times \biggl\{\mr{tr}\left[\frac{\Slash{p}-\Slash{k}_1+\Slash{a}}{(p-k_1+a)^2-i\epsilon}\gamma^\nu \frac{\Slash{p}+\Slash{a}}{(p+a)^2-i\epsilon}\gamma^\rho \frac{1+\gamma_5}{2}\right]\mr{tr}[T_\beta T_\gamma T_\alpha] \\
&-\mr{tr}\left[\gamma^\nu \frac{\Slash{p}+\Slash{a}}{(p+a)^2-i\epsilon}\gamma^\rho \frac{\Slash{p}+\Slash{k}_2+\Slash{a}}{(p+k_2+a)^2-i\epsilon} \frac{1-\gamma_5}{2}\right]\mr{tr}[T_\beta T_\gamma T_\alpha] \\
&+\mr{tr}\left[\frac{\Slash{p}-\Slash{k}_2+\Slash{b}}{(p-k_2+b)^2-i\epsilon}\gamma^\rho \frac{\Slash{p}+\Slash{b}}{(p+b)^2-i\epsilon}\gamma^\nu \frac{1+\gamma_5}{2}\right]\mr{tr}[T_\gamma T_\beta T_\alpha] \\
&-\mr{tr}\left[\gamma^\rho \frac{\Slash{p}+\Slash{b}}{(p+b)^2-i\epsilon}\gamma^\nu \frac{\Slash{p}+\Slash{k}_1+\Slash{b}}{(p+k_1+b)^2-i\epsilon} \frac{1-\gamma_5}{2}\right]\mr{tr}[T_\gamma T_\beta T_\alpha]  \biggr\} \\
=&\frac{1}{(2\pi)^{12}}\int d^4k_1 d^4k_2e^{-i(k_1+k_2)\cdot x}e^{ik_1\cdot y}e^{ik_2\cdot z}\int d^4p \\
&\times \biggl\{\mr{tr}\left[\frac{\Slash{p}-\Slash{k}_1+\Slash{a}}{(p-k_1+a)^2-i\epsilon}\gamma^\nu \frac{\Slash{p}+\Slash{a}}{(p+a)^2-i\epsilon}\gamma^\rho \frac{1+\gamma_5}{2}\right]\mr{tr}[T_\beta T_\gamma T_\alpha] \\
&-\mr{tr}\left[ \frac{\Slash{p}+\Slash{a}}{(p+a)^2-i\epsilon}\gamma^\rho \frac{\Slash{p}+\Slash{k}_2+\Slash{a}}{(p+k_2+a)^2-i\epsilon} \gamma^\nu\frac{1+\gamma_5}{2}\right]\mr{tr}[T_\beta T_\gamma T_\alpha] \\
&+\mr{tr}\left[\frac{\Slash{p}-\Slash{k}_2+\Slash{b}}{(p-k_2+b)^2-i\epsilon}\gamma^\rho \frac{\Slash{p}+\Slash{b}}{(p+b)^2-i\epsilon}\gamma^\nu \frac{1+\gamma_5}{2}\right]\mr{tr}[T_\gamma T_\beta T_\alpha] \\
&-\mr{tr}\left[ \frac{\Slash{p}+\Slash{b}}{(p+b)^2-i\epsilon}\gamma^\nu \frac{\Slash{p}+\Slash{k}_1+\Slash{b}}{(p+k_1+b)^2-i\epsilon} \gamma^\rho\frac{1+\gamma_5}{2}\right]\mr{tr}[T_\gamma T_\beta T_\alpha]  \biggr\}
\end{align*}
となる.\par
この時点で,3点関数を群の指標$\alpha\beta\cdots$について対称な部分と反対称な部分に分けると便利だ.
\begin{align*}
&\mr{tr}[T_\alpha T_\beta]=N\delta_{\alpha\beta} \\
&D_{\alpha\beta\gamma}=\frac{1}{2}\mr{tr}[\{T_\beta , T_\gamma \}T_\alpha]
\end{align*}
という量を定義すると
\begin{align*}
\mr{tr}[T_\beta T_\gamma T_\alpha]-\mr{tr}[T_\gamma T_\beta T_\alpha]=& iC_{\beta\gamma\delta}\mr{tr}[T_\delta T_\alpha] \\
=&iC_{\beta\gamma\delta}N\delta_{\delta \alpha}=iNC_{\beta \gamma \alpha} \\
\mr{tr}[T_\beta T_\gamma T_\alpha]+\mr{tr}[T_\gamma T_\beta T_\alpha]=&\mr{tr}[\{T_\beta , T_\gamma \}T_\alpha]=2D_{\beta \gamma \alpha}
\end{align*}
前者は$C_{\alpha\beta\gamma}$が完全反対称であったことを思い出せば自明に完全反対称だ.$D_{\beta\gamma\alpha}$は$\beta\gamma$については反交換子より対称であることがわかり,またトレースの巡回性より
\begin{align*}
2D_{\alpha\gamma\beta}=&\mr{tr}[T_\alpha T_\gamma T_\beta]+\mr{tr}[T_\gamma T_\alpha T_\beta] \\
=&\mr{tr}[T_\gamma T_\beta T_\alpha]+\mr{tr}[T_\beta T_\gamma T_\alpha]=2D_{\beta\gamma\alpha}
\end{align*}
が示せるから
\begin{align*}
D_{\beta\alpha\gamma}=D_{\alpha\beta\gamma}=D_{\gamma\beta\alpha}
\end{align*}
となって完全対称であることがわかる.よってこれらの完全対称な量と完全反対称な量によって
\begin{align*}
&\mr{tr}[T_\beta T_\gamma T_\alpha]=D_{\alpha\beta\gamma}+\frac{1}{2}iNC_{\alpha\beta\gamma} \\
&\mr{tr}[T_\gamma T_\beta T_\alpha]=D_{\alpha\beta\gamma}-\frac{1}{2}iNC_{\alpha\beta\gamma}
\end{align*}
と書ける.群の指標に関して反対称な項は一般にゼロではない.しかしこれはなんら対称性の破れを表すものではない.行列要素(22.3.5)の発散を形式的に計算する際には時間順序積の$\theta$関数の時間微分から以下の寄与がある.(20.4節の計算を見直すと良い)
\begin{align*}
&\left[ \frac{\partial}{\partial x^\mu}\Gamma^{\mu\nu\rho}_{\alpha\beta\gamma}(x,y,z) \right]_{\mr{formal}}=\frac{\partial}{\partial x^\mu}\braket{T\{ j^\mu_\alpha(x),j^\nu_\beta(y),j^\rho_\gamma(z) \}}_{\VAC} \\
=&\frac{\partial}{\partial x^\mu}\langle j^\mu_\alpha(x)\theta(x^0-y^0)j^\nu_\beta(y)\theta(y^0-z^0)j^\rho_\gamma(z)+ j^\nu_\beta (y)\theta(y^0-x^0)j^\mu_\alpha(x)\theta(x^0-z^0)j^\rho_\gamma(z) \\
&\qquad+j^\nu_\beta(y)\theta(y^0-z^0)j^\rho_\gamma(z)\theta(z^0-x^0)j^\mu_\alpha(x) \\
&\qquad +j^\mu_\alpha(x)\theta(x^0-z^0)j^\rho_\gamma(z)\theta(z^0-y^0)j^\nu_\beta(y)+ j^\rho_\gamma (z)\theta(z^0-x^0)j^\mu_\alpha(x)\theta(x^0-y^0)j^\nu_\beta(y)  \\
&\qquad +j^\rho_\gamma(z)\theta(z^0-y^0)j^\nu_\beta(y)\theta(y^0-x^0)j^\mu_\alpha(x) \rangle_{\VAC} \\
=&\langle j^0_\alpha(x)\delta(x^0-y^0)j^\nu_\beta(y)\theta(y^0-z^0)j^\rho_\gamma(z) \\
&-j^\nu_\beta (y)\delta(y^0-x^0)j^0_\alpha(x)\theta(x^0-z^0)j^\rho_\gamma(z) +j^\nu_\beta (y)\theta(y^0-x^0)j^0_\alpha(x)\delta(x^0-z^0)j^\rho_\gamma(z) \\
&-j^\nu_\beta(y)\theta(y^0-z^0)j^\rho_\gamma(z)\delta(z^0-x^0)j^0_\alpha(x) \\
&+j^0_\alpha(x)\delta(x^0-z^0)j^\rho_\gamma(z)\theta(z^0-y^0)j^\nu_\beta(y) \\
&-j^\rho_\gamma (z)\delta(z^0-x^0)j^0_\alpha(x)\theta(x^0-y^0)j^\nu_\beta(y)+j^\rho_\gamma (z)\theta(z^0-x^0)j^0_\alpha(x)\delta(x^0-y^0)j^\nu_\beta(y) \\
&-j^\rho_\gamma(z)\theta(z^0-y^0)j^\nu_\beta(y)\delta(y^0-x^0)j^0_\alpha(x) \rangle_{\VAC} \\
=&\langle \delta(x^0-y^0)[j^0_\alpha (\mathbf{x},y^0),j^\nu_\beta(\mathbf{y},y^0)]\theta(y^0-z^0)j^\rho_\gamma(z) +\delta(x^0-z^0)j^\nu_\beta(y) \theta(y^0-z^0)[j^0_\alpha(\mathbf{x},z^0),j^\rho_\gamma(\mathbf{z},z^0)] \\
&+\delta(x^0-z^0)[j^0_\alpha (\mathbf{x},z^0),j^\rho_\gamma(\mathbf{z},z^0)]\theta(z^0-y^0)j^\nu_\beta(y)+\delta(x^0-y^0)j^\rho_\gamma(z) \theta(z^0-y^0)[j^0_\alpha(\mathbf{x},y^0),j^\nu_\beta(\mathbf{y},y^0)]\rangle_{\VAC}
\end{align*}
ここで(22.3.6)より,$\bar{\chi}=\chi^\dagger \gamma_4,\gamma_4 =i\gamma^0$を用いれば
\begin{align*}
[j^0_\alpha (\mathbf{x},y^0),j^\nu_\beta(\mathbf{y},y^0)]=&(-i)^2\bar{\chi}(\mathbf{x})T_\alpha \gamma^0 \chi(\mathbf{x})\bar{\chi}(\mathbf{y})T_\beta \gamma^\nu \chi(\mathbf{y}) \\
&-(-i)^2\bar{\chi}(\mathbf{y})T_\beta \gamma^\nu \chi(\mathbf{y})\bar{\chi}(\mathbf{x})T_\alpha \gamma^0 \chi(\mathbf{x}) \\
=&-\bar{\chi}(\mathbf{x})T_\alpha \gamma^0 \chi(\mathbf{x})\chi^\dagger(\mathbf{y})i\gamma^0T_\beta \gamma^\nu \chi(\mathbf{y}) \\
&+\bar{\chi}(\mathbf{y})T_\beta \gamma^\nu \chi(\mathbf{y})\chi^\dagger(\mathbf{x})(-i)T_\alpha \chi(\mathbf{x}) \\
=&i\delta^3(\mathbf{x}-\mathbf{y})\left\{ \bar{\chi}(\mathbf{x})T_\alpha T_\beta \gamma^\nu \chi(\mathbf{y}) -\bar{\chi}(\mathbf{y}) T_\beta T_\alpha \gamma^\nu \chi(\mathbf{x}) \right\} \\
=&i\delta^3(\mathbf{x}-\mathbf{y})\bar{\chi}(\mathbf{y})[T_\alpha,T_\beta]\gamma^\nu \chi(\mathbf{y}) \\
=&-C_{\alpha\beta\delta}\delta^3(\mathbf{x}-\mathbf{y})\bar{\chi}(\mathbf{y})T_\delta\gamma^\nu \chi(\mathbf{y})=-i\delta^3(\mathbf{x}-\mathbf{y})C_{\alpha\beta\delta}j^\nu_\delta(y)
\end{align*}
が示せる.(ここで$\chi(\mathbf{y},y^0)$等における$y^0$は簡単のため省略した.)したがって
\begin{align*}
&\left[ \frac{\partial}{\partial x^\mu}\Gamma^{\mu\nu\rho}_{\alpha\beta\gamma}(x,y,z) \right]_{\mr{formal}} \\
=&\langle -iC_{\alpha\beta\delta} \delta^4(x-y)j^\nu_\delta(y) \theta(y^0-z^0) j^\rho_\gamma(z) -iC_{\alpha\gamma\delta}\delta^4(x-z)j^\nu_\beta(y) \theta(y^0-z^0)j^\rho_\delta(z) \\
&-iC_{\alpha\beta\delta}\delta^4(x-z)j^\rho_\delta(z) \theta(z^0-y^0)j^\nu_\beta(y)-iC_{\alpha\gamma\delta}\delta^4(x-y)j^\rho_\gamma(z) \theta(z^0-y^0)j^\nu_\delta(y)\rangle_{\VAC} \\
=&-iC_{\alpha\beta\delta}\delta^4(x-y)\braket{T\{j^\nu_\delta(y),j^\rho_\gamma(z)\}}_{\VAC}-iC_{\alpha\gamma\delta}\delta^4(x-z)\braket{T\{j^\nu_\beta(y),j^\rho_\delta(z) \}}_{\VAC}
\end{align*}
となる.(22.3.11)の反対称項がこれを再現することを示そう.(22.3.11)は
\begin{align*}
&\frac{\partial}{\partial x^\mu}\Gamma^{\mu\nu\rho}_{\alpha\beta\gamma}(x,y,z) \\
=&\frac{1}{(2\pi)^{12}}\int d^4k_1 d^4k_2e^{-i(k_1+k_2)\cdot x}e^{ik_1\cdot y}e^{ik_2\cdot z}\int d^4p \\
&\times \biggl\{ \left[D_{\alpha\beta\gamma}+\frac{1}{2}iNC_{\alpha\beta\gamma}\right]\mr{tr}\left[\frac{\Slash{p}-\Slash{k}_1+\Slash{a}}{(p-k_1+a)^2-i\epsilon}\gamma^\nu \frac{\Slash{p}+\Slash{a}}{(p+a)^2-i\epsilon}\gamma^\rho \frac{1+\gamma_5}{2}\right] \\
&-\left[D_{\alpha\beta\gamma}+\frac{1}{2}iNC_{\alpha\beta\gamma}\right]\mr{tr}\left[ \frac{\Slash{p}+\Slash{a}}{(p+a)^2-i\epsilon}\gamma^\rho \frac{\Slash{p}+\Slash{k}_2+\Slash{a}}{(p+k_2+a)^2-i\epsilon} \gamma^\nu\frac{1+\gamma_5}{2}\right] \\
&+\left[D_{\alpha\beta\gamma}-\frac{1}{2}iNC_{\alpha\beta\gamma}\right]\mr{tr}\left[\frac{\Slash{p}-\Slash{k}_2+\Slash{b}}{(p-k_2+b)^2-i\epsilon}\gamma^\rho \frac{\Slash{p}+\Slash{b}}{(p+b)^2-i\epsilon}\gamma^\nu \frac{1+\gamma_5}{2}\right] \\
&-\left[D_{\alpha\beta\gamma}-\frac{1}{2}iNC_{\alpha\beta\gamma}\right]\mr{tr}\left[ \frac{\Slash{p}+\Slash{b}}{(p+b)^2-i\epsilon}\gamma^\nu \frac{\Slash{p}+\Slash{k}_1+\Slash{b}}{(p+k_1+b)^2-i\epsilon} \gamma^\rho\frac{1+\gamma_5}{2}\right]  \biggr\} \\
=&\frac{1}{(2\pi)^{12}}D_{\alpha\beta\gamma}\int d^4k_1 d^4k_2e^{-i(k_1+k_2)\cdot x}e^{ik_1\cdot y}e^{ik_2\cdot z} \\
&\times \int d^4p \biggl\{ \mr{tr}\left[\frac{\Slash{p}-\Slash{k}_1+\Slash{a}}{(p-k_1+a)^2-i\epsilon}\gamma^\nu \frac{\Slash{p}+\Slash{a}}{(p+a)^2-i\epsilon}\gamma^\rho \frac{1+\gamma_5}{2}\right]  \\
&\qquad \quad -\mr{tr}\left[ \frac{\Slash{p}+\Slash{a}}{(p+a)^2-i\epsilon}\gamma^\rho \frac{\Slash{p}+\Slash{k}_2+\Slash{a}}{(p+k_2+a)^2-i\epsilon} \gamma^\nu\frac{1+\gamma_5}{2}\right] \\
&\qquad \quad +\mr{tr}\left[\frac{\Slash{p}-\Slash{k}_2+\Slash{b}}{(p-k_2+b)^2-i\epsilon}\gamma^\rho \frac{\Slash{p}+\Slash{b}}{(p+b)^2-i\epsilon}\gamma^\nu \frac{1+\gamma_5}{2}\right] \\
&\qquad\quad -\mr{tr}\left[ \frac{\Slash{p}+\Slash{b}}{(p+b)^2-i\epsilon}\gamma^\nu \frac{\Slash{p}+\Slash{k}_1+\Slash{b}}{(p+k_1+b)^2-i\epsilon} \gamma^\rho\frac{1+\gamma_5}{2}\right]  \biggr\} \\
&+\frac{i}{2(2\pi)^{12}}NC_{\alpha\beta\gamma}\int d^4k_1 d^4k_2e^{-i(k_1+k_2)\cdot x}e^{ik_1\cdot y}e^{ik_2\cdot z} \\
&\times \int d^4p\biggl\{ \mr{tr}\left[\frac{\Slash{p}-\Slash{k}_1+\Slash{a}}{(p-k_1+a)^2-i\epsilon}\gamma^\nu \frac{\Slash{p}+\Slash{a}}{(p+a)^2-i\epsilon}\gamma^\rho \frac{1+\gamma_5}{2}\right]  \\
&\qquad \quad -\mr{tr}\left[ \frac{\Slash{p}+\Slash{a}}{(p+a)^2-i\epsilon}\gamma^\rho \frac{\Slash{p}+\Slash{k}_2+\Slash{a}}{(p+k_2+a)^2-i\epsilon} \gamma^\nu\frac{1+\gamma_5}{2}\right] \\
&\qquad \quad -\mr{tr}\left[\frac{\Slash{p}-\Slash{k}_2+\Slash{b}}{(p-k_2+b)^2-i\epsilon}\gamma^\rho \frac{\Slash{p}+\Slash{b}}{(p+b)^2-i\epsilon}\gamma^\nu \frac{1+\gamma_5}{2}\right] \\
&\qquad\quad +\mr{tr}\left[ \frac{\Slash{p}+\Slash{b}}{(p+b)^2-i\epsilon}\gamma^\nu \frac{\Slash{p}+\Slash{k}_1+\Slash{b}}{(p+k_1+b)^2-i\epsilon} \gamma^\rho\frac{1+\gamma_5}{2}\right]  \biggr\}
\end{align*}
と分解される.ここで$D_{\alpha\beta\gamma}$に比例する項を
\begin{align*}
&\left[ \frac{\partial}{\partial x^\mu}\Gamma^{\mu\nu\rho}_{\alpha\beta\gamma}(x,y,z) \right]_{\mr{anom}} \\
=&\frac{1}{(2\pi)^{12}}D_{\alpha\beta\gamma}\int d^4k_1 d^4k_2e^{-i(k_1+k_2)\cdot x}e^{ik_1\cdot y}e^{ik_2\cdot z} \\
&\times \int d^4p \biggl\{ \mr{tr}\left[\frac{\Slash{p}-\Slash{k}_1+\Slash{a}}{(p-k_1+a)^2-i\epsilon}\gamma^\nu \frac{\Slash{p}+\Slash{a}}{(p+a)^2-i\epsilon}\gamma^\rho \frac{1+\gamma_5}{2}\right]  \\
&\qquad \quad -\mr{tr}\left[ \frac{\Slash{p}+\Slash{a}}{(p+a)^2-i\epsilon}\gamma^\rho \frac{\Slash{p}+\Slash{k}_2+\Slash{a}}{(p+k_2+a)^2-i\epsilon} \gamma^\nu\frac{1+\gamma_5}{2}\right] \\
&\qquad \quad +\mr{tr}\left[\frac{\Slash{p}-\Slash{k}_2+\Slash{b}}{(p-k_2+b)^2-i\epsilon}\gamma^\rho \frac{\Slash{p}+\Slash{b}}{(p+b)^2-i\epsilon}\gamma^\nu \frac{1+\gamma_5}{2}\right] \\
&\qquad\quad -\mr{tr}\left[ \frac{\Slash{p}+\Slash{b}}{(p+b)^2-i\epsilon}\gamma^\nu \frac{\Slash{p}+\Slash{k}_1+\Slash{b}}{(p+k_1+b)^2-i\epsilon} \gamma^\rho\frac{1+\gamma_5}{2}\right]  \biggr\}
\end{align*}
として,また$C_{\alpha\beta\gamma}$に比例する項を
\begin{align*}
&\left[ \frac{\partial}{\partial x^\mu}\Gamma^{\mu\nu\rho}_{\alpha\beta\gamma}(x,y,z) \right]'_{\mr{formal}} \\
=&\frac{i}{2(2\pi)^{12}}NC_{\alpha\beta\gamma}\int d^4k_1 d^4k_2e^{-i(k_1+k_2)\cdot x}e^{ik_1\cdot y}e^{ik_2\cdot z} \\
&\times \int d^4p\biggl\{ \mr{tr}\left[\frac{\Slash{p}-\Slash{k}_1+\Slash{a}}{(p-k_1+a)^2-i\epsilon}\gamma^\nu \frac{\Slash{p}+\Slash{a}}{(p+a)^2-i\epsilon}\gamma^\rho \frac{1+\gamma_5}{2}\right]  \\
&\qquad \quad -\mr{tr}\left[ \frac{\Slash{p}+\Slash{a}}{(p+a)^2-i\epsilon}\gamma^\rho \frac{\Slash{p}+\Slash{k}_2+\Slash{a}}{(p+k_2+a)^2-i\epsilon} \gamma^\nu\frac{1+\gamma_5}{2}\right] \\
&\qquad \quad -\mr{tr}\left[\frac{\Slash{p}-\Slash{k}_2+\Slash{b}}{(p-k_2+b)^2-i\epsilon}\gamma^\rho \frac{\Slash{p}+\Slash{b}}{(p+b)^2-i\epsilon}\gamma^\nu \frac{1+\gamma_5}{2}\right] \\
&\qquad\quad +\mr{tr}\left[ \frac{\Slash{p}+\Slash{b}}{(p+b)^2-i\epsilon}\gamma^\nu \frac{\Slash{p}+\Slash{k}_1+\Slash{b}}{(p+k_1+b)^2-i\epsilon} \gamma^\rho\frac{1+\gamma_5}{2}\right]  \biggr\}
\end{align*}
とおく.一方,(22.3.13)からは次のファインマンダイアグラムから生じる項がある.
\begin{figure}[H]
  \centering
  \begin{tikzpicture}[decoration={markings, 
    mark= at position 1.46cm with {\arrow[line width=0.5mm]{Stealth}}}]
    \draw[thick,decorate,decoration={snake,amplitude=.5mm,segment length=4mm,post length=1mm}](0,0)--(2.5,0);
    \draw[thick,decorate,decoration={snake,amplitude=.5mm,segment length=4mm,post length=1mm}](6,0)--(3.5,0);
    \filldraw[thick,fill=white](3,0)circle[x radius=1,y radius=0.7];
    \draw[thick,postaction={decorate}](2,0)arc(180:80:1cm and 0.7cm);
    \draw[thick,postaction={decorate}](4,0)arc(0:-100:1cm and 0.7cm);
   
    
  \end{tikzpicture}
  
\end{figure}
したがって,
\begin{align*}
&\left[ \frac{\partial}{\partial x^\mu}\Gamma^{\mu\nu\rho}_{\alpha\beta\gamma}(x,y,z) \right]_{\mr{formal}} \\
=&-iC_{\alpha\beta\delta}\delta^4(x-y)\left\{-(-i)^2\mr{Tr}\left[S(y-z)T_\delta \gamma^\nu S(z-y)T_\gamma \gamma^\rho P_L\right] \right\} \\
&-iC_{\alpha\gamma\delta}\delta^4(x-z)\left\{-(-i)^2\mr{Tr}\left[S(y-z)T_\beta \gamma^\nu S(z-y) T_\delta \gamma^\rho P_L \right]\right\} \\
=&-iC_{\alpha\beta\delta}\delta^4(x-y)\int d^4q_1 d^4q_2 e^{iq_1\cdot( y-z)}e^{iq_2\cdot (z-y)}\mr{Tr}\left[\frac{-i}{(2\pi)^4}\frac{-i\Slash{q_1}}{q^2_1-i\epsilon}T_\delta \gamma^\nu \frac{-i}{(2\pi)^4}\frac{-i\Slash{q_2}}{q^2_2-i\epsilon}T_\gamma \gamma^\rho \frac{1+\gamma_5}{2}\right] \\
&-iC_{\alpha\gamma\delta}\delta^4(x-z)\int d^4q_1 d^4q_2 e^{iq_1\cdot (y-z)}e^{iq_2\cdot (z-y)} \mr{Tr}\left[\frac{-i}{(2\pi)^4}\frac{-i\Slash{q_1}}{q^2_1-i\epsilon}T_\beta \gamma^\nu \frac{-i}{(2\pi)^4}\frac{-i\Slash{q_2}}{q^2_2-i\epsilon}T_\delta \gamma^\rho \frac{1+\gamma_5}{2}\right] \\
=&-iC_{\alpha\beta\delta}\delta^4(x-y)\frac{1}{(2\pi)^8}\int d^4q_1 d^4q_2 e^{i(q_1-q_2)\cdot y}e^{i(q_2-q_1)\cdot z}\mr{tr}\left[\frac{\Slash{q_1}}{q^2_1-i\epsilon}\gamma^\nu \frac{\Slash{q_2}}{q^2_2-i\epsilon}\gamma^\rho \frac{1+\gamma_5}{2} \right]\mr{tr}[T_\delta T_\gamma] \\
&-iC_{\alpha\gamma\delta}\delta^4(x-z)\frac{1}{(2\pi)^8}\int d^4q_1 d^4q_2 e^{i(q_1-q_2)\cdot y}e^{i(q_2-q_1)\cdot z}\mr{tr}\left[\frac{\Slash{q_1}}{q^2_1-i\epsilon}\gamma^\nu \frac{\Slash{q_2}}{q^2_2-i\epsilon}\gamma^\rho \frac{1+\gamma_5}{2} \right]\mr{tr}[T_\beta T_\delta] \\
=&-iNC_{\alpha\beta\gamma}\delta^4(x-y)\frac{1}{(2\pi)^8}\int d^4q_1 d^4q_2 e^{i(q_1-q_2)\cdot y}e^{i(q_2-q_1)\cdot z}\mr{tr}\left[\frac{\Slash{q_1}}{q^2_1-i\epsilon}\gamma^\nu \frac{\Slash{q_2}}{q^2_2-i\epsilon}\gamma^\rho \frac{1+\gamma_5}{2} \right] \\
&+iNC_{\alpha\beta\gamma}\delta^4(x-z)\frac{1}{(2\pi)^8}\int d^4q_1 d^4q_2 e^{i(q_1-q_2)\cdot y}e^{i(q_2-q_1)\cdot z}\mr{tr}\left[\frac{\Slash{q_1}}{q^2_1-i\epsilon}\gamma^\nu \frac{\Slash{q_2}}{q^2_2-i\epsilon}\gamma^\rho \frac{1+\gamma_5}{2} \right]
\end{align*}
ここで$\mr{tr}[T_\alpha T_\beta]=N\delta_{\alpha\beta}$を用いた.第一項目は
\begin{align*}
&-iNC_{\alpha\beta\gamma}\delta^4(x-y)\frac{1}{(2\pi)^8}\int d^4q_1 d^4q_2 e^{i(q_1-q_2)\cdot y}e^{i(q_2-q_1)\cdot z}\mr{tr}\left[\frac{\Slash{q_1}}{q^2_1-i\epsilon}\gamma^\nu \frac{\Slash{q_2}}{q^2_2-i\epsilon}\gamma^\rho \frac{1+\gamma_5}{2} \right] \\
=&-iNC_{\alpha\beta\gamma}\delta^4(x-y)\frac{1}{2(2\pi)^8}\int d^4q_1 d^4q_2 e^{i(q_1-q_2)\cdot y}e^{i(q_2-q_1)\cdot z}\mr{tr}\left[\frac{\Slash{q_1}}{q^2_1-i\epsilon}\gamma^\nu \frac{\Slash{q_2}}{q^2_2-i\epsilon}\gamma^\rho \frac{1+\gamma_5}{2} \right] \\
&-iNC_{\alpha\beta\gamma}\delta^4(x-y)\frac{1}{2(2\pi)^8}\int d^4q_1 d^4q_2 e^{i(q_1-q_2)\cdot y}e^{i(q_2-q_1)\cdot z}\mr{tr}\left[\frac{\Slash{q_1}}{q^2_1-i\epsilon}\gamma^\nu \frac{\Slash{q_2}}{q^2_2-i\epsilon}\gamma^\rho \frac{1+\gamma_5}{2} \right]  \\
=&-iNC_{\alpha\beta\gamma}\delta^4(x-y)\frac{1}{2(2\pi)^8}\int d^4k_2 d^4p e^{-ik_2\cdot y}e^{ik_2\cdot z} \mr{tr}\left[\frac{\Slash{p}+\Slash{a}}{(p+a)^2-i\epsilon}\gamma^\nu \frac{\Slash{p}+\Slash{k}_2+\Slash{a}}{(p+k_2+a)^2-i\epsilon}\gamma^\rho \frac{1+\gamma_5}{2} \right] \\
&-iNC_{\alpha\beta\gamma}\delta^4(x-y)\frac{1}{2(2\pi)^8}\int d^4k_2 d^4p e^{-ik_2\cdot y}e^{ik_2\cdot z}\mr{tr}\left[\frac{\Slash{p}-\Slash{k}_2+\Slash{b}}{(p-k_2+b)^2-i\epsilon}\gamma^\nu \frac{\Slash{p}+\Slash{b}}{(p+b)^2-i\epsilon}\gamma^\rho \frac{1+\gamma_5}{2} \right] \\
=&-iNC_{\alpha\beta\gamma}\frac{1}{2(2\pi)^{12}}\int d^4k_1 e^{-ik_1\cdot (x-y)}\int d^4k_2 d^4p e^{-ik_2\cdot x}e^{ik_2\cdot z}\qquad \because \delta^4(x-y)e^{ikx}=\delta^4(x-y)e^{iky} \\
&\times \mr{tr}\left[\frac{\Slash{p}+\Slash{a}}{(p+a)^2-i\epsilon}\gamma^\nu \frac{\Slash{p}+\Slash{k}_2+\Slash{a}}{(p+k_2+a)^2-i\epsilon}\gamma^\rho \frac{1+\gamma_5}{2} \right] \\
&-iNC_{\alpha\beta\gamma}\frac{1}{2(2\pi)^{12}}\int d^4k_1 e^{-ik_1\cdot (x-y)}\int d^4k_2 d^4p e^{-ik_2\cdot y}e^{ik_2\cdot z}\\
&\times \mr{tr}\left[\frac{\Slash{p}-\Slash{k}_2+\Slash{b}}{(p-k_2+b)^2-i\epsilon}\gamma^\nu \frac{\Slash{p}+\Slash{b}}{(p+b)^2-i\epsilon}\gamma^\rho \frac{1+\gamma_5}{2} \right] \\
=&-iNC_{\alpha\beta\gamma}\frac{1}{2(2\pi)^{12}}\int d^4k_1 d^4k_2\, e^{-i(k_1+k_2)\cdot x}e^{ik_1\cdot y}e^{ik_2\cdot z}\int d^4p \\
&\times \mr{tr}\left[\frac{\Slash{p}+\Slash{a}}{(p+a)^2-i\epsilon}\gamma^\nu \frac{\Slash{p}+\Slash{k}_2+\Slash{a}}{(p+k_2+a)^2-i\epsilon}\gamma^\rho \frac{1+\gamma_5}{2} \right] \\
&-iNC_{\alpha\beta\gamma}\frac{1}{2(2\pi)^{12}}\int d^4k_1 d^4k_2\, e^{-i(k_1+k_2)\cdot x}e^{ik_1\cdot y}e^{ik_2\cdot z}\int d^4p \\
&\times\mr{tr}\left[\frac{\Slash{p}-\Slash{k}_2+\Slash{b}}{(p-k_2+b)^2-i\epsilon}\gamma^\nu \frac{\Slash{p}+\Slash{b}}{(p+b)^2-i\epsilon}\gamma^\rho \frac{1+\gamma_5}{2} \right]
\end{align*}
となり,第二項目も同様にして
\begin{align*}
&+iNC_{\alpha\beta\gamma}\delta^4(x-z)\frac{1}{(2\pi)^8}\int d^4q_1 d^4q_2 e^{i(q_1-q_2)\cdot y}e^{i(q_2-q_1)\cdot z}\mr{tr}\left[\frac{\Slash{q_1}}{q^2_1-i\epsilon}\gamma^\nu \frac{\Slash{q_2}}{q^2_2-i\epsilon}\gamma^\rho \frac{1+\gamma_5}{2} \right] \\
=&+iNC_{\alpha\beta\gamma}\delta^4(x-z)\frac{1}{2(2\pi)^8}\int d^4q_1 d^4q_2 e^{i(q_1-q_2)\cdot y}e^{i(q_2-q_1)\cdot z}\mr{tr}\left[\frac{\Slash{q_1}}{q^2_1-i\epsilon}\gamma^\nu \frac{\Slash{q_2}}{q^2_2-i\epsilon}\gamma^\rho \frac{1+\gamma_5}{2} \right] \\
&+iNC_{\alpha\beta\gamma}\delta^4(x-z)\frac{1}{2(2\pi)^8}\int d^4q_1 d^4q_2 e^{i(q_1-q_2)\cdot y}e^{i(q_2-q_1)\cdot z}\mr{tr}\left[\frac{\Slash{q_1}}{q^2_1-i\epsilon}\gamma^\nu \frac{\Slash{q_2}}{q^2_2-i\epsilon}\gamma^\rho \frac{1+\gamma_5}{2} \right] \\
=&+iNC_{\alpha\beta\gamma}\delta^4(x-z)\frac{1}{2(2\pi)^8}\int d^4q_1 d^4q_2 e^{-i(q_1-q_2)\cdot y}e^{-i(q_2-q_1)\cdot z}\mr{tr}\left[\frac{-\Slash{q_1}}{q^2_1-i\epsilon}\gamma^\nu \frac{-\Slash{q_2}}{q^2_2-i\epsilon}\gamma^\rho \frac{1+\gamma_5}{2} \right] \\
&+iNC_{\alpha\beta\gamma}\delta^4(x-z)\frac{1}{2(2\pi)^8}\int d^4q_1 d^4q_2 e^{-i(q_1-q_2)\cdot y}e^{-i(q_2-q_1)\cdot z}\mr{tr}\left[\frac{-\Slash{q_1}}{q^2_1-i\epsilon}\gamma^\nu \frac{-\Slash{q_2}}{q^2_2-i\epsilon}\gamma^\rho \frac{1+\gamma_5}{2} \right] \\
=&+iNC_{\alpha\beta\gamma}\delta^4(x-z)\frac{1}{2(2\pi)^8}\int d^4k_1 d^4p e^{ik_1\cdot y}e^{-ik_1 \cdot z}\mr{tr}\left[\frac{\Slash{p}-\Slash{k}_1+\Slash{a}}{(p-k_1+a)^2-i\epsilon}\gamma^\nu \frac{\Slash{p}+\Slash{a}}{(p+a)^2-i\epsilon}\gamma^\rho \frac{1+\gamma_5}{2} \right] \\
&+iNC_{\alpha\beta\gamma}\delta^4(x-z)\frac{1}{2(2\pi)^8}\int d^4k_1 d^4p e^{ik_1 \cdot y}e^{-ik_1\cdot z}\mr{tr}\left[\frac{\Slash{p}+\Slash{b}}{(p+b)^2-i\epsilon}\gamma^\nu \frac{\Slash{p}+\Slash{k}_1+\Slash{b}}{(p+k_1+b)^2-i\epsilon}\gamma^\rho \frac{1+\gamma_5}{2} \right] \\
=&+iNC_{\alpha\beta\gamma}\frac{1}{2(2\pi)^{12}}\int d^4k_2 e^{-ik_2\cdot(x-z)} \int d^4k_2 d^4p e^{ik_1\cdot y}e^{-ik_1 \cdot x} \qquad \because \delta^4(x-z)e^{ikz}=\delta^4(x-z)e^{ikx}\\
&\times \mr{tr}\left[\frac{\Slash{p}-\Slash{k}_1+\Slash{a}}{(p-k_1+a)^2-i\epsilon}\gamma^\nu \frac{\Slash{p}+\Slash{a}}{(p+a)^2-i\epsilon}\gamma^\rho \frac{1+\gamma_5}{2} \right] \\
&+iNC_{\alpha\beta\gamma}\frac{1}{2(2\pi)^{12}}\int d^4k_2 e^{-ik_2\cdot(x-z)}\int d^4k_1 d^4p e^{ik_1 \cdot y}e^{-ik_1\cdot x} \\
&\times \mr{tr}\left[\frac{\Slash{p}+\Slash{b}}{(p+b)^2-i\epsilon}\gamma^\nu \frac{\Slash{p}+\Slash{k}_1+\Slash{b}}{(p+k_1+b)^2-i\epsilon}\gamma^\rho \frac{1+\gamma_5}{2} \right] \\
=&+iNC_{\alpha\beta\gamma}\frac{1}{2(2\pi)^{12}}\int d^4k_1d^4k_2 e^{-i(k_1+k_2)\cdot x}  e^{ik_1\cdot y}e^{ik_2 \cdot z} \int d^4p \\
&\times \mr{tr}\left[\frac{\Slash{p}-\Slash{k}_1+\Slash{a}}{(p-k_1+a)^2-i\epsilon}\gamma^\nu \frac{\Slash{p}+\Slash{a}}{(p+a)^2-i\epsilon}\gamma^\rho \frac{1+\gamma_5}{2} \right] \\
&+iNC_{\alpha\beta\gamma}\frac{1}{2(2\pi)^{12}}\int d^4k_1d^4k_2 e^{-i(k_1+k_2)\cdot x}  e^{ik_1\cdot y}e^{ik_2 \cdot z} \int d^4p \\
&\times \mr{tr}\left[\frac{\Slash{p}+\Slash{b}}{(p+b)^2-i\epsilon}\gamma^\nu \frac{\Slash{p}+\Slash{k}_1+\Slash{b}}{(p+k_1+b)^2-i\epsilon}\gamma^\rho \frac{1+\gamma_5}{2} \right]
\end{align*}
まとめると
\begin{align*}
&\left[ \frac{\partial}{\partial x^\mu}\Gamma^{\mu\nu\rho}_{\alpha\beta\gamma}(x,y,z) \right]_{\mr{formal}} \\
=&\frac{i}{2(2\pi)^{12}}NC_{\alpha\beta\gamma}\int d^4k_1 d^4k_2e^{-i(k_1+k_2)\cdot x}e^{ik_1\cdot y}e^{ik_2\cdot z} \\
&\times \int d^4p\biggl\{ \mr{tr}\left[\frac{\Slash{p}-\Slash{k}_1+\Slash{a}}{(p-k_1+a)^2-i\epsilon}\gamma^\nu \frac{\Slash{p}+\Slash{a}}{(p+a)^2-i\epsilon}\gamma^\rho \frac{1+\gamma_5}{2}\right]  \\
&\qquad \quad -\mr{tr}\left[ \frac{\Slash{p}+\Slash{a}}{(p+a)^2-i\epsilon}\gamma^\rho \frac{\Slash{p}+\Slash{k}_2+\Slash{a}}{(p+k_2+a)^2-i\epsilon} \gamma^\nu\frac{1+\gamma_5}{2}\right] \\
&\qquad \quad -\mr{tr}\left[\frac{\Slash{p}-\Slash{k}_2+\Slash{b}}{(p-k_2+b)^2-i\epsilon}\gamma^\rho \frac{\Slash{p}+\Slash{b}}{(p+b)^2-i\epsilon}\gamma^\nu \frac{1+\gamma_5}{2}\right] \\
&\qquad\quad +\mr{tr}\left[ \frac{\Slash{p}+\Slash{b}}{(p+b)^2-i\epsilon}\gamma^\nu \frac{\Slash{p}+\Slash{k}_1+\Slash{b}}{(p+k_1+b)^2-i\epsilon} \gamma^\rho\frac{1+\gamma_5}{2}\right]  \biggr\} \\
=&\left[ \frac{\partial}{\partial x^\mu}\Gamma^{\mu\nu\rho}_{\alpha\beta\gamma}(x,y,z) \right]'_{\mr{formal}}
\end{align*}
となって,(22.3.11)の反対称項が(22.3.13)を再現することが示せた.したがってアノマリーは(22.3.11)の対称部分
\begin{align*}
&\left[ \frac{\partial}{\partial x^\mu}\Gamma^{\mu\nu\rho}_{\alpha\beta\gamma}(x,y,z) \right]_{\mr{anom}} \\
&=\frac{1}{(2\pi)^{12}}D_{\alpha\beta\gamma}\int d^4k_1 d^4k_2e^{-i(k_1+k_2)\cdot x}e^{ik_1\cdot y}e^{ik_2\cdot z} \\
&\times \int d^4p \biggl\{ \mr{tr}\left[\frac{\Slash{p}-\Slash{k}_1+\Slash{a}}{(p-k_1+a)^2-i\epsilon}\gamma^\nu \frac{\Slash{p}+\Slash{a}}{(p+a)^2-i\epsilon}\gamma^\rho \frac{1+\gamma_5}{2}\right]  \\
&\qquad \quad -\mr{tr}\left[ \frac{\Slash{p}+\Slash{a}}{(p+a)^2-i\epsilon}\gamma^\rho \frac{\Slash{p}+\Slash{k}_2+\Slash{a}}{(p+k_2+a)^2-i\epsilon} \gamma^\nu\frac{1+\gamma_5}{2}\right] \\
&\qquad \quad +\mr{tr}\left[\frac{\Slash{p}-\Slash{k}_2+\Slash{b}}{(p-k_2+b)^2-i\epsilon}\gamma^\rho \frac{\Slash{p}+\Slash{b}}{(p+b)^2-i\epsilon}\gamma^\nu \frac{1+\gamma_5}{2}\right] \\
&\qquad\quad -\mr{tr}\left[ \frac{\Slash{p}+\Slash{b}}{(p+b)^2-i\epsilon}\gamma^\nu \frac{\Slash{p}+\Slash{k}_1+\Slash{b}}{(p+k_1+b)^2-i\epsilon} \gamma^\rho\frac{1+\gamma_5}{2}\right]  \biggr\} \\
&=\frac{1}{(2\pi)^{12}}D_{\alpha\beta\gamma}\int d^4k_1 d^4k_2e^{-i(k_1+k_2)\cdot x}e^{ik_1\cdot y}e^{ik_2\cdot z} \\
&\times \int d^4p \biggl\{ \mr{tr}\left[\gamma^\kappa \gamma^\nu \gamma^\lambda \gamma^\rho \frac{1+\gamma_5}{2}\right]\left[\frac{p_\kappa-k_{1\kappa}+a_\kappa}{(p-k_1+a)^2-i\epsilon}\frac{p_\lambda+a_\lambda}{(p+a)^2-i\epsilon} \right]  \\
&\qquad \qquad -\mr{tr}\left[\gamma^\kappa \gamma^\rho \gamma^\lambda \gamma^\nu\frac{1+\gamma_5}{2}\right]\left[ \frac{p_\kappa+a_\kappa}{(p+a)^2-i\epsilon} \frac{p_\lambda+k_{2\lambda}+a_\lambda }{(p+k_2+a)^2-i\epsilon}\right] \\
&\qquad \qquad +\mr{tr}\left[\gamma^\kappa \gamma^\rho \gamma^\lambda \gamma^\nu \frac{1+\gamma_5}{2}\right]\left[\frac{p_\kappa-k_{2\kappa}+b_\kappa}{(p-k_2+b)^2-i\epsilon} \frac{p_\lambda+b_\lambda }{(p+b)^2-i\epsilon}\right] \\
&\qquad\qquad -\mr{tr}\left[\gamma^\kappa \gamma^\nu \gamma^\lambda \gamma^\rho \frac{1+\gamma_5}{2}\right]\left[ \frac{p_\kappa+b_\kappa}{(p+b)^2-i\epsilon} \frac{p_\lambda+k_{1\lambda}+b_\lambda}{(p+k_1+b)^2-i\epsilon} \right]  \biggr\} \\
&=\frac{1}{(2\pi)^{12}}D_{\alpha\beta\gamma}\int d^4k_1 d^4k_2e^{-i(k_1+k_2)\cdot x}e^{ik_1\cdot y}e^{ik_2\cdot z} \\
&\times \int d^4p \biggl\{ \mr{tr}\left[\gamma^\kappa \gamma^\nu \gamma^\lambda \gamma^\rho \frac{1+\gamma_5}{2}\right] \\
&\qquad \qquad \times \left[\frac{(p-k_1+a)_\kappa}{(p-k_1+a)^2-i\epsilon}\frac{(p+a)_\lambda}{(p+a)^2-i\epsilon}- \frac{(p+b)_\kappa}{(p+b)^2-i\epsilon} \frac{(p+k_1+b)_\lambda}{(p+k_1+b)^2-i\epsilon} \right]  \\
&\qquad \qquad +\mr{tr}\left[\gamma^\kappa \gamma^\rho \gamma^\lambda \gamma^\nu\frac{1+\gamma_5}{2}\right] \\
&\qquad \qquad \times \left[\frac{(p-k_2+b)_\kappa}{(p-k_2+b)^2-i\epsilon} \frac{(p+b)_\lambda }{(p+b)^2-i\epsilon}- \frac{(p+a)_\kappa}{(p+a)^2-i\epsilon} \frac{(p+k_2+a)_\lambda }{(p+k_2+a)^2-i\epsilon}\right] \biggr\} \\
&=\frac{1}{(2\pi)^{12}}D_{\alpha\beta\gamma}\int d^4k_1 d^4k_2e^{-i(k_1+k_2)\cdot x}e^{ik_1\cdot y}e^{ik_2\cdot z} \\
&\times \int d^4p \biggl\{ \mr{tr}\left[\gamma^\kappa \gamma^\nu \gamma^\lambda \gamma^\rho \frac{1+\gamma_5}{2}\right] \\
&\times \left[\frac{(p+(a-b-k_1)+b)_\kappa}{(p+(a-b-k_1)+b)^2-i\epsilon}\frac{(p+(a-b-k_1)+(b+k_1))_\lambda}{(p+(a-b-k_1)+(b+k_1))^2-i\epsilon}- \frac{(p+b)_\kappa}{(p+b)^2-i\epsilon} \frac{(p+k_1+b)_\lambda}{(p+k_1+b)^2-i\epsilon} \right]  \\
&+\mr{tr}\left[\gamma^\kappa \gamma^\rho \gamma^\lambda \gamma^\nu\frac{1+\gamma_5}{2}\right] \\
&\times \left[\frac{(p+(b-a-k_2)+a)_\kappa}{(p+(b-a-k_2)+a)^2-i\epsilon} \frac{(p+(b-a-k_2)+(a+k_2))_\lambda }{(p+(b-a-k_2)+(a+k_2))^2-i\epsilon}- \frac{(p+a)_\kappa}{(p+a)^2-i\epsilon} \frac{(p+k_2+a)_\lambda }{(p+k_2+a)^2-i\epsilon}\right] \biggr\} \\
\end{align*}
から生じる.ここで
\begin{align*}
f_{\kappa\lambda}(p,c,d)\equiv \frac{(p+c)_\kappa (p+d)_\lambda}{[(p+c)^2-i\epsilon][(p+d)^2-i\epsilon]}
\end{align*}
とすると
\begin{align*}
&\left[ \frac{\partial}{\partial x^\mu}\Gamma^{\mu\nu\rho}_{\alpha\beta\gamma}(x,y,z) \right]_{\mr{anom}} \\
=&\frac{1}{(2\pi)^{12}}D_{\alpha\beta\gamma}\int d^4k_1 d^4k_2e^{-i(k_1+k_2)\cdot x}e^{ik_1\cdot y}e^{ik_2\cdot z} \\
&\int d^4p \biggl\{ \mr{tr}\left[\gamma^\kappa \gamma^\nu \gamma^\lambda \gamma^\rho \frac{1+\gamma_5}{2}\right] \left[f_{\kappa\lambda}(p+(a-b-k_1),b,b+k_1)-f_{\kappa\lambda}(p,b,b+k_1) \right]  \\
&\qquad \quad +\mr{tr}\left[\gamma^\kappa \gamma^\rho \gamma^\lambda \gamma^\nu\frac{1+\gamma_5}{2}\right] \left[f_{\kappa\lambda}(p+(b-a-k_2),a,a+k_2) -f_{\kappa\lambda}(p,a,a+k_2) \right] \\
\end{align*}
ここで
\begin{align*}
I_{\kappa,\lambda}(k,c,d)\equiv \int d^4 p[f_{\kappa\lambda}(p+k,c,d)-f_{\kappa\lambda}(p,c,d)]
\end{align*}
とすると
\begin{align*}
&\left[ \frac{\partial}{\partial x^\mu}\Gamma^{\mu\nu\rho}_{\alpha\beta\gamma}(x,y,z) \right]_{\mr{anom}} \\
=&\frac{1}{(2\pi)^{12}}D_{\alpha\beta\gamma}\int d^4k_1 d^4k_2e^{-i(k_1+k_2)\cdot x}e^{ik_1\cdot y}e^{ik_2\cdot z} \\
&\times \biggl\{ \mr{tr}\left[\gamma^\kappa \gamma^\nu \gamma^\lambda \gamma^\rho \frac{1+\gamma_5}{2}\right] I_{\kappa\lambda}(a-b-k_1,b,b+k_1)  \\
&+\mr{tr}\left[\gamma^\kappa \gamma^\rho \gamma^\lambda \gamma^\nu\frac{1+\gamma_5}{2}\right] I_{\kappa\lambda}(b-a-k_2,a,a+k_2) \biggr\}
\end{align*}
となる.これらの積分を計算するには,関数$f_{\kappa\lambda}(p+k,c,d)$の$k$についてのベキ展開を考える.
\begin{align*}
f_{\kappa\lambda}(p+k,c,d)=&\sum_{n=0}^\infty \frac{1}{n!}k^{\mu_1}\cdots k^{\mu_n}\frac{\partial^n f_{\kappa\lambda}(p,c,d)}{\partial p^{\mu_1}\cdots \partial p^{\mu_n}} \\
=&f_{\kappa\lambda}(p,c,d)+k^\mu \frac{\partial f_{\kappa\lambda}(p,c,d)}{\partial p^{\mu}}+\frac{1}{2} k^\mu k^\nu \frac{\partial^2 f_{\kappa\lambda}(p,c,d)}{\partial p^\mu \partial p^\nu}+ \cdots
\end{align*}
ゼロ次の項は明らかに(22.3.16)でゼロとなる.(22.3.16)の他の項は全て$p$に関する微分の積分だから,4次元版Stokesの定理より,ウィック回転の後には大きな4次元球の表面積分と書くことができる.その球の半径をいま仮に$P$としよう.そうすると,
\begin{align*}
\int d^4p k^{\mu_1}\cdots k^{\mu_n}\frac{\partial^n f_{\kappa\lambda}(p,c,d)}{\partial p^{\mu_1}\cdots \partial p^{\mu_n}}=i\int P^3 n_{\mu_1}\left\{k^{\mu_1}\cdots k^{\mu_n}\frac{\partial^{(n-1)} f_{\kappa\lambda}(p,c,d)}{\partial p^{\mu^2}\cdots \partial p^{\mu_n}}\right\} dS^3
\end{align*}
$f$は$P^{-2}$として振る舞うことが(22.3.17)からわかるから,右辺の,$f$の$n-1$階微分は$P^{-2-(n-1)}$として振る舞い,半径$P$の4次元球の表面積は$P^3$と振る舞うから,全体は$P^{2-n}$と振る舞う.したがって,$P\to \infty$において寄与できるのは$n=1$か$n=2$の項のみだ.
\begin{align*}
I_{\kappa\lambda}(k,c,d)=k^\mu\int d^4p \frac{\partial f_{\kappa\lambda}(p,c,d)}{\partial p^{\mu}}+\frac{1}{2} k^\mu k^\nu \int d^4p \frac{\partial^2 f_{\kappa\lambda}(p,c,d)}{\partial p^\mu \partial p^\nu}
\end{align*}
これを計算する.Stokesの定理よりこれらの積分は(SrednickiのChapter75参照)
\begin{align*}
\int d^4p \frac{\partial f_{\kappa\lambda}(p,c,d)}{\partial p^\mu}=&i\lim_{p\to \infty}\int d\Omega_4 p^2p_\mu f_{\kappa\lambda}(p,c,d) \\
\int d^4p \frac{\partial^2 f_{\kappa\lambda}(p,c,d)}{\partial p^\mu \partial p^\nu}=&i\lim_{p\to \infty}\int d\Omega_4 p^2p_\mu \frac{\partial f_{\kappa\lambda}(p,c,d)}{\partial p^\nu} 
\end{align*}
となることを用いる.ここで$\Omega_4$は4次元立体角で,全立体角は$\Omega_4=2\pi^2$となる.$f$の微分は
\begin{align*}
\frac{\partial f_{\kappa\lambda}(p,c,d)}{\partial p^{\nu}}=&\frac{\partial}{\partial p^\nu}\frac{(p+c)_\kappa (p+d)_\lambda}{[(p+c)^2-i\epsilon][(p+d)^2-i\epsilon]} \\
=&\eta_{\nu\kappa}\frac{(p+d)_\lambda}{[(p+c)^2-i\epsilon][(p+d)^2-i\epsilon]}+\eta_{\nu\lambda}\frac{(p+c)_\kappa}{[(p+c)^2-i\epsilon][(p+d)^2-i\epsilon]} \\
&-2\frac{(p+c)_\nu(p+c)_\kappa (p+d)_\lambda}{[(p+c)^2-i\epsilon]^2[(p+d)^2-i\epsilon]} \\
&-2\frac{(p+d)_\nu(p+c)_\kappa (p+d)_\lambda}{[(p+c)^2-i\epsilon][(p+d)^2-i\epsilon]^2}
\end{align*}
となる.(11.2.4)(11.3.2)のファインマンのパラメータ技法を用いると,
\begin{align*}
&\frac{1}{[(p+c)^2-i\epsilon][(p+d)^2-i\epsilon]}=\frac{1}{[p'^2-i\epsilon][(p'-q)^2-i\epsilon]} \quad(p'=p+c,q=c-d) \\
=&\int^1_0 dx\left[ (p'^2-i\epsilon)(1-x) +((p'-q)^2-i\epsilon)x \right]^{-2} \\
=&\int^1_0 dx\left[ (p'-qx)^2-i\epsilon +q^2x(1-x) \right]^{-2} \\
=&\int^1_0 dx\left[ (p+c-(c-d)x)^2-i\epsilon +(c-d)^2x(1-x) \right]^{-2} \\
&\frac{1}{[(p+c)^2-i\epsilon]^2[(p+d)^2-i\epsilon]}=\frac{1}{[p'^2-i\epsilon]^2[(p'-q)^2-i\epsilon]} \\
=&2\int^1_0 dx \int^x_0 dy \left[ (p'^2-i\epsilon)(1-x)+(p'^2-i\epsilon)(x-y)+((p'-q)^2-i\epsilon)y \right]^{-3} \\
=&2\int^1_0 dx \int^x_0 dy \left[(p'^2-i\epsilon)(1-x)+((p'-q)^2-i\epsilon)x \right ]^{-3} \\
=&2\int^1_0 dx \left[(p'-qx)^2-i\epsilon +q^2x(1-x)  \right]^{-3} x \\
=&2\int^1_0 dx \left[(p+c-(c-d)x)^2-i\epsilon +(c-d)^2x(1-x) \right]^{-3} x\\
&\frac{1}{[(p+c)^2-i\epsilon][(p+d)^2-i\epsilon]^2}=\frac{1}{[p'^2-i\epsilon][(p'-q)^2-i\epsilon]^2} \\
=&2\int^1_0 dx \int^x_0 dy \left[ (p'^2-i\epsilon)(1-x)+((p'-q)^2-i\epsilon)(x-y)+((p'-q)^2-i\epsilon)y \right]^{-3} \\
=&2\int^1_0 dx \int^x_0 dy \left[ (p'^2-i\epsilon)(1-x)+((p'-q)^2-i\epsilon)x \right]^{-3} \\
=&2\int^1_0 dx \left[(p+c-(c-d)x)^2-i\epsilon +(c-d)^2x(1-x) \right]^{-3} x
\end{align*}
ここで$p\to p'=p-c+(c-d)x$と変数変換しウィック回転を施し,Stokesの定理と,角度平均をとる際に(11.2.8)(11.2.9)の処方を用いれば
\begin{align*}
&\int d^4p \frac{\partial f_{\kappa\lambda}(p,c,d)}{\partial p^\mu} =i\int d^4p_E \frac{\partial f_{\kappa\lambda}(p,c,d)}{\partial p^\mu}\\
=&i\int d^4p'_E \frac{\partial f_{\kappa\lambda}(p',c,d)}{\partial p'^\mu}=i\lim_{p\to\infty}\int d\Omega_4 p'^2 p'_\mu f_{\kappa\lambda}(p',c,d) \\
=&i\lim_{p\to\infty}\int^1_0 dx \int d\Omega_4\frac{ [p-c+(c-d)x ]^2[p-c+(c-d)x]_\mu[p+(c-d)x]_\kappa [p-(c-d)(1-x)]_\lambda }{\left[ p^2 +(c-d)^2x(1-x) \right]^2} \\
=&i\lim_{p\to\infty}\int^1_0 dx\int d\Omega_4 [p^2-2p\cdot(c-(c-d)x)+(c-(c-d)x)^2] \\
&\qquad \times [p-c+(c-d)x]_\mu[p+(c-d)x]_\kappa [p-(c-d)(1-x)]_\lambda \\
&\qquad \times \left[ p^2 +(c-d)^2x(1-x) \right]^{-2} \\
=&i\lim_{p\to\infty}\int^1_0 dx \int d\Omega_4 \\
&\times [ -p^2p_\mu p_\kappa(c-d)_\lambda(1-x) + p^2p_\mu(c-d)_\kappa p_\lambda x -p^2[c-(c-d)x]_\mu p_\kappa p_\lambda  \\
&-2p\cdot [c-(c-d)x] p_\mu p_\kappa p_\lambda]\times  \left[ p^2 +(c-d)^2x(1-x) \right]^{-2} \\
=&i\Omega_4\lim_{p\to\infty}\int^1_0 dx \\
&\times \left[-\frac{(p^2)^2}{4}\eta_{\mu\kappa}(c-d)_\lambda (1-x)+\frac{(p^2)^2}{4}\eta_{\mu\lambda}(c-d)_\kappa x -\frac{(p^2)^2}{4}\eta_{\kappa\lambda}[c-(c-d) x]_\mu  \right. \\
&\qquad \left. -2\frac{(p^2)^2}{24}[c-(c-d)x]^\rho\{\eta_{\rho\mu}\eta_{\kappa\lambda}+\eta_{\rho\kappa}\eta_{\mu\lambda}+\eta_{\rho\lambda}\eta_{\mu\kappa} \} \right]\left[ p^2 +(c-d)^2x(1-x) \right]^{-2} \\
=&2\pi^2i\int^1_0 dx \left[-\frac{1}{4}\eta_{\mu\kappa}(c-d)_\lambda(1-x)+\frac{1}{4}\eta_{\mu\lambda}(c-d)_\kappa x+ \frac{1}{4}\eta_{\kappa\lambda}(c-d)_\mu x-\frac{1}{4}\eta_{\kappa\lambda}c_\mu \right. \\
&\qquad \left.-\frac{1}{12}\left\{(c-(c-d)x)_\mu\eta_{\kappa\lambda}+(c-(c-d)x)_\kappa\eta_{\mu\lambda}+(c-(c-d)x)_\lambda\eta_{\mu\kappa} \right\} \right] \\
=&2\pi^2i\left[-\frac{1}{8}\eta_{\mu\kappa}(c-d)_\lambda+\frac{1}{8}\eta_{\mu\lambda}(c-d)_\kappa - \frac{1}{8}\eta_{\kappa\lambda}(c+d)_\mu \right. \\
&\qquad \left.-\frac{1}{24}\{(c+d)_\mu\eta_{\kappa\lambda}+(c+d)_\kappa\eta_{\mu\lambda}+(c+d)_\lambda\eta_{\mu\kappa} \} \right] \\
=&\pi^2i\left[-\frac{1}{4}\eta_{\mu\kappa}c_\lambda +\frac{1}{4}\eta_{\mu\kappa}d_\lambda +\frac{1}{4}\eta_{\mu\lambda}c_\kappa -\frac{1}{4}\eta_{\mu\lambda}d_\kappa -\frac{1}{4}\eta_{\kappa\lambda}(c+d)_\mu\right. \\
&\qquad \left.-\frac{1}{12}\{(c+d)_\mu\eta_{\kappa\lambda}+(c+d)_\kappa\eta_{\mu\lambda}+(c+d)_\lambda\eta_{\mu\kappa} \} \right] \\
=&i\pi^2\left[-\frac{1}{3}\eta_{\mu\kappa}c_\lambda +\frac{1}{6}\eta_{\mu\kappa}d_\lambda +\frac{1}{6}\eta_{\mu\lambda}c_\kappa-\frac{1}{3}\eta_{\mu\lambda}d_\kappa -\frac{1}{3}\eta_{\kappa\lambda}(c+d)_\mu\right]
\end{align*}
となる.したがって
\begin{align*}
k^\mu\int d^4p \frac{\partial f_{\kappa\lambda}(p,c,d)}{\partial p^\mu}=i\pi^2\left[\frac{1}{6}k_\lambda c_\kappa+\frac{1}{6}k_\kappa d_\lambda-\frac{1}{3}k_\lambda d_\kappa-\frac{1}{3}k_\kappa c_\lambda -\frac{1}{3}\eta_{\kappa\lambda}k\cdot(c+d)\right]
\end{align*}
となる.これは係数が少し違う…どこかでミスがあるかもしれない.$I_{\kappa\lambda}(k,c,d)$の第二項目は
\begin{align*}
&\int d^4p \frac{\partial^2 f_{\kappa\lambda}(p,c,d)}{\partial p^\mu \partial p^\nu}=i\int d^4p_E \frac{\partial}{\partial p^\mu}\left\{\frac{\partial f_{\kappa\lambda}(p,c,d)}{\partial p^\nu}\right\} \\
=&i\int d^4p'_E \frac{\partial}{\partial p'^\mu}\left\{\frac{\partial f_{\kappa\lambda}(p',c,d)}{\partial p'^\nu}\right\} =i\int d\Omega_4 p'^2p'_\mu \left\{\frac{\partial f_{\kappa\lambda}(p',c,d)}{\partial p'^\nu} \right\}\\
=&i\lim_{p\to \infty} \int^1_0 dx \int d\Omega_4 (p-c+(c-d)x)^2(p-c+(c-d)x)_\mu \\
&\qquad \biggl\{ \frac{\eta_{\nu\kappa}(p-(c-d)(1-x))_\lambda +\eta_{\nu\lambda}(p+(c-d)x)_\kappa }{\left[ p^2+(c-d)^2x(1-x) \right]^2} \\
&\qquad -4\frac{(p+(c-d)x)_\nu(p+(c-d)x)_\kappa (p-(c-d)(1-x))_\lambda}{\left[p^2+(c-d)^2x(1-x) \right]^3}x \biggr\} \\
&\qquad -4\frac{(p-(c-d)(1-x))_\nu(p+(c-d)x)_\kappa (p-(c-d)(1-x))_\lambda}{\left[p^2+(c-d)^2x(1-x) \right]^3}x \biggr\} \\
=&i\lim_{p\to \infty}\int^1_0 dx \int d\Omega_4 \biggl\{ \frac{\eta_{\nu\kappa}p^2p_\mu p_\lambda+\eta_{\nu\lambda}p^2p_\mu p_\kappa}{\left[ p^2+(c-d)^2x(1-x) \right]^2} \\
&\qquad \qquad - 8\frac{p^2p_\mu p_\nu p_\kappa p_\lambda }{\left[p^2+(c-d)^2x(1-x) \right]^3} x +O(p^{-1}) \biggr\} \\
=&i\Omega_4\lim_{p\to \infty}\int^1_0 dx \biggl\{ \frac{[\eta_{\nu\kappa}\eta_{\mu\lambda}+\eta_{\nu\lambda}\eta_{\mu\kappa}](p^2)^2}{4\left[ p^2+(c-d)^2x(1-x) \right]^2}- 8\frac{(p^2)^3(\eta_{\mu\nu}\eta_{\kappa\lambda}+\eta_{\mu\kappa}\eta_{\nu\lambda}+\eta_{\mu\lambda}\eta_{\nu\kappa}) }{24\left[p^2+(c-d)^2x(1-x) \right]^3} x\biggr\} \\
=&i\Omega_4\int^1_0 dx \left\{ \frac{1}{4}(\eta_{\nu\kappa}\eta_{\mu\lambda}+\eta_{\nu\lambda}\eta_{\mu\kappa})-\frac{1}{3}(\eta_{\mu\nu}\eta_{\kappa\lambda}+\eta_{\mu\kappa}\eta_{\nu\lambda}+\eta_{\mu\lambda}\eta_{\nu\kappa})x \right\} \\
=&2\pi^2 i \left\{ \frac{1}{4}(\eta_{\nu\kappa}\eta_{\mu\lambda}+\eta_{\nu\lambda}\eta_{\mu\kappa})-\frac{1}{6}(\eta_{\mu\nu}\eta_{\kappa\lambda}+\eta_{\mu\kappa}\eta_{\nu\lambda}+\eta_{\mu\lambda}\eta_{\nu\kappa}) \right\} \\
=&i\pi^2\left\{ \frac{1}{6}\eta_{\nu\kappa}\eta_{\mu\lambda}+\frac{1}{6}\eta_{\nu\lambda}\eta_{\mu\kappa} -\frac{1}{3}\eta_{\mu\nu}\eta_{\kappa\lambda}\right\}
\end{align*}
となる.したがって
\begin{align*}
\frac{1}{2} k^\mu k^\nu \int d^4p \frac{\partial^2 f_{\kappa\lambda}(p,c,d)}{\partial p^\mu \partial p^\nu}=&\frac{1}{2}\left\{ \frac{1}{6}k_\kappa k_\lambda +\frac{1}{6}k_\lambda k_\kappa -\frac{1}{3}\eta_{\kappa\lambda}k^2 \right\} \\
=&\frac{1}{6}i\pi^2\left\{k_\kappa k_\lambda -\eta_{\kappa\lambda}k^2\right\}
\end{align*}
となる.この結果から分かる通り(22.3.19)は誤植を含んでいて,正しい結果は
\begin{align*}
I_{\kappa\lambda}(k,c,d)=\frac{1}{6}i\pi^2\left[k_\kappa k_\lambda +2k_\lambda c_\kappa +2k_\kappa d_\lambda -k_\lambda d_\kappa -k_\kappa c_\lambda -\eta_{\kappa\lambda} k\cdot (k+c+d) \right]
\end{align*}
となる…と思われる.\par

\vskip\baselineskip

(追記)この計算は以下のようにして得られることが分かった.\par
$f$には$f_{\kappa\lambda}(p;c,d)=f_{\lambda\kappa}(p;d,c)$という対称性があることに注意すると$I_{\kappa\lambda}$は以下の形に限られる.
\begin{align*}
I_{\kappa\lambda}(k;c,d)=A(k_\lambda c_\kappa+k_\kappa d_\lambda)+B(k_\lambda d_\kappa +k_\kappa c_\lambda)+C \eta_{\kappa\lambda} k\cdot (c+d)+D\eta_{\kappa\lambda} k^2 +E k_\kappa k_\lambda
\end{align*}
Stokesの定理により
\begin{align*}
\int d^4p \frac{\partial f_{\kappa\lambda}(p,c,d)}{\partial p^\mu}=&i\lim_{p\to \infty}\int d\Omega_4 p^2p_\mu f_{\kappa\lambda}(p,c,d) \\
=&i 2\pi^2 \lim_{p\to \infty} p_\mu p^2 f^o_{\kappa\lambda}(p;c,d) \\
\int d^4p \frac{\partial^2 f_{\kappa\lambda}(p,c,d)}{\partial p^\mu \partial p^\nu}=&i\lim_{p\to \infty}\int d\Omega_4 p^2p_\mu \frac{\partial f_{\kappa\lambda}(p,c,d)}{\partial p^\nu} \\
=&i2\pi^2 \lim_{p\to \infty } p_\mu p^2  \frac{\partial f^e_{\kappa\lambda}(p,c,d)}{\partial p^\nu}
\end{align*}
となる.ここで$f^o,f^e$は$f$の$p\to -p$による反対称(odd)部分と対称(even)部分である.これは角度平均をとる際に(11.2.8)(11.2.9)の処方を用いると,$p$について偶数次のみの項が残ることに拠る.\par
まず,$f_{\kappa\lambda}(p;c,d)$の反対称部分を計算して,$A,B,C$を求める.簡単のため$d=0$とすると
\begin{align*}
\frac{1}{2} \left( \frac{(p+c)_\kappa p_\lambda }{(p+c)^2 p^2}- \frac{(-p+c)_\kappa (-p)_\lambda}{(-p+c)^2 p^2} \right)=&\frac{(-p+c)^2(p+c)_\kappa p_\lambda +(p+c)^2 (-p+c)_\kappa p_\lambda}{2p^2 (p+c)^2 (-p+c)^2} \\
=&p_\lambda \frac{[(-p+c)^2-(p+c)^2]p_\kappa+[(-p+c)^2+(p+c)^2]c_\kappa}{2p^2 (p+c)^2 (-p+c)^2} \\
=&p_\lambda \frac{-2(p\cdot c)p_\kappa +(p^2+c^2) c_\kappa}{p^2 (p+c)^2 (-p+c)^2}
\end{align*}
であるから
\begin{align*}
\int d^4p \frac{\partial f_{\kappa\lambda}(p,c,0)}{\partial p^\mu}=&i 2\pi^2 \lim_{p\to \infty} p_\mu p^2 f^o_{\kappa\lambda}(p;c,0) \\
=&i2\pi^2 \lim_{p\to \infty }p_\mu p_\lambda \frac{-2(p\cdot c)p_\kappa +(p^2+c^2) c_\kappa}{(p+c)^2 (-p+c)^2} \\
=&i2\pi^2 \lim_{p\to \infty }\left( \frac{-2p_\mu p_\lambda p_\kappa p_\rho}{p^4}c^\rho + \frac{p_\mu p_\lambda}{p^2}c_\kappa \right) \\
=& i2\pi^2 \left( -\frac{1}{12}(\eta_{\mu\lambda}\eta_{\kappa\rho}+\eta_{\mu\kappa}\eta_{\lambda\rho}+\eta_{\mu\rho}\eta_{\kappa\lambda})c^\rho +\frac{1}{4} \eta_{\mu\lambda}c_\kappa \right) \\
=& i\pi^2 \left( -\frac{1}{6}(\eta_{\mu\kappa}\eta_{\lambda\rho}+\eta_{\mu\rho}\eta_{\kappa\lambda})c^\rho +\frac{1}{3} \eta_{\mu\lambda}c_\kappa \right)  \\
\therefore \quad k^\mu \int d^4p \frac{\partial f_{\kappa\lambda}(p;c,0)}{\partial p^\mu}=&-\frac{i\pi^2 }{6}(k_\kappa c_\lambda+\eta_{\kappa\lambda}k\cdot c)+\frac{i\pi^2}{3} k_\lambda c_\kappa
\end{align*}
したがって$A=-i\pi^2/6,B=C=i\pi^2/3$を得る.\par
次に,$f_{\kappa\lambda}(p;c.d)$の対称部分を計算して,$D,E$を求める.これらは$c,d$に依らないので$c=d=0$としてよい.このとき,$f$は偶関数となるから
\begin{align*}
\frac{\partial }{\partial p^\nu}f^e_{\kappa\lambda}(p;0,0)=&\frac{\partial}{\partial p^\nu} \frac{p_\kappa p_\lambda }{p^4} \\
=&-4\frac{p_\nu p_\kappa p_\lambda}{p^6}+\eta_{\nu\kappa}\frac{p_\lambda}{p^4}+\eta_{\nu\lambda}\frac{p_\kappa}{p^4} \\
\int d^4p \frac{\partial^2 f_{\kappa\lambda}(p,0,0)}{\partial p^\mu \partial p^\nu}=&i2\pi^2 \lim_{p\to \infty } p_\mu p^2  \frac{\partial f^e_{\kappa\lambda}(p,0,0)}{\partial p^\nu} \\
=&i2\pi^2\lim_{p\to \infty } \left[-4\frac{p_\mu p_\nu p_\kappa p_\lambda}{p^4}+\eta_{\nu\kappa}\frac{p_\mu p_\lambda}{p^2}+\eta_{\nu\lambda}\frac{p_\mu p_\kappa}{p^2} \right] \\
=&i2\pi^2 \left[-\frac{1}{6}(\eta_{\mu\nu}\eta_{\kappa\lambda}+\eta_{\mu\kappa}\eta_{\nu\lambda}+\eta_{\mu\lambda}\eta_{\nu\kappa})+\frac{1}{4}\eta_{\nu\kappa}\eta_{\mu\lambda}+\frac{1}{4}\eta_{\nu\lambda}\eta_{\mu\kappa}\right] \\
=&\frac{i\pi^2}{6}(\eta_{\nu\kappa}\eta_{\mu\lambda}+\eta_{\nu\lambda}\eta_{\mu\kappa})-\frac{i\pi^2}{3}\eta_{\mu\nu}\eta_{\kappa\lambda} \\
\therefore \quad \frac{1}{2}k^\mu k^\nu \int d^4p \frac{\partial^2 f_{\kappa\lambda}(p,c,d)}{\partial p^\mu \partial p^\nu}=&\frac{i\pi^2}{3}(-\eta_{\kappa\lambda}k^2+k_\kappa k_\lambda)
\end{align*}
したがって$D=-i\pi^2/3,E=i\pi^2/3$を得る.\par
以上より
\begin{align*}
I_{\kappa\lambda}(k;c,d)=\frac{1}{6}i\pi^2\left[k_\kappa k_\lambda +2k_\lambda c_\kappa +2k_\kappa d_\lambda -k_\lambda d_\kappa -k_\kappa c_\lambda -\eta_{\kappa\lambda} k\cdot (k+c+d) \right]
\end{align*}
となる.これが示したかった.


\vskip\baselineskip

さて,(22.3.15)において射影行列$\frac{1}{2}(1+\gamma_5)$の$1$と$\gamma_5$のトレースの項を別個に考察しなければならない.$1$から発生する項は,$\mr{tr}[\gamma^\kappa \gamma^\nu \gamma^\lambda \gamma^\rho]$を含む.これは(8.A.6)により$\kappa$と$\lambda$について,また$\nu$と$\rho$について対称だ.したがって積分は以下の組み合わせで表れる.
\begin{align*}
&\mr{tr}\left[\gamma^\kappa \gamma^\nu \gamma^\lambda \gamma^\rho \right] I_{\kappa\lambda}(a-b-k_1,b,b+k_1)  \\
&+\mr{tr}\left[\gamma^\kappa \gamma^\rho \gamma^\lambda \gamma^\nu \right] I_{\kappa\lambda}(b-a-k_2,a,a+k_2) \\
=&\frac{1}{2}\mr{tr}\left[\gamma^\kappa \gamma^\nu \gamma^\lambda \gamma^\rho \right]\left\{ I_{\kappa\lambda}(a-b-k_1,b,b+k_1)+I_{\lambda\kappa }(a-b-k_1,b,b+k_1)\right\} \\
&+\frac{1}{2}\mr{tr}\left[\gamma^\kappa \gamma^\rho \gamma^\lambda \gamma^\nu \right]\left\{ I_{\kappa\lambda}(b-a-k_2,a,a+k_2)+ I_{\lambda\kappa}(b-a-k_2,a,a+k_2)\right\} \\
=&\frac{1}{2}\mr{tr}\left[\gamma^\kappa \gamma^\nu \gamma^\lambda \gamma^\rho \right]\bigl\{ I_{\kappa\lambda}(a-b-k_1,b,b+k_1)+I_{\lambda\kappa }(a-b-k_1,b,b+k_1) \\
&\qquad +I_{\kappa\lambda}(b-a-k_2,a,a+k_2)+ I_{\lambda\kappa}(b-a-k_2,a,a+k_2)\bigr\} \\
\Rightarrow \quad & I_{\kappa\lambda}(a-b-k_1,b,b+k_1)+I_{\lambda\kappa }(a-b-k_1,b,b+k_1) \\
&\quad +I_{\kappa\lambda}(b-a-k_2,a,a+k_2)+ I_{\lambda\kappa}(b-a-k_2,a,a+k_2)
\end{align*}
(22.3.19)を使うと,
\begin{align*}
&I_{\kappa\lambda}(k,c,d)+I_{\lambda\kappa}(k,c,d) \\
=&\frac{1}{6}i\pi^2\left[k_\kappa k_\lambda +2k_\lambda c_\kappa +2k_\kappa d_\lambda -k_\lambda d_\kappa -k_\kappa c_\lambda -\eta_{\kappa\lambda} k\cdot (k+c+d) \right] \\
&+\frac{1}{6}i\pi^2\left[k_\lambda k_\kappa +2k_\kappa c_\lambda +2k_\lambda d_\kappa -k_\kappa d_\lambda -k_\lambda c_\kappa -\eta_{\kappa\lambda} k\cdot (k+c+d) \right] \\
=&\frac{1}{6}i\pi^2\left[2k_\kappa k_\lambda +k_\lambda (c_\kappa+d_\kappa) + k_\kappa(c_\lambda +d_\lambda)-2\eta_{\kappa\lambda}k\cdot(k+c+d) \right] \\
=&\frac{1}{6}i\pi^2\left[k_\lambda (k_\kappa+c_\kappa+d_\kappa) + k_\kappa(k_\lambda +c_\lambda +d_\lambda)-2\eta_{\kappa\lambda}k\cdot(k+c+d) \right]
\end{align*}
であるから
\begin{align*}
&I_{\kappa\lambda}(a-b-k_1,b,b+k_1)+I_{\lambda\kappa }(a-b-k_1,b,b+k_1) \\
& \quad +I_{\kappa\lambda}(b-a-k_2,a,a+k_2)+ I_{\lambda\kappa}(b-a-k_2,a,a+k_2) \\
=&\frac{1}{6}i\pi^2\left[(a-b-k_1)_\lambda (a+b)_\kappa + (a-b-k_1)_\kappa(a+b)_\lambda-2\eta_{\kappa\lambda}(a-b-k_1)\cdot(a+b) \right] \\
+&\frac{1}{6}i\pi^2\left[(b-a-k_2)_\lambda (a+b)_\kappa + (b-a-k_2)_\kappa(a+b)_\lambda-2\eta_{\kappa\lambda}(b-a-k_2)\cdot(a+b) \right] 
\end{align*}
となる.これが消えるのは任意の定数ベクトルを
\begin{align*}
a=-b
\end{align*}
と選んだときのみであることがわかる.このように選ぶと3つ全部のカレントについて非カイラル的なアノマリーを避けることができる.これは(22.3.11)の導出まで戻って,今度は$y^\nu$で微分すると
\begin{align*}
\Slash{k}_1=(\Slash{p}+\Slash{a})-(\Slash{p}-\Slash{k}_1+\Slash{a})=(\Slash{p}+\Slash{k}_1+\Slash{b})-(\Slash{p}+\Slash{b})
\end{align*}
を用いて
\begin{align*}
&\frac{\partial}{\partial y^\nu}\Gamma^{\mu\nu\rho}_{\alpha\beta\gamma}(x,y,z) \\
=&\frac{-1}{(2\pi)^{12}}\int d^4k_1 d^4k_2 \, k_{1\nu} e^{-i(k_1+k_2)\cdot x}e^{ik_1\cdot y}e^{ik_2\cdot z}\int d^4p \\
&\times \biggl\{\mr{tr}\left[\frac{\Slash{p}-\Slash{k}_1+\Slash{a}}{(p-k_1+a)^2-i\epsilon}\gamma^\nu \frac{\Slash{p}+\Slash{a}}{(p+a)^2-i\epsilon}\gamma^\rho \frac{\Slash{p}+\Slash{k}_2+\Slash{a}}{(p+k_2+a)^2-i\epsilon}\gamma^\mu \frac{1+\gamma_5}{2}\right]\mr{tr}[T_\beta T_\gamma T_\alpha] \\
&+\mr{tr}\left[\frac{\Slash{p}-\Slash{k}_2+\Slash{b}}{(p-k_2+b)^2-i\epsilon}\gamma^\rho \frac{\Slash{p}+\Slash{b}}{(p+b)^2-i\epsilon}\gamma^\nu \frac{\Slash{p}+\Slash{k}_1+\Slash{b}}{(p+k_1+b)^2-i\epsilon}\gamma^\mu \frac{1+\gamma_5}{2}\right]\mr{tr}[T_\gamma T_\beta T_\alpha]  \biggr\} \\
=&\frac{-1}{(2\pi)^{12}}\int d^4k_1 d^4k_2e^{-i(k_1+k_2)\cdot x}e^{ik_1\cdot y}e^{ik_2\cdot z}\int d^4p \\
&\times \biggl\{\mr{tr}\left[\frac{\Slash{p}-\Slash{k}_1+\Slash{a}}{(p-k_1+a)^2-i\epsilon} \Slash{k}_1 \frac{\Slash{p}+\Slash{a}}{(p+a)^2-i\epsilon}\gamma^\rho \frac{\Slash{p}+\Slash{k}_2+\Slash{a}}{(p+k_2+a)^2-i\epsilon}\gamma^\mu \frac{1+\gamma_5}{2}\right]\mr{tr}[T_\beta T_\gamma T_\alpha] \\
&+\mr{tr}\left[\frac{\Slash{p}-\Slash{k}_2+\Slash{b}}{(p-k_2+b)^2-i\epsilon}\gamma^\rho \frac{\Slash{p}+\Slash{b}}{(p+b)^2-i\epsilon}\Slash{k}_1 \frac{\Slash{p}+\Slash{k}_1+\Slash{b}}{(p+k_1+b)^2-i\epsilon}\gamma^\mu \frac{1+\gamma_5}{2}\right]\mr{tr}[T_\gamma T_\beta T_\alpha]  \biggr\} \\
=&\frac{-1}{(2\pi)^{12}}\int d^4k_1 d^4k_2e^{-i(k_1+k_2)\cdot x}e^{ik_1\cdot y}e^{ik_2\cdot z}\int d^4p \\
&\times \biggl\{\mr{tr}\left[\frac{\Slash{p}-\Slash{k}_1+\Slash{a}}{(p-k_1+a)^2-i\epsilon}(\Slash{p}+\Slash{a}) \frac{\Slash{p}+\Slash{a}}{(p+a)^2-i\epsilon}\gamma^\rho \frac{\Slash{p}+\Slash{k}_2+\Slash{a}}{(p+k_2+a)^2-i\epsilon}\gamma^\mu \frac{1+\gamma_5}{2}\right]\mr{tr}[T_\beta T_\gamma T_\alpha] \\
&-\mr{tr}\left[\frac{\Slash{p}-\Slash{k}_1+\Slash{a}}{(p-k_1+a)^2-i\epsilon}(\Slash{p}-\Slash{k}_1+\Slash{a}) \frac{\Slash{p}+\Slash{a}}{(p+a)^2-i\epsilon}\gamma^\rho \frac{\Slash{p}+\Slash{k}_2+\Slash{a}}{(p+k_2+a)^2-i\epsilon}\gamma^\mu \frac{1+\gamma_5}{2}\right]\mr{tr}[T_\beta T_\gamma T_\alpha] \\
&+\mr{tr}\left[\frac{\Slash{p}-\Slash{k}_2+\Slash{b}}{(p-k_2+b)^2-i\epsilon}\gamma^\rho \frac{\Slash{p}+\Slash{b}}{(p+b)^2-i\epsilon}(\Slash{p}+\Slash{k}_1+\Slash{b}) \frac{\Slash{p}+\Slash{k}_1+\Slash{b}}{(p+k_1+b)^2-i\epsilon}\gamma^\mu \frac{1+\gamma_5}{2}\right]\mr{tr}[T_\gamma T_\beta T_\alpha] \\
&-\mr{tr}\left[\frac{\Slash{p}-\Slash{k}_2+\Slash{b}}{(p-k_2+b)^2-i\epsilon}\gamma^\rho \frac{\Slash{p}+\Slash{b}}{(p+b)^2-i\epsilon}(\Slash{p}+\Slash{b}) \frac{\Slash{p}+\Slash{k}_1+\Slash{b}}{(p+k_1+b)^2-i\epsilon}\gamma^\mu \frac{1+\gamma_5}{2}\right]\mr{tr}[T_\gamma T_\beta T_\alpha]  \biggr\} \\
=&\frac{-1}{(2\pi)^{12}}\int d^4k_1 d^4k_2e^{-i(k_1+k_2)\cdot x}e^{ik_1\cdot y}e^{ik_2\cdot z}\int d^4p \\
&\times \biggl\{\mr{tr}\left[\frac{\Slash{p}-\Slash{k}_1+\Slash{a}}{(p-k_1+a)^2-i\epsilon}\gamma^\rho \frac{\Slash{p}+\Slash{k}_2+\Slash{a}}{(p+k_2+a)^2-i\epsilon}\gamma^\mu \frac{1+\gamma_5}{2}\right]\mr{tr}[T_\beta T_\gamma T_\alpha] \\
&-\mr{tr}\left[ \frac{\Slash{p}+\Slash{a}}{(p+a)^2-i\epsilon}\gamma^\rho \frac{\Slash{p}+\Slash{k}_2+\Slash{a}}{(p+k_2+a)^2-i\epsilon}\gamma^\mu \frac{1+\gamma_5}{2}\right]\mr{tr}[T_\beta T_\gamma T_\alpha] \\
&+\mr{tr}\left[\frac{\Slash{p}-\Slash{k}_2+\Slash{b}}{(p-k_2+b)^2-i\epsilon}\gamma^\rho \frac{\Slash{p}+\Slash{b}}{(p+b)^2-i\epsilon}\gamma^\mu \frac{1+\gamma_5}{2}\right]\mr{tr}[T_\gamma T_\beta T_\alpha] \\
&-\mr{tr}\left[\frac{\Slash{p}-\Slash{k}_2+\Slash{b}}{(p-k_2+b)^2-i\epsilon}\gamma^\rho \frac{\Slash{p}+\Slash{k}_1+\Slash{b}}{(p+k_1+b)^2-i\epsilon}\gamma^\mu \frac{1+\gamma_5}{2}\right]\mr{tr}[T_\gamma T_\beta T_\alpha]  \biggr\} \\
=&\frac{-1}{(2\pi)^{12}}\int d^4k_1 d^4k_2e^{-i(k_1+k_2)\cdot x}e^{ik_1\cdot y}e^{ik_2\cdot z}\int d^4p \\
&\times \biggl\{\mr{tr}\left[ \frac{\Slash{p}+\Slash{k}_2+\Slash{a}}{(p+k_2+a)^2-i\epsilon}\gamma^\mu \frac{\Slash{p}-\Slash{k}_1+\Slash{a}}{(p-k_1+a)^2-i\epsilon}\gamma^\rho \frac{1+\gamma_5}{2}\right]\mr{tr}[T_\beta T_\gamma T_\alpha] \\
&-\mr{tr}\left[ \frac{\Slash{p}+\Slash{a}}{(p+a)^2-i\epsilon}\gamma^\rho \frac{\Slash{p}+\Slash{k}_2+\Slash{a}}{(p+k_2+a)^2-i\epsilon}\gamma^\mu \frac{1+\gamma_5}{2}\right]\mr{tr}[T_\beta T_\gamma T_\alpha] \\
&+\mr{tr}\left[ \frac{\Slash{p}-\Slash{k}_2+\Slash{b}}{(p-k_2+b)^2-i\epsilon}\gamma^\rho \frac{\Slash{p}+\Slash{b}}{(p+b)^2-i\epsilon}\gamma^\mu \frac{1+\gamma_5}{2}\right]\mr{tr}[T_\gamma T_\beta T_\alpha] \\
&-\mr{tr}\left[ \frac{\Slash{p}+\Slash{k}_1+\Slash{b}}{(p+k_1+b)^2-i\epsilon}\gamma^\mu \frac{\Slash{p}-\Slash{k}_2+\Slash{b}}{(p-k_2+b)^2-i\epsilon}\gamma^\rho \frac{1+\gamma_5}{2}\right]\mr{tr}[T_\gamma T_\beta T_\alpha]  \biggr\} 
\end{align*}
となってアノマリー部分は
\begin{align*}
&\left[ \frac{\partial}{\partial y^\nu}\Gamma^{\mu\nu\rho}_{\alpha\beta\gamma}(x,y,z) \right]_{\mr{anom}} \\
=&\frac{-1}{(2\pi)^{12}}D_{\alpha\beta\gamma}\int d^4k_1 d^4k_2e^{-i(k_1+k_2)\cdot x}e^{ik_1\cdot y}e^{ik_2\cdot z}\int d^4p \\
&\times \biggl\{\mr{tr}\left[ \frac{\Slash{p}+\Slash{k}_2+\Slash{a}}{(p+k_2+a)^2-i\epsilon}\gamma^\mu \frac{\Slash{p}-\Slash{k}_1+\Slash{a}}{(p-k_1+a)^2-i\epsilon}\gamma^\rho \frac{1+\gamma_5}{2}\right] \\
&-\mr{tr}\left[ \frac{\Slash{p}+\Slash{a}}{(p+a)^2-i\epsilon}\gamma^\rho \frac{\Slash{p}+\Slash{k}_2+\Slash{a}}{(p+k_2+a)^2-i\epsilon}\gamma^\mu \frac{1+\gamma_5}{2}\right] \\
&+\mr{tr}\left[ \frac{\Slash{p}-\Slash{k}_2+\Slash{b}}{(p-k_2+b)^2-i\epsilon}\gamma^\rho \frac{\Slash{p}+\Slash{b}}{(p+b)^2-i\epsilon}\gamma^\mu \frac{1+\gamma_5}{2}\right] \\
&-\mr{tr}\left[ \frac{\Slash{p}+\Slash{k}_1+\Slash{b}}{(p+k_1+b)^2-i\epsilon}\gamma^\mu \frac{\Slash{p}-\Slash{k}_2+\Slash{b}}{(p-k_2+b)^2-i\epsilon}\gamma^\rho \frac{1+\gamma_5}{2}\right] \biggr\} \\
=&\frac{-1}{(2\pi)^{12}}D_{\alpha\beta\gamma}\int d^4k_1 d^4k_2e^{-i(k_1+k_2)\cdot x}e^{ik_1\cdot y}e^{ik_2\cdot z} \\
&\times \int d^4p \biggl\{\mr{tr}\left[\gamma^\kappa \gamma^\mu \gamma^\lambda \gamma^\rho \frac{1+\gamma_5}{2}\right] \\
&\qquad \qquad \times \left[ \frac{(p+k_2+a)_\kappa}{(p+k_2+a)^2-i\epsilon} \frac{(p-k_1+a)_\lambda }{(p-k_1+a)^2-i\epsilon}-\frac{(p+k_1+b)_\kappa}{(p+k_1+b)^2-i\epsilon} \frac{(p-k_2+b)_\lambda }{(p-k_2+b)^2-i\epsilon} \right]  \\
&\qquad \qquad + \mr{tr}\left[\gamma^\kappa \gamma^\rho \gamma^\lambda \gamma^\mu \frac{1+\gamma_5}{2}\right] \\
&\qquad \qquad \times \left[ \frac{(p-k_2+b)_\kappa}{(p-k_2+b)^2-i\epsilon} \frac{(p+b)_\lambda}{(p+b)^2-i\epsilon}-\frac{(p+a)_\kappa}{(p+a)^2-i\epsilon}\frac{(p+k_2+a)_\lambda}{(p+k_2+a)^2-i\epsilon} \right] \biggr\} \\
=&\frac{-1}{(2\pi)^{12}}D_{\alpha\beta\gamma}\int d^4k_1 d^4k_2e^{-i(k_1+k_2)\cdot x}e^{ik_1\cdot y}e^{ik_2\cdot z} \\
&\times \int d^4p \biggl\{\mr{tr}\left[\gamma^\kappa \gamma^\mu \gamma^\lambda \gamma^\rho \frac{1+\gamma_5}{2}\right] \\
&\times \biggl[ \frac{(p+(a-b-k_1+k_2)+b+k_1)_\kappa}{(p+(a-b-k_1+k_2)+b+k_1)^2-i\epsilon} \frac{(p+(a-b-k_1+k_2)+b-k_2)_\lambda }{(p+(a-b-k_1+k_2)+b-k_2)^2-i\epsilon} \\
&\qquad -\frac{(p+b+k_1)_\kappa}{(p+b+k_1)^2-i\epsilon} \frac{(p+b-k_2)_\lambda }{(p+b-k_2)^2-i\epsilon} \biggr]  \\
&+ \mr{tr}\left[\gamma^\kappa \gamma^\rho \gamma^\lambda \gamma^\mu \frac{1+\gamma_5}{2}\right] \\
&\times \biggl[ \frac{(p+(b-a-k_2)+a)_\kappa}{(p+(b-a-k_2)+a)^2-i\epsilon} \frac{(p+(b-a-k_2)+a+k_2)_\lambda}{(p+(b-a-k_2)+a+k_2)^2-i\epsilon} \\
&\qquad -\frac{(p+a)_\kappa}{(p+a)^2-i\epsilon}\frac{(p+a+k_2)_\lambda}{(p+a+k_2)^2-i\epsilon} \biggr] \biggr\} \\
=&\frac{-1}{(2\pi)^{12}}D_{\alpha\beta\gamma}\int d^4k_1 d^4k_2e^{-i(k_1+k_2)\cdot x}e^{ik_1\cdot y}e^{ik_2\cdot z} \\
&\times \int d^4p\biggl\{\mr{tr}\left[\gamma^\kappa \gamma^\mu \gamma^\lambda \gamma^\rho \frac{1+\gamma_5}{2}\right][f_{\kappa\lambda}(p+(a-b-k_1+k_2),b+k_1,b-k_2)-f_{\kappa\lambda}(p,b+k_1,b-k_2)] \\
&\qquad +\mr{tr}\left[\gamma^\kappa \gamma^\rho \gamma^\lambda \gamma^\mu \frac{1+\gamma_5}{2}\right][f_{\kappa\lambda}(p+(b-a-k_2),a,a+k_2)-f_{\kappa\lambda}(p,a,a+k_2)] \\
=&\frac{-1}{(2\pi)^{12}}D_{\alpha\beta\gamma}\int d^4k_1 d^4k_2e^{-i(k_1+k_2)\cdot x}e^{ik_1\cdot y}e^{ik_2\cdot z} \\
&\times \biggl\{ \mr{tr}\left[\gamma^\kappa \gamma^\mu \gamma^\lambda \gamma^\rho \frac{1+\gamma_5}{2}\right] I_{\kappa\lambda}(a-b-k_1+k_2,b+k_1,b-k_2) \\
&\qquad +\mr{tr}\left[\gamma^\kappa \gamma^\rho \gamma^\lambda \gamma^\mu \frac{1+\gamma_5}{2}\right]I_{\kappa\lambda}(b-a-k_2,a,a+k_2) \biggr\}
\end{align*}
となる.これは(22.3.15)における第一項目で$a\to a'=a-k_1,b\to b'=b+k_1,k_1\to k'_1=-k_1-k_2 $と置き換えたものと等しい.$a=-b$を仮定すれば$a'=-b'$も保証され,非カイラル項は打ち消し合わされる.$z^\rho$で微分した場合も同様に
\begin{align*}
&\frac{\partial}{\partial z^\rho}\Gamma^{\mu\nu\rho}_{\alpha\beta\gamma}(x,y,z) \\
=&\frac{-1}{(2\pi)^{12}}\int d^4k_1 d^4k_2 \, k_{2\rho} e^{-i(k_1+k_2)\cdot x}e^{ik_1\cdot y}e^{ik_2\cdot z}\int d^4p \\
&\times \biggl\{\mr{tr}\left[\frac{\Slash{p}-\Slash{k}_1+\Slash{a}}{(p-k_1+a)^2-i\epsilon}\gamma^\nu \frac{\Slash{p}+\Slash{a}}{(p+a)^2-i\epsilon}\gamma^\rho \frac{\Slash{p}+\Slash{k}_2+\Slash{a}}{(p+k_2+a)^2-i\epsilon}\gamma^\mu \frac{1+\gamma_5}{2}\right]\mr{tr}[T_\beta T_\gamma T_\alpha] \\
&+\mr{tr}\left[\frac{\Slash{p}-\Slash{k}_2+\Slash{b}}{(p-k_2+b)^2-i\epsilon}\gamma^\rho \frac{\Slash{p}+\Slash{b}}{(p+b)^2-i\epsilon}\gamma^\nu \frac{\Slash{p}+\Slash{k}_1+\Slash{b}}{(p+k_1+b)^2-i\epsilon}\gamma^\mu \frac{1+\gamma_5}{2}\right]\mr{tr}[T_\gamma T_\beta T_\alpha]  \biggr\} \\
=&\frac{-1}{(2\pi)^{12}}\int d^4k_1 d^4k_2\, e^{-i(k_1+k_2)\cdot x}e^{ik_1\cdot y}e^{ik_2\cdot z}\int d^4p \\
&\times \biggl\{\mr{tr}\left[\frac{\Slash{p}-\Slash{k}_1+\Slash{a}}{(p-k_1+a)^2-i\epsilon}\gamma^\nu \frac{\Slash{p}+\Slash{a}}{(p+a)^2-i\epsilon}\Slash{k}_2 \frac{\Slash{p}+\Slash{k}_2+\Slash{a}}{(p+k_2+a)^2-i\epsilon}\gamma^\mu \frac{1+\gamma_5}{2}\right]\mr{tr}[T_\beta T_\gamma T_\alpha] \\
&+\mr{tr}\left[\frac{\Slash{p}-\Slash{k}_2+\Slash{b}}{(p-k_2+b)^2-i\epsilon}\Slash{k}_2 \frac{\Slash{p}+\Slash{b}}{(p+b)^2-i\epsilon}\gamma^\nu \frac{\Slash{p}+\Slash{k}_1+\Slash{b}}{(p+k_1+b)^2-i\epsilon}\gamma^\mu \frac{1+\gamma_5}{2}\right]\mr{tr}[T_\gamma T_\beta T_\alpha]  \biggr\} \\
=&\frac{-1}{(2\pi)^{12}}\int d^4k_1 d^4k_2\, e^{-i(k_1+k_2)\cdot x}e^{ik_1\cdot y}e^{ik_2\cdot z}\int d^4p \\
&\times \biggl\{\mr{tr}\left[\frac{\Slash{p}-\Slash{k}_1+\Slash{a}}{(p-k_1+a)^2-i\epsilon}\gamma^\nu \frac{\Slash{p}+\Slash{a}}{(p+a)^2-i\epsilon}(\Slash{p}+\Slash{k}_2+\Slash{a}) \frac{\Slash{p}+\Slash{k}_2+\Slash{a}}{(p+k_2+a)^2-i\epsilon}\gamma^\mu \frac{1+\gamma_5}{2}\right]\mr{tr}[T_\beta T_\gamma T_\alpha] \\
&-\mr{tr}\left[\frac{\Slash{p}-\Slash{k}_1+\Slash{a}}{(p-k_1+a)^2-i\epsilon}\gamma^\nu \frac{\Slash{p}+\Slash{a}}{(p+a)^2-i\epsilon}(\Slash{p}+\Slash{a}) \frac{\Slash{p}+\Slash{k}_2+\Slash{a}}{(p+k_2+a)^2-i\epsilon}\gamma^\mu \frac{1+\gamma_5}{2}\right]\mr{tr}[T_\beta T_\gamma T_\alpha] \\
&+\mr{tr}\left[\frac{\Slash{p}-\Slash{k}_2+\Slash{b}}{(p-k_2+b)^2-i\epsilon}(\Slash{p}+\Slash{k}_2) \frac{\Slash{p}+\Slash{b}}{(p+b)^2-i\epsilon}\gamma^\nu \frac{\Slash{p}+\Slash{k}_1+\Slash{b}}{(p+k_1+b)^2-i\epsilon}\gamma^\mu \frac{1+\gamma_5}{2}\right]\mr{tr}[T_\gamma T_\beta T_\alpha] \\
&-\mr{tr}\left[\frac{\Slash{p}-\Slash{k}_2+\Slash{b}}{(p-k_2+b)^2-i\epsilon}(\Slash{p}-\Slash{k}_2+\Slash{b}) \frac{\Slash{p}+\Slash{b}}{(p+b)^2-i\epsilon}\gamma^\nu \frac{\Slash{p}+\Slash{k}_1+\Slash{b}}{(p+k_1+b)^2-i\epsilon}\gamma^\mu \frac{1+\gamma_5}{2}\right]\mr{tr}[T_\gamma T_\beta T_\alpha]  \biggr\} \\
=&\frac{-1}{(2\pi)^{12}}\int d^4k_1 d^4k_2\, e^{-i(k_1+k_2)\cdot x}e^{ik_1\cdot y}e^{ik_2\cdot z}\int d^4p \\
&\times \biggl\{\mr{tr}\left[\frac{\Slash{p}-\Slash{k}_1+\Slash{a}}{(p-k_1+a)^2-i\epsilon}\gamma^\nu \frac{\Slash{p}+\Slash{a}}{(p+a)^2-i\epsilon}\gamma^\mu \frac{1+\gamma_5}{2}\right]\mr{tr}[T_\beta T_\gamma T_\alpha] \\
&-\mr{tr}\left[\frac{\Slash{p}-\Slash{k}_1+\Slash{a}}{(p-k_1+a)^2-i\epsilon}\gamma^\nu \frac{\Slash{p}+\Slash{k}_2+\Slash{a}}{(p+k_2+a)^2-i\epsilon}\gamma^\mu \frac{1+\gamma_5}{2}\right]\mr{tr}[T_\beta T_\gamma T_\alpha] \\
&+\mr{tr}\left[\frac{\Slash{p}-\Slash{k}_2+\Slash{b}}{(p-k_2+b)^2-i\epsilon} \gamma^\nu \frac{\Slash{p}+\Slash{k}_1+\Slash{b}}{(p+k_1+b)^2-i\epsilon}\gamma^\mu \frac{1+\gamma_5}{2}\right]\mr{tr}[T_\gamma T_\beta T_\alpha] \\
&-\mr{tr}\left[ \frac{\Slash{p}+\Slash{b}}{(p+b)^2-i\epsilon}\gamma^\nu \frac{\Slash{p}+\Slash{k}_1+\Slash{b}}{(p+k_1+b)^2-i\epsilon}\gamma^\mu \frac{1+\gamma_5}{2}\right]\mr{tr}[T_\gamma T_\beta T_\alpha]  \biggr\} \\
=&\frac{-1}{(2\pi)^{12}}\int d^4k_1 d^4k_2\, e^{-i(k_1+k_2)\cdot x}e^{ik_1\cdot y}e^{ik_2\cdot z}\int d^4p \\
&\times \biggl\{\mr{tr}\left[\frac{\Slash{p}-\Slash{k}_1+\Slash{a}}{(p-k_1+a)^2-i\epsilon}\gamma^\nu \frac{\Slash{p}+\Slash{a}}{(p+a)^2-i\epsilon}\gamma^\mu \frac{1+\gamma_5}{2}\right]\mr{tr}[T_\beta T_\gamma T_\alpha] \\
&-\mr{tr}\left[ \frac{\Slash{p}+\Slash{k}_2+\Slash{a}}{(p+k_2+a)^2-i\epsilon}\gamma^\mu \frac{\Slash{p}-\Slash{k}_1+\Slash{a}}{(p-k_1+a)^2-i\epsilon}\gamma^\nu \frac{1+\gamma_5}{2}\right]\mr{tr}[T_\beta T_\gamma T_\alpha] \\
&+\mr{tr}\left[ \frac{\Slash{p}+\Slash{k}_1+\Slash{b}}{(p+k_1+b)^2-i\epsilon}\gamma^\mu \frac{\Slash{p}-\Slash{k}_2+\Slash{b}}{(p-k_2+b)^2-i\epsilon} \gamma^\nu \frac{1+\gamma_5}{2}\right]\mr{tr}[T_\gamma T_\beta T_\alpha] \\
&-\mr{tr}\left[ \frac{\Slash{p}+\Slash{b}}{(p+b)^2-i\epsilon}\gamma^\nu \frac{\Slash{p}+\Slash{k}_1+\Slash{b}}{(p+k_1+b)^2-i\epsilon}\gamma^\mu \frac{1+\gamma_5}{2}\right]\mr{tr}[T_\gamma T_\beta T_\alpha]  \biggr\}
\end{align*}
これのアノマリー部分は
\begin{align*}
&\left[ \frac{\partial}{\partial y^\nu}\Gamma^{\mu\nu\rho}_{\alpha\beta\gamma}(x,y,z) \right]_{\mr{anom}} \\
=&\frac{-1}{(2\pi)^{12}}D_{\alpha\beta\gamma}\int d^4k_1 d^4k_2\, e^{-i(k_1+k_2)\cdot x}e^{ik_1\cdot y}e^{ik_2\cdot z}\int d^4p \\
&\times \biggl\{\mr{tr}\left[\frac{\Slash{p}-\Slash{k}_1+\Slash{a}}{(p-k_1+a)^2-i\epsilon}\gamma^\nu \frac{\Slash{p}+\Slash{a}}{(p+a)^2-i\epsilon}\gamma^\mu \frac{1+\gamma_5}{2}\right] \\
&-\mr{tr}\left[ \frac{\Slash{p}+\Slash{k}_2+\Slash{a}}{(p+k_2+a)^2-i\epsilon}\gamma^\mu \frac{\Slash{p}-\Slash{k}_1+\Slash{a}}{(p-k_1+a)^2-i\epsilon}\gamma^\nu \frac{1+\gamma_5}{2}\right] \\
&+\mr{tr}\left[ \frac{\Slash{p}+\Slash{k}_1+\Slash{b}}{(p+k_1+b)^2-i\epsilon}\gamma^\mu \frac{\Slash{p}-\Slash{k}_2+\Slash{b}}{(p-k_2+b)^2-i\epsilon} \gamma^\nu \frac{1+\gamma_5}{2}\right] \\
&-\mr{tr}\left[ \frac{\Slash{p}+\Slash{b}}{(p+b)^2-i\epsilon}\gamma^\nu \frac{\Slash{p}+\Slash{k}_1+\Slash{b}}{(p+k_1+b)^2-i\epsilon}\gamma^\mu \frac{1+\gamma_5}{2}\right] \biggr\} \\
=&\frac{-1}{(2\pi)^{12}}D_{\alpha\beta\gamma}\int d^4k_1 d^4k_2e^{-i(k_1+k_2)\cdot x}e^{ik_1\cdot y}e^{ik_2\cdot z} \\
&\times \int d^4p \biggl\{\mr{tr}\left[\gamma^\kappa \gamma^\nu \gamma^\lambda \gamma^\mu \frac{1+\gamma_5}{2}\right] \\
&\qquad \qquad \times \left[ \frac{(p-k_1+a)_\kappa}{(p-k_1+a)^2-i\epsilon} \frac{(p+a)_\lambda }{(p+a)^2-i\epsilon}-\frac{(p+b)_\kappa}{(p+b)^2-i\epsilon} \frac{(p+k_1+b)_\lambda }{(p+k_1+b)^2-i\epsilon} \right]  \\
&\qquad \qquad + \mr{tr}\left[\gamma^\kappa \gamma^\mu \gamma^\lambda \gamma^\nu \frac{1+\gamma_5}{2}\right] \\
&\qquad \qquad \times \left[ \frac{(p+k_1+b)_\kappa}{(p+k_1+b)^2-i\epsilon} \frac{(p-k_2+b)_\lambda}{(p-k_2+b)^2-i\epsilon}-\frac{(p+k_2+a)_\kappa}{(p+k_2+a)^2-i\epsilon}\frac{(p-k_1+a)_\lambda}{(p-k_1+a)^2-i\epsilon} \right] \biggr\} \\
=&\frac{-1}{(2\pi)^{12}}D_{\alpha\beta\gamma}\int d^4k_1 d^4k_2e^{-i(k_1+k_2)\cdot x}e^{ik_1\cdot y}e^{ik_2\cdot z} \\
&\times \int d^4p \biggl\{\mr{tr}\left[\gamma^\kappa \gamma^\nu \gamma^\lambda \gamma^\mu \frac{1+\gamma_5}{2}\right] \\
&\times \biggl[ \frac{(p+(a-b-k_1)+b)_\kappa}{(p+(a-b-k_1)+b)^2-i\epsilon} \frac{(p+(a-b-k_1)+b+k_1)_\lambda }{(p+(a-b-k_1)+b+k_1)^2-i\epsilon} \\
&\qquad -\frac{(p+b)_\kappa}{(p+b)^2-i\epsilon} \frac{(p+b+k_1)_\lambda }{(p+b+k_1)^2-i\epsilon} \biggr]  \\
&+ \mr{tr}\left[\gamma^\kappa \gamma^\mu \gamma^\lambda \gamma^\nu \frac{1+\gamma_5}{2}\right] \\
&\times \biggl[ \frac{(p+(b-a+k_1-k_2)+a+k_2)_\kappa}{(p+(b-a+k_1-k_2)+a+k_2)^2-i\epsilon} \frac{(p+(b-a+k_1-k_2)+a-k_1)_\lambda}{(p+(b-a+k_1-k_2)+a-k_1)^2-i\epsilon} \\
&\qquad -\frac{(p+a+k_2)_\kappa}{(p+a+k_2)^2-i\epsilon}\frac{(p+a-k_1)_\lambda}{(p+a-k_1)^2-i\epsilon} \biggr] \biggr\} \\
=&\frac{-1}{(2\pi)^{12}}D_{\alpha\beta\gamma}\int d^4k_1 d^4k_2e^{-i(k_1+k_2)\cdot x}e^{ik_1\cdot y}e^{ik_2\cdot z} \\
&\times \int d^4p\biggl\{\mr{tr}\left[\gamma^\kappa \gamma^\nu \gamma^\lambda \gamma^\mu \frac{1+\gamma_5}{2}\right][f_{\kappa\lambda}(p+(a-b-k_1),b,b+k_1)-f_{\kappa\lambda}(p,b,b+k_1)] \\
&\qquad +\mr{tr}\left[\gamma^\kappa \gamma^\mu \gamma^\lambda \gamma^\nu \frac{1+\gamma_5}{2}\right][f_{\kappa\lambda}(p+(b-a+k_1-k_2),a+k_2,a-k_1)-f_{\kappa\lambda}(p,a+k_2,a-k_1)] \\
=&\frac{-1}{(2\pi)^{12}}D_{\alpha\beta\gamma}\int d^4k_1 d^4k_2e^{-i(k_1+k_2)\cdot x}e^{ik_1\cdot y}e^{ik_2\cdot z} \\
&\times \biggl\{ \mr{tr}\left[\gamma^\kappa \gamma^\nu \gamma^\lambda \gamma^\mu \frac{1+\gamma_5}{2}\right] I_{\kappa\lambda}(a-b-k_1,b,b+k_1) \\
&\qquad +\mr{tr}\left[\gamma^\kappa \gamma^\mu \gamma^\lambda \gamma^\nu \frac{1+\gamma_5}{2}\right]I_{\kappa\lambda}(b-a+k_1-k_2,a+k_2,a-k_1) \biggr\}
\end{align*}
となるが,これは(22.3.15)の第二項目で$a\to a''=a+k_2,b\to b''=b-k_2,k''_2=-k_1-k_2$と置き換えたものと等しい.$a=-b$を仮定すれば$a''=-b''$も保証される.

\vskip\baselineskip

$\gamma_5$を含むトレース項が残った.これは(8.A.12)より
\begin{align*}
\mr{tr}\left[\gamma^\kappa \gamma^\nu \gamma^\lambda \gamma^\rho \gamma_5 \right]=-4i\epsilon^{\kappa\nu\lambda\rho}
\end{align*}
となって完全反対称だ.(マイナスは$\gamma_0=-\gamma^0$より生じる.$\kappa,\nu,\lambda,\rho$のどれか一つは0を取るからだ.)これと$b=-a$を(22.3.15)に使うと
\begin{align*}
&I_{\kappa\lambda}(a-b-k_1,b,b+k_1)=I_{\kappa\lambda}(2a-k_1,-a,-a+k_1) \\
=&\frac{1}{6}i\pi^2[ (2a-k_1)_\kappa (2a-k_1)_\lambda-2a_\kappa(2a-k_1)_\lambda  +2(2a-k_1)_\kappa (-a+k_1)_\lambda \\
&- (-a+k_1)_\kappa (2a-k_1)_\lambda + (2a-k_1)_\kappa a_\lambda  ] \\
=&\frac{1}{6}i\pi^2[-k_{1\kappa}(2a-k_1)_\lambda -(-a+k_1)_\kappa(2a-k_1)_\lambda +2(2a-k_1)_\kappa (-a+k_1)_\lambda +(2a-k_1)_\kappa a_\lambda ]\\
=&\frac{1}{6}i\pi^2[(a-2k_1)_\kappa(2a-k_1)_\lambda +(2a-k_1)_\kappa (-a+2k_1)_\lambda ]\\
=&\frac{1}{6}i\pi^2[-a_\kappa k_{1\lambda}-4k_{1\kappa}a_{\lambda}+k_{1\kappa}a_\lambda +4a_\kappa k_{1\lambda}] \\
=&\frac{1}{2}i\pi^2[a_\kappa k_{1\lambda}- k_{1\kappa}a_\lambda]
\end{align*}
と,対称性より
\begin{align*}
&I_{\kappa\lambda}(b-a-k_2,a,a+k_2)=I_{\kappa\lambda}(2b-k_2,-b,-b+k_2) \\
=&\frac{1}{2}i\pi^2[b_\kappa k_{2\lambda}- k_{2\kappa}b_\lambda] \\
=&\frac{1}{2}i\pi^2[-a_\kappa k_{2\lambda}+ k_{2\kappa}a_\lambda]
\end{align*}
を用いると
\begin{align*}
&\left[ \frac{\partial}{\partial x^\mu}\Gamma^{\mu\nu\rho}_{\alpha\beta\gamma}(x,y,z) \right]_{\mr{anom}} \\
=&\frac{-2i}{(2\pi)^{12}}D_{\alpha\beta\gamma}\int d^4k_1 d^4k_2e^{-i(k_1+k_2)\cdot x}e^{ik_1\cdot y}e^{ik_2\cdot z} \\
&\times \biggl\{ \epsilon^{\kappa\nu\lambda\rho} I_{\kappa\lambda}(a-b-k_1,b,b+k_1) +\epsilon^{\kappa\rho\lambda\nu} I_{\kappa\lambda}(b-a-k_2,a,a+k_2) \biggr\} \\
=&\frac{\pi^2}{2(2\pi)^{12}}D_{\alpha\beta\gamma}\int d^4k_1 d^4k_2e^{-i(k_1+k_2)\cdot x}e^{ik_1\cdot y}e^{ik_2\cdot z} \\
&\times \biggl\{ \epsilon^{\kappa\nu\lambda\rho} [a_\kappa k_{1\lambda}- k_{1\kappa}a_\lambda] +\epsilon^{\kappa\rho\lambda\nu} [-a_\kappa k_{2\lambda}+ k_{2\kappa}a_\lambda] \biggr\} \\
=&\frac{2}{(2\pi)^{12}}D_{\alpha\beta\gamma}\int d^4k_1 d^4k_2e^{-i(k_1+k_2)\cdot x}e^{ik_1\cdot y}e^{ik_2\cdot z}\pi^2\epsilon^{\kappa\nu\lambda\rho}a_{\kappa}(k_1+k_2)_\lambda
\end{align*}
となる.\par
$J^\mu_\alpha(x)$のカレントでは,$a\propto k_1+k_2$として$\epsilon^{\kappa\nu\lambda\rho}$の完全反対称によりアノマリー(22.3.22)を消去できる.これは可能だが,アノマリーが全て消えるわけではない.アノマリーは$(\partial/\partial y^\nu)\Gamma^{\mu\nu\rho}_{\alpha\beta\gamma}(x,y,z)$か$(\partial/\partial z^\rho)\Gamma^{\mu\nu\rho}_{\alpha\beta\gamma}(x,y,z)$に現れる.以前見たように,$(\partial/\partial y^\nu)\Gamma^{\mu\nu\rho}_{\alpha\beta\gamma}(x,y,z)$では(22.3.15)の第一項目の$I_{\kappa\lambda}(a-b-k_1,b,b+k_1)$で$a\to a'=a-k_1,b\to b'=b+k_1,k_1\to k'_1=-k_1-k_2$と置き換えたものと等しいのであったから,
\begin{align*}
&I_{\kappa\lambda}(a'-b'-k_1,b',b'+k'_1)=I_{\kappa\lambda}(2a'-k_1,-a',-a'+k_1) \\
=&\frac{1}{2}i\pi^2[a'_\kappa k'_{1\lambda}- k'_{1\kappa}a'_\lambda] \\
=&\frac{1}{2}i\pi^2[(a-k_1)_\kappa (-k_1-k_2)_{1\lambda}- (-k_1-k_2)_{1\kappa}(a-k_1)_\lambda] \\
=&\frac{1}{2}i\pi^2[-(a-k_1)_\kappa (k_1+k_2)_{1\lambda}+(k_1+k_2)_{1\kappa}(a-k_1)_\lambda] \\
=&\frac{1}{2}i\pi^2[-a_\kappa (k_1+k_2)_{1\lambda}+k_{1\kappa}k_{2\lambda}+(k_1+k_2)_{1\kappa}a_\lambda-k_{2\kappa}k_{1\lambda}] \\
\end{align*}
を用いて
\begin{align*}
&\left[ \frac{\partial}{\partial y^\nu}\Gamma^{\mu\nu\rho}_{\alpha\beta\gamma}(x,y,z) \right]_{\mr{anom}} \\
=&\frac{2i}{(2\pi)^{12}}D_{\alpha\beta\gamma}\int d^4k_1 d^4k_2e^{-i(k_1+k_2)\cdot x}e^{ik_1\cdot y}e^{ik_2\cdot z} \\
&\times \biggl\{ \epsilon^{\kappa\mu\lambda\rho} I_{\kappa\lambda}(a'-b'-k'_1,b',b'+k'_1) +\epsilon^{\kappa\rho\lambda\mu} I_{\kappa\lambda}(b-a-k_2,a,a+k_2) \biggr\} \\
=&\frac{-\pi^2}{(2\pi)^{12}}D_{\alpha\beta\gamma}\int d^4k_1 d^4k_2e^{-i(k_1+k_2)\cdot x}e^{ik_1\cdot y}e^{ik_2\cdot z} \\
&\times \biggl\{ \epsilon^{\kappa\mu\lambda\rho} [-a_\kappa (k_1+k_2)_{1\lambda}+k_{1\kappa}k_{2\lambda}+(k_1+k_2)_{1\kappa}a_\lambda-k_{2\kappa}k_{1\lambda}] \\
& +\epsilon^{\kappa\rho\lambda\mu} [-a_\kappa k_{2\lambda}+ k_{2\kappa}a_\lambda] \biggr\} \\
=&\frac{-\pi^2}{(2\pi)^{12}}D_{\alpha\beta\gamma}\int d^4k_1 d^4k_2e^{-i(k_1+k_2)\cdot x}e^{ik_1\cdot y}e^{ik_2\cdot z} \\
&\times \biggl\{ \epsilon^{\kappa\mu\lambda\rho} [-2a_\kappa (k_1+k_2)_{1\lambda}+2k_{1\kappa}k_{2\lambda}] +2\epsilon^{\kappa\mu\lambda\rho} a_\kappa k_{2\lambda} \biggr\} \\
=&\frac{-2}{(2\pi)^{12}}D_{\alpha\beta\gamma}\int d^4k_1 d^4k_2e^{-i(k_1+k_2)\cdot x}e^{ik_1\cdot y}e^{ik_2\cdot z} \pi^2 \epsilon^{\kappa\rho\lambda\mu}(a+k_1)_\kappa k_{2\lambda}
\end{align*}
となるが,これが消えるのは$a+k_1\propto k_2$のときだ.同様に,$(\partial/\partial z^\rho)\Gamma^{\mu\nu\rho}_{\alpha\beta\gamma}(x,y,z)$の場合も,今度は(22.3.15)の第二項目で$a\to a''=a+k_2,b\to b''=b-k_2,k''_2=-k_1-k_2$と置き換えたものと等しいのであったから
\begin{align*}
&I_{\kappa\lambda}(b''-a''-k''_2,a'',a''+k''_2)=I_{\kappa\lambda}(2b''-k''_2,-b'',-b''+k''_2) \\
=&\frac{1}{2}i\pi^2[-a''_\kappa k''_{2\lambda}+ k''_{2\kappa}a''_\lambda] \\
=&\frac{1}{2}i\pi^2[-(a+k_2)_\kappa (-k_1-k_2)_{\lambda}+ (-k_1-k_2)_{\kappa}(a+k_2)_\lambda] \\
=&\frac{1}{2}i\pi^2[(a+k_2)_\kappa (k_1+k_2)_{\lambda}- (k_1+k_2)_{\kappa}(a+k_2)_\lambda] \\
=&\frac{1}{2}i\pi^2[a_\kappa (k_1+k_2)_{\lambda}+k_{2\kappa}k_{1\lambda}- (k_1+k_2)_{\kappa}a_\lambda-k_{1\kappa}k_{2\lambda}]
\end{align*}
を用いて
\begin{align*}
&\left[ \frac{\partial}{\partial z^\rho}\Gamma^{\mu\nu\rho}_{\alpha\beta\gamma}(x,y,z) \right]_{\mr{anom}} \\
=&\frac{2i}{(2\pi)^{12}}D_{\alpha\beta\gamma}\int d^4k_1 d^4k_2e^{-i(k_1+k_2)\cdot x}e^{ik_1\cdot y}e^{ik_2\cdot z} \\
&\times \biggl\{ \epsilon^{\kappa\nu\lambda\mu} I_{\kappa\lambda}(a-b-k_1,b,b+k_1) +\epsilon^{\kappa\mu\lambda\nu} I_{\kappa\lambda}(b''-a''-k''_2,a'',a''+k''_2) \biggr\} \\
=&\frac{-\pi^2}{(2\pi)^{12}}D_{\alpha\beta\gamma}\int d^4k_1 d^4k_2e^{-i(k_1+k_2)\cdot x}e^{ik_1\cdot y}e^{ik_2\cdot z} \\
&\times \biggl\{ \epsilon^{\kappa\nu\lambda\mu}[a_\kappa k_{1\lambda}- k_{1\kappa}a_\lambda] \\
& +\epsilon^{\kappa\mu\lambda\nu} [a_\kappa (k_1+k_2)_{\lambda}+k_{2\kappa}k_{1\lambda}- (k_1+k_2)_{\kappa}a_\lambda-k_{1\kappa}k_{2\lambda}] \biggr\} \\
=&\frac{-\pi^2}{(2\pi)^{12}}D_{\alpha\beta\gamma}\int d^4k_1 d^4k_2e^{-i(k_1+k_2)\cdot x}e^{ik_1\cdot y}e^{ik_2\cdot z} \\
&\times \biggl\{ -2\epsilon^{\kappa\mu\lambda\nu}a_\kappa k_{1\lambda}+\epsilon^{\kappa\mu\lambda\nu}[2a_\kappa (k_1+k_2)_{\lambda}+2k_{2\kappa}k_{1\lambda}] \biggr\} \\
=&\frac{-2}{(2\pi)^{12}}D_{\alpha\beta\gamma}\int d^4k_1 d^4k_2e^{-i(k_1+k_2)\cdot x}e^{ik_1\cdot y}e^{ik_2\cdot z} \pi^2 \epsilon^{\kappa\mu\lambda\nu}(a-k_2)_\kappa k_{1\lambda}
\end{align*}
となるが,これが消えるのは$a-k_2\propto k_1$のときだ.$a^\mu$をこれらの条件のどれか二つを満たすように選び,アノマリーがカレントのどれか二つから取り除かれるようにすることができる.しかし$k_1$と$k_2$が平行でなければ,三つの条件$a\propto k_1+k_2,a+k_1\propto k_2,a-k_2\propto k_1$を同時に満たすことは不可能だ.(たとえば,最初の二つの条件は$a=-k_1-k_2$を意味するが,これは三番目の条件に反する.)したがって,どのカレントがアノマリーを持つかを決める自由度はある程度存在するものの,$D_{\alpha\beta\gamma}$がゼロでなければカレント$J^\mu_\alpha(x),J^\nu_\beta(y),J^\rho_\gamma(z)$のどれか最低一つはアノマリーを持つ.これが上の計算の主要な結果だ.

\vskip\baselineskip

非常に重要な問題として,$J^\mu_\alpha(x)$が大域的対称性のカレントであり,$J^\nu_\beta(y)$と$J^\rho_\gamma(z)$がゲージ対称性のカレント,つまりゲージ場が結合するカレントである場合がある.(そのような問題は前節扱った.そのときは$J^\mu_\alpha(x)$が大域的カイラル対称性のカレントであった.)このような場合は$a^\mu$を$J^\nu_\beta(y)$や$J^\rho_\gamma(z)$ではなく,$J^\mu_\alpha(x)$のみにアノマリーが生じるように選ばなければならない.既に見たように,このためには$a+k_1\propto k_2$と$a-k_2\propto k_1$が満たされなければならず
\begin{align*}
a=k_2-k_1
\end{align*}
という唯一の結果が得られる.この$a^\mu$の値を採用して,アノマリー(22.3.22)は
\begin{align*}
&\left[ \frac{\partial}{\partial x^\mu}\Gamma^{\mu\nu\rho}_{\alpha\beta\gamma}(x,y,z) \right]_{\mr{anom}} \\
=&\frac{2}{(2\pi)^{12}}D_{\alpha\beta\gamma}\int d^4k_1 d^4k_2e^{-i(k_1+k_2)\cdot x}e^{ik_1\cdot y}e^{ik_2\cdot z} \pi^2\epsilon^{\kappa\nu\lambda\rho}(k_2-k_1)_{\kappa}(k_1+k_2)_\lambda \\
=&\frac{-1}{(2\pi)^{12}}D_{\alpha\beta\gamma}\int d^4k_1 d^4k_2e^{-i(k_1+k_2)\cdot x}e^{ik_1\cdot y}e^{ik_2\cdot z}4\pi^2\epsilon^{\kappa\nu\lambda\rho}k_{1\kappa}k_{2\lambda} \\
=&\frac{1}{(2\pi)^{12}}D_{\alpha\beta\gamma}\int d^4k_1 d^4k_2\frac{\partial}{\partial y^\kappa}e^{ik_1\cdot (y-x)}\frac{\partial}{\partial z^\rho}e^{ik_2\cdot (z-x)}4\pi^2\epsilon^{\kappa\nu\lambda\rho} \\
=&\frac{1}{(2\pi)^{4}}D_{\alpha\beta\gamma}\frac{\partial \delta^4(y-x)}{\partial y^\kappa}\frac{\partial\delta^4(z-x)}{\partial z^\rho}4\pi^2\epsilon^{\kappa\nu\lambda\rho} \\
=&\frac{1}{4\pi^2}D_{\alpha\beta\gamma}\epsilon^{\kappa\nu\lambda\rho}\frac{\partial \delta^4(y-x)}{\partial y^\kappa}\frac{\partial\delta^4(z-x)}{\partial z^\rho}
\end{align*}
となる.\par
この結果をカレント$J^\nu_\beta$と$J^\rho_\gamma$に結合するゲージ場があるときのカレント$J^\mu_\alpha$の真空期待値を使って表す.
\begin{align*}
\braket{J^\mu_\alpha(x)}&=\int[d\psi][d\bar{\psi}]J^\mu_\alpha(x) \exp\left(i\int d^4 x' \mc{L}(x')\right) \\
&=\int[d\psi][d\bar{\psi}]\sum_{N=0}^\infty\frac{1}{N!}\int d^4x_1 d^4x_2\cdots d^4x_N \, T\left\{ J^\mu_\alpha(x),i\mc{L}(x_1),i\mc{L}(x_2),\cdots,i\mc{L}(x_N) \right\}  \\
&=\sum_{N=0}^\infty\frac{1}{N!}\int d^4x_1 d^4x_2\cdots d^4x_N \braket{ T\left\{ J^\mu_\alpha(x),i\mc{L}(x_1),i\mc{L}(x_2),\cdots,i\mc{L}(x_N) \right\}}_{\VAC}
\end{align*}
このとき,物質場の共変微分から生じるゲージ場と結合するカレントから,三角ダイアグラムによる以下の寄与が存在する.
\begin{align*}
\braket{J^\mu_\alpha(x)}_{\triangle}=&i^2\frac{1}{2!}\int d^4y d^4z \braket{ T\left\{ J^\mu_{\alpha}(x),J^\nu_\beta(y)A^\beta_\nu(y),J^\rho_\gamma(z)A^\gamma_\rho(z) \right\} }_{\VAC} \\
=&-\frac{1}{2}\int d^4y d^4z \braket{T\left\{ J^\mu_\alpha(x),J^\nu_\beta(y),J^\rho_\gamma(z) \right\}}A^\beta_\nu(y)A^\gamma_\rho(z) \\
=&-\frac{1}{2}\int d^4y d^4z \Gamma^{\mu\nu\rho}_{\alpha\beta\gamma}(x,y,z) A^\beta_\nu(y)A^\gamma_\rho(z)
\end{align*}
(22.3.24)を用いると,これはアノマリーな発散
\begin{align*}
\left[\braket{\partial_\mu J^\mu_\alpha(x)}_\triangle \right]_{\mr{anom}}=&-\frac{1}{2}\int d^4y d^4z \left[\frac{\partial}{\partial x^\mu}\Gamma^{\mu\nu\rho}_{\alpha\beta\gamma}(x,y,z)\right] A^\beta_\nu(y)A^\gamma_\rho(z) \\
=&-\frac{1}{8\pi^2} \int d^4y d^4z D_{\alpha\beta\gamma} \epsilon^{\kappa\nu\lambda\rho}\frac{\partial \delta^4(y-x)}{\partial y^\kappa}\frac{\partial \delta^4(z-x)}{\partial z^\rho} A^\beta_\nu(y)A^\gamma_\rho(z) \\
=&-\frac{1}{8\pi^2} \int d^4y d^4z D_{\alpha\beta\gamma} \epsilon^{\kappa\nu\lambda\rho}\frac{\partial A^\beta_\nu(y)}{\partial y^\kappa}\frac{\partial A^\gamma_\rho(z)}{\partial z^\rho} \delta^4(y-x)  \delta^4(z-x) \\
&=-\frac{1}{8\pi^2}D_{\alpha\beta\gamma}\epsilon^{\kappa\nu\lambda\rho}\partial_\kappa A^\beta_\nu(x) \partial_\lambda A^\gamma_\rho(x)
\end{align*}
を持つことがわかる.他に,四角形ダイアグラムや五角形ダイアグラムからもアノマリーを生じる.ゲージ不変性から,それらダイアグラムの和全体は以下のゲージ不変な結果にならなければならない.
\begin{align*}
\left[\braket{\partial_\mu J^\mu_\alpha(x)} \right]_{\mr{anom}}=-\frac{1}{32\pi^2}D_{\alpha\beta\gamma}\epsilon^{\kappa\nu\lambda\rho}F^\beta_{\kappa\nu}(x)F^\gamma_{\lambda\rho}(x)
\end{align*}
(係数が$1/8\pi^2$から$1/32\pi^2$になっているのは,(22.3.27)を展開し$\epsilon^{\kappa\nu\lambda\rho}$の反対称性を用いたときに(22.3.26)を再現するようにだ.実際
\begin{align*}
&-\frac{1}{32\pi^2}D_{\alpha\beta\gamma}\epsilon^{\kappa\nu\lambda\rho}F^\beta_{\kappa\nu}(x)F^\gamma_{\lambda\rho}(x) \\
=&-\frac{1}{32\pi^2}D_{\alpha\beta\gamma}\epsilon^{\kappa\nu\lambda\rho}\left( \partial_\kappa A^\beta_\nu(x) -\partial_\nu A^\beta_\kappa(x)+C_{\beta\delta\epsilon}A^\delta_\kappa(x)A^\epsilon_\nu(x) \right) \\
&\qquad \times\left( \partial_\lambda A^\gamma_\rho(x) -\partial_\rho A^\gamma_\lambda(x)+C_{\gamma\delta'\epsilon'}A^{\delta'}_\lambda(x)A^{\epsilon'}_\rho(x) \right) \\
=&-\frac{1}{32\pi^2}D_{\alpha\beta\gamma}\epsilon^{\kappa\nu\lambda\rho}\left( 2\partial_\kappa A^\beta_\nu(x)+C_{\beta\delta\epsilon}A^\delta_\kappa(x)A^\epsilon_\nu(x) \right) \left( 2\partial_\lambda A^\gamma_\rho(x)+C_{\gamma\delta'\epsilon'}A^{\delta'}_\lambda(x)A^{\epsilon'}_\rho(x) \right) \\
=&-\frac{1}{8\pi^2}D_{\alpha\beta\gamma}\epsilon^{\kappa\nu\lambda\rho}\partial_\kappa A^\beta_\nu(x) \partial_\lambda A^\gamma_\rho(x)+\cdots
\end{align*}
となる.)\par
確認のためにフェルミオン数を保存する理論を考え,その生成子$T_\alpha$が(22.3.4)の形だとする.アノマリー(22.3.26)の定数$D_{\alpha\beta\gamma}$は以下で与えられる.
\begin{align*}
D_{\alpha\beta\gamma}=&\frac{1}{2}\mr{tr}\left[\left\{T_\alpha,T_\beta\right\} T_\gamma \right] \\
=&\frac{1}{2}\mr{tr}\left[ T_\alpha T_\beta T_\gamma \right]+\frac{1}{2}\mr{tr}\left[ T_\beta T_\alpha T_\gamma \right] \\
=&\frac{1}{2}\mr{tr}\left[ t_\alpha^L t_\beta^L t_\gamma^L \right]-\frac{1}{2}\mr{tr}\left[ (t_\alpha^R)^T (t_\beta^R)^T (t_\gamma^R)^T \right]+\frac{1}{2}\mr{tr}\left[ t_\beta^L t_\alpha^L t_\gamma^L \right]-\frac{1}{2}\mr{tr}\left[ (t_\beta^R)^T (t_\alpha^R)^T (t_\gamma^R)^T \right] \\
=&\frac{1}{2}\mr{tr}\left[ t_\alpha^L t_\beta^L t_\gamma^L \right]-\frac{1}{2}\mr{tr}\left[ t_\gamma^R t_\beta^R t_\alpha^R \right]+\frac{1}{2}\mr{tr}\left[ t_\beta^L t_\alpha^L t_\gamma^L \right]-\frac{1}{2}\mr{tr}\left[ t_\gamma^R t_\alpha^R t_\beta^R \right]  \\
=&\frac{1}{2}\mr{tr}\left[\left\{t^L_\alpha,t^L_\beta\right\} t^L_\gamma \right]-\frac{1}{2}\mr{tr}\left[\left\{t^R_\alpha,t^R_\beta\right\} t^R_\gamma \right]
\end{align*}
特に,22.2節でゲージ場の$t^L_\beta=t^R_\beta \equiv t_\beta$として(すなわち,(22.3.2)より$\delta \psi=i\theta_\beta t_\beta \psi$という変換),また$t_\gamma$についても同様にとったベクトルカレント$J^\nu_\beta$と$J^\rho_\gamma$(22.2節では$J^\nu_\alpha$と$J^\rho_\beta$と呼んだ)との相互作用による$t^L=-t^R\equiv t$(すなわち$\delta \psi=i\theta \gamma_5 t\psi$)の軸性ベクトルカレント$J^\mu_5$の発散を計算した.したがって,この場合には
\begin{align*}
D_{\alpha\beta\gamma}=&\frac{1}{2}\mr{tr}\left[\left\{t^L_\alpha,t^L_\beta\right\} t^L_\gamma \right]-\frac{1}{2}\mr{tr}\left[\left\{t^R_\alpha,t^R_\beta\right\} t^R_\gamma \right] \\
=&\frac{1}{2}\mr{tr}\left[\left\{t,t_\beta \right\} t_\gamma \right]-\frac{1}{2}\mr{tr}\left[\left\{(-t),t_\beta\right\} t_\gamma \right] \\
=&\mr{tr}\left[\left\{t,t_\beta \right\} t_\gamma \right]=\mr{tr}\left[\left\{t_\beta , t_\gamma \right\} t \right]\quad \because 完全対称性
\end{align*}
で置き換えられ,(22.3.27)は
\begin{align*}
\left[\braket{\partial_\mu J^\mu_5(x)} \right]_{\mr{anom}}=&-\frac{1}{32\pi^2}\mr{tr}\left[\left\{t_\beta , t_\gamma \right\} t \right] \epsilon^{\kappa\nu\lambda\rho}F^\beta_{\kappa\nu}(x)F^\gamma_{\lambda\rho}(x) \\
=&-\frac{1}{32\pi^2}\mr{tr}\left[t_\beta t_\gamma t \right] \epsilon^{\kappa\nu\lambda\rho}F^\beta_{\kappa\nu}(x)F^\gamma_{\lambda\rho}(x) -\frac{1}{32\pi^2}\mr{tr}\left[ t_\gamma t_\beta t \right] \epsilon^{\kappa\nu\lambda\rho}F^\beta_{\kappa\nu}(x)F^\gamma_{\lambda\rho}(x) \\
=&-\frac{1}{32\pi^2}\mr{tr}\left[t_\beta t_\gamma t \right] \epsilon^{\kappa\nu\lambda\rho}F^\beta_{\kappa\nu}(x)F^\gamma_{\lambda\rho}(x) -\frac{1}{32\pi^2}\mr{tr}\left[t_\gamma t_\beta t \right] \epsilon^{\lambda\rho\kappa\nu}F^\gamma_{\lambda\rho}(x)F^\beta_{\kappa\nu}(x) \\
=&-\frac{1}{32\pi^2}\mr{tr}\left[t_\beta t_\gamma t \right] \epsilon^{\kappa\nu\lambda\rho}F^\beta_{\kappa\nu}(x)F^\gamma_{\lambda\rho}(x) -\frac{1}{32\pi^2}\mr{tr}\left[t_\beta t_\gamma t \right] \epsilon^{\kappa\nu\lambda\rho}F^\beta_{\kappa\nu}(x)F^\gamma_{\lambda\rho}(x) \\
=&-\frac{1}{16\pi^2}\mr{tr}\left[t_\beta t_\gamma t \right] \epsilon^{\kappa\nu\lambda\rho}F^\beta_{\kappa\nu}(x)F^\gamma_{\lambda\rho}(x)
\end{align*}
となり,(22.2.26)と一致する.

\vskip\baselineskip

どんなゲージ場も$J^\mu_\alpha(x),J^\nu_\beta(y),J^\rho_\gamma(z)$のいずれかのカレントとも結合していない場合は,ずらすベクトル場$a^\mu$の選び方は便宜的なものでしかない.なぜなら,ゲージ場が結合するカレントからアノマリーが生じることは禁止されるが,全てのカレントがそうでないならばどのカレントからアノマリーが生じても問題はないからだ.もし,全部ではなく幾つかのカレントが「自発的に破れない」対称性に伴っていれば,破れていない対称性に対応したカレントにアノマリーがないように$a^\mu$を選んで,破れていない対称性を自明に保つ.\par
たとえば,量子色力学(あるいはそれと同等の理論)では,カイラル$SU(3)\times SU(3)$のような大域的対称性の生成子$T_\alpha$が(22.3.4)の形となり,全てのカレントは破れていない対称性に相当する$t^R_\alpha=t^L_\alpha$の(つまり$\delta \psi=i\theta_\alpha t_\alpha \psi$のときの)ベクトルカレントか,破れている対称性に相当する$t^R_\alpha=-t^L_\alpha$の(つまり$\delta \psi=i\theta_\alpha \gamma_5 t_\alpha \psi$)軸性ベクトルカレントか,だ.\par
(22.3.28)から,三つのカレントが全てベクトルカレントである場合は
\begin{align*}
D_{\alpha\beta\gamma}=&\frac{1}{2}\mr{tr}[\{ t^L_\alpha,t^L_\beta \} t^L_\gamma ]-\frac{1}{2}\mr{tr}[\{ t^L_\alpha,t^L_\beta \} t^L_\gamma ] \\
=&\frac{1}{2}\mr{tr}[\{ t^L_\alpha,t^L_\beta \} t^L_\gamma ]-\frac{1}{2}\mr{tr}[\{ t^L_\alpha,t^L_\beta \} t^L_\gamma ] \quad \because t^L_\alpha=t^R_\alpha \\
=&0
\end{align*}
となってアノマリーは生じない.一つのベクトルカレントと二つの軸性カレントの場合も,$\alpha,\beta$が軸性カレントの添え字で$\gamma$がベクトルカレントの添え字だとすると
\begin{align*}
D_{\alpha\beta\gamma}=&\frac{1}{2}\mr{tr}[\{ t^L_\alpha,t^L_\beta \} t^L_\gamma ]-\frac{1}{2}\mr{tr}[\{ t^L_\alpha,t^L_\beta \} t^L_\gamma ] \\
=&\frac{1}{2}\mr{tr}[\{ t^L_\alpha,t^L_\beta \} t^L_\gamma ]-\frac{1}{2}\mr{tr}[\{ (-t^L_\alpha),(-t^L_\beta) \} t^L_\gamma ] \quad \because t^L_\alpha=-t^R_\alpha , t^L_\beta=-t^R_\beta , t^L_\gamma=t^R_\gamma \\
=&0
\end{align*}
となってアノマリーは生じない.(ベクトルカレントに対応する添え字が$\alpha,\beta,\gamma$のどれであっても,それぞれの項は完全対称であるからどの場合でもゼロとなる.)一方,前に計算したような,二つのベクトルカレントと一つの軸性カレントの場合はアノマリーを持つ.
\begin{align*}
D_{\alpha\beta\gamma}=&\frac{1}{2}\mr{tr}[\{ t^L_\alpha,t^L_\beta \} t^L_\gamma ]-\frac{1}{2}\mr{tr}[\{ t^L_\alpha,t^L_\beta \} t^L_\gamma ] \\
=&\frac{1}{2}\mr{tr}[\{ t^L_\alpha,t^L_\beta \} t^L_\gamma ]-\frac{1}{2}\mr{tr}[\{ (-t^L_\alpha),t^L_\beta \} t^L_\gamma ] \quad \because t^L_\alpha=-t^R_\alpha , t^L_\beta=t^R_\beta , t^L_\gamma=t^R_\gamma \\
=&\mr{tr}[\{ t^L_\alpha,t^L_\beta \} t^L_\gamma ] \neq 0
\end{align*}
さらに,軸性カレントが三つの場合もアノマリーを持つ.
\begin{align*}
D_{\alpha\beta\gamma}=&\frac{1}{2}\mr{tr}[\{ t^L_\alpha,t^L_\beta \} t^L_\gamma ]-\frac{1}{2}\mr{tr}[\{ t^L_\alpha,t^L_\beta \} t^L_\gamma ] \\
=&\frac{1}{2}\mr{tr}[\{ t^L_\alpha,t^L_\beta \} t^L_\gamma ]-\frac{1}{2}\mr{tr}[\{ (-t^L_\alpha),(-t^L_\beta) \} (-t^L_\gamma) ] \quad \because t^L_\alpha=-t^R_\alpha , t^L_\beta=-t^R_\beta , t^L_\gamma=-t^R_\gamma \\
=&\mr{tr}[\{ t^L_\alpha,t^L_\beta \} t^L_\gamma ]\neq 0
\end{align*}
軸性カレント一つとベクトルカレント二つの場合には,アノマリーがベクトルカレントの保存則を妨げないように$a^\mu$を選ぶ.つまり,今回計算したように,もし$J^\mu_\alpha(x)$が軸性カレントで,$J^\nu_\beta(y),J^\rho_\gamma(z)$が二つのベクトルカレントならば(22.3.23)と同じように$a^\mu=k_2^\mu-k_1^\mu$ととり,アノマリーが(22.3.24)で与えられるようにする.一方,三つの軸性ベクトルカレントの場合には,そのどれもアノマリーを持たないことを要請する理由はない.そのため,三つのカレントに対するアノマリーが対称的になるようにして$a^\mu$を与えるのが自然だ.ローレンツ不変性から,$\alpha,\beta$を定数として$a=\alpha k_1+\beta k_2$を試してみることにする.それぞれのアノマリーは
\begin{align*}
\left[ \frac{\partial}{\partial x^\mu}\Gamma^{\mu\nu\rho}_{\alpha\beta\gamma}(x,y,z) \right]_{\mr{anom}} &=\frac{2}{(2\pi)^{12}}D_{\alpha\beta\gamma}\int d^4k_1 d^4k_2e^{-i(k_1+k_2)\cdot x}e^{ik_1\cdot y}e^{ik_2\cdot z}\pi^2\epsilon^{\kappa\nu\lambda\rho}a_{\kappa}(k_1+k_2)_\lambda \\
&=\frac{2}{(2\pi)^{12}}D_{\alpha\beta\gamma}\int d^4k_1 d^4k_2e^{-i(k_1+k_2)\cdot x}e^{ik_1\cdot y}e^{ik_2\cdot z}\pi^2\epsilon^{\kappa\nu\lambda\rho}(\alpha-\beta)k_{1\kappa}k_{2\lambda} \\
\left[ \frac{\partial}{\partial y^\nu}\Gamma^{\mu\nu\rho}_{\alpha\beta\gamma}(x,y,z) \right]_{\mr{anom}} &=\frac{-2}{(2\pi)^{12}}D_{\alpha\beta\gamma}\int d^4k_1 d^4k_2e^{-i(k_1+k_2)\cdot x}e^{ik_1\cdot y}e^{ik_2\cdot z} \pi^2 \epsilon^{\kappa\rho\lambda\mu}(a+k_1)_\kappa k_{2\lambda} \\
&=\frac{-2}{(2\pi)^{12}}D_{\alpha\beta\gamma}\int d^4k_1 d^4k_2e^{-i(k_1+k_2)\cdot x}e^{ik_1\cdot y}e^{ik_2\cdot z} \pi^2 \epsilon^{\kappa\rho\lambda\mu}(\alpha+1)k_{1\kappa} k_{2\lambda} \\
\left[ \frac{\partial}{\partial z^\rho}\Gamma^{\mu\nu\rho}_{\alpha\beta\gamma}(x,y,z) \right]_{\mr{anom}} &=\frac{-2}{(2\pi)^{12}}D_{\alpha\beta\gamma}\int d^4k_1 d^4k_2e^{-i(k_1+k_2)\cdot x}e^{ik_1\cdot y}e^{ik_2\cdot z} \pi^2 \epsilon^{\kappa\mu\lambda\nu}(a-k_2)_\kappa k_{1\lambda} \\
&=\frac{-2}{(2\pi)^{12}}D_{\alpha\beta\gamma}\int d^4k_1 d^4k_2e^{-i(k_1+k_2)\cdot x}e^{ik_1\cdot y}e^{ik_2\cdot z} \pi^2 \epsilon^{\kappa\rho\lambda\mu}(1-\beta)k_{1\kappa} k_{2\lambda}
\end{align*}
であったが,これらが等しくなるという要請より$\alpha=-\beta=-1/3$となる.したがって
\begin{align*}
a=\frac{1}{3}(k_2-k_1)
\end{align*}
となる.これを(22.3.22)に使い(22.3.23)と比較すると,三つの軸性ベクトルカレントのアノマリーは一つの軸性ベクトルカレントと二つのベクトルカレントのものの3分の1だということがわかる.

\vskip\baselineskip

カレントの発散には,さらに22.2図のダイアグラムからのアノマリーも含まれる.(22.3.26)の通り,三角ダイアグラムさえ$\braket{\partial_\mu J^\mu_\alpha(x)}=0$となるような保存カレントを与えないので,ここではゲージ不変性はなんの指針にもならない.強い相互作用の$SU(3)\times SU(3)$カイラル対称性の全アノマリーは,ベクトルカレントが保存され,軸性カレントがそれらのカレントについて対称なように運動量をずらすベクトル$a^\mu$を選んで,バーディーンによって計算された.量子色力学では,この$SU(3)\times SU(3)$(これはクォークの色ではなくフレーバーに働く)はゲージ化されていないが,このアノマリーをベクトル場$V^\mu_a(x)$の8重項と軸性ベクトル場$A^\mu_a$の8重項に強く結合する仮想的なゲージ場の汎関数$\Gamma[V,A]$におけるゲージ不変性の破れとして表すと便利だ.また,非カイラル$SU(3)$とカイラル$SU(3)$の微小ゲージ変換は(15.1.17)よりそれぞれ
\begin{align*}
t_a(V_a^\mu+\gamma_5 A_a^\mu)_\epsilon =& \exp(it_b\epsilon_b)t_a(V_a^\mu+\gamma_5 A_a^\mu)\exp(-it_b \epsilon_b) -i[\partial^\mu \exp(it_b \epsilon_b)]\exp(-it_b \epsilon_b) \\
=&t_a(V_a^\mu+\gamma_5 A_a^\mu)+[it_c\epsilon_c,t_b](V_b^\mu+\gamma_5 A_b^\mu)+t_b\partial_\mu \epsilon_b \quad \because \mr{BCH}公式\\
=&t_a(V_a^\mu+\gamma_5 A_a^\mu)+f_{abc}t_a \epsilon_c (V_b^\mu+\gamma_5 A_b^\mu)+t_a\partial_\mu \epsilon_a \\
\therefore \delta(V_a^\mu+\gamma_5 A_a^\mu)=&\partial^\mu \epsilon_a+f_{abc} \epsilon_c (V_b^\mu+\gamma_5 A_b^\mu) \\
t_a(V_a^\mu+\gamma_5 A_a^\mu)_{5}=&\exp(i\gamma_5 t_b\epsilon_b)t_a(V_a^\mu+\gamma_5 A_a^\mu)\exp(-i\gamma_5 t_b \epsilon_b) -i[\partial^\mu \exp(i\gamma_5 t_b \epsilon_b)]\exp(-i\gamma_5 t_b \epsilon_b) \\
=&t_a(V_a^\mu+\gamma_5 A_a^\mu)+[i\gamma_5 t_c \epsilon_c , t_b](V_b^\mu+\gamma_5 A_b^\mu)+\gamma_5 t_a \partial \epsilon_a \quad \because \mr{BCH}公式\\
=&t_a(V_a^\mu+\gamma_5 A_a^\mu)+\gamma_5 f_{abc}t_a \epsilon_c (V_b^\mu+\gamma_5 A_b^\mu)+\gamma_5 t_a \partial \epsilon_a \\
\therefore \delta_5 (V_a^\mu+\gamma_5 A_a^\mu)=&\partial^\mu (\gamma_5 \epsilon_a) +\gamma_5 f_{abc} \epsilon_c (V_b^\mu+\gamma_5 A_b^\mu)
\end{align*}
となる.ここで$\delta$は生成子$t_a$による非カイラルゲージ変換を表し,$\delta_5$は生成子$\gamma_5t_a$によるカイラルゲージ変換を表す.$\gamma_5$の係数比較により
\begin{align*}
&\delta V^\mu_a=\partial^\mu \epsilon_a+f_{abc}\epsilon_c V^\mu_b,\quad \delta A^\mu_a=f_{abc}\epsilon_c A^\mu_b \\
&\delta_5 V^\mu_a=f_{abc}\epsilon_c A^\mu_b,\quad \delta_5 A^\mu_a=\partial^\mu \epsilon_a+f_{abc}\epsilon_c V^\mu_b
\end{align*}
となる.したがって微小ゲージ変換演算子は
\begin{align*}
\int d^4x i\epsilon_a(x) \mc{Y}_a(x)=&\int d^4x\, \delta V_{a\mu}(x) \frac{\delta}{\delta V_{a\mu}(x)}+\int d^4x \, \delta A_{a\mu}(x)\frac{\delta}{\delta A_{a\mu}(x)} \\
=&\int d^4x \left[\partial_\mu \epsilon_a(x)+f_{abc}\epsilon_c(x) V_{b\mu}(x)\right] \frac{\delta}{\delta V_{a\mu}(x)} +\int d^4x \, f_{abc}\epsilon_c(x) A_{b\mu}(x)\frac{\delta}{\delta A_{a\mu}(x)} \\
=&\int d^4x \left[-\epsilon_a(x)\frac{\partial}{\partial x^\mu}\frac{\delta}{\delta V_{a\mu}(x)}-\epsilon_a(x)f_{abc}V_{b\mu}(x)\frac{\delta}{\delta V_{c\mu}(x)}-\epsilon_a(x)f_{abc}A_{b\mu}(x)\frac{\delta}{\delta A_{c\mu}(x)} \right] \\
\therefore i\mc{Y}_a(x)=&-\frac{\partial}{\partial x^\mu}\frac{\delta}{\delta V_{a\mu}(x)}-f_{abc}V_{b\mu}(x)\frac{\delta}{\delta V_{c\mu}(x)}-f_{abc}A_{b\mu}(x)\frac{\delta}{\delta A_{c\mu}(x)} \\
\int d^4x i\epsilon_a(x) \mc{X}_a(x)=&\int d^4x \, \delta_5 V_{a\mu}(x) \frac{\delta}{\delta V_{a\mu}(x)}+\int d^4x \delta_5 A_{a\mu}(x)\frac{\delta}{\delta A_{a\mu}(x)} \\
=&\int d^4x \, f_{abc}\epsilon_c(x) A_{b\mu}(x) \frac{\delta}{\delta V_{a\mu}(x)} +\int d^4x \left[\partial_\mu \epsilon_a(x)+f_{abc}\epsilon_c(x) V_{b\mu}(x)\right]\frac{\delta}{\delta A_{a\mu}(x)} \\
=&\int d^4x \left[-\epsilon_a(x)\frac{\partial}{\partial x^\mu}\frac{\delta}{\delta A_{a\mu}(x)}-\epsilon_a(x)f_{abc}V_{b\mu}(x)\frac{\delta}{\delta A_{c\mu}(x)}-\epsilon_a(x)f_{abc}A_{b\mu}(x)\frac{\delta}{\delta V_{c\mu}(x)}\right] \\
\therefore i\mc{X}_a(x)=&-\frac{\partial}{\partial x^\mu}\frac{\delta}{\delta A_{a\mu}(x)}-f_{abc}V_{b\mu}(x)\frac{\delta}{\delta A_{c\mu}(x)}-f_{abc}A_{b\mu}(x)\frac{\delta}{\delta V_{c\mu}(x)}
\end{align*}
となる.ここで$f_{abc}$は$SU(3)$の構造定数だ.(これが実際に微小ゲージ変換演算子であることを確かめたければ,これを$V^\nu_b(y)$などに作用させてみれば良い.たとえば
\begin{align*}
\left[\int d^4x i\epsilon_a(x) \mc{Y}_a(x)\right] V^\nu_b(y)=&\int d^4x \, \delta V_{a\mu}(x) \frac{\delta V^\nu_b(y)}{\delta V_{a\mu}(x)}+\int d^4x \,\delta A_{a\mu}(x)\frac{\delta V^\nu_b(y)}{\delta A_{a\mu}(x)} \\
=&\int d^4x \,\delta V_{a\mu}(x)\eta^{\mu\nu}\delta^a_b \delta^4(x-y) +0 \\
=&\delta V^\nu_b(y)
\end{align*}
となる.)これらの演算子を$\Gamma[V,A]$に作用することは$\delta \Gamma[V,A],\delta_5 \Gamma[V,A]$を計算することであり,これはすなわち(22.2.12)における$\mc{A}(x)$に等しい.内線の運動量の添え字は,ベクトルカレントがアノマリーを持たないように選ばれている.
\begin{align*}
\mc{Y}_a\Gamma[V,A]=0
\end{align*}
そうすると,軸性ベクトルカレントにアノマリーが現れる.(以下の表式は,参考文献8のバーディーンの論文での(45)式を参照.)
\begin{align*}
\mc{X}_a\Gamma[V,A]=\frac{i}{32\pi^2}\epsilon^{\mu\nu\rho\sigma}\mr{Tr}\biggl\{\lambda_a\biggl[ V_{\mu\nu}V_{\rho\sigma}+\frac{1}{3}A_{\mu\nu}A_{\rho\sigma}-\frac{32}{3}A_\mu A_\nu A_\rho A_\sigma \\
+\frac{8}{3}i\left( A_\mu A_\nu V_{\rho\sigma}+A_\mu V_{\rho\sigma}A_\nu +V_{\rho\sigma}A_\mu A_\nu \right) \biggr]  \biggr\}
\end{align*}
ここで$\lambda_a$は(19.7.2)で与えられる$SU(3)$行列で
\begin{align*}
&V_\mu\equiv \frac{1}{2}\lambda_a V_{a\mu},\quad A_\mu\equiv \frac{1}{2}\lambda_a A_{a\mu} \\
&V_{\mu\nu}=\partial_\mu V_\nu -\partial_\nu V_\mu-i[V_\mu,V_\nu]-i[A_\mu,A_\nu] \\
&A_{\mu\nu}=\partial_\mu A_\nu-\partial_\nu A_\mu-i[V_\mu,A_\nu]-i[A_\mu,V_\nu]
\end{align*}
である.すでに説明したように(22.3.34)右辺第二項に伴う因子1/3は$AVV$と$AAA$のダイアグラムでの$a^\mu$の選び方が異なることによる.

\vskip\baselineskip

対称性が全て自発的に破れる場合のアノマリーについては,異なるカレントを区別するように$a^\mu$を選ぶ理由は全くない.その代わりに,フェルミオンの内線について,それについたゲージ・ボゾンについて対称なように添え字を付けるのが自然だ.既にみたように,これは$a=\frac{1}{3}(k_1-k_2)$と選んで三角ダイアグラムを計算することを意味する.これにより,(22.3.26)で与えられるものの三分の一が三角アノマリーに与えられる.四角形と五角形のダイアグラムを含めると,この結果は
\begin{align*}
\left[\braket{D_\mu J^\mu_\alpha }\right]_{\mr{anom}}=-\frac{1}{24\pi^2}\epsilon^{\kappa\nu\lambda\rho}\mr{Tr}\left\{T_\alpha \left[\partial_\kappa A_\nu \partial_\lambda A_\rho -\frac{1}{2} i\partial_\kappa A_\nu A_\lambda A_\rho +\frac{1}{2}iA_\kappa \partial_\nu A_\lambda A_\rho -\frac{1}{2}iA_\kappa A_\nu \partial_\lambda A_\rho \right]\right\}
\end{align*}
となる.ここで$A^\mu=A^\mu_\alpha T_\alpha$だ.(この結果は中原「理論物理学のための幾何学とトポロジー」を参照するといい.)

\vskip\baselineskip

(22.3.22)を導いた議論は,摂動論の任意の次数で使えて,一般に表面積分として表せる運動量積分からアノマリーが生じることが示せる.その結果,(22.3.18)で見たようにアノマリーに寄与するカレントの発散ダイアグラムはフェルミオンループを回る運動量についての積分の次元が(運動量のベキで)ゼロ,あるいは正のものだけが残る.ループのフェルミオンが仮想的ゲージボゾンと相互作用すると,フェルミオンループの運動量積分の次元が下がる.(例えば,今回三角形ダイアグラムを計算したときは(22.3.17)の次元は$-2$だったが,四角形ダイアグラムを計算するとフェルミオンプロパゲータがひとつ増えるために$-3$の次元に下がる.)そのためアノマリーへの寄与が無くなることがある.したがってそのような輻射補正からアノマリーへの寄与はない.(例えば,前の計算では六角形ダイアグラム以降は寄与しなかった.)フェルミオンループについているゲージボゾンと,他のゲージボゾンや他のフェルミオンループとの相互作用は,アノマリーに影響する.(例えば,ゲージボゾンの外線に真空偏極のときのようなループがついているとき.)しかし,これはゲージ場をくりこめば影響を消せるので,$\epsilon^{\kappa\nu\lambda\rho}F^\beta_{\kappa\nu}(x)F^\gamma_{\lambda\rho}(x)$のような演算子をくりこむ効果しかない.同じ理由で,問題の対称性を破らないフェルミオンの質量は,この質量の因子を取り出すと運動量積分の次元が下がるのでアノマリーを変えない.\par
一般の種類の理論では,ラグランジアン密度の質量項は
\begin{align*}
\mc{L}_{\mr{mass}}=-\sum_{nn',\sigma\sigma'}\chi_{\sigma n}\epsilon_{\sigma\sigma'}M_{nn'}\chi_{\sigma' n'}+\mr{h.c.}
\end{align*}
の形をとる.ここで$\sigma$はローレンツ群の$(\frac{1}{2},0)$表現の2成分スピノル指標,$\epsilon_{\sigma\sigma'}$はローレンツ不変性に必要な$\epsilon_{\frac{1}{2},-\frac{1}{2}}=+1$の反対称行列,$M$は対称質量行列だ.(これがローレンツ不変であることは,表記が違ってわかりにくいが九後ゲージの1章における点付きスピノルと点なしスピノルの話と同じことであることに気付けばわかる.)$\mc{L}_{\mr{mass}}$がゲージ不変性を保つためには(22.3.3)より
\begin{align*}
\delta \mc{L}_{\mr{mass}}=&-\sum_{nn',\sigma\sigma'}\delta \chi_{\sigma n}\epsilon_{\sigma\sigma'}M_{nn'}\chi_{\sigma' n'} -\sum_{nn',\sigma\sigma'}\chi_{\sigma n}\epsilon_{\sigma\sigma'}M_{nn'}\delta \chi_{\sigma' n'}+\mr{h.c.} \\
=&-\sum_{nn',\sigma\sigma'}i\epsilon_\alpha \chi_{\sigma n} T^{\mr{T}}_\alpha \epsilon_{\sigma\sigma'}M_{nn'}\chi_{\sigma' n'}-\sum_{nn',\sigma\sigma'}\chi_{\sigma n}\epsilon_{\sigma\sigma'}M_{nn'}i\epsilon_\alpha T_\alpha \chi_{\sigma' n'}+\mr{h.c.} \\
=&-\sum_{nn',\sigma\sigma'}i\epsilon_\alpha \chi_{\sigma n}\epsilon_{\sigma\sigma'}\left(T_\alpha^{\mr{T}}M_{nn'}+M_{nn'}T_\alpha\right)\chi_{\sigma' n'}+\mr{h.c.}=0
\end{align*}
であるから,質量行列は
\begin{align*}
-T^{\mr{T}}_\alpha M=MT_\alpha
\end{align*}
を満たさなければならない.指標$n$はゲージ群の既約表現自体を表す指標$r$と,それぞれの既約表現内の成分を表す指標$s$に置き換えることができる.(深非弾性散乱での(20.6.19)あたりの議論と同様.)ある既約表現$r$の場がゲージ変換によって別の既約表現$r'$に移り変わることはないから,$T_\alpha$は異なる既約表現同士を結びつける効果を持たない.したがって
\begin{align*}
(T_\alpha)_{rs,r's'}=\delta_{rr'}(T^{(r)}_\alpha)_{s,s'}
\end{align*}
となって,また
\begin{align*}
M_{rs,r's'}=(M^{rr'})_{ss'}
\end{align*}
と書くことができる.すると,(22.3.240)は以下のように書き直せる.
\begin{align*}
-T_\alpha^{(r)\mr{T}}M^{(r,r')}=M^{(r,r')}T_\alpha^{(r')}
\end{align*}
ここで$r$や$r'$について和はとっていない.この変形が分かりにくれば,変換(22.3.3)が以下の形になることに留意して(22.3.40)の導出をやり直した方がいいかもしれない.
\begin{align*}
\delta \chi_{\sigma,sr}=&\sum_{rs,r's'}i\epsilon_\alpha \delta_{rr'}(T_\alpha^{(r)})_{s,s'}\chi_{\sigma,s'r'} \\
=&\sum_{ss'}i\epsilon_\alpha (T_\alpha^{(r)})_{s,s'}\chi_{\sigma,s'r}
\end{align*}
これにより
\begin{align*}
\delta \mc{L}_{\mr{mass}}=&-\sum_{\sigma,\sigma'}\sum_{rr',ss'}\delta \chi_{\sigma rs}\epsilon_{\sigma\sigma'}M^{(rr')}_{ss'}\chi_{\sigma' ,r's'} -\sum_{\sigma\sigma'}\sum_{rr',ss'}\chi_{\sigma,rs}\epsilon_{\sigma\sigma'}M^{(rr')}_{ss'}\delta \chi_{\sigma', r's'}+\mr{h.c.} \\
=&-\sum_{\sigma\sigma'}\sum_{rr',ss's''}i\epsilon_\alpha \chi_{\sigma,s''r} (T^{(r)\mr{T}}_\alpha)_{s'',s} \epsilon_{\sigma\sigma'}M^{(rr')}_{ss'} \chi_{\sigma' s'r'} \\
&-\sum_{\sigma\sigma'}\sum_{rr',ss's''}\chi_{\sigma,sr}\epsilon_{\sigma\sigma'}M^{(rr')}_{ss'}i\epsilon_\alpha (T^{(r')}_\alpha)_{s's''} \chi_{\sigma' r's''}+\mr{h.c.} \\
=&-\sum_{\sigma\sigma'}\sum_{rr',ss's''}i\epsilon_\alpha \chi_{\sigma, rs}\epsilon_{\sigma\sigma'}\left((T_\alpha^{(r)\mr{T}})_{s,s''}M^{(rr')}_{s''s'}+M^{(rr')}_{ss''}(T_\alpha^{(r')})_{s'',s'}\right)\chi_{\sigma', s'r'}+\mr{h.c.}=0
\end{align*}
より
\begin{align*}
-T_\alpha^{(r)\mr{T}}M^{(r,r')}=M^{(r,r')}T_\alpha^{(r')}
\end{align*}
となる.ここでシューアの補題とは,ふたつの既約表現$D_{(r)}:\chi_r \to \chi'_r$と$D'_{(r')}:\chi_{r'} \to \chi'_{r'}$と,線形写像に対応する質量行列$M^{(rr')}:\chi_{r'}\to(M\chi)_{r}$との間に
\begin{align*}
D_{(r)}M^{(rr')}=M^{(rr')}D'_{(r')}
\end{align*}
が成り立つとき,$D_{(r)}$と$D'_{(r')}$が同型でない(すなわち相似変換で関係していない)ならば$M^{(rr')}=0$,また同型である(相似変換で関係している)ならば$M^{(rr')}$は同型写像すなわち正則行列となる,という補題だ.今回の場合は,ゲージ不変性よりゲージ変換$D_{(r)}$によって
\begin{align*}
M^{(rr')}=&D_{(r)}^{\mr{T}}M^{(rr')}D_{(r')} \\
=&M^{(rr')}+i\epsilon_{\alpha}[M^{(r,r')}T_\alpha^{(r')}+T_\alpha^{(r)\mr{T}}M^{(rr')}]
\end{align*}
となって(22.3.42)が導かれるのであったから,シューアの補題を適用することができ(i)$M^{(r,r')}=0$か(ii)$-T^{(r)\mr{T}}_\alpha$と$T^{(r')}_\alpha$が相似変換で関係している.
\begin{align*}
D_{(r)}&=1+i\epsilon_\alpha T^{(r)}_\alpha \\
=&U[D^{\mr{T}}_{(r')}]^{-1}U^{-1} \quad(U は相似変換行列)\\
=&1-i\epsilon_\alpha U T^{(r')T}_\alpha U^{-1} \\
\therefore &\quad T^{(r)}_\alpha=U [- T_\alpha^{(r')\mr{T}}]U^{-1}
\end{align*}
したがってこのとき,アノマリー定数(22.3.12)への寄与は
\begin{align*}
D^{(r)}_{\alpha\beta\gamma}=&\frac{1}{2}\mr{tr}[\{T_\alpha^{(r)} ,T^{(r)}_\beta \} T^{(r)}_\gamma] \\
=&\frac{1}{2}\mr{tr}[T_\alpha^{(r)} T^{(r)}_\beta T^{(r)}_\gamma]+\frac{1}{2}\mr{tr}[T^{(r)}_\beta T^{(r)}_\alpha T^{(r)}_\gamma] \\
=&\frac{1}{2}\mr{tr}[U[- T_\alpha^{(r')\mr{T}}]U^{-1} U[-T^{(r')}_\beta]U^{-1} U[-T^{(r')}_\gamma]U^{-1}] \\
&+\frac{1}{2}\mr{tr}[U[-T^{(r')}_\beta]U^{-1} U[- T_\alpha^{(r')\mr{T}}]U^{-1} U[-T^{(r')}_\gamma]U^{-1}] \\
=&-\frac{1}{2}\mr{tr}[T_\alpha^{(r')\mr{T}} T^{(r')\mr{T}}_\beta T^{(r')\mr{T}}_\gamma]-\frac{1}{2}\mr{tr}[T^{(r')\mr{T}}_\beta T_\alpha^{(r')\mr{T}} T^{(r')\mr{T}}_\gamma ] \\
=&-\frac{1}{2}\mr{tr}[T^{(r')}_\gamma T_\alpha^{(r')} T^{(r')}_\beta ]-\frac{1}{2}\mr{tr}[T^{(r')}_\gamma T^{(r')}_\beta T_\alpha^{(r')}  ]=-\frac{1}{2}\mr{tr}[\{T^{(r')}_\alpha,T^{(r')}_\beta\}T^{(r')}_\gamma] \\
=&-D^{(r')}_{\alpha\beta\gamma}
\end{align*}
と関係している.したがって,($r=r'$のとき)$D_{\alpha\beta\gamma}=0$となってアノマリーはゼロとなるか,もしくは($r\neq r'$のとき)二つの既約表現の間で$D_{\alpha\beta\gamma}^{(r)}+D_{\alpha\beta\gamma}^{(r')}=0$となって相殺する.したがって,ある対称性の与えられた集合のアノマリーはそれらの対称性のもとで質量を持つことが許されるフェルミオンがあっても,それに影響されない.


\newpage

\subsection{アノマリーを持たないゲージ理論}
ここまで,アノマリーの一般的なカレント$J^\mu_\alpha$の保存に対する影響を計算してきた.このカレントがそれ自身ゲージ場に結合していると,ゲージ不変性よりアノマリーが無いことが要請される.前の節ではアノマリーは(22.3.12)で定義される完全に対称な定数因子$D_{\alpha\beta\gamma}$に比例していなければらなず,このためにゲージカレントについてはアノマリーが消えるために
\begin{align*}
D_{\alpha\beta\gamma}\equiv \frac{1}{2}\mr{tr}[\{T_\alpha ,T_\beta \} T_\gamma]=0
\end{align*}
となっていなければならない.ここで$T_\alpha$は全ての左手フェルミオンと左手反フェルミオン場に対するゲージ代数の表現,また「$\mr{tr}$」は以前と同様にこれらのフェルミオンと反フェルミオンの種類についてのトレース和を意味する.もしフェルミオン場が群の適切な既約または可約な表現になっていれば,任意のゲージ群においてこの条件を満たすことができる.さらにいくつかのゲージ群では,フェルミオンがその群の任意の表現になっていても(22.4.1)が満たされる.\par
もし左手フェルミオン(と反フェルミオン)場がゲージ代数の表現であり,その複素共役がそれ自身に同値ならば,(つまり後に言う「広い意味での実表現」ならば)
\begin{align*}
(iT_\alpha)^*=S(iT_\alpha)S^{-1}
\end{align*}
となり,あるいは書き換えて($T_\alpha$を常にエルミートととるから)
\begin{align*}
-iT^*_\alpha&=-iT_\alpha^{\mr{T}}=iST_\alpha S^{-1} \quad \because T^\dagger_\alpha=T_\alpha \\
T_\alpha^{\mr{T}}&=-ST_\alpha S^{-1}
\end{align*}
となる.これを(22.3.12)に代入すると,22.3節最後の計算と同様にして$D_{\alpha\beta\gamma}=-D_{\alpha\beta\gamma}$が得られゼロとなる.ここで(22.4.2)を満たす表現は実か擬実だ.\par

\vskip\baselineskip

表現$(\pi,V)$とその複素表現$(\bar{\pi},\bar{V})$が同値であるとき,$(\pi,V)$は広い意味での実表現と呼ぶ.「広い意味」とつけた理由はすぐに分かる.$\kappa:V\to \bar{V}$を,成分全てに複素共役をとる自然な写像とする.これは反ユニタリである.
\begin{align*}
\kappa (cv)=c^* \kappa v
\end{align*}
また当たり前だが
\begin{align*}
\kappa \pi(g)=\bar{\pi}(g)\kappa
\end{align*}
だ.さて,$(\pi,V)$が広い意味で実表現だとする.すると連続な線形写像$M:\bar{V}\to V$が存在して
\begin{align*}
\pi(g)M=M\bar{\pi}(g)
\end{align*}
という相似変換が成り立っていることになる.$M$はユニタリである.これに両辺右から$\kappa$をかけて,$M$と$\kappa$を組み合わせた
\begin{align*}
\tau=M\kappa:V\to V
\end{align*}
という反ユニタリ写像を定義すると
\begin{align*}
&\pi(g)M\kappa =M\bar{\pi}(g)\kappa = M \kappa \pi(g) \\
\Rightarrow \quad & \pi (g)\tau =\tau \pi(g)
\end{align*}
となる.いま,$V$がさらに既約だとする.すると$\tau^2:V\to V$はユニタリかつ
\begin{align*}
\tau^2 \pi(g)=\pi(g)\tau^2
\end{align*}
を満たす.よってシューアの補題より$\tau^2=\lambda \mr{id}$となる.ただし$\lambda$は絶対値$1$の複素数となる.ここで
\begin{align*}
\lambda \tau=(\tau)^2\tau=\tau(\tau)^2=\tau \lambda
\end{align*}
となるが,$\tau$は反ユニタリであるから$\lambda\tau =\tau \lambda^*$となるから,$\lambda=\lambda^*$となる.したがって$\lambda=\pm 1$となる.$\tau^2=+1$の場合,任意の$v\in V$を
\begin{align*}
v=w+w' \quad \mr{where} ,w=\frac{v+\tau v}{2} ,w'=\frac{v-\tau v}{2}
\end{align*}
として,実部分
\begin{align*}
W=\left\{ w\in V |w=\tau w \right\}
\end{align*}
と純虚部分
\begin{align*}
W'=\left\{ w'\in V |w'=-\tau w' \right\}
\end{align*}
に分けることができる.そうすると,$\pi(g):V\to V$かつ$\tau \pi(g)=\pi(g)\tau$より,$\pi(g):W\to W$となる.これは実ベクトルから実ベクトルへの写像となり,行列表示は全成分が実表現となる.したがって狭義の実表現とは,$\tau^2=+1$となる表現である.対して$\tau^2=-1$になる表現を擬実表現と呼ぶ.\par
まとめると,自然な反ユニタリ写像$\tau :V\to V$が存在し,$\pi(g)$と交換するならば広い意味での実表現であり,$\tau^2=+1$ならば実,$\tau^2=-1$ならば擬実である.\par
擬実表現の例としては,21.3節で示したように,$SU(2)$二重項$(\phi_1,\phi_2)$に対して
\begin{align*}
\tau :\left(
\begin{array}{cc}
\phi_1 \\
\phi_2
\end{array}
\right) \to i\sigma_2\left(
\begin{array}{cc}
\phi_1^* \\
\phi_2^*
\end{array}
\right)
\end{align*}
を定めると,$\exp(i\theta_\alpha t_\alpha)\tau =\tau \exp(i\theta_\alpha t_\alpha)$を満たし,これは$\tau^2=-1$を満たすので擬実表現となる.\par
補足すると,$j=\tau$と定めると虚数単位$i$と合わせて$i^2=j^2=-1$となり,また$\tau$が反ユニタリーであるから$ij=-ji$を満たし,$k=ij$を定義すると$i,j,k$は四元数の関係式を満たす.このことから擬実表現は四元数表現とも呼ぶことがある.

\vskip\baselineskip

実表現の場合には相似変換$T'_\alpha=RT_\alpha R^{-1}$でこの表現を$T'_\alpha$が虚で反対称の表現に変換できる.擬実表現の場合はこれは不可能だ.(たとえば,$SU(2)$の三次元既約表現は$SO(3)$に同型であるから自明に実で,$SU(2)$の二次元表現は前述の通り擬実だ.)ゲージ代数が実か擬実な表現しか持たなければ,アノマリーは存在しないということだ.そのような代数とは,$SO(2n+1)$,$n\geq 2$の$SO(4n)$,$n\geq 3$の$USp(2n)$,$G_2,F_4,E_7,E_8$とそれらの直和の全て,らしい.他の代数でも,いくつかの表現は実でも擬実でもないが$D_{\alpha\beta\gamma}$がゼロとなる表現のみを持っていて,それらは$SO(2)\equiv U(1)$と$SO(6)\equiv SU(4)$を除いた$SO(4n+2)$,$E_6$,それら同士あるいは上のどれかの代数との直和となる.したがってアノマリーは,$n\geq 3$の$SU(n)$か$U(1)$因子を持つゲージ代数だ.標準理論はゲージ群$SU(3)\times SU(2)\times U(1)$に基づいているから,理論にアノマリーが無いようにするためにクォークやレプトンの間で相殺が起こるようにしなければならない.\par
ここで$T_\alpha,T_\beta,T_\gamma$が$G=SU(3)\times SU(2)\times U(1)$の生成子の全てをとるとき$D_{\alpha\beta\gamma}$がゼロとなるかをチェックする.$T_\alpha,T_\beta,T_\gamma$の積が$G$のもとで不変になる生成子の組み合わせだけを考えれば良い.$SU(3)$の生成子がゼロ,二つ,三つの場合には不変量を作ることができる.なぜなら$SU(3)$の生成子は$SU(3)$のもとで$8$重項となり,二つの場合には
\begin{align*}
8\times 8=1+8+8+10+\bar{10}+27
\end{align*}
で1重項を含み,三つの場合にも
\begin{align*}
8\times 8\times 8&=(1+8+8+10+\bar{10}+27)\times 8 \\
=&(1\times 8)+ (8\times 8)+(8\times 8)+\cdots \\
=&8+1+8+8+10+\bar{10}+27 +\cdots
\end{align*}
となって,これも1重項を含む.ただし一つの場合はゲルマン行列$\lambda_a$に対して$\mr{tr}(\lambda_a)=0$により$D_{\alpha\beta\gamma}$はゼロとなる.(他の$SU(2),U(1)$の生成子は作用する箇所が違うので,$SU(3)$添え字のみのトレースでゼロとなる.)また同様に,$SU(2)$生成子がゼロ,二つ,三つの場合にも不変量が作ることができる.なぜなら,$SU(2)$生成子は$SU(2)$のもとで$3$重項となり,二つの場合には
\begin{align*}
3\times 3=1+3+5
\end{align*}
となって,1重項を含む.三つの場合にも
\begin{align*}
3\times 3\times 3=&(1+3+5)\times 3 \\
=&3 +1+3+5 +(5\times 3) \\
=&1+3+3+3+5+5+7
\end{align*}
となって,これも1重項を含む.ただし一つの場合は$SU(2)$生成子$t_i$に対して$\mr{tr}(t_i)=0$により$D_{\alpha\beta\gamma}$はゼロとなる.$U(1)$生成子については任意の個数で良い.したがって,以下の組み合わせのみチェックすれば良い.

\vskip\baselineskip

(1)$SU(3)-SU(3)-SU(3)$\par
この場合,$u_L,d_L$が$3$表現,$u^*_R,d^*_R$が$\bar{3}$表現,他のフェルミオンが1表現であるから,左手フェルミオン(と反フェルミオン)が$SU(3)$の$3+3+\bar{3}+\bar{3}+1+1+1$表現になり,これは実なので$D_{\alpha\beta\gamma}$はゼロとなる.

\vskip\baselineskip

(2)$SU(3)-SU(3)-U(1)$\par
トレースは固有値の和に等しく,$SU(3)$の固有ベクトルになれるのは$u_L,d_L,u^*_R,d^*_R$のみである.したがって$D_{\alpha\beta\gamma}=\frac{1}{2}\mr{tr}[\{\lambda_a,\lambda_b\}y/g']$は,それらの$U(1)$固有値の和に比例する.すなわち
\begin{align*}
\sum_{3,\bar{3}}y/g'=-\frac{1}{6}-\frac{1}{6}+\frac{2}{3}-\frac{1}{3}=0
\end{align*}
となって,$D_{\alpha\beta\gamma}$はゼロとなる.

\vskip\baselineskip

(3)$SU(2)-SU(2)-SU(2)$\par
この場合は,前述の通り$SU(2)$が実か擬実の表現しか持たないのでアノマリーは存在しない.

\vskip\baselineskip

(4)$SU(2)-SU(2)-U(1)$\par
この場合は,$SU(3)-SU(3)-U(1)$のときと同様に,今度は$SU(2)$生成子の固有ベクトルになれるのは$SU(2)$二重項のみとなるので,$D_{\alpha\beta\gamma}$は以下に比例する.(カラーが3つあるので,その分足し合わせなければならない.)
\begin{align*}
\sum_{\mr{doublet}}y/g'=3\left(-\frac{1}{6}\right)+\frac{1}{2}=0
\end{align*}
したがって$D_{\alpha\beta\gamma}$はゼロとなる.

\vskip\baselineskip

(5)$U(1)-U(1)-U(1)$\par
この場合,全ての左手フェルミオンが固有ベクトルとなるから,$D_{\alpha\beta\gamma}$は以下に比例する.(今度は$U(1)$生成子が三つだから,固有値は三乗になる.)
\begin{align*}
\sum(y/g')^3=6\left(-\frac{1}{6}\right)^3+3\left(+\frac{2}{3}\right)^3+3\left(-\frac{1}{3}\right)^3+2\left(+\frac{1}{2}\right)^3+(-1)^3=0
\end{align*}
したがって$D_{\alpha\beta\gamma}$はゼロとなる.

\vskip\baselineskip

以上により,標準模型のゲージ対称性については全てのアノマリーが相殺することがわかる.この結果は,大統一理論に関する議論でも述べた通り$SU(3)\times SU(2)\times U(1)$が$SO(10)$に埋め込めることから簡潔に理解できる.たまたま,1世代分のクォーク・レプトン・反クォーク・反レプトン,さらにそれに($SU(3)\times SU(2)\times U(1)$)の一重項,つまり$y/g'=0$の項を追加すると$SO(10)$の16次元表現(基本スピノル表現)が完全に形成され,$SO(10)$は前述の通りアノマリーが存在しない.この一重項はアノマリーに寄与しないので,標準模型のゲージ対称性にはアノマリーが存在しないと分かる.\par
計算しなければならないアノマリーはもうひとつある.フェルミオンはどの種類も重力と同じように相互作用する.カレント$\bar{\chi}T\gamma^\mu \chi$の期待値について,重力外場のもとでフェルミオンループを持つダイアグラムを計算すると以下に比例するアノマリー$\braket{\partial_\mu (\bar{\chi}T\gamma^\mu \chi)}$が得られるらしい.
\begin{align*}
\mr{tr}\{T \}\epsilon^{\mu\nu\rho\sigma}R_{\mu\nu\kappa\lambda}R_{\rho\sigma}^{\quad \kappa \lambda}
\end{align*}
(22.3.3)のようなゲージ対称性を重力が破らないためには,このアノマリーがゼロになる,つまり
\begin{align*}
\mr{tr}\{T_\alpha \}=0
\end{align*}
を満たさなければならない.純粋なゲージアノマリーのように,これは(22.4.2)を満たすゲージ群の生成子についてはゼロになる.
\begin{align*}
\mr{tr}\{T_\alpha\}&=\mr{tr}\{T_\alpha^{\mr{T}}\}=-\mr{tr}\{ST_\alpha S^{-1}\}=-\mr{tr}\{T_\alpha \} \\
\therefore &\quad \mr{tr}\{T_\alpha\}=0
\end{align*}
したがって,ゲージ群の実か擬実の表現をなすフェルミオンからの寄与はこのアノマリーにはない.よってこのアノマリーはゲージ対称性を破ることによって質量を得るフェルミオンのみを考慮に入れて計算してやれば良い.また,この条件は$SU(2),SU(3)$の生成子については定義より明らかに満たされていることが分かるだろう.よって「$重力子-重力子-SU(2)$」「$重力子-重力子-SU(2)$」の場合にはアノマリーは存在せず,「$重力子-重力子-U(1)$」の場合のみを確かめれば良い.これは同様に以下に比例する.
\begin{align*}
\sum (y/g')=6\left( -\frac{1}{6} \right)+3\left(\frac{2}{3}\right) +3\left(-\frac{1}{3}\right)+2\left(\frac{1}{2}\right)+(-1)=0
\end{align*}
したがって,標準模型のカレントには重力アノマリーは存在しない.

\vskip\baselineskip

アノマリーがゼロとなる要請は現実的な理論を作る上の指針となる.たとえば,21章で作った表のような,$SU(3)\times SU(2)$の多重項ハイパーチャージは元々実験から得られた.しかし,なぜこれらの弱ハイパーチャージ(および対応する各多重項の電荷)が観測された値をとるか不思議だ.多重項$(u_L,d_L),u_R^*,d^*_R,(\nu_L,e_L),e^*_R$にそれぞれ,任意の弱ハイパーチャージ$a,b,c,d,e$を与えたとしよう.アノマリーが相殺する条件から \par
\noindent $SU(3)-SU(3)-U(1)$
\begin{align*}
\sum_{3,\bar{3}} y=2a+b+c=0
\end{align*}
$SU(2)-SU(2)-U(1)$
\begin{align*}
\sum_{\mr{doublet}}y=3a+d=0
\end{align*}
$U(1)-U(1)-U(1)$
\begin{align*}
\sum y^3=6a^3+3b^3+3c^3+2d^3+e^3=0
\end{align*}
$重力子-重力子-U(1)$
\begin{align*}
\sum y=6a+3b+3c+2d+e=0
\end{align*}
となる.最初の式より$b+c=-2a$,二番目の式より$d/a=-3$がすぐに得られ,四番目の式より
\begin{align*}
6a+3(-2a)+-6a+e=0
\end{align*}
で$e/a=6$を得る.三番目の式より
\begin{align*}
&6a^3+3(b^3+c^3)+2(-3a)^3+(6a)^3=0 \\
&b^3+c^3=-56a^3 \\
&=(b+c)(b^2-bc+c^2)=-2a(b^2-bc+c^2)
\end{align*}
ここで$b/a=x,c/a=y$とすると,解くべき方程式は
\begin{align*}
&x+y=-2 \\
&x^2-xy+y^2=28
\end{align*}
の連立方程式となる.上の式を下の式に代入して整理すると$x^2+2x-8=0$となって,$x=2,-4$となる.$x,y$は対称性があるので,$b/a,c/a$のどちらかが$2$で,どちらかが$-4$となる.$u^*_R,d^*_R$が交換する可能性を除くと,$d/a$の方を$-4$に決めて
\begin{align*}
b/a=-4,\quad c/a=2,\quad d/a=-3,\quad e/a=6
\end{align*}
という解を得る.あるいは$a=0$とすることで$b=-c$以外全てゼロの
\begin{align*}
b=-c,\quad a=d=e=0
\end{align*}
という解も得られる.前者の解を$U(1)$,後者を$U(1)'$とする.これらの解は排他的だ.つまり,$U(1)$と$U(1)'$の両方が局所対称性だとすることはできない.なぜなら,もしどちらも局所対称性であるとすると,$(-4)+(-1)^2(+2)\neq 0$に比例する$U(1)'-U(1)'-U(1)$アノマリーと,$(-4)^2-(+2)^2\neq 0$に比例する$U(1)'-U(1)-U(1)$アノマリーが生じるからだ.$U(1)$生成子は$a=-\frac{1}{6}g'$とすれば標準電弱理論の弱ハイパーチャージであり,$U(1)'$対称性は自然界に観測されるどのような対称性にも似ていない.この計算より,標準理論での$y$の値,すなわち電荷の与え方について論理的な説明が得られた.これより,クォークとレプトンの一世代内で全てのゲージアノマリーが相殺しなければならないならば,ゲージボゾンが弱ハイパーチャージ$y$以外のどんな$U(1)$量子数($U(1)'$のような)とも結合するのは不可能なことがわかる.\par
一方,標準模型で$SU(3)\times SU(2)\times U(1)$のゲージボゾンが,知られているクォークとレプトンにのみ結合すると仮定するのは妥当だ.しかし,他の$U(1)'$ゲージボゾンが他の発見されていない$SU(3)\times SU(2)\times U(1)$について中性なフェルミオンや,知られているクォークとレプトンに結合しているかもしれない.多重項$(u_L,d_L),u_R^*,d^*_R,(\nu_L,e_L),e^*_R$の$U(1)'$量子数$y'$をそれぞれ改めて$a',b',c',d',e'$としよう.$SU(3)\times SU(2)\times U(1)$について中性のフェルミオンに対しては何も知らないから,$U(1)'-U(1)'-U(1)'$や$重力子-重力子-U(1)'$アノマリーが消える要請をしても,中性フェルミオンが何個存在しているか分からず,条件式を立てることができない.残りのアノマリーがゼロとなる条件から以下がわかる.\par
\noindent $SU(3)-SU(3)-U(1)'$
\begin{align*}
\sum_{3,\bar{3}}y'=2a'+b'+c'=0
\end{align*}
$SU(2)-SU(2)-U(1)'$
\begin{align*}
\sum_{\mr{doublet}}y'=3a'+d'=0
\end{align*}
$U(1)-U(1)-U(1)'$
\begin{align*}
\sum (y/a)^2 y'=6(1)^2a'+3(-4)^2b'+3(+2)^2c'+2(-3)^2d'+(+6)^2e'=0
\end{align*}
$U(1)-U(1)'-U(1)'$
\begin{align*}
\sum (y/a)y'^2=6(1)a'^2+3(-4)b'^2+3(+2)c'^2+2(-3)d'^2+(+6)e'^2=0
\end{align*}
これを解くと,(もはや面倒なのでwolfram alphaに入れて,このままでは条件不足なので$b'=c'$を仮定すると)
\begin{align*}
a'=e'/3,\quad b'=c'=-e/3,\quad d=-e
\end{align*}
となる.$e=1$とすると
\begin{align*}
a=1/3,\quad b=-1/3 ,\quad c=-1/3, \quad d=-1,\quad e=1
\end{align*}
となる.これが多重項のそれぞれの値に対応していると考えると,これはまさに$B-L$の数だ.つまり,量子数$B-L$はカイラルアノマリーや重力アノマリーによって破れず,保存する.通常の物体は巨視的な$B-L$の値を持つから,もし$B-L$が局所対称性であり(つまり結合するゲージ場が存在し),その結合定数が$e$より何桁も小さい,ということがなければ,より大きな対称性から$SU(3)\times SU(2)\times U(1)$に破れる際に自発的に破れていなければならない.この対称性の破れの典型的スケール$F$は,電弱相互作用のスケールより大きくなければならないが,何桁もかなり大きい必然性はない.したがって,$B-L$カレントに結合する中性ベクトルボゾン($Z^0$より重い)が標準模型に付け加わるのは,ありえる.

\newpage

\setcounter{subsection}{5}
\subsection{無矛盾条件}
どんな対称性のアノマリーに現れる数係数も理論の物質の内容(どの群の表現か,など)に依存する.一方,アノマリーの形は理論の詳細にはあまり依存しないようだ.これはヴェス-ズミノの無矛盾条件によって決まっている.\par
大域的対称性のアノマリーに興味があるときにでも,無矛盾条件を導くには全ての対称性カレントがゲージ場に結合し,非可換対称性ではゲージ場同士も(ヤンミルズ項で)結合し,そのためにこれらの対称性がラグランジアン密度の局所対称性になると想定すると便意らしい.対応するゲージ結合定数を微小にすることで,いつでも対称性が大域的な場合に戻ってくることができる.この形式では,アノマリーを除いては背景ゲージ$A_{\alpha \mu}(x)$の有効作用$\Gamma[A]$はゲージ場に対する微小ゲージ変換
\begin{align*}
A_{\beta\mu}(y)\to A_{\beta\mu}(y)-i\int d^4 x\epsilon_\alpha(x)\mc{T}_\alpha(x)A_{\beta\mu}(y)
\end{align*}
のもとで不変だ.ここで(15.1.9)を再現するように(22.3.31)(22.3.32)の導出と同様にして
\begin{align*}
-i\mc{T}_\alpha(x)=-\frac{\partial}{\partial x^\mu}\frac{\delta}{\delta A_{\alpha\mu}(x)}-C_{\alpha\beta\gamma}A_{\beta\mu}(x)\frac{\delta}{\delta A_{\gamma\mu}(x)}
\end{align*}
ととらねばならない.アノマリーを考慮すると,$\mc{T}_\alpha (x)$は作用は消すが有効作用$\Gamma[A]$を消さず
\begin{align*}
\mc{T}_\alpha (x)\Gamma[A]=G_\alpha [x;A]
\end{align*}
となる.$G_\alpha [x;A]$はアノマリーの影響を表す.(22.6.2)はまたカレントの期待値の共変発散の表式として書くこともできる.
\begin{align*}
iG_\alpha [x;A]&=i\mc{T}_\alpha (x)\Gamma[A]=\left\{\frac{\partial}{\partial x^\mu}\frac{\delta}{\delta A_{\alpha\mu}(x)}+C_{\alpha\beta\gamma}A_{\beta\mu}(x)\frac{\delta}{\delta A_{\gamma\mu}(x)}\right\}\Gamma[A] \\
&=\frac{\partial}{\partial x^\mu} \frac{\delta}{\delta A_{\alpha\mu}(x)}\Gamma[A]+iA_{\beta\mu}(t_\beta)_{\gamma\alpha}\frac{\delta}{\delta A_{\gamma\mu}(x)}\Gamma[A] \\
&=\frac{\partial}{\partial x^\mu} \frac{\delta}{\delta A_{\alpha\mu}(x)}\Gamma[A]-iA_{\beta\mu}(t_\beta)_{\alpha\gamma}\frac{\delta}{\delta A_{\gamma\mu}(x)}\Gamma[A] \\
&=D_\mu \frac{\delta}{\delta A_{\alpha\mu}(x)}\Gamma[A] =-D_\mu \braket{J^\mu_\alpha(x)} \\
\therefore &\quad D_\mu \braket{J^\mu_\alpha(x)}=-iG_\alpha [x;A]
\end{align*}
ここで(16.1.7)より
\begin{align*}
\braket{J^\mu_\alpha(x)}\equiv -\frac{\delta}{\delta A_{\alpha\mu}(x)}\Gamma[A]
\end{align*}
であり$D_\mu$は$(t_\beta)_{\gamma\alpha}=-iC_{\alpha\beta\gamma}$の随伴表現(15.1.6)でのゲージ共変微分(15.1.10)(あるいは(17.4.13)の方がわかりやすいだろうか)だ.\par
(22.6.1)を二回作用させることで
\begin{align*}
\mc{T}_\beta(y) \mc{T}_\alpha (x)A_{\gamma\mu}(z)=&i\mc{T}_\beta (y)\left[ \frac{\partial}{\partial x^\nu}\frac{\delta}{\delta A_{\alpha\nu}(x)}+C_{\alpha\delta\epsilon}A_{\delta\nu}(x)\frac{\delta}{\delta A_{\epsilon\nu}(x)} \right]A_{\gamma\mu}(z) \\
=&i\mc{T}_\beta(y)\left[\frac{\partial}{\partial x^\mu}\delta^4(z-x)\delta_{\alpha\gamma}+C_{\alpha\delta\gamma}A_{\delta\mu}(x)\delta^4(z-x)\right] \\
=&-C_{\alpha\delta\gamma}\left[\frac{\partial}{\partial y^\rho}\frac{\delta}{\delta A_{\beta\rho}(y)}+C_{\beta\eta\theta}A_{\eta\rho}(y)\frac{\delta}{\delta A_{\theta\rho}(y)} \right]\delta^4(z-x)A_{\delta\mu}(x) \\
=&-C_{\alpha\beta\gamma}\delta^4(z-x)\frac{\partial}{\partial y^\mu}\delta^4(x-y)-C_{\alpha\delta\gamma}C_{\beta\eta\delta}A_{\eta\mu}(y)\delta^4(x-y)\delta^4(x-z) \\
\mc{T}_\alpha(x) \mc{T}_\beta (y)A_{\gamma\mu}(z)=&C_{\alpha\beta\gamma}\delta^4(z-y)\frac{\partial}{\partial x^\mu}\delta^4(y-x)-C_{\beta\delta\gamma}C_{\alpha\eta\delta}A_{\eta\mu}(x)\delta^4(x-y)\delta^4(x-z) \\
[\mc{T}_\alpha(x),\mc{T}_\beta(y)]A_{\gamma\mu}(z)=&C_{\alpha\beta\gamma}\left[\delta^4(z-y)\frac{\partial}{\partial x^\mu}\delta^4(y-x)+\delta^4(z-x)\frac{\partial}{\partial y^\mu}\delta^4(x-y)\right] \\
&+[C_{\alpha\eta\delta}C_{\delta\beta\gamma}+C_{\alpha\gamma\delta}C_{\delta\eta\beta}]A_{\eta\mu}(x)\delta^4(x-y)\delta^4(x-z) \\
=&C_{\alpha\beta\gamma}\left[\delta^4(z-y)\frac{\partial}{\partial x^\mu}\delta^4(y-x)+\delta^4(z-x)\frac{\partial}{\partial y^\mu}\delta^4(x-y)\right]  \\
&-C_{\alpha\beta\delta}C_{\delta\gamma\eta}A_{\eta\mu}(x)\delta^4(x-y)\delta^4(x-z)
\end{align*}
一方
\begin{align*}
iC_{\alpha\beta\delta}\delta^4(x-y)\mc{T}_\delta(x)A_{\gamma\mu}(z)=&C_{\alpha\beta\delta}\delta^4(x-y)\left[\frac{\partial}{\partial x^\nu}\frac{\delta}{\delta A_{\delta\nu}(x)}+C_{\delta\eta\epsilon}A_{\eta\nu}(x)\frac{\delta}{\delta A_{\epsilon\nu}(x)}\right]A_{\gamma\mu}(z) \\
=&C_{\alpha\beta\gamma}\delta^4(x-y)\frac{\partial}{\partial x^\mu}\delta^4(z-x)+C_{\alpha\beta\delta}C_{\delta\eta\gamma}A_{\eta\mu}(x)\delta^4(x-y)\delta^4(x-z) \\
=&C_{\alpha\beta\gamma}\delta^4(x-y)\frac{\partial}{\partial x^\mu}\delta^4(z-x)-C_{\alpha\beta\delta}C_{\delta\gamma\eta}A_{\eta\mu}(x)\delta^4(x-y)\delta^4(x-z)
\end{align*}
ここで
\begin{align*}
&\delta^4(z-y)\frac{\partial}{\partial x^\mu}\delta^4(y-x)+\delta^4(z-x)\frac{\partial}{\partial y^\mu}\delta^4(x-y) \\
&=\frac{\partial}{\partial x^\mu}[\delta^4(z-y)\delta^4(y-x)]-\delta^4(z-x)\frac{\partial}{\partial x^\mu}\delta^4(x-y) \\
&=\frac{\partial}{\partial x^\mu}[\delta^4(z-x)\delta^4(y-x)]-\delta^4(z-x)\frac{\partial}{\partial x^\mu}\delta^4(x-y) \\
&=\delta^4(x-y)\frac{\partial}{\partial x^\mu}\delta^4(z-x)
\end{align*}
より交換関係
\begin{align*}
[\mc{T}_\alpha(x),\mc{T}_\beta(y)]=iC_{\alpha\beta\gamma}\delta^4(x-y)\mc{T}_\gamma(x)
\end{align*}
が得られる.((22.6.1)左辺のマイナス符号はこの交換関係を得るために必要なものだった.)(22.6.2)と(22.6.5)より一般の無矛盾条件
\begin{align*}
\mc{T}_\alpha(x)G_\beta[y;A]-\mc{T}_\beta(y)G_\alpha[x;A]=iC_{\alpha\beta\gamma}\delta^4(x-y)G_\gamma[x;A]
\end{align*}
が導かれる.\par
(22.3.31)(22.3.32)で定義されるゲージ変換演算子
\begin{align*}
-i\mc{Y}_a(x)=&-\frac{\partial}{\partial x^\mu}\frac{\delta}{\delta V_{a\mu}(x)}-f_{abc}V_{b\mu}(x)\frac{\delta}{\delta V_{c\mu}(x)}-f_{abc}A_{b\mu}(x)\frac{\delta}{\delta A_{c\mu}(x)} \\
-i\mc{X}_a(x)=&-\frac{\partial}{\partial x^\mu}\frac{\delta}{\delta A_{a\mu}(x)}-f_{abc}V_{b\mu}(x)\frac{\delta}{\delta A_{c\mu}(x)}-f_{abc}A_{b\mu}(x)\frac{\delta}{\delta V_{c\mu}(x)}
\end{align*}
は,交換関係
\begin{align*}
[\mc{Y}_a(x),\mc{Y}_b(y)]=if_{abc}\delta^4(x-y)\mc{Y}_c(x) \\
[\mc{Y}_a(x),\mc{X}_b(y)]=if_{abc}\delta^4(x-y)\mc{X}_c(x) \\
[\mc{X}_a(x),\mc{X}_b(y)]=if_{abc}\delta^4(x-y)\mc{Y}_c(x)
\end{align*}
を満たす.導出は(22.6.5)と同様にできるだろう.ここで$f_{abc}$は$SU(3)$の構造定数だ.$\mc{Y}_a$で生成される$SU(3)$部分群は自発的に破れていないから,フェルミオン運動量に関する積分を$\mc{Y}_a$で生成されるゲージ変換のもとでの不変性が保たれるように取り扱うと便利だ.こうすると
\begin{align*}
\mc{Y}_a (x)\Gamma=0
\end{align*}
となり,代わりにゼロでないアノマリー
\begin{align*}
\mc{X}_a(x)\Gamma=G_a(x)
\end{align*}
が残る.すると,自明でない無矛盾条件は
\begin{align*}
&\mc{Y}_a(x)G_b(y)=i\delta^4(x-y)f_{abc}G_c(x) \\
&\mc{X}_a(x)G_b(y)-\mc{X}_b(y)G_a(x)=0
\end{align*}
になる.上の第一の条件は,単に$G_a(x)$は通常の非カイラル$SU(3)$変換のもとで8重項として変換されることを意味し,第二の条件は$G_a(x)$に他の強い拘束を課す.バーディーン表式(22.3.34)はこの条件を満たしている.\par
一般のゲージ理論で全てのカレントを対象に扱う場合のアノマリーとして(22.3.28)を引用していたが,この表式のゲージ場についての二次より高い次数の項の係数は導出しなかった.ここで,これらの高次項が無矛盾条件(22.6.6)によって決まることを示そう.この目的のために,そしてさらに一般化をするために,ヴェス-ズミノ無矛盾条件をBRST変換のもとでの不変性条件として再構成するのが便利だ.一般のゲージ場でゴースト場$\omega_\alpha$を導入して,ベキ零のBRST演算子$s$を以下のように書く.((15.7.8)(15.7.10)参照)
\begin{align*}
sA_{\alpha\mu}=\partial_\mu \omega_\alpha +C_{\alpha\beta\gamma}A_{\beta\mu}\omega_\gamma \\
s\omega_\alpha =-\frac{1}{2}C_{\alpha\beta\gamma}\omega_\beta \omega_\gamma
\end{align*}
ここで$s$は分配則$s(AB)=(sA)B\pm A(sB)$を満たし,第二項目の符号は$A$がフェルミオン的ならば負,そうでなければ正とする.またアノマリー関数$G_\alpha[x;A]$の代わりに汎関数
\begin{align*}
G[\omega,A]=\int \omega_\alpha(x)G_\alpha[x;A]d^4x
\end{align*}
を用いる.これをBRST変換すると($\omega_\alpha$がフェルミオン的なので$G_\alpha$をBRST変換する際は負符号がつくことを忘れずに)
\begin{align*}
sG[\omega,A]=&\int \left\{s\omega_\alpha(x)\right\} G_\alpha[x;A]d^4x -\int \omega_\alpha(x)\left\{sG_\alpha[x;A]\right\}d^4x \\
=&-\frac{1}{2}C_{\alpha\beta\gamma}\int d^4x \omega_\beta (x)\omega_\gamma(x)G_\alpha[x;A] \\
&-\int d^4x \omega_\alpha(x) \int d^4y \, sA_{\beta\mu}(y)\frac{\delta G_\alpha[x;A]}{\delta A_{\beta\mu}(y)} \\
=&-\frac{1}{2}C_{\alpha\beta\gamma}\int d^4x \omega_\beta (x)\omega_\gamma(x)G_\alpha[x;A] \\ 
&-\int d^4x \omega_\alpha(x) \int d^4y \, \left[\frac{\partial \omega_\beta(y)}{\partial y^\mu}+C_{\beta\gamma\delta}A_{\gamma\mu}(y)\omega_\delta(y)\right]\frac{\delta G_\alpha[x;A]}{\delta A_{\beta\mu}(y)} \\
=&-\frac{1}{2}C_{\alpha\beta\gamma}\int d^4x \omega_\beta (x)\omega_\gamma(x)G_\alpha[x;A] \\
&+\int d^4x \omega_\alpha(x) \int d^4y \, \left[\omega_\beta(y)\frac{\partial }{\partial y^\mu}+\omega_\gamma(y)C_{\gamma\delta\beta}A_{\gamma\mu}(y)\right]\frac{\delta G_\alpha[x;A]}{\delta A_{\beta\mu}(y)} \\
=&-\frac{1}{2}C_{\alpha\beta\gamma}\int d^4x \omega_\beta (x)\omega_\gamma(x)G_\alpha[x;A] \\
&+\int d^4x \int d^4y \omega_\alpha(x)\omega_\beta(y) \left[\frac{\partial }{\partial y^\mu}\frac{\delta G_\alpha[x;A]}{\delta A_{\beta\mu}(y)}+C_{\beta\gamma\delta}A_{\gamma\mu}(y)\frac{\delta G_\alpha[x;A]}{\delta A_{\delta\mu}(y)}\right] \\
=&-\frac{1}{2}C_{\alpha\beta\gamma}\int d^4x \int d^4y \delta^4(x-y)\omega_\alpha (x)\omega_\beta(y)G_\gamma[x;A] \\
&+i\int d^4x \int d^4y \omega_\alpha(x)\omega_\beta(y)i\mc{T}_\beta(y)G_\alpha[x;A] \\
=&\int d^4x \int d^4y \omega_\alpha(x)\omega_\beta(y)\left[-\frac{1}{2}C_{\alpha\beta\gamma}\delta^4(x-y)G_\gamma[x;A]+i\mc{T}_\beta(y)G_\alpha[x;A]\right]
\end{align*}
ゴースト場は互いに反可換だから,
\begin{align*}
sG[\omega,A]=&-\frac{1}{2}C_{\alpha\beta\gamma}\int d^4x \int d^4y \delta^4(x-y)\omega_\alpha (x)\omega_\beta(y)G_\gamma[x;A] \\
&+i\int d^4x \int d^4y \omega_\alpha(x)\omega_\beta(y)i\mc{T}_\beta(y)G_\alpha[x;A] \\
=&-\frac{1}{2}C_{\alpha\beta\gamma}\int d^4x \int d^4y \delta^4(x-y)\omega_\alpha (x)\omega_\beta(y)G_\gamma[x;A] \\
&+i\frac{1}{2}\int d^4x \int d^4y \omega_\alpha(x)\omega_\beta(y)i\mc{T}_\beta(y)G_\alpha[x;A] \\
&+i\frac{1}{2}\int d^4x \int d^4y \omega_\beta(y)\omega_\alpha(x)i\mc{T}_\alpha(x)G_\beta[y;A] \quad \because x\leftrightarrow y,\alpha\leftrightarrow \beta \\
=&\int d^4x \int d^4y \omega_\alpha(x)\omega_\beta(y)\left[-\frac{1}{2}C_{\alpha\beta\gamma}\delta^4(x-y)G_\gamma[x;A]+i\frac{1}{2}\mc{T}_\beta(y)G_\alpha[x;A]-i\frac{1}{2}\mc{T}_\alpha(y)G_\beta[x;A]\right] \\
=&\frac{1}{2}i\int d^4x \int d^4y \omega_\alpha(x)\omega_\beta(y)\left[iC_{\alpha\beta\gamma}\delta^4(x-y)G_\gamma[x;A]+\mc{T}_\beta(y)G_\alpha[x;A]-\mc{T}_\alpha(y)G_\beta[x;A]\right] 
\end{align*}
この括弧の中を見てみよう.無矛盾条件(22.6.6)が成立するならばこの積分はゼロであり,逆にこの積分がゼロであるならば(22.6.6)は成り立っていなければならない!すなわち,無矛盾条件(22.6.6)は$G[\omega,A]$が全てのゴースト場$\omega_\alpha(x)$についてBRST不変
\begin{align*}
sG[\omega,A]=0
\end{align*}
と同値であることがわかる!\par
ここで,アノマリー$G[\omega,A]$が局所的汎関数$F[A]$にBRST演算子$s$を作用させたものと書ける可能性を考える.
\begin{align*}
G[\omega,A]=sF[A]
\end{align*}
(汎関数$F[A]$はゴースト場から独立だ.なぜなら,アノマリー汎関数$G[\omega,A]$は(22.6.8)より既にゴースト場について一次だが,演算子$s$はゴースト場の因子を一つ加えるからだ.もし$F$がゴースト場について一次以上ならば,$sF$はゴースト場について二次以上の項が現れ矛盾する.)BRST演算子は$s^2=0$を満たすから,このようなアノマリーは無矛盾条件$sG=0$を明らかに満たす.もし$F[A]$がゲージ場の局所的汎関数(ある与えられた点の場のと場の微分の関数,の積分の意)ならば,それを作用から差し引いてアノマリーを相殺できる.すなわち,BRST変換はパラメータを$\epsilon_\alpha(x)=\theta \omega_\alpha(x)$としたゲージ変換と見なせるから,測度からは$i\int d^4x \theta \omega_\alpha(x)G_\alpha [x;A]$が生じる((22.2.25)あたりを見ると分かりやすい).$\delta_\theta =\theta s$に気を付けると
\begin{align*}
&\delta_\theta \int[d\psi][d\bar{\psi}]e^{i(S[\psi,A]-F[A])} \\
= & \int [d\psi][d\bar{\psi}]\left\{i\delta_\theta S[\psi,A]-i\delta_\theta F[A]+i\int d^4x\theta \omega_\alpha(x) G_\alpha[x;A] \right\}e^{i(S[\psi,A]-F[A])} \\
=&\int [d\psi][d\bar{\psi}]\left\{i\delta_\theta S[\psi,A]-i\theta sF[A]+i\theta G[\omega,A] \right\}e^{i(S[\psi,A]-F[A])} \\
=& i\int [d\psi][d\bar{\psi}]\delta_\theta S[\psi,A]e^{i(S[\psi,A]-F[A])}=0
\end{align*}
と,$\delta_\theta S=\int d^4x J^\mu_\alpha(x)\partial_\mu(\theta\omega_\alpha(x))$より$\braket{\partial_\mu J_\alpha^\mu}=0$となる.これより,(22.6.11)とできる分のアノマリーは局所汎関数によって相殺できる.同じことはアノマリーの中の,局所汎関数にBRST演算子$s$を作用させた形に書けるような任意の項にも当てはまる.つまり,そのような項はそれ自身で無矛盾条件を満たし,作用に局所的な項を加えることで相殺できる.したがって,我々の興味のあるアノマリーは,無矛盾条件(22.6.10)を満たしゴースト数が1の局所汎関数$G[\omega,A]$から,ゴースト数ゼロの局所汎関数に$s$が作用した形の部分を全て引き去ったものだ.ベキ零の演算子の通常の用語に従えば,そのような汎関数の同値類は$s$演算子のゴースト数1のコホモロジー$H[\omega,A]$を成すという.
\begin{align*}
&Z[\omega,A]=\{G[\omega,A]|\mr{gh}(G[\omega,A])=1,G[\omega,A]\in \mr{Ker}(s) \} \\
&B[\omega,A]=\{G[\omega,A]|G[\omega,A]=sF[A]\in \mr{Im}(s),\mr{gh}(F[A])=0\} \\
&H[\omega,A]=Z[\omega,A]/B[\omega,A]
\end{align*}
これは局所密度自身を使って表すこともできる.アノマリー(もしくはアノマリーの中の任意の項)を$G[\omega,A]=\int d^4x \mc{G}(x)$と書くことができる.ここで$\mc{G}(x)$は時空の点$x$でのゲージ場とゴースト場,およびそれらの微分についてのベキ級数だ.$sG=0$という条件は以下の表式と同値だ.
\begin{align*}
s\mc{G}(x)=\partial_\mu \mc{J}^\mu(x)
\end{align*}
ここで,$\mc{J}^\mu(x)$は場と場の微分のある関数だ.同様に,作用に局所的な項を付け加えて相殺できる$\mc{G}$の中の項は,微分を除いて$s\mc{F}$の形のものだ.さらに,もし$\mc{G}=\partial_\mu \mc{H}^\mu+\cdots $と書ける場合,微分の項はアノマリーに寄与しない.したがって,我々に興味のあるアノマリーは,無矛盾条件(22.6.12)を満たしゴースト数が1の局所関数から,ゴースト数がゼロのある局所関数に$s$が作用した分だけ差し引き,さらにそれから微分だけ差し引いたものだ.これは,微分の文だけ差し引いた局所関数の空間でゴースト数が1の$s$のコホモロジーとして知られていて,通常$H^1(s|d)$と書かれるらしい.\par
代数的方法を使ってBRST演算子$s$のコホモロジーが調べられている.これより一般のゲージ理論でのアノマリーの形が分かる.この節の最初に述べた通り,この手法はアノマリーの形を決めるものであり,定数係数は理論に含まれる物質の内容に依存しているためこの方法ではわからない.これは22.2節と22.3節の方法で計算しなければならない.一般のゲージ理論でゲージ場について2次のアノマリー項は,その定数係数も既に計算したので,ここでは無矛盾条件(22.6.12)を使って場について高次の項を計算する.\par
22.3節で見たように,全てのカレントを対称に扱うと,場について二次の項は(22.3.26)の表式の1/3だ.質量次元4の演算子もあり,ヴェス-ズミノ無矛盾条件(22.6.6)は同じ次元の演算子を関係付けるので,この条件を満たすためには次元が同じで場について二次以上の項のみ足せば良い.したがって,以下の形の無矛盾条件の解を探す.
\begin{align*}
G_\alpha[x;A]=&i[\braket{D_\mu J^\mu_\alpha}]_{\mr{anom}} \\
=&-\frac{i}{24\pi^2}\epsilon^{\kappa\nu\lambda\rho}\mr{Tr}\biggl\{T_\alpha \Bigl[\partial_\kappa A_\nu \partial_\lambda A_\rho +ic_1 \partial_\kappa A_\nu A_\lambda A_\rho  \\
& +ic_2 A_\kappa \partial_\nu A_\lambda A_\rho +ic_3 A_\kappa A_\nu \partial_\lambda A_\rho -c_4 A_\kappa A_\nu A_\lambda A_\rho \Bigr]\biggl\}
\end{align*}
ここで$A_\mu \equiv A_{\alpha\mu}T_\alpha$で,$c_i(i=1,\cdots, 4)$は決めるべき定数だ.\par
努力を大幅に軽減するために,これを微分形式の言葉を使って書くと良い.c数のパラメータの集合$dx^\mu$を導入し,これは互いに,またゴースト場$\omega_\alpha$を含む全てのフェルミオン的な場と反可換とする.
\begin{align*}
dx^\mu dx^\nu=-dx^\nu dx^\mu,\quad \omega_\alpha dx^\mu=-dx^\mu \omega_\alpha
\end{align*}
このとき,$dx^\mu$はBRST演算子$s$と反可換だ.これは以下からすぐにわかる.
\begin{align*}
(sA_{\alpha\mu})dx^\mu=&\left\{\partial_\mu \omega_\alpha +C_{\alpha\beta\gamma}A_{\beta\mu}\omega_\gamma\right\}dx^\mu \\
=&-dx^\mu\left\{\partial_\mu \omega_\alpha +C_{\alpha\beta\gamma}A_{\beta\mu}\omega_\gamma\right\} =-dx^\mu(sA_{\alpha\mu})
\end{align*}
$dx^\kappa dx^\nu dx^\lambda dx^\rho$は完全反対称だから,それは以下のように書ける.
\begin{align*}
dx^\kappa dx^\nu dx^\lambda dx^\rho=\epsilon^{\kappa\nu\lambda\rho}d^4x ,\quad d^4x =dx^0dx^1dx^2dx^3
\end{align*}
また外微分
\begin{align*}
d \equiv dx^\mu \frac{\partial}{\partial x^\mu}
\end{align*}
も導入する.これは微分が可換なためにベキ零
\begin{align*}
&d^2=dx^\mu dx^\nu \frac{\partial^2}{\partial x^\mu \partial x^\nu}=-dx^\nu dx^\mu \frac{\partial^2}{\partial x^\nu \partial x^\mu}=-d^2 \\
\therefore &\quad d^2=0
\end{align*}
であり,また$s$と反可換
\begin{align*}
&ds=dx^\mu \frac{\partial}{\partial x^\mu} s=-sdx^\mu\frac{\partial}{\partial x^\mu}=-sd \\
\therefore &\quad  ds+sd=0
\end{align*}
だ.最後に,反可換量
\begin{align*}
A\equiv iA_\mu dx^\mu =iA_{\alpha\mu}T_\alpha dx^\mu ,\quad \omega \equiv i\omega_\alpha T_\alpha 
\end{align*}
を導入する.これらの記法を用いれば,(22.6.13)は
\begin{align*}
G[\omega,A]=&\int \omega_\alpha(x) G_\alpha[x;A]d^4x \\
=&-\frac{i}{24\pi^2}\int \omega_\alpha \epsilon^{\kappa\nu\lambda\rho}\mr{Tr}\biggl\{T_\alpha \Bigl[\partial_\kappa A_\nu \partial_\lambda A_\rho +ic_1 \partial_\kappa A_\nu A_\lambda A_\rho  \\
& +ic_2 A_\kappa \partial_\nu A_\lambda A_\rho +ic_3 A_\kappa A_\nu \partial_\lambda A_\rho -c_4 A_\kappa A_\nu A_\lambda A_\rho \Bigr]\biggl\} d^4x \\
=&-\frac{1}{24\pi^2}\int \mr{Tr}\biggl\{i\omega_\alpha T_\alpha \biggl[\left(dx^\kappa \frac{\partial}{\partial x^\kappa} A_\nu dx^\nu \right) \left(dx^\lambda \frac{\partial}{\partial x^\lambda} A_\rho dx^\rho \right) \\
&+ic_1A_\kappa dx^\kappa \left(dx^\nu \frac{\partial}{\partial x^\nu} A_\lambda dx^\lambda \right)A_\rho dx^\rho +ic_3A_\kappa dx^\kappa A_\nu dx^\nu \left(dx^\lambda \frac{\partial}{\partial x^\lambda} A\rho dx^\rho \right) \\
&-c_4 A_\kappa dx^\kappa A_\nu dx^\nu A_\lambda dx^\lambda A_\rho dx^\rho \biggr]\biggl\} \\
=&\frac{1}{24\pi^2}\int \mr{Tr}\biggl\{\omega \Bigl[(dA)^2 +c_1 A(dA)A+c_3 A^2(dA)+c_4A^4 \Bigr]\biggl\}
\end{align*}
となる.これはわかりやすい!\par
無矛盾条件(22.6.10)を満たすために,BRST変換則(22.6.7)(22.6.8)が
\begin{align*}
sA=&i\left\{\partial_\mu \omega_\alpha +C_{\alpha\beta\gamma}A_{\beta\mu}\omega_\gamma\right\}T_\alpha dx^\mu \\
=&-idx^\mu \partial_\mu \omega_\alpha T_\alpha - iC_{\alpha\beta\gamma}T_\alpha A_{\beta\mu}dx^\mu \omega_\gamma \\
=&-d\omega-[T_\beta,T_\gamma]A_{\beta\mu}dx^\mu \omega_\gamma \\
=&-d\omega-(A_{\beta\mu}T_\beta dx^\mu \omega_\gamma T_\gamma+\omega_\gamma T_\gamma A_{\beta\mu}T_\beta dx^\mu )\quad \because 反可換性 \\
=&-d\omega+\{A,\omega\} \\
s\omega=&-i\frac{1}{2} C_{\alpha\beta\gamma}T_\alpha \omega_\beta \omega_\gamma \\
=&-\frac{1}{2}[T_\beta,T_\gamma]\omega_\beta \omega_\gamma \\
=&-\frac{1}{2}[\omega_\beta T_\beta \omega_\gamma T_\gamma +\omega_\gamma T_\gamma \omega_\beta T_\beta ] \\
=&\omega^2
\end{align*}
と書けることに着目する.すると,(22.6.17)の$c_4$の項のBRST変換は
\begin{align*}
s\mr{Tr}[\omega A^4]=&\mr{Tr}[\omega^2 A^4 -\omega \{A,\omega\}A^3+\omega A\{A,\omega\}A^2 -\omega A^2 \{A,\omega\}A+\omega A^3\{A,\omega\} ]+[\omega d\omega A^3 項] \\
=&\mr{Tr}[\omega^2 A^4-\omega A\omega A^3 -\omega^2 A^4 +\omega A^2 \omega A^2 +\omega A \omega A^3 \\
&-\omega A^3\omega A-\omega A^2 \omega A^2 +\omega A^4 \omega +\omega A^3\omega A]+[\omega d\omega A^3 項] \\
=&\mr{Tr}[\omega^2 A^4]+[\omega d\omega A^3 項]
\end{align*}
となる.$\mr{Tr}[\omega^2 A^4]$に比例する$sG$への寄与は他の項から生じないので,無矛盾条件(22.6.10)は$c_4=0$でのみ満たすことができる.$c_4$のもとで,他の項のBRST変換を実行する.(ここで$d^2=0,sd+ds=0$に注意する.また,トレースの巡回性を用いるときは,$\omega_\alpha,A_{\alpha\mu},T_\alpha$などには巡回性を用いることができるが,$dx^\mu$だけは行列ではなく添え字の足し合わせには関係ないので巡回性を用いることができない.他の$dx^\nu$や$\omega$と反交換させながら注意して移動させること.)
\begin{align*}
s(dA)=&-d(sA)=-d(-d\omega+\{A,\omega\})=-d\{A,\omega\} \\
=&-d(A\omega)-d(\omega A)=-(dA)\omega+A(d\omega)-(d\omega)A+\omega(dA)
\end{align*}
となるから
\begin{align*}
s\mr{Tr}[\omega(dA)^2]=&\mr{Tr}[s\omega (dA)^2-\omega (sdA)(dA)-\omega (dA)(sdA)] \\
=&\mr{Tr}[\omega^2(dA)^2+\omega d\{A,\omega\} (dA)+\omega (dA)d\{A,\omega\}] \\
=&\mr{Tr}[-\omega^2(dA)^2+\omega (dA) \omega (dA)-\omega A (d\omega) (dA)+\omega (d\omega) A (dA)-\omega^2 (dA)^2 \\
&+\omega (dA)^2 \omega -\omega(dA)A(d\omega) +\omega(dA)(d\omega) A-\omega (dA)\omega (dA)] \\
=&\mr{Tr}[-\omega^2(dA)^2-\omega A (d\omega) (dA)+\omega (d\omega) A (dA) -\omega(dA)A(d\omega) +\omega(dA)(d\omega) A]
\end{align*}
となる.また$c_1$の項は
\begin{align*}
s\mr{Tr}[\omega(dA)A^2]=&\mr{Tr}[s\omega (dA)A^2-\omega(sdA)A^2-\omega(dA)(sA)A+\omega(dA)A(sA)] \\
=&\mr{Tr}[\omega^2 (dA)A^2+\omega d\{A,\omega \}A^2-\omega(dA)(-d\omega+\{A,\omega\})A+\omega(dA)A(-d\omega+\{A,\omega\})] \\
=&\mr{Tr}[\omega^2 (dA)A^2+\omega (dA)\omega A^2-\omega A(d\omega) A^2+\omega (d\omega)A^3- \omega^2(dA)A^2 \\
&+\omega (dA)(d\omega)A -\omega (dA)A\omega A -\omega(dA)\omega A^2 \\
&-\omega (dA)A(d\omega)+\omega(dA)A^2\omega +\omega(dA)A\omega A] \\
=&\mr{Tr}[-\omega A(d\omega) A^2+\omega (d\omega)A^3 +\omega (dA)(d\omega)A -\omega (dA)A(d\omega)+\omega(dA)A^2\omega]
\end{align*}
となる.$c_2$の項からは
\begin{align*}
s\mr{Tr}[\omega A(dA)A]=&\mr{Tr}[s\omega A(dA)A-\omega(sA)(dA)A+\omega A(sdA)A + \omega A(dA)(sA) ] \\
=&\mr{Tr}[\omega^2 A(dA)A-\omega (-d\omega +\{A,\omega \} )(dA)A-\omega A d\{A,\omega\}A+\omega A(dA)(-d\omega+\{\omega,A\})]\\
=&\mr{Tr}[\omega^2A(dA)A+\omega(d\omega)(dA)A-\omega A\omega (dA)A-\omega^2 A(dA)A \\
&-\omega A (dA)\omega A +\omega A^2(d\omega)A - \omega A(d\omega)A^2+\omega A\omega (dA)A \\
&-\omega A(dA)(d\omega)+\omega A(dA)\omega A+\omega A(dA)A\omega] \\
=&\mr{Tr}[\omega(d\omega)(dA)A+\omega A^2(d\omega)A-\omega A(d\omega)A^2-\omega A(dA)(d\omega)+\omega A(dA)A\omega]
\end{align*}
となる.$c_3$の項からは
\begin{align*}
s\mr{Tr}[\omega A^2(dA)]=&\mr{Tr}[s\omega A^2(dA)-\omega (sA)A(dA)+\omega A(sA)(dA)-\omega A^2(sdA)] \\
=&\mr{Tr}[\omega^2 A^2(dA)-\omega (-d\omega+\{A,\omega \})A(dA)+\omega A(-d\omega+\{A,\omega \})(dA)+\omega A^2d\{A,\omega\}] \\
=&\mr{Tr}[\omega^2A^2(dA)+\omega(d\omega)A(dA)-\omega A\omega A(dA)-\omega^2 A^2(dA) \\
&-\omega A(d\omega)(dA)+\omega A^2\omega (dA)+\omega A\omega A(dA) \\
&+\omega A^2(dA)\omega -\omega A^3 (d\omega)+\omega A^2(d\omega)A-\omega A^2\omega (dA)] \\
=&\mr{Tr}[\omega(d\omega)A(dA)-\omega A(d\omega)(dA)+\omega A^2(dA)\omega -\omega A^3 (d\omega)+\omega A^2(d\omega)A]
\end{align*}
となる.これらを合わせると
\begin{align*}
sG=&\frac{1}{24\pi^2}\int \mr{Tr}\Bigl\{ -\omega^2(dA)^2-\omega A(d\omega)(dA)+\omega(d\omega)A(dA)-\omega (dA)A(d\omega)+\omega(dA)(d\omega)A \\
&+c_1[\omega(dA)(d\omega)A-\omega(dA)A(d\omega)] \\
&+c_2[\omega(d\omega)(dA)A-\omega A(dA)(d\omega)] \\
&+c_3[\omega(d\omega)A(dA)-\omega A(d\omega)(dA)] \\
&+c_1[-\omega A(d\omega)A^2+\omega(d\omega)A^3+\omega(dA)A^2\omega] \\
&+c_2[\omega A^2(d\omega)A-\omega A(d\omega)A^2+\omega A(dA)A\omega] \\
&+c_3[\omega A^2(dA)\omega-\omega A^3(d\omega)+\omega A^2(d\omega)A]\Bigr\}
\end{align*}
となる.被積分関数がゼロと仮定する必要はなく,なんらかの局所関数の全微分の形になっていてその積分が消えるとするだけで良い.この条件は微分を一つだけ含む項と,二つ含む項のそれぞれで別々に満たされる必要がある.なぜなら微分の数が異なる項の間では相殺が起こらないからだ.微分を一つだけ含む項では,$+c_1=-c_2=+c_3\equiv c$とすると
\begin{align*}
\mr{Tr}\Bigl\{&[-\omega A(d\omega)A^2+\omega(d\omega)A^3+\omega(dA)A^2\omega] \\
&-[\omega A^2(d\omega)A-\omega A(d\omega)A^2+\omega A(dA)A\omega] \\
&+[\omega A^2(dA)\omega-\omega A^3(d\omega)+\omega A^2(d\omega)A]\Bigr\} \\
=&\mr{Tr}\Bigl\{ \omega(d\omega)A^3+\omega(dA)A^2\omega -\omega A(dA)A\omega +\omega A^2(dA)\omega -\omega A^3(d\omega) \Bigr\} \\
=&\mr{Tr}\Bigl\{\omega(d\omega)A^3+\omega^2(dA)A^2-\omega^2A(dA)A+\omega^2 A^2(dA)-(d\omega)\omega A^3 \Bigr\} \\
=&-\mr{Tr}\Bigl\{d(\omega^2 A^3)\Bigr\}
\end{align*}
となり全微分$d\mc{F}$の形になることが分かる.残りの項は,もし$c=-1/2$ならば
\begin{align*}
\mr{Tr}\Bigl\{&-\omega^2(dA)^2-\omega A(d\omega)(dA)+\omega(d\omega)A(dA)-\omega (dA)A(d\omega)+\omega(dA)(d\omega)A \\
&-\frac{1}{2}[\omega(dA)(d\omega)A-\omega(dA)A(d\omega)]\\
&+\frac{1}{2}[\omega(d\omega)(dA)A-\omega A(dA)(d\omega)]\\
&-\frac{1}{2}[\omega(d\omega)A(dA)-\omega A(d\omega)(dA)] \Bigr\} \\
=\mr{Tr}\Bigl\{&-\omega^2(dA)^2-\frac{1}{2}\omega A(d\omega)(dA)+\frac{1}{2}\omega(d\omega)A(dA)-\frac{1}{2}\omega(dA)A(d\omega)+\frac{1}{2}\omega(dA)(d\omega)A \\
&+\frac{1}{2}\omega(d\omega)(dA)A-\frac{1}{2}\omega A(dA)(d\omega) \Bigr\}
\end{align*}
ここで
\begin{align*}
d(\omega^2 A (dA))=&(d\omega)\omega A(dA)-\omega(d\omega)A(dA)+\omega^2(dA)^2 \\
d(\omega^2 (dA) A)=&(d\omega)\omega(dA)A-\omega(d\omega)(dA)A+\omega^2(dA)^2 \\
d(\omega (dA) \omega A)=&(d\omega)(dA)\omega A-\omega (dA)(d\omega)A+\omega(dA)\omega(dA) \\
d(\omega A \omega (dA))=&(d\omega)A\omega(dA)-\omega(dA)\omega(dA)+\omega A(d\omega dA)
\end{align*}
を足し合わせると
\begin{align*}
&d\mr{Tr}\Bigl\{\omega^2 A(dA)+\omega^2(dA)A+\frac{1}{2}\omega(dA)\omega A+\frac{1}{2}\omega A\omega (dA)\Bigr\} \\
=&\mr{Tr}\Bigl\{2\omega^2(dA)^2-\omega(d\omega)A(dA)+(d\omega)\omega (dA)A-\omega(d\omega)(dA)A+(d\omega)\omega A(dA) \\
&-\frac{1}{2}\omega(dA)(d\omega)A+\frac{1}{2}\omega A(d\omega)(dA)-\frac{1}{2}\omega(dA)(d\omega)A+\frac{1}{2}\omega A(d\omega)(dA)\Bigr\} \\
=&\mr{Tr}\Bigl\{2\omega^2(dA)^2-\omega(d\omega)A(dA)+(d\omega)\omega (dA)A-\omega(d\omega)(dA)A+(d\omega)\omega A(dA) \\
&+\omega A(d\omega)(dA)-\omega(dA)(d\omega)A\Bigr\} \\
=&-2\mr{Tr}\Bigl\{-\omega^2(dA)^2-\frac{1}{2}\omega A(d\omega)(dA)+\frac{1}{2}\omega(d\omega)A(dA)-\frac{1}{2}\omega(dA)A(d\omega)+\frac{1}{2}\omega(dA)(d\omega)A \\
&+\frac{1}{2}\omega(d\omega)(dA)A-\frac{1}{2}\omega A(dA)(d\omega) \Bigr\}
\end{align*}
となり,これは残りの項と同じであるから,全微分で書けることが分かる.
\begin{align*}
sG=\frac{1}{24\pi^2}\int d \mr{Tr}\left\{-\frac{1}{2}[\omega^2 A(dA)+\omega^2(dA)A+\frac{1}{2}\omega(dA)\omega A+\frac{1}{2}\omega A \omega (dA)]-\omega^2 A^3\right\}
\end{align*}
以上より,$c=-1/2=c_1=-c_2=c_3,c_4=0$を(22.6.13)(22.6.17)に用いると
\begin{align*}
&-iG_\alpha=[\braket{D_\mu J^\mu_\alpha}]_{\mr{anom}} \\
=&-\frac{1}{24\pi^2}\epsilon^{\kappa\nu\lambda\rho}\mr{Tr}\biggl\{T_\alpha \Bigl[\partial_\kappa A_\nu \partial_\lambda A_\rho -\frac{i}{2} \partial_\kappa A_\nu A_\lambda A_\rho +\frac{i}{2} A_\kappa \partial_\nu A_\lambda A_\rho -\frac{i}{2} A_\kappa A_\nu \partial_\lambda A_\rho \Bigr]\biggl\}
\end{align*}
となり,これは(22.3.38)だ.また,この結果はしばしばより簡潔に
\begin{align*}
G[\omega,A]=&\frac{1}{24\pi^2}\int \mr{Tr}\left\{\omega\left[(dA)^2-\frac{1}{2}(dA)A^2+\frac{1}{2}A(dA)A-\frac{1}{2}A^2(dA)\right]\right\} \\
=&\frac{1}{24\pi^2}\int \mr{Tr}\left\{\omega\left[(dA^2)-\frac{1}{2}d(A^3)\right]\right\} \\
=&\frac{1}{24\pi^2}\int \mr{Tr}\left\{\omega d\left[A(dA)-\frac{1}{2}A^3\right]\right\}
\end{align*}
あるいは,場の強度の二形式
\begin{align*}
F\equiv & \frac{1}{2}iT_\alpha F_{\alpha\mu\nu}dx^\mu dx^\nu \\
=&\frac{1}{2}iT_\alpha (\partial_\mu A_{\alpha\nu}-\partial_\nu A_{\alpha\mu}+C_{\alpha\beta\gamma}A_{\beta\mu}A_{\gamma\nu})dx^\mu dx^\nu \\
=&\frac{1}{2}i(2dx^\mu \partial_\mu A_{\alpha\nu}T_\alpha dx^\nu -i[T_\beta,T_\gamma]A_{\beta\mu}dx^\mu A_{\gamma\nu}dx^\nu) \\
=&\frac{1}{2}i(2dx^\mu \partial_\mu A_{\alpha\nu}T_\alpha dx^\nu -i(A_{\beta\mu}T_\beta dx^\mu A_{\gamma\nu}T_\gamma dx^\nu)-i(A_{\gamma\nu}T_\gamma dx^\nu A_{\beta\mu}T_\beta dx^\mu)) \\
=&dA-A^2
\end{align*}
を用いて
\begin{align*}
G[\omega,A]=\frac{1}{24\pi^2}\int \mr{Tr}\left\{\omega d\left[AF+\frac{1}{2}A^3\right]\right\}
\end{align*}
と書ける.アノマリーは必ずしも(22.6.13)の形にする必要はないので,(22.6.20)は$G[\omega,A]$についての唯一の結果ではない.

\vskip\baselineskip

無矛盾条件の解を構成するには,ストラ-ズミノ降下方程式として知られるエレガントな代数方程式がある.この方法を任意の偶数次元時空で説明するのは4次元時空でと同じくらい簡単なので,以下では時空の次元を$2n$とする.まず,$2n$個の時空座標に少なくとも二つの余分な変数を加え,$(2n+2)$形式$\mr{Tr}F^{n+1}$が意味を持つと考える.ここで(22.6.21)より
\begin{align*}
dF=-d(A^2)=-(dA)A+A(dA)=[A,F]
\end{align*}
に注意すると,$\mr{Tr}F^{n+1}$は閉形式だ.
\begin{align*}
d\mr{Tr}F^{n+1}=&\mr{Tr}\left\{(dF)F^n\right\}+\mr{Tr}\left\{F(dF)F^{n-1}\right\}+\cdots +\mr{Tr}\left\{F^n(dF)\right\} \\
=&(n+1)\mr{Tr}\left\{(dF)F^n \right\} \\
=&\mr{Tr}\left\{[A,F]F^n\right\}+\mr{Tr}\left\{F[A,F]F^{n-1}\right\}+\cdots +\mr{Tr}\left\{F^n[A,F]\right\} \\
=&\mr{Tr}\left\{[A,F^{n+1}]\right\}=0
\end{align*}
ポアンカレの定理より,拡張した時空が単連結ならば,その時空上の任意の閉形式は完全形式,すなわち$(2n+1)$形式$\Omega_{2n+1}$が存在して(これはチャーン・サイモンズ形式として知られている)
\begin{align*}
\mr{Tr}F^{n+1}=d\Omega_{2n+1}
\end{align*}
となっている.さらに,ゲージ変換のもとで(22.6.21)は$F\to gFg^{-1}$と変換し,$\mathrm{Tr}F^{n+1}$は明白にゲージ不変で,かつゲージ場のみに依存するから,それはBRST不変だ.(ゲージ変換はBRST変換の一種であることに気付けば良い.)
\begin{align*}
s\mr{Tr}F^{n+1}=0
\end{align*}
(22.6.15)より
\begin{align*}
sd+ds=0
\end{align*}
であるから,これにより$s\Omega_{2n+1}$もまた閉形式となる.
\begin{align*}
d(s\Omega_{2n+1})=-s(d\Omega_{2n+1})=-s\mr{Tr}F^{n+1}=0
\end{align*}
BRST演算子$s$がゴースト数を1つ増やす演算子であることと,(22.6.24)より$\Omega_{2n+1}$はゴースト数ゼロであることに注意する.ポアンカレの定理を再び使うと,閉形式$s\Omega_{2n+1}$は$\omega_\alpha$について1次の$2n$形式$\Omega^1_{2n}$が存在し
\begin{align*}
s\Omega_{2n+1}=d\Omega^1_{2n}
\end{align*}
となる.さらに,$s$はベキ零だから,
\begin{align*}
d(s\Omega^1_{2n})=-s(d\Omega^1_{2n})=-s(s\Omega_{2n+1})=0
\end{align*}
となって,$s\Omega^1_{2n}$もまた閉形式だから,ゴースト場について2次の$(2n-1)$形式$\Omega^2_{2n-1}$も存在して
\begin{align*}
s\Omega^1_{2n}=d\Omega^2_{2n-1}
\end{align*}
となっている.したがって,$\Omega^1_{2n}$はそれ自身BRST不変ではない(つまり$s\Omega^1_{2n}\neq 0$)が,$\Omega^1_{2n}$の$2n$次元時空についての積分はBRST不変だ.
\begin{align*}
s\int_{\mr{spacetime}} \Omega^1_{2n}=\int_{\mr{spacetime}} d\Omega^2_{2n-1}=0
\end{align*}
これより,アノマリー汎関数$G[\omega,A]$の候補$\int \Omega^1_{2n}$を二つの一次微分方程式$d\Omega_{2n+1}=\mr{Tr}F^{n+1}$と$d\Omega^1_{2n}=s\Omega_{2n+1}$を積分して求めることができる.これらの方程式の一般解(唯一の解ではないが)は以下となるらしい.
\begin{align*}
\Omega_{2n+1}&=(n+1)\int^1_0 dt \mr{Tr}\left\{ AF^n_t \right\} \\
\Omega^1_{2n}&=-(n+1)\int^1_0 dt(1-t)\mr{Tr}\sum^{n-1}_{r=0}\left\{ \omega d(F^r_t A F^{n-1-r}_t) \right\}
\end{align*}
ここで,$F_t=tF+(t-t^2)A^2=tdA-t^2A^2$だ.($t:[0,1]$で$F_t:[0,F]$を走る.)積分(22.6.30)を$n=2$で計算すると,
\begin{align*}
\Omega^1_4=&-3\int^1_0 dt(1-t)\mr{Tr}\left\{ \omega d(AF_t+F_t A) \right\} \\
=&-3\int^1_0 dt(1-t)\mr{Tr}\left\{ \omega d[A(tF+t(1-t)A^2)+(tF+t(1-t)A^2)A ]\right\} \\
=&-3\int^1_0 dt \mr{Tr}\left[t(1-t)\omega d(AF)+t(1-t)^2\omega d(A^3)+t(1-t)\omega d(FA)+t(1-t)^2\omega d(A^3)\right] \\
=&-3\mr{Tr}\left[\frac{1}{6}\omega d(AF)+\frac{1}{12}\omega d(A^3)+\frac{1}{6}\omega d(FA)+\frac{1}{12}\omega d(A^3)\right] \\
=&-\frac{1}{2}\mr{Tr}\left[ \omega d(AF+FA+A^3) \right]=-\frac{1}{2}\mr{Tr}\left[\omega d(A(dA)+(dA)A-A^3)\right] \\
=&-\frac{1}{2}\mr{Tr}\left[\omega d(2A(dA)-A^3)\right]=-\frac{1}{4}\mr{Tr}\left[ \omega d\left[A(dA)-\frac{1}{2}A^3\right] \right]
\end{align*}
となって,(22.6.20)は$G[\omega,A]$について4次元時空の場合は$\int \Omega^1_4$に比例する結果を与える.\par
この降下を続けて,他の有用な結果を導くことができる.とくに(22.6.27)と$s$のベキ零性から,
\begin{align*}
d(s\Omega^2_{2n-1})=-s(d\Omega^2_{2n-1})=-s(s\Omega^1_{2n})=0
\end{align*}
であり,再びポアンカレの定理から$s\Omega^2_{2n-1}$が$d\Omega^3_{2n-2}$の形になり,$\Omega^2_{2n-1}$の$2n-1$空間座標積分(例えば$n=2$ならば,3次元空間座標積分)はBRST不変だと分かる.
\begin{align*}
s\int_{\mr{space}} \Omega^2_{2n-1}=\int_{\mr{space}} d\Omega^3_{2n-2}=0
\end{align*}
そのような,ゴースト場について二次のBRST不変な汎関数はいわゆるシュウィンガー項の候補となる.

\vskip\baselineskip

この節で,ここまでに与えたアノマリーの解析は1ループのアノマリーについてのみ厳密に適用できる.量子ゲージ場が結合するカレントに1ループ・アノマリーがある理論は(つまり,破れていないのにアノマリーで破れているというのは)矛盾しているので,それ以上高次を調べる必要はない.しかし,その逆は正しくない.つまり量子ゲージ場の理論が1ループの次数でアノマリーを持たない場合には,高次ループでもアノマリーがないことを示す必要がある.また,量子色力学のカイラル対称性のような大域的対称性にアノマリーを持つ理論の場合はどうせ破れている対称性なのだから矛盾は起きないが,そのような理論でもアノマリーに高次の補正があるかどうか調べる必要はある.\par
BRST変換は場に非線形に作用するから,アノマリーが無くても量子有効作用$\Gamma[\omega,A]$が1ループ近似を越えてBRST不変と期待する必然性はない.17.1節で見たように,この近似を越えるとゲージ場とゴースト場だけでなく,それらの反場も考慮する必要がある.\par
作用に反場を含めてアノマリーを調べると,これがコホモロジー形式で表させることもわかることを見る.この問題の解析はバタリン・ビルコビスキーがマスター方程式と呼んだもののジン・ジュスタン版(17.1.10)にもとづいている.有効作用がBRST不変ならば,すなわちアノマリーがなければ17.1節の議論を繰り返して$(\Gamma,\Gamma)=0$だ.\footnote{(17.1.10)での反括弧は$\chi$と$K$の組によるものだが,(15.9.1)にて$\chi^\ddagger_n=K_n+\delta\Psi/\delta \chi^n$とおけば(17.1.4)にすることができてジン・ジュスタン方程式が導出できる.$\chi^\ddagger$と$K$は反正準変換で関係しているので$\chi$と$\chi^\ddagger$の組での反括弧に置き換えてよい.}これにより,$S$をゼロ次の作用,$\Gamma_1$を量子有効作用への1ループの寄与として,1ループの次数で($S$ と$\Gamma_1$ は共にボゾン的であるから(15.9.15)より反括弧が交換できて)
\begin{align*}
(\Gamma,\Gamma)=(S+\Gamma_1,S+\Gamma_1)=(S,S)+2(S,\Gamma_1)=0
\end{align*}
となり,マスター方程式(15.9.14)より第一項目はゼロとなり,よって$(S,\Gamma_1)=0$となる.\footnote{今回はあらわに書いていないが,$\Gamma_1$には$\hbar$が係数としてついていることに意識すること.17.2節の脚注で説明している通り,ループ数を数えるパラメータとして$\hbar$が使える.この式で$(\Gamma_1,\Gamma_1)$の項が出ていないのは,それが$\hbar$について二次であり,2ループの次数だからだ.}アノマリーがあれば,その代わりに
\begin{align*}
(S,\Gamma_1)=G_1
\end{align*}
となる.ここで$G_1$は場と反場のある汎関数で,$S$と$\Gamma_1$がゴースト数ゼロなので,$G_1$はゴースト数1だ.作用は古典的なマスター方程式$(S,S)=0$を満たすと仮定するから,(15.9.22)より(22.6.35)の反括弧演算はベキゼロだ.したがって
\begin{align*}
(S,G_1)=(S,(S,\Gamma_1))=0
\end{align*}
が成り立つ.しかし,もしゴースト数ゼロのある局所汎関数$F_1$が存在して$G_1=(S,F_1)$が成り立っているならば,作用$\Gamma_1$から項$F_1$を差し引いて,1ループの次数でアノマリーを相殺
\begin{align*}
(S,\Gamma_1-F_1)=0
\end{align*}
させることができる.したがって,候補となるアノマリーは$(S,G_1)=0$の意味で閉形式であるが,ゴースト数1の局所汎関数を用いて$G_1=(S,F_1)$と書くことはできないという意味で完全ではないゴースト数1の局所汎関数$G_1$だ.言い換えると,候補となるアノマリーとは,場とその反場の局所関数の空間における,反括弧演算$X\mapsto (S,X)$のゴースト数1のコホモロジーに相当する.\par
これはまさに,この節の前半で得た結果で,BRST演算子$s$を反括弧$(S,\cdots)$に入れ替えたものだ.(22.6.35)は(15.9.16)(15.9.17)より
\begin{align*}
(S,\Gamma_1)=&\frac{\delta_R S}{\delta \chi^n}\frac{\delta_L \Gamma_1}{\delta \chi^\ddagger_n}-\frac{\delta_R \Gamma_1}{\delta \chi^n}\frac{\delta_L S}{\delta \chi^\ddagger_n} \\
=&-s\chi^\ddagger_n\frac{\delta_L \Gamma_1}{\delta \chi^\ddagger_n}-\frac{\delta_R \Gamma_1}{\delta \chi^n}s\chi_n=-2s\Gamma_1=G_1
\end{align*}
を与える.これは以前の議論と同等だ.しかし,(22.6.35)にもとづいた解析は高次に拡張できるという利点がある.\par
これを見るため,反括弧演算$X\mapsto(S,X)$がゴースト数1の局所汎関数の空間において,コホモロジーが空であって(つまり全てのアノマリーがBRST変換の像として書けるとして),上で述べたように$G_1=(S,\Gamma_1)=0$となるように作用を再定義できると仮定する.2ループでマスター方程式を破るアノマリーは
\begin{align*}
&(S+\Gamma_1+\Gamma_2,S+\Gamma_1+\Gamma_2) \\
&=(S,S)+2(S,\Gamma_1)+2(S,\Gamma_2)+(\Gamma_1,\Gamma_1)+2(\Gamma_1,\Gamma_2)+(\Gamma_2,\Gamma_2)
\end{align*}
の$\hbar$の二次の項を取り出して
\begin{align*}
(\Gamma_1,\Gamma_1)+2(S,\Gamma_2)=G_2
\end{align*}
となる関数$G_2$で表される.しかし仮定より$(S,\Gamma_1)=0,(S,S)=0$だから,そのような任意の$G_2$はヤコビ恒等式(15.9.21)と(15.9.22)より
\begin{align*}
(S,G_2)=&(S,(\Gamma_1,\Gamma_1))+2(S,(S,\Gamma_2))=0
\end{align*}
が成り立つ.これは,コホモロジーが空だという仮定により$G_2=(S,F_2)$と表せることを意味する.ここで$F_2$はゴースト数ゼロの局所汎関数だ.したがって,この次数ではアノマリーは$F_2$を作用から差し引くことで
\begin{align*}
&(S+\Gamma_1+\Gamma_2-F_2,S+\Gamma_1+\Gamma_2-F_2)=0
\end{align*}
となって相殺できる.\par
この議論は全次数に拡張できる.
\begin{align*}
(\Gamma,\Gamma)=&(S+\Gamma_1+\cdots +\Gamma_N,S+\Gamma_1+\cdots +\Gamma_N) \\
=&(S,S) +(S,\Gamma_1)+(\Gamma_1,S) \\
&+ (S,\Gamma_2) \\
&+(\Gamma_1,\Gamma_1)+(\Gamma_2,S) \\
&+\cdots \\
&+(S,\Gamma_N)+(\Gamma_1,\Gamma_{N-1})+\cdots (\Gamma,L,\Gamma_{N-L})+\cdots (\Gamma_N,S) \\
=&\sum^N_{M=0}\sum^M_{L=0}(\Gamma_L ,\Gamma_{M-L})
\end{align*}
マスター方程式でアノマリーを$N-1$次まで相殺し
\begin{align*}
0=G_M=\sum^M_{L=0}(\Gamma_L,\Gamma_{M-L})=2(S,\Gamma_M)+\sum^{M-1}_{L=1}(\Gamma_L,\Gamma_{M-L})
\end{align*}
が全ての$M<N$で成り立っているとする.(ただし$M=1$のときは第二項目はゼロだ.)反括弧$(\Gamma,\Gamma)$の$N$次の項は同様に
\begin{align*}
G_N=2(S,\Gamma_N)+\sum^{N-1}_{M=1}(\Gamma_M,\Gamma_{N-M})
\end{align*}
だから,ヤコビ恒等式(15.9.21)が,(ボゾン演算子三つの場合については全ての項の符号は負であるから)
\begin{align*}
0=-(S,(\Gamma_M,\Gamma_{N-M}))-(\Gamma_{N-M},(S,\Gamma_M))-(\Gamma_M,(\Gamma_{N-M},S))
\end{align*}
であることを用いて,上の表式全体を$S$と反括弧演算してやると
\begin{align*}
(S,G_N)=&\sum^{N-1}_{M=1}\left(S,(\Gamma_M,\Gamma_{N-M})\right) \\
=&-\sum^{N-1}_{M=1}(\Gamma_{N-M},(S,\Gamma_M))-\sum^{N-1}_{M=1}(\Gamma_M,(S,\Gamma_{N-M}))\\
=&-2\sum^{N-1}_{M=1}(\Gamma_{N-M},(S,\Gamma_M))
\end{align*}
ここで$(S,\Gamma_M)$についての仮定を用いれば
\begin{align*}
(S,G_N)=&\sum^{N-1}_{M=1}\sum^{M-1}_{L=1}(\Gamma_{N-M},(\Gamma_L,\Gamma_{M-L})) \\
=&\sum^{N-1}_{M=2}\sum^{M-1}_{L=1}(\Gamma_{N-M},(\Gamma_L,\Gamma_{M-L}))
\end{align*}
を得る.ここで$M=1$でゼロになることを用いた.この反括弧の中の$\Gamma$の添え字の和は$(N-M)+L+(M-L)=N$であることに気付く.これを用いれば,より対称な形に書き換えることができる.
\begin{align*}
(S,G_N)=\sum^{N-2}_{M_1=1}\sum^{N-2}_{M_2=1}\sum^{N-2}_{M_3=1}\delta_{N,M_1+M_2+M_3}(\Gamma_{M_1},(\Gamma_{M_2},\Gamma_{M_3}))
\end{align*}
$M_1,M_2,M_3$の範囲は同じだから,この和の二重反括弧をこれらの添え字の$3!$個の置換についての和と書くことができる.
\begin{align*}
(S,G_N)=&\frac{1}{3!}\sum^{N-2}_{M_1=1}\sum^{N-2}_{M_2=1}\sum^{N-2}_{M_3=1}\delta_{N,M_1+M_2+M_3}(\Gamma_{M_1},(\Gamma_{M_2},\Gamma_{M_3})) \\
&+\frac{1}{3!}\sum^{N-2}_{M_1=1}\sum^{N-2}_{M_3=1}\sum^{N-2}_{M_2=1}\delta_{N,M_1+M_2+M_3}(\Gamma_{M_1},(\Gamma_{M_3},\Gamma_{M_2})) \\
&+\frac{1}{3!}\sum^{N-2}_{M_2=1}\sum^{N-2}_{M_1=1}\sum^{N-2}_{M_3=1}\delta_{N,M_1+M_2+M_3}(\Gamma_{M_2},(\Gamma_{M_1},\Gamma_{M_3})) \\
&+\frac{1}{3!}\sum^{N-2}_{M_2=1}\sum^{N-2}_{M_3=1}\sum^{N-2}_{M_1=1}\delta_{N,M_1+M_2+M_3}(\Gamma_{M_2},(\Gamma_{M_3},\Gamma_{M_1})) \\
&+\frac{1}{3!}\sum^{N-2}_{M_3=1}\sum^{N-2}_{M_1=1}\sum^{N-2}_{M_2=1}\delta_{N,M_1+M_2+M_3}(\Gamma_{M_3},(\Gamma_{M_1},\Gamma_{M_2})) \\
&+\frac{1}{3!}\sum^{N-2}_{M_3=1}\sum^{N-2}_{M_2=1}\sum^{N-2}_{M_1=1}\delta_{N,M_1+M_2+M_3}(\Gamma_{M_3},(\Gamma_{M_2},\Gamma_{M_1})) \\
=&\frac{1}{3!}\sum^{N-2}_{M_1=1}\sum^{N-2}_{M_2=1}\sum^{N-2}_{M_3=1}\delta_{N,M_1+M_2+M_3}\\
&\quad \times \left[(\Gamma_{M_1},(\Gamma_{M_2},\Gamma_{M_3}))+(\Gamma_{M_3},(\Gamma_{M_1},\Gamma_{M_2}))+\Gamma_{M_2},(\Gamma_{M_3},\Gamma_{M_1}))\right] \\
+&\frac{1}{3!}\sum^{N-2}_{M_1=1}\sum^{N-2}_{M_2=1}\sum^{N-2}_{M_3=1}\delta_{N,M_1+M_2+M_3}\\
&\quad\times\left[(\Gamma_{M_1},(\Gamma_{M_3},\Gamma_{M_2}))+(\Gamma_{M_2},(\Gamma_{M_1},\Gamma_{M_3}))+\Gamma_{M_3},(\Gamma_{M_2},\Gamma_{M_1}))\right]
\end{align*}
これはヤコビ恒等式(15.9.21)によりゼロとなるから,$(S,G_N)=0$という結論に達する.仮定したようにコホモロジーが空ならば,$G_N=(S,F_N)$となる局所汎関数$F_N$が存在することになるから,$F_N$を作用から差し引くことでアノマリーが$N$次で相殺される.数学的帰納法により,コホモロジーが空ならば任意の次数のアノマリーが相殺できることが証明できた.これが証明したいことだった.\par
純粋に代数的な方法を用いて,バーニッヒ,ブラント,ヘネーによって半単純ゲージ群の(4次元時空の)ヤンミルズ理論では反括弧$X\mapsto (S,X)$のコホモロジーは,(ゴースト数1の局所汎関数の空間において)ゲージ群の各単純部分群に一つずつの(22.6.20)の形の項の線型結合のみからなることを証明した.この線型結合の係数は未知だ.これにより,理論の物質の内容に全く依らず,1ループの次数でアノマリー$G_1$が自動的に$(S,G_1)=0$を満たすときは,半単純ゲージ群のアノマリーは(22.6.20)の形の項の線型結合でなければならないことがわかる.その各単純群の定数係数は理論の物質の内容を考慮した詳細な計算で決定される.\par
さらに,22.4節では(22.6.20)のトレースがどのようなフェルミオン場についても自動的にゼロとなるようなゲージ群が存在することを見た.(それらは,$n\leq 3$の$SU(n)$群を因子として持たない半単純群ゲージ群だった.)そのような場合には反括弧演算$X\mapsto (S,X)$のゴースト数1のコホモロジーは空らしい.既に見たように,そのような理論では摂動論の任意の次数でアノマリーが存在しないことを意味する.つまり,標準理論のようなゲージ理論ではアノマリーは1ループ近似を越えても存在しない.

\vskip\baselineskip

別の意味でも,アノマリーは反括弧演算のコホモロジーに関係する.スラブノフ・テイラー恒等式(16.4.6)を導く際に,測度$\prod_{n,x}d\chi^n(x)$は問題の対称性変換のもとで不変だと仮定した.17.1節では対称性変換$\chi^n\to \chi^n+\theta \delta S/\delta \chi^\ddagger_n$((15.9.16)参照)のスラブノフ・テイラー恒等式からジン・ジュスタン方程式を導いたから,17.1節で与えたジン・ジュスタン方程式は$\prod_{n,x}d\chi^n(x)$がこの変換のもとで不変でなければ,言い換えると$\Delta S=0$でなければ,成立しない.


\newpage


\subsection{アノマリーとゴールドストンボゾン}
閉じ込められた質量ゼロのフェルミオンを持つある基本理論で線型に実現されていて,破れている大域的対称性群$G$を考える.その例としては3種の質量ゼロのクォークを持つ量子色力学の大域的カイラル対称性$SU(3)\times SU(3)$がある.この基本理論に「架空のゲージ場」を導入し,可能なアノマリーを除いては大域的対称性$G$が局所的になるようにする.この局所対称性はアノマリーで破られるだろう.なぜなら,この拡張した対称性はこの世界においては純粋に大域的で,大域的対称性は必ずしもアノマリーを持たない局所対称性に拡張できなくても良いからだ.しかし,これらのアノマリーは適切な質量ゼロの傍観フェルミオンを加えることによって消去することができる.このように導入したゲージ結合が十分小さくて,傍観フェルミオンがこの非常に弱いゲージ相互作用のみを持つ限り,これらの変更において理論のダイナミクスはあまり変わることはないはずだ.\par
次に,閉じ込められたフェルミオンが観測されないほどの低いエネルギーでの有効場の理論を考える.この理論の自由度は,質量ゼロの架空ゲージボゾンと傍観フェルミオン,そしてゴールドストンボゾンの集合だけだ.このゴールドストンボゾンは破れた対称性の独立なものについて一つずつあり,その場を$\xi_a$と書く.基本理論はゲージ不変かつアノマリーがない,と仮定したから,有効場の理論でも同じことが成り立っていなければならない.しかし傍観フェルミオンは,基本理論のアノマリーを相殺するだけのアノマリーを生じているのであったのだから,それを打ち消すためにはゴールドストンボゾンによるアノマリーと一致していなければならない.したがってゴールドストンボゾンの有効ゲージ理論は,基本理論の閉じ込められたフェルミオンが生じるアノマリーに等しい,架空的な局所対称性のアノマリーを持たなければならない.つまり,(22.6.2)の代わりに,架空のゲージ場とゴールドストンボゾンの有効作用$\Gamma[\xi,A]$は以下の条件を満たさなければならない.
\begin{align*}
\mc{T}_\beta(x)\Gamma[\xi,A]=G_\beta[x;A]
\end{align*}
ここで$G_\beta[x;A]$は基本理論の時点で既に生じていたアノマリー関数であり,つまりこれにはゴールドストンボゾンは含まれていない.また$\mc{T}_\beta$はいまやゲージ場とゴールドストンボゾン場の両方に作用するゲージ群$G$の生成子だ.$\mc{T}_\beta$の添え字$\beta$は,破れていない対称性部分群$H$の独立な生成子$\mc{Y}_i$の集合の添え字$i$と,破れた対称性の生成子$\mc{X}_a$の添え字$a$の値をとる.書く$\mc{X}_a$について一つずつゴールドストンボゾン$\xi_a$が存在する.\par

\vskip\baselineskip

(22.7.1)は,現実の弱く結合するゲージ場とゴールドストンボゾンの相互作用を調べることにも使えるらしい.たとえば,基本理論がクォークに結合した電弱ゲージ場を含む場合には,「架空の」ゲージ場と呼んだものを電弱ゲージ場とする.そのような場合には,「傍観」フェルミオンも実際に存在し,閉じ込められたフェルミオンのループによって生じる,現実の弱く結合したゲージ場のゲージ対称性のアノマリーを相殺しなければならない.これは,22.4節でレプトンがクォークによって生じる電弱アノマリーを相殺することに相当する.

\vskip\baselineskip

さて,(22.7.1)から得られる結論を論じる.この目的のために,ゲージ対称性の生成子$\mc{T}_\beta(x)$を計算するには,一般の群の線型変換
\begin{align*}
A_{\alpha\mu}(x)\to&A'_{\alpha\mu}(x)= A_{\alpha\mu}(x)-i\int d^4x \zeta_\beta(y) \mc{T}^A_\beta(y) A_{\alpha\mu}(x) \\
=&\left[1-i\int d^4x \zeta_\beta(y) \mc{T}^A_\beta(y) \right]A_{\alpha\mu}(x)=\exp\left(-i\int \zeta_\beta(y)\mc{T}^A_\beta(y)d^4x\right) A_{\alpha\mu}(x) \\
\xi_a(x)\to& \xi'_a(x)=\xi_a-i\int d^4x \zeta_\beta(y) \mc{T}^\xi_\beta(y)\xi_a(x) \\
=&\exp\left(-i\int \zeta_\beta(y)\mc{T}^\xi_\beta(y)d^4x\right)\xi_a(x) \\
\therefore g=& \exp\left(-i\int \zeta_\beta(x)\mc{T}_\beta(x)d^4x\right)
\end{align*}
のもとでゴールドストンボゾンゴールドストンボゾン場$\xi_a(x)$は(19.6.18)によって与えられる場$\xi'_a(x)$に変換され,ゲージ場$A^\mu_\alpha(x)$はゲージ変換された場$A'^\mu_\alpha(x)$に変換されることに気付くことだ.これにより
\begin{align*}
\mc{T}_\beta(x)=\mc{T}_\beta^A(x)+\mc{T}_\beta^\xi(x)
\end{align*}
となる(まぁ当たり前だと思うが).ここで$\mc{T}_\beta^A(x)$はゲージ場に働き,(22.6.1)で与えられる(導出は(22.3.31)(22.3.32)のときと同様).
\begin{align*}
-i\mc{T}^A_\beta=-\frac{\partial}{\partial x^\mu}\frac{\delta}{\delta A_{\beta\mu}(x)}-C_{\beta\gamma\alpha} A_{\gamma\mu}(x) \frac{\delta}{\delta A_{\alpha\mu}(x)}
\end{align*}
ここで$C_{\alpha\beta\gamma}$はゲージ群$G$の完全反対称構造定数だ.また$\mc{T}_\beta^\xi(x)$は(19.6.17)の微小極限で与えられる.これは$\gamma(\xi)=\exp(i\xi_a X_a)$の指数パラメータ表示(19.6.12)では以下となる.
\begin{align*}
&g\gamma(\xi(x))=\gamma(\xi'(x))h(\xi,g) \\
&\exp(i\Lambda_\beta(x)T_\beta)\exp(i\xi_a(x)X_a)=\exp\left(-i\int \Lambda_\beta(y)\mc{T}_\beta(y)d^4y\right)\left\{\exp(i\xi_a(x)X_a)\right\} \exp(i\theta_i(x) Y_i) \\
&(1+i\Lambda_\beta(x)T_\beta)\exp(i\xi_a(x)X_a)=\left(1-i\int \Lambda_\beta(y)\mc{T}_\beta(y)d^4y \right)\exp(i\xi_a(x)X_a)(1+i\theta_i(x)Y_i) \\
&i\Lambda_\beta(x)T_\beta\exp(i\xi_a(x)X_a)=-i\int \Lambda_\beta(y)\mc{T}_\beta(y)d^4y\exp(i\xi_a(x)X_a) +\exp(i\xi_a(x)X_a)i\theta_i(x)Y_i \\
&\Lambda_\beta(x)T_\beta\exp(i\xi_a(x)X_a)+\int \Lambda_\beta(y)\mc{T}_\beta(y)d^4y\exp(i\xi_a(x)X_a)=\exp(i\xi_a(x)X_a)\theta_i(x)Y_i \\
&\exp(-i\xi_a(x)X_a)\left[\Lambda_\beta(x)T_\beta+\int \Lambda_\beta(y)\mc{T}_\beta(y)d^4y\right]\exp(i\xi_a(x)X_a)=\theta_i(x)Y_i
\end{align*}
((22.7.4)の形式にしたければ,デルタ関数などを用いて積分を外して$\Lambda_\beta(x)$で全体を割れば良いと思うが,誤植があるのは明らかだし今後の証明においてはこの形式のまま使った方が便利なので,このままにしておく.重要なのは,左辺のこの和が全体として破れていない対称性の生成子に比例するという事実だ.)ここで$T_\beta$は任意の表現で$G$の生成子を表す行列だ.通常通り$T_\beta$は,それぞれ破れた対称性と破れていない対称性の生成子$X_a,Y_i$の集合に分けられる.\par
アノマリー関数$G_\beta[x;A]$については無矛盾条件(22.6.6)
\begin{align*}
\mc{T}_\alpha(x)G_\beta[y;A]-\mc{T}_\beta(y)G_\alpha[x;A]=iC_{\alpha\beta\gamma}\delta^4(x-y)G_\gamma[x;A]
\end{align*}
と,破れていない対称性にはアノマリーがないこと
\begin{align*}
\mc{T}_i(x)\Gamma[\xi,A]=G_i[x;A]=0
\end{align*}
のみを仮定する.破れていない対称性部分群の生成子について,トレース$\mr{Tr}[T_i\{T_j ,T_k \}]$がゼロとなるかぎり,作用に局所汎関数を加えて(22.7.6)が満たされるようにすることが常に可能らしい.\par
(22.7.5)と(22.7.6)の仮定のもとでは,アノマリーを持つスラブノフ・テイラー恒等式(16.4.6)(22.7.1)の解を見つけることが「常に」可能だ.
\begin{align*}
\Gamma[\xi,A]=-i\int^1_0 dt \int \xi_b(y)G_b[y;A_{-t\xi}]d^4y
\end{align*}
ここで$[A_{-t\xi}(x)]_\mu$は,$A_\mu\equiv T_\beta A_{\beta\mu}$に$\Lambda_a=-t\xi_a$と$\Lambda_i=0$のゲージ変換(15.1.17)を働かせた結果だ.
\begin{align*}
[A_{-t\xi}(x)]_\mu=&\exp(i\Lambda_\alpha(x)T_\alpha)\left[A_\mu(x)+i\partial_\mu \right]\exp(-i\Lambda_\alpha(x)T_\alpha) \\
=&\exp(-it\xi_a(x) X_a)A_\mu(x)\exp(it\xi_a(x)X_a)-i\left[\partial_\mu \exp(-it\xi_a(x)X_a)\right]\exp(it\xi_a(x)X_a)
\end{align*}
作用(22.7.7)が(22.7.1)を満たす証明の概略は以下となる.局所的な生成子$\mc{T}_\beta(x)$を使う代わりに任意の関数$\eta_\beta(x)$を導入し
\begin{align*}
\mc{T}[\eta]=\int d^4x \eta_\beta(x)\mc{T}_\beta(x)
\end{align*}
と定義する.(22.7.7)両辺に$\mc{T}[\eta]$を作用させたとき,振る舞いが気になるのは右辺に生じる$\mc{T}[\eta]\xi_b(x)$と$\mc{T}[\eta]G_b[y,A]$だ.$\mc{T}[\eta]\xi_b(x)$を計算するには,行列
\begin{align*}
\eta_{-t\xi}(x)&\equiv \exp(-it\xi_a(x)X_a)\left[\eta(x)+\mc{T}[\eta]\right]\exp(it\xi_a(x)X_a) \\
&\equiv [\eta_{-t\xi}(x)]_\beta T_\beta
\end{align*}
を導入する.ここで$\eta(x)\equiv \eta_\beta(x)T_\beta$だ.これを$t$微分すると(少し複雑だが単純に積の微分だ.)
\begin{align*}
\frac{\partial}{\partial t}\eta_{-t\xi}(x)=&-i\xi_b X_b e^{-it\xi_a X_a}\left[\eta(x)+\mc{T}[\eta]\right]e^{it\xi_a X_a} \\
&+ e^{-it\xi_a X_a}\left[\eta(x)+\mc{T}[\eta]\right]\Bigl\{i\xi_b X_b e^{it\xi_a X_a}\Bigr\} \\
=&-i\xi_b X_b\eta_{-t\xi}(x) \\
&+i e^{-it\xi_a X_a}\eta(x)\xi_b X_b e^{it\xi_a X_a} \\
&+ie^{-it\xi_a X_a}\mc{T}[\eta]\Bigl\{ e^{it\xi_a X_a}\xi_b X_b\Bigr\} \\
=&-i\xi_b X_b\eta_{-t\xi}(x) \\
&+ie^{-it\xi_a X_a}\eta(x)\xi_b X_b e^{it\xi_a X_a} \\
&+ie^{-it\xi_a X_a}e^{it\xi_a X_a}\Bigl\{\mc{T}[\eta]\xi_b\Bigr\}X_b \\
&+ie^{-it\xi_a X_a}\Bigl\{\mc{T}[\eta]e^{it\xi_a X_a}\Bigr\}\xi_b X_b \\
=&-i\xi_b X_b\eta_{-t\xi}(x) \\
&+ie^{-it\xi_a X_a}\eta(x)e^{it\xi_a X_a}\xi_b X_b \\
&+i\Bigl\{\mc{T}[\eta]\xi_b \Bigr\}X_b \\
&+ie^{-it\xi_a X_a}\Bigl\{\mc{T}[\eta]e^{it\xi_a X_a}\Bigr\}\xi_b X_b \\
=&-i\xi_b X_b\eta_{-t\xi}(x) +i\eta_{-t\xi}(x)\xi_b X_b +i\Bigl\{\mc{T}[\eta]\xi_b \Bigr\}X_b \\
=&-i[\xi_a X_a,\eta_{-t\xi}(x)]+i\Bigl\{\mc{T}[\eta]\xi_b \Bigr\}X_b 
\end{align*}
三つ目の等号では$\mc{T}[\eta]$の積のライプニッツ則を用いていることに注意.よって
\begin{align*}
&\frac{\partial}{\partial t}\eta_{-t\xi}(x)=\frac{\partial}{\partial t}[\eta_{-t\xi}(x)]_\beta T_\beta \\
=&-i[\xi_a X_a,[\eta_{-t\xi}(x)]_\gamma T_\gamma]+i\left(\mc{T}[\eta]\xi_b \right)X_b \\
=&\xi_a [\eta_{-t\xi}(x)]_\gamma C_{a\gamma \beta}T_\beta+i\left(\mc{T}[\eta]\xi_b \right)X_b
\end{align*}
より,$X_b$の係数比較より
\begin{align*}
\mc{T}[\eta]\xi_b(x)=-i\frac{\partial}{\partial t}[\eta_{-t\xi}(x)]_b+i C_{a\gamma b}\xi_a(x)[\eta_{-t\xi}(x)]_\gamma
\end{align*}
が得られる.$\mc{T}[\eta]G_b[y,A]$を求めるには,ゲージ場に$\mc{T}[\eta]$を作用させ
\begin{align*}
\mc{T}[\eta][A_{-t\xi}(x)]_\mu=\left(\mc{T}^A[\eta_{-t\xi}]A_\mu(x)\right)_{A\to A_{-t\xi}}
\end{align*}
を示す必要がある.右辺の$A\to A_{-t\xi}$は,$\mc{T}^A$を$A_\mu$に作用させた後に$A_\mu$を$[A_{-t\xi}]_\mu$にする操作である.これを示すのはかなり労力がいるが,右辺から左辺を示すのが楽だろう.以下で示す.\par
まず(22.7.9)(22.7.3)より
\begin{align*}
\left(\mc{T}^A[\eta_{-t\xi}]A_\mu \right)_{A\to A_{-t\xi}}=&\left(\int d^4 y [\eta_{-t\xi}(y)]_\beta \mc{T}^A_\beta(y) A_{\gamma \mu}T_\gamma \right)_{A\to A_{-t\xi}} \\
=&i\Bigl\{\partial _\mu [\eta_{-t\xi}(x)]_\gamma +C_{\gamma\beta\alpha}[\eta_{-t\xi}(x)]_\alpha A_{\beta\mu}(x)\Bigr\}_{A\to A_{-t\xi}}T_\gamma \\
=&\Bigl\{ i\partial_\mu \eta_{-t\xi}(x)+[T_\beta,T_\alpha][\eta_{-t\xi}(x)]_\alpha A_{\beta\mu}(x) \Bigr\}_{A\to A_{-t\xi}} \\
=&\Bigl\{i\partial_\mu \eta_{-t\xi}(x)+[A_{\mu}(x),\eta_{-t\xi}(x)]\Bigr\}_{A\to A_{-t\xi}} \\
=&i\partial_\mu \eta_{-t\xi}(x)+[[A_{-t\xi}(x)]_\mu,\eta_{-t\xi}]
\end{align*}
となり,(22.7.8)(22.7.10)より
\begin{align*}
=&i\partial_\mu\Bigl\{ e^{-it\xi_a X_a}[\eta(x)+\mc{T}[\eta]]e^{it\xi_a X_a} \Bigr\} \\
&+\Bigl[ e^{-it\xi_a X_a}A_\mu(x)e^{it\xi_a X_a},e^{-it\xi_a X_a}[\eta(x)+\mc{T}[\eta]]e^{it\xi_a X_a} \Bigr] \\
&+\Bigl[-i[\partial_\mu e^{-it\xi_a X_a}]e^{it\xi_a X_a},e^{-it\xi_a X_a}[\eta(x)+\mc{T}[\eta]]e^{it\xi_a X_a} \Bigr] \\
=&i\partial_\mu\Bigl\{ e^{-it\xi_a X_a}\eta(x)e^{it\xi_a X_a} \Bigr\}+i\partial_\mu\Bigl\{ e^{-it\xi_a X_a}\mc{T}[\eta]e^{it\xi_a X_a} \Bigr\} \\
&+e^{-it\xi_a X_a}A_\mu(x)[\eta(x)+\mc{T}[\eta]]e^{it\xi_a X_a} \\
&-e^{-it\xi_a X_a}[\eta(x)+\mc{T}[\eta]]e^{it\xi_a X_a}e^{-it\xi_a X_a}A_\mu(x)e^{it\xi_a X_a} \\
&-i[\partial_\mu e^{-it\xi_a X_a}][\eta(x)+\mc{T}[\eta]]e^{it\xi_a X_a} \\
&+ie^{-it\xi_a X_a}[\eta(x)+\mc{T}[\eta]]e^{it\xi_a X_a}[\partial_\mu e^{-it\xi_a X_a}]e^{it\xi_a X_a}
\end{align*}
ここで注意するべきことは,$\mc{T}[\eta]$が作用しているのは後ろに隣接している$\exp(it\xi_a X_a)$のみで,そのさらに後ろにはこの時点では例外なく作用していない,ということだ.複雑になってきたので,各項に分けて計算する.第四項目を
\begin{align*}
&-e^{-it\xi_a X_a}[\eta(x)+\mc{T}[\eta]]e^{it\xi_a X_a}e^{-it\xi_a X_a}A_\mu(x)e^{it\xi_a X_a} \\
=&-e^{-it\xi_a X_a}\eta(x)A_\mu(x)e^{it\xi_a X_a} \\
&-e^{-it\xi_a X_a}\Bigl\{\mc{T}[\eta]e^{it\xi_a X_a}\Bigr\}e^{-it\xi_a X_a}A_\mu(x)e^{it\xi_a X_a} \\
=&-e^{-it\xi_a X_a}\eta(x)A_\mu(x)e^{it\xi_a X_a} \\
&-e^{-it\xi_a X_a}\mc{T}[\eta]\Bigl\{e^{it\xi_a X_a}e^{-it\xi_a X_a }A_\mu(x)e^{it\xi_a X_a}\Bigr\} \\
&+e^{-it\xi_a X_a}e^{it\xi_a X_a}\mc{T}[\eta]\Bigl\{e^{-it\xi_a X_a}A_\mu(x)e^{it\xi_a X_a}\Bigr\} \\
=&-e^{-it\xi_a X_a}\eta(x)A_\mu(x)e^{it\xi_a X_a} \\
&-e^{-it\xi_a X_a}\mc{T}[\eta]\Bigl\{A_\mu(x)e^{it\xi_a X_a}\Bigr\} \\
&+\mc{T}[\eta]\Bigl\{e^{-it\xi_a X_a}A_\mu(x)e^{it\xi_a X_a}\Bigr\}
\end{align*}
ここまで変形する.次に第六項目は
\begin{align*}
&ie^{-it\xi_a X_a}[\eta(x)+\mc{T}[\eta]]e^{it\xi_a X_a}[\partial_\mu e^{-it\xi_a X_a}]e^{it\xi_a X_a} \\
=&ie^{-it\xi_a X_a}\eta(x)e^{it\xi_a X_a}[\partial_\mu e^{-it\xi_a X_a}]e^{it\xi_a X_a} \\
&+ie^{-it\xi_a X_a}\Bigl\{\mc{T}[\eta]e^{it\xi_a X_a}\Bigr\}[\partial_\mu e^{-it\xi_a X_a}]e^{it\xi_a X_a} \\
=&-ie^{-it\xi_a X_a}\eta(x)[\partial_\mu e^{it\xi_a X_a}] \\
&+ie^{-it\xi_a X_a}\mc{T}[\eta]\Bigl\{e^{it\xi_a X_a}[\partial_\mu e^{-it\xi_a X_a}]e^{it\xi_a X_a}\Bigr\} \\
&-ie^{-it\xi_a X_a}e^{it\xi_a X_a}\mc{T}[\eta]\Bigl\{[\partial_\mu e^{-it\xi_a X_a}]e^{it\xi_a X_a}\Bigr\} \\
=&-ie^{-it\xi_a X_a}\eta(x)[\partial_\mu e^{it\xi_a X_a}] \\
&-ie^{-it\xi_a X_a}\mc{T}[\eta][\partial_\mu e^{it\xi_a X_a}] \\
&-i\mc{T}[\eta]\Bigl\{[\partial_\mu e^{-it\xi_a X_a}]e^{it\xi_a X_a}\Bigr\}
\end{align*}
ここまで変形する.ここで用いたのは以下の関係式だ.
\begin{align*}
0=&\partial_\mu 1=\partial_\mu [e^{it\xi_a X_a}e^{-it\xi_a X_a}] \\
=&[\partial_\mu e^{it\xi_a X_a}]e^{-it\xi_a X_a}+e^{it\xi_a X_a}[\partial_\mu e^{-it\xi_a X_a}] \\
\therefore \quad & e^{it\xi_a X_a}[\partial_\mu e^{-it\xi_a X_a}]e^{it\xi_a X_a}=-[\partial_\mu e^{it\xi_a X_a}]
\end{align*}
ここまでの変形を用いることによって,元の式は
\begin{align*}
&\left(\mc{T}^A[\eta_{-t\xi}]A_\mu \right)_{A\to A_{-t\xi}} \\
=&i\partial_\mu\Bigl\{ e^{-it\xi_a X_a}\eta(x)e^{it\xi_a X_a} \Bigr\}+i\partial_\mu\Bigl\{ e^{-it\xi_a X_a}\mc{T}[\eta]e^{it\xi_a X_a} \Bigr\} \\
&+e^{-it\xi_a X_a}A_\mu(x)\eta(x)e^{it\xi_a X_a}+e^{-it\xi_a X_a}A_\mu(x)\mc{T}[\eta]e^{it\xi_a X_a} \\
&-e^{-it\xi_a X_a}\eta(x)A_\mu(x)e^{it\xi_a X_a} \\
&-e^{-it\xi_a X_a}\mc{T}[\eta]\Bigl\{A_\mu(x)e^{it\xi_a X_a}\Bigr\} \\
&+\mc{T}[\eta]\Bigl\{e^{-it\xi_a X_a}A_\mu(x)e^{it\xi_a X_a}\Bigr\} \\
& -i[\partial_\mu e^{-it\xi_a X_a}]\eta(x)e^{it\xi_a X_a}-i[\partial_\mu e^{-it\xi_a X_a}]\mc{T}[\eta]e^{it\xi_a X_a}  \\
&-ie^{-it\xi_a X_a}\eta(x)[\partial_\mu e^{it\xi_a X_a}] \\
&-ie^{-it\xi_a X_a}\mc{T}[\eta][\partial_\mu e^{it\xi_a X_a}] \\
&-i\mc{T}[\eta]\Bigl\{[\partial_\mu e^{-it\xi_a X_a}]e^{it\xi_a X_a}\Bigr\}
\end{align*}
もう少しだ.第七項目と第十二項目以外が全て相殺すればゴールとなる.さて,第一項目が
\begin{align*}
&i\partial_\mu\Bigl\{ e^{-it\xi_a X_a}\eta(x)e^{it\xi_a X_a} \Bigr\} \\
=&i[\partial_\mu e^{-it\xi_a X_a}]\eta(x)e^{it\xi_a X_a} +ie^{-it\xi_a X_a}\eta(x)[\partial_\mu e^{it\xi_a X_a}] \\
&+ie^{-it\xi_a X_a}[\partial_\mu \eta(x)]e^{it\xi_a X_a}
\end{align*}
であり,第三項目と第五項目が
\begin{align*}
&e^{-it\xi_a X_a}A_\mu(x)\eta(x)e^{it\xi_a X_a}-e^{-it\xi_a X_a}\eta(x)A_\mu(x)e^{it\xi_a X_a} \\
=&e^{-it\xi_a X_a}[A_\mu(x),\eta(x)]e^{it\xi_a X_a}
\end{align*}
とまとめられ,第四項目と第六項目が
\begin{align*}
&e^{-it\xi_a X_a}A_\mu(x)\mc{T}[\eta]e^{it\xi_a X_a} -e^{-it\xi_a X_a}\mc{T}[\eta]\{A_\mu(x)e^{it\xi_a X_a}\} \\
=&-e^{-it\xi_a X_a}\{\mc{T}[\eta]A_\mu(x)\}e^{it\xi_a X_a} \\
=&-e^{-it\xi_a X_a}\{i\partial_\mu \eta(x)+[A_\mu,\eta(x)]\}e^{it\xi_a X_a}
\end{align*}
とまとめられることを用いると,第一項目と第三項目・第五項目と第四項目・第六項目と第八項目と第十項目は相殺してゼロとなる.残りの第二項目と第九項目と第十一項目はライプニッツ則で明らかに相殺する.したがって結局残るのは
\begin{align*}
&\left(\mc{T}^A[\eta_{-t\xi}]A_\mu \right)_{A\to A_{-t\xi}} \\
=&\mc{T}[\eta]\Bigl\{e^{-it\xi_a X_a}A_\mu(x)e^{it\xi_a X_a}\Bigr\}-i\mc{T}[\eta]\Bigl\{[\partial_\mu e^{-it\xi_a X_a}]e^{it\xi_a X_a}\Bigr\} \\
=&\mc{T}[\eta][A_{-t\xi}(x)]_\mu
\end{align*}
となって,(22.7.12)が示される.\par
(22.7.12)と無矛盾条件(22.7.5)を用いれば
\begin{align*}
\mc{T}[\eta]G_b[y;A_{-t\xi}]=&\Bigl(\mc{T}^A[\eta_{-t\xi}] G_b[y;A]\Bigr)_{A\to A_{-t\xi}} \\
=&\int d^4x[\eta_{-t\xi}(x)]_\gamma \left(\mc{T}^A_\gamma(x)G_b[y,A]\right)_{A\to A_{-t\xi}} \\
=&\int d^4x[\eta_{-t\xi}(x)]_\gamma \left(\mc{T}^A_b(y)G_\gamma[x,A]+iC_{\gamma b \alpha }\delta^4(x-y)G_\alpha[x;A]\right)_{A\to A_{-t\xi}} \\
=&\int d^4x[\eta_{-t\xi}(x)]_\gamma \left(\mc{T}^A_b(y)G_\gamma[x,A]\right)_{A\to A_{-t\xi}}+iC_{\gamma b\alpha}[\eta_{-t\xi}(x)]_\gamma G_\alpha[y;A_{-t\xi}] \\
=&\int d^4x[\eta_{-t\xi}(x)]_a \left(\mc{T}^A_b(y)G_a[x,A]\right)_{A\to A_{-t\xi}}+iC_{\gamma ba}[\eta_{-t\xi}(x)]_\gamma G_a[y;A_{-t\xi}]
\end{align*}
最後の等号では(22.7.6)を用いた.ここまで計算した結果を用いると,(22.7.7)の両辺に$\mc{T}[\eta]$を作用させると
\begin{align*}
\mc{T}[\eta]\Gamma[\xi,A]=&-i\mc{T}[\eta]\int^1_0 dt \int \xi_b(y) G_b[y;A_{-t\xi}]d^4y \\
=&-i\int^1_0 dt \int d^4y\Bigl\{\mc{T}[\eta]\xi_b(y)\Bigr\} G_b[y;A_{-t\xi}]d^4y-i\int^1_0 dt \int \xi_b(y)\Bigl\{ \mc{T}[\eta]G_b[y;A_{-t\xi}]\Bigr\} \\
=&-i\int^1_0 dt\int d^4y\left\{-i\frac{\partial}{\partial t}[\eta_{-t\xi}(y)]_b+i C_{a\gamma b}\xi_a(y)[\eta_{-t\xi}(y)]_\gamma\right\}G_b[y;A_{-t\xi}] \\
&-i\int^1_0 dt\int d^4y\xi_b(y)\biggl\{\int d^4x[\eta_{-t\xi}(x)]_a \left(\mc{T}^A_b(y)G_a[x,A]\right)_{A\to A_{-t\xi}} \\
&\qquad+iC_{\gamma ba}[\eta_{-t\xi}(x)]_\gamma G_a[y;A_{-t\xi}]\biggr\} \\
=&-\int^1_0 dt \int d^4y \frac{\partial}{\partial t}[\eta_{-t\xi}(y)]_b G_b[y;A_{-t\xi}]+\int^1_0 dt \int d^4y C_{a\gamma b}\xi_a(y) [\eta_{-t\xi}(y)]_\gamma G_b[y;A_{-t\xi}] \\
&-i\int^1_0 dt \int d^4y \int d^4x \xi_b(y)[\eta_{-t\xi}(x)]_a \left(\mc{T}^A_b(y)G_a[x,A]\right)_{A\to A_{-t\xi}} \\
&-\int^1_0 dt \int d^4y C_{a\gamma b}\xi_a(y) [\eta_{-t\xi}(y)]_\gamma G_b[y;A_{-t\xi}] \\
=&-\int^1_0 dt \int d^4y\frac{\partial}{\partial t}[\eta_{-t\xi}(y)]_b G_b[y;A_{-t\xi}] \\
&-i\int^1_0 dt \int d^4x  [\eta_{-t\xi}(x)]_a \left(\int d^4y\xi_b(y)\mc{T}^A_b(y)G_a[x,A]\right)_{A\to A_{-t\xi}} \\
=&\int^1_0 dt \int d^4y\left\{ -\frac{\partial}{\partial t}[\eta_{-t\xi}(y)]_b G_b[y;A_{-t\xi}]-i[\eta_{-t\xi}(y)]_a \left(\mc{T}^A[\xi]G_a[y,A]\right)_{A\to A_{-t\xi}} \right\}
\end{align*}
となる.また
\begin{align*}
i\Bigl( \mc{T}^A[\xi]A_\mu (x) \Bigr)_{A\to A_{-t\xi}}=&-(\partial_\mu \xi_\alpha +C_{\alpha \beta \gamma}\xi_\gamma A_{\beta\mu})_{A\to A_{-t\xi}} T_\alpha \\
=&(-\partial_\mu \xi_\alpha T_\alpha +i[T_\beta,T_\gamma]\xi_\gamma A_{\beta\mu})_{A\to A_{-t\xi}} \\
=&(-\partial_\mu \xi_a X_a +i[A_\mu,\xi_aX_a])_{A\to A_{-t\xi}} \quad \because \xi_i=0 \\
=&-\partial_\mu \xi_a X_a +i[[A_{-t\xi}]_\mu,\xi_aX_a] \\
=&-\partial_\mu \xi_a X_a +i[e^{-it\xi_a X_a}A_\mu e^{it\xi_a X_a}-i[\partial_\mu e^{-it\xi_a X_a}]e^{it\xi_a X_a},\xi_aX_a]\\
=&-\partial_\mu \xi_a X_a +i[e^{-it\xi_a X_a}A_\mu e^{it\xi_a X_a},\xi_aX_a]+[[\partial_\mu e^{-it\xi_a X_a}]e^{it\xi_a X_a},\xi_aX_a]
\end{align*}
一方
\begin{align*}
\frac{\partial}{\partial t}[A_{-t\xi}(x)]_\mu=&-i\xi_a X_a e^{it\xi_a X_a}A_\mu e^{it\xi_a X_a}+e^{-it\xi_a X_a}A_\mu e^{it\xi_a X_a}i\xi_a X_a \\
&-i\left[\partial_\mu \{ -i\xi_a X_a e^{-it\xi_a X_a} \} \right]e^{it\xi_a X_a}-i\left[\partial_\mu e^{-it\xi_a X_a} \right](i\xi_a X_a)e^{it\xi_a X_a} \\
=&i[e^{-it\xi_a X_a}A_\mu e^{it\xi_a X_a},\xi_aX_a] \\
&-\partial_\mu \xi_a X_a-\xi_a X_a [\partial_\mu e^{-it\xi_a X_a}]e^{it\xi_a X_a}+[\partial_\mu e^{-it\xi_a X_a}]e^{it\xi_a X_a}\xi_a X_a
\end{align*}
であるから
\begin{align*}
\frac{\partial}{\partial t}[A_{-t\xi}(x)]_\mu=i\Bigl( \mc{T}^A[\xi]A_\mu (x) \Bigr)_{A\to A_{-t\xi}}
\end{align*}
が示される.したがって
\begin{align*}
\mc{T}[\eta]\Gamma[\xi,A]=&\int^1_0 dt \int d^4y\left\{ -\frac{\partial}{\partial t}[\eta_{-t\xi}(y)]_b G_b[y;A_{-t\xi}]-i[\eta_{-t\xi}(y)]_b \left(\mc{T}^A[\xi]G_b[y,A]\right)_{A\to A_{-t\xi}} \right\} \\
=&\int^1_0 dt \int d^4y\left\{ -\frac{\partial}{\partial t}[\eta_{-t\xi}(y)]_b G_b[y;A_{-t\xi}]-[\eta_{-t\xi}(y)]_b \frac{\partial}{\partial t}G_b[y;A_{-t\xi}] \right\} \\
=&-\int^1_0 dt \int d^4y \frac{\partial}{\partial t}\Bigl\{ [\eta_{-t\xi}(y)]_b G_b[y;A_{-t\xi}] \Bigr\} \\
=&-\int d^4y \biggl[[\eta_{-t\xi}(y)]_b G_b[y;A_{-t\xi}] \biggr]^1_0
\end{align*}
となる.$t=1$では(22.7.10)は
\begin{align*}
\eta_{-\xi}(x)&\equiv \exp(-i\xi_a(x)X_a)\left[\eta(x)+\mc{T}[\eta]\right]\exp(i\xi_a(x)X_a)
\end{align*}
となるが,これは(22.7.4)の誤植を訂正した式により,破れていない対称性部分群$H$の生成子の線型結合だと分かるので,任意の破れた対称性の生成子$X_b$の係数$[\eta_{-\xi}(y)]_b$はゼロとなる!また,(22.7.10)と(22.7.8)から$t=0$では$\eta_{-t\xi}(y)=\eta(y)$かつ$[A_{-t\xi}(y)]_\mu=A_\mu(y)$となる.したがって(22.7.16)より
\begin{align*}
\mc{T}[\eta]\Gamma[\xi,A]=\int d^4y [\eta_0(y)]_b G_b[y;A_0]=\int d^4y \eta_b(y)G_b[y;A]
\end{align*}
(22.7.6)(22.7.9)よりこれは
\begin{align*}
\mc{T}_\beta(y)\Gamma[\xi,A]=G_\beta[y;A]
\end{align*}
と同等の結果であることがわかる!これで証明が完成した!\par
この解は唯一の解ではない.しかしこれは$\xi=0$でゼロになるという条件のもとでは唯一の解となる.これを見るには,ゲージ変換によって
\begin{align*}
\exp\left(-i \int \eta_\beta (x)\mc{T}_\beta(x)d^4x \right)\Gamma[\xi,A]=\Gamma[\xi',A']
\end{align*}
となることに注目すれば良い.
\begin{align*}
\exp(z)=1+\int^1_0 dt \exp(zt)z
\end{align*}
を用いると(22.7.1)より
\begin{align*}
\Gamma[\xi',A']=&\Gamma[\xi,A]+\int^1_0 dt \exp\left(-it \int \eta_\beta (x)\mc{T}_\beta(x)d^4x \right)\left[ -i \int \eta_\beta (y)\mc{T}_\beta(y)d^4y \right] \Gamma[\xi,A] \\
=&\Gamma[\xi,A]-i\int^1_0 dt \exp\left(-it \int \eta_\beta (x)\mc{T}^A_\beta(x)d^4x \right)\int \eta_b(y)G_b[y;A]d^4y
\end{align*}
となる.もし$\eta_a=-\xi_a$かつ$\eta_i=0$とすると
\begin{align*}
&g\gamma(\xi)=\gamma(\xi')h \\
&\exp(-i\xi_a X_a +0)\exp(i\xi_a X_a)=\exp(i\xi'_a X_a) \\
&\xi'_a=0
\end{align*}
となり,この場合は仮定により$\Gamma[\xi',A']$はゼロとなる.汎関数演算子$\exp(it \int \xi_a(x)\mc{T}^A_a(x)d^4x)$は,単にゲージ場に対しゲージパラメータ$\Lambda_\beta=-t\xi_a(x)$のゲージ変換(15.1.17)を行うだけであることを念頭におくと,以下の表式を得る.
\begin{align*}
\Gamma[\xi,A]=&-i\int^1_0 dt \exp\left(it \int \xi_b (x)\mc{T}^A_b(x)d^4x \right)\int \xi_b(y)G_b[y;A]d^4y \\
=&-i\int^1_0 dt \int \xi_b(y) \exp \left(-i \int (-t\xi_b (x))\mc{T}^A_b(x)d^4x \right)G_b[y;A]d^4y \\
=&-i\int^1_0 dt \int \xi_b(y) G_b[y;A_{-t\xi}]d^4y
\end{align*}
つまり,$\xi=0$では$\Gamma[\xi,A]=0$の条件のもとでは(22.7.7)は唯一の形であることがわかる!

\vskip\baselineskip

(22.7.7)は擬スカラーのゴールドストンボゾン8重項の電弱相互作用を調べるのに使うことができるらしい.ゲージ場が実際に無くても,ゴールドストンボゾン自身の相互作用については重要な結論を導くことができるらしい.$A=0$の場合,(22.7.8)は純ゲージ場となる.
\begin{align*}
[A_{-t\xi}(x)]_\mu=&-i\left[\partial_\mu \exp(-it\xi_a(x)X_a)\right]\exp(it\xi_a(x)X_a) \\
=&-i[\partial_\mu V(t\xi(x))]V^{-1}(t\xi(x))
\end{align*}
(一般に$A=-i(\partial_\mu g)g^{-1}$の形のゲージ場を純ゲージ場と呼ぶ.)ここで
\begin{align*}
V(t\xi(x))\equiv \exp(-it \xi_a X_a)
\end{align*}
と定義した.$A_{\alpha\mu}(x)=0$では,アノマリー項はゲージ場の関数であるから消えて
\begin{align*}
\mc{T}_\beta(x)\Gamma[\xi,0]=0
\end{align*}
となる.したがって,(22.7.21)を(22.7.7)に使うと,ゴールドストンボゾン場$\xi_a(x)$の$G$不変局所汎関数を得る.ただし,それは一般には19.6節で構成したような$\xi_a(x)$とその微分の$G$不変な関数の時空座標についての積分ではない.\par
最も単純な例は,対称性が完全に破れている場合だ.この場合は条件(22.7.6)は空で,アノマリーの対称形(22.3.38)を使える.
\begin{align*}
G_a[x;A]=-\frac{i}{24\pi^2}\epsilon^{\kappa\nu\lambda\rho}\mr{Tr}\left\{X_a \left[\partial_\kappa A_\nu \partial_\lambda A_\rho -\frac{1}{2} i\partial_\kappa A_\nu A_\lambda A_\rho +\frac{1}{2}iA_\kappa \partial_\nu A_\lambda A_\rho -\frac{1}{2}iA_\kappa A_\nu \partial_\lambda A_\rho \right]\right\}
\end{align*}
ここで$X_a$は理論の左手フェルミオン(反フェルミオンとの差異が重要なときには反フェルミオンも含む)によって与えられる群の生成子の特定の表現だ.($T_\alpha$ではないのは,仮定(22.7.6)より$T_i$に比例するアノマリーはないとしているからだ.)例えば,フェルミオン数を保存する場合には(22.3.4)で与えられる.このアノマリー形の場合には,(22.7.24)に(22.7.21)を使って,(22.7.24)のトレース内の項で$\epsilon^{\kappa\nu\lambda\rho}$と縮約をとっても生き残る項は,全て以下に比例することが分かる.
\begin{align*}
\mr{Tr}\left\{ X_a (\partial_\kappa V)V^{-1} (\partial_\nu V) V^{-1} (\partial_\lambda V) V^{-1} (\partial_\rho V) V^{-1} \right\}
\end{align*}
なぜなら,例えば(22.7.24)で出てくる微分
\begin{align*}
\partial_\kappa [A_{-t\xi}]_\nu=&-i\partial_\kappa \{(\partial_\nu V)V^{-1}\} \\
=&-i(\partial_\kappa \partial_\nu V)V^{-1}-i(\partial_\nu V)(\partial_\kappa V^{-1}) \\
=&-i(\partial_\kappa \partial_\nu V)V^{-1}+i(\partial_\nu V)V^{-1}(\partial_\kappa V)V^{-1}
\end{align*}
の第一項目は$\epsilon^{\kappa\nu\lambda\rho}$の反対称性によりゼロとなるからだ.よって第二項目のみが残り,(22.7.24)の各項でこの操作をすれば
\begin{align*}
&\epsilon^{\kappa\nu\lambda\rho}\mr{Tr}\left\{ X_a \partial_\kappa [A_{-t\xi}]_\nu \partial_\lambda [A_{-t\xi}]_\rho \right\} \\
=&-\epsilon^{\kappa\nu\lambda\rho}\mr{Tr}\Bigl\{ X_a (\partial_\nu V)V^{-1}(\partial_\kappa V)V^{-1}(\partial_\rho V)V^{-1}(\partial_\lambda V)V^{-1} \Bigr\} \\
=&-\epsilon^{\kappa\nu\lambda\rho}\mr{Tr}\Bigl\{ X_a (\partial_\kappa V)V^{-1}(\partial_\nu V)V^{-1}(\partial_\lambda V)V^{-1}(\partial_\rho V)V^{-1} \Bigr\}
\end{align*}
(22.7.24)第一項目の係数は$-1$となる.
\begin{align*}
&\epsilon^{\kappa\nu\lambda\rho}\mr{Tr}\left\{ X_a \left[-\frac{1}{2}i \partial_\kappa [A_{-t\xi}]_\nu [A_{-t\xi}]_\lambda [A_{-t\xi}]_\rho \right] \right\} \\
=&-\frac{1}{2}\epsilon^{\kappa\nu\lambda\rho}\mr{Tr}\Bigl\{X_a(\partial_\nu V)V^{-1}(\partial_\kappa V)V^{-1}(\partial_\lambda V)V^{-1}(\partial_\rho V)V^{-1}\Bigr\} \\
=&\frac{1}{2}\epsilon^{\kappa\nu\lambda\rho}\mr{Tr}\Bigl\{X_a(\partial_\kappa V)V^{-1}(\partial_\nu V)V^{-1}(\partial_\lambda V)V^{-1}(\partial_\rho V)V^{-1}\Bigr\}
\end{align*}
第二項目は$+1/2$となる.
\begin{align*}
&\epsilon^{\kappa\nu\lambda\rho}\mr{Tr}\left\{ X_a \left[\frac{1}{2}i [A_{-t\xi}]_\kappa \partial_\nu [A_{-t\xi}]_\lambda [A_{-t\xi}]_\rho \right] \right\} \\
=&\frac{1}{2}\epsilon^{\kappa\nu\lambda\rho}\mr{Tr}\Bigl\{X_a(\partial_\kappa V)V^{-1}(\partial_\lambda V)V^{-1}(\partial_\nu V)V^{-1}(\partial_\rho V)V^{-1}\Bigr\} \\
=&-\frac{1}{2}\epsilon^{\kappa\nu\lambda\rho}\mr{Tr}\Bigl\{X_a(\partial_\kappa V)V^{-1}(\partial_\nu V)V^{-1}(\partial_\lambda V)V^{-1}(\partial_\rho V)V^{-1}\Bigr\}
\end{align*}
第三項目は$-1/2$となる.
\begin{align*}
&\epsilon^{\kappa\nu\lambda\rho}\mr{Tr}\left\{ X_a \left[-\frac{1}{2}i [A_{-t\xi}]_\kappa [A_{-t\xi}]_\nu \partial_\lambda [A_{-t\xi}]_\rho \right] \right\} \\
=&-\frac{1}{2}\epsilon^{\kappa\nu\lambda\rho}\mr{Tr}\Bigl\{X_a(\partial_\kappa V)V^{-1}(\partial_\nu V)V^{-1}(\partial_\rho V)V^{-1}(\partial_\lambda V)V^{-1}\Bigr\} \\
=&\frac{1}{2}\epsilon^{\kappa\nu\lambda\rho}\mr{Tr}\Bigl\{X_a(\partial_\kappa V)V^{-1}(\partial_\nu V)V^{-1}(\partial_\lambda V)V^{-1}(\partial_\rho V)V^{-1}\Bigr\}
\end{align*}
第四項目は$+1/2$となる.これらを足し合わせると
\begin{align*}
G_a[x;A]=&-\frac{i}{24\pi^2}\epsilon^{\kappa\nu\lambda\rho}\mr{Tr}\left\{X_a \left[\partial_\kappa A_\nu \partial_\lambda A_\rho -\frac{1}{2} i\partial_\kappa A_\nu A_\lambda A_\rho +\frac{1}{2}iA_\kappa \partial_\nu A_\lambda A_\rho -\frac{1}{2}iA_\kappa A_\nu \partial_\lambda A_\rho \right]\right\} \\
=&\frac{i}{48\pi^2}\epsilon^{\kappa\nu\lambda\rho}\mr{Tr}\left\{X_a (\partial_\kappa V)V^{-1} (\partial_\nu V) V^{-1} (\partial_\lambda V) V^{-1} (\partial_\rho V) V^{-1}\right\}
\end{align*}
となる.これを(22.7.7)に代入すると
\begin{align*}
\Gamma[\xi,0]=&i\int^1_0 dt \int \xi_b(y)G_b[y;A_{-t\xi}]d^4y \\
=&\frac{1}{48\pi^2}\epsilon^{\kappa\nu\lambda\rho} \int d^4y\xi_a (y)\int^1_0 dt \mr{Tr}\biggl\{X_a \Bigl[\partial_\kappa V \Bigl(t\xi(y)\Bigr) \Bigr]V^{-1}\Bigl(t\xi(y)\Bigr) \Bigl[\partial_\nu V\Bigl(t\xi(y)\Bigr)\Bigr] V^{-1}\Bigl(t\xi(y)\Bigr) \\
&\qquad \qquad \times  \Bigl[\partial_\lambda V\Bigl(t\xi(y)\Bigr)\Bigr] V^{-1}\Bigl(t\xi(y)\Bigr) \Bigl[\partial_\rho V\Bigl(t\xi(y)\Bigr)\Bigr] V^{-1}\Bigl(t\xi(y)\Bigr)\biggr\}
\end{align*}
約束したように,これは19.6節で構成したような,場と場の微分の不変関数の4次元積分ではない.たとえば,ゴールドストンボゾン場が弱いとき,これは
\begin{align*}
\Gamma[\xi,0]=&\frac{1}{48\pi^2}\epsilon^{\kappa\nu\lambda\rho} \int d^4y\int^1_0 dt \mr{Tr}\Bigl\{X_a \xi_a(y)(-it\partial_\kappa \xi_b X_b)(-it\partial_\nu \xi_c X_c)(-it \partial_\lambda \xi_d X_d)(-it\partial_\rho \xi_e X_e)\Bigr\} \\
&+O(\xi^6) \\
=&\frac{1}{48\pi^2}\epsilon^{\kappa\nu\lambda\rho} \mr{Tr}\left\{X_a X_b X_c X_d X_e\right\} \int^1_0 t^4 dt \int d^4y \xi_a \partial_\kappa \xi_b \partial_\nu \xi_c \partial_\lambda \xi_d \partial_\rho \xi_e +O(\xi^6) \\
=&\frac{1}{240\pi^2}\epsilon^{\kappa\nu\lambda\rho} \mr{Tr}\left\{X_a X_b X_c X_d X_e\right\} \int d^4y \xi_a \partial_\kappa \xi_b \partial_\nu \xi_c \partial_\lambda \xi_d \partial_\rho \xi_e +O(\xi^6)
\end{align*}
となる.19.6節で構成したように,ゴールドストンボゾン場は共変微分から発生するから,(22.7.26)のような形の最低次の項は持てない.

\vskip\baselineskip

実用上さらに重要な場合として,質量ゼロの$u,d,s$クォークを持つ量子色力学の$SU(3)\times SU(3)$カイラル対称性が,ゲルマン・ネーマンの対角$SU(3)$部分群に自発的に破れる例を考える.22.3節で述べたように,破れていない対称性である対角$SU(3)$部分群のベクトルカレントがアノマリーを持っていないように扱わなければならないため,この場合に対応するアノマリーとしてはバーディーン形(22.3.34)となる.
\begin{align*}
G_a[V,A]=-\frac{in}{16\pi^2}\epsilon^{\mu\nu\rho\sigma}\mr{Tr}\biggl\{t_a\biggl[ V_{\mu\nu}V_{\rho\sigma}+\frac{1}{3}A_{\mu\nu}A_{\rho\sigma}-\frac{32}{3}A_\mu A_\nu A_\rho A_\sigma \\
+\frac{8}{3}i\left( A_\mu A_\nu V_{\rho\sigma}+A_\mu V_{\rho\sigma}A_\nu +V_{\rho\sigma}A_\mu A_\nu \right) \biggr]  \biggr\}
\end{align*}
ここで$t_a$は(19.7.2)で定義されるゲルマン行列$\lambda_a$の半分$t_a=\lambda_a/2$だ.また$V_\mu,A_\mu,V_{\mu\nu},A_{\mu\nu}$は(22.3.35)-(22.3.37)と同様に定義される.そして$n$は各フレーバーのクォークの種類(色)の数だ.ここでは群の生成子を表す行列に小文字の$t$を使っているが,これはこのトレース$\mr{Tr}$では左手クォークについてのみ和をとり,左手反クォークについては和をとっていないからだ.(いわゆる,(22.3.4)の小文字の$t$かな.)場の強度テンソルはゲージ変換に対して$F\to g^{-1}F g$と変換するので,(22.7.21)のような純ゲージ場については自明にゼロになる.なぜなら,ゲージ変換前のゲージ場がゼロなのだから,ゼロから構成された場の強度テンソルはゼロで,ゲージ変換してもこれは変わらないからだ.すると生き残るのは第三項目だけで
\begin{align*}
G_a[V,A]=\frac{2in}{3\pi^2}\epsilon^{\mu\nu\rho\sigma}\mr{Tr}\{t_a [A_{-t\xi}]_\mu [A_{-t\xi}]_\nu [A_{-t\xi}]_\rho [A_{-t\xi}]_\sigma \}
\end{align*}
だ.これを(22.7.7)に使うと
\begin{align*}
\Gamma[\xi,0]=\frac{2n}{3\pi^2}\epsilon^{\mu\nu\rho\sigma}\int^1_0 dt \int d^4x \mr{Tr}\{\xi_a t_a [A_{-t\xi}]_\mu [A_{-t\xi}]_\nu [A_{-t\xi}]_\rho [A_{-t\xi}]_\sigma \}
\end{align*}
ここで注意すべきは,この$A_{-t\xi}$は(22.7.21)と同じではなくて,ベクトル場$V_\mu$と軸性ベクトル場$A_\mu$に分けた$A_\mu$だ.$[A_{-t\xi}]_\mu$が知りたいので,(22.7.21)を
\begin{align*}
[V_{-t\xi}(x)]_\mu +\gamma_5 [A_{-t\xi}(x)]_\mu =-i[\partial_\mu \exp(-it\xi_a \gamma_5 t_a)]\exp(it\xi_a \gamma_5 t_a)
\end{align*}
と書く.$(1+\gamma_5)/2$をかけて$\gamma_5$に比例しない項を集めれば
\begin{align*}
&\frac{1+\gamma_5}{2}[V_{-t\xi}(x)]_\mu+\frac{1+\gamma_5}{2}[A_{-t\xi}(x)]_\mu=-\frac{1+\gamma_5}{2}i[\partial_\mu \exp(-it\xi_a t_a)]\exp(+it\xi_a t_a) \\
\therefore \quad &\frac{1}{2}[V_{-t\xi}(x)]_\mu+\frac{1}{2}[A_{-t\xi}(x)]_\mu=-\frac{1}{2}i[\partial_\mu \exp(-it\xi_a t_a)]\exp(+it\xi_a t_a)
\end{align*}
$(1-\gamma_5)/2$をかけて同様に
\begin{align*}
&\frac{1-\gamma_5}{2}[V_{-t\xi}(x)]_\mu-\frac{1-\gamma_5}{2}[A_{-t\xi}(x)]_\mu=-\frac{1-\gamma_5}{2}i[\partial_\mu \exp(+it\xi_a t_a)]\exp(-it\xi_a t_a) \\
\therefore \quad &\frac{1}{2}[V_{-t\xi}(x)]_\mu-\frac{1}{2}[A_{-t\xi}(x)]_\mu=-\frac{1}{2}i[\partial_\mu \exp(+it\xi_a t_a)]\exp(-it\xi_a t_a)
\end{align*}
前者から後者を引けば,$[V_{-t\xi}(x)]_\mu$を消すことができて
\begin{align*}
[A_{-t\xi}(x)]_\mu=&-\frac{1}{2}i[\partial_\mu \exp(-it\xi_a t_a)]\exp(+it\xi_a t_a)+\frac{1}{2}i[\partial_\mu \exp(+it\xi_a t_a)]\exp(-it\xi_a t_a) \\
=&\frac{1}{2}i\exp(-it\xi_a t_a)[\partial_\mu \exp(+it\xi_a t_a)]+\frac{1}{2}i[\partial_\mu \exp(+it\xi_a t_a)]\exp(-it\xi_a t_a) \\
=&\frac{1}{2}i\exp(it\xi_a t_a)\exp(-2it\xi_a t_a)\biggl\{[\partial_\mu \exp(+it\xi_a t_a)]\exp(it\xi_a t_a)\biggr\}\exp-(it\xi_a t_a) \\
&+\frac{1}{2}i\exp(it\xi_a t_a)\exp(-2it\xi_a t_a)\biggl\{\exp(it\xi_a t_a)[\partial_\mu \exp(+it\xi_a t_a)]\biggr\}\exp(-it\xi_a t_a) \\
=&\frac{1}{2}i\exp(it\xi_a t_a)\exp(-2it\xi_a t_a)[\partial_\mu \exp(2it\xi_a t_a )]\exp(-it\xi_a t_a) \\
=&\frac{1}{2}i\exp(it\xi_a t_a)U^{-1}\Bigl(t\xi(x)\Bigr)\Bigl[\partial_\mu U\Bigl(t\xi(x)\Bigr)\Bigr]\exp(-it\xi_a t_a)
\end{align*}
となる.ここで
\begin{align*}
U\Bigl(t\xi \Bigr) \equiv \exp(2it\xi_a t_a )
\end{align*}
だ.これを(22.7.28)に使うと,
\begin{align*}
\Gamma[\xi,0]=&\frac{n}{24\pi^2}\epsilon^{\mu\nu\rho\sigma}\int^1_0 dt \int d^4x \mr{Tr}\biggl\{\xi_a t_a U^{-1}\Bigl(t\xi(x)\Bigr)\Bigl[\partial_\mu U\Bigl(t\xi(x)\Bigr)\Bigr] U^{-1}\Bigl(t\xi(x)\Bigr)\Bigl[\partial_\nu U\Bigl(t\xi(x)\Bigr)\Bigr] \\
&\times U^{-1}\Bigl(t\xi(x)\Bigr)\Bigl[\partial_\rho U\Bigl(t\xi(x)\Bigr)\Bigr] U^{-1}\Bigl(t\xi(x)\Bigr)\Bigl[\partial_\sigma U\Bigl(t\xi(x)\Bigr)\Bigr] \biggr\}
\end{align*}
となる.これは便利な5次元形にすることができる.$t$を5番目の座標として,$\xi_a(x,t)\equiv t\xi_a(x)$と定義する.すると
\begin{align*}
\Gamma[\xi,0]=&-\frac{in}{240\pi^2}\epsilon^{ijk\ell m}\int d^5z \mr{Tr}\biggl\{ U^{-1}\Bigl(\xi(z)\Bigr)\Bigl[\partial_i U\Bigl(\xi(z)\Bigr)\Bigr] U^{-1}\Bigl(\xi(z)\Bigr)\Bigl[\partial_j U\Bigl(\xi(z)\Bigr)\Bigr]  \\
&\times U^{-1}\Bigl(\xi(z)\Bigr)\Bigl[\partial_k  U\Bigl(\xi(z)\Bigr)\Bigr] U^{-1}\Bigl(\xi(z)\Bigr)\Bigl[\partial_\ell U\Bigl(\xi(z)\Bigr)\Bigr] U^{-1}\Bigl(\xi(z)\Bigr)\Bigl[\partial_m U\Bigl(\xi(z)\Bigr)\Bigr] \biggr\}
\end{align*}
と表せる.ここで$i,j$等は$1,2,3,0,5$の値をとる.$i=1,2,3,0$については$z^i=x^i$で,$z^5=t$だ.また$z^5$積分は$0\leq z^5 \leq 1$の領域についてする.5つの添え字$i,j,k,\ell,m$のどれでも5の値をとる可能性があることを勘定に入れて,余分な因子1/5を入れてある.このように表せる理由は,外微分形式(22.6.14)を使うと見やすい.
\begin{align*}
\Gamma[\xi,0]=&-\frac{in}{240\pi^2}\int \mr{Tr}\biggl\{ U^{-1}\Bigl(\xi(z)\Bigr)\Bigl[dz^i \partial_i U\Bigl(\xi(z)\Bigr)\Bigr] U^{-1}\Bigl(\xi(z)\Bigr)\Bigl[dz^j\partial_j U\Bigl(\xi(z)\Bigr)\Bigr]  \\
&\times U^{-1}\Bigl(\xi(z)\Bigr)\Bigl[dz^k \partial_k  U\Bigl(\xi(z)\Bigr)\Bigr] U^{-1}\Bigl(\xi(z)\Bigr)\Bigl[dz^\ell \partial_\ell U\Bigl(\xi(z)\Bigr)\Bigr] U^{-1}\Bigl(\xi(z)\Bigr)\Bigl[dz^m\partial_m U\Bigl(\xi(z)\Bigr)\Bigr] \biggr\} \\
=&-\frac{in}{48\pi^2}\int \mr{Tr}\biggl\{ U^{-1}\Bigl(t\xi(x)\Bigr)\Bigl[dt \frac{\partial}{\partial t} U\Bigl(t\xi(x)\Bigr)\Bigr] U^{-1}\Bigl(t\xi(x)\Bigr)\Bigl[dx^\mu \partial_\mu U\Bigl(t\xi(x)\Bigr)\Bigr]  \\
&\times U^{-1}\Bigl(t\xi(x)\Bigr)\Bigl[dx^\nu \partial_\nu  U\Bigl(t\xi(x)\Bigr)\Bigr] U^{-1}\Bigl(t\xi(x)\Bigr)\Bigl[dx^\rho \partial_\rho U\Bigl(t\xi(x)\Bigr)\Bigr] U^{-1}\Bigl(t\xi(x)\Bigr)\Bigl[dx^\sigma\partial_\sigma U\Bigl(t\xi(x)\Bigr)\Bigr] \biggr\}
\end{align*}
ここで(22.7.31)
\begin{align*}
U^{-1}\Bigl(t\xi(x)\Bigr)\Bigl[ \frac{\partial}{\partial t} U\Bigl(t\xi(x)\Bigr)\Bigr]=2i\xi_a t_a
\end{align*}
を使えば,元の形式と同じであることが分かる.$z^5=0$においては,他の$z^i$のどの成分$z^\mu$の値によらず$\xi_a(z)=0\times \xi_a(x)=0$は固定値ゼロをとるので,これらの$z^i=$の値を一つの点に同定して,(22.7.33)の積分領域を5次元球体と考えることができる.その4次元境界$z^5=1$は通常の時空だ.(19.8節参照)このようにして(22.7.33)は(19.8.1)と(19.8.3)で与えられるWZW作用の特別な場合となる.これはやはり整数$n$に比例する.\par
唯一の違いは,$n$はここでは色の数と特定されていることだ.19.8節では積分(19.8.3)が5次元球体の時空境界での$\xi_a(z)$の値のみに依存することを見た.したがって(22.7.33)を導く際に(19.8.2)は$\xi_a(x,t)=t\xi_a(x)$のみならず$\xi_a(x)$の5次元球体の内部への任意の接続$\xi_a(x) \to \xi_a(x,t)$の仕方について当てはまる.


\end{document}