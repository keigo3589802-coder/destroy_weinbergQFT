\documentclass[dvipdfmx]{jsarticle}
\let\headfont=\gtfamily
\usepackage[dvips]{graphicx}
\usepackage{amsmath}
\usepackage{mathrsfs} % 花文字\mathscr{M}, 筆記体\mathcal{M}, 黒板文字\mathbb{M},ドイツ文字\mathfrak{M}
\usepackage{bm} %太文字
\usepackage{amssymb}
\usepackage{latexsym}
\usepackage{braket}
\usepackage{tikz}
\usepackage{tikz-feynhand}
\usepackage{ulem}
\usepackage{bigdelim}
\usepackage{multirow}
\usepackage{tcolorbox}
\usepackage{here}
\tcbuselibrary{theorems,skins}
\usetikzlibrary{decorations}
\usepackage{color}

\usetikzlibrary{intersections, calc, arrows.meta}
 \usetikzlibrary{patterns}

\newfont{\bg}{cmr9 scaled\magstep4}
\newcommand{\bigzerol}{\smash{\lower1.0ex\hbox{\bg 0}}}
\newcommand{\bigzerou}{%
   \smash{\hbox{\bg 0}}}
\newcommand{\mcO}{\mathcal{O}}
\newcommand{\VAC}{\mathrm{VAC}}
\newcommand{\Slash}[1]{{\ooalign{\hfil/\hfil\crcr$#1$}}} %ファインマンのスラッシュ記号
\renewcommand{\mc}{\mathcal}
\newcommand{\mr}[1]{\mathrm{#1}}

% \textrm{Roman デフォルト}
% \textgt{Gothic 和文ゴシック体}*専門用語に
% \textbf{Boldface 太字}*専門用語(英語)に
% \textit{Italic 斜体}
% \textsl{Slanted ローマンを傾けただけ}
% \textsf{Sans Serif サンセリフ体}
% \texttt{Typewriter タイプライタ体、等幅}
% \textsc{Small Caps 小文字が大文字に}

\setlength{\textwidth}{\fullwidth}
\setlength{\textheight}{44\baselineskip}
\addtolength{\textheight}{\topskip}
\setlength{\voffset}{-0.6in}

\allowdisplaybreaks[4]

\makeatletter
  \renewcommand{\theequation}
  {\arabic{section}.\arabic{equation}}
  \@addtoreset{equation}{section}
 \makeatother

\title{\vspace{-1cm}\Huge{WeinbergQFT4}}
\author{Laplacyan}
\date{}
\begin{document}



\maketitle
\setcounter{part}{19}
\part{演算子積展開}
\setcounter{section}{20}
\subsection{展開:記述と導出}
ウィルソンは二つの演算子の積$A(x)B(y)$で$x\to y$での特異部分が,他の局所演算子の和
\begin{eqnarray*}
A(x)B(y)\to \sum_C F^{AB}_C(x-y) C(y)
\end{eqnarray*}
で与えられると仮定した.ここで$F^{AB}_C(x-y)$は特異なc関数だ\par
$A(x)B(y)$は質量次元で$d_A+d_B$で,$C(y)$が$d_C$なので,$F^{AB}_C(x-y)$は$d_A+d_B-d_C$,つまり$x-y$の$d_C-d_A-d_B$乗として振る舞う.\par
演算子$\mathcal{O}$に場や微分を加えると$d_{\mathcal{O}}$は増加するので,$F^{AB}_C(x-y)$の特異性の度合は,$C$が複雑になるにつれ,つまり$d_C$が大きくなるにつれ減少する($d_C-d_A-d_B$が大きくなるので,$(x-y)^{-n}$の$n$が大きくなる).演算子積展開の重要な性質は,これが演算子の関係式である点だ.つまり,これを$\bra{\beta}A(x)B(y)\ket{\alpha}$のような行列要素に使うとき,任意の$\ket{\alpha}$と$\bra{\beta}$の状態について同じ関数$F^{AB}_C(x-y)$が現れる.\par
\vskip\baselineskip

演算子$C(y)$が複雑になるにつれ(20.1.1)の特異性が弱まるため,この展開は積$A(x)B(y)$の$x\to y$での振る舞いを導く際に有効となる.上の,簡単なベキ勘定はくりこみの効果によって変更を受ける.\par
つまり(20.1.1)の展開はあるスケール$\mu$でくりこまれた演算子で行うべき行為であり,このために係数関数$F^{AB}_C(x-y)$に,$x-y$に加えて$\mu$が登場する.\par
20.3節にて,漸近的自由な理論において$F^{AB}_C(x-y)$は次元解析で分かる$d_C-d_A-d_B$乗に加えて$\log (x-y)^2$のベキも持つことが分かる.より一般の理論でも,各種演算子$C(y)$が複雑になるにしたがって,$F^{AB}_C(x-y)$の特異性が弱まると思われる.
\vskip\baselineskip

運動量空間でこれに対応するのは,$k\to\infty$において
\begin{eqnarray*}
\int d^{4} x e^{-i k x} A(x) B(0) \rightarrow \sum_{C} \int d^{4} x e^{-i k x} F_C^{A B}(x) C(0)=\sum_{C} V_{C}^{A B}(k) C(0)
\end{eqnarray*}
となり,これに対応して
\begin{eqnarray*}
\int d^{4} x e^{-i k x} T\left\{A(x) B(0) \right\} = \sum_{C} U_{C}^{A B}(k) C(0)
\end{eqnarray*}
となることだ.ここで$V_C^{AB}(k),U_C^{AB}(k)$は$k^\mu$の関数であり,級数$C(0)$の項が複雑になるにつれ,大きな$k$でより早く減少する.\par
\vskip\baselineskip

ここではより一般に,任意個数の演算子が運ぶ運度量が無限大になるというウィルソン展開を導く.この目的のために,Green関数
\begin{eqnarray*}
&&\braket{T\left\{A_{1}\left(x_{1}\right), A_{2}\left(x_{2}\right), \cdots, B_{1}\left(y_{1}\right), B_{2}\left(y_{2}\right), \cdots \right\} }_{0} \\
&=&\int \left[\prod_{\ell, z} d \phi_{\ell}(z)\right] a_{1}\left(x_{1}\right) a_{2}\left(x_{2}\right) \cdots b_{1}\left(y_{1}\right) b_{2}\left(y_{2}\right) \cdots \exp (i I[\phi])
\end{eqnarray*}
において,局所演算子$A_1(x_1),A_2(x_2)$等の引数が点$x$に近づき,他の局所演算子$B_1(y_1),B_2(y_2)$等の引数が固定された場合を考える.ここで小文字$a$と$b$は,$A$や$B$の中の場の演算子をc数の場で置き換えることを意味する.
さて,点$x$を半径$R$の球$B(R)$で囲む.$R$は$x_1,x_2$等との間の距離よりずっと大きく,また,$y_1,y_2$等との間の距離よりはずっと小さくとる.\par

\begin{figure}[H]
\centering
\begin{tikzpicture}
\draw(1,1) circle (2);
\draw(1,1) circle (0.02);
\draw (1,1)node[above]{$x$};

\draw(2,2) circle (0.02);
\draw (2,2)node[above]{$x_1$};

\draw(2,0) circle (0.02);
\draw (2,0)node[right]{$x_2$};

\draw(-0.5,0) circle (0.02);
\draw (-0.5,0)node[above]{$x_3$};

\draw(5,2) circle (0.02);
\draw (5,2)node[above]{$y_1$};

\draw(6,0) circle (0.02);
\draw (6,0)node[above]{$y_2$};

\draw[->] (2 - 0.1 , 2 - 0.1)--(1+0.1 ,1+0.1);
\draw[->] (2 - 0.1 , 0 + 0.1)--(1+0.1 ,1-0.1);
\draw[->] (-0.5 + 0.1 , 0 + 0.1)--(1-0.1 ,1-0.1);

\draw (1,1)--(1,-1);
\draw (1,-0.5)node[right]{$R$};

\end{tikzpicture}
\end{figure}
\noindent
作用は局所的だから,これを
\begin{eqnarray*}
I=\int_{z\in B(R)}d^4z\mathcal{L}(z)+\int_{z\notin B(R)}d^4z \mathcal{L}(z)
\end{eqnarray*}
と書くことができる.\par
\vskip\baselineskip

ここでは球の内部の場についての経路積分は,球の表面で場が球の外部の場と滑らかにつながるという境界条件で拘束されている.この\uwave{境界条件を除いては},球の内部の場についての経路積分は球の外の場の振る舞いに全く影響されない.\par
$\Rightarrow$境界条件,つまり球の表面についての振る舞いが分かれば積分は収束する.したがって球の内部の場についての積分は,\uwave{球の表面での場と場の微分の関数となる}.\par
また,球の表面は球の中心$x$が分かれば決まる.\par
$\Rightarrow$積分は\uwave{点$x$での場と,場の微分の関数となる}.\par
もしこの積分を$x$におけるc数の場の積$\mathcal{O}(x)$の級数として表すと,その係数は座標の差のみの関数$U_{\mathcal{O}}(x_1-x,x_2-x,\cdots)$となる.点$y_1,y_2$等は全て球$B(R)$のはるか外部にあるから,球の外部の作用からこの球を除いても$R\to 0$の極限では何の影響もない.($x_1,x_2,\cdots$が$x$に近づくにつれ$R$もゼロに近付ける.)\par

\begin{figure}[H]
\centering
\begin{tikzpicture}
\draw(1,1) circle (2);
\draw(1,1) circle (0.02);
\draw (1,1)node[above]{$x$};

\draw(2,2) circle (0.02);
\draw (2,2)node[above]{$x_1$};

\draw(2,0) circle (0.02);
\draw (2,0)node[right]{$x_2$};

\draw(-0.5,0) circle (0.02);
\draw (-0.5,0)node[above]{$x_3$};

\draw(5,2) circle (0.02);
\draw (5,2)node[above]{$y_1$};

\draw(6,0) circle (0.02);
\draw (6,0)node[above]{$y_2$};

\draw[->] (2 - 0.1 , 2 - 0.1)--(1+0.1 ,1+0.1);
\draw[->] (2 - 0.1 , 0 + 0.1)--(1+0.1 ,1-0.1);
\draw[->] (-0.5 + 0.1 , 0 + 0.1)--(1-0.1 ,1-0.1);

\draw (1,1)--(1,-1);
\draw (1,-0.5)node[right]{$R$};

\draw[->](7,1)--(9,1);
\draw (8 ,1)node[above]{$R\to 0$};

\draw (10,1) circle (0.35);
\draw (10,1) circle (0.02);
\draw (10,1)node[above]{$x$};

\draw(12,2) circle (0.02);
\draw (12,2)node[above]{$y_1$};

\draw(13,0) circle (0.02);
\draw (13,0)node[above]{$y_2$};

\end{tikzpicture}
\end{figure}
\noindent
$R\to 0$の極限で,$x_1,x_2,\cdots$が$x$に近づくとき(20.1.6)は
\begin{eqnarray*}
&&\braket{T\left\{A_{1}\left(x_{1}\right), A_{2}\left(x_{2}\right), \cdots, B_{1}\left(y_{1}\right), B_{2}\left(y_{2}\right), \cdots \right\} }_{0} \\
&=&\int \left[\prod_{\ell, z} d \phi_{\ell}(z)\right] \sum_o U_{o}^{A_1,A_2,\cdots}(x_1-x,x_2-x,\cdots )o(x) b_{1}\left(y_{1}\right) b_{2}\left(y_{2}\right) \cdots \exp (i I[\phi]) \\
&=&\sum_{\mathcal{O}}U^{A_1,A_2,\cdots}_\mathcal{O}(x_1-x,x_2-x,\cdots)\braket{T\left\{ \mathcal{O}(x),B_{1}\left(y_{1}\right), B_{2}\left(y_{2}\right), \cdots \right\} }_{0} \\
\end{eqnarray*}
ここで$\mathcal{O}(x)$は$o(x)$に対応するハイゼンベルグ表示の量子力学的演算子だ.\par
特に,$y$変数についてフーリエ変換をして適切な係数関数をかけると
\begin{eqnarray*}
\bra{\beta} T\left\{A_{1}\left(x_{1}\right), A_{2}\left(x_{2}\right), \cdots \right\}\ket{\alpha} \to \sum_{\mathcal{O}}U^{A_1,A_2,\cdots}_{\mathcal{O}}(x_1-x,x_2-x,\cdots)\bra{\beta}\mathcal{O}(x)\ket{\alpha}
\end{eqnarray*}
これは任意の状態について成り立つので,以下の一般的な演算子積展開が成立する.
\begin{eqnarray*}
T\left\{A_{1}\left(x_{1}\right), A_{2}\left(x_{2}\right), \cdots \right\} \to \sum_{\mathcal{O}}U^{A_1,A_2,\cdots}_{\mathcal{O}}(x_1-x,x_2-x,\cdots) \mathcal{O}(x)
\end{eqnarray*}

\newpage

\subsection{運動量の流れ}
\,OPEの簡単な場合を考える.一つの質量$m$の実スカラー場$\phi(x)$が$-g\phi^4/4!$の相互作用をする理論での,$(n+2)$点ファインマン振幅において,大きな運動量$k$が一本の線から流れ込み,別の線から流れ出し,他の外線の運動量は固定されているときの漸近的振る舞いを考える.\par
$n$個の粒子の散乱振幅の連結ダイアグラムの和で,外線から$k,p-k$が流入し,$p_1,p_2,\cdots,p_n$が流出し$p=p_1+p_2+\cdots +p_n$となっているものを$\Gamma(k,p_1,\cdots,p_n)$と定義する.ここで便利のために,$\Gamma(k,p_1,\cdots,p_n)$は運動量$k,p-k$の線のプロパゲータを含むが,固定された運動量$p_1,p_2,\cdots,p_n$のプロパゲータは含まないもの,としておく.\par

\begin{figure}[H]
\centering
\begin{tikzpicture}[scale=0.7]
\draw[very thick] (-4,1)--(-{sqrt(3)},1);
\draw (-3,1)node[above]{$k$};

\draw[very thick](-4,-1)--(-{sqrt(3)},-1);
\draw (-3,-1)node[above]{$p-k$};

\draw[very thick](1,{sqrt(3)})--(4,{sqrt(3)});
\draw (3,{sqrt(3)})node[below]{$p_1$};

\draw[very thick]({sqrt(3)},1)--(4,1);
\draw (3,1)node[below]{$p_2$};

\draw(3,0.2) circle (0.02);
\draw(3,0) circle (0.02);
\draw(3,-0.2) circle (0.02);

\draw[very thick](1,-{sqrt(3)})--(4,-{sqrt(3)});
\draw (3,-{sqrt(3)})node[above]{$p_n$};

\path[clip, preaction={draw, thick}] (0,0) circle (2);
\fill[draw=black, thick, pattern=north west lines] (-2,2) -- (2,2) -- (2,-2) -- (-2,-2) -- cycle;
\end{tikzpicture}
\end{figure}

摂動論の任意の有限次で,$k\to \infty$のときに以下が成立することを示したい.(これがこの節で示したいこと!)
\begin{eqnarray*}
\Gamma(k,p_1,\cdots,p_n) \to U_{\phi^2}(k)F_{\phi^2}(p_1,p_2,\cdots,p_n)+O(k^{-5})
\end{eqnarray*}
ここで$U_{\phi^2}(k)$は$k^{-4}$の大きさの項の和だ.\par
$F_{\phi^2}(p_1,\cdots,p_n)$は,$n$本の$\phi$の線と一つの$\phi^2$の挿入があり,これを有限にするために適切なくりこみ定数$Z_{\phi^2}$がかけられた振幅だ.
$F_{\phi^2}(p_1,\cdots,p_n)$はくりこまれた演算子$(\phi^2)_R(0)=Z_{\phi^2}\phi^2(0)$の行列要素であるから,(20.2.1)は
\begin{eqnarray*}
\int d^{4} x e^{-i k x} T\left\{\phi_{R}(x) \phi_{R}(0)\right\}_{C} \rightarrow U_{\phi^{2}}(k)\left(\phi^{2}(0)\right)_{R}
\end{eqnarray*}
が$k\to\infty$での主要な項であることを示している.ここで$C$は連結ダイアグラムのみを含めることを意味している.また,展開する演算子はこの節を通じて(20.1.3)において$C=\phi^2$のみとなる.\par

\vskip\baselineskip

ファインマン振幅の漸近的振る舞いを知るには,運動量空間の積分領域の中には外線(運動量が無限となる)と一部の内線の運動量が同じオーダーとなり,他の内線はそうはならない,という領域もあることに留意しなければならない.\par
ある部分ダイアグラム$\mathcal{S}$の外線が$k$の大きさの運動量を持つ積分領域からの$\Gamma(k,p_1,\cdots,p_n)$への寄与は,漸近的に$k^{D_\mathcal{S}}$となる.ここで$D_{\mathcal{S}}$は12.1節の法則に従って計算した部分ダイアグラム$\mathcal{S}$の次元:
\begin{eqnarray*}
D_{\mathcal{S}}=4-E_\phi(s_\phi+1)-4
\end{eqnarray*}
である.ここで$E_\phi$は\uwave{部分ダイアグラム$\mathcal{S}$における}外線の数で,最後の$-4$は,$k,p-k$のプロパゲータから来る因子である.スカラー場$\phi$では$s_\phi=0$である.もし$\mathcal{S}$がダイアグラムの残りの部分と$m$本の外線で繋がっているとすると,$D_{\mathcal{S}}=4-(2+m)-4=-2-m$となる.\par
したがって$\Gamma(k,p_1,\cdots,p_n)$の漸近的振る舞いは,大きな運動量$k$がダイアグラム全体か,または部分ダイアグラムのうち,外線の数$m$が最も少ないものを流れる時の運動量空間の積分で決まってくる!

\begin{figure}[H]
\centering
\begin{tikzpicture}[scale=0.5]

\draw[very thick] (-4,1)--(-{sqrt(3)},1);
\draw (-4,1)node[above]{$k$};

\draw[very thick](-4,-1)--(-{sqrt(3)},-1);
\draw (-4,-1)node[above]{$p-k$};

\begin{scope}
\path[clip, preaction={draw, thick}] (0,0) circle (2);
\fill[draw=black, thick, pattern=north west lines] (-2,2) -- (2,2) -- (2,-2) -- (-2,-2) -- cycle;
\end{scope}

\begin{scope}
\path[clip, preaction={draw, thick}] (8,0) circle (2);
\fill[draw=black, thick, pattern=north west lines] (6,2) -- (10,2) -- (10,-2) -- (6,-2) -- cycle;
\end{scope}

\draw[very thick](1,{sqrt(3)})--(7,{sqrt(3)});

\draw[very thick]({sqrt(3)},1)--({8-sqrt(3)},1);

\draw(4,0.2) circle (0.02);
\draw(4,0) circle (0.02);
\draw(4,-0.2) circle (0.02);
\draw (4.2,-0.5)node[below]{$m$本の線};

\draw[very thick](1,-{sqrt(3)})--(7,-{sqrt(3)});


\draw[very thick](9,{sqrt(3)})--(12,{sqrt(3)});
\draw (11,{sqrt(3)})node[above]{$p_1$};

\draw[very thick]({8+sqrt(3)},1)--(12,1);
\draw (11,1)node[below]{$p_2$};

\draw(11,-0.2) circle (0.02);
\draw(11,-0.4) circle (0.02);
\draw(11,-0.6) circle (0.02);

\draw[very thick](9,-{sqrt(3)})--(12,-{sqrt(3)});
\draw (11,-{sqrt(3)})node[below]{$p_n$};


\draw[dashed] (-3,2.5)--(2.35,2.5)--(2.35,-2.5)--(-3,-2.5)--(-3,2.5);
\draw (0,2.-4.5)node[below]{部分ダイアグラム$\mathcal{S}$};

\end{tikzpicture}
\end{figure}
\noindent
$n=0$のとき,これはダイアグラム全体のみだ.つまり積分の主要な部分は,どの線も$k$の大きさの運動量を持つような積分領域から来る.そのため,これは漸近的に$k^{-2}$の大きさとなる.ここでは連結ダイアグラムのみを考えているので,$n>0$ではこの項は除外される.

\begin{figure}[H]
\centering
\begin{tikzpicture}[scale=0.5]

\begin{scope}
\draw[very thick] (-4,1)--(-{sqrt(3)},1);
\draw (-3,1)node[above]{$k$};

\draw[very thick](-4,-1)--(-{sqrt(3)},-1);
\draw (-3,-1)node[above]{$-k$};

\draw (0,-2)node[below]{$n=0$};

\path[clip, preaction={draw, thick}] (0,0) circle (2);
\fill[draw=black, thick, pattern=north west lines] (-2,2) -- (2,2) -- (2,-2) -- (-2,-2) -- cycle;
\end{scope}

\draw [very thick](3,2)--(3,-2);

\begin{scope}
\draw[very thick] (4,1)--({8-sqrt(3)},1);
\draw (5,1)node[above]{$k$};

\draw[very thick](4,-1)--({8-sqrt(3)},-1);
\draw (5,-1)node[above]{$-k$};

\path[clip, preaction={draw, thick}] (8,0) circle (2);
\fill[draw=black, thick, pattern=north west lines] (6,2) -- (10,2) -- (10,-2) -- (6,-2) -- cycle;
\end{scope}

\begin{scope}
\draw[very thick](15,{sqrt(3)})--(18,{sqrt(3)});

\draw[very thick]({14+sqrt(3)},1)--(18,1);

\draw(17,0.2) circle (0.02);
\draw(17,0) circle (0.02);
\draw(17,-0.2) circle (0.02);

\draw[very thick](15,-{sqrt(3)})--(18,-{sqrt(3)});

\path[clip, preaction={draw, thick}] (14,0) circle (2);
\fill[draw=black, thick, pattern=north west lines] (12,2) -- (16,2) -- (16,-2) -- (12,-2) -- cycle;
\end{scope}

\draw (11,-2)node[below]{このダイアグラムは非連結なので除外される};

\end{tikzpicture}
\end{figure}
\noindent
($n=1$のときは$\phi\to-\phi$の対称性より禁止される.)$n=2$のとき,主要な寄与は全ダイアグラムと,運動量$p-k$と$k$を持つ外線が他の2本の外線と,2本の内線からなる橋で繋がっている部分ダイアグラムとの両方からなる.したがってそれは漸近的に$k^{-4}$の振る舞いをする.

\begin{figure}[H]
\centering
\begin{tikzpicture}[scale=0.5]

\begin{scope}
\draw[very thick] (-4,1)--(-{sqrt(3)},1);
\draw (-3,1)node[above]{$k$};

\draw[very thick](-4,-1)--(-{sqrt(3)},-1);
\draw (-3.5,-1)node[above]{$p-k$};

\draw[very thick]({sqrt(3)},1)--({6-sqrt(3)},1);
\draw (3,1)node[above]{$p_1$};

\draw[very thick]({sqrt(3)},-1)--({6-sqrt(3)},-1);
\draw (3,-1)node[above]{$p_2$};

\path[clip, preaction={draw, thick}] (0,0) circle (2);
\fill[draw=black, thick, pattern=north west lines] (-2,2) -- (2,2) -- (2,-2) -- (-2,-2) -- cycle;
\end{scope}

\draw [very thick](5,2)--(5,-2);
\draw (0,-2)node[below]{$n=2$のダイアグラム};

\begin{scope}
\draw[very thick] (6,1)--({10-sqrt(3)},1);
\draw (6.5,1)node[above]{$k$};

\draw[very thick](6,-1)--({10-sqrt(3)},-1);
\draw (6.5,-1)node[above]{$p-k$};

\draw[very thick]({10+sqrt(3)},1)--({16-sqrt(3)},1);

\draw[very thick]({10+sqrt(3)},-1)--({16-sqrt(3)},-1);

\path[clip, preaction={draw, thick}] (10,0) circle (2);
\fill[draw=black, thick, pattern=north west lines] (8,2) -- (12,2) -- (12,-2) -- (8,-2) -- cycle;
\end{scope}

\begin{scope}
\draw[very thick]({16+sqrt(3)},1)--({22-sqrt(3)},1);
\draw (19,1)node[above]{$p_1$};

\draw[very thick]({16+sqrt(3)},-1)--({22-sqrt(3)},-1);
\draw (19,-1)node[above]{$p_2$};

\path[clip, preaction={draw, thick}] (16,0) circle (2);
\fill[draw=black, thick, pattern=north west lines] (14,2) -- (18,2) -- (18,-2) -- (14,-2) -- cycle;
\end{scope}
\draw[dashed] (7.5,2.5)--(12.5,2.5)--(12.5,-2.5)--(7.5,-2.5)--(7.5,2.5);
\draw (10,-3)node[below]{$\mathcal{S}$};

\end{tikzpicture}
\end{figure}
\noindent
$n\geq 4$($n$が奇数のときは$n=1$のときと同様に禁止)のとき,主要な項は運動量$k$と$p-k$を持つ2本の外線が他の$n>2$本の外線と,2本の内線からなる橋で繋がっている部分ダイアグラムのみ(他は$k^{-5}$より大きな次数となり主要でなくなる)から生じ,これも$k^{-4}$の振る舞いをする.

\begin{figure}[H]
\centering
\begin{tikzpicture}[scale=0.5]

\begin{scope}
\draw[very thick] (-4,1)--(-{sqrt(3)},1);
\draw (-3.5,1)node[above]{$k$};

\draw[very thick](-4,-1)--(-{sqrt(3)},-1);
\draw (-3.5,-1)node[above]{$p-k$};

\draw[very thick]({sqrt(3)},1)--({6-sqrt(3)},1);

\draw[very thick]({sqrt(3)},-1)--({6-sqrt(3)},-1);

\path[clip, preaction={draw, thick}] (0,0) circle (2);
\fill[draw=black, thick, pattern=north west lines] (-2,2) -- (2,2) -- (2,-2) -- (-2,-2) -- cycle;
\end{scope}

\begin{scope}
\draw[very thick](7,{sqrt(3)})--(10,{sqrt(3)});
\draw (9,{sqrt(3)})node[below]{$p_1$};

\draw[very thick]({6+sqrt(3)},1)--(10,1);
\draw (9,1)node[below]{$p_2$};

\draw(9,0.2) circle (0.02);
\draw(9,0) circle (0.02);
\draw(9,-0.2) circle (0.02);

\draw[very thick](7,-{sqrt(3)})--(10,-{sqrt(3)});
\draw (9,-{sqrt(3)})node[above]{$p_n$};

\path[clip, preaction={draw, thick}] (6,0) circle (2);
\fill[draw=black, thick, pattern=north west lines] (4,2) -- (8,2) -- (8,-2) -- (4,-2) -- cycle;
\end{scope}
\draw[dashed] (-2.5,2.5)--(3,2.5)--(3,-2.5)--(-2.5,-2.5)--(-2.5,2.5);
\draw (0,-3)node[below]{$\mathcal{S}$};

\end{tikzpicture}
\end{figure}

$n=2$と$n\geq 4$のダイアグラムの解析は,一般のダイアグラムがこれらの2粒子からなる橋を複数含むために複雑となる.\par
まず$n=2$の場合を考える.$I(k,k^{\prime},p)$を($p_1=k^{\prime},p_2=p-k^{\prime}$として)「$\Gamma(k,p_1,p_2)$に寄与し,2粒子既約な,全てのダイアグラムの和」と定義する.2粒子既約とはここでは,「運動量$k$と$p-k$を持つ2本の外線の対と,運動量$k'$と$p-k'$を持つ2本の外線の対が,内線を2本切っても分離されないもの」として定義する.\par

\begin{figure}[H]
\centering
\begin{tikzpicture}[scale=0.5]

\begin{scope}
\draw[very thick] (-4,1)--(-{sqrt(3)},1);
\draw (-3.5,1)node[above]{$k$};

\draw[very thick](-4,-1)--(-{sqrt(3)},-1);
\draw (-3.5,-1)node[above]{$p-k$};

\draw[very thick]({sqrt(3)},1)--({6-sqrt(3)},1);

\draw[very thick]({sqrt(3)},-1)--({6-sqrt(3)},-1);

\path[clip, preaction={draw, thick}] (0,0) circle (2);
\fill[draw=black, thick, pattern=north west lines] (-2,2) -- (2,2) -- (2,-2) -- (-2,-2) -- cycle;
\end{scope}

\begin{scope}
\draw[very thick]({6+sqrt(3)},1)--(10,1);
\draw (9,1)node[above]{$k''$};

\draw[very thick]({6+sqrt(3)},-1)--(10,-1);
\draw (9.5,-1)node[above]{$p-k''$};

\path[clip, preaction={draw, thick}] (6,0) circle (2);
\fill[draw=black, thick, pattern=north west lines] (4,2) -- (8,2) -- (8,-2) -- (4,-2) -- cycle;
\end{scope}

\draw[very thick](0,2)--(0,-2);
\draw[very thick](2.5,1.5)--(3,0.5);
\draw[very thick](2.5,-0.5)--(3,-1.5);

\draw(3,-2)node[below]{←分離できる→};
\end{tikzpicture}
\end{figure}
\noindent
これより$\Gamma-I$は2本の内線を切れば$p-k,k$の対と$p-k',k'$の対が分離できるダイアグラムからなり,したがって次のように書かれる.
\begin{eqnarray*}
\Gamma(k;k',p-k')-I(k,k',p)=\int d^4k'' I(k,k'',p)\Gamma(k'';k',p-k')
\end{eqnarray*}
(20.2.3)の右辺の振る舞いを調べるために,まず核$I(k,k',p)$において$k'$と$p$を固定しておいて$k\to \infty$としたときの漸近的振る舞いを見る.
$I(k,k',p)$の主な寄与は外線が4本のときの$k^{-4}$の大きさだ.なぜなら,$I(k,k',p)$は2粒子既約であるため,どの内線も$k$程度の大きさの部分ダイアグラムと,そうでない他の部分ダイアグラムとで分解すると,その間の外線は3本以上の線である必要があり,その(どの内線も$k$の大きさで構成された)部分ダイアグラムは外線を5本以上持つことになる.
\begin{figure}[H]
\centering
\begin{tikzpicture}[scale=0.5]

\begin{scope}
\draw[very thick] (-4,1)--(-{sqrt(3)},1);
\draw (-3.5,1)node[above]{$k$};

\draw[very thick](-4,-1)--(-{sqrt(3)},-1);
\draw (-3.5,-1)node[above]{$p-k$};

\draw[very thick](1,{sqrt(3)})--(5,{sqrt(3)});

\draw[very thick](2,0)--(4,0);

\draw[very thick](1,-{sqrt(3)})--(5,-{sqrt(3)});

\path[clip, preaction={draw, thick}] (0,0) circle (2);
\fill[draw=black, thick, pattern=north west lines] (-2,2) -- (2,2) -- (2,-2) -- (-2,-2) -- cycle;
\end{scope}

\begin{scope}
\draw[very thick]({6+sqrt(3)},1)--(10,1);
\draw (10,1)node[above]{$k''$};

\draw[very thick]({6+sqrt(3)},-1)--(10,-1);
\draw (10,-1)node[above]{$p-k''$};

\path[clip, preaction={draw, thick}] (6,0) circle (2);
\fill[draw=black, thick, pattern=north west lines] (4,2) -- (8,2) -- (8,-2) -- (4,-2) -- cycle;
\end{scope}
\draw[very thick](6,2)--(6,-2);
\draw[very thick](0,2)--(0,-2);

\draw (0,-2)node[below]{内線が$k$程度};
\draw (6,-2)node[below]{他のダイアグラム};
\end{tikzpicture}
\end{figure}
\noindent
つまり.$k^{-5}$以上の速さで減少することになる.これにより主要な寄与は$I(k,k',p)$全体が,どの内線も$k$の大きさを持つものであると分かる.これにより
\begin{eqnarray*}
I(k,k',p)=k^{-4}I\left(1,\frac{k'}{k},\frac{p}{k}\right)
\end{eqnarray*}
を$k'$や$p$で微分すると
\begin{eqnarray*}
\frac{\partial I}{\partial k'}&=&k^{-4}\left(\frac{k'}{k}\right)'\frac{\partial}{\partial (k'/k)}I\left(1,\frac{k'}{k},\frac{p}{k}\right) \\
&=&k^{-5}I_{k'}'\left(1,\frac{k'}{k},\frac{p}{k}\right) \\
\frac{\partial I}{\partial p}&=&k^{-5}I_{p}'\left(1,\frac{k'}{k},\frac{p}{k}\right)
\end{eqnarray*}
となって,$k^{-1}$だけ漸近的振る舞いが落ちる.すなわち$k$が大きいところでは$k',p$を変化させても関数$I(k,k',p)$はほぼ変化せず,これは$k$のみの関数であると見なすことができる.よって
\begin{eqnarray*}
I(k,k',p)\to I_{\infty}(k)
\end{eqnarray*}
となる.ここで$I_\infty(k)$は$k$のみの関数で$k^{-4}$の大きさだ.\par
\vskip\baselineskip
(20.2.3)の右辺で,単に$I(k,k'',p)\to I_\infty(k)$とはできない.なぜなら$k$がどれだけ大きくなろうと,$k''$に関する積分範囲が$-\infty\sim\infty$なので,$k''$が$k$の大きさのところから大きな寄与を受けるからだ.\par
これに対応するため,数学的帰納法に基づく「あるトリック」使う.

\begin{figure}[H]
 \centering
\begin{tikzpicture}
\draw(0,0) circle(0.03);
\draw(-0.5,1/3) circle(0.03);
\draw(-0.5,-1/3) circle(0.03);

\draw (-1.5,1)--(1.5,-1);
\draw (-1.5,-1)--(1.5,1);

\draw (-1.4,1)node[right]{$p-k$};
\draw (-1.4,-1)node[right]{$k$};
\draw (1.5,1)node[right]{$p-k'$};
\draw (1.5,-1)node[right]{$k'$};

\draw[->] (2,0)--(1,0.4);
\draw[->] (2,0)--(1,-0.4);

\draw (2,0)node[right]{ここのプロパゲータは$\Gamma$の定義より,含まない};
\end{tikzpicture}
\end{figure}
$\Gamma(k;k',p-k')$の最低次では,2本の裸のプロパゲータがついた一つの頂点で表される.
\begin{eqnarray*}
\Gamma(k;k',p-k')_{最低次}&=&-i(2\pi)^4g \frac{-i}{(2\pi)^4}\frac{1}{(k^2+m^2)}\frac{-i}{(2\pi)^4}\frac{1}{((p-k)^2+m^2)} \\
&=&\frac{ig}{(2\pi)^4(k^2+m^2)((p-k^2)+m^2)}
\end{eqnarray*}
これは(20.2.1)の振る舞いをする.
\begin{eqnarray*}
\Gamma(k;k',p-k')_{最低次}=k^{-4}\times \frac{ig}{(2\pi)^4(1+m^2/k^2)((p/k -1)^2+m^2/k^2)}
\end{eqnarray*}
$n=2$で$g$の次数$=N$まで(20.2.1)が成り立つと仮定する.すなわち,この次数までで$k\to \infty$で漸近的振る舞いが
\begin{eqnarray*}
\Gamma(k;k',p-k')\to U_{\phi^2}(k) F_{\phi^2}(k',p-k')+O(k^{-5})
\end{eqnarray*}
という形になると仮定する.この振る舞いが$g$の次数$=N+1$でも成り立つことを確かめることができれば,数学的帰納法により(20.2.1)が証明できる!\par
(20.2.3)を次のように書き換えよう.
\begin{eqnarray*}
\underset{N+1次までの展開}{\uwave{\Gamma(k;k',p-k')}}&=&I(k,k',p) +\int d^4k'' \underset{少なくとも1次}{\uwave{I(k,k''p)}}\underset{←これより,高々N次なので(20.2.5)が使える}{\uwave{\left[ \Gamma(k'';k',p-k')-U_{\phi^2}(k'')F_{\phi^2}(k',p-k') \right]}} \\
&+&F_{\phi^2}(k',p-k')\int d^4k'' I(k,k'',p)U_{\phi^2}(k'')
\end{eqnarray*}
$I$は(図20.2を見れば明らかなように)少なくとも$g$について1次なので,2項目の括弧の中は高々$N$次であるから(20.2.5)が適用できる.

\begin{eqnarray*}
\int_{k''\sim k} \underset{k^{+4}}{\uwave{d^4 k''}} \underset{k^{-4}}{\uwave{I(k,k'',p)}}\underset{k^{-5}}{\uwave{\left[ \Gamma(k'';k'p-k')-U_{\phi^2}(k'')F_{\phi^2}(k',p-k') \right]}}
\end{eqnarray*}
(注意!$k$ではなく$k''$の次数で$k''^{-5}$,かつ積分が$k''\sim k$の領域だということ)
すなわち$k''$が$k$の大きさの積分領域からの寄与は$k^{-5}$のように減少する.よって$k''が\infty$に近い領域での積分は$k''$が有限の領域に比べ無視でき,さらに$k''$が有限のときは(20.2.4)が適用できて,以下の収束する積分が得られる.
\begin{eqnarray*}
I_{\infty}(k)\int d^4k' \left[ \Gamma(k'';k',p-k')-U_{\phi^2}(k'')F_{\phi^2}(k',p-k') \right]
\end{eqnarray*}
さらに,(20.2.4)を導いた方法を全く同様にして(ただし今回は$k''$は$k$の大きさなので$p$のみを固定して),この積分の中で$I(k,k',p)$は漸近的に$I(k,k')\equiv I(k,k',0)$で置き換えられる.したがって$k\to\infty$では(20.2.6)は
\begin{eqnarray*}
I(k; k', p-k') &&\to F_{\phi^2}(k', p-k') \int d^{4} k'' I (k, k'') U_{\phi^2}\left(k''\right) \\
&&+I_{\infty}(k)\left\{1+\int d^4 k'' \left[ \Gamma\left(k'', k' , p-k' \right)-U_{\phi^{2}}\left(k'' \right) F_{\phi^{2}}\left(k', p-k'\right)\right]\right\}
\end{eqnarray*}
となる.このため$g$について$N+1$次までの$U_{\phi^2}(k),F_{\phi^2}(k',p-k')$を,以下の等式が成り立つとして定義する.
\begin{eqnarray*}
U_{\phi^2}(k)&=&CI_{\infty}+\int d^4 k' I(k,k')U_{\phi^2}(k') \\
F_{\phi^2}(k',p-k')&=&C^{-1}\left\{ 1+\int d^4 k'' [\Gamma(k'';k',p-k')-U_{\phi^2}(k'')F_{\phi^2}(k',p-k')] \right\}
\end{eqnarray*}
ここで$C$は任意に選べる.これらの定義より,(20.2.5)($N$次での仮定)から(20.2.7)($N+1$の次数)が導かれる.実際
\begin{eqnarray*}
U_{\phi^2}(k)F_{\phi^2}(k',p-k')&=&CI_{\infty}(k)F_{\phi^2}(k',p-k')\\
&&+\int d^4 k'' I(k,k'')U_{\phi^2}(k'')F_{\phi^2}(k',p-k') \quad (U_{\phi^2}だけ展開)\\
=I_{\infty}&(k)&\left\{1+\int d^4k'' \left[\Gamma(k'';k',p-k')-U_{\phi^2}(k'')F_{\phi^2}(k',p-k')\right]\right\} \\
&+&F_{\phi^2}(k',p-k')\int d^4 k'' I(k,k'')U_{\phi^2}(k'') \\
&=&(20.2.7)
\end{eqnarray*}
よって数学的帰納法より証明が完了した!


\vskip\baselineskip


あるくりこみ点$k'=k(\mu)$と$p=p(\mu)$で$F_{\phi^2}(k',p-k')=1$を満たすように定数$C$を選ぶと便利だ.ここで$k(\mu),p(\mu)$は$\mu$の大きさの基準的な運動量だ.そうすると
\begin{eqnarray*}
C=1+\int d^4 k'' \Gamma(k'';k(\mu),p(\mu)-k(\mu))-\int d^4 k''U_{\phi^2}(k'')
\end{eqnarray*}
と具体的に$C$を選ぶことができる.
(20.2.5)を用いると,($F_{\phi^2}=1$なので)$k\to\infty$での(20.2.10)の二つの積分の発散は相殺することが分かる.(20.2.9)はこれにより
\begin{eqnarray*}
CF_{\phi^2}(k',p-k')&=&1+\int d^4 k'' \Gamma(k'';k',p-k')-\int d^4 k'' U_{\phi^2}(k'')F_{\phi^2}(k',p-k')\\
\mathrm{LHS}&=& \left\{ 1+\int d^4 k'' \Gamma(k'';k(\mu),p(\mu)-k(\mu)) \right\}F_{\phi^2}(k',p-k') \\
&&-\int d^4k''U_{\phi^2}(k'')F_{\phi^2}(k'-,p-k')
\end{eqnarray*}
右辺第三項目と左辺第二項目が等しいことが分かるので
\begin{align*}
&\left\{ 1+\int d^4 k'' \Gamma(k'';k(\mu),p(\mu)-k(\mu)) \right\}F_{\phi^2}(k',p-k')= 1+\int d^4 k'' \Gamma(k'';k',p-k') \\
&F_{\phi^2}(k',p-k')=\left[ 1+\int d^4 k'' \Gamma(k'';k(\mu),p(\mu)-k(\mu)) \right]^{-1} \left[1+\int d^4 k'' \Gamma(k'';k',p-k')\right] \\
&\equiv Z_{\phi^2}\left[ 1+\int d^4 k'' \Gamma(k'';k',p-k') \right]
\end{align*}
ここで$Z_{\phi^2}$は複合演算子$\phi^2$のくりこみ定数と解釈すれば,演算子$(\phi^2)_R=Z_{\phi^2}\phi^2$が有限な2粒子行列要素$F_{\phi^2}(k,p-k)$を持ち,$F_{\phi^2}(k,p-k)$は$k=k(\mu),p=p(\mu)$で1になるように定義されたものだと解釈できる.

$U_{\phi^2}(k)$や$F_{\phi^2}(k,p-k)$を(20.2.8)(20.2.11)を用いて計算するのは便利ではない.それよりも$\Gamma(k;k',p-k')$を計算し,(20.2.5)と比較して$U_{\phi^2}(k),F_{\phi^2}(k,p-k)$を読み取る方が便利だ.1ループの次数で一般的な2-2粒子散乱振幅である関数(12.2.26)を用いよう.

\begin{figure}[H]
  \centering
\begin{tikzpicture}[scale=0.5]
\draw[very thick] (-4,1)--(-{sqrt(3)},1);
\draw (-3,1)node[above]{$p_1=k$};

\draw[very thick](-4,-1)--(-{sqrt(3)},-1);
\draw (-4,-1)node[above]{$p_2=p-k$};

\draw[very thick]({sqrt(3)},1)--({6-sqrt(3)},1);
\draw (3,1)node[above]{$p'_1=k'$};

\draw[very thick]({sqrt(3)},-1)--({6-sqrt(3)},-1);
\draw (4,-1)node[above]{$p'_2=p-k'$};

\path[clip, preaction={draw, thick}] (0,0) circle (2);
\fill[draw=black, thick, pattern=north west lines] (-2,2) -- (2,2) -- (2,-2) -- (-2,-2) -- cycle;
\end{tikzpicture}
\end{figure}
(12.2.26)の関数$F$に運動量$k$と$p-k$の線のプロパゲータの積をかけると,1ループの次数で
\begin{align*}
\Gamma(k;k',p-k')&=\left[\frac{-i}{(2 \pi)^{4}} \frac{1}{p^{2}+m^{2}}\right]\left[\frac{-i}{(2 \pi)^{4}} \frac{1}{(k-p)^{2}+m^{2}}\right]\left[-i(2 \pi)^{4} F\right]\\
&=\left[\frac{-i}{(2 \pi)^{4}\left(p^{2}+m^{2}\right)}\right]\left[\frac{-i}{(2 \pi)^{4}\left((k-p)^{2}+m^{2}\right)}\right]\left[-i(2 \pi)^{4} g\right] \\
&\times \left\{1-\frac{g}{32 \pi^{2}} \int_{0}^{1} d x\left\{ \ln \left(\frac{m^{2}+4 x(1-x) \mu^{2} / 3}{m^{2}-sx(1-x)}\right) \right. \right. \\
+& \left.\left. \ln \left(\frac{m^{2}+4x(1-x) \mu^{2} / 3}{m^{2}-tx(1-x)}\right)+\ln \left(\frac{m^{2}+4 x(1-x) \mu^{2} / 3}{m^{2}-u x(1-x)}\right) \right\} + \cdots \right\}
\end{align*}
ここで$s,t,u$はマンデルスタム変数
\begin{eqnarray*}
s&=&-(p_1+p_2)^2=-p^2 \\
t&=&-(p_1-p'_1)^2=-(k-k')^2 \\
u&=&-(p_1-p_2)^=-(p-k-k')^2
\end{eqnarray*}
($p_1=k,p_2=p-k,p_1'=k',p_2'=p-k'$であることを用いた)であり,$\mu$はくりこみスケール,そして$g$はくりこまれた結合定数(すなわち(12.2.26)の$g_R$)でファインマン振幅の$s=t=u=-4\mu^2/3$における$F$の値であったことを思い出そう.\par
$k\to \infty$でマンデルスタム変数は$s=-p^2,t\to-k^2,u\to -k^2$であるから,$\Gamma$は$k\to \infty$で以下の漸近的な振る舞いをする.
\begin{eqnarray*}
\Gamma(k;k',p-k')\to \frac{ig}{(2\pi)^4(k^2)^2}\left\{ 1-\frac{g}{32\pi^2} \int^1_0 dx \left\{ \ln \left(\frac{m^{2}+4 x(1-x) \mu^{2} / 3}{m^{2}+p^2x(1-x)}\right) \right. \right. \\
\left.\left.+2\ln \left(\frac{m^{2}+4 x(1-x) \mu^{2} / 3}{m^{2}+k^2x(1-x)}\right) \right\} +\cdots \right\}
\end{eqnarray*}
$g^2$の次数では,これは
\begin{align*}
&U_{\phi^2}(k)=\frac{ig}{(2\pi)^4(k^2)^2}\left\{ 1-\frac{g}{16\pi^2}\int^1_0 \ln \left(\frac{m^2+4x(1-x)\mu^2/3}{m^2+k^2x(1-x)}\right)+\cdots \right\} \\
&F_{\phi^2}(k,p-k)= 1-\frac{g}{32\pi^2}\int^1_0 \ln \left( \frac{m^2+4x(1-x)\mu^2/3}{m^2+p^2x(1-x)} \right)+\cdots
\end{align*}
とすると(20.2.5)と一致する.(実際にこれらを掛けた$U_{\phi^2}(k)F_{\phi^2}(k',p-k')$は$\Gamma(k;k',p-k')$と$g^2$の次数までで一致していることが確認できる)
ここで演算子$\phi^2$のくりこみ点$k(\mu),p(\mu)$を$p^2(\mu)=4\mu^2/3$となるように結合定数$g$のくりこみ点と関連付けていたので,くりこみ点$p^2=4\mu^2/3$のとき,($\ln1=0$なので)実際に$F_{\phi^2}(k,p-k)=1$が確認できる.\par
3巻の内容をよく覚えていれば気付くけど,これは(18.1.9)
\begin{eqnarray*}
F(p)=1-\frac{g}{32\pi^2}\int^1_0 \ln \left( \frac{\Lambda^2}{m^2+p^2x(1-x)} \right) +\cdots
\end{eqnarray*}
を,今度はくりこみ点$p^2(\mu)=4\mu^2/3$でくりこんだもの
\begin{align*}
&Z_{\phi^2}F(4\mu^2/3)=1 \\
&Z_{\phi^2}=1+\frac{g}{32\pi^2}\int^1_0 \ln \left( \frac{\Lambda^2}{m^2+4x(1-x)\mu^2/3} \right)\cdots \\
&F_{R}(p)=Z_{\phi^2}F(p)=1-\frac{g}{32\pi^2}\int^1_0 \ln \left( \frac{m^2+4x(1-x)\mu^2/3}{m^2+p^2x(1-x)} \right)+\cdots
\end{align*}
と同じものだ.そして(20.2.11)は同様に(18.1.8)を示していることが分かる.そしてこの$F(p)$が表すダイアグラムは図18.3である.(20.2.11)の第一項目は図18.3の左図であるから1であり,第二項はその右図である.\par
ここでようやく$F_{\phi^2}$の物理的意味が分かる.すなわち$\phi^2$によってもたらされる運動量が$p$で,持ち出される運動量が$k,p-k$の頂点関数がこの$F_{\phi^2}(k,p-k)$であったのだ.そしてこの$\phi^2$はOPEで展開される(今回は唯一の)演算子である.\par
この一連の繋がりが補遺の理解に非常に関わってくる.



\vskip\baselineskip


さて,固定された運動量を持つ外線の本数$n$が2より大きい場合を考える.p7~p8にかけての議論より,$k\to \infty$の極限では,$\Gamma(k;p_1,p_2,\cdots,p_n)$の主要なダイアグラムは,運動量$k$と$p-k$を持つ2本の外線が固定された運動量を持つ$n$本の外線から,1対の内線2本を切ることによって分離できるものだと分かる.

\begin{figure}[H]
  \centering
\begin{tikzpicture}[scale=0.5]

\draw[very thick] (-4,1)--(-{sqrt(3)},1);
\draw (-3,1)node[above]{$k$};

\draw[very thick](-4,-1)--(-{sqrt(3)},-1);
\draw (-3,-1)node[above]{$p-k$};

\begin{scope}
\path[clip, preaction={draw, thick}] (0,0) circle (2);
\fill[draw=black, thick, pattern=north west lines] (-2,2) -- (2,2) -- (2,-2) -- (-2,-2) -- cycle;
\end{scope}

\begin{scope}
\path[clip, preaction={draw, thick}] (6,0) circle (2);
\fill[draw=black, thick, pattern=north west lines] (4,2) -- (8,2) -- (8,-2) -- (4,-2) -- cycle;
\end{scope}

\draw[very thick]({sqrt(3)},1)--({6-sqrt(3)},1);
\draw (3,1)node[above]{$k'$};

\draw[very thick]({sqrt(3)},-1)--({6-sqrt(3)},-1);
\draw (3,-1)node[above]{$p-k'$};

\draw[very thick](7,{sqrt(3)})--(10,{sqrt(3)});
\draw (9,{sqrt(3)})node[below]{$p_1$};

\draw[very thick]({6+sqrt(3)},1)--(10,1);
\draw (9,1)node[below]{$p_2$};

\draw(9,0.2) circle (0.02);
\draw(9,0) circle (0.02);
\draw(9,-0.2) circle (0.02);

\draw[very thick](7,-{sqrt(3)})--(10,-{sqrt(3)});
\draw (9,-{sqrt(3)})node[above]{$p_n$};

\draw (0,-2)node[below]{$I$};
\draw (6,-2)node[below]{$\Gamma$};
\draw (3,2)node[above]{主要なダイアグラム};
\end{tikzpicture}
\end{figure}
\noindent
すなわち
\begin{eqnarray*}
\Gamma(k;p_1,\cdots,p_n)\to \int d^4k' I(k,k',p)\Gamma(k',p_1,\cdots,p_n)
\end{eqnarray*}
以前と同様,右辺において核$I(k,k',p)$の$k\to\infty$における漸近的極限を単純には用いることができない.これは$k'$の積分領域が$k$の大きさのところで大きな寄与を受けるからだ.これは(20.2.17)を以下のように書き換えることで対処できる.
\begin{eqnarray*}
\Gamma(k;p_1,\cdots,p_n)\to&& \int d^4 k' I(k,k',p)\left[  \Gamma(k';p_1,\cdots,p_n)-U_{\phi^2}(k')F_{\phi^2}(p_1,\cdots,p_n)  \right] \\
&&+F_{\phi^2}(p_1,\cdots,p_n)\int d^4 k' I(k,k',p)U_{\phi^2}(k')
\end{eqnarray*}
前と同様,ある次数$N$までで
\begin{eqnarray*}
\Gamma(k,p_1,\cdots p_n)\to U_{\phi^2}(k)F_{\phi^2}(p_1,\cdots,p_n)+O(k^{-5})
\end{eqnarray*}
が成り立つとする.まず第一項目については$I_\infty(k')$が少なくとも1次なので括弧の中に仮定(20.2.19)が適用できる.すると括弧の中が$k^{-5}$の大きさなので$k'$が$k$の大きさのところからの寄与は$k\to \infty$で小さくなり,したがって$k'$積分の領域は有限のところから寄与があるが,このときは(20.2.4)の極限が使える.第二項目については,$n=2$のときに定義した(20.2.8)を使う.これらにより
\begin{eqnarray*}
\Gamma(k;p_1,\cdots,p_n)&\to&I_{\infty}(k)\int d^4 k'\left[ \Gamma(k',p_1,\cdots,p_n)- U_{\phi^2}(k')F_{\phi^2}(p_1,\cdots,p_n) \right] \\
&&+F_{\phi^2}(p_1,\cdots,p_n)\left[U_{\phi^2}(k)-CI_{\infty}(k)  \right] \\
&&=U_{\phi^2}(k)F_{\phi^2}(p_1,\cdots,p_n) \\
&&+I_{\infty}(k)\left\{ \int d^4 k'\left[ \Gamma(k',p_1,\cdots,p_n)- U_{\phi^2}(k')F_{\phi^2}(p_1,\cdots,p_n) \right] \right. \\
&&\biggl. -CF_{\phi^2}(p_1,\cdots,p_n) \biggr\}
\end{eqnarray*}
これは
\begin{eqnarray*}
CF_{\phi^2}(p_1,\cdots,p_n)=\int d^4 k'\left[ \Gamma(k',p_1,\cdots,p_n)- U_{\phi^2}(k')F_{\phi^2}(p_1,\cdots,p_n) \right] \quad (*)
\end{eqnarray*}
を仮定すれば$N+1$の次数で(20.2.19)と一致することが確認できる!\par
最後の式$(*)$については,(20.2.10)(20.2.12)を使って(今回の(20.2.10)はくりこみ点$p_i(\mu)$で$F_{\phi^2}(p_1(\mu),p_2(\mu),\cdots,p_n(\mu))=1$となるようにして)
\begin{align*}
&\mathrm{LHS}=\left[1+\int d^4 k' \Gamma(k',p_1(\mu),\cdots,p_n(\mu))-\int d^4 k' U_{\phi^2}(k')\right]F_{\phi^2}(p_1,\cdots,p_n) \\
&\mathrm{RHS}=\int d^4k'\Gamma(k';p_1,\cdots,p_n)-\int d^4k'U_{\phi^2}(k')F_{\phi^2}(p_1,\cdots,p_n) \\
&\Rightarrow F_{\phi^2}(p_1,\cdots,p_n)=\left[ 1+ \int d^4k''\Gamma(k'';p_1(\mu),\cdots,p_n(\mu)) \right]^{-1}  \int  d^4 k'\Gamma(k';p_1,\cdots,p_n) \\
&\qquad\qquad\qquad\qquad =Z_{\phi^2}\int  d^4 k'\Gamma(k';p_1,\cdots,p_n)
\end{align*}
が導かれる.これは,$F_{\phi^2}(p_1,\cdots,p_n)$はくりこまれた演算子$(\phi^2)_R\equiv Z_{\phi^2}\phi^2$の行列要素であり,(20.2.1)は演算子積の表式(20.2.2)に対応することを意味している,と解釈できる.特に,$n=2$のときと$n\geq 4$のときの証明で用いた(20.2.8)は共通している,すなわちどちらの場合の$U_{\phi^2}(k)$も同じ関数方程式を満たしている.$U_{\phi^2}(k)$はもちろん$p_1\sim p_n$に依らないし$F_{\phi^2}(p_1,\cdots,p_n)$が満たす関数方程式の形に依らないので,$n$や$p_1\sim p_n$がどんな値を取ろうとも,$U_{\phi^2}(k)$は同じ係数関数だと分かる!これが証明したかったことだ!

\newpage

\subsection*{補遺}
次の節に進む前に補遺を見ておいた方が良いので,それを先に書いておく.\par
任意の2本以上の外線の運動量が大きい振幅の漸近的振る舞いを,任意個数の場の因子と微分を含み次元がある限界$N$以下の演算子を取り入れることで考察する.
\vskip\baselineskip
文字$\ell,\ell'$等は,ファインマンダイアグラムか部分ダイアグラムに入るか出ていく特定の種類の外線$i$の集合を意味する.\par
文字$k,k'$等はそのような線の4元運動量の集合で,その和がある固定された値$p$に等しい,という条件を満たすものを意味する.(例えば20.2節ならば,$k_1+k_2=p$で$k_1=k,k_2=p-k$,$p_1+p_2+\cdots +p_n=p$だった.)\par
振幅$\Gamma_{\ell\ell'}(k,k',p)$は運動量$k$を持ち入射する線の集合が$\ell$で,運動量$k'$を持ち出ていく線の集合が$\ell '$の,全てのダイアグラムの和だ.これには集合$\ell$の裸のプロパゲータを含むが$\ell'$については含まない,と定義する.

\vskip\baselineskip
12.1節p279で示されているように,「結合定数も全て含めた運動量積分の次数は
\begin{eqnarray*}
4-\sum_f E_f (1+s_f) \quad (E_fは種類fの粒子の外線の数)
\end{eqnarray*}
となる.」だから,部分ダイアグラムの次元$d(\ell,\ell'')$は
\begin{eqnarray*}
d(\ell,\ell'')=4-\sum_{i\in \ell,\ell''}(1+s_i)-\sum_{i\in \ell}(2-2s_i)
\end{eqnarray*}
最後の項は,集合$\ell$の線のプロパゲータから生じる.ここで$s_i$は種類$i$の線のスピンだ.スカラーと\uwave{ゲージ・ボゾン}では$s_i=0$,スピン1/2では$s_i=1/2$だ.12.1節でも述べているように,ゲージボゾン(例えば光子)はスピン1を持つが$s_i=0$として扱うことに注意すること.\par
上を導いたのにあたって,$\sum_i(1+s_i)$と$\sum_f E_f(1+s_f)$は同じ意味だということを一応明らかに述べておく.例えば,スピンゼロの線が5本,スピン1/2の線が3本の集合$\ell=\{\ell_i|\ell_1\sim\ell_5はスピンゼロの線,\ell_6\sim\ell_8はスピン1/2の線\}$とすれば
\begin{eqnarray*}
\sum_f E_f(1+s_f)&=&\underset{スピンゼロについて}{5(1+0)}+\underset{スピン1/2について}{3(1+(1/2))} \\
\sum_i (1+s_i)&=&\underset{\ell_1について}{(1+0)}+\underset{\ell_2について}{(1+0)}+\cdots+\underset{\ell_5について}{(1+0)}\\
&&+\underset{\ell_6について}{(1+(1/2))}+\cdots \underset{\ell_8について}{(1+(1/2))} \\
&=& 5(1+0)+3(1+(1/2))=\sum_f E_f(1+s_f)
\end{eqnarray*}
が分かる.一般にこの二つは等しい.まぁ少しイメージしてみれば当たり前なんだけど…
\vskip\baselineskip
(20.A.1)を
\begin{eqnarray*}
d(\ell,\ell')&=&4-\sum_{i\in\ell,\ell''}(1+s_i)-\sum_{i\in\ell}(2-2s_i) \\
&=& 4-\sum_{i\in \ell}(3-s_i)-\sum_{i\in\ell''}(1+s_i) \\
&=& 4+\sum_{i\in\ell}(1+s_i)-4\sum_{\i\in\ell}1-\sum_{i\in\ell''}(1+s_i) \\
&=& 4-4n(\ell)-N(\ell')+N(\ell)
\end{eqnarray*}
とすると便利らしい.ここで
\begin{eqnarray*}
n(\ell) \equiv \sum_{i\in \ell}1,\quad N(\ell) \equiv \sum_{i\in\ell}(1+s_i)
\end{eqnarray*}

これらの部分ダイアグラムに伴う漸近的振る舞いを勘定に入れて,$k',p$を固定して$k\to\infty$(各入射線の4元運動量が一斉に無限大にいく)とすると,$\Gamma_{\ell\ell'}(k,k',p)$の摂動論の各次数で
\begin{eqnarray*}
\Gamma_{\ell\ell'}(k,k',p)\to \sum^{(N)}_{\mathcal{O}} U^{\ell}_{\mathcal{O}}(k)F_{\mathcal{O},\ell'}(k',p)+o(k^{4-4n(\ell)+N(\ell)-N})
\end{eqnarray*}
という漸近形をしていることを示したい.ここで和$\sum_{\mathcal{O}}^{(N)}$は次元が$N(\mathcal{O})(\leq N)$の演算子$\mathcal{O}$についてとる.関数$U_{\mathcal{O}}^{\ell}(k)$は$k^{4+N(\ell)-N(\mathcal{O})}$の大きさである.$o(k^A)$は$k^A$より(少なくとも$1/k$の因子一つ分は,例えば20.2節においては$k^{-4}$に対して$k^{-5}$がこの項に属していた)速くゼロになる項を意味する.

\vskip\baselineskip

($N$以下の次元の)演算子の寄与を分離するために「$N$既約」振幅$I^{N}_{\ell\ell'}(k,k',p)$を「$N(\ell'')\leq N$の内線のどんな集合$\ell''$を切っても,集合$\ell$に属する線は集合$\ell'$の線から分離されない$\Gamma_{\ell\ell'}(k,k',p)$の全てのダイアグラムの和」と定義する(20.2節の「2粒子既約」の定義と比較すると分かりやすい).\par
差$\Gamma-I^N$は,定義中の操作によって分離できる全てのダイアグラムから構成されることに留意すれば,
\begin{eqnarray*}
\Gamma_{\ell\ell'}(k,k',p)-I^N_{\ell\ell'}(k,k',p)=\sum^{(N)}_{\ell''}\int dk'' I_{\ell\ell'}^N(k,k'',p)\Gamma_{\ell''\ell'}(k'',k',p)
\end{eqnarray*}
と書ける.

\begin{tikzpicture}[scale=0.4]

\begin{scope}
\draw[very thick] (-4,1)--(-{sqrt(3)},1);
\draw[very thick] (-4,0)--(-2,0);
\draw[very thick](-4,-1)--(-{sqrt(3)},-1);

\draw (-3,-2)node{$\ell$};
\draw (9,-2)node{$\ell'$};

\draw[very thick](2,0)--(4,0);

\path[clip, preaction={draw, thick}] (0,0) circle (2);
\fill[draw=black, thick, pattern=north west lines] (-2,2) -- (2,2) -- (2,-2) -- (-2,-2) -- cycle;
\end{scope}

\begin{scope}
\draw[very thick]({6+sqrt(3)},1)--(10,1);
\draw[very thick] (8,0)--(10,0);
\draw[very thick]({6+sqrt(3)},-1)--(10,-1);

\path[clip, preaction={draw, thick}] (6,0) circle (2);
\fill[draw=black, thick, pattern=north west lines] (4,2) -- (8,2) -- (8,-2) -- (4,-2) -- cycle;
\end{scope}

\begin{scope}
\draw[very thick](0,2)--(0,-2);
\draw (11,0)node{$+$};
\draw[very thick](16,2)--(16,-2);

\draw[very thick] (12,1)--({16-sqrt(3)},1);
\draw[very thick] (12,0)--(14,0);
\draw[very thick](12,-1)--({16-sqrt(3)},-1);

\draw (-3,-2)node{$\ell$};
\draw (9,-2)node{$\ell'$};

\draw[very thick]({16+sqrt(3)},1)--({22-sqrt(3)},1);
\draw[very thick]({16+sqrt(3)},-1)--({22-sqrt(3)},-1);

\path[clip, preaction={draw, thick}] (16,0) circle (2);
\fill[draw=black, thick, pattern=north west lines] (14,2) -- (18,2) -- (18,-2) -- (14,-2) -- cycle;
\end{scope}

\begin{scope}
\draw[very thick]({22+sqrt(3)},1)--(26,1);
\draw[very thick] (24,0)--(26,0);
\draw[very thick]({22+sqrt(3)},-1)--(26,-1);

\path[clip, preaction={draw, thick}] (22,0) circle (2);
\fill[draw=black, thick, pattern=north west lines] (20,2) -- (24,2) -- (24,-2) -- (20,-2) -- cycle;
\end{scope}

\draw (13,-2)node{$\ell$};
\draw (25,-2)node{$\ell'$};

\draw (27,0)node{$+$};
\draw (29,0)node{$\cdots$};

\draw (-3,-6)node{$+$};


\begin{scope}
\draw[very thick] (-2,-5)--({2-sqrt(3)},-5);
\draw[very thick] (-2,-6)--(0,-6);
\draw[very thick](-2,-7)--({2-sqrt(3)},-7);
\draw[very thick](2,-4)--(2,-8);

\draw (-1,-8)node{$\ell$};
\draw (11,-8)node{$\ell'$};

\draw[very thick](3,{-6+sqrt(3)})--(7,{-6+sqrt(3)});
\draw[very thick]({2+sqrt(3)},-5)--({8-sqrt(3)},-5);
\draw(5,0.2-7) circle (0.02);
\draw(5,-7) circle (0.02);
\draw(5,-0.2-7) circle (0.02);
\draw[very thick](3,{-6-sqrt(3)})--(7,{-6-sqrt(3)});

\path[clip, preaction={draw, thick}] (2,-6) circle (2);
\fill[draw=black, thick, pattern=north west lines] (0,-4) -- (4,-4) -- (4,-8) -- (0,-8) -- cycle;
\end{scope}

\begin{scope}
\draw[very thick]({8+sqrt(3)},-5)--(12,-5);
\draw[very thick] (10,-6)--(12,-6);
\draw[very thick]({8+sqrt(3)},-7)--(12,-7);

\path[clip, preaction={draw, thick}] (8,-6) circle (2);
\fill[draw=black, thick, pattern=north west lines] (6,-4) -- (10,-4) -- (10,-8) -- (6,-8) -- cycle;
\end{scope}

\draw (20,-6)node{(内線の集合は$N$を超えない次元)};

\end{tikzpicture}

\vskip\baselineskip

\begin{tikzpicture}[scale=0.4]

\draw (-4,0)node{$=\sum^{(N)}_{\ell''}$};

\begin{scope}
\draw[very thick] (-2,1)--({2-sqrt(3)},1);
\draw[very thick] (-2,0)--(0,0);
\draw[very thick](-2,-1)--({2-sqrt(3)},-1);

\draw (-1,-2)node{$\ell$};
\draw (11,-2)node{$\ell'$};
\draw (5,-3)node{$\ell''$};

\draw[very thick](3,{sqrt(3)})--(7,{sqrt(3)});
\draw[very thick]({2+sqrt(3)},1)--({8-sqrt(3)},1);
\draw(5,0.2-1) circle (0.02);
\draw(5,0-1) circle (0.02);
\draw(5,-0.2-1) circle (0.02);
\draw[very thick](3,{-sqrt(3)})--(7,{-sqrt(3)});

\draw[very thick](2,2)--(2,-2);

\path[clip, preaction={draw, thick}] (2,0) circle (2);
\fill[draw=black, thick, pattern=north west lines] (0,2) -- (4,2) -- (4,-2) -- (0,-2) -- cycle;
\end{scope}

\begin{scope}
\draw[very thick]({8+sqrt(3)},1)--(12,1);
\draw[very thick] (10,0)--(12,0);
\draw[very thick]({8+sqrt(3)},-1)--(12,-1);

\path[clip, preaction={draw, thick}] (8,0) circle (2);
\fill[draw=black, thick, pattern=north west lines] (6,2) -- (10,2) -- (10,-2) -- (6,-2) -- cycle;
\end{scope}

\end{tikzpicture}

\noindent
ここで$\sum^{(N)}_{\ell''}$は$N$以下の次元$N(\ell'')$を持つ粒子線$\ell''$についての和,$\int dk''$は集合$\ell''$の4元運動量の成分について,その和を$p$と拘束した積分.\par

核$I^N_{\ell\ell''}(k,k'',p)$の漸近的振る舞いは,$\Gamma_{\ell\ell''}(k,k'',p)$の漸近的振る舞いよりはるかに簡単(らしい).$k'',p$を固定して$k\to\infty$としたとき,$I^N_{\ell\ell''}(k,k'',p)$には「どの内線も$k$の大きさの運動量を持っている」積分領域から主要な寄与を受ける.なぜなら,$I^N$は$N$粒子既約なダイアグラムであるから,部分領域だけが大きな運動量を担っていて他の部分はそうでない部分である項は,ダイアグラムの他の部分と(橋$\ell'''$の次元$N(\ell''')$が$N$より高い)$\ell'''$によって結合されていなければならない.(20.A.5)の和より$N(\ell'')\leq N$であったから,$N(\ell''')>N \geq N(\ell'')$が成り立っている.

\begin{figure}[H]
  \centering
\begin{tikzpicture}[scale=0.5]
\begin{scope}
\draw[very thick] (-4,1)--(-{sqrt(3)},1);
\draw[very thick] (-4,0)--(-2,0);
\draw[very thick](-4,-1)--(-{sqrt(3)},-1);

\draw (-3,-2)node{$\ell$};
\draw (3,-2)node{$\ell''$};

\draw[very thick]({sqrt(3)},1)--(4,1);
\draw[very thick](2,0)--(4,0);
\draw[very thick]({sqrt(3)},-1)--(4,-1);

\path[clip, preaction={draw, thick}] (0,0) circle (2);
\fill[draw=black, thick, pattern=north west lines] (-2,2) -- (2,2) -- (2,-2) -- (-2,-2) -- cycle;
\end{scope}

\draw[very thick](0,2)--(0,-2);
\draw (5,0)node{$=$};


\begin{scope}
\draw[very thick] (6,1)--({10-sqrt(3)},1);
\draw[very thick] (6,0)--(8,0);
\draw[very thick](6,-1)--({10-sqrt(3)},-1);

\draw (-3,-2)node{$\ell$};
\draw (3,-2)node{$\ell''$};

\draw[very thick]({sqrt(3)},1)--(4,1);
\draw[very thick](2,0)--(4,0);
\draw[very thick]({sqrt(3)},-1)--(4,-1);

\draw[very thick](11,{sqrt(3)})--(15,{sqrt(3)});
\draw[very thick]({10+sqrt(3)},1)--({16-sqrt(3)},1);
\draw(13,0.2-1) circle (0.02);
\draw(13,0-1) circle (0.02);
\draw(13,-0.2-1) circle (0.02);
\draw[very thick](11,{-sqrt(3)})--(15,{-sqrt(3)});

\path[clip, preaction={draw, thick}] (10,0) circle (2);
\fill[draw=black, thick, pattern=north west lines] (8,2) -- (12,2) -- (12,-2) -- (8,-2) -- cycle;
\end{scope}

\begin{scope}

\path[clip, preaction={draw, thick}] (16,0) circle (2);
\fill[draw=black, thick, pattern=north west lines] (14,2) -- (18,2) -- (18,-2) -- (14,-2) -- cycle;
\end{scope}

\draw[very thick]({16+sqrt(3)},1)--(20,1);
\draw[very thick](18,0)--(20,0);
\draw[very thick]({16+sqrt(3)},-1)--(20,-1);

\draw[very thick](10,2)--(10,-2);
\draw[very thick](16,2)--(16,-2);

\draw (10,-2)node[below]{内線$k$程度};
\draw (16,-2)node[below]{それ以外};
\draw (7,-2)node{$\ell$};
\draw (19,-2)node{$\ell''$};
\draw (13,-2.5)node{$\ell'''$};
\draw (13,2.5)node{$N(\ell''')>N$};
\end{tikzpicture}
\end{figure}
\noindent
この(部分ダイアグラムのさらに)部分ダイアグラム(20.A.2)より,
\begin{eqnarray*}
k^{4-4n(\ell)-N(\ell''')+N(\ell)}&=&k^{4-4n(\ell)-N(\ell'')+N(\ell)}k^{N(\ell'')-N(\ell''')} \\
&=& k^{d(\ell,\ell'')}k^{N(\ell'')-N(\ell''')} \leq k^{d(\ell,\ell'')}k^{N(\ell'')-(N+1)} \quad \because N(\ell''')> N
\end{eqnarray*}
となって,全ての内線が$k$程度の大きさを持つ項に比べて最低$k^{N(\ell'')-N-1}$の因子分は小さくなる.よってこれは主要な項でなくなるため,主要な項は「内線が全て$k$の大きさを持つダイアグラム」で,$k^{d(\ell,\ell'')}$の大きさのものだと分かる.\par
よって,$I^N_{\ell\ell''}(k,k'',p)$を$k'',p$で微分すると
\begin{eqnarray*}
\frac{\partial}{\partial k''}I^N_{\ell\ell''}(k,k'',p)&=&\frac{\partial}{\partial k''}\left\{k^{d(\ell,\ell'')}I^N_{\ell\ell''}\left(1,\frac{k''}{k},\frac{p}{k}\right)\right\} \\
&=&k^{d(\ell,\ell'')}\left(\frac{k''}{k}\right)' \frac{\partial }{\partial (k''/k)}I_{\ell\ell''}^N\left(1,\frac{k''}{k},\frac{p}{k}\right) \\
&=&k^{d(\ell,\ell'')-1}I_{\ell\ell'',k''}^N\left(1,\frac{k''}{k},\frac{p}{k}\right) \\
\frac{\partial}{\partial k''}I^N_{\ell\ell''}(k,k'',p)&=&k^{d(\ell,\ell'')-1}I_{\ell\ell'',p}^N\left(1,\frac{k''}{k},\frac{p}{k}\right)
\end{eqnarray*}
となって漸近的振る舞いは$k^{d(\ell,\ell'')-1}$に落ちる.しかし$d$回微分するときは,その漸近的振る舞いは\uwave{$d\leq N-N(\ell'')+1$の場合のみ}$k^{d(\ell,\ell'')-d}$に落ちる(20.2節の場合では$N=2,N(\ell'')=2$だったので幸運にも特に考えず微分して$k^{-5}$に下げることができた).なぜなら,それより高次の微分は,上で述べたような一部の部分ダイアグラムのみが$k$程度の運動量を持つダイアグラムにおいて,積の微分より
\begin{align*}
&\left[\frac{\partial}{\partial k''(\mathrm{or}\, p)}\right]^dI^N_{\ell\ell''}(k,k'',p)=\left[ \frac{\partial}{\partial k''(\mathrm{or}\, p)}\right]^d \left\{ \int dk''' \underset{内線がk程度}{\uwave{I^N_{\ell\ell'''}(k,k''',p)}} \, \underset{それ以外}{\uwave{I^N_{\ell'''\ell''}(k''',k'',p)}} \right\} \\
&=\int dk'''I^N_{\ell\ell'''}(k,k''',p) \left\{\left[ \frac{\partial}{\partial k''(\mathrm{or}\, p)}\right]^d I^N_{\ell'''\ell''}(k''',k'',p)\right\} +(積の微分の他の項)
\end{align*}
となって,この最初の項は全体として
\begin{eqnarray*}
k^{4-4n(\ell)-N(\ell''')-N(\ell)}&=&k^{d(\ell,\ell'')+N(\ell'')-N(\ell''')}\\
&\leq& k^{d(\ell,\ell'')+N(\ell'')-N-1} \quad \because N(\ell''')>N
\end{eqnarray*}
の振る舞いをするからだ.もし$d> -N(\ell'')+N+1$を満たす場合はこの項は,主要なダイアグラムを$d$回微分したもの($k^{d(\ell,\ell'')-d}$の大きさ)よりも,大きな寄与があるかもしれない.\par
したがって,場の微分を含む演算子の寄与を考慮に入れるためには(?),$I^N$の漸近的振る舞いを以下の通りに書く.
\begin{eqnarray*}
I^N_{\ell\ell''}(k,k'',p)=\sum_{\nu:d_\nu \leq N-N(\ell'')}I^N_{\ell\ell''\nu}(k)P_{\ell''\nu}(\ell'',p)+o(k^{d(\ell,\ell'')-N+N(\ell'')})
\end{eqnarray*}
ここで$P_{\ell''\nu}(k'',p)$は$n(\ell'')$個の運動量$k'',p$((20.A.5)の展開の間の外線の数一本一本に対応した運動量は$n(\ell'')$個ある)の$d_\nu$次の斉次多項式の完全系.そして$I^N_{\ell\ell''\nu}(k)$は$k$のみの関数で$k\to\infty$で$k^{d(\ell,\ell'')-d_\nu}$の大きさだ.一部の内線だけが$k$の大きさの部分ダイアグラムの寄与は(次数が最低でも$k^{d(\ell,\ell'')+N(\ell'')-N-1}$だったから),$o(k^{d(\ell,\ell'')-N+N(\ell'')})$の項にまとめられていると分かる.\par
(20.2節においては,$N=N(\ell'')=2$だったので和は$d_\nu=0$のみで,$I^N_\nu$は$k^{d(\ell,\ell'')}=k^{-4}$の大きさを持つ$I_{\infty}(k)$で,$P_\nu$は$k'',p$についてゼロ次の多項式となって存在しなかったのだと分かる.)\par
この展開は線形代数でいう基底で展開したようなもので,実際$P_\nu$は完全系であるから内積はゼロであるので基底の役割をしていることが分かる.またこの$I_{\ell\ell''}^N$を$k'',p$で微分すると,第一項目において$d_\nu=0$の項は$k'',p$に依存しないのでゼロとなり,また$k'',p$について$d_\nu$次の$P_\nu$の項は斉次多項式なので$d_\nu-1$次となるが,その係数$I^N_\nu$は$k\to\infty$で$1/k\times k^{d(\ell,\\ell'')-(d_\nu+1)}$の大きさであるから,実際に全体として$k^{-1}$だけ漸近的振る舞いが落ちていることが分かる.

\vskip\baselineskip

20.2節と同様に,(20.A.5)において(20.A.6)をすぐには適用することはできない.これは$k$をどれだけ大きくとっても,$k''$積分の領域には$k''$が$k$の大きさの領域が存在し,それが大きな寄与をするからだ.(20.A.6)は$k'',p$がどちらも$k$に対して有限の大きさでなければならない.これに対処するためには,やはり以前と同じく数学的帰納法を用いる.つまり,まず(20.A.4)が摂動論のある次数まで成立することを仮定して,それを(20.A.5)の右辺に使って$\Gamma$の摂動論の次の次数での漸近的振る舞いを計算し(20.A.4)を導けば証明できる.\par
(20.A.5)を以下のように書き換える.
\begin{align*}
&\Gamma_{\ell\ell'}(k,k',p)=I^N_{\ell\ell'}(k,k',p) \\
&+\sum_{\ell''}^{(N)}\int dk'' I^N_{\ell\ell'}(k,k'',p)\left[\Gamma_{\ell''\ell'}(k'',k',p)-\sum^{(N)}_{\mathcal{O}} U^{\ell''}_{\mathcal{O}}(k'')F_{\mathcal{O},\ell'}(k',p) \right] \\
&+\sum^{(N)}_{\mathcal{O}}F_{\mathcal{O},\ell'}(k',p) \sum_{\ell'':N(\ell')<N} \int dk'' I^N_{\ell\ell'}(k,k'',p)U_{\mathcal{O}}^{\ell''}(k'')
\end{align*}
(第三項目の$\sum_{\ell''}$は表記が違うだけで第二項目と同じ意味)(20.A.4)によると,(20.A.7)の右辺第二項目の括弧の中の量は$k''\to\infty$において$(k'')^{4-4n(\ell'')+N(\ell'')-N}$\uwave{より}速く減少するから,この因子と$d_\nu\leq N-N(\ell'')$の次数の多項式$P_{\ell''\nu}$との積は$(k'')^{4-4n(\ell'')}$より速く減少する.\par
($0 \leq N-N(\ell'')$であるから,$d_\nu=0$の多項式との積は$(k'')^{4-4n(\ell'')+N(\ell'')-N}$より速く減少し,それは$(k'')^{4-4n(\ell'')}$より速く減少する.$d_\nu=N-N(\ell'')$の多項式との積でも$(k'')^{4-4n(\ell'')}$より速く減少するので,成り立つ.)\par
運動量$k''$は$n(\ell'')$個存在あるが,その和が$p$だという条件で拘束されているので独立な運動量の数は$n(\ell'')-1$個.一つの運動量には4つの成分が存在するので独立成分の数は$4(n(\ell'')-1))$個である.よって積分は$4(n(\ell'')-1)$重積分となる.被積分関数は$k^{-4(n(\ell'')-1)}$より小さいから,$k''$が大きい領域での積分は速く減少し,したがって有限領域のみが寄与する.このとき(20.A.6)を適用することができて
\begin{eqnarray*}
\Gamma_{\ell\ell'}(k,k',p)&\to& \sum_{\nu: d_\nu +N(\ell')\leq N} I^N_{\ell\ell'\nu}(k)P_{\ell'\nu}(k',p) \\
&&+\sum^{(N)}_{\ell''}\sum_{\nu : d_\nu+N(\ell'')\leq N} I^N_{\ell\ell''\nu}(k)\int dk'' P_{\ell''\nu}(k'',p) \\
&&\quad \times \left[ \Gamma_{\ell''\ell'}(k'',k',p)-\sum^{(N)}_{\mathcal{O}}U^{\ell''}_{\mathcal{O}}(k'')F_{\mathcal{O},\ell'}(k',p) \right] \\
&&+\sum^{(N)}_{\mathcal{O}}F_{\mathcal{O},\ell'}(k',p) \sum_{\ell'':N(\ell')<N} \int dk'' I^N_{\ell\ell'}(k,k'',p)U_{\mathcal{O}}^{\ell''}(k'')
\end{eqnarray*}
となる.ここで補正項は書かれている項より$1/k$の因子以上小さい.第一項目は$N(\ell')\leq N$の場合のみに存在する.実際,20.2節において$n\geq 4$のときの(20.2.17)等では第一項目に相当するものは存在しなかった.\par
さて,$d_\nu+N(\ell)\leq N$となっている$\ell$と$\nu$のそれぞれの値について,$\ell$の線に対応する場の因子,と$d_\nu$個の微分,を持つ演算子$\mathcal{O}$(次元は$N(\mathcal{O})=d_\nu +N(\ell) \leq N$)がある.$F_{\phi^2}$の物理的意味を検討したときと同じ議論を用いると,摂動論のゼロ次では演算子$\mathcal{O}$が入射運動量$p$を持ち,外線$\ell$が流出運動量$k$を持つ頂点関数,が多項式$P_{\ell\nu}(k,p)$となっている.このときの(発散するであろう)頂点関数は
\begin{eqnarray*}
F^B_{\mathcal{O},\ell}(k,p)=P_{\ell_{\mathcal{O}}\nu_{\mathcal{O}}}(k,p)\delta_{\ell_{\mathcal{O}},\ell}+ \int dk' P_{\ell_{\mathcal{O}}\nu_{\mathcal{O}}}(k',p)\Gamma_{\ell_{\mathcal{O}}\ell}(k',k,p)
\end{eqnarray*}
したがってくりこまれた演算子$\mathcal{O}_R=\sum_{\mathcal{O}'}Z_{\mathcal{O,O'}}\mathcal{O}'$の対応する完全な頂点関数は
\begin{eqnarray*}
F_{\mathcal{O},\ell}(k,p)&=&\sum_{\mathcal{O}'}Z_{\mathcal{OO'}}\biggl\{ P_{\ell_{\mathcal{O'}}\nu_{\mathcal{O'}}}(k,p)\delta_{\ell_{\mathcal{O'}},\ell}\biggr. \\
&&\qquad\qquad \biggl. + \int dk' P_{\ell_{\mathcal{O'}}\nu_{\mathcal{O'}}}(k',p)\Gamma_{\ell_{\mathcal{O'}}\ell}(k',k,p) \biggr\}
\end{eqnarray*}
となる.

\begin{figure}[H]
  \centering
\begin{tikzpicture}[scale=0.4]

\begin{scope}
\draw[very thick](-4,0)--(-2,0);
\draw (-3,0)node[below]{$\mathcal{O}(p)$};

\draw[very thick](2,0)--(8,0);
\draw(3.5,0)node[below]{$\ell_{\mathcal{O}}(k)$};

\draw(7,0)node[below]{$\ell(k)$};

\draw[thick](5,0.5)--(6,-0.5);
\draw[thick](5,-0.5)--(6,0.5);

\path[clip, preaction={draw, thick}] (0,0) circle (2);
\fill[draw=black, thick, pattern=north west lines] (-2,2) -- (2,2) -- (2,-2) -- (-2,-2) -- cycle;
\end{scope}

\draw (9,0)node{$+$};
\draw (0,-2)node[below]{$P$};

\begin{scope}
\draw[very thick](10,0)--(12,0);
\draw (11,0)node[below]{$\mathcal{O}(p)$};

\draw[very thick](16,0)--(20,0);
\draw (18,0)node[below]{$\ell_{\mathcal{O}}(k')$};

\draw (14,-2)node[below]{$P$};
\draw (22,-2)node[below]{$\Gamma$};

\path[clip, preaction={draw, thick}] (14,0) circle (2);
\fill[draw=black, thick, pattern=north west lines] (12,2) -- (16,2) -- (16,-2) -- (12,-2) -- cycle;
\end{scope}

\begin{scope}
\draw[very thick](24,0)--(26,0);
\draw (25,0)node[below]{$\ell(k)$};

\path[clip, preaction={draw, thick}] (22,0) circle (2);
\fill[draw=black, thick, pattern=north west lines] (20,2) -- (24,2) -- (24,-2) -- (20,-2) -- cycle;
\end{scope}

\end{tikzpicture}
\end{figure}
\noindent
(この説明は非常に分かりにくいし,言語化するのも非常に難しいのだけれど,この図と図18.3を見比べることが理解の手助けになるのではないかと思う.)

このもとで次のように$U^{\ell}_{\mathcal{O}}$を定義すれば(20.A.4)を示すことができる.
\begin{eqnarray*}
U^{\ell}_{\mathcal{O}}(k)&=&\sum^{(N)}_{\mathcal{O'}}I_{\ell \, \ell_{\mathcal{O'}} \, \nu_{\mathcal{O'}}}(k)\left[ Z^{-1}_{\mathcal{O'}\mathcal{O}} -\int dk'' U^{\ell_{\mathcal{O'}}}_{\mathcal{O}}(k'') P_{\ell_{\mathcal{O'}}\nu_{\mathcal{O'}}}(k'') \right] \\
&&\qquad +\sum^{(N)}_{\ell''}\int dk'' I^N_{\ell\ell''}(k,k'')U^{\ell''}_{\mathcal{O}}(k'')
\end{eqnarray*}
(最後の項の$U^{\ell''}_{\mathcal{O}}$の二重プライムは誤植で上下逆であることに注意)省略記法はここでは使わないで全てあらわに書いた.
実際$\Gamma(k,k',p)$を以上の$U^{\ell}_{\mathcal{O}}(k)$と$F_{\mathcal{O},\ell}(k,p)$で計算してみると(補正項は面倒なので書かない)
\begin{align*}
&\Gamma(k,k',p)\to \sum^{(N)}_{\mathcal{O}}U^{\ell}_{\mathcal{O}}(k)F_{\mathcal{O},\ell'}(k',p) \\
&=\sum^{(N)}_{\mathcal{O,O'}} I^N_{\ell \, \ell_{\mathcal{O'}} \, \nu_{\mathcal{O'}}}(k)\left[ Z^{-1}_{\mathcal{O'O}}-\int dk'' U^{\ell_{\mathcal{O'}}}_{\mathcal{O}}(k'') P_{\ell_{\mathcal{O'}}\nu_{\mathcal{O'}}}(k'') \right]F_{\mathcal{O}\ell'}(k',p) \\
&\quad +\sum^{(N)}_{\mathcal{O}}\sum^{(N)}_{\ell''}F_{\mathcal{O}\ell'}(k',p)\int dk'' I^N_{\ell \ell''}(k,k')U^{\ell''}_{\mathcal{O}}(k'') \qquad(U^{\ell}_{\mathcal{O}}(k)について展開)\\
&=\sum^{(N)}_{\mathcal{O,O''}} I^N_{\ell \, \ell_{\mathcal{O''}} \, \nu_{\mathcal{O''}}}(k)Z^{-1}_{\mathcal{O''O}} F_{\mathcal{O},\ell'}(k',p) \\
&\quad -\sum^{(N)}_{\mathcal{O,O'}} I^N_{\ell \, \ell_{\mathcal{O'}} \, \nu_{\mathcal{O'}}}(k)\int dk'' P_{\ell_{\mathcal{O'}}\nu_{\mathcal{O'}}}(k'')U^{\ell_{\mathcal{O'}}}_{\mathcal{O}}(k'') F_{\mathcal{O},\ell}(k',p) \\
&\quad +\sum^{(N)}_{\mathcal{O}}F_{\mathcal{O},\ell}(k',p)\sum^{(N)}_{\ell''}\int dk'' I^N_{\ell \ell'}(k,k')U^{\ell''}_{\mathcal{O}}(k'')
\end{align*}
各項に分けて展開する.
\begin{align*}
&(第一項目)=\sum^{(N)}_{\mathcal{O,O',O''}} I^N_{\ell \, \ell_{\mathcal{O''}} \, \nu_{\mathcal{O''}}}(k)Z^{-1}_{\mathcal{O''O}} Z_{\mathcal{OO'}}\biggl\{ P_{\ell_{\mathcal{O'}}\nu_{\mathcal{O'}}}(k,p)\delta_{\ell_{\mathcal{O'}},\ell}\biggr. \\
&\qquad\qquad \biggl. + \int dk'' P_{\ell_{\mathcal{O'}}\nu_{\mathcal{O'}}}(k',p)\Gamma_{\ell_{\mathcal{O'}}\ell}(k'',k',p) \biggr\}\\
&=\sum^{(N)}_{\mathcal{O',O''}} I^N_{\ell \, \ell_{\mathcal{O''}} \, \nu_{\mathcal{O''}}}(k)\delta _{\mathcal{O''O'}} \left\{ P_{\ell_{\mathcal{O'}}\nu_{\mathcal{O'}}}(k,p)\delta_{\ell_{\mathcal{O'}},\ell}+ \int dk'' P_{\ell_{\mathcal{O'}}\nu_{\mathcal{O'}}}(k'',p)\Gamma_{\ell_{\mathcal{O'}}\ell}(k'',k',p) \right\} \\
&=\sum^{(N)}_{\mathcal{O'}} I^N_{\ell \, \ell_{\mathcal{O'}} \, \nu_{\mathcal{O'}}}(k)\left\{ P_{\ell_{\mathcal{O'}}\nu_{\mathcal{O'}}}(k,p)\delta_{\ell_{\mathcal{O'}},\ell}+ \int dk'' P_{\ell_{\mathcal{O'}}\nu_{\mathcal{O'}}}(k'',p)\Gamma_{\ell_{\mathcal{O'}}\ell}(k'',k',p) \right\} \\
&=\sum^{(N)}_{\mathcal{O'}} I^N_{\ell \, \ell_{\mathcal{O'}} \, \nu_{\mathcal{O'}}}(k)P_{\ell_{\mathcal{O'}}\nu_{\mathcal{O'}}}(k,p)\delta_{\ell_{\mathcal{O'}},\ell} \\
& \qquad +\sum^{(N)}_{\mathcal{O'}}I^N_{\ell \, \ell_{\mathcal{O'}} \, \nu_{\mathcal{O'}}}(k)\int dk'' P_{\ell_{\mathcal{O'}}\nu_{\mathcal{O'}}}(k'',p)\Gamma_{\ell_{\mathcal{O'}}\ell}(k'',k',p) \\
&=\sum_{\nu:d_\nu \leq N-N(\ell')} I^N_{\ell \ell' \nu}(k)P_{\ell' \nu }(k,p) \\
&\qquad +\sum^{(N)}_{\ell''}\sum^{(N)}_{\nu:d_\nu \leq N-N(\ell'')}I^N_{\ell \ell'' \nu}(k)\int dk'' P_{\ell'' \nu}(k'',p)\Gamma_{\ell'' \ell}(k'',k',p)
\end{align*}
最後の変形では,$\sum^{(N)}_{\mathcal{O}}=\sum^{(N)}_{\ell}\sum^{(N)}_{\nu:d_\nu\leq N-N(\ell)}$であることを用いて,第一項目では$\ell$についての和をとった.これは,微分と場の因子の合計の次元が$N$を超えないように足し合わせることは,OPEで$N$を超えない演算子で展開することと同値であるからだ.
\begin{align*}
&(第二項目)=-\sum^{(N)}_{\mathcal{O'}} I^N_{\ell \, \ell_{\mathcal{O'}} \, \nu_{\mathcal{O'}}}(k)\int dk'' P_{\ell_{\mathcal{O'}}\nu_{\mathcal{O'}}}(k'')\sum^{(N)}_{\mathcal{O}}U^{\ell_{\mathcal{O'}}}_{\mathcal{O}}(k'') F_{\mathcal{O},\ell}(k',p) \\
&=-\sum^{(N)}_{\ell''}\sum^{(N)}_{\nu:d_\nu\leq N-N(\ell'')} I^N_{\ell \ell'' \nu}(k)\int dk'' P_{\ell'' \nu}(k'')\sum^{(N)}_{\mathcal{O}}U^{\ell''}_{\mathcal{O}}(k'') F_{\mathcal{O},\ell}(k',p)
\end{align*}
第三項目はそのままで十分だ.これらを合わせると,
\begin{eqnarray*}
\Gamma(k,k',p)&\to& \sum^{(N)}_{\mathcal{O}}U^{\ell}_{\mathcal{O}}(k)F_{\mathcal{O},\ell'}(k',p) \\
&&=\sum_{\nu:d_\nu \leq N-N(\ell')} I^N_{\ell \ell' \nu}(k)P_{\ell' \nu }(k,p) \\
&&\qquad +\sum^{(N)}_{\ell''}\sum^{(N)}_{\nu:d_\nu \leq N-N(\ell'')}I^N_{\ell \ell'' \nu}(k)\int dk'' P_{\ell'' \nu}(k'',p) \\
&&\qquad\qquad\qquad \times \left[\Gamma_{\ell'' \ell}(k'',k',p) -U^{\ell''}_{\mathcal{O}}(k'') F_{\mathcal{O},\ell}(k',p) \right] \\
&&\qquad +\sum^{(N)}_{\mathcal{O}}F_{\mathcal{O},\ell}(k',p)\sum^{(N)}_{\ell''}\int dk'' I^N_{\ell \ell'}(k,k')U^{\ell''}_{\mathcal{O}}(k'')
\end{eqnarray*}
一方,(20.A.8)の第三項目では,p10-p11での議論と同様にして$I^N_{\ell \ell'}(k,k')\equiv I^N_{\ell \ell'}(k,k',0)$として$p$依存性を消去することができるから,上の式は(20.A.8)と等しいことが確認できる.くりこまれた定数$Z_{\mathcal{OO'}}$はくりこみ点$k(\mu),p(\mu)$において関数$F_{\mathcal{O},\ell}(k(\mu),p(\mu))$が,相互作用がないときと同じ値$P_{\ell_{\mathcal{O}}\nu_{\mathcal{O}}}(k(\mu),p(\mu))\delta_{\ell_{\mathcal{O}},\ell}$に等しいととるように定義する.
\begin{eqnarray*}
P_{\ell_{\mathcal{O}}\nu_{\mathcal{O}}}(k(\mu),p(\mu))\delta_{\ell_{\mathcal{O}},\ell}&=&F_{\mathcal{O},\ell}(k(\mu),p(\mu)) \\
&=&\sum_{\mathcal{O}'}Z_{\mathcal{O,O'}}\biggl\{ P_{\ell_{\mathcal{O'}}\nu_{\mathcal{O'}}}(k(\mu),p(\mu))\delta_{\ell_{\mathcal{O'}},\ell}\biggr. \\
&&\qquad\qquad \biggl. + \int dk' P_{\ell_{\mathcal{O'}}\nu_{\mathcal{O'}}}(k',p(\mu))\Gamma_{\ell_{\mathcal{O'}}\ell}(k',k(\mu),p(\mu)) \biggr\}
\end{eqnarray*}
$\Gamma=0$では$Z_{\mathcal{OO'}}=\delta_{\mathcal{OO'}}$という解が存在する.線の集合の取り方についての和は,各項が線型独立であるからこれは唯一の解だ.したがって連続性より(20.A.12)は唯一の解を常に持つ.このため,(20.A.10)はOPE(20.A.4)の係数関数の再帰的な定義を与える.これが証明したかったことだ.

\newpage

\subsection{係数関数のくりこみ群方程式}
ここでは補遺を先に解説しているので冗長かもしれないが,一応再び定義を説明しておく.グリーン関数$\Gamma_{\ell\ell'}(k,k',p)$の演算子積展開を考える.そこでは,線の集合$\ell$の入射運動量(これらをまとめて$k$,その和を$p$と書く)が全て揃って無限大に行き,残りの線の集合$\ell'$が固定された流出運動量(これらをまとめて$k'$,その和は$p$と書く)を持つ.
\begin{eqnarray*}
\Gamma_{\ell\ell'}(k,k',p)\to \sum_{\mathcal{O}}U^{\ell}_{\mathcal{O}}(k)F_{\mathcal{O},\ell'}(k',p)
\end{eqnarray*}

(20.A.9)のように,$F_{\mathcal{O},\ell'}(k',p)$はくりこまれた演算子$\mathcal{O}_R=\sum_{\mathcal{O}}Z_{\mathcal{OO'}}\mathcal{O'}$の行列要素で$Z_{\mathcal{OO'}}$に比例するので,$Z_{\mathcal{OO'}}$を打ち消すように$U_{\mathcal{O}}^{\ell}(k)$は$Z_{\mathcal{O'O}}^{-1}$に比例している.\\
$\Rightarrow$この説明でも納得はできるが,補遺を先に見たのでもう少し分かりやすく見ることができる.(20.A.10)の積分方程式
\begin{eqnarray*}
U_{\mathcal{O}}^{\ell}(k)&=&\sum^{(N)}_{\mathcal{O'}}I_{\ell \, \ell_{\mathcal{O'}} \, \nu_{\mathcal{O'}}}(k)\left[ Z^{-1}_{\mathcal{O'}\mathcal{O}} -\int dk'' U^{\ell_{\mathcal{O'}}}_{\mathcal{O}}(k'') P_{\ell_{\mathcal{O'}}\nu_{\mathcal{O'}}}(k'') \right] \\
&&\qquad +\sum^{(N)}_{\ell''}\int dk'' I^N_{\ell\ell''}(k,k'')U^{\ell''}_{\mathcal{O}}(k'') \\
&=&\sum^{(N)}_{\mathcal{O'}}I_{\ell \, \ell_{\mathcal{O'}} \, \nu_{\mathcal{O'}}}(k)Z^{-1}_{\mathcal{O'}\mathcal{O}} \\
&& -\sum^{(N)}_{\mathcal{O'}}I_{\ell \, \ell_{\mathcal{O'}} \, \nu_{\mathcal{O'}}}(k)\int dk'' U^{\ell_{\mathcal{O'}}}_{\mathcal{O}}(k'') P_{\ell_{\mathcal{O'}}\nu_{\mathcal{O'}}}(k'') \\
&&+\sum^{(N)}_{\ell''}\int dk'' I^N_{\ell\ell''}(k,k'')U^{\ell''}_{\mathcal{O}}(k'')
\end{eqnarray*}
の右辺に左辺を代入すると
\begin{eqnarray*}
U_{\mathcal{O}}^{\ell}(k)&=&\sum^{(N)}_{\mathcal{O'}}I_{\ell \, \ell_{\mathcal{O'}} \, \nu_{\mathcal{O'}}}(k)Z^{-1}_{\mathcal{O'}\mathcal{O}} \\
&&-\sum^{(N)}_{\mathcal{O''}}I_{\ell \, \ell_{\mathcal{O''}} \, \nu_{\mathcal{O''}}}(k)\int dk'' \left\{ \sum^{(N)}_{\mathcal{O'}}I_{\ell_{\mathcal{O''}} \, \ell_{\mathcal{O'}} \, \nu_{\mathcal{O'}}}(k'')Z^{-1}_{\mathcal{O'}\mathcal{O}} \right\} P_{\ell_{\mathcal{O''}}\nu_{\mathcal{O''}}}(k'') \\
&&+\sum^{(N)}_{\mathcal{O''}}I_{\ell \, \ell_{\mathcal{O''}} \, \nu_{\mathcal{O''}}}(k)\int dk'' \\
&&\qquad \times \left\{ \sum^{(N)}_{\mathcal{O'}}I_{\ell_{\mathcal{O''}} \, \ell_{\mathcal{O'}} \, \nu_{\mathcal{O'}}}(k'')\int dk' U^{\ell_{\mathcal{O'}}}_{\mathcal{O}}(k') P_{\ell_{\mathcal{O'}}\nu_{\mathcal{O'}}}(k') \right\} P_{\ell_{\mathcal{O''}}\nu_{\mathcal{O''}}}(k'') \\
&&-\sum^{(N)}_{\mathcal{O''}}I_{\ell \, \ell_{\mathcal{O''}} \, \nu_{\mathcal{O''}}}(k)\int dk'' \left\{ \sum^{(N)}_{\ell''}\int dk' I^N_{\ell_{\mathcal{O''}}\ell''}(k'',k')U^{\ell''}_{\mathcal{O}}(k') \right\} P_{\ell_{\mathcal{O''}}\nu_{\mathcal{O''}}}(k'') \\
&&+\sum^{(N)}_{\ell''}\int dk'' I^N_{\ell \ell''}(k,k'') \left\{ \sum^{(N)}_{\mathcal{O'}}I_{\ell'' \, \ell_{\mathcal{O'}} \, \nu_{\mathcal{O'}}}(k'')Z^{-1}_{\mathcal{O'}\mathcal{O}} \right\} \\
&& -\sum^{(N)}_{\ell''}\int dk'' I^N_{\ell \ell''}(k,k'') \left\{ \sum^{(N)}_{\mathcal{O'}}I_{\ell'' \, \ell_{\mathcal{O'}} \, \nu_{\mathcal{O'}}}(k'')\int dk' U^{\ell_{\mathcal{O'}}}_{\mathcal{O}}(k') P_{\ell_{\mathcal{O'}}\nu_{\mathcal{O'}}}(k')\right\} \\
&& +\sum^{(N)}_{\ell''}\int dk'' I^N_{\ell \ell''}(k,k'') \left\{ \sum^{(N)}_{\ell'}\int dk' I^N_{\ell''\ell'}(k'',k')U^{\ell'}_{\mathcal{O}}(k')\right\} 
\end{eqnarray*}
となるが,第1,2,5項目は$Z_{\mathcal{O'O}}^{-1}$に比例していることが分かる.残りの項には$U_{\mathcal{O}}^{\ell}(k)$に再び(20.A.10)を代入することができ,これを繰り返し無限に続ければ$U_{\mathcal{O}}^{\ell}(k)$全体は$Z_{\mathcal{O'O}}^{-1}$に比例していることが明らかに分かる.\par

\vskip\baselineskip

また,同時に$U_{\mathcal{O}}^{\ell}$は集合$\ell'$の場の因子全てのくりこみ行列因子$Z_{\ell\ell'}$にも比例する.したがって
\begin{eqnarray*}
U_{\mathcal{O}}^{\ell}=\sum_{\ell'\mathcal{O'}}Z_{\ell\ell'}\mathcal{U}^{\ell'}_{\mathcal{O'}}Z^{-1}_{\mathcal{O'O}}
\end{eqnarray*}
と形式的に書くことができる.\par
(18.2.25)と同様,くりこみ行列には
\begin{eqnarray*}
\mu\frac{\partial}{\partial \mu}Z_{\ell\ell'}=\sum_{\ell''}\gamma_{\ell\ell''}Z_{\ell''\ell'} ,\quad \mu\frac{\partial}{\partial \mu}Z_{\mathcal{OO'}}=\sum_{\mathcal{O''}}\gamma_{\mathcal{OO''}}Z_{\mathcal{O''O'}}
\end{eqnarray*}
の関係がある.一般に逆行列の微分には
\begin{align*}
&0=\frac{d}{dx}{A(x)A^{-1}(x)}=\frac{dA(x)}{dx}A^{-1}(x)+A(x)\frac{dA^{-1}(x)}{dx}\\
&\Rightarrow \frac{d}{dx}A^{-1}(x)=-A^{-1}(x)\frac{dA(x)}{dx}A^{-1}(x)
\end{align*}
の関係があるから$Z_{\mathcal{OO'}}$の逆行列には
\begin{eqnarray*}
\frac{\partial}{\partial \mu}Z^{-1}_{\mathcal{O'O}}&=&-\sum_{\mathcal{O''O'''}}Z^{-1}_{\mathcal{O'O''}}\frac{\partial}{\partial \mu}Z_{\mathcal{O''O'''}}Z^{-1}_{\mathcal{O'''O}} \\
\mu\frac{\partial}{\partial \mu}Z^{-1}_{\mathcal{O'O}}&=&-\sum_{\mathcal{O''O'''}}Z^{-1}_{\mathcal{O'O''}}\left\{ \mu\frac{\partial}{\partial \mu}Z_{\mathcal{O''O'''}}\right\} Z^{-1}_{\mathcal{O'''O}} \\
&=&-\sum_{\mathcal{O''O'''}}Z^{-1}_{\mathcal{O'O''}}\gamma_{\mathcal{O''O''''}}Z_{\mathcal{O''''O'''}} Z^{-1}_{\mathcal{O'''O}} \\
&=&-\sum_{\mathcal{O''}}Z^{-1}_{\mathcal{O'O''}}\gamma_{\mathcal{O''O}}
\end{eqnarray*}
の関係があると分かる.したがって$U_{\mathcal{O}}^{\ell}$に対しては
\begin{eqnarray*}
\mu\frac{d}{d\mu}U_{\mathcal{O}}^{\ell}&=&\mu\frac{d}{d\mu}\left[\sum_{\ell'\mathcal{O'}}Z_{\ell\ell'}\mathcal{U}^{\ell'}_{\mathcal{O'}}Z^{-1}_{\mathcal{O'O}}\right] \\
&=&\sum_{\ell'\ell''\mathcal{O'}}\gamma_{\ell\ell''}Z_{\ell''\ell'}\mathcal{U}^{\ell'}_{\mathcal{O'}}Z^{-1}_{\mathcal{O'O}}-\sum_{\ell'\mathcal{O'O''}}Z_{\ell''\ell'}\mathcal{U}^{\ell'}_{\mathcal{O'}}Z^{-1}_{\mathcal{O'O''}}\gamma_{\mathcal{O''O}}+\mu \frac{dg}{d\mu}\frac{\partial}{\partial g}U_{\mathcal{O}}^{\ell} \\
&=&\sum_{\ell'}\gamma_{\ell\ell'}U_{\mathcal{O}}^{\ell'}-\sum_{\mathcal{O'}}U_{\mathcal{O'}}^{\ell}\gamma_{\mathcal{O'O}}+\beta(g)\frac{\partial}{\partial g}U_{\mathcal{O}}^{\ell}
\end{eqnarray*}
となる.簡単のために唯一のくりこみ可能な結合定数$g_\mu$は運動量が$\mu$の大きさのくりこみ点におけるファインマン振幅の値として定義され,$\mu dg_\mu /d\mu=\beta(g_\mu)$だとする.\par

次元解析が使えるように,全ての演算子に$\mu$のベキをかけて無次元化する.つまり
\begin{align*}
&\mathcal{O}\to \mu^{-N(\mathcal{O})}\mathcal{O} \\
&\mathcal{O}_R=\sum_{\mathcal{O'}}Z_{\mathcal{OO'}}\mu^{-N(\mathcal{O'})}\mathcal{O'}
\end{align*}
とすると,$Z$は無次元となる(元々無次元だと思うかもしれないが,それぞれ次元が違う演算子に$Z$をかけて足し合わせてくりこまれた演算子を作っていたので,$Z$に次元を持たせなければならなかった.例えば$\mathcal{O}_R=Z_{\mathcal{O}\phi^2}\phi^2+Z_{\mathcal{O}\psi}\psi$のような感じ).このとき$U_{\mathcal{O}}^{\ell}$は
\begin{eqnarray*}
U_{\mathcal{O}}^{\ell}&=&\sum_{\ell'\mathcal{O'}}Z_{\ell\ell'}\mathcal{U}^{\ell'}_{\mathcal{O'}}Z^{-1}_{\mathcal{O'O}} \\
&&\to \sum_{\ell'\mathcal{O'}}Z_{\ell\ell'}\mu^{-N(\ell')}\mathcal{U}^{\ell'}_{\mathcal{O'}}\mu^{N(\mathcal{O'})}Z^{-1}_{\mathcal{O'O}} \\
&=&\sum_{\ell'\mathcal{O'}}Z'_{\ell\ell'}\mathcal{U}^{\ell'}_{\mathcal{O'}}Z'^{-1}_{\mathcal{O'O}} 
\end{eqnarray*}
となる.ここで$Z'_{\ell\ell'}=Z_{\ell\ell'}\mu^{-N(\ell')},Z'_{\mathcal{OO'}}=Z_{\mathcal{OO'}}\mu^{-N(\mathcal{O'})}$結合定数がゼロの極限の場合,くりこみ因子は不要で$Z'$は定数値であるから
\begin{eqnarray*}
0&=&\mu \frac{\partial}{\partial \mu}Z'_{\mathcal{OO'}} \\
&=&\sum_{\mathcal{O''}}\gamma_{\mathcal{OO''}}Z_{\mathcal{O''O'}}\mu^{-N(\mathcal{O'})}-N(\mathcal{O}')Z_{\mathcal{OO'}}\mu^{-N(\mathcal{O'})} \\
0&=&\mu \frac{\partial}{\partial \mu}Z'_{\ell\ell'} \\
&=&\sum_{\ell''}\gamma_{\ell\ell''}Z_{\ell''\ell'}\mu^{-N(\ell')}-N(\ell')Z_{\ell\ell'}\mu^{-N(\ell')} 
\end{eqnarray*}
すなわち結合定数のゼロの極限で$\gamma$は
\begin{eqnarray*}
\gamma_{\mathcal{OO'}}\to \delta_{\mathcal{OO'}}N(\mathcal{O}), \qquad \gamma_{\ell\ell'}\to \delta_{\ell\ell'}N(\ell')
\end{eqnarray*}
という値をとると分かる.\par
また,次元解析により$k^\mu=\kappa n^\mu$として$n^\mu$を固定すると,振幅はフーリエ変換を定義するのに用いた$k$の$4-4n(\ell)$重積分から生じる$\kappa^{4-4n(\ell)}$の因子を除いては今は(演算子を無次元化したので)$U^\ell_{\mathcal{O}}$は無次元なので比$\kappa/\mu$を通じてのみ$\kappa$に依存する.それぞれの$Z$の解の形は(18.2.27)でわかっているため(20.3.2)の解は
\begin{align*}
&U^\ell_{\mathcal{O}}(\kappa n)=\kappa^{4-4n(\ell)}\sum_{\ell'\mathcal{O'}}Z_{\ell\ell'}\mathcal{U}^{\ell'}_{\mathcal{O'}}Z^{-1}_{\mathcal{O'O}}\\
&=\kappa^{4-4n(\ell)}\sum_{\ell'\mathcal{O}'}\left[M\left\{ \exp\left(\int^ \kappa \frac{d\mu}{\mu} \gamma(g_\mu)\right) \right\} \right]_{\ell\ell'}\mathcal{U}^{\ell'}_{\mathcal{O'}}(g_\kappa,n)\left[M\left\{ \exp\left(\int^ \kappa \frac{d\mu}{\mu} \gamma(g_\mu)\right) \right\} \right]^{-1}_{\mathcal{O'O}}
\end{align*}
という形であることが分かる.ここで$M$は$\mu$順序積($T$順序積のようなもの)で,指数関数のベキ展開を考えたときに$n$乗の項が現れるが,その並び順は左から右に$\mu$が減る順番に並び変えられている.$\mu$順序積によって$exp^{-1}(\sim)$は逆行列なので逆順となって,右から左に減っているように並び変えられていることに留意する.\par

\vskip\baselineskip

$\mu\to\infty$において$g_\mu$が固定点$g_*$に近づくとき,特に簡単な結果を与える.
$\mu\to\infty$で$g_\mu\to g_*$は(18.3.13)より
\begin{eqnarray*}
\gamma(g_\mu)=\gamma(g_*)+c(g_*-g_\mu)+O((g_*-g_\mu)^2)
\end{eqnarray*}
この場合,$M\left\{ \exp(\int^\kappa \gamma(g_\mu)d\mu/\mu) \right\}$への大きな$\mu$の寄与は$\mu$順序積によって\uwave{左}に現れる$\kappa^{\gamma(g_*)}$の因子だ.
\begin{align*}
&M\left\{ \exp\left( \int^\kappa[\underset{最も\mu が大きい}{\uwave{\gamma(g_*)}}+c\underset{速さの因子}{(g_*-g_\mu)} ] \frac{d\mu}{\mu}\right) \right\}=M\left\{ \exp(\gamma(g_*)\ln\kappa )\times\sim\right\} \\
&=\kappa^{\gamma(g_*)}\times\sim[ここは\mathcal{U}と一緒にして\mathcal{C}と定義する]
\end{align*}
したがって(20.3.5)は
\begin{eqnarray*}
U^\ell_{\mathcal{O}}(\kappa n)=\kappa^{4-4n(\ell)}\sum_{\ell',\mathcal{O'}}\left[\kappa^{\gamma(g_*)}\right]_{\ell\ell'}\mathcal{C}_{\ell'\mathcal{O'}}\left[ \kappa^{-\gamma(g_*)} \right]_{\mathcal{O'O}}
\end{eqnarray*}
となる.($\exp(\alpha\ln\kappa)=\kappa^{\alpha}$を用いた.)
ここで$\mathcal{C}$は定数または$\ln \kappa$のベキの和であり,$g_\kappa$が$g_*$に近づく速さ(上の途中式で出てきた$(g_*-g_\mu)/\mu$の因子)に依存する.\par

\vskip\baselineskip

量子色力学のように,漸近的自由な理論は特に物理的に興味のある場合だ.\par
ここでは固定点は(18.3節で議論した通り)$g_*=0$にとれて,(20.3.4)によればこの点の近傍では$\gamma$行列は
\begin{eqnarray*}
\gamma(g)_{\ell\ell'}\to N(\ell)\delta_{\ell\ell'}+g^2 c_{\ell\ell'},\qquad \gamma(g)_{\mathcal{OO'}}\to N(\mathcal{O})\delta_{\mathcal{OO'}}+g^2 c_{\mathcal{OO'}} 
\end{eqnarray*}
と振る舞う.($g$が小さい時に一次の項が存在しないで二次の項を無視したときに(20.3.4)が再現される.)量子色力学におけるくりこみ群方程式(18.7.4)から
\begin{align*}
&\mu\frac{d}{d\mu}g_\mu=-\frac{b}{16\pi^2}g^3_\mu \qquad\left(b=11-\frac{n_f}{24}\right)\\
\Rightarrow \quad&\mu\frac{d}{d\mu}g^2_\mu=2g_\mu \left[\mu\frac{d}{d\mu}g_\mu\right]=-\frac{b}{8\pi^2}g^4_\mu
\end{align*}
となるから,ここで結合定数に対するくりこみ群方程式を
\begin{eqnarray*}
\mu\frac{d}{d\mu}g^2_\mu=-\frac{b}{8\pi^2}g^4_\mu
\end{eqnarray*}
と仮定することは標準的だ.もしこのように仮定すると
\begin{align*}
&\frac{d\mu}{\mu}g_\mu^2=-\frac{8\pi^2}{b}\frac{1}{g^2_\mu}dg^2_\mu \\
&\int^\kappa \frac{d\mu}{\mu}g^2_\mu=\int^{g_\kappa}\left[-\frac{8\pi^2}{b}\frac{1}{g^2_\mu}\right] dg^2_\mu \\
& \qquad\qquad =-\frac{8\pi^2}{b}\ln g^2_\kappa+\mathrm{Const}
\end{align*}
となる.これを(20.3.5)に代入して
\begin{align*}
&U^\ell_{\mathcal{O}}(\kappa n)\to \kappa^{4-4n(\ell)}\sum_{\ell'\mathcal{O}'}\left[M\left\{ \exp\left(\int^ \kappa \frac{d\mu}{\mu} (N(\ell)+g^2_\mu c)\right) \right\} \right]_{\ell\ell'}\mathcal{U}^{\ell'}_{\mathcal{O'}}(g_\kappa,n)\\
& \qquad \times \left[M\left\{ \exp\left(\int^ \kappa \frac{d\mu}{\mu} (N(\mathcal{O})+g^2_\mu c)\right) \right\} \right]^{-1}_{\mathcal{O'O}} \\
&=\kappa^{4-4n(\ell)}\sum_{\ell'\mathcal{O}'}\left[M\left\{ \exp\left(N(\ell)\ln \kappa - \frac{8\pi^2c}{b}\ln g^2_\kappa +\mathrm{Const}\right) \right\} \right]_{\ell\ell'}\mathcal{U}^{\ell'}_{\mathcal{O'}}(g_\kappa,n)\\
& \qquad\times \left[M\left\{ \exp\left(N(\mathcal{O})\ln \kappa - \frac{8\pi^2c}{b}\ln g^2_\kappa +\mathrm{Const}\right) \right\} \right]^{-1}_{\mathcal{O'O}} \\
&=\kappa^{4-4n(\ell)+N(\ell)-N(\mathcal{O})}\\
& \quad \times\sum_{\ell'\mathcal{O}'} \left[ (g_\kappa^2)^{-8\pi^2c/b} \right]_{\ell\ell'}\left[\exp(\mathrm{Const})\mathcal{U}(g_\kappa,n)\exp(\mathrm{Const})\right]^{\ell'}_{\mathcal{O'}}\left[(g_\kappa^2)^{8\pi^2c/b} \right]_{\mathcal{O'O}} \\
&=\kappa^{4-4n(\ell)+N(\ell)-N(\mathcal{O})}\sum_{\ell'\mathcal{O}'} \left[ (g_\kappa^2)^{-8\pi^2c/b} \right]_{\ell\ell'}\mathcal{C}^{\ell'}_{\mathcal{O'}}\left[(g_\kappa^2)^{8\pi^2c/b} \right]_{\mathcal{O'O}}
\end{align*}
ここで$g_\kappa$は$\kappa\to\infty$で固定点$g_*=0$に近づき,$n^\mu$は前述の通り固定しているので,定数行列$\mathcal{U}^{\ell'}_{\mathcal{O'}}(0,n)$に定数因子をかけたものである$\mathcal{C}^{\ell'}_{\mathcal{O'}}$は定数行列となる.この定数因子は(20.3.5)の積分で$g_\mu$が($\mu$順序により)それほど小さくない(すなわち,$\mu$が大きくないので固定点$g_*=0$に十分近付けていない)ところから来るために,摂動論では計算できない.\par
$\kappa\to\infty$では結合定数の振る舞いは(20.3.8)を解いて
\begin{align*}
&-\int \frac{1}{g^4_\mu}dg^2_\mu=\frac{b}{8\pi^2}\int \frac{d\mu}{\mu} \\
&\Rightarrow \frac{1}{g^2_\mu}=\frac{b}{8\pi^2}(\ln \mu+ \ln M) \qquad (Mは積分定数)\\
&\Rightarrow g^2_\mu=\frac{8\pi^2}{b\ln(\mu/M)}
\end{align*}
よって$\kappa\to\infty$では$g^2_\kappa\to 8\pi^2/b\ln\kappa$である.よって(20.3.9)は
\begin{eqnarray*}
U^\ell_{\mathcal{O}}(\kappa n)&\to&\kappa^{4-4n(\ell)+N(\ell)-N(\mathcal{O})}\sum_{\ell'\mathcal{O}'} \left[ \left(\frac{8\pi^2}{b\ln\kappa}\right)^{-8\pi^2c/b} \right]_{\ell\ell'}\mathcal{C}^{\ell'}_{\mathcal{O'}}\left[\left(\frac{8\pi^2}{b\ln\kappa}\right)^{8\pi^2c/b} \right]_{\mathcal{O'O}}\\
&=&\kappa^{4-4n(\ell)+N(\ell)-N(\mathcal{O})}\sum_{\ell'\mathcal{O}'}\left[(\ln \kappa)^{8\pi^2c/b}\right]_{\ell\ell'}\mathcal{B}^{\ell'}_{\mathcal{O}'}\left[(\ln \kappa)^{-8\pi^2c/b}\right]_{\mathcal{O'O}}
\end{eqnarray*}
とも書ける.ここで$\mathcal{B}$は$\mathcal{C}$とはまた別の定数行列だ.(20.3.9)(20.3.10)が正しいための条件は以前と同じく「結合定数が小さい」であるから,今回この条件は$g_\mu$が固定点に十分近いということであるから$\kappa\to\infty$と同じで,すなわち$\ln \kappa\to\infty\Rightarrow 1/\ln\kappa <<1 \Rightarrow g^2_\kappa/8\pi^2<<1$である.$\ln \kappa$はそこまで大きくなくて良いらしい.

\newpage

\subsection{係数関数の対称性}
くりこまれた演算子$\mathcal{O}_i(x)$の積の演算子積展開を考える.これらの演算子は,保存カレント$J^\mu(x)$を持つある対称性のもとで以下のように線型に変換するとする.
\begin{eqnarray*}
\left[ J^0(\mathbf{x},t), \mathcal{O}_i(\mathbf{y},t)\right]=-\delta^3(\mathbf{x-y})\sum_j t_{ij}\mathcal{O}_j(\mathbf{y},t)
\end{eqnarray*}
ここで$t_{ij}$は定数行列だ.(両辺を$\mathbf{x}$で空間積分すれば$\int d^3x J^0(\mathbf{x},t)=Q(t)$より
\begin{eqnarray*}
\left[ Q(t),\mathcal{O}_i(\mathbf{y},t) \right]=-\sum_j t_{ij}\mathcal{O}_j(\mathbf{y},t)
\end{eqnarray*}
となって見慣れた形になるだろう.)\par
演算子積展開は,$x_1,x_2,\cdots ,x_n$が揃って$x$に($x_1-x,x_2-x,\cdots,x_n-x$の比が全て固定されたまま)近づくとき
\begin{align*}
&\bra{\beta} T\left\{ \mathcal{O}_{i_1}(x_1), \mathcal{O}_{i_2}(x_2), \cdots , \mathcal{O}_{i_n}(x_n)\right\} \ket{\alpha} \\
&\to \sum_i U^{i_1\cdots i_n}_i(x_1-x,x_2-x,\cdots,x_n-x)\bra{\beta} \mathcal{O}_i(x)\ket{\alpha}
\end{align*}
と書ける.さて,カレント$J^\mu(x)$を持つ対称性が自発的に破れて,対応するNGボゾン$\pi$が以下を満たすとする.(恐らく誤植ではないとは思うが,以前との議論との繋がりを持たせるために(19.2.34)同様に$i$をつけておく.もし本文と完全に同じ式を導きたければ$iF\to F$とすれば良い.)
\begin{eqnarray*}
\bra{\mathrm{VAC}} J^\mu(0)\ket{\pi}=\frac{iFp^\mu_\pi}{(2\pi)^{3/2}\sqrt{2p^0_\pi}}
\end{eqnarray*}
(20.4.4)の導出は,19.2節のNGボゾンの極の理論と同様に示そうとしたが,以前とは式の状況が違うので同様にして示すのは難しかった.これを示すためには,極の理論の原点に立ち返って(10.2.7)を用いるのが良い.(10.2.1)から
\begin{align*}
&G(p_1,p_2,\cdots,p_n,q) \qquad(x_1,x_2,\sim,x_n,x\to p_1,p_2\sim p_n,qにフーリエ変換)\\
&\int d^4x_1d^4x_2 \cdots d^4x_n d^4x \, e^{-ip_1x_1}e^{-ip_2x_2}\cdots e^{-ip_nx_n}e^{-iqx} \\
& \qquad\qquad\times\frac{\partial}{\partial x^\mu}\bra{\beta}T\left\{ \mathcal{O}_{i_1}(x_1), \mathcal{O}_{i_2}(x_2), \cdots , \mathcal{O}_{i_n}(x_n) J^\mu(x) \right\} \ket{\alpha} \\
&=\int d^4x_1\cdots d^4x_n d^4x \, e^{-ip_1x_1}\cdots e^{-ip_nx_n}e^{-iqx}  \\
& \qquad\qquad \times iq_\mu  \bra{\beta}T\left\{ \mathcal{O}_{i_1}(x_1), \mathcal{O}_{i_2}(x_2), \cdots , \mathcal{O}_{i_n}(x_n) J^\mu(x) \right\} \ket{\alpha} \quad(\because 部分積分)\\
&=i\int d^4k\left[ (2\pi)^{3/2}\sqrt{2q^0_\pi}\int d^4x_1\cdots d^4x_n \, e^{-ip_1x_1}\cdots e^{-ip_nx_n} \right. \\
&\quad \times \biggl. \bra{\beta}T\left\{ \mathcal{O}_{i_1}(x_1), \mathcal{O}_{i_2}(x_2), \cdots , \mathcal{O}_{i_n}(x_n) \right\} \ket{\alpha+\pi} \biggr] \\
&\quad \times\left[ \frac{-i}{(2\pi)^4}\frac{1}{k^2} \right] \left[ (2\pi)^4\delta^4(q-k) q_\mu \bra{\mathrm{VAC}}J^\mu(0)\ket{\pi}(2\pi)^{3/2}\sqrt{2q^0_\pi}\right] \because (10.2.6) \\
&=\int d^4 k (2\pi)^{3/2}\sqrt{2q^0_\pi}\int d^4x_1\cdots d^4x_n \, e^{-ip_1x_1}\cdots e^{-ip_nx_n} \\
&\quad \times \bra{\beta}T\left\{ \mathcal{O}_{i_1}(x_1), \cdots , \mathcal{O}_{i_n}(x_n) \right\} \ket{\alpha+\pi} \\
&\quad \times \frac{1}{k^2}\delta^4(q-k)\frac{iFq^2}{(2\pi)^{3/2}\sqrt{2q^0_\pi}}(2\pi)^{3/2}\sqrt{2q^0_\pi} \quad \because (訂正された)(20.4.3)を代入 \\
&=i(2\pi)^{3/2}\sqrt{2q^0_\pi}F \int d^4x_1\cdots d^4x_n \, e^{-ip_1x_1}\cdots e^{-ip_nx_n} \\
& \qquad \times\bra{\beta}T\left\{ \mathcal{O}_{i_1}(x_1), \cdots , \mathcal{O}_{i_n}(x_n) \right\} \ket{\alpha+\pi}
\end{align*}
最初の式と最後の式で,フーリエ逆変換($p_1\sim p_n\to x_1\sim x_n$)を用いれば,
\begin{align*}
&\int d^4x e^{-iqx}\frac{\partial}{\partial x^\mu}\bra{\beta}T\left\{ \mathcal{O}_{i_1}(x_1), \mathcal{O}_{i_2}(x_2), \cdots , \mathcal{O}_{i_n}(x_n) J^\mu(x) \right\} \ket{\alpha} \\
&=i(2\pi)^{3/2}\sqrt{2q^0_\pi}F\bra{\beta}T\left\{ \mathcal{O}_{i_1}(x_1), \cdots , \mathcal{O}_{i_n}(x_n) \right\} \ket{\alpha+\pi} 
\end{align*}
両辺$q\to 0$($\pi$中間子の運動量ゼロの極限)で
\begin{align*}
&\bra{\beta}T\left\{ \mathcal{O}_{i_1}(x_1), \cdots , \mathcal{O}_{i_n}(x_n) \right\} \ket{\alpha+\pi} \\
&=\frac{-i}{(2\pi)^{3/2}\sqrt{2q^0_\pi}F}\int d^4 x \frac{\partial}{\partial x^\mu}\bra{\beta}T\left\{ \mathcal{O}_{i_1}(x_1), \mathcal{O}_{i_2}(x_2), \cdots , \mathcal{O}_{i_n}(x_n) J^\mu(x) \right\} \ket{\alpha}
\end{align*}
となる.\par


\vskip\baselineskip


(20.4.5)を導くのはさらに骨が折れる.これを示すためには,通常定義するよりも一般的な時間順序積の定義$(*)$
\begin{eqnarray*}
&T&\left\{ \mcO_{i_1}(x_1)\cdots \mcO_{i_n}(x_n) \right\} \\
&&\equiv \mcO_{i_1}(x_1)\theta(x^0_1-x^0_2)\mcO_{i_2}(x_2)\theta(x^0_2-x^0_3)\times\cdots \\
&&\quad \times\mcO_{i_{m-1}}(x_{m-1})\theta(x^0_{m-1}-x^0_m)\mcO_{m}(x_m)\theta(x^0_m-x^0_{m+1})\mcO_{i_{m+1}}(x_{m+1})\cdots \theta(x^0_{n-1}-x^0_n)\mcO_{i_n}(x_n) \\
&&\quad +(演算子の順序を交換したもの)
\end{eqnarray*}
を使うのが良い.この定義はwell-definedであることはすぐ分かるだろう.実際例えば$x_1>x_2>\cdots >x_m>x_{m+1}>\cdots>x_n$の場合にはこの定義($*$)を使うと
\begin{eqnarray*}
\mcO_{i_1}(x_1)\mcO_{i_2}(x_2)\cdots \mcO_{i_m}\mcO_{i_{m+1}}\cdots\mcO_{i_n}(x_n)
\end{eqnarray*}
の項が残って,時間順序を果たしていることが分かる.階段関数に時間微分がかかる項以外はカレントの保存則
\begin{eqnarray*}
\frac{\partial}{\partial x^\mu}J^\mu(x)=0
\end{eqnarray*}
により消えることに留意すれば
\begin{align*}
& \int d^4 x \frac{\partial}{\partial x^\mu}\bra{\beta}T\left\{ \mathcal{O}_{i_1}(x_1), \mathcal{O}_{i_2}(x_2), \cdots , \mathcal{O}_{i_n}(x_n) J^\mu(x) \right\} \ket{\alpha} \\
= & \int d^4x \left\langle \beta \middle| \biggl[J^0(x)\delta(x^0-x^0_1)\mcO_{i_1}(x_1)\theta(x^0_1-x^0_2)\mcO_{i_2}(x_2) \cdots \theta(x^0_{n-1}-x^0_n)\mcO_{i_n}(x_n)  \right. \\
& -\mcO_{i_1}(x_1)\delta(x^0_1-x^0)J^0(x)\theta(x^0-x^0_2)\mcO_{i_2}(x_2) \cdots \theta(x^0_{n-1}-x^0_n)\mcO_{i_n}(x_n) \\
& +\mcO_{i_1}(x_1)\theta(x^0_1-x^0)J^0(x)\delta(x^0-x^0_2)\mcO_{i_2}(x_2) \cdots \theta(x^0_{n-1}-x^0_n)\mcO_{i_n}(x_n) \\
& -\sim \\
&-\mcO_{i_1}(x_1)\cdots \mcO_{i_m}(x_m)\delta(x^0_m-x)J^0(x)\theta(x^0-x^0_{m+1})\mcO_{i_{m+1}}(x_{m+1})\cdots \theta(x^0_{n-1}-x^0_n)\mcO_{i_n}(x_n) \\
& +\mcO_{i_1}(x_1)\cdots \mcO_{i_m}(x_m)\theta(x^0_m-x)J^0(x)\delta(x^0-x^0_{m+1})\mcO_{i_{m+1}}(x_{m+1})\cdots \theta(x^0_{n-1}-x^0_n)\mcO_{i_n}(x_n) \\
& -\mcO_{i_1}(x_1)\cdots \mcO_{i_{m+1}}(x_{m+1})\delta(x^0_{m+1}-x)J^0(x)\theta(x^0-x^0_{m+2})\mcO_{i_{m+2}}(x_{m+2})\cdots \theta(x^0_{n-1}-x^0_n)\mcO_{i_n}(x_n) \\
&  +\mcO_{i_1}(x_1)\cdots \mcO_{i_{m+1}}(x_{m+1})\theta(x^0_{m+1}-x)J^0(x)\delta(x^0-x^0_{m+2})\mcO_{i_{m+2}}(x_{m+2})\cdots \theta(x^0_{n-1}-x^0_n)\mcO_{i_n}(x_n) \\
& -\sim \\
& -\mcO_{i_1}(x_1)\cdots \mcO_{i_{n-1}}(x_{n-1})\delta(x^0_{n-1}-x^0)J^0(x)\theta(x^0-x^0_n)\mcO_{i_n}(x_n) \\
& +\mcO_{i_1}(x_1)\cdots \mcO_{i_{n-1}}(x_{n-1})\theta(x^0_{n-1}-x^0)J^0(x)\delta(x^0-x^0_n)\mcO_{i_n}(x_n) \\
& -\mcO_{i_1}(x_1)\cdots\mcO_{i_n}(x_n)\delta(x^0_n-x^0)J^0(x) \\
& \left. +(\mcO_{i_1}(x_1)\sim \mcO_{i_n}(x_n)の順番を交換した全ての場合)←上の展開と同様\biggr]\middle| \alpha \right\rangle
\end{align*}
と展開される.詳しく説明しておくと,第1項目は順序積の定義$(*)$において$J\mcO_{i_1}\mcO_{i_2}\cdots\mcO_{i_n}$の順番の項を微分したときから来る項だ.この場合においては$x^0$に依存する階段関数は$J$の右の一つしか無いので,一つの項のみ生じる.第2項目と第3項目は$\mcO_{i_1}J\mcO_{i_2}\cdots\mcO_{i_n}$の順番から来る項だ.定義($*$)を見ればすぐ分かる通り,この項においては$x^0$に依存する階段関数は$J$の両側二つが存在しており,積の微分により二つの項が現れる.マイナス因子は,$J$の左側の階段関数の中身の$-x^0$から合成関数の微分により生じる.途中項を飛ばして,第4項目と第5項目は定義($*$)において$\mcO_{i_1}\cdots\mcO_{i_m}J\mcO_{i_{m+1}}\cdots\mcO_{i_n}$の項から生じる.これも第2,3項目と同じ理由により二つの項が現れる.第6,7項目と第8,9項目も第5,6項目と同様である.最後の項は定義($*$)において$\mcO_{i_1}\cdots \mcO_{i_n}J$の項から生じる.これも最初の項と同様の理由で,定義($*$)より$J$の左にしか$x^0$を含む階段関数は存在しないので一つの項しか生じない.\par
ここで$x^0$積分すれば,それぞれの項のデルタ関数のはたらきにより
\begin{align*}
=&\int d^3\mathbf{x} \left\langle \beta \middle| \biggl[\uwave{J^0(\mathbf{x},x^0_1)\mcO_{i_1}(\mathbf{x}_1,x^0_1)}\theta(x^0_1-x^0_2)\mcO_{i_2}(x_2)\cdots \theta(x^0_{n-1}-x^0_n)\mcO_{i_n}(x_n) \right. \\
&\qquad\qquad \uwave{-\mcO_{i_1}(\mathbf{x}_1,x^0_1)J^0(\mathbf{x},x^0_1)}\theta(x^0_1-x^0_2)\mcO_{i_2}(x_2)\cdots \theta(x^0_{n-1}-x^0_n)\mcO_{i_n}(x_n) \\
&+\mcO_{i_1}(x_1)\theta(x^0_1-x^0_2)\uwave{J^0(\mathbf{x},x^0_2)\mcO_{i_2}(\mathbf{x}_2,x^0_2)}\theta(x^0_2-x^0_3)\cdots \theta(x^0_{n-1}-x^0_n)\mcO_{i_n}(x_n) \\
& -\sim \\
& -\mcO_{i_1}\cdots \theta(x^0_{m-1}-x^0_m)\uwave{\mcO_{i_m}(\mathbf{x}_m,x^0_m)J(\mathbf{x},x^0_m)}\theta(x^0_m-x^0_{m+1})\mcO_{i_{m+1}}(x_{m+1})\cdots \mcO_{i_n}(x_n) \\
& +\mcO_{i_1}\cdots \theta(x^0_m-x^0_{m+1})\uwave{J(\mathbf{x},x^0_{m+1})\mcO_{i_{m+1}}(\mathbf{x}_{m+1},x^0_{m+1})}\theta(x^0_{m+1}-x^0_{m+2})\mcO_{i_{m+2}}(x_{m+2})\cdots \mcO_{i_n}(x_n) \\
& -\mcO_{i_1}\cdots \theta(x^0_m-x^0_{m+1})\uwave{\mcO_{i_{m+1}}(\mathbf{x}_{m+1},x^0_{m+1})J(\mathbf{x},x^0_{m+1})}\theta(x^0_{m+1}-x^0_{m+2})\mcO_{i_{m+2}}(x_{m+2})\cdots \mcO_{i_n}(x_n) \\
& +\mcO_{i_1}\cdots \mcO_{i_{m+1}}(x_{m+1})\theta(x^0_{m+1}-x^0_{m+2})\uwave{J(\mathbf{x},x^0_{m+2})\mcO_{i_{m+2}}(\mathbf{x}_{m+2},x^0_{m+2})}\theta(x^0_{m+2}-x^0_{m+3})\cdots \mcO_{i_n}(x_n) \\
& -\sim \\
&-\mcO_{i_1}\cdots \theta(x^0_{n-2}-x^0_{n-1})\uwave{\mcO_{i_{n-1}}(\mathbf{x}_{n-1},x^0_{n-1})J^0(\mathbf{x},x_{n-1})}\theta(x^0_{n-1}-x^0_n)\mcO_{i_n}(x_n) \\
& +\mcO_{i_1}\cdots \theta(x^0_{n-2}-x^0_{n-1})\mcO_{i_{n-1}}(x_{n-1})\theta(x^0_{n-1}-x^0_n)\uwave{J^0(\mathbf{x},x_n)\mcO_{i_n}(\mathbf{x}_n,x_n)} \\
& \left. -\mcO_{i_1}\cdots \theta(x^0_{n-2}-x^0_{n-1})\mcO_{i_{n-1}}(x_{n-1})\theta(x^0_{n-1}-x^0_n)\uwave{\mcO_{i_n}(\mathbf{x}_n,x_n)J^0(\mathbf{x},x_n)}+(交換項)\biggr]\middle| \alpha \right\rangle
\end{align*}
ここで(20.4.1)を第1,2項目の組,第2,3項目の組…に適応すれば次を得る.
\begin{align*}
=\int d^3 &\mathbf{x} \left\langle \beta \middle| \biggl[ -\delta^3(\mathbf{x-x_1}) \sum_{j_1}t_{i_1 j_1}\mcO_{j_1}(x_1)\theta(x^0_1-x^0_2)\mcO_{i_2}(x_2)\cdots \mcO_{i_n}(x_n) \right. \\
& -\delta^3(\mathbf{x-x_2})\sum_{j_2}t_{i_2 j_2}\mcO_{i_1}(x_1)\theta(x^0_1-x^0_2)\mcO_{j_2}(x_2)\theta(x^0_2-x^0_3)\cdots \mcO_{i_n}(x_n) \\
& -\cdots \\
& -\delta^3(\mathbf{x}-\mathbf{x}_m)\sum_{j_m}t_{i_m j_m}\mcO_{i_1}(x_1)\cdots\theta(x^0_{m-1}-x^0_m)\mcO_{j_m}(x_m)\theta(x^0_m-x^0_{m+1})\cdots \mcO_{i_n}(x_n) \\
& -\cdots \\
& -\delta^3(\mathbf{x}-\mathbf{x}_n)\sum_{j_n}t_{i_n j_n}\mcO_{i_1}(x_1)\cdots \theta(x^0_{n-1}-x^0_n)\mcO_{j_n}(x_n) \left. -(交換項)\biggr]\middle| \alpha \right\rangle \\
=-\sum^n_{r=1}&\sum_{j_r}t_{i_r j_r}\Braket{ \beta|T\left\{ \mcO_{i_1}(x_1)\cdots \mcO_{j_r}(x_r)\cdots \mcO_{i_n}(x_n) \right\}|\alpha}
\end{align*}
よって
\begin{align*}
&\bra{\beta}T\left\{ \mathcal{O}_{i_1}(x_1), \cdots , \mathcal{O}_{i_n}(x_n) \right\} \ket{\alpha+\pi} \\
&=\frac{i}{(2\pi)^{3/2}\sqrt{2q^0_\pi}F}\sum^n_{r=1}\sum_{j_r}t_{i_r j_r}\Braket{ \beta|T\left\{ \mcO_{i_1}(x_1)\cdots \mcO_{j_r}(x_r)\cdots \mcO_{i_n}(x_n) \right\}|\alpha}
\end{align*}
が導かれる.

\vskip\baselineskip

さて,この表式の両辺に演算子積展開(20.4.2)を適用する.$x_1,x_2,\cdots,x_n$が全て揃って$x$に近づく極限では以下を得る.
\begin{align*}
&\sum_i U^{i_1\cdots i_n}_i(x_1-x,x_2-x,\cdots,x_n-x)\bra{\beta} \mathcal{O}_i(x)\ket{\alpha+\pi} \\
&=\frac{i}{(2\pi)^{3/2}\sqrt{2q^0_\pi}F}\sum^n_{r=1}\sum_{j_r}\sum_{j}t_{i_r j_r}U^{i_1\cdots j_r \cdots i_n}_j(x_1-x,x_2-x,\cdots,x_n-x)\bra{\beta} \mathcal{O}_j(x)\ket{\alpha}
\end{align*}
また(20.4.5)の$n=1$の特別な場合について
\begin{align*}
\bra{\beta} \mathcal{O}_i(x)\ket{\alpha+\pi}=\frac{i}{(2\pi)^{3/2}\sqrt{2q^0_\pi}F}\sum_j t_{ij} \bra{\beta} \mathcal{O}_j(x)\ket{\alpha}
\end{align*}
が成り立つ.これらは全て任意の状態$\bra{\beta},\ket{\alpha}$について成り立つから,(20.4.6)の両辺の$\bra{\beta} \mathcal{O}_j(x)\ket{\alpha}$の係数は等しくなければならない.したがって
\begin{eqnarray*}
&&\frac{i}{(2\pi)^{3/2}\sqrt{2q^0_\pi}F}\sum_{ij}t_{ij} U^{i_1\cdots i_n}_i(x_1-x,x_2-x,\cdots,x_n-x)\bra{\beta} \mathcal{O}_j(x)\ket{\alpha} \\
&&=\frac{i}{(2\pi)^{3/2}\sqrt{2q^0_\pi}F}\sum^n_{r=1}\sum_{j_r}\sum_{j}t_{i_r j_r}U^{i_1\cdots i_n}_j(x_1-x,x_2-x,\cdots,x_n-x)\bra{\beta} \mathcal{O}_i(x)\ket{\alpha} \\
\Rightarrow && \quad \sum_{ij}t_{ij} U^{i_1\cdots i_n}_i(x_1-x,x_2-x,\cdots,x_n-x)= \sum^n_{r=1}\sum_{j_r}t_{i_r j_r}U^{i_1\cdots j_r \cdots i_n}_j(x_1-x,x_2-x,\cdots,x_n-x)
\end{eqnarray*}
これは,$U^{i_1\cdots i_n}_i(x_1-x,x_2-x,\cdots,x_n-x)$が$t$によって生成される対称性のもとで不変であることを意味する.すなわち
\begin{eqnarray*}
\left[Ut\right]_j^{i_1\cdots i_n} &\equiv& \sum_{ij} U^{i_1\cdots i_n}_i t_{ij} \\
\left[tU\right]_{j}^{i_1\cdots i_n}&\equiv& \sum^n_{r=1 }\sum_{j_r} t_{i_r j_r} U^{i_1\cdots j_r \cdots i_n}_j
\end{eqnarray*}
と定義すると交換関係$[U,t]=0$を満たす.

\vskip\baselineskip

時間順序積の定義($*$)はもう少しコンパクトに書けると思う.置換を用いれば,例えば
\begin{align*}
T\left\{\prod_{j=1}^n \mcO_{i_j}(x_j) \right\}=\sum_{\sigma_n\in S_n}\left[\left( \prod_{j=1}^n \mcO_{i_{\sigma_n(j)}} \right)\left(\prod_{k=1}^{n-1}\theta(x^0_{\sigma_n(k)}-x^0_{\sigma_n(k+1)})\right)\right]
\end{align*}
とできる.ここで置換$\sigma_n$は,$1$から$n$までの自然数の集合$N_n=\{1,2,\cdots ,n\}$に対しての全単射$\sigma_n:N_n\to N_n$であり,$S_n$は全ての$\sigma_n$の集合である.こっちの表記で今回の議論をすることもできるとは思うが,総積の中のひとつの階段関数を微分によりデルタ関数にするという操作が必要になるため,どちらにしても面倒なことになると思う.



\newpage

\subsection{スペクトル関数の和則}
ローレンツ不変性を用いて,ローレンツの4元ベクトルになっている以外は任意のカレント$J^\mu_{\alpha}$の集合は
\begin{align*}
&\sum_N \delta^4(p-p_N) \braket{\VAC|J^{\mu}_{\alpha}(0)| N }\braket{\VAC|J^\nu_{\beta}(0)|N}^* \\
&= \frac{1}{(2\pi)^3}\theta(p^0)\left[ \left (\eta^{\mu\nu}-\frac{p^\mu p^\nu}{p^2} \right)\rho^{(1)}_{\alpha\beta}(-p^2)+p^\mu p^\nu \rho^{(0)}_{\alpha\beta}(-p^2) \right]
\end{align*}
これは(10.7.4)(19.2.19)(19.2.20)と類似のもので,自然な定義だといえる.状態$\ket{N}$の完全性を用いてフーリエ変換することで
\begin{align*}
\mathrm{LHS}&=\sum_N  \delta^4(p-p_N) \braket{\VAC|J^{\mu}_{\alpha}(0)| N }\braket{\VAC|J^\nu_{\beta}(0)|N}^* \\
&=\frac{1}{(2\pi)^4}\sum_{N} \int d^4 x e^{-i(p-p_N)x}\braket{\VAC|J^\mu_{\alpha}(0)|N}\braket{\VAC|J^\nu_{\beta}(0)|N}^* \\
&=\frac{1}{(2\pi)^4}\sum_{N} \int d^4 x e^{-ipx}\braket{\VAC|J^\mu_{\alpha}(x)|N}\braket{N|J^\nu_{\beta}(0)|\VAC} \qquad \because(10.1.4)\\
&=\frac{1}{(2\pi)^4} \int d^4 x e^{-ipx}\braket{J^\mu_{\alpha}(x)J^\nu_{\beta}(0)}_{\VAC} \\
\mathrm{RHS}&= \frac{1}{(2\pi)^3}\theta(p^0)\left[ \left (\eta^{\mu\nu}-\frac{p^\mu p^\nu}{p^2} \right)\rho^{(1)}_{\alpha\beta}(-p^2)+p^\mu p^\nu \rho^{(0)}_{\alpha\beta}(-p^2) \right] \\
&=\frac{1}{(2\pi)^3}\int d\mu^2 \delta(p^2+\mu^2) \theta(p^0)\left[ \left (\eta^{\mu\nu}+\frac{p^\mu p^\nu}{\mu^2} \right)\rho^{(1)}_{\alpha\beta}(\mu^2)+p^\mu p^\nu \rho^{(0)}_{\alpha\beta}(\mu^2) \right]  
\end{align*}
両辺を逆フーリエ変換してやれば
\begin{align*}
\mathrm{LHS}&=\frac{1}{(2\pi)^4} \int d^4p e^{ipx} \int d^4 x' e^{-ipx'}\braket{J^\mu_{\alpha}(x')J^\nu_{\beta}(0)}_{\VAC} \\
&=\frac{1}{(2\pi)^4} \int d^4x' \int d^4 p e^{-ip(x'-x)} \braket{J^\mu_{\alpha}(x')J^\nu_{\beta}(0)}_{\VAC} \\
&=\frac{1}{(2\pi)^4} \int d^4x' \delta^4(x'-x) \braket{J^\mu_{\alpha}(x')J^\nu_{\beta}(0)}_{\VAC} \\
&=\braket{J^\mu_{\alpha}(x)J^\nu_{\beta}(0)}_{\VAC} \\
\mathrm{RHS}&=\frac{1}{(2\pi)^3} \int d^4x e^{ipx} \int d\mu^2 \delta(p^2+\mu^2) \theta(p^0)\left[ \left (\eta^{\mu\nu}+\frac{p^\mu p^\nu}{\mu^2} \right)\rho^{(1)}_{\alpha\beta}(\mu^2)+p^\mu p^\nu \rho^{(0)}_{\alpha\beta}(\mu^2) \right] \\
&= \int d\mu^2 \left[ \left (\eta^{\mu\nu}+\frac{(-i\partial^\mu) (-i\partial^\nu)}{\mu^2} \right)\rho^{(1)}_{\alpha\beta}(\mu^2)+(-i\partial^\mu) (-i\partial^\nu) \rho^{(0)}_{\alpha\beta}(\mu^2) \right] \\
&\qquad \times\frac{1}{(2\pi)^3}\int d^4p \theta(p^0)\delta(p^2+\mu^2)e^{ipx} \\
&=\int d\mu^2 \left[ \eta^{\mu\nu}\rho^{(1)}_{\alpha\beta}(\mu^2)-\left(\rho^{(0)}_{\alpha\beta}(\mu^2)+\frac{\rho^{(1)}_{\alpha\beta}(\mu^2)}{\mu^2}\right) \partial^\mu \partial^\nu \right] \Delta_{+}(x;\mu^2)
\end{align*}
ここで$\Delta_{+}(x;\mu^2)$は(5.2.7)(10.7.7)で定義される標準的な関数だ.

\vskip\baselineskip


カレントがエルミート演算子に選ばれていると仮定すると(20.5.1)より
\begin{align*}
\left[\sum_N \delta^4(p-p_N) \braket{\VAC|J^{\mu}_{\alpha}(0)| N }\braket{\VAC|J^\nu_{\beta}(0)|N}^* \right]^{\dagger }=\sum_N \delta^4(p-p_N) \braket{\VAC|J^{\nu}_{\beta}(0)| N }\braket{\VAC|J^\mu_{\alpha}(0)|N}^*
\end{align*}
であるが,右辺は(20.5.1)の$(\mu,\alpha)\leftrightarrow(\nu,\beta) $という入れ替えをしたものと等しい.したがって
\begin{align*}
&\frac{1}{(2\pi)^3}\theta(p^0)\left[ \left (\eta^{\mu\nu}-\frac{p^\mu p^\nu}{p^2} \right)\rho^{(1)\dagger}_{\alpha\beta}(-p^2)+p^\mu p^\nu \rho^{(0)\dagger}_{\alpha\beta}(-p^2) \right] \\
&= \frac{1}{(2\pi)^3}\theta(p^0)\left[ \left (\eta^{\mu\nu}-\frac{p^\mu p^\nu}{p^2} \right)\rho^{(1)}_{\beta\alpha}(-p^2)+p^\mu p^\nu \rho^{(0)}_{\beta\alpha}(-p^2) \right] \\
\Rightarrow &\rho^{(1)}_{\beta\alpha}=\rho^{(1)\dagger}_{\alpha\beta}=\rho^{(1)*}_{\beta\alpha},\qquad \rho^{(0)}_{\beta\alpha}=\rho^{(0)\dagger}_{\alpha\beta}=\rho^{(0)*}_{\beta\alpha}
\end{align*}
が導かれ,すなわち実であることが分かる.(エルミートではないと思う.)
また,$x$が空間的(このとき$\Delta_+(x;\mu^2)$は偶関数)にとって,(20.5.2)で並進不変性と因果律(空間的なら場の交換子がゼロであるという要請)を用いると
\begin{align*}
\braket{J^\mu_\alpha(x)J^\nu_\beta(0)}_{\VAC}&=\int d\mu^2 \left[ \eta^{\mu\nu}\rho^{(1)}_{\alpha\beta}(\mu^2)-\left(\rho^{(0)}_{\alpha\beta}(\mu^2)+\frac{\rho^{(1)}_{\alpha\beta}(\mu^2)}{\mu^2}\right) \partial^\mu \partial^\nu \right] \Delta_{+}(x;\mu^2) \\
&=\braket{J^\mu_\alpha(0)J^\nu_\beta(-x)}_{\VAC}\qquad \because 並進不変性 \\
&=\braket{J^\nu_\beta(-x)J^\mu_\alpha(0)}_{\VAC} \qquad \because 因果律 \\
&=\int d\mu^2 \left[ \eta^{\nu\mu}\rho^{(1)}_{\beta\alpha}(\mu^2)-\left(\rho^{(0)}_{\beta\alpha}(\mu^2)+\frac{\rho^{(1)}_{\beta\alpha}(\mu^2)}{\mu^2}\right) \partial^\nu \partial^\mu \right] \Delta_{+}(-x;\mu^2)
\end{align*}
$\Delta_+(-x;\mu^2)=\Delta_+(x;\mu^2)$が今回成り立っており,$\mu,\nu$については自明に対称なので
\begin{align*}
\rho^{(1)}_{\alpha\beta}=\rho^{(1)}_{\beta\alpha},\qquad \rho^{(0)}_{\alpha\beta}=\rho^{(0)}_{\beta\alpha}
\end{align*}
以上より,$\rho^{(1)}_{\alpha\beta},\rho^{(0)}_{\alpha\beta}$は実対称であることが分かる.(実対称はエルミートの一種だが,前半のみの議論でエルミートとは断ずることはできないと思う.)

\vskip\baselineskip

さて,$\Delta_+(x;\mu^2)$の$x\to 0$での漸近展開は(5.2.9)での第二種変形ベッセル関数(単に虚数倍の違いだが,ハンケル関数ではない)をwolfram alphaにでも入れるか小松勇作「特殊関数」「特殊関数演習」を参照すれば良いらしい.ここでは一応(5.2.7)から(5.2.9)をもう一度導いておく(他の本でも導出が省略されているらしいので).(5.2.7)より
\begin{align*}
x^0=0,\quad |\mathbf{x}|=\sqrt{x^2}
\end{align*}
として
\begin{align*}
\Delta_+(x)&=\frac{1}{(2\pi)^3}\int \frac{d^3\mathbf{p}}{2\sqrt{\mathbf{p}^2+m^2}}e^{ipx}\\ 
&=\frac{1}{(2\pi)^3}\int^{\infty}_{0}\int^{\pi}_{0}\int^{2\pi}_{0}\frac{e^{ipr\cos\theta}}{2\sqrt{\mathbf{p}^2+m^2}}p^2\sin\theta d\phi d\theta dp \\
&=\frac{2\pi}{(2\pi)^3}\int^{\infty}_{0} \frac{p^2}{2\sqrt{\mathbf{p}^2+m^2}}\frac{2\sin(pr)}{pr}dp \qquad \because \int^{\pi}_{0}\sin(x)e^{iax}dx=\frac{2\sin(a)}{a} \\
&=\frac{4\pi}{(2\pi)^3}\int^{\infty}_{0} \frac{p^2}{2\sqrt{\mathbf{p}^2+m^2}}\frac{\sin(pr)}{pr}dp  \quad(*)\\
&\underset{u=p/m}{=} \frac{m}{4\pi^2\sqrt{x^2}}\int^{\infty}_{0}\frac{u}{\sqrt{u^2+1}}\sin (m\sqrt{x^2}u)du
\end{align*}
ここで第二種変形ベッセル関数
\begin{align*}
&K_{\nu}(z)=\frac{\Gamma(\nu+\frac{1}{2})(2z)^\nu}{\sqrt{\pi}}\int^{\infty}_{0}dt \frac{\cos t}{(t^2+z^2)^{\nu+\frac{1}{2}}} \\
\Rightarrow \quad& \frac{\partial}{\partial z}K_0(z)=-K_1(z)
\end{align*}
を用いると($*$)式は
\begin{align*}
\frac{1}{4\pi^2 r}\int^\infty_0\frac{p^2dp}{\sqrt{p^2+m^2}}\frac{\sin(pr)}{p}&=\frac{1}{4\pi^2r}\int^\infty_0 \frac{p}{\sqrt{p^2+m^2}}\sin (pr)dp \\
&=\frac{-1}{4\pi^2r}\frac{\partial}{\partial r}\int^\infty_0\frac{1}{\sqrt{p^2+m^2}}\cos (pr) dp \\
&=\frac{-1}{4\pi^2r}\frac{\partial}{\partial r}\int^\infty_0 \frac{dq}{q}\frac{\cos(q)}{\sqrt{(q/r)^2+m^2}} \\
&=\frac{-1}{4\pi^2r}\frac{\partial}{\partial r}\int^\infty_0 dq \frac{\cos(q)}{\sqrt{q^2+(mr)^2}} \\
&=\frac{-1}{4\pi^2 r}\frac{\partial}{\partial r} K_0(mr) \\
&=\frac{m}{4\pi^2r}K_1(mr)
\end{align*}
よって$x^0=0,\quad |\mathbf{x}|=\sqrt{x^2}$で
\begin{align*}
\Delta_+(x)=\frac{m}{4\pi^2r}K_1(mr)
\end{align*}
で,(5.2.9)が導かれた.\par
第二種変形ベッセル関数の$x\to0$のときの漸近展開はwolfram alphaに入れると,$\gamma$をオイラー定数として
\begin{align*}
K_1(x)=\frac{1}{x}+\frac{1}{4}x(2\ln x+2\gamma-1-\ln 4)+O(x^3)
\end{align*}
と振る舞うらしい.よって$\Delta_+(x)$は$x^2>0$で$x\to0$のとき
\begin{align*}
\Delta_+(x)&=\frac{\mu}{4\pi^2\sqrt{x^2}}K_1(\mu\sqrt{x^2}) \\
&\to \frac{1}{4\pi x^2}+\frac{\mu^2}{8\pi^2}\left( \ln \left(\frac{\mu\sqrt{x^2}}{2}\right) -\gamma \right)+O(x^2)
\end{align*}
と振る舞う.したがって$\braket{J^\mu_\alpha(x)J^\nu_\beta(0)}_{\VAC}$の展開の最初の数項は(20.5.2)より
\begin{align*}
\braket{J^\mu_\alpha(x)J^\nu_\beta(0)}_{\VAC} &\to \int d\mu^2 \left[ \eta^{\mu\nu}\rho^{(1)}_{\alpha\beta}(\mu^2)-\left(\rho^{(0)}_{\alpha\beta}(\mu^2)+\frac{\rho^{(1)}_{\alpha\beta}(\mu^2)}{\mu^2}\right) \partial^\mu \partial^\nu \right] \\
&\qquad\qquad\qquad \times \left[ \frac{1}{4\pi x^2}+\frac{\mu^2}{8\pi^2}\left( \ln \left(\frac{\mu\sqrt{x^2}}{2}\right) -\gamma \right)\right]+O(\ln x^2) \\
&=\int d\mu^2 \Biggl[ \frac{\eta^{\mu\nu}}{4\pi^2x^2}\rho^{(1)}_{\alpha\beta}(\mu^2) -\left( \rho^{(0)}_{\alpha\beta}(\mu^2)+\frac{\rho^{(1)}_{\alpha\beta}(\mu^2)}{\mu^2} \right) \left( \partial^\mu \partial^\nu \frac{1}{4\pi^2x^2} \right) \\
&\quad -\frac{1}{8\pi^2}\left( \mu^2\rho^{(0)}_{\alpha\beta}(\mu^2)+\rho^{(1)}_{\alpha\beta}(\mu^2) \right)\left( \partial^\mu\partial^\nu \ln\left( \frac{\mu\sqrt{x^2}}{2} \right) \right) \Biggr]+O(\ln x^2)
\end{align*}
ここで
\begin{align*}
&\partial^\mu\partial^\nu \left[\frac{1}{x^2}\right]=\partial^\mu\left(-\frac{2x^\nu}{(x^2)^2}\right)=-\frac{2\eta^{\mu\nu}}{(x^2)^2}+\frac{4x^\mu\cdot 2x^\nu}{(x^2)^3} \qquad \because \partial^\mu x^\nu=\eta^{\mu\rho}\partial_\rho x^\nu=\eta^{\mu\rho}\delta^\nu_\rho=\eta^{\mu\nu} \\
&\partial^\mu\partial^\nu \left[\ln \frac{\mu\sqrt{x^2}}{2}\right]=\partial^\mu \left( \frac{x^\nu}{x^2} \right)=\frac{\eta^{\mu\nu}}{x^2}-\frac{2x^\mu x^\nu}{(x^2)^2}
\end{align*}
よって
\begin{align*}
\braket{J^\mu_\alpha(x)J^\nu_\beta(0)}_{\VAC} \to &\frac{1}{2\pi^2}\left[ \frac{\eta^{\mu\nu}}{(x^2)^2}-\frac{4x^\mu x^\nu}{(x^2)^3} \right]\int d\mu^2 \left( \rho^{(0)}_{\alpha\beta}(\mu^2)+\frac{\rho^{(1)}_{\alpha\beta}(\mu^2)}{\mu^2} \right) \\
&+\frac{\eta^{\mu\nu}}{4\pi^2x^2}\int d\mu^2 \rho^{(1)}_{\alpha\beta}(\mu^2)-\left[ \frac{\eta^{\mu\nu}}{8\pi^2 x^2}-\frac{x^\mu x^\nu}{4\pi^2(x^2)^2} \right]\int d\mu^2 \left(\mu^2\rho^{(0)}_{\alpha\beta}(\mu^2)+\rho^{(1)}_{\alpha\beta}(\mu^2)\right) \\
&+O(\ln x^2)
\end{align*}
(結構違うけど多分誤植.次の結論に違いは出ないが.)もし二点関数のなんらかの線形結合$\sum_{\alpha\beta}c_{\alpha\beta}\braket{J^\mu_\alpha(x) J^\nu_\beta(0)}_{\VAC}$が$x\to0$において特異性を持っていて,それが$1/x^4$の次数より弱いと示すことができたならば
\begin{align*}
\sum_{\alpha\beta}c_{\alpha\beta}\int d\mu^2 \left( \rho^{(0)}_{\alpha\beta}(\mu^2)+\frac{\rho^{(1)}_{\alpha\beta}(\mu^2)}{\mu^2} \right)=0 
\end{align*}
が要請される.また,もしその特異性がさらに$1/x^2$より弱ければ
\begin{align*}
\sum_{\alpha\beta}c_{\alpha\beta}\int d\mu^2 \rho^{(1)}_{\alpha\beta}(\mu^2)=0,\quad \sum_{\alpha\beta}c_{\alpha\beta}\int d\mu^2 \mu^2 \rho^{(0)}_{\alpha\beta}(\mu^2)=0
\end{align*}
が要請される.これらは,第1~3のスペクトル関数の和則として知られている,らしい.

\vskip\baselineskip

これが最も興味のある場合,すなわち量子色力学においてどのようにはたらくか見る.$J^\mu_\alpha(x)$が量子色力学のような理論の保存カレントの場合,カレントの保存則から(20.5.1)は$p_\mu$と縮約すると消える.なぜなら
\begin{align*}
0&=\bra{\VAC}\partial_\mu J^\mu_\alpha(x)\ket{N}=\partial_\mu \left( \bra{\VAC}J^\mu_\alpha(0) \ket{N}e^{ip_Nx}\right) \\
&=ip^N_\mu\bra{\VAC}J^\mu_\alpha(0) \ket{N}e^{ip_Nx} \\
\Rightarrow &\quad p^N_\mu \bra{\VAC}J^\mu_\alpha(0) \ket{N}=0 \qquad\because e^{ip_Nx}\neq 0
\end{align*}
よって,(20.5.1)を$p_\mu$と縮約すると
\begin{align*}
\mathrm{LHS}&=\sum_N \delta^4(p-p_N) p_\mu \braket{\VAC|J^{\mu}_{\alpha}(0)| N }\braket{\VAC|J^\nu_{\beta}(0)|N}^* \\
&=\sum_N \delta^4(p-p_N) \underset{=0}{\uwave{p^N_\mu \braket{\VAC|J^{\mu}_{\alpha}(0)| N }}}\braket{\VAC|J^\nu_{\beta}(0)|N}^* \\
\mathrm{RHS}&= \frac{1}{(2\pi)^3}\theta(p^0)\left[ \underset{=0}{\left (\uwave{p^\nu-\frac{p^\nu p^2}{p^2} }\right)}\rho^{(1)}_{\alpha\beta}(-p^2)+p^2 p^\nu \rho^{(0)}_{\alpha\beta}(-p^2) \right] \\
&=\frac{\theta(p^0)}{(2\pi)^3}p^2p^\nu \rho^{(0)}_{\alpha\beta}(-p^2) \\
\Rightarrow &\quad p^2 \rho^{(0)}_{\alpha\beta}(-p^2)=0 \\
\Rightarrow &\quad \rho^{(0)}_{\alpha\beta}(-p^2)\propto \delta(-p^2) \qquad \because x\delta(x)=0
\end{align*}
となって,$\rho^{(0)}_{\alpha\beta}(-p^2)$は$\delta(-p^2)$に比例していなければならない.\par
$x\delta(x)=0$の性質を用いれば,(20.5.8)の被積分関数はゼロとなり自動的に第3の和則は満たされていることが分かる.3巻p231の$\ell$12の議論と同様,$\braket{\VAC|J^{\mu}_{\alpha}(0)| N }$は$J^\mu_\alpha$と同じパリティ・内部量子数を持つ$N=B_a$のみから寄与受ける.この粒子は実質NGボゾンだ.\par
ローレンツ不変性により
\begin{eqnarray*}
\bra{\VAC}J^\mu_\alpha(0)\ket{B_a}=\frac{iF_{\alpha a}p^\mu_B}{(2\pi)^{3/2}\sqrt{2p^0_B}}
\end{eqnarray*}
で,(19.2.36)の上の計算と同様,(20.5.1)より
\begin{align*}
&\sum_N \delta^4(p-p_N) \bra{\VAC}J^\mu_\alpha(0)\ket{N}\bra{\VAC}J^\nu_\beta(0)\ket{N}^* \\
&=\frac{1}{(2\pi)^4} \int d^4 x e^{-ipx}\braket{J^\mu_{\alpha}(x)J^\nu_{\beta}(0)}_{\VAC} \\
&=\frac{1}{(2\pi)^4} \int d^4 x e^{-ipx}\int d^3\mathbf{p}_B \bra{\VAC}J^\mu_{\alpha}(x)\ket{B_a}\bra{B_a}J^\nu_{\beta}(0)\ket{\VAC} \\
&=\frac{1}{(2\pi)^4} \int d^4 x e^{-i(p-p_B )x}\int d^3\mathbf{p}_B \bra{\VAC}J^\mu_{\alpha}(0)\ket{B_a}\bra{B_a}J^\nu_{\beta}(0)\ket{\VAC} \\
&=\sum_a \int d^3\mathbf{p}_B \bra{\VAC}J^\mu_\alpha(0)\ket{B_a}\bra{\VAC}J^\nu_\beta(0)\ket{B_a}^*\delta^4(p-p_B) \\
&=\sum_a \int d^3 \mathbf{p}_B \frac{F_{\alpha a}F^*_{\beta a}}{(2\pi)^3(2p^0_B)}p^\mu_B p^\nu_B \delta^4(p-p_B) \\
&=\sum_a \frac{F_{\alpha a}F^*_{\beta a}}{(2\pi)^3(2p^0)}p^\mu p^\nu \delta(p^0-|\mathbf{p}|) \qquad \because デルタ関数により\mathbf{p}_B=\mathbf{p}でp^0_B=|\mathbf{p}_B|=|\mathbf{p}| 
\end{align*}
デルタ関数の公式より$\theta(p^0)\delta(-p^2)=\delta(p^0-|\mathbf{p}|)(2p^0)^{-1}$であるから
\begin{align*}
=\sum_a \frac{F_{\alpha a}F^*_{\beta a}}{(2\pi)^3}p^\mu p^\nu \theta(p^0)\delta(-p^2)
\end{align*}
これが(20.5.1)と等しいことから,$p^\mu p^\nu$の係数比較をして
\begin{align*}
\rho^{(0)}_{\alpha\beta}(-p^2)=\delta(-p^2)\sum_a F_{\alpha a}F^*_{\beta a}
\end{align*}
を得る.対して$\rho^{(1)}_{\alpha\beta}(-p^2)$は$-p^2>0(-p^2は正で,-p^2\neq 0)$のときのみゼロでない.$\rho^{(0)}_{\alpha\beta}$は$-p^2\neq0$でゼロなので,そのときは$\rho^{(1)}_{\alpha\beta}$が(20.5.1)の寄与となる.


\vskip\baselineskip


より特定の例を扱うために,くりこみ可能で漸近的自由なゲージ理論において,$N$個のスピン1/2の質量ゼロ(もしくはほとんどゼロ)のフェルミオンがゲージ群の同じ表現に属している場合を考える.量子色力学は$u,d,s$クォークの質量を無視すれば$N=3$,また$u,d$クォークの質量を無視すれば$N=2$としてこの描像に当てはまる.\par
19章で見たように,そのような理論には$SU(N)\times SU(N)$対称性が存在する.この対称性は軽いフェルミオン場の左手と右手成分を,それぞれ$(N,1)$と$(1,N)$表現として変換する.ここで$Nと1$はそれぞれ$SU(N)$の基本表現と単位表現を意味する.つまり,$SU(N)\times SU(N)$の元を
\begin{align*}
g=\exp \left[ i\sum_a(\theta^V_a \lambda_a +\theta^A_a \gamma_5\lambda_a) \right] \in SU(N)_V \times SU(N)_A
\end{align*}
あるいは同等に($\theta_L=\theta_V+\theta_A,\theta_R=\theta_V-\theta_A$とすれば上と同じ)
\begin{align*}
g=\exp \left[ i\sum_a\left(\frac{1+\gamma_5}{2}\theta^L_a \lambda_a +\frac{1-\gamma_5}{2}\theta^R_a \lambda_a\right) \right] \in SU(N)_L \times SU(N)_R
\end{align*}
と書ける.このとき左手フェルミオン場$\psi_L$と右手フェルミオン場$\psi_R$は
\begin{align*}
g\psi_L=g\frac{1+\gamma_5}{2}\psi&=\exp \left[ i\sum_a\left(\frac{1+\gamma_5}{2}\theta^L_a \lambda_a +\frac{1-\gamma_5}{2}\theta^R_a \lambda_a\right) \right]\frac{1+\gamma_5}{2}\psi \\
&=\exp \left[ i\sum_a\left(\frac{1+\gamma_5}{2}\theta^L_a \lambda_a\right) \right]\frac{1+\gamma_5}{2}\psi=\exp \left[ i\sum_a\left(\frac{1+\gamma_5}{2}\theta^L_a \lambda_a\right) \right]\psi_L \\
g\psi_R=g\frac{1-\gamma_5}{2}\psi&=\exp \left[ i\sum_a\left(\frac{1+\gamma_5}{2}\theta^L_a \lambda_a +\frac{1-\gamma_5}{2}\theta^R_a \lambda_a\right) \right]\frac{1-\gamma_5}{2}\psi \\
&=\exp \left[ i\sum_a\left(\frac{1-\gamma_5}{2}\theta^R_a \lambda_a\right) \right]\frac{1-\gamma_5}{2}\psi=\exp \left[ i\sum_a\left(\frac{1-\gamma_5}{2}\theta^R_a \lambda_a\right) \right]\psi_R
\end{align*}
と変換する.$\psi_L$は$SU(N)_L$の元に対しては基本表現として変換するが,$SU(N)_R$の元に対しては単位表現として変換することが分かる.$\psi_R$はその逆である.すなわち,フェルミオン場の左手と右手成分はそれぞれ$(N,1)表現と(1,N)$表現であることがわかる.\par
左手と右手の$SU(N)$対称性のカレントは(7.3.15)より($F=\lambda_a \psi_{L,R}$なので)
\begin{align*}
\mathcal{L}_M&=-\bar{\psi}(\gamma^\mu\partial_\mu +m)\psi=-\bar{\psi}_L(\gamma^\mu\partial_\mu +m)\psi_L-\bar{\psi}_R(\gamma^\mu\partial_\mu +m)\psi_R \\
J^\mu_{L a}&=-i\frac{\partial \mathcal{L}_M}{\partial (\partial_\mu \psi_L)}\lambda_a\psi_L=i\bar{\psi}_L\gamma^\mu\lambda_a \psi_L=i\bar{\psi}\gamma^\mu \frac{1+\gamma_5}{2}\lambda_a \psi \\
J^\mu_{R a}&=-i\frac{\partial \mathcal{L}_M}{\partial (\partial_\mu \psi_R)}\lambda_a\psi_R=i\bar{\psi}_R\gamma^\mu\lambda_a \psi_R=i\bar{\psi}\gamma^\mu \frac{1-\gamma_5}{2}\lambda_a \psi
\end{align*}
で与えられる.(-1/2がないが,誤植ではない.定数倍違ってもカレントではある.(20.5.14)の定義も少し違ってくるがスペクトル関数の和則(20.5.18)(20.5.19)は結局変わらない.より合理的な起源にするためにここではこの定義を用いることにする.)ここで$\lambda_a$は($N=2$でのパウリ行列や$N=3$でのゲルマン行列のような)$N$個の軽いクォークを区別する「フレーバー」の添え字にはたらくトレースゼロ・エルミートの行列の完全系をなす.($\psi$がスピン1/2なので次元は3/2より)これらのカレントは質量次元+3を持ち,2つのカレントの積の展開(20.3.6)における次元$d(\mcO)$の演算子の係数は$x\to0$で$x^{-6+d(\mcO)}$の次数の特異性を持つ.($x$は質量とは逆の次元なので$x^{-6+d(\mcO)}$)
\begin{align*}
\underset{+6}{\uwave{\braket{J(x)J(0)}}}\to \sum_\mcO \underset{6-d(\mcO)}{\uwave{U_\mcO(x)}}\underset{d(\mcO)}{\uwave{\mcO(x)}}
\end{align*}
したがって,もしカレントの積の線型結合展開が単位演算子$\mcO=I$を含めば,$d(I)=0$なので,対応するスペクトル関数の線形結合は$x^{-6}$のように振る舞う.\par
$\Rightarrow$一般に,スペクトル関数の第1,2の和則はどちらも満たさない.\par
一方,この展開の真空期待値の最低次の演算子が微分(次元+1)をゼロ個または1個しか持たないフェルミオンの二次形式(+3)なら,スペクトル関数の線型結合は$x^{-6+3}=x^{-3}$か$x^{-6+4}=x^{-2}$のように振る舞い,したがってスペクトル関数の第1の和則は満たすが,一般に第2の和則は満たさない.\par
また,この展開の真空期待値の最低次の演算子が二つ以上の微分(+2)を含むフェルミオンの二次形式(+3)か,フェルミオンの四次形式(+6)ならば,対応するスペクトル関数の線型結合は$x^{-2}$よりも特異性は弱く,したがってスペクトル関数の第1,2の和則を満たす.\par
二つのカレントの積の展開に,どの演算子が現れるかを知るためには,積に含まれる$SU(N)\times SU(N)$の表現の分類する必要がある.また,これらの演算易の真空期待値がゼロでないかどうかを知るには,自発的に破れていない$SU(N)\times SU(N)$の部分群のもとでどれが不変であるかを問わなくてはいけない.(演算子積の展開で現れる演算子は,演算子の積全体の表現と同じ表現に属していなければならない.)\par
$\Rightarrow$これらの質問に答えるためには,カレント$J^\mu_{L a}(x),J^\mu_{R b}(x)$は$SU(N)\times SU(N)$のもとで,それぞれ$SU(N)\times SU(N)$のもとで$(A,1)表現と(1,A)表現$として変換されることに注意する.ここで$A$と1は,$SU(N)$の随伴表現と単位表現である.実際$SU(N)\times SU(N)$
\begin{align*}
&g=\exp \left[ i\sum_a\left(\frac{1+\gamma_5}{2}\theta^L_a \lambda_a +\frac{1-\gamma_5}{2}\theta^R_a \lambda_a\right) \right] 
\end{align*}
の変換の下で
\begin{align*}
J^\mu_{L a}(x)&=i\bar{\psi}(x)\gamma^\mu \frac{1+\gamma_5}{2}\lambda_a \psi(x) \\
\rightarrow J'^\mu_{L a}(x)&=i[g\psi(x)]^\dagger \beta\gamma^\mu \frac{1+\gamma_5}{2}\lambda_a g\psi(x) \\
&=i\psi^\dagger (x) \exp \left[ -i\sum_a\left(\frac{1+\gamma_5}{2}\theta^L_a \lambda_a +\frac{1-\gamma_5}{2}\theta^R_a \lambda_a\right) \right] \beta \\
& \quad \times \gamma^\mu \frac{1+\gamma_5}{2}\lambda_a \exp \left[ i\sum_a\left(\frac{1+\gamma_5}{2}\theta^L_a \lambda_a +\frac{1-\gamma_5}{2}\theta^R_a \lambda_a\right) \right]\psi(x) \quad \because \lambda_aはエルミート \\
&=i\bar{\psi}(x) \exp \left[ -i\sum_a\left(\frac{1-\gamma_5}{2}\theta^L_a \lambda_a +\frac{1+\gamma_5}{2}\theta^R_a \lambda_a\right) \right] \\
& \quad \times \frac{1-\gamma_5}{2} \gamma^\mu \frac{1+\gamma_5}{2}\lambda_a \exp \left[ i\sum_a\left(\frac{1+\gamma_5}{2}\theta^L_a \lambda_a +\frac{1-\gamma_5}{2}\theta^R_a \lambda_a\right) \right]\psi(x) \quad \because \gamma_5\gamma^\mu=-\gamma^\mu \gamma_5 \\
&=i\bar{\psi}(x) \exp \left[ -i\sum_a \theta^L_a \lambda_a \right] \frac{1-\gamma_5}{2} \gamma^\mu \frac{1+\gamma_5}{2}\lambda_a \exp \left[ i\sum_a \theta^L_a \lambda_a \right]\psi(x) \\
&=i\bar{\psi}(x) \exp \left[ -i\sum_a \theta^L_a \lambda_a \right] \lambda_a \exp \left[ i\sum_a \theta^L_a \lambda_a \right]\gamma^\mu \frac{1+\gamma_5}{2}\psi(x) \\
&=i\bar{\psi}(x) \left[\mathrm{Ad}(g_L^{-1})\lambda_a \right] \gamma^\mu \frac{1+\gamma_5}{2}\psi(x)
\end{align*}
ここでリー群の随伴表現
\begin{align*}
\mathrm{Ad}(g)X=gXg^{-1} ,\quad g\in G ,X\in \mathfrak{g}
\end{align*}
を用いた.(小林俊行・大島利雄著「リー群と表現論」p211参照)また$g_L$は$SU(N)_L$の元
\begin{align*}
g_L=\exp \left[ i\sum_a \theta^L_a \lambda_a \right] \in SU(N)_L
\end{align*}
である.全く同様にして
\begin{align*}
J^\mu_{R a}(x)&=i\bar{\psi}(x)\gamma^\mu \frac{1-\gamma_5}{2}\lambda_a \psi(x) \\
\rightarrow J'^\mu_{R a}(x)&=i\bar{\psi}(x) \left[\mathrm{Ad}(g_R^{-1})\lambda_a \right] \gamma^\mu \frac{1-\gamma_5}{2}\psi(x)
\end{align*}
となる.ここで$g_R$は$SU(N)_R$の元
\begin{align*}
g_R=\exp \left[ i\sum_a \theta^R_a \lambda_a \right] \in SU(N)_R
\end{align*}
である.したがってカレント$J^\mu_{L a}(x),J^\mu_{R b}(x)$は$SU(N)\times SU(N)$のもとで,それぞれ$SU(N)\times SU(N)$のもとで$(A,1)表現と(1,A)表現$として変換されることが確認できた.\par
ここで,$SU(N)\times SU(N)$の自発的に破れていない部分群はベクトル的な$SU(N)_V$であって,そのカレントは$J^\mu_{L a}(x)+J^\mu_{R a}(x)(=J^\mu_{V a}(x))$であるとする.ここで
\begin{align*}
J^\mu_{V a}=i\bar{\psi}\gamma^\mu\lambda_a\psi ,\quad J^\mu_{A a}=i\bar{\psi}\gamma^\mu \gamma_5 \lambda_a \psi
\end{align*}
と定義する.
これは量子色力学と(19.9節で見たように)他の広い種類の理論でも成立している.また,パリティ保存は自発的には\uwave{破れていない}(すなわち$\mathbf{P}\ket{0}=\ket{0}$)とする.以上より
\begin{align*}
\braket{J^\mu_{\alpha}(x)J^\nu_{\beta}(0)}&=\bra{0}\mathbf{P^\dagger P}J^\mu_{\alpha}(x)J^\nu_{\beta}(0)\mathbf{P^\dagger P}\ket{0}\\
&=\bra{0}\mathbf{P}J^\mu_{\alpha}(x)J^\nu_{\beta}(0)\mathbf{P}^{-1}\ket{0}
\end{align*}
が要請される.よって(5.4.30)(5.5.57)等を用いると
\begin{align*}
\braket{J^\mu_{V a}(x)J_{\mu A b}(0)}&=\braket{\mathbf{P}\left\{ -\bar{\psi}(x)\gamma^\mu \lambda_a \psi(x) \bar{\psi}(0)\gamma_\mu \gamma_5 \lambda_b \psi(0) \right\}\mathbf{P}} \\
&=\braket{ -\bar{\psi}(\mathcal{P}x) \beta \gamma^\mu \beta \lambda_a \psi(\mathcal{P}x) \bar{\psi}(0)\beta \gamma_\mu \gamma_5 \beta \lambda_b \psi(0) } \\
&=\braket{ +\bar{\psi}(\mathcal{P}x) \gamma^{\mu\dagger} \lambda_a \psi(\mathcal{P}x) \bar{\psi}(0) \gamma_\mu^\dagger \gamma_5 \lambda_b \psi(0) } \quad \because \beta\gamma^\mu\beta =-\gamma^{\mu\dagger},\gamma_5\beta=-\beta\gamma_5 \\
&=\braket{ \bar{\psi}(x)\gamma^\mu \lambda_a \psi(x) \bar{\psi}(0)\gamma_\mu \gamma_5 \lambda_b \psi(0) } \quad \because \gamma^{0\dagger}=-\gamma^0,\gamma^{i\dagger}=\gamma^i\\
&=-\braket{J^\mu_{Va}(\mathcal{P}x)J_{\mu Ab}(0)}
\end{align*}
ここで$x$が空間的とすると$\rho_{Va\, Ab}=0$が分かる.同様にして$\rho_{A a\, Vb}=0$も分かる.また
\begin{align*}
\braket{J^\mu_{L a}(x)J_{\mu L b}(0)}&=\braket{-\bar{\psi}(x)\gamma^\mu \frac{1+\gamma_5}{2}\lambda_a \psi(x)\bar{\psi}(0)\gamma^\mu \frac{1+\gamma_5}{2}\lambda_a \psi(0)} \\
&=\braket{-\bar{\psi}(\mathcal{P}x)\gamma^\mu \frac{1-\gamma_5}{2}\lambda_a \psi(\mathcal{P}x)\bar{\psi}(0)\gamma^\mu \frac{1-\gamma_5}{2}\lambda_a \psi(0)} \\
&=\braket{J^\mu_{R a}(\mathcal{P}x)J_{\mu R b}(0)}
\end{align*}
ここで$x$が空間的だとすれば$\rho_{La\, Lb}=\rho_{Ra\, Rb}$が分かる.同様に
\begin{align*}
\braket{J^\mu_{L a}(x)J_{\mu R b}(0)}&=\braket{-\bar{\psi}(x)\gamma^\mu \frac{1+\gamma_5}{2}\lambda_a \psi(x)\bar{\psi}(0)\gamma^\mu \frac{1-\gamma_5}{2}\lambda_a \psi(0)} \\
&=\braket{-\bar{\psi}(\mathcal{P}x)\gamma^\mu \frac{1-\gamma_5}{2}\lambda_a \psi(\mathcal{P}x)\bar{\psi}(0)\gamma^\mu \frac{1+\gamma_5}{2}\lambda_a \psi(0)} \\
&=\braket{J^\mu_{R a}(\mathcal{P}x)J_{\mu R b}(0)}
\end{align*}
より$\rho_{La\, Rb}=\rho_{Ra\, Lb}$が分かる.\par
以上より
\begin{align*}
\braket{J^\mu_{L a}(x)J^\nu_{L b}(0)}&=\frac{1}{4}\braket{\left\{ J^\mu_{Va}(x)+J^\mu_{Aa}(x) \right\}\left\{ J^\nu_{Vb}(0)+J^\nu_{Ab}(0)\right\}} \\
&=\frac{1}{4} \Bigl[ \braket{J^\mu_{V a}(x)J^\nu_{V b}(0)} +\braket{J^\mu_{V a}(x)J^\nu_{A b}(0)} + \braket{J^\mu_{A a}(x)J^\nu_{V b}(0)} + \braket{J^\mu_{A a}(x)J^\nu_{A b}(0)}\Bigr] 
\end{align*}
を用いると
\begin{align*}
\rho^{(1)}_{L a L b}(\mu^2)=\rho^{(1)}_{R a R b}(\mu^2)=\frac{1}{4}\delta_{ab}[\rho^{(1)}_V(\mu^2)+\rho^{(1)}_A{(\mu^2 )}]
\end{align*}
がわかる.ここで$\delta_{ab}\rho^{(1)}_V,\delta_{ab}\rho^{(1)}_A$は
\begin{align*}
&\braket{J^\mu_{V a}(x)J^\nu_{V b}(0)}=\int d\mu^2 \left[ \eta^{\mu\nu}\delta_{ab}\rho^{(1)}_{V}(\mu^2)-\delta_{ab}\left(\rho^{(0)}_{V}(\mu^2)+\frac{\rho^{(1)}_{V}(\mu^2)}{\mu^2}\right) \partial^\mu \partial^\nu \right] \Delta_{+}(x;\mu^2) \\
&\braket{J^\mu_{A a}(x)J^\nu_{A b}(0)}=\int d\mu^2 \left[ \eta^{\mu\nu}\delta_{ab}\rho^{(1)}_{A}(\mu^2)-\delta_{ab}\left(\rho^{(0)}_{A}(\mu^2)+\frac{\rho^{(1)}_{A}(\mu^2)}{\mu^2}\right) \partial^\mu \partial^\nu \right] \Delta_{+}(x;\mu^2)
\end{align*}
で定められる関数だ.同様にして
\begin{align*}
\braket{J^\mu_{L a}(x)J^\nu_{R b}(0)}&=\frac{1}{4}\braket{\left\{ J^\mu_{Va}(x)+J^\mu_{Aa}(x) \right\}\left\{ J^\nu_{Vb}(0)-J^\nu_{Ab}(0)\right\}} \\
&=\frac{1}{4} \Bigl[ \braket{J^\mu_{V a}(x)J^\nu_{V b}(0)} -\braket{J^\mu_{V a}(x)J^\nu_{A b}(0)} + \braket{J^\mu_{A a}(x)J^\nu_{V b}(0)} + \braket{J^\mu_{A a}(x)J^\nu_{A b}(0)}\Bigr] 
\end{align*}
を用いると
\begin{align*}
\rho^{(1)}_{L a R b}(\mu^2)=\rho^{(1)}_{R a L b}(\mu^2)=\frac{1}{4}\delta_{ab}[\rho^{(1)}_V(\mu^2)-\rho^{(1)}_A{(\mu^2 )}] 
\end{align*}
が分かる.\par
また,(19.4.24)と同様に
\begin{align*}
\braket{\VAC|J^\mu_{A a}(0)|B_b}=\frac{i\delta_{ab} F p^\mu_B}{(2\pi)^{3/2}\sqrt{2p^0_B}}
\end{align*}
とすれば,
\begin{align*}
\frac{i F_{L ab} p^\mu_B}{(2\pi)^{3/2}\sqrt{2p^0_B}}&=\braket{\VAC|J^\mu_{L a}(0)|B_b}\\
&=\frac{1}{2}\uwave{\braket{\VAC|J^\mu_{V a}(0)|B_b}}+\frac{1}{2}\braket{\VAC|J^\mu_{A a}(0)|B_b}=\frac{i\delta_{ab} F p^\mu_B}{2(2\pi)^{3/2}\sqrt{2p^0_B}} \\
\frac{i F_{R ab} p^\mu_B}{(2\pi)^{3/2}\sqrt{2p^0_B}}&=\braket{\VAC|J^\mu_{L a}(0)|B_b}\\
&=\frac{1}{2}\uwave{\braket{\VAC|J^\mu_{V a}(0)|B_b}}-\frac{1}{2}\braket{\VAC|J^\mu_{A a}(0)|B_b}=-\frac{i\delta_{ab} F p^\mu_B}{2(2\pi)^{3/2}\sqrt{2p^0_B}} \\
\Rightarrow &F_{Lab}=-F_{Rab}=\delta_{ab}F/2
\end{align*}
(19.4.44)と同じ形式のものができる.ここで,第二式目において$J_V$は破れていないカレントなので NGボゾンの量子数と異なり,第一項目はゼロであることに留意する.これにより(20.5.10)は
\begin{align*}
&\rho^{(0)}_{LaLb}(\mu^2)=\delta(\mu^2)\sum_c F_{L ac}F^*_{L bc}=\frac{1}{4}\delta(\mu^2)\sum_c F^2 \delta_{ac}\delta_{bc}=\frac{1}{4}F^2 \delta(\mu^2)\delta_{ab}\\
&\rho^{(0)}_{RaRb}(\mu^2)=\delta(\mu^2)\sum_c F_{R ac}F^*_{R bc}=\frac{1}{4}\delta(\mu^2)\sum_c F^2 \delta_{ac}\delta_{bc}=\frac{1}{4}F^2 \delta(\mu^2)\delta_{ab}\\
&\rho^{(0)}_{LaRb}(\mu^2)=\delta(\mu^2)\sum_c F_{L ac}F^*_{R bc}=-\frac{1}{4}\delta(\mu^2)\sum_c F^2 \delta_{ac}\delta_{bc}=-\frac{1}{4}F^2 \delta(\mu^2)\delta_{ab}\\
&\rho^{(0)}_{RaLb}(\mu^2)=\delta(\mu^2)\sum_c F_{R ac}F^*_{L bc}=-\frac{1}{4}\delta(\mu^2)\sum_c F^2 \delta_{ac}\delta_{bc}=-\frac{1}{4}F^2 \delta(\mu^2)\delta_{ab}
\end{align*}
となる.


\vskip\baselineskip


カイラリティが同じ場合,積$J^\mu_{La}(x)J^\nu_{Lb}(0)$と$J^\mu_{Ra}(x)J^\nu_{Rb}(0)$は$SU(N)\times SU(N)$のもとでそれぞれ
\begin{align*}
J^\mu_{La}(x)J^\nu_{Lb}(0)&=  -\bar{\psi}(x) \lambda_a \gamma^\mu \frac{1+\gamma_5}{2}\psi(x) \bar{\psi}(x) \lambda_b \gamma^\mu \frac{1+\gamma_5}{2}\psi(x)\\
&\to -\bar{\psi}(x) \left[\mathrm{Ad}(g_L^{-1})\lambda_a \right] \gamma^\mu \frac{1+\gamma_5}{2}\psi(x) \bar{\psi}(x) \left[\mathrm{Ad}(g_L^{-1})\lambda_b \right] \gamma^\mu \frac{1+\gamma_5}{2}\psi(x) \\
J^\mu_{Ra}(x)J^\nu_{Rb}(0)&=  -\bar{\psi}(x) \lambda_a \gamma^\mu \frac{1-\gamma_5}{2}\psi(x) \bar{\psi}(x) \lambda_b \gamma^\mu \frac{1-\gamma_5}{2}\psi(x)\\
&\to -\bar{\psi}(x) \left[\mathrm{Ad}(g_R^{-1})\lambda_a \right] \gamma^\mu \frac{1-\gamma_5}{2}\psi(x) \bar{\psi}(x) \left[\mathrm{Ad}(g_R^{-1})\lambda_b \right] \gamma^\mu \frac{1-\gamma_5}{2}\psi(x) 
\end{align*}
と変換されるから,それぞれ$(A\times A , 1)表現と(1,A\times A)表現$として変換されることが分かる.任意の群は単位元を含むので,これらカレントの積は部分空間として単位演算子を含む.(小林俊行・大島利雄著「リー群と表現論」p13参照)これらの積の演算子積展開に現れる演算子には,単位演算子が含まれる.$\rho_{LaLb},\rho_{RaRb}$はどちらも$\delta_{ab}$に比例するので$\braket{J^\mu_{La}J^\nu_{Lb}},\braket{J^\mu_{Ra}J^\nu_{Rb}}$の演算子積展開も$\delta_{ab}$に比例していなければならず,よって単位演算子の係数には$\delta_{ab}$がある.(この単位演算子は,$SU(N)$の作用するフレーバー添え字に関して単位表現なのであって,$\delta_{ab}$が単位演算子ということではない.)前の説明通り,単位演算子がある場合には一般に和則を満たさないが,スペクトル関数がゼロ;
\begin{align*}
\delta_{ab}[\rho^{(1)}_V(\mu^2)+\rho^{(1)}_A{(\mu^2 )}]=0
\end{align*}
である場合には和則を満たしている.($\delta_{ab}$に比例,すなわち対角成分であるからトレースゼロ)しかし(20.5.1)の左辺は$\alpha=\beta$のとき$|\braket{\VAC|J^\mu_\alpha|N}|>0$があるので,スペクトル関数は正定値で,ゼロとはならない.よってカイラリティが同じときは常にスペクトル関数の和則が満たされない.


\vskip\baselineskip


カイラリティが異なる場合,積$J^\mu_{La}(x)J^\nu_{Rb}(0),J^\mu_{Ra}(x)J^\nu_{Lb}(0)$は$SU(N)\times SU(N)$のもとでそれぞれ
\begin{align*}
J^\mu_{La}(x)J^\nu_{Rb}(0)&=  -\bar{\psi}(x) \lambda_a \gamma^\mu \frac{1+\gamma_5}{2}\psi(x) \bar{\psi}(x) \lambda_b \gamma^\mu \frac{1-\gamma_5}{2}\psi(x)\\
&\to -\bar{\psi}(x) \left[\mathrm{Ad}(g_L^{-1})\lambda_a \right] \gamma^\mu \frac{1+\gamma_5}{2}\psi(x) \bar{\psi}(x) \left[\mathrm{Ad}(g_R^{-1})\lambda_b \right] \gamma^\mu \frac{1-\gamma_5}{2}\psi(x) \\
J^\mu_{Ra}(x)J^\nu_{Lb}(0)&=  -\bar{\psi}(x) \lambda_a \gamma^\mu \frac{1-\gamma_5}{2}\psi(x) \bar{\psi}(x) \lambda_b \gamma^\mu \frac{1+\gamma_5}{2}\psi(x)\\
&\to -\bar{\psi}(x) \left[\mathrm{Ad}(g_R^{-1})\lambda_a \right] \gamma^\mu \frac{1-\gamma_5}{2}\psi(x) \bar{\psi}(x) \left[\mathrm{Ad}(g_L^{-1})\lambda_b \right] \gamma^\mu \frac{1+\gamma_5}{2}\psi(x) 
\end{align*}
と変換されるから,どちらも$(A,A)$表現として変換されることが分かる.よってこれらはまとめて$SU(N)\times SU(N)$二重項を形成する.単位演算子や$F_{a\mu\nu}F_a^{\mu\nu}$のような演算子は$SU(N)\times SU(N)$一重項なので,これらの演算子積展開には現れない.また$\bar{\psi}\psi$のような,微分のないフェルミオン二次形式は
\begin{align*}
\bar{\psi}\psi &=\bar{\psi}_L\psi_R +\bar{\psi}_R\psi_L \\
&\to \left\{ g_L\psi_L \right\}^\dagger \beta g_R \psi_R +\left\{ g_R\psi_R \right\} ^\dagger \beta g_L \psi_L  \\
&=\bar{\psi}_L g^\dagger_L g_R \psi_R +\bar{\psi}_R g^\dagger_R g_L \psi_L
\end{align*}
と変換し,第一項目は$(\bar{N},N)$と変換し第二項目は$(N,\bar{N})$と変換することがわかる.ここで$\bar{N}$は共役表現だ.したがって$(A,A)$の部分空間でないし二重項でもないので,これも演算子積展開には出てこない.微分を一つ含み,ゲージ不変でローレンツ不変なフェルミオン二次形式は,ゲージ共変微分$\gamma^\mu \mathcal{D}_\mu$が$\psi$にかかる$\bar{\psi}\Slash{\mathcal{D}}\psi$だけだがディラック方程式より
\begin{align*}
\bar{\psi}\gamma^\mu \mc{D}_\mu \psi=-m\bar{\psi}\psi
\end{align*}
となり,上の議論よりやはりこの項は現れない.\par
以上の議論より,p24上から$\ell 1 \sim \ell 7$の話と合わせると,カイラリティが異なる場合はスペクトル関数の和則が成り立つことが分かる.
\begin{align*}
&\sum_{ab}c_{ab}\int d\mu^2 \left[ \rho^{(0)}_{LaRb}(\mu^2)+\frac{\rho^{(1)}_{LaRb}(\mu^2)}{\mu^2} \right]=0 \qquad(第1の和則) \\
\Rightarrow\quad &\sum_{ab} c_{ab }\int d\mu^2 \delta_{ab}\left[ -\frac{\delta(\mu^2)F^2}{4} + \frac{\rho^{(1)}_V(\mu^2)-\rho^{(1)}_A(\mu^2)}{4\mu^2} \right]=0 \\
\Rightarrow \quad & \int d\mu^2 [\rho^{(1)}_V(\mu^2)-\rho^{(1)}_A(\mu^2)]/\mu^2=F^2 \\
& \sum_{ab}c_{ab}\int d\mu^2 \rho^{(1)}_{La Rb}(\mu^2) =0 \qquad (第2の和則) \\
\Rightarrow \quad& \int d\mu^2 \left[ \rho^{(1)}_V(\mu^2)-\rho^{(1)}_A(\mu^2) \right]=0
\end{align*}
$N=2$で$\lambda_a$をパウリ行列とするときには,(19.7.2)のように,$F=F_\pi=184\mathrm{MeV}$が成り立つ.


\vskip\baselineskip


$SU(2)\times SU(2)$のスペクトル関数は,$\rho^{(1)}_{V}(\mu^2)$については$\mu=m_\rho=770\mathrm{MeV}$において,また$\rho^{(1)}_A(\mu^2)$については未知の質量$\mu=m_A$において鋭いピークを持つと仮定された.

\begin{figure}[H]
  \centering
\begin{tikzpicture}[scale=1.5]
\draw (2,0)[below]node{$m_\rho,m_A$};
\draw (0,2.8)[left]node{$\rho_{V,A}$};
\draw (4,0)[right]node{$\mu$};
\draw(2.03,0) circle (0.02);
\draw[dashed](2.03,0)--(2.03,2);
\draw[->] (-0.2,0) -- (4,0);
\draw[->] (0,-0.2) -- (0,3);
\draw [smooth,samples=100,domain=0.035:1] plot({\x+2},{\x - ln(\x) -0.8});
\draw [smooth,samples=100,domain=0.035:1] plot({-\x+2.07},{\x - ln(\x) -0.8});
\end{tikzpicture}
\end{figure}
\noindent
つまり
\begin{align*}
\rho^{(1)}_{V}(\mu^2)\simeq g^2_\rho \delta(\mu^2-m^2_\rho) ,\quad \rho^{(1)}_A(\mu^2)\simeq g^2_A(\mu^2)\delta(\mu^2-m^2_A)
\end{align*}
これよりスペクトル関数の和則から
\begin{align*}
&\frac{g^2_\rho}{m^2_\rho}-\frac{g^2_A}{m^2_A}=F_\pi^2 \quad(第1の和則)\\
&g^2_\rho=g^2_A (第2の和則)\\
\Rightarrow \quad& g^2_\rho=F^2_\pi \left( \frac{1}{m^2_\rho}-\frac{1}{m^2_A} \right)^{-1}
\end{align*}
が導ける.1967年に$g^2_\rho=2F^2_\pi m^2_\rho$という式と共に用いられ
\begin{align*}
&2m^2_\rho\left( \frac{1}{m^2_\rho}-\frac{1}{m^2_A} \right)=1 \quad\Rightarrow \quad 2-\frac{2m^2_\rho}{m^2_A}=1 \quad \Rightarrow \quad m^2_A=2m^2_\rho \\
\Rightarrow \quad & m_A=\sqrt{2}m_\rho \qquad \because m_A,m_\rho>0
\end{align*}
という結果が導かれる.

\newpage

\subsection{深非弾性散乱}
深非弾性散乱を学ぶ前に,散乱断面積について復習しておく.\\
・ラザフォード散乱(非相対論的弾性散乱)素粒子物理学の基礎(以下「同著」)p92参照
\begin{align*}
\left[ \frac{d\sigma}{d\Omega} \right]_{\mathrm{Ruth}}&=\left( \frac{Ze^2}{4\pi} \right)^2\frac{4m^2}{q^4} \\
&=\frac{Z^2\alpha^2m^2}{4p^4\sin^4 \frac{\theta}{2}} \qquad \because 同著\text{p92(5.22)}参照
\end{align*}

\begin{figure}[H]
  \centering
\begin{tikzpicture}
      \draw[thick](0,0)node[above=2mm]{$\mathbf{p}$}node[above=6mm]{運動量}node[below=2mm]{点電子}circle[radius=0.2];
      \draw[thick,dashed](-1,0)--(4,0);
      \draw[very thick,-{Stealth[length=3mm]}](0.2,0)--(1,0)to[out=0,in=-135](3.5,1.5);
      \draw[thick](3.7,1.7)node[above right=1mm]{$\mathbf{p}$}circle[radius=0.2];
      \draw[thick](3,0)node[below=2mm]{点電荷}node[below=6mm]{固定粒子}circle[radius=0.2];
      \draw[thick](3.5,0)arc(0:35:2);
      \node(a) at (3.7,0.7){$\theta$};
\end{tikzpicture}
\end{figure}
\noindent
・mott散乱(散乱粒子のスピンを考慮した相対論的ラザフォード散乱)同著p102参照
\begin{align*}
\left[ \frac{d\sigma}{d\Omega} \right]_{\mathrm{mott}}&=\frac{4Z^2\alpha^2}{q^4}E^2(1-\beta^2 \sin^2 \frac{\theta}{2}) \\
&\to \frac{4Z^2\alpha^2}{q^4}E_e^2 \cos^2 \frac{\theta}{2} \qquad \because E_e>>m_e,v=cより\beta=1,|\mathbf{p}|=E_e \\
&=\frac{Z^2\alpha^2}{4E^2_e}\frac{\cos^2\frac{\theta}{2}}{\sin^4 \frac{\theta}{2}}
\end{align*}
ここで$E_e>>m_e,v=cより\beta=1,|\mathbf{p}|=E_e $であり
\begin{align*}
q^2&=(p-p')^2=p^2+p'^2-2p\cdot p' =-m_e^2-m_e^2-2p\cdot p' \\
&=-2(-EE'+|\mathbf{p}| |\mathbf{p'}|\cos \theta) \qquad\because E>>m \\
&=2E^2(1-\cos \theta)=4E^2\sin^2\frac{\theta}{2} \qquad \because E'=E , |\mathbf{p}|=|\mathbf{p}'|=E
\end{align*}
を用いた.電子-電子散乱では$Z=1$なので(20.6.4)式となる(係数若干誤植).

\begin{figure}[H]
  \centering
\begin{tikzpicture}
      \draw[thick](0,0)node[above=2mm]{$p=(\mathbf{p},E)$}node[above=6mm]{4元運動量}node[below=2mm]{点電子}circle[radius=0.2];
      \draw[thick,dashed](-1,0)--(4,0);
      \draw[very thick,-{Stealth[length=3mm]}](0.2,0)--(1,0)to[out=0,in=-135](3.5,1.5);
      \draw[thick](3.7,1.7)node[above right=1mm]{$p=(\mathbf{p},E)$}circle[radius=0.2];
      \draw[thick](3,0)node[below=2mm]{点電荷}node[below=6mm]{固定粒子}circle[radius=0.2];
      \draw[thick](3.5,0)arc(0:35:2);
      \node(a) at (3.7,0.7){$\theta$};
\end{tikzpicture}
\end{figure}
\noindent
・$e-\mu$散乱(スピン1/2同士の弾性散乱)同著p106(5.80)参照
\begin{align*}
\left[ \frac{d\sigma}{d\Omega} \right]_{\mathrm{LAB}}=\left[ \frac{d\sigma}{d\Omega} \right]_{\mathrm{mott}}\frac{p_3}{p_1}\left[ 1-\frac{q^2}{2M^2}\tan^2\left(\frac{\theta}{2}\right) \right]
\end{align*}

\begin{figure}[H]
  \centering
\begin{tikzpicture}[decoration={markings, 
mark= at position -1cm with {\arrow[line width=0.5mm]{Stealth}}}]
\coordinate (b1) at (-2,2){};
\coordinate (b2) at (-2,-2){};
\coordinate (m1) at (0,1){};
\coordinate (a1) at (2,2){};
\coordinate (a2) at (2,-2){};
\coordinate (m2) at (0,-1){};
\draw[thick,postaction={decorate}](b1)node[left]{$e$}node[above right]{$p_1$}--(m1);
\draw[thick,postaction={decorate}](m1)--(a1)node[right]{$e$}node[above left]{$p_3$};

\draw[thick,postaction={decorate}](b2)node[left]{$\mu$}node[below right]{$p_2$}--(m2);
\draw[thick,postaction={decorate}](m2)--(a2)node[right]{$\mu$}node[below left]{$p_4$};

\begin{feynhand}
\propag[photon,thick](m1)--(m2);
\end{feynhand}

\node(q) at (-0.3,0){$q$};
\node(g) at (0.3,0){$\gamma^*$};
    
\end{tikzpicture}
\end{figure}
\noindent
・形状因子(標的粒子が大きさを持つとする弾性散乱)同著p428参照\par
ラザフォード散乱では標的粒子は点状だった. \\
$\Rightarrow$陽子との$e-p$散乱では大きさによる寄与である形状因子$F(q^2)$を持たせる必要がある.このときの散乱断面積は
\begin{align*}
\frac{d\sigma}{d\Omega}=\left[\frac{d\sigma}{d\Omega}\right]_{\mathrm{Ruth}}|F(q^2)|^2
\end{align*}
となる.\\
・Rosenbluthの公式(標的に形状因子を持たせた非弾性散乱)同著p434,ワインバーグ2巻p219,導出についてはランダウ相対論的量子力学2巻p207参照
\begin{align*}
\frac{d\sigma}{d\Omega}=\left[\frac{d\sigma}{d\Omega}\right]_{\mathrm{mott}}\frac{p_3}{p_1}\left[ \frac{G^2_E+\tau G^2_M}{1+\tau}-2\tau G^2_M \tan^2\frac{\theta}{2} \right] 
\end{align*}
ここで$\tau=-q^2/4M^2$だ.ワインバーグ2巻p219の表式は一見これと形が違うように見えるが,括弧の外は同じである(形状因子に関しては定義の違い).それを確かめておく.始状態の,終状態の電子の運動量をそれぞれ$p,p'$,またエネルギーを$p^0=E_e,p'^0=E'_e$とする.また始状態,終状態の標的粒子の運動量をそれぞれ$q,q'$,また質量とエネルギーをそれぞれ$M,Q$とおく.このもとで電子の質量を無視すると
\begin{align*}
k^2&=(p-p')^2=-2m_e^2-2(p\cdot p')=-2(-p^0p'^0+|\mathbf{p}||\mathbf{p'}|\cos\theta) \\
&=4E_eE'_e\sin^2\frac{\theta}{2} =4E_e(E_e+M-Q)\sin^2\frac{\theta}{2} \quad \because p'+q'=p+qの時間成分E'_e+Q=E_e+M \\
&=4E_e^2\sin^2 \frac{\theta}{2}+4E_e(M-Q)\sin^2\frac{\theta}{2}
\end{align*}
また標的粒子に関して
\begin{align*}
k^2&=(q'-q)^2=-2M^2-2(q'\cdot q)=-2M^2+2MQ \\
\Rightarrow \quad & M-Q=-\frac{k^2}{2M}
\end{align*}
を得るので,結局
\begin{align*}
k^2&=4E_e^2\sin^2 \frac{\theta}{2}+4E_e(M-Q)\sin^2\frac{\theta}{2} \\
&=4E_e^2\sin^2 \frac{\theta}{2}-2E_e\frac{k^2}{M}\sin^2\frac{\theta}{2} \\
\Rightarrow \quad & k^2=\frac{4E_e^2\sin^2(\theta/2)}{1+2E_e\sin^2(\theta/2)/M}
\end{align*}
これがまずワインバーグ2巻p220の$k^2$だ.そして$p_3/p_1=E'_e/E_e$は
\begin{align*}
\frac{E'_e}{E_e}&=\frac{E_e+M-Q}{E_e}=1+\frac{M-Q}{E_e}=1-\frac{k^2}{2E_eM} \\
&=1-\frac{1}{2E_e M}\frac{4E_e^2\sin^2(\theta/2)}{1+2E_e\sin^2(\theta/2)/M} =1-\frac{2E_e \sin^2(\theta/2)}{M+2E_e \sin^2 (\theta/2)} \\
&=\frac{M}{M+2E_e \sin^2 (\theta/2)}=\left[1+\frac{2E_e}{M} \sin^2 (\theta/2)\right]^{-1}  \\
\end{align*}
がわかる.したがってワインバーグ2巻p219の公式も他の文献と同じ形であることが分かる.

\vskip\baselineskip

深非弾性散乱とは何か,を先に説明する.クォークが存在することの決定的な証拠をもたらしたのがこの散乱現象である.相対論的なエネルギースケールで電子などのレプトンが入射され陽子や中性子などの核子に衝突すると,核子が「破砕」され,多数の新粒子を生み出す.これらはハドロンだ.これを単純化すると,核子を構成しているクォークが電子によって叩き出され,クォークの閉じ込めにより観測不可能なクォークがハドロン化して散乱する.こうして核子の内部構造を調べることができるため「深部」と言える.これらハドロンは電子の運動量を一部持っていくのでこれは「非弾性」散乱だ.これらの性質により「深非弾性散乱」と呼ばれている.

\begin{figure}[H]
  \centering
\begin{tikzpicture}
\draw[thick](0,0)node[above=2mm]{$k$}node[above=6mm]{運動量}node[below=2mm]{電子}circle[radius=0.2];
\draw[thick,dashed](-1,0)--(4,0);
\draw[very thick,-{Stealth[length=3mm]}](0.2,0)--(1,0)to[out=0,in=-135](3.5,1.5);
\draw[thick](3.7,1.7)node[above right=1mm]{$k'$}circle[radius=0.2];
\draw[thick](3,0)node[below=4mm]{核子$N$}circle[radius=0.4];
\draw[thick](3.5,0)arc(0:35:2);
\node(a) at (3.7,0.7){$\theta$};

\draw[very thick,-{Stealth[length=3mm]}](3,0)--(5,-0.5);
\draw(5,-0.5)node[right]{$\pi$};

\draw[very thick,-{Stealth[length=3mm]}](3,0)--(5,-0.9);
\draw(5,-0.9)node[right]{$\pi \Biggr\}H$};

\draw[very thick,-{Stealth[length=3mm]}](3,0)--(5,-1.3);
\draw(5,-1.3)node[right]{$p$};

\draw(6,1)node{$\Longrightarrow$};
\draw(6,1.5)node{};
\end{tikzpicture}
\begin{tikzpicture}[decoration={markings, 
mark= at position -1cm with {\arrow[line width=0.5mm]{Stealth}}}]
\coordinate (b1) at (-2,2){};
\coordinate (b2) at (-2,-1){};
\coordinate (m1) at (0,1){};
\coordinate (a1) at (2,2){};
\coordinate (a2) at (2.5,-0.5){};
\coordinate (m2) at (1,0){};
\coordinate (a3) at (2,-2.4){};
\coordinate (a4) at (2,-1.6){};
\draw[thick,postaction={decorate}](b1)node[left]{$e$}node[above right]{$k$}--(m1);
\draw[thick,postaction={decorate}](m1)--(a1)node[right]{$e$}node[above left]{$k'$};

\draw[thick,postaction={decorate}](b2)node[above=2mm]{$N$}--(m2);
\draw[thick,postaction={decorate}](m2)--(a2)node[right]{$H$};
\draw(-2,-1.5)circle(0.8);
\draw(b2)circle(0.2);
\draw(-1.7,-1.5)circle(0.2);
\draw(-2.2,-1.9)circle(0.2);
\draw[thick,postaction={decorate}](-1.7,-1.5)--(2.5,-1.5);
\draw[thick,postaction={decorate}](-2.2,-1.9)--(2.5,-1.9);
\draw(2.5,-1.5)node[right]{$H$};
\draw(2.5,-1.9)node[right]{$H$};

\draw(m1)node[above]{$x$};
\draw(m2)node[below]{$y$};

\begin{feynhand}
\propag[photon,thick](m1)--(m2);
\end{feynhand}

\node(q) at (0,0.5){$q$};
\node(g) at (0.9,0.6){$\gamma*$};

\end{tikzpicture}
\end{figure}

さて,今までの散乱断面積の表式から類推すれば,(20.6.3)の形は自然であると分かる.また$W_1,W_2$は形状因子に対応する関数(構造関数と呼ぶ)であるということも類推できる.この感覚のもと,以下で(20.6.3)を頑張って導出することにしよう.せっかくなので,復習も兼ねて散乱断面積の初歩的なところから導出しようと思う.\par
相互作用ハミルトニアンは(相互作用ラグランジアンと符号だけ違うので)
\begin{align*}
&\mc{L}=-\bar[{\psi}\Slash{\mc{D}}+m]\psi =-\bar{\psi}[\Slash{\partial}+m]\psi-ieA\bar{\psi}\gamma^\mu \psi \\
&\mc{H}=ieA_\mu\bar{\psi} \gamma^\mu \psi \equiv ieA_\mu J^\mu
\end{align*}
で与えられるので$S$行列は,$\mc{J}^\mu$が(20.6.1)で出てくる「電荷$e$で割った(またフェルミオンの外線$u,\bar{u}$に付随する$1/(2\pi)^{3/2}$の因子2つで割った)カレント」とすれば,
\begin{align*}
S&=\frac{(-i)^2}{2!} \int d^4x \int d^4y \bra{e(k'),H}T\left\{ \mc{H}(x)\mc{H}(y) \right\} \ket{e(k),N} \\
&=\int d^4x \int d^4y \, e^2T\{ \bra{H} J^\mu (x)\ket{N}[\bra{0} A_\mu(x)A_\nu(y) \ket{0}] \bra{e(k')}J^\nu(y) \ket{e(k)} \}
\end{align*}
ここで$e(k') \leftrightarrow N,e(k)\leftrightarrow H$の寄与もあるが,$ x\leftrightarrow y$で対称なので,1/2!で相殺されている.真中の項はプロパゲータ(6.2.31)であるから
\begin{align*}
&=\int d^4x d^4y \, e^2\bra{H} J^\mu (x)\ket{N}[-i\Delta_{\mu\nu}(x-y)]\bra{e(k')}J^\nu(y)\ket{e(k)} \\
&=\int d^4 x d^4y \, e^2\bra{H}J^\mu (0)\ket{N}e^{i(p_N-p_H)x}\left[\frac{-i}{(2\pi)^4}\int d^4 q \frac{\eta_{\mu\nu}}{q^2-i\epsilon}e^{iq(x-y)} \right]e^{i(k-k')y}\bra{e(k')}J^\nu(0)\ket{e(k)} \\
&=\int d^4q \int d^4xd^4y \, e^2\bra{H}J^\mu (0)\ket{N}e^{i(p_N-p_H+q)x}\left[\frac{-i}{(2\pi)^4} \frac{\eta_{\mu\nu}}{q^2-i\epsilon} \right]e^{i(k-k'-q)y}\bra{e(k')}J^\nu(0)\ket{e(k)} \\
&= \int d^4q \, e^2\bra{H}J^\mu (0)\ket{N}(2\pi)^4\delta^4(p_N-p_H+q)\left[\frac{-i}{(2\pi)^4} \frac{\eta_{\mu\nu}}{q^2-i\epsilon} \right](2\pi)^4\delta^4(k-k'-q)\bra{e(k')}J^\nu(0)\ket{e(k)} \\
&=e^2\bra{H}J^\mu (0)\ket{N}(2\pi)^4\delta^4(p_N-p_H+k-k')\left[\frac{-i}{(2\pi)^4} \frac{\eta_{\mu\nu}}{q^2} \right](2\pi)^4\bra{e(k')}J^\nu(0)\ket{e(k)} \\
&=-(2\pi)^4 ie^2 \delta^4(p_N-p_H+k-k') \bra{H}J^\mu (0)\ket{N}\frac{1}{q^2}\bra{e(k')}J_\mu(0)\ket{e(k)} \\
&=-(2\pi)^4 i e^2 \delta^4(p_N-p_H+k-k') \frac{1}{(2\pi)^3}\bra{H}\mc{J}^\mu (0)\ket{N}\frac{1}{q^2} \frac{1}{(2\pi)^3} \bra{e(k')}\mc{J}_\mu(0)\ket{e(k)} \\
&=-(2\pi)^4 i \delta^4(p_N-p_H+k-k') e^2 \frac{1}{(2\pi)^6}\bra{H}\mc{J}^\mu (0)\ket{N}\frac{1}{q^2} \bar{u}(k')\gamma_\mu u(k)
\end{align*}
で与えられ,ファインマン振幅$M$は一般に(3.3.2)で(散乱が起きるとしているので)
\begin{align*}
&S=-2\pi i \delta^4(p_N-p_H+k-k') M \\
\Rightarrow \quad & M=\frac{e^2}{(2\pi)^3q^2}\bra{H}\mc{J}^\mu (0)\ket{N} \bar{u}(k')\gamma_\mu u(k)
\end{align*}
2体散乱の微分散乱断面積は(3.4.15)で与えられ
\begin{align*}
d\sigma =(2\pi)^4 u^{-1}|M|^2\delta^4(p_N-p_H+k-k') d^3k'
\end{align*}
となる.(ハドロン$H$については測定しないので$d\beta$は$d^3k'$ひとつだけ)核子$N$は静止していて,電子の質量については無視するので,相対速度(3.4.17)は$u=1$となる.したがって
\begin{align*}
d\sigma=\frac{e^4}{(2\pi)^2q^4}\biggl[\bra{H}\mc{J}^\mu (0)\ket{N}\bra{H}\mc{J}^\nu (0)\ket{N}^{*}\biggr]\biggl[\bar{u}(k')\gamma_\mu u(k)u^\dagger(k)\gamma^\dagger_\nu \beta u(k')\biggr]\delta^4(p_N-p_H+k-k')d^3k'
\end{align*}
ハドロンについては測定しないので$H$について和をとり,また始状態と終状態の$z$成分については測定しないのが通常なので,電子のスピン$\sigma,\sigma'$と核子のスピン$\sigma_N$については平均をとる.よって
\begin{align*}
d\bar\sigma &\equiv \frac{1}{4}\sum_{\sigma \sigma'}\sum_{\sigma_N,H}d\sigma \\
&=\frac{e^4}{(2\pi)^2q^4}\biggl[\frac{1}{2}\sum_{\sigma_N,H} \bra{H}\mc{J}^\mu (0)\ket{N}\bra{H}\mc{J}^\nu (0)\ket{N}^{*}\biggr] \\
&\quad \times\biggl[\frac{1}{2}\sum_{\sigma \sigma'}\bar{u}(k')\gamma_\mu u(k)u^\dagger(k)\gamma^\dagger_\nu \beta u(k') \biggr]\delta^4(p_N-p_H+k-k')d^3k'  \\
&=\frac{e^4}{(2\pi)^2q^4}W^{\mu\nu}L_{\mu\nu}d^3k'=\frac{e^4}{(2\pi)^2q^4}W^{\mu\nu}L_{\mu\nu}(k'^0)^2dk'^0d\Omega \quad\because |\mathbf{k}|=k^0
\end{align*}
が計算すべき断面積だ.ここで$d\Omega=\sin \theta d\theta d \phi$は終状態の電子が散乱される立体角であり,$\theta$は$\mathbf{k}$と$\mathbf{k'}$の間の角度,そして
\begin{align*}
&W^{\mu\nu}=\frac{m_N}{p^0_N}W^{\mu\nu} \equiv \frac{1}{2}\sum_{\sigma_N,H} \delta^4(p_N-p_H+k-k') \bra{H}\mc{J}^\mu (0)\ket{N}\bra{H}\mc{J}^\nu (0)\ket{N}^{*} \\
&L_{\mu\nu} \equiv \frac{1}{2}\sum_{\sigma \sigma'}\bar{u}(k')\gamma_\mu u(k)u^\dagger(k)\gamma^\dagger_\nu \beta u(k')=\frac{1}{2}\sum_{\sigma \sigma'}\bar{u}(k')\gamma_\mu u(k)\bar u(k)(-\gamma_\nu)  u(k')
\end{align*}
である.(8.7.20)を用いると
\begin{align*}
&L_{\mu\nu}=\frac{1}{2}\mathrm{Tr} \left\{ \gamma_\mu \left( \frac{-i\Slash{k}+m_e}{2k^0} \right)(-\gamma_\nu) \left(\frac{-i\Slash{k}'+m_e}{2k'^0}\right)  \right\} \\
&=\frac{1}{8k^0k'^0}\left[  \mathrm{Tr}\left\{ \gamma_\mu \gamma_\rho \gamma_\nu \gamma_\sigma \right\}k^\rho k'^\sigma -m_e^2 \mathrm{Tr}\left\{ \gamma_\mu \gamma_\nu \right\} \right]  \qquad \because 奇数個の\gamma 行列のトレースはゼロ \\
&=\frac{1}{8k^0k'^0}\left[  \left\{ 4\eta_{\mu\rho}\eta_{\nu\sigma}+\eta_{\mu\sigma}\eta_{\rho\nu}-\eta_{\mu\nu}\eta_{\rho\sigma} \right\}k^\rho k'^\sigma -4m_e^2\eta_{\mu\nu}  \right] \\
&=\frac{1}{2k^0k'^0}\left[  \left\{ k_\mu k'_\nu+k'_\mu k_\nu-(k\cdot k')\eta_{\mu\nu} \right\} -m_e^2\eta_{\mu\nu}  \right]  \\
&=\frac{1}{2k^0k'^0}\left[ k_\mu k'_\nu+k'_\mu k_\nu+\left\{-(k\cdot k') +\frac{k^2}{2}+ \frac{k'^2}{2}\right\}\eta_{\mu\nu} \right] \\
&=\frac{1}{2k^0k'^0}\left[ k_\mu k'_\nu+k'_\mu k_\nu+\frac{q^2}{2} \eta_{\mu\nu} \right] 
\end{align*}
となる.さて,カレントの保存則からも分かる通り
\begin{align*}
&(k-k')^\mu\bra{e(k')}\mc{J}_\mu(0)\ket{e(k)}=q^\mu\bra{e(k')}\mc{J}_\mu(0)\ket{e(k)}=0 \\
&(p_H-p_N)^\mu\bra{H}\mc{J}_\mu(0)\ket{N}=q^\mu\bra{H}\mc{J}_\mu(0)\ket{N}=0
\end{align*}
が成り立っている.したがって
\begin{align*}
&q^\mu L_{\mu\nu}=q^\nu L_{\mu\nu}=0 \\
&q_\mu W^{\mu\nu}=q_\nu W^{\mu\nu}=0
\end{align*}
という条件がつく.前者の条件は自明だ.なぜなら$q=k-k',k^2=k'^2$に留意すれば
\begin{align*}
q^\mu L_{\mu\nu}&=\frac{1}{2k^0 k'^0}\left[ (q\cdot k) k'_\nu+(q \cdot k') k_\nu+\frac{q^2}{2} q_\nu \right] \\
&=\frac{1}{2k^0 k'^0}\biggl[ (k)^2 k'_\nu-(k'\cdot k)k'_\nu +(k \cdot k') k_\nu-(k')^2 k_\nu  \\
&\qquad \qquad + \frac{(k)^2}{2}k_\nu +\frac{(k')^2}{2}k_\nu-(k\cdot k')k_\nu -\frac{(k)^2}{2}k'_\nu- \frac{(k')^2}{2}k'_\nu +  (k\cdot k')k_\nu \biggr] \\
&=0
\end{align*}
となるからだ.後者の条件について考えてみる.$J^\mu$はエルミートなので$W^{\mu\nu}=W^{\nu\mu*}$であり,定義より正低値行列だ.実際
\begin{align*}
v_\mu W^{\mu\nu}v_\nu=\frac{1}{2}\sum_{\sigma_N,H}\delta^4(p_N-p_H+k-k')|\bra{H}J^\mu(0) v_\mu\ket{N}|^2>0
\end{align*}
である.また核子のスピンについては平均をとってしまっているので,$W^{\mu\nu}$は$qとp$のみの関数であり,しかもローレンツテンソルである.したがって$\eta^{\mu\nu},q^\mu q^\nu ,p^\mu p^\nu , p^\mu q^\nu , q^\mu p^\nu ,i\epsilon^{\mu\nu\rho\sigma}p_\rho q_\sigma$で展開できる.($i\epsilon^{\mu\nu\rho\sigma}p_\rho p_\sigma,i\epsilon^{\mu\nu\rho\sigma}q_\rho q_\sigma$は反対称性により自動的にゼロ)ここで条件$q_\mu W^{\mu\nu}=q_\nu W^{\mu\nu}=0$を満たす量は
\begin{align*}
\left(\frac{q^\mu q^\nu}{q^2}-\eta_{\mu\nu}\right),\quad \left(p^\mu -\frac{(p\cdot q)}{q^2}q^\mu \right)\left(p^\nu -\frac{(p\cdot q)}{q^2}q^\nu \right), \quad i\epsilon^{\mu\nu\rho\sigma}p_\rho q_\sigma
\end{align*}
である.前者二つはグラムシュミットの直交化法(懐かしい!)を思い出せば自然に導かれる.実際にこれらに$q_\mu,q_\nu$を縮約してみればゼロになることを確認してみるといい.さらに電磁相互作用では,パリティが保存するので三つ目の項は存在しない(実際に$L_{\mu\nu}$にかければ反対称性より自動的にゼロになることはすぐ確認できる)したがって自然に
\begin{align*}
W^{\mu\nu}(q,p)=-\left(\frac{q^\mu q^\nu}{q^2}-\eta^{\mu\nu}\right)W_1(\nu,q^2)+\frac{1}{m_N^2}\left(p^\mu -\frac{(p\cdot q)}{q^2}q^\mu \right)\left(p^\nu -\frac{(p\cdot q)}{q^2}q^\nu \right)W_2(\nu,q^2)
\end{align*}
という形に導かれる.ここで$W^{\mu\nu}=W^{\nu\mu*}$より$W_1,W_2$は実関数であり$W^{\mu\nu}$は正定値
\begin{align*}
0<v_\mu W^{\mu\nu} v_\nu =-\left(\frac{(v\cdot q)^2}{q^2}-v^2\right)W_1+\frac{1}{m_N^2}\left|(v\cdot p) -\frac{(p\cdot q)}{q^2}(v\cdot q) \right|^2W_2
\end{align*}
であるから,$W_2$の係数は自明に正,また$W_1$の係数はコーシー・シュワルツの不等式$(v\cdot q)^2\leq v^2 q^2$を用いれば正であることから,$W_1,W_2$は実かつ正の関数であることが要請される.これらはスカラー関数であるから,$p,q$から作られる独立な二つのスカラー量$q^2,\nu \equiv -q\cdot p/m_N(=q^0)$のみに依存する($p^2$は$-m^2_N$であるから変数ではない).第二項目は次元を合わせるために係数$1/m_N^2$をかけてある.条件$q^\mu L_{\mu\nu}=q^\nu L_{\mu\nu}=0$と,電子の質量を無視することから
\begin{align*}
q^2=(k-k')^2&=-2m_e^ 2 - 2(k\cdot k') \\
&=-2(k\cdot k') =2(k^0k'^0-|\mathbf{k}||\mathbf{k'}|\cos\theta)=2 k^0k'^0(1-\cos\theta) \quad \because |\mathbf{k}|=k^0 \\
&=4k^0k'^0 \sin^2\frac{\theta}{2} 
\end{align*}
を用いると
\begin{align*}
W^{\mu\nu}L_{\mu\nu}&=\left[-\left(\frac{q^\mu q^\nu}{q^2}-\eta^{\mu\nu}\right)W_1+\frac{1}{m_N^2}\left(p^\mu -\frac{(p\cdot q)}{q^2}q^\mu \right)\left(p^\nu -\frac{(p\cdot q)}{q^2}q^\nu \right)W_2\right]L_{\mu\nu} \\
&=\eta^{\mu\nu}L_{\mu\nu}W_1 + \frac{1}{m_N^2}p^\mu p^\nu L_{\mu\nu}W_2 \\
&=\eta^{\mu\nu}\frac{1}{2k^0k'^0}\left[ k_\mu k'_\nu+k'_\mu k_\nu+\frac{q^2}{2} \eta_{\mu\nu} \right]W_1 +\frac{1}{m_N^2}p^\mu p^\nu\frac{1}{2k^0k'^0}\left[ k_\mu k'_\nu+k'_\mu k_\nu+\frac{q^2}{2} \eta_{\mu\nu} \right]  \\
&=\frac{1}{2k^0k'^0}\left[ 2(k\cdot k')+2q^2 \right] W_1 +\frac{1}{2k^0k'^0m_N^2}\left[ 2(p\cdot k)(p\cdot k')+\frac{q^2 p^2}{2} \right]  \quad \because \eta^{\mu\nu}\eta_{\mu\nu}=4 \\
&=\frac{1}{2k^0k'^0}\left[ -2(k\cdot k') \right] W_1 +\frac{1}{2k^0k'^0m_N^2}\left[ 2m_N^2 k^0k'^0 -m_N^2\frac{q^2}{2} \right]  \quad \because p=(0,0,0,m_N)\\
&=2\sin^2\frac{\theta}{2} W_1 +\frac{1}{2k^0k'^0}\left[ 2 k^0k'^0 -\frac{q^2}{2} \right] =2\sin^2\frac{\theta}{2} W_1 +\frac{1}{2k^0k'^0}\left[ 2 k^0k'^0 -2k^0k'^0\sin^2\frac{\theta}{2} \right] \\
&=2\sin^2\frac{\theta}{2} W_1 + \cos^2 \frac{\theta}{2}W_2=\cos^2 \frac{\theta}{2}\left[W_2 +2\tan^2 \frac{\theta}{2}W_1 \right]
\end{align*}
であるから,求めるべき断面積は
\begin{align*}
d\bar{\sigma}&=\frac{e^4}{(2\pi)^2q^4}W^{\mu\nu}L_{\mu\nu}(k'^0)^2dk'^0d\Omega \\
&=\frac{4\alpha^2}{q^4}E_e'^2\cos^2 \frac{\theta}{2}\left[W_2 +2\tan^2 \frac{\theta}{2}W_1 \right]dE_e' d\Omega \\
&=\frac{4\alpha^2}{q^4}E_e'^2\cos^2 \frac{\theta}{2}\left[W_2 +2\tan^2 \frac{\theta}{2}W_1 \right]d\nu d\Omega \quad \because \nu=q^0=k^0-k'^0
\end{align*}
(最後はマイナスがつきそうだが,積分範囲が$E_e':[-\infty\sim\infty]\to \nu:[\infty\sim-\infty]$となるので,それをマイナスで戻してやると打ち消し合うのでマイナスは出てこない)mott散乱の断面積の表式を思い出せば
\begin{align*}
\frac{d\bar{\sigma}}{d\nu d\Omega}&=\frac{4\alpha^2}{q^4}E_e'^2\cos^2 \frac{\theta}{2}\left[W_2 +2W_1\tan^2 \frac{\theta}{2} \right] \\
&=\left[ \frac{d\sigma}{d\Omega}\right]_{\mathrm{mott}}\left[W_2 +2W_1\tan^2 \frac{\theta}{2} \right] 
\end{align*}
を得る.これが求めたかったものだ!

\vskip\baselineskip

$-p^2_H=-(p+q)^2=-q^2-2p\cdot q -p^2 =-q^2 +2m_N \nu +m^2_N$が固定された値のとき,微分断面積は$q^{-4}$の因子により$q^2\to \infty$のとき非常に速く減少すると思われる.\\
$\Rightarrow$しかし$\nu W_2(\nu,q^2)$は$\omega\equiv 2m_N\nu/q^2$
\begin{align*}
-p^2_H&=-q^2 +\frac{2m_N \nu}{q^2}q^2 +m^2_N \overset{q^2\to \infty}{\longrightarrow} (\omega-1)q^2 >0 \quad \because -p^2_H>0\\
\Rightarrow \quad &\omega>1
\end{align*}
が固定された値のときに$q^2$に対して定数だと分かった(らしい).


\vskip\baselineskip


$\nu W_2(\nu,q^2)$と$W_1(\nu,q^2)$は,$\nu$と$q^2$が共に無限大にいくときに,$q^2$と$\nu$の2変数関数ではなく比$\omega=2m_N\nu/q^2$に依存する,とブヨルケンは仮定した.
\begin{align*}
\nu W_2(\nu,q^2)\to F_2(\omega),\quad W_1(\nu,q^2)\to F_1(\omega)
\end{align*}
(素粒子物理学の基礎p440と同様に$m_NW_1\to F_1$とした方が後の表式は綺麗になるが,ここはワインバーグに合わせる)少し後にファインマンがより直感的な説明を与えた.非常に相対論的な核子の深非弾性散乱では核子はあたかも各種のパートンからできているように振る舞うとした.(これはクォークとグルーオンだ!)このパートンには$i$という添え字を与え,核子の運動量を$\mathbf{p}$として,それぞれのパートンの運動量が$x\mathbf{p}$と$(x+dx)\mathbf{p}$との間にある確率は$\mc{F}_i(x)dx$で与えられるとする(分布関数の話を復習しておくと良い.すなわちパートンの運動量$x\mathbf{p}$が従う分布関数が$\mc{F}_i(x)$だ.それぞれのパートンは種類が同じとは限らないので,それぞれのパートン$i$の運動量はそれぞれ分布関数$\mc{F}_i(x)$に従うとおく).\par
各パートンの運動量は$0(x=0)\sim \mathbf{p}(x=1)$の中には必ずあるので,各$i$について
\begin{align*}
\int^1_0 \mc{F}_i(x)dx=1
\end{align*}
が成り立つ.また核子全体の運動量が$\mathbf{p}$なので,それぞれの核子の運動量期待値を全て足し合わせたら$\mathbf{p}$となることが要請される.
\begin{align*}
&\sum_i \int^1_0 (x\mathbf{p})\mc{F}_i(x)dx=\mathbf{p} \\
\Rightarrow \quad &\int^1_0 \sum_i \mc{F}_i(x)xdx=1
\end{align*}
が成り立つ.\par
4元運動量$xp$のパートンから電子弾性散乱について
\begin{align*}
&-(xp)^2=x^2m_N^2=-(q+xp)^2 \\
&=-q^2-2x(p\cdot q)-x^2p^2 = -q^2 +2m_N \nu x+x^2m_N^2 \quad \because \nu=-\frac{(p\cdot q)}{m_N}\\
\Rightarrow \quad & \nu=\frac{q^2}{2m_N x}
\end{align*}

\begin{figure}[H]
  \centering
\begin{tikzpicture}[decoration={markings, 
mark= at position -1cm with {\arrow[line width=0.5mm]{Stealth}}}]
\coordinate (b1) at (-2,2){};
\coordinate (b2) at (-2,-1){};
\coordinate (m1) at (0,1){};
\coordinate (a1) at (2,2){};
\coordinate (a2) at (2,-1){};
\coordinate (m2) at (0,0){};
\coordinate (a3) at (2,-2.4){};
\coordinate (a4) at (2,-1.6){};
\draw[thick,postaction={decorate}](b1)node[left]{$e$}node[above right]{$k$}--(m1);
\draw[thick,postaction={decorate}](m1)--(a1)node[right]{$e$}node[above left]{$k'$};

\draw[thick,postaction={decorate}](b2)node[above=2mm]{$N$}--(m2);
\draw[thick,postaction={decorate}](m2)--(a2)node[above=2mm]{$q+xp$};
\draw(-2,-1.5)circle(0.8);
\draw(b2)circle(0.2);
\draw(-1.7,-1.5)circle(0.2);
\draw(-2.2,-1.9)circle(0.2);
\draw(-0.5,-0.5)node[below]{$xp$};
;\draw(a2)circle(0.2);

\begin{feynhand}
\propag[photon,thick](m1)--(m2);
\end{feynhand}

\node(q) at (-0.3,0.5){$q$};
\node(g) at (0.3,0.5){$\gamma*$};

\end{tikzpicture}
\end{figure}

この散乱はパートンひとつひとつについて$e-\mu$散乱と(電荷を除いて)同じだ.つまり(20.6.3)の断面積を$\nu$で積分したら$e-\mu
$散乱断面積であるから,各$i$について($M^2\to x^2m_N^2$として)
\begin{align*}
\left[\frac{d\sigma}{d\Omega}\right]_i=\left[ \frac{d\sigma}{d\Omega} \right]_{\mathrm{mott}} Q_i^2 \frac{E'_e}{E_e}\left(1+\frac{q^2}{2m_N^2x^2} \tan^2 \frac{\theta}{2} \right)
\end{align*}
ここで
\begin{align*}
\frac{q^2}{2m_N x}=\frac{2E_e E'_e}{m_N x}\sin^2 \frac{\theta}{2}
\end{align*}
であることに留意すると,($q^2$は$\nu$の関数なので以下の積分は単純には行うことはできない)
\begin{align*}
\int^\infty_{-\infty}\delta \left(\nu -\underset{\nu の関数}{\uwave{\frac{q^2}{2m_N x} }}\right)d\nu&=-\int^{-\infty}_{\infty}\delta \left(E_e -E'_e -\frac{2E_e E'_e}{m_N x}\sin^2 \frac{\theta}{2} \right) dE'_e \\
&=\int^\infty_{-\infty}\delta \left( -E'_e\left[ 1+\frac{2E_e}{m_N x}\sin^2 \frac{\theta}{2} \right]+E_e \right)dE'_e \\
&=\int^\infty_{-\infty}\delta(-E'_e \alpha +E_e)dE'_e \\
&=\int^\infty_{-\infty}\frac{1}{\alpha}\delta\left(-E'_e +\frac{E_e}{\alpha}\right)dE'_e \\
&=\frac{1}{\alpha} \quad \left(E'_e=\frac{E_e}{\alpha}\right) \\
&=\frac{E'_e}{E_e}
\end{align*}
よって
\begin{align*}
\left[ \frac{d^2\sigma}{d\Omega d\nu} \right]_i =\left[ \frac{d\sigma}{d\Omega} \right]_{\mathrm{mott}}Q^2_i\left( 1+\frac{q^2}{2m_N^2x^2}\tan^2 \frac{\theta}{2} \right)\delta\left(\nu-\frac{q^2}{2m_N x}\right)
\end{align*}
(実際に$\nu$で積分すれば$e-\mu$散乱断面積と形式が一致することが分かる)各$i$についての散乱断面積が分かったので,これらをパートン分布関数とかけて$x$について積分し,パートン全てについて足し合わせることで核子全体についての散乱断面積となる.
\begin{align*}
\frac{d^2\sigma}{d\Omega d\nu}&=\sum_i \int^1_0\left[ \frac{d^2\sigma}{d\Omega d\nu} \right]_i \mc{F}_i(x)dx \\
&=\left[\frac{d\sigma}{d\Omega}\right]_{\mathrm{mott}}\sum_i Q^2_i \int^1_0 \mc{F}_i(x)\left( 1+\frac{q^2}{2m_N^2x^2}\tan^2 \frac{\theta}{2} \right)\delta\left(\nu-\frac{q^2}{2m_N x}\right) dx
\end{align*}
これと(20.6.3)と比較すると
\begin{align*}
W_2(\nu,q^2)&=\sum_i Q^2_i \int^1_0 dx \mc{F}_i (x)\delta\left(\nu-\frac{q^2}{2m_N x}\right) \\
&=\sum_i Q^2_i \int^1_0 dx \mc{F}_i(x)\frac{1}{\nu}\delta\left(1-\frac{q^2}{2m_N\nu x}\right) =\sum_i Q^2_i \int^1_0 dx\mc{F}_i(x)\frac{1}{\nu}\delta\left( 1-\frac{1}{\omega x} \right) \\
&=\sum_i Q^2_i \int^1_0 dx\mc{F}_i(x)\frac{1}{\nu\omega}\delta\left(x-\frac{1}{\omega}\right) \quad \because デルタ関数の公式 \, \delta(f(x))=\frac{1}{|f'(x_0)|}\delta(x-x_0) \\
&=\frac{1}{\nu\omega}\sum_i Q^2_i \mc{F}_i\left(\frac{1}{\omega}\right) \\
2W_1(\nu,q^2)&=\sum_i Q_i^2 \int^1_0 dx \mc{F}_i(x)\frac{q^2}{2m_N^2 x^2}\delta\left( \nu-\frac{q^2}{2m_Nx} \right) \\
&=\sum_i Q^2_i \int^1_0 dx \mc{F}_i(x)\frac{1}{\omega\nu}\frac{q^2}{2m_N^2 x^2}\delta\left( x-\frac{1}{\omega} \right) \\
&=\sum_i Q^2_i \mc{F}_i\left(\frac{1}{\omega}\right)\frac{1}{\omega\nu}\frac{q^2\omega^2}{2m_N^2}=\sum_i Q^2 \mc{F}_i\left(\frac{1}{\omega}\right)\frac{1}{\omega\nu}\frac{q^2}{2m_N^2}\frac{2m_N\nu}{q^2}\omega \\
&=\sum_i Q^2_i \mc{F}_i\left(\frac{1}{\omega}\right)\frac{1}{m_N}=\frac{\omega\nu}{m_N}W_2(\nu,q^2) \\
\Rightarrow \quad &W_1(\nu,q^2)=\frac{\omega\nu}{2m_N}W_2(\nu,q^2)
\end{align*}
これは
\begin{align*}
&F_2(\omega)=\frac{1}{\omega}\sum_iQ^2_i \mc{F}_i\left(\frac{1}{\omega}\right) \\
&F_1(\omega)=\frac{\omega}{2m_N}F_2(\omega)
\end{align*}
としてブヨルケンのスケーリング則(20.6.5)に一致する.

\vskip\baselineskip

陽子が二つのアップクォーク$u$(電荷+3/2)と一つのダウンクォーク$d$(電荷-1/3),中性子が一つの$u$(+3/2)と二つの$d$(-1/3),およびそれぞれさらに任意個数の中性パートンからなるという仮説に基づくと(20.6.11)と(20.6.6)は以下の$W_2$についての和則を与える.
\begin{align*}
1&=\int^1_0 \mc{F}_i(x)dx=\int^1_\infty \mc{F}_i\left(\frac{1}{\omega}\right)\left(-\frac{1}{\omega^2}d\omega\right) \quad \because x=\frac{1}{\omega},dx=-\frac{1}{\omega^2}d\omega \\
&=\int^\infty_1 \mc{F}_i\left(\frac{1}{\omega}\right)\frac{1}{\omega^2}d\omega \\
\Rightarrow \quad & \int^\infty_1 F_2(\omega)\frac{d\omega}{\omega}=\int^\infty_1 \frac{1}{\omega^2}\sum_i Q^2_i \mc{F}_i\left(\frac{1}{\omega}\right)d\omega =\sum_iQ^2_i \\
&\qquad\qquad\qquad =\left\{
\begin{array}{ll}
(2/3)^2+(-1/3)^2+( -1/3)^2=2/3 &(中性子nのとき) \\
(2/3)^2 +(-2/3)^2+( -1/3)^2=1 &(陽子pのとき)
\end{array}
\right.
\end{align*}
核子の運動量は3つのクォークが等しく担っており中性パートンは運動量ゼロと仮定すると,クォークの運動量期待値はそれぞれ$\mathbf{p}/3$となる.
\begin{align*}
\int^1_0 \mc{F}_i(x)(x\mathbf{p})dx=\frac{\mathbf{p}}{3}
\end{align*}
したがって
\begin{align*}
&\int^1_0\mc{F}_i(x)xdx=\int^1_\infty \mc{F}_i\left(\frac{1}{\omega}\right)\frac{1}{\omega}\left(-\frac{1}{\omega^2}d\omega \right)=\int^\infty_1 \mc{F}_i\left(\frac{1}{\omega}\right)\frac{d\omega}{\omega^3}=\frac{1}{3} \\
\Rightarrow \quad &\int^\infty_1 F_2(\omega)\frac{d\omega}{\omega^2}=\int^\infty_1 \frac{1}{\omega^3}\sum_i Q^2_i \mc{F}_i \left(\frac{1}{\omega}\right)d\omega=\frac{1}{3}\sum_iQ^2_i \\
&\qquad\qquad =\left\{
\begin{array}{ll}
1/3 & (陽子pのとき) \\
2/9 &(中性子nのとき)
\end{array}
\right.
\end{align*}
これは測定値とずれるらしい.つまり,核子の運動量の多くは中性パートン(これはグルーオンに対応する)によって担われていることを示している.

\vskip\baselineskip

以上の現象論は,特定の場の理論(QED.QCD等)に依存していない.基礎の場の理論を深非弾性散乱に用いる方法を最終的に与えるのはOPEだ.特にOPEは(20.6.5)のスケール仮定を説明するのに「漸近的自由な場の理論」が必要ということを明らかにした.この「漸近的自由な場の理論」を我々は知っている.すなわち,QCDだ!\par
OPEを用いるために,まず(20.6.1)をフーリエ変換する.並進不変性とハドロン状態の完全性を用いると
\begin{align*}
\left(\frac{m_N}{p^0_N}\right)W^{\mu\nu}(q,p)&=\frac{1}{2}\sum_{\sigma_N}\sum_H \delta^4(p_H-p-q)\bra{H}J^\mu (0)\ket{N}\bra{H}J^\nu (0)\ket{N}^* \\
&=\frac{1}{2(2\pi)^4}\sum_{\sigma_N}\sum_H \int d^4z \, e^{i(p_H-p-q)z}\bra{N}J^\nu (0)\ket{H}\bra{H}J^\mu (0)\ket{N} \\
&=\frac{1}{2(2\pi)^4}\sum_{\sigma_N}\sum_H \int d^4z \, e^{-iqz}\bra{N}J^\nu (z)\ket{H}\bra{H}J^\mu (0)\ket{N} \\
&=\frac{1}{2(2\pi)^4}\sum_{\sigma_N} \int d^4z \, e^{-iqz}\bra{N}J^\nu (z)J^\mu (0)\ket{N}
\end{align*}
となる.(20.1.2)で述べているように,$q\to \infty$での$W^{\mu\nu}(q,p)$の漸近的振る舞いは,この関係式によりOPEの$z\to 0$における特異性と関係する.\par
これは以下の2点グリーン関数の行列要素に関係する.
\begin{align*}
\left(\frac{m_N}{p^0_N}\right)T^{\mu\nu}(q,p)\equiv\frac{1}{2(2\pi)^4}\sum_{\sigma_N} \int d^4z \, e^{-iqz}\bra{N}T\{J^\nu (z),J^\mu (0)\} \ket{N}
\end{align*}
これは$W^{\mu\nu}$と同様に$q_\mu$と縮約するとカレント保存によりゼロとなるので,(20.6.2)と同じように
\begin{align*}
T^{\mu\nu}(q,p)=-\left(\frac{q^\mu q^\nu}{q^2}-\eta^{\mu\nu}\right)T_1(\nu,q^2)+\frac{1}{m_N^2}\left(p^\mu -\frac{(p\cdot q)}{q^2}q^\mu \right)\left(p^\nu -\frac{(p\cdot q)}{q^2}q^\nu \right)T_2(\nu,q^2)
\end{align*}
として展開できる.(10.8.5)と同様に,$T^{\mu\nu}$の中の時間順序は交換子を用いて,以下のように異なった仕方で書き換えることができる.
\begin{align*}
T\{J^\nu (z),J^\mu (0)\}&=\theta(z^0)[J^\nu(z),J^\mu(0)]+J^\mu(0)J^\nu(z) \\
&=-\theta(-z^0)[J^\nu(z),J^\mu(0)]+J^\nu(z)J^\mu(0)
\end{align*}
($z^0>0$のときは$J^\nu(z)J^\mu(0)$,$z^0<0$のときは$J^\mu(0)J^\nu(z)$となることが確認できる)これを用いて
\begin{align*}
T^{\mu\nu}(q,p)&=\frac{1}{2(2\pi)^4}\sum_{\sigma_N} \int d^4z \, e^{-iqz}\bra{N}T\{J^\nu (z),J^\mu (0)\} \ket{N} \\
&=\frac{1}{2(2\pi)^4}\sum_{\sigma_N} \int d^4z \, \theta(z^0)e^{-iqz}\bra{N}[J^\nu (z),J^\mu (0)] \ket{N} \\
&\qquad \qquad +\frac{1}{2(2\pi)^4}\sum_{\sigma_N} \int d^4z \, e^{-iqz}\bra{N}J^\mu (0)J^\nu (z) \ket{N} \\
&=-\frac{1}{2(2\pi)^4}\sum_{\sigma_N} \int d^4z \, \theta(-z^0)e^{-iqz}\bra{N}[J^\nu (z),J^\mu (0)] \ket{N} \\
&\qquad \qquad +\frac{1}{2(2\pi)^4}\sum_{\sigma_N} \int d^4z \, e^{-iqz}\bra{N}J^\nu (z)J^\mu (0) \ket{N}
\end{align*}
とできる.これらを(10.8.7)$\sim$(10.8.10)と同様に
\begin{align*}
W^{\mu\nu}_A(q,p)&=\frac{1}{2(2\pi)^4}\sum_{\sigma_N} \int d^4z \, \theta(z^0)e^{-iqz}\bra{N}[J^\nu (z),J^\mu (0)] \ket{N} \\
W^{\mu\nu}_R(q,p)&=-\frac{1}{2(2\pi)^4}\sum_{\sigma_N} \int d^4z \, \theta(-z^0)e^{-iqz}\bra{N}[J^\nu (z),J^\mu (0)] \ket{N} \\
W^{\mu\nu}_+(q,p)&=\frac{1}{2(2\pi)^4}\sum_{\sigma_N} \int d^4z \, e^{-iqz}\bra{N}J^\mu (0)J^\nu (z) \ket{N} \\
&= \frac{1}{2(2\pi)^4}\sum_{\sigma_N} \int d^4z \, e^{iqz}\bra{N}J^\mu (z)J^\nu (0) \ket{N}=W^{\nu\mu}(-q,p) \\
W^{\mu\nu}_-(q,p)&=\frac{1}{2(2\pi)^4}\sum_{\sigma_N} \int d^4z \, e^{-iqz}\bra{N}J^\nu (z)J^\mu (0) \ket{N}=W^{\mu\nu}(q,p)
\end{align*}
としておこう.すると
\begin{align*}
T^{\mu\nu}(q,p)=W^{\mu\nu}_A(q,p)+W^{\mu\nu}(q,p)=W^{\mu\nu}_R(q,p)+W^{\mu\nu}(-q,p)
\end{align*}
(最後の項は$\mu\nu$について対称であることを用いている)\par
さて,今回も10.8節での$\ell^\mu$と同様のベクトルを定義する.脚注にあるとおり,$q^2=0$の場合は10.8節の方法と同様だ.$q^\mu=\nu\ell^\mu$とおき,$\ell^\mu$は$\ell^\mu \ell_\mu=0$かつ$\ell^0=1$(すなわち三次元ベクトル部分のノルムは1だ:$|\bm{\ell}|=1$)と定義すれば以降は同じ議論が使える.($\nu$は定義より$q^0$だったことを思い出そう)もし$q^2\neq 0$の場合はこのように定義できないため,別の方法を使う必要がある.それについては分からなかったので,ここでは簡単のため$q^2=0$の場合に限定して話をする.この場合,上の各$W^{\mu\nu}$は以下のようになる.
\begin{align*}
W^{\mu\nu}_A(\nu,p)&=\frac{1}{2(2\pi)^4}\sum_{\sigma_N} \int d^4z \, \theta(z^0)e^{-i\nu(\ell\cdot z)}\bra{N}[J^\nu (z),J^\mu (0)] \ket{N} \\
W^{\mu\nu}_R(\nu,p)&=-\frac{1}{2(2\pi)^4}\sum_{\sigma_N} \int d^4z \, \theta(-z^0)e^{-i\nu(\ell\cdot z)}\bra{N}[J^\nu (z),J^\mu (0)] \ket{N} \\
W^{\mu\nu}_+(\nu,p)&=\frac{1}{2(2\pi)^4}\sum_{\sigma_N} \int d^4z \, e^{-i\nu(\ell\cdot z)}\bra{N}J^\mu (0)J^\nu (z) \ket{N} \\
&= \frac{1}{2(2\pi)^4}\sum_{\sigma_N} \int d^4z \, e^{i\nu(\ell\cdot z)}\bra{N}J^\mu (z)J^\nu (0) \ket{N}=W^{\nu\mu}(-\nu,p) \\
W^{\mu\nu}_-(\nu,p)&=\frac{1}{2(2\pi)^4}\sum_{\sigma_N} \int d^4z \, e^{-i\nu(\ell\cdot z)}\bra{N}J^\nu (z)J^\mu (0) \ket{N}=W^{\mu\nu}(\nu,p)
\end{align*}
微視的因果律によれば,$W_A,W_R$の中の交換子は$z^\mu$が光錐の中にないとゼロになる.したがって階段関数の効果も合わせると,$W_A$では$z^\mu$が未来の光錐内($z^2<0$かつ$z^0>0$,したがって$z^0>|\mathbf{z}|$),つまり
\begin{align*}
\ell\cdot z=-\ell^0 z^0+|\bm{\ell}||\mathbf{z}|\cos\varphi=-z^0+|\mathbf{z}|\cos\varphi < -z^0+z^0\cos\varphi<0
\end{align*}
となって$\ell \cdot z<0$を要求する.また$W_R$も同様に過去の光錐内($z^2<0$かつ$z^0<0$,したがって$z^0<-|\mathbf{z}|$),つまり
\begin{align*}
\ell\cdot z=-\ell^0 z^0+|\bm{\ell}||\mathbf{z}|\cos\varphi=-z^0+|\mathbf{z}|\cos\varphi > |\mathbf{z}|+|\mathbf{z}|\cos\varphi>0
\end{align*}
となって$\ell \cdot z>0$を要求する.これより$W_A$は$\mathrm{Im}\,\nu>0$で解析的で,$W_R$は$\mathrm{Im}\, \nu<0$で解析的だ.(指数関数の肩の$-i\nu(\ell\cdot z)$全体で負である必要を考える)このため,以下の関数を定義する.
\begin{align*}
\mc{W}^{\mu\nu}(\nu,p)=\left\{
\begin{array}{ll}
W^{\mu\nu}_A(\nu,p) &\quad \mathrm{Im}\, \nu>0 \\
W^{\mu\nu}_R(\nu,p) & \quad  \mathrm{Im}\, \nu<0
\end{array}
\right.
\end{align*}
これは$\nu$の実軸上のカットを除き全平面で解析的だ.これよりどんな実の$\nu'$での不連続性も
\begin{align*}
\mc{W}^{\mu\nu}(\nu'+i\epsilon,p)-\mc{W}^{\mu\nu}(\nu'-i\epsilon,p)=W^{\mu\nu}_A(\nu',p)-W^{\mu\nu}_R(\nu',p)=W^{\mu\nu}(-\nu',p)-W^{\mu\nu}(\nu',p)
\end{align*}
を満たす.\par
深非弾性散乱の散乱断面積はmott散乱断面積に比例し,それは$\nu\to\infty$で(すなわち$E_e\to\infty$で)ゼロになる.よって多項式$P(\nu)$は1とすることができて,
\begin{align*}
\mc{W}^{\mu\nu}(\nu,p)=\frac{1}{2\pi i}\oint_C \frac{\mc{W}^{\mu\nu}(z,p)}{z-\nu}dz
\end{align*}
となる.ここで$\nu$は実軸上にはないがそれ以外は任意の点だとする.また$C$は二つの部分からなる.$-\infty+i\epsilon$から$+\infty +i\epsilon$へ実軸上のすぐ上を通り,大きな半円をまわって$-\infty+i\epsilon$に戻る部分と,$+\infty-i\epsilon$から$-\infty-i\epsilon$へ実軸のすぐ下を通り,大きな半円を通って$+\infty-i\epsilon$に戻る部分だ.$\mc{W}^{\mu\nu}$は$|\nu|\to\infty$でゼロになるから,大きな半円からの寄与は無視できる.したがって
\begin{align*}
\mc{W}^{\mu\nu}(\nu,p)=\frac{1}{2\pi i}\int^\infty_{-\infty} \frac{W^{\mu\nu}(-\nu',p)-W^{\mu\nu}(\nu',p)}{\nu'-\nu}d\nu'
\end{align*}
となる.$\nu$を実軸に向かって上から近付けると,
\begin{align*}
W^{\mu\nu}_A(\nu,p)=\frac{1}{2\pi i}\int^\infty_{-\infty} \frac{W^{\mu\nu}(-\nu',p)-W^{\mu\nu}(\nu',p)}{\nu'-\nu-i\epsilon}d\nu'
\end{align*}
となる.(3.1.25)を使うと,これは
\begin{align*}
\frac{1}{\nu'-\nu-i\epsilon}=\frac{\mc{P}}{\nu'-\nu}+i\pi\delta(\nu'-\nu)
\end{align*}
であるから
\begin{align*}
W^{\mu\nu}_A(\nu,p)&=\frac{1}{2\pi i}\int^\infty_{-\infty} \frac{W^{\mu\nu}(-\nu',p)-W^{\mu\nu}(\nu',p)}{\nu'-\nu}d\nu' \\
&\qquad\qquad +\frac{1}{2\pi i}\int^\infty_{-\infty} i\pi\delta(\nu'-\nu)\{W^{\mu\nu}(-\nu',p)-W^{\mu\nu}(\nu',p)\}d\nu' \\
&=\frac{1}{2\pi i}\int^\infty_{-\infty} \frac{W^{\mu\nu}(-\nu',p)-W^{\mu\nu}(\nu',p)}{\nu'-\nu}d\nu'+\frac{1}{2}W^{\mu\nu}(-\nu,p)-\frac{1}{2}W^{\mu\nu}(\nu,p) \\
T^{\mu\nu}(\nu,p)&=W^{\mu\nu}_A(\nu,p)+W^{\mu\nu}(\nu,p) \\
&=\frac{1}{2}W^{\mu\nu}(\nu,p)+\frac{1}{2}W^{\mu\nu}(-\nu,p)+\frac{1}{2\pi i}\int^\infty_{-\infty} \frac{W^{\mu\nu}(-\nu',p)-W^{\mu\nu}(\nu',p)}{\nu'-\nu}d\nu'
\end{align*}
が分かる.ここで(20.6.2)と(20.6.17)を用いれば,$T_1,T_2$についての関係式が得られる.$W_1,W_2$や$T_1,T_2$の係数には$q^\mu,p\cdot q$などがあり,安易に積分の外にこの係数を出せなさそうだが,$q^\mu=\nu\ell^\mu$としていたので$\nu$の寄与はこの係数には存在せず,積分の外に出してよい.
\begin{align*}
\left(\frac{q^\mu q^\nu}{q^2}-\eta^{\mu\nu}\right)=\left( \frac{\ell^\mu \ell^\nu}{\ell^2}-\eta^{\mu\nu} \right),\quad \left(p^\mu-\frac{p\cdot q}{q^2}q^\mu\right)=\left( p^\mu-\frac{p\cdot \ell}{\ell^2}\ell^\mu \right)
\end{align*}
よって
\begin{align*}
T_r(\nu,q^2)=\frac{1}{2}W_r(-\nu,q^2)+\frac{1}{2}W_r(\nu,q^2)+\frac{1}{2\pi i}\int^\infty_{-\infty}d\nu'\frac{W_r(-\nu',q^2)-W_r(\nu',q^2)}{\nu'-\nu}
\end{align*}
が示された.これが求めたかったものだ!$q^2\neq0$が固定された場合についてもこれが導出できるようだが,脚注にある通りそれは難しそうだ.$\nu>q^2/2m_N$なので,$W_r(\nu,q^2)$は$\nu<q^2/2m_N$での値はゼロだ.したがって分散関係の第三項目は
\begin{align*}
+&\frac{1}{2\pi i}\int^\infty_{-\infty}d\nu'\frac{W_r(-\nu',q^2)-W_r(\nu',q^2)}{\nu'-\nu} \\
&=\frac{1}{2\pi i}\int^\infty_{-\infty}d\nu'\frac{W_r(-\nu',q^2)}{\nu'-\nu}-\frac{1}{2\pi i}\int^\infty_{-\infty}d\nu'\frac{W_r(\nu',q^2)}{\nu'-\nu} \\
&=-\frac{1}{2\pi i}\int^{-\infty}_{\infty}d\nu'\frac{W_r(\nu',q^2)}{-\nu'-\nu}-\frac{1}{2\pi i}\int^\infty_{-\infty}d\nu'\frac{W_r(\nu',q^2)}{\nu'-\nu} \quad \because 第一項目で\nu'\to -\nu' \\
&=-\frac{1}{2\pi i}\int^{\infty}_{-\infty}d\nu'\frac{W_r(\nu',q^2)}{\nu'+\nu}-\frac{1}{2\pi i}\int^\infty_{-\infty}d\nu'\frac{W_r(\nu',q^2)}{\nu'-\nu} \\
&=-\frac{1}{2\pi i}\int^\infty_{-\infty}d\nu' W_r(\nu',q^2)\left( \frac{1}{\nu'+\nu}+\frac{1}{\nu'-\nu} \right) \\
&=-\frac{1}{2\pi i}\int^\infty_{q^2/2m_N}d\nu' W_r(\nu',q^2)\left( \frac{1}{\nu'+\nu}+\frac{1}{\nu'-\nu} \right)
\end{align*}
とできる.\par
$T^{\mu\nu}$のOPEを(20.1.3)より
\begin{align*}
\left(\frac{m_N}{p_N^0}\right)T^{\mu\nu}(q,p)&=\frac{1}{2(2\pi)^4}\sum_{\sigma_N} \int d^4z \, e^{-iqz}\bra{N}T\{J^\nu (z),J^\mu (0)\} \ket{N} \\
&\to \frac{1}{2}\sum_{\sigma_N}\left\{ \sum_{n=2}^\infty U_{\mc{O},\mu_1 \cdots \mu_{n-2}}(q) \bra{N} \underset{n階のテンソル}{\uwave{\mc{O}^{\mu\nu \mu_1\cdots \mu_{n-2}}(p)}} \ket{N} \right\}
\end{align*}
$n$階のテンソル$\mc{O}^{\mu_1 \mu_2\cdots \mu_n}$は,ローレンツ変換性より$p^{\mu_1}p^{\mu_2}\cdots p^{\mu_n}$に比例する…と思うだろうが,$p^{\mu_1}p^{\mu_2}\cdots p^{\mu_n}$はローレンツ群の下で様々な変換性を持つ表現の足し合わせである.\par
$\Rightarrow$\uwave{既約表現}として分類したものに比例する,とするのが良い!\\
既約表現である条件は(ジョージアイp147などを参照)「対称」かつ「トレースレス$\eta_{\mu_i \mu_j}T^{\mu_1 \cdots \mu_i \cdots \mu_j \cdots \mu_n}=0$」だ.もしトレースレスでなければ縮約により$n-2$階のテンソルとして再び表現になるので,既約表現ではないということになる.これらを念頭に置いて既約表現を構成しよう.例えば$n=2の\mc{O}^{\mu\nu}$の場合,$p^\mu p^\nu$は
\begin{align*}
p^\mu p^\nu=\uwave{\left( p^\mu p^\nu - \frac{1}{4}\eta^{\mu\nu}p^2 \right)}+\frac{1}{4}\eta^{\mu\nu}p^2
\end{align*}
と既約分解され,第一項目は対称かつトレースレスだ.したがって$\mc{O}^{\mu\nu}\propto p^\mu p^\nu - \eta^{\mu\nu}p^2/4 $だと分かる.$n=3$のとき,$p^\mu p^\nu p^\rho$は
\begin{align*}
p^\mu p^\nu p^\rho=&\left( p^\mu p^\nu p^\rho -\frac{1}{6}\eta^{\mu\nu}p^2p^\rho -\frac{1}{6}\eta^{\mu\rho}p^2p^\nu -\frac{1}{6}\eta^{\nu\rho}p^2p^\mu \right) +\frac{1}{6}\eta^{\mu\nu}p^2p^\rho +\frac{1}{6}\eta^{\mu\rho}p^2p^\nu +\frac{1}{6}\eta^{\nu\rho}p^2p^\mu
\end{align*}
と既約分解され,第一項目は対称かつトレースレスであることが確認できる.($\eta^{\mu\nu},\eta^{\nu\rho},\eta^{\rho\mu}$をかけるとゼロになる)よって
\begin{align*}
\mc{O}^{\mu\nu\rho}\propto p^\mu p^\nu p^\rho -\frac{1}{6}\eta^{\mu\nu}p^2p^\rho -\frac{1}{6}\eta^{\mu\rho}p^2p^\nu -\frac{1}{6}\eta^{\nu\rho}p^2p^\mu
\end{align*}
が分かる.(まぁ偶数個の添え字のものしか以降では出てこないので,これは考える必要ないが)これ以上高階のものは手計算で既約分解するのはかなり大変だが,原理的には適切なトレース項($\eta^{\mu_i\mu_j}に比例$)を差し引くことで対称かつトレースレスのものが得られる.よって
\begin{align*}
\frac{1}{2}\sum_{\sigma_N}\bra{N} \mc{O}_{si}^{\mu_1 \mu_2\cdots \mu_n} \ket{N}=\left(\frac{m_N}{p^0_N}\right)\biggl[ p^{\mu_1}p^{\mu_2}\cdots p^{\mu_s} - [\mathrm{Traces}(\eta^{\mu_i \mu_j}に比例)] \biggr]\braket{\mc{O}_{si}}
\end{align*}
とすることができる.$i$は同じテンソル構造でも異なる演算子を区別するための添え字だ.$s$はローレンツ変換性を区別するための添え字だ.$\braket{\mc{O}_{si}}$は定数係数だ.(peskin,p628も参照すると良い)これによって(演算子についての和は,演算子の種類を区別する添え字$i$の和に変えて)
\begin{align*}
T^{\mu\nu}(q,p)&\to \frac{1}{2}\sum_{\sigma_N}\left[\sum_i \sum^\infty_{s=2}U_{i,\mu_1\cdots \mu_{s-2}}(q)\bra{N} \mc{O}_{si}^{\mu\nu \mu_1\cdots \mu_{s-2}}(p) \ket{N}\right] \\
&=\sum_i \sum_{s=2}^\infty \left[\frac{1}{2}\sum_{\sigma_N}U_{i,\mu_1\cdots \mu_{s-2}}(q)\bra{N} \mc{O}_{si}^{\mu\nu \mu_1\cdots \mu_{s-2}}(p) \ket{N}\right] =\sum_{i,s}T^{\mu\nu}_{si}(q,p) 
\end{align*}
としてローレンツ群の既約表現で関数を演算子積展開することができる!また$T^{\mu\nu}_{si}$は$p$の$s$乗に比例する寄与があると分かるので
\begin{align*}
\left(\frac{m_N}{p_N^0}\right)T^{\mu\nu}_{si}(q,p)=-\left(\frac{q^\mu q^\nu}{q^2}-\eta^{\mu\nu}\right)T_{1,si}(\nu,q^2)+\frac{1}{m_N^2}\left(p^\mu -\frac{(p\cdot q)}{q^2}q^\mu \right)\left(p^\nu -\frac{(p\cdot q)}{q^2}q^\nu \right)T_{2,si}(\nu,q^2)
\end{align*}
と書いておくと便利だ.この演算子$\mc{O}^{\mu_1\cdots \mu_s}_{si}$は左辺には$p^s$の大きさの寄与をするから,右辺の$T_{1,si},T_{2,si}$にはそれぞれ$p^s,p^{s-2}$に比例していると分かる($[\mathrm{Traces}]$の項は考えない.なぜならその項は$\eta^{\mu_i \mu_j}$によって添え字が奪われた分$p^2$を含んでおり,これは$-m_N^2$なので$m^2_N/q^2$か$m^2_N/p\cdot q$の因子で抑えられるから).これらは$\nu=-p\cdot q/m_Nとq^2$の関数なので,$T_{1,si},T_{2,si}$にはそれぞれ$\nu^s,\nu^{s-2}$の寄与があると分かる.もし対数的補正を無視すれば(あとで考える),漸近的自由な理論では係数関数$U(q)$の$q^2$依存性は
\begin{align*}
\left(\frac{m_N}{p^0_N}\right)T^{\mu\nu}(q,p) &= \frac{1}{2(2\pi)^4}\sum_{\sigma_N} \int \underset{-4}{\uwave{d^4z}} \, e^{-iqz}\bra{N}\underset{+3+3}{\uwave{T\{J^\nu (z),J^\mu (0)\} }}\ket{N} \\
&\to \frac{1}{2}\sum_{\sigma_N}\left[\sum_i \sum^\infty_{s=2}\underset{-4+6-d(s,i)}{\uwave{U_{i,\mu_1\cdots \mu_{s-2}}(q)}}\bra{N} \underset{+d(s,i)}{\uwave{ \mc{O}_{si}^{\mu\nu \mu_1\cdots \mu_{s-2}}(p)}} \ket{N}\right]
\end{align*}
より$(q^2)^{(-4+6-d(s,i))/2}$だと分かる.ここで$d(s,i)$は演算子$\mc{O}_{si}$の次元だ.したがって$T_{1,si},T_{2,si}$にはさらにそれぞれ$(q^2)^{(-4+6-d-s)/2},(q^2)^{(-4+6-d-(s-2))/2}$だけの寄与があると分かる.ここで$(q^2)^{-s/2},(q^2)^{-(s-2)/2}$の分は,$\nu^s,\nu^{s-2}$の寄与の中に既に$(q^2)^{+s/2},(q^2)^{+(s-2)/2}$だけの分があることに因る.(定義$\nu=-p\cdot q/m_N$を思い出す)$\omega=2m_N\nu/q^2$より$\nu\propto q^2\omega$を使うと,以上より
\begin{align*}
T_{1,si} \propto \nu^s (q^2)^{(2-d(s,i)-s)/2} & \propto \omega^s (q^2)^{(2-d(s,i)+s)/2} \\
&\propto \omega^s(q^2)^{(2-\tau(s,i))/2} 
\end{align*}
と
\begin{align*}
\nu T_{2,si} \propto \nu^{s-1}(q^2)^{(4-d(s,i)-s)/2} & \propto \omega^{s-1}(q^2)^{(2-d(s,i)+s)/2} \\
&\propto \omega^{s-1}(q^2)^{(2-\tau(s,i))/2}
\end{align*}
となることがわかる.ここで$\tau(s,i)$は演算子$\mc{O}_{si}$のツイストであり,以下で定義される.
\begin{align*}
\tau(s,i)\equiv d(s,i)-s
\end{align*}
これにより$\omega$が固定されて$q^2\to\infty$としたときの$T_1,\nu T_2$への主要な寄与は,ツイストが最小の演算子から得られることが分かる.また(20.6.18)より$T_r(\nu,q^2)$には$\nu$について奇数次の項はないので,これに寄与できる演算子$\mc{O}_{si}$は$s$が偶数のものだけだ.これを見るには
\begin{align*}
\frac{1}{\nu'+\nu}+\frac{1}{\nu'-\nu}&=\frac{2\nu'}{\nu'^2-\nu^2} \\
&=\frac{2}{\nu'}\frac{1}{1-(\nu/\nu')^2} \\
&=\frac{2}{\nu'}\left\{ 1+\left(\frac{\nu}{\nu'}\right)^2+\left(\frac{\nu}{\nu'}\right)^4+\cdots \right\}
 \quad\because \frac{1}{1-x}=1+x+x^2+\cdots =\sum_{i=0}^{\infty}x^i 
\end{align*}
であることを用いれば良い.($\nu'$の積分範囲は$q^2/2m_N=\nu<\nu'<\infty$なので$\nu/\nu'<1$が満たされており,安心してマクローリン展開を用いて良い.)
\begin{align*}
T_r(\nu,q^2)&=\frac{1}{2}W_r(-\nu,q^2)+\frac{1}{2}W_r(\nu,q^2)-\frac{1}{2\pi i}\int^\infty_{q^2/2m_N}d\nu' W_r(\nu',q^2)\left( \frac{1}{\nu'+\nu}+\frac{1}{\nu'-\nu} \right) \\
&=\frac{1}{2}W_r(-\nu,q^2)+\frac{1}{2}W_r(\nu,q^2)-\frac{2}{2\pi i}\int^\infty_{q^2/2m_N}d\nu' W_r(\nu',q^2)\sum_{s=偶数}\nu^s \nu'^{-s-1}
\end{align*}
第1,2項目は全体で明らかに$\nu$に対して偶関数だ.第3項目も明らかに$\nu$の偶数次の項のみだ.したがって$T_r$は全体として$\nu'$に対して偶数次の項しか存在しない.(この表式は後に用いる)\par
対称・トレースレスで$s$階のテンソルで最低の次元をもつ演算子は,場と微分から共変なように構成すれば良いから,
\begin{align*}
&(\mc{O}_{sf})_{\mu_1\cdots \mu_s}\equiv \left( \frac{i^{s-2}}{s!} \right)\bar{\psi}_f \gamma_{\{ \mu_1}D_{\mu_2}^{\leftrightarrow}\cdots D_{\mu_s\}}^{\leftrightarrow}\psi_f \\
&(\mc{O}_{s0})_{\mu_1\cdots \mu_s}\equiv \left( \frac{i^{s-2}}{s!} \right)F_{\alpha\, \nu\{\mu_1}D_{\mu_3}^{\leftrightarrow}\cdots D_{\mu_s}^{\leftrightarrow}F^{\,\,\, \nu}_{\alpha \,\,\, \mu_s\}}\quad(\alpha はリー代数添え字)
\end{align*}
ここで$\leftrightarrow$は,右の場を微分したものから左の場を微分したものを引いたものの半分,という意味だ.$J^\mu$がエルミートであったから,全体をエルミートに保つためにこの演算が必要となる.(足すと全微分となってしまう.同じ運動量をもつ状態で挟んだ行列要素は,$D_\mu \psi$からは$ip_\mu\psi$が,$D_\mu\bar{\psi}からは-ip_\mu\psi$が出てくるので足し合わせるとゼロになってしまう.そこで,引いてそれぞれ半分,として定義すると$\bar{\psi}\cdots D^\leftrightarrow_\mu\psi$からは丁度$\bar{\psi}\cdots ip_\mu\psi$が出てくる)$\{\}$は置換について和をとり,適切なトレース項を引くことを意味する.例えば$s=2$なら
\begin{align*}
(\mc{O}_{2f})_{\mu_1\mu_2} & =\frac{1}{2}\bar{\psi}_f \gamma_{\{\mu_1}D^{\leftrightarrow}_{\mu_2\}}\psi_f \\
&=\frac{1}{2}\bar{\psi}_f\gamma_{\mu_1}D^{\leftrightarrow}_{\mu_2}\psi_f+\frac{1}{2}\bar{\psi}_f\gamma_{\mu_2}D^{\leftrightarrow}_{\mu_1}\psi_f-\frac{1}{4}\eta_{\mu_1\mu_2}[\bar{\psi}_f\gamma^{\nu}D^{\leftrightarrow}_{\nu}\psi_f] \\
&=\frac{1}{4}\bar{\psi}_f\gamma_{\mu_1}D_{\mu_2}\psi_f-\frac{1}{4}D_{\mu_2}\bar{\psi}_f\gamma_{\mu_1}\psi_f \\
&\quad +\frac{1}{4}\bar{\psi}_f\gamma_{\mu_2}D_{\mu_1}\psi_f-\frac{1}{4}D_{\mu_1}\bar{\psi}_f\gamma_{\mu_2}\psi_f \\
&\quad -\frac{1}{8}\eta_{\mu_1\mu_2}[\bar{\psi}_f\gamma^{\nu}D_{\nu}\psi_f-D_{\nu}\bar{\psi}_f\gamma^{\nu}\psi_f] \\
&=\frac{1}{4} \left\{ \bar{\psi}_f\gamma^{\mu}D^{\nu}\psi_f +\bar{\psi}_f\gamma^{\nu}D^{\mu}\psi_f  -\frac{1}{2}\eta^{\mu\nu}\bar{\psi}_f\gamma^{\nu}D_{\nu}\psi_f \right\} \\
&\quad -\frac{1}{4}\left\{ D^{\mu}\bar{\psi}_f\gamma^{\nu}\psi_f -D^{\nu}\bar{\psi}_f\gamma^{\mu}\psi_f  +\frac{1}{2}\eta^{\mu\nu}D_{\nu}\bar{\psi}_f\gamma^{\nu}\psi_f \right\} \\
&=\frac{1}{4} \left\{ \bar{\psi}_f\gamma^{\mu}D^{\nu}\psi_f +\bar{\psi}_f\gamma^{\nu}D^{\mu}\psi_f  -\frac{1}{2}\eta^{\mu\nu}\bar{\psi}_f\gamma^{\nu}D_{\nu}\psi_f \right\} +\mathrm{H.c.} \quad\because (5.4.33)
\end{align*}
($D_\mu\bar{\psi}$は$[D_\mu\psi]^\dagger\beta$の意だ.そうでなければ対称性変換のもとで不変でなくなるし,全体は実でもなくなる.)これは実際に$\eta^{\mu_1\mu_2}$をかけるとゼロであり,トレースレスであることが確認できる.また$\mu_1\leftrightarrow\mu_2$で対称であることも明らかに分かる.\par
(20.6.23)(20.6.24)の演算子の次元は共に$3+(s-1)=4+(s-2)=2+s$なので,ツイストは2だ.よって$T_{r,si}$は共に$q^2$依存性がなくなり,対数補正を除いてブヨルケン・スケーリングを保証する.

\vskip\baselineskip

対数補正を考える.漸近的自由な理論の係数関数の漸近的振る舞いは(20.3.9)で決まるのだった.\par
さて,10.4節で学んだ通り,(電荷で割っておく)カレント
\begin{align*}
J^\mu=\frac{1}{e}\frac{\delta\mc{L}}{\delta A_\mu}=i\bar{\psi}\gamma^\mu\psi
\end{align*}
で,くりこまれた電磁場および電荷は$A_\mu=Z^{-1/2}_3A_{\mu B},e=\sqrt{Z_3}e_B$なので$J^\mu$には電磁場に関するくりこみ因子は存在しない.またくりこまれた$\psi$場を用いることで電磁輻射補正をくりこみ因子に押し付けることができるのであった.しかし電磁輻射補正を無視する場合には$\psi$にくりこみは必要ない.したがって電磁輻射補正を無視するとすれば電磁カレントにくりこみ因子は不要だ.このとき$J^\mu$から作られている$T^{\mu\nu}$において,(20.3.9)の行列は,くりこみ因子が存在しないのでゼロだ:$c_{\ell\ell'}=0$.また,行列$c_{\mc{O}\mc{O}'}=c_{si,s'i'}$は,ローレンツ変換性の違う二つの演算子$\mc{O}_s,\mc{O}_{s'}(s\neq s')$を結びつける効果を持たないので,$c_{si,s'i'}=\delta_{ss'}c_{ii'}(s)$となる.\par
以上を踏まえて(20.3.9)がどのような形になるかを見る.$\ell^2=q^2$を係数関数を定義するために選んだ$q^2$の特定の値とすると,
\begin{align*}
\int^{q}_{\ell} \frac{d\mu}{\mu}g^2_\mu=-\frac{8\pi^2}{b}\ln g^2_q+\frac{8\pi^2}{b}\ln g^2_\ell=-\frac{8\pi^2}{b}\ln \left(g^2_q/g^2_{\ell}\right)
\end{align*}
よって(20.3.9)は,
\begin{align*}
U^\ell_{si}(q n)&\to q^{-4+6-d(s,i)}\sum_{\ell's'i'}\delta_{\ell\ell'}\mc{C}^{\ell'}_{s'i'}\left[\left(g^2_q/g^2_{\ell} \right)^{8\pi^2c(s)/b} \right]_{s's,i'i} \\
&=\kappa^{-4+6-d(s,i)}\sum_{i'}\mc{C}^{\ell}_{si'}\left[\left(g^2_q/g^2_{\ell} \right)^{8\pi^2c(s)/b} \right]_{i'i}
\end{align*}
ここで$q$は$q^\mu=q n^\mu$で$n^\mu$を固定したときの値,$\mc{C}^{\ell}_{si}$は定数係数だ.これより
\begin{align*}
T^{\mu\nu}(q,p) &\to \sum_{s j} \left[\frac{1}{2}\sum_{\sigma_N}U_{j,\mu_1\cdots \mu_{s-2}}(q)\bra{N} \mc{O}_{sj}^{\mu\nu \mu_1\cdots \mu_{s-2}}(p) \ket{N}\right] \\
&=\sum_{sij}q^{-4+6-d(s,i)}p^s\, \mc{C}_{si}\left[\left(g^2_q/g^2_{\ell} \right)^{8\pi^2c(s)/b} \right]_{ij}\braket{O_{sj}}
\end{align*}
と振る舞う.ここで$p$は$p^\mu=pm^\mu$で$m^\mu$を固定したときの値だ.$qとp$のベキについては対数補正以外の部分なので既に計算に取り入れた.したがって$T_r$は
\begin{align*}
T_1(\nu,q^2)\to \sum_{sij}\omega^s\mc{A}_{si}\left[\left(g^2_q/g^2_{\ell} \right)^{8\pi^2c(s)/b} \right]_{ij}\braket{O_{sj}} \\
T_2(\nu,q^2)\to \sum_{sij}\omega^{s-1}\mc{B}_{si}\left[\left(g^2_q/g^2_{\ell} \right)^{8\pi^2c(s)/b} \right]_{ij}\braket{O_{sj}}
\end{align*}
と振る舞うことが分かる.(主要な演算子は(20.6.23)(20.6.24)であるから$q^2$依存性は現れない.)ここで$\mc{A,B}$は$\mc{O}$の係数関数に現れる$\mc{C}_{si}$から来る定数係数だ.\par
OPEの係数は問題の特定の過程(今回の$eN\to eH$などの散乱過程)や,クォークの閉じ込めという束縛条件からは独立だ.よって係数$\mc{A}_{si}$と$\mc{B}_{si}$は仮想的に単純な過程を考察して求めることができる!\par
$\Rightarrow$ フレーバー$f$の自由クォークによる電子の散乱を考える.このとき,自由クォークは相互作用前と相互作用後では運動量は等しいので,前に述べた通り$\bar{\psi}\cdots D_\mu^\leftrightarrow \psi$は丁度$\bar{\psi}\cdots ip_\mu\psi$となる.したがって
\begin{align*}
\bra{f',\sigma'}\mc{O}_{sf}\ket{f'',\sigma''}&=\frac{i^{s-2}}{s!}\delta_{f'f}\bar{u}'\gamma^{\{ \mu_1}(ip^{\mu_2})\cdots (ip^{\mu_s\}})u\delta_{ff''} \\
&=\frac{i^{s-2}}{s!}i^{s-1}(\bar{u}'\gamma^{\{\mu_1}u)p^{\mu_2}\cdots p^{\mu_s\}}\delta_{ff'}\delta_{ff''} \\
&=\frac{(-1)^s i}{s!}(\bar{u}'\gamma^{\{\mu_1}u)p^{\mu_2}\cdots p^{\mu_s\}}\delta_{ff'}\delta_{ff''}
\end{align*}
となる.(20.6.27)を$\sigma'=\sigma''$について平均し,その結果を(20.6.19)と比べる.$\bra{N}\mc{O}\ket{N}$のために$f'=f''$として(2巻p92参照)
\begin{align*}
\frac{1}{2}\sum_{\sigma'=\sigma''}\bra{f',\sigma'}\mc{O}_{sf}\ket{f',\sigma''}&=\frac{(-1)^s i}{2s!}\left(\sum_{\sigma'}\bar{u}_\alpha(\mathbf{p},\sigma') \gamma^{\{\mu_1}_{\alpha\beta} u_\beta(\mathbf{p},\sigma')\right)p^{\mu_2}\cdots p^{\mu_s\}}\delta_{ff'} \\
&=\frac{(-1)^s i}{2s!}\left(\sum_{\sigma'}u_\beta(\mathbf{p},\sigma')\bar{u}_\alpha(\mathbf{p},\sigma') \gamma^{\{\mu_1}_{\alpha\beta} \right)p^{\mu_2}\cdots p^{\mu_s\}}\delta_{ff'} \\
&=\frac{(-1)^s i}{2s!}\mathrm{Tr}\left[ \left(\frac{-i\Slash{p}+m_f}{2p^0}\right) \gamma^{\{\mu_1} \right]p^{\mu_2}\cdots p^{\mu_s\}}\delta_{ff'} \\
&=\frac{(-1)^s}{4p^0 s!}\mathrm{Tr}\left[\gamma^\nu \gamma^{\{\mu_1} \right]p_\nu p^{\mu_2}\cdots p^{\mu_s\}}\delta_{ff'}  \\
&\qquad +m_f\frac{(-1)^si}{4p^0 s!}\mathrm{Tr}\left[\gamma^{\{\mu_1} \right]p^{\mu_2}\cdots p^{\mu_s\}}\delta_{ff'} \\
&=\frac{(-1)^s}{p^0 s!}\eta^{\nu\{\mu_1}p_\nu p^{\mu_2}\cdots p^{\mu_s\}}\delta_{ff'}  \quad \because 奇数個の\gamma 行列のトレースはゼロ\\
&=\frac{(-1)^s}{p^0 s!}p^{\{\mu_1} p^{\mu_2}\cdots p^{\mu_s\}}\delta_{ff'} \\
&=\frac{(-1)^s}{p^0}\biggl[ p^{\mu_1}p^{\mu_2}\cdots p^{\mu_s} - [\mathrm{Traces}] \biggr]\delta_{ff'}
\end{align*}
最後の等号については,括弧$\{\}$の意味を思い出す.これは$\mu_1\sim\mu_s$について置換して適切なトレース項を引くのであった.置換した項は$s!$個あるが,$p$同士は可換なので$s!p^{\mu_1}\cdots p^{\mu_s}$が現れ,トレース項も同様に同じ項が$s!$個現れるからこれが成り立つ.これが(20.6.19)(ただし$m_N\to m_f$としたもの)に等しいので
\begin{align*}
\left(\frac{m_f}{p^0}\right)\biggl[ p^{\mu_1}p^{\mu_2}\cdots p^{\mu_s} - [\mathrm{Traces}] \biggr]\braket{\mc{O}_{sf}}_{f'}=\frac{(-1)^s}{p^0}\biggl[ p^{\mu_1}p^{\mu_2}\cdots p^{\mu_s} - [\mathrm{Traces}] \biggr]\delta_{ff'}
\end{align*}
p35$\ell$7の通り,$s$が偶数のときのみが寄与するので
\begin{align*}
\braket{\mc{O}_{sf}}_{f'}=\delta_{ff'}/m_f,\quad \braket{\mc{O}_{s0}}_{f'}=0
\end{align*}
がわかる.\par
(20.6.9)と(20.6.10)を導くパートン模型の計算を,この$Q-e$散乱についてもう一度たどる.まず(20.6.2)が$Q-e$散乱では
\begin{align*}
W_f^{\mu\nu}(q,p)&=-\left(\frac{q^\mu q^\nu}{q^2}-\eta^{\mu\nu}\right)W'_{1}(\nu,q^2)+\frac{1}{m_f^2}\left(p^\mu-\frac{p\cdot q}{q^\mu}\right)\left(p^\nu-\frac{p\cdot q}{q^\nu}\right)W'_{2}(\nu,q^2) \\
&=-\left(\frac{q^\mu q^\nu}{q^2}-\eta^{\mu\nu}\right)W'_{1}(\nu,q^2)+\frac{1}{m_N^2}\left(p^\mu-\frac{p\cdot q}{q^\mu}\right)\left(p^\nu-\frac{p\cdot q}{q^\nu}\right)\left\{\frac{m_N^2}{m_f^2}W'_{2}(\nu,q^2)\right\}
\end{align*}
すなわち
\begin{align*}
W'_{1}(\nu,q^2)=W_1(\nu,q^2),\quad W'_{2}(\nu,q^2)=\left(m_f/m_N\right)^2W_2(\nu,q^2)
\end{align*}
と対応付けられることがわかる.また,(20.6.3)と同様に
\begin{align*}
\left[\frac{d^2\sigma}{d\Omega d\nu}\right]_f=\left(\frac{d\sigma}{d\Omega}\right)_{\mathrm{mott}}\left(W'_2+2W'_1\tan^2\frac{\theta}{2}\right)
\end{align*}
となるが,(20.6.8)と同様にこの散乱は$e-\mu$散乱と同様の($\mu$粒子をクォークに置き換えただけの)散乱断面積であるから
\begin{align*}
\left[ \frac{d^2\sigma}{d\Omega d\nu} \right]_f =\left[ \frac{d\sigma}{d\Omega} \right]_{\mathrm{mott}}Q^2_f\left( 1+\frac{q^2}{2m_f^2}\tan^2 \frac{\theta}{2} \right)\delta\left(\nu'-\frac{q^2}{2m_f}\right)
\end{align*}
となる.ここで$\nu'=-p\cdot q/m_f=(m_N/m_f)\nu$だ.したがって
\begin{align*}
W'_{2}(\nu,q^2)&=\left(m_f/m_N\right)^2W_2(\nu,q^2)=Q^2_f\delta\left(\frac{m_N}{m_f}\nu-\frac{q^2}{2m_N}\frac{m_N}{m_f} \right) \\
&=Q^2_f\frac{m_f}{m_f   }\delta\left(\nu-\frac{q^2}{2m_N}\right) \\
&=Q^2_f\frac{m_f}{m_N}\delta\left(\frac{q^2}{2m_N}\omega-\frac{q^2}{2m_N}\right) \quad \because 定義より\frac{q^2}{2m_N}=\frac{\nu}{\omega} \\
&=Q^2_f\frac{m_f}{m_N}\frac{2m_N}{q^2}\delta\left(\omega-1\right)=Q^2_f\frac{m_f}{m_N}\frac{\omega}{\nu}\delta\left(\omega-1\right) \\
&=Q^2_f\frac{m_f}{m_N}\frac{1}{\nu}\delta\left(\omega-1\right) \quad \because x\delta(x-a)=a\delta(x-a)
\end{align*}
すなわち
\begin{align*}
\nu W_2(\nu,q^2)=Q^2_f \left(\frac{m_N}{m_f}\right)\delta(\omega-1)
\end{align*}
$W_1$についても
\begin{align*}
W'_{2}(\nu,q^2)&=W_1(\nu,q^2)=\frac{1}{2}Q_f^2\frac{q^2}{2m_f^2}\delta\left(\nu'-\frac{q^2}{2m_f}\right) \\
&=\frac{1}{2}Q_f^2\frac{q^2}{2m_f^2}\left(\frac{m_f}{m_N}\right)\delta\left(\nu-\frac{q^2}{2m_N}\right) \\
&=\frac{1}{2}Q_f^2\frac{q^2}{2m_f^2}\left(\frac{m_f}{m_N}\right)\delta\left(\nu-\frac{q^2}{2m_N}\right) \\
&=\frac{1}{2}Q_f^2\frac{q^2}{2m_fm_N}\frac{2m_N}{q^2}\delta\left(\omega-1\right) \\
&=\frac{1}{2}Q_f^2\frac{1}{m_f}\delta(\omega-1)=Q^2_f \delta(\omega-1)/2m_f 
\end{align*}
となる.(脚注の通り(20.6.10)を用いてもいい)
(20.6.31)(20.6.32)を分散関係(20.6.18)に使うと,$\omega\neq1$では($\delta(\omega-1)$より第1,2項目がゼロ)
\begin{align*}
T_{1,f}(\nu,q^2)&=-\frac{1}{2\pi i}\int^\infty_{q^2/2m_N}d\nu' W_{1,f}(\nu',q^2)\left( \frac{1}{\nu'+\nu}+\frac{1}{\nu'-\nu} \right) \\
&=-\frac{1}{2\pi i}\int^\infty_{q^2/2m_N}d\omega' \frac{q^2}{2m_N} \frac{Q^2_f}{2m_f} \delta(\omega'-1) \left( \frac{2m_N}{q^2(\omega'+\omega)}+\frac{2m_N}{q^2(\omega'-\omega)} \right) \quad \because \nu'=q^2\omega'/2m_N\\
&=-\frac{Q^2_f}{4\pi im_f}\int^\infty_1 d\omega' \delta(\omega-1)\left( \frac{1}{\omega'+\omega}+\frac{1}{\omega'-\omega} \right) \\
&=-\frac{Q^2_f}{4\pi im_f}\left( \frac{1}{1+\omega}+\frac{1}{1-\omega} \right) =\frac{Q^2_f}{4\pi im_f}\frac{2}{\omega^2-1} \\
&=\frac{Q^2_f}{2\pi im_f}\frac{1}{\omega^2-1} \\
\nu T_{2,f}(\nu,q^2)&=-\frac{\nu}{2\pi i}\int^\infty_{q^2/2m_N}d\nu' W_{2,f}(\nu',q^2)\left( \frac{1}{\nu'+\nu}+\frac{1}{\nu'-\nu} \right) \\
&=-\frac{\nu}{2\pi i}\int^\infty_{q^2/2m_N}d\omega'\frac{q^2}{2m_N} \left[\frac{1}{\omega'}\frac{2m_N}{q^2}Q^2_f \left(\frac{m_N}{m_f}\right)\delta(\omega-1)\right]\frac{2m_N}{q^2}\left( \frac{1}{\omega'+\omega}+\frac{1}{\omega'-\omega} \right)  \\
&=-\frac{\nu}{2\pi i}\int^\infty_{q^2/2m_N}d\omega' \frac{1}{\omega'}Q^2_f \left(\frac{m_N}{m_f}\right)\delta(\omega-1)\frac{2m_N}{q^2}\left( \frac{1}{\omega'+\omega}+\frac{1}{\omega'-\omega} \right) \\
&=\frac{\nu}{2\pi i}Q^2_f\frac{m_N}{m_f}\frac{2m_N}{q^2}\frac{2}{1-\omega^2} \\
&=\frac{2\nu}{2\pi i}Q^2_f \frac{m_N}{m_f}\frac{\omega}{\nu}\frac{1}{\omega^2-1} \quad \because \frac{q^2}{2m_N}=\frac{\omega}{\nu} \\
&=\frac{2Q^2_f}{2\pi i}\frac{m_N}{m_f}\frac{\omega}{\omega^2-1}
\end{align*}
を得る.マクローリン展開
\begin{align*}
\frac{1}{1-x}=1+x+x^2+\cdots =\sum_{i=0}^\infty x^i
\end{align*}
を用いると,(20.6.33)(20.6.34)は
\begin{align*}
T_{1,f}&=\frac{-Q^2_f}{2\pi i m_f}(1+\omega^2+\omega^4+\cdots)=\frac{iQ^2_f}{2\pi m_f}\sum_{s=偶数} \omega^s \\
\nu T_{2,f}&=\frac{-2Q^2_f}{2\pi i}\frac{m_N}{m_f}(\omega+\omega^3+\omega^5+\cdots)=\frac{iQ^2_f}{\pi}\frac{m_N}{m_f}\sum_{s=偶数}\omega^{s-1}
\end{align*}
よって(20.6.25)(20.6.26)と比較すると,くりこみ点$q^2=\ell^2$においては$g^2_q/g^2_\ell=1$なので
\begin{align*}
T_{1}(\nu,\ell^2)&=\sum_{sij}\omega^s \mc{A}_{si}\delta_{ij}\braket{\mc{O}_{sj}} =\sum_{si}\omega^s \mc{A}_{si}\braket{\mc{O}_{si}} \\
&=\sum_{sf}\omega^s \mc{A}_{sf}\braket{\mc{O}_{sf}}_{f'}+\sum_{s}\omega^s \mc{A}_{s0}\braket{\mc{O}_{s0}}_{f'} \\
&=\sum_{sf}\omega^s \mc{A}_{sf}\delta_{ff'}/m_f \quad \because(20.6.29)(20.6.30)\\
&=\sum_{s=偶数}\omega^s \frac{\mc{A}_{sf}}{m_f}=\sum_{s=偶数} \omega^s\frac{iQ^2_f}{2\pi m_f} \\
\Rightarrow \quad &\mc{A}_{si}=\frac{iQ^2_i}{2\pi} \\
\nu T_{2}(\nu,\ell^2)&=\sum_{sij}\omega^{s-1} \mc{B}_{si}\delta_{ij}\braket{\mc{O}_{sj}} =\sum_{si}\omega^s \mc{B}_{si}\braket{\mc{O}_{si}} \\
&=\sum_{sf}\omega^s \mc{B}_{sf}\braket{\mc{O}_{sf}}_{f'}+\sum_{s}\omega^s \mc{B}_{s0}\braket{\mc{O}_{s0}}_{f'} \\
&=\sum_{sf}\omega^s \mc{B}_{sf}\delta_{ff'}/m_f \quad \because(20.6.29)(20.6.30)\\
&=\sum_{s=偶数}\omega^s \frac{\mc{B}_{sf}}{m_f}=\sum_{s=偶数} \omega^s\frac{iQ^2_f}{\pi }\frac{m_N}{m_f} \\
\Rightarrow \quad &\mc{B}_{si}=\frac{im_N Q^2_i}{\pi}
\end{align*}
を得る.ここでグルーオンの電荷$Q_0$は(中性粒子なので)ゼロとおいた.よって下を得る.
\begin{align*}
T_1(\nu,q^2)\to \frac{i}{2\pi}\sum_{sij}\omega^s Q^2_i \left[\left(g^2_q/g^2_{\ell} \right)^{8\pi^2c(s)/b} \right]_{ij}\braket{O_{sj}} \\
\nu T_2(\nu,q^2) \to \frac{im_N}{\pi}\sum_{sij}\omega^{s-1}Q^2_i \left[\left(g^2_q/g^2_{\ell} \right)^{8\pi^2c(s)/b} \right]_{ij}\braket{O_{sj}}
\end{align*}

構造関数$W_r(\nu,q^2)$に戻る.$T_r(\nu,q^2)$の$\omega^s=(2m_N\nu/q^2)^s$の係数を求める.
\begin{align*}
T_r(\nu,q^2)=&\frac{1}{2}W_r(-\nu,q^2)+\frac{1}{2}W_r(\nu,q^2)\qquad\leftarrow \omega^sに比例していないので以降省略 \\
&-\frac{1}{2\pi i}\int^\infty_{q^2/2m_N}d\nu' W_r(\nu',q^2)\left\{ \frac{1}{\nu'+\nu}+\frac{1}{\nu'-\nu} \right\} \\
=&-\frac{2}{2\pi i}\int^\infty_{q^2/2m_N}d\nu' W_r(\nu',q^2)\sum_{s=偶数}\nu^s \nu'^{-s-1} \\
=&-\frac{2}{2\pi i}\int^\infty_{q^2/2m_N}d\nu' W_r(\nu',q^2)\sum_{s=偶数}\left(\frac{q^2}{2m_N}\right)^s \nu'^{-s-1}\omega^s
\end{align*}
よって$\omega^s$の係数は
\begin{align*}
-&\frac{2}{2\pi i}\left(\frac{q^2}{2m_N}\right)^s\int^\infty_{q^2/2m_N}d\nu' \nu'^{-1-s}W_r(\nu',q^2) \\
&=\frac{i}{\pi}\left(\frac{q^2}{2m_N}\right)\int^\infty_1\left(d\omega \frac{q^2}{2m_N}\right)\left( \frac{q^2}{2m_N}\omega \right)^{-1-s}W_r\left(\frac{\omega q^2}{2m_N},q^2\right) \\
&=\frac{i}{\pi}\int^\infty_1d\omega\,\omega^{-1-s}W_r\left(\frac{\omega q^2}{2m_N},q^2\right)
\end{align*}
(20.6.36)および(20.6.37)と比較する.$\omega^s$の係数を比べると
\begin{align*}
\int^\infty_1d\omega\, \omega^{-1-s}W_1\left(\frac{\omega q^2}{2m_N},q^2\right)\to\frac{1}{2}\sum_{ij}Q^2_i \left[\left(g^2_q/g^2_{\ell} \right)^{8\pi^2c(s)/b} \right]_{ij}\braket{O_{sj}} 
\end{align*}
また$\omega^{s-1}$の係数を比べると
\begin{align*}
&\int^\infty_1d\omega\, \omega^{-1-(s-1)}\nu W_2\left(\frac{\omega q^2}{2m_N},q^2\right) \\
&=\int^\infty_1d\omega\, \omega^{-s}\nu W_2\left(\frac{\omega q^2}{2m_N},q^2\right)\to m_N\sum_{ij}Q^2_i \left[\left(g^2_q/g^2_{\ell} \right)^{8\pi^2c(s)/b} \right]_{ij}\braket{O_{sj}}
\end{align*}
となる.これらが満たされるためには,パートン分布関数$\mc{F}_i$がモーメント方程式
\begin{align*}
\int^1_0 dx\, x^{s-1}\mc{F}_i(x,q^2)=\frac{1}{2}\sum_j \left[\left(g^2_q/g^2_{\ell} \right)^{8\pi^2c(s)/b} \right]_{ij}\braket{O_{sj}}
\end{align*}
を満たすとして
\begin{align*}
W_1\left(\frac{\omega q^2}{2m_N},q^2\right)&\to \sum_iQ^2_i \mc{F}_i\left(\frac{1}{\omega},q^2\right) \\
\nu W_2\left(\frac{\omega q^2}{2m_N},q^2\right)&\to \frac{2m_N}{\omega}\sum_iQ^2_i \mc{F}_i\left(\frac{1}{\omega},q^2\right)
\end{align*}
であるならば良い.実際
\begin{align*}
\int^1_0 dx \, x^{s-1}\mc{F}_i(x,q^2)&=-\int^1_\infty d\omega\frac{1}{\omega^2}\left(\frac{1}{\omega}\right)^{s-1}\mc{F}_i\left(\frac{1}{\omega},q^2\right) \quad \because x=\frac{1}{\omega},dx=-\frac{d\omega}{\omega^2} \\
&=\int^\infty_1d\omega\,\omega^{-1-s}\mc{F}_i\left(\frac{1}{\omega},q^2\right)
\end{align*}
であるから
\begin{align*}
(20.6.38)左辺=\int^\infty_1 d\omega \, \omega^{-1-s}W_1\left(\frac{\omega q^2}{2m_N},q^2\right)&\to\int^\infty_1d\omega\, \omega^{-1-s}\sum_iQ^2_i \mc{F}_i\left(\frac{1}{\omega},q^2\right) \\
&=\frac{1}{2}\sum_{ij}Q^2_i \left[\left(g^2_q/g^2_{\ell} \right)^{8\pi^2c(s)/b} \right]_{ij}\braket{O_{sj}} =(20.6.38)右辺 \\
(20.6.39)左辺=\int^\infty_1d\omega\, \omega^{-s}\nu W_2\left(\frac{\omega q^2}{2m_N},q^2\right) &\to \int^\infty_1d\omega \, \omega^{-s-1}(2m_N)\sum_i Q^2_i \mc{F}_i\left(\frac{1}{\omega},q^2\right) \\
&=m_N\sum_{ij}Q^2_i \left[\left(g^2_q/g^2_{\ell} \right)^{8\pi^2c(s)/b} \right]_{ij}\braket{O_{sj}} =(20.6.39)右辺
\end{align*}
となっている.\par
(20.6.40)(20.6.41)は明らかにパートン模型の表式(20.6.9)(20.6.10)だ!\\
ここまでに用いた条件はOPEと漸近的自由性のみだ.\par
$\Rightarrow$つまり,漸近的自由性はブヨルケン・スケーリングの補正版だけでなく,$W_1$と$W_2$の間のキャラン・グロス関係式(20.6.12)も意味している!

\vskip\baselineskip

(20.6.42)とくりこみ群方程式(20.3.8)より以下の微分方程式が導かれる.
\begin{align*}
q^2\frac{d}{dq^2}\int^1_0 dx \, x^{s-1}\mc{F}_i(x,q^2)&=q^2\frac{d}{dq^2}\left\{ \frac{1}{2}\sum_k \left[\left(g^2_q/g^2_{\ell} \right)^{8\pi^2c(s)/b} \right]_{ik}\braket{\mc{O}_{sk}} \right\} \\
&=\frac{1}{2}\sum_{jk}\left(q^2\frac{d}{dq^2}g^2_q\right)\left(\frac{1}{g^2_\ell} \right)\frac{8\pi^2c_{ij}(s)}{b}\left[\left(g^2_q/g^2_{\ell} \right)^{8\pi^2c(s)/b-1} \right]_{jk} \braket{\mc{O}_{sk}} \\
&=\frac{1}{2}\sum_{jk}\left( -\frac{b}{8\pi^2}g^4_q \right) \left(\frac{1}{g^2_\ell} \right)\frac{8\pi^2c_{ij}(s)}{b}\left[\left(g^2_q/g^2_{\ell} \right)^{8\pi^2c(s)/b-1} \right]_{jk} \braket{\mc{O}_{sk}} \\
&=-\frac{1}{2}\sum_{jk}g^2_qc_{ij}(s)\left[\left(g^2_q/g^2_{\ell} \right)^{8\pi^2c(s)/b} \right]_{jk} \braket{\mc{O}_{sk}} \\
&=-g^2_q \sum_j c_{ij}(s)\left[ \frac{1}{2}\sum_k \left[\left(g^2_q/g^2_{\ell} \right)^{8\pi^2c(s)/b} \right]_{jk}\braket{\mc{O}_{sk}} \right] \\
&=-g^2_q\sum_j c_{ij}(s)\int^1_0 dx \, x^{s-1}\mc{F}_j(x,q^2)
\end{align*}
これらの方程式は,くりこみ点$\ell^2$での初期条件
\begin{align*}
\int^1_0 dx \, x^{s-1}\mc{F}_i(x,\ell^2)=\frac{1}{2}\braket{\mc{O}_{si}} 
\end{align*}
を用いると,解を一つだけ持つ.\par
$\Rightarrow$この微分方程式と初期条件の組を,モーメント方程式の代わりに用いることができる!
(20.6.43)は以下の$\mc{F}_i$の微分方程式の解によって満たされている.
\begin{align*}
q^2\frac{d}{dq^2}\mc{F}_i(x,q^2)=\frac{g^2_q}{4\pi^2}\sum_j \int^1_x \frac{dy}{y}P_{ij}\left(\frac{x}{y}\right)\mc{F}_j(y,q^2)
\end{align*}
ここで行列関数$P_{ij}(z)$は以下の条件を満たすとする.
\begin{align*}
\int^1_0 z^{s-1}P_{ij}(z)dz=-4\pi^2c_{ij}(s)
\end{align*}
この微分方程式の解が(20.6.43)を満たすことを見よう.
\begin{align*}
(20.6.43)左辺&=q^2\frac{d}{dq^2}\int^1_0 dx \, x^{s-1}\mc{F}_i(x,q^2)=\int^1_0 dx \, x^{s-1} q^2\frac{d}{dq^2}\mc{F}_i(x,q^2) \\
&=\frac{g^2_q}{4\pi^2}\sum_j \int^1_0 dx\,x^{s-1}\int^1_y \frac{dy}{y}P_{ij}\left(\frac{x}{y}\right)\mc{F}_j(y,q^2) \\
&=\frac{g^2_q}{4\pi^2}\sum_j \int^1_0 \frac{dy}{y}\mc{F}_j(y,q^2)\int^y_0 dx \, x^{s-1}P_{ij}\left(\frac{x}{y}\right) \\
&=\frac{g^2_q}{4\pi^2}\sum_j \int^1_0 \frac{dy}{y}\mc{F}_j(y,q^2)\int^1_0 (dz\, y) \, (yz)^{s-1}P_{ij}(z) \quad \because x=yz ,x:[1\sim y]\to z:[0\sim 1]\\
&=\frac{g^2_q}{4\pi^2}\sum_j \int^1_0 \frac{dy}{y}\mc{F}_j(y,q^2)\, y^s\uwave{\int^1_0 dz \, z^{s-1}P_{ij}(z)} \\
&=-g^2_q\sum_j c_{ij}(s)\int^1_0 dy\, y^{s-1}\mc{F}_j(y,q^2) \quad \because (20.6.46)\\
&=-g^2_q\sum_j c_{ij}(s)\int^1_0 dx\, x^{s-1}\mc{F}_j(x,q^2) =(20.6.43)右辺
\end{align*}
2行目から3行目にかけての等号では,面積分の入れ替え
\begin{align*}
\int^1_0 dx\int^1_x dy=\int^1_0 dy \int^y_0 dx \quad (面積分範囲は0<x<y<1)
\end{align*}
を用いた.これにより,微分方程式(20.6.45)を満たす解$\mc{F}_i$によって(20.6.43)は満たされていることがわかる.

\vskip\baselineskip

行列$c_{ij}(s)$は量子色力学ではジョージァイとポリッツァー,およびグロスとウィルチェックによって計算された.$N$フレーバーのクォーク($f=f_1,f_2,\cdots f_N$の$N$個のフレーバーがある)が,十分軽く,質量ゼロとして扱えるとした.また,その他の全てのクォークは非常に重く,場について積分されて,それらは強い相互作用の結合定数への影響以外は無視できるとした.その結果は,(20.6.23)(20.6.24)演算子に対し以下となる.(導出はPeskinのp637等参照)
\begin{align*}
&c_{00}(s)=\frac{1}{2\pi^2}\left\{ C_1\left[ \frac{1}{12}-\frac{1}{s(s-1)}-\frac{1}{(s+1)(s+2)}+\sum^s_{t=2}\frac{1}{t} \right] +\frac{N}{3}C_2\right\} \\
&c_{f0}(s)=\frac{1}{\pi^2}C_2\left[ \frac{1}{s+2}+\frac{2}{s(s+1)(s+2)} \right] \\
&c_{of}(s)=\frac{1}{8\pi^2}C_3\left[\frac{1}{s+1}+\frac{2}{s(s-1)}\right] \\
&c_{ff'}(s)=\frac{1}{8\pi^2}C_3\left[ 1-\frac{2}{s(s+1)}+4\sum^s_{t=2}\frac{1}{t} \right]\delta_{ff'}
\end{align*}
ここで$0,f$はそれぞれ演算子(20.6.24)(20.6.23)を意味する.定数$C_1,C_2$は(17.5.33)(17.5.34)で定義される.$N$はフレーバーの個数.$C_3$は17.4節の記法で
\begin{align*}
t_\alpha t_\alpha=C_3 g^2 \bm{1}
\end{align*}
$SU(3)$ゲージ群でクォークがその基本表現$\bm{3}$だという現実的な場合には
\begin{align*}
C_1=3,C_2=\frac{1}{2},C_3=\frac{4}{3}
\end{align*}
となる.\par
これらの複雑な結果は,アルタレリ・パリジ関数$P_{ij}(z)$を用いることでより簡潔になる(らしいがとてもそうは思えない…)以下が成立すると(20.6.46)が満たされていることが確認できる.
\begin{align*}
&P_{ff'}=\delta_{ff'}\left[ \frac{4}{3}\left( \frac{1+x^2}{(1-x)_+} \right)+2\delta(1-x) \right] \\
&P_{f0}=-4\left(x^2-x+\frac{1}{2}\right) \\
&P_{0f}=-\frac{2}{3}\left( \frac{2}{x}-2+x \right) \\
&P_{00}=6\left[ \frac{1-x}{x}+x(1-x)+\frac{x}{(1-x)_+}+\frac{11}{12}\delta(1-x) \right]-\frac{N}{3}\delta(1-x)
\end{align*}
(多分誤植.確認の計算はこれからする)ここで
\begin{align*}
\int^1_0\frac{f(x)}{(1-x)_+}dx=\int^1_0\frac{f(x)-f(1)}{1-x}dx
\end{align*}
だ.若干計算ルールを誤解しやすいので注意が必要である.例えば$P_{ff'}$には$(1+x^2)/(1-x)_+$という項があるが,先に計算するのではなく積分してから演算を開始する必要がある.つまり例えばこの項ならば
\begin{align*}
\int^1_0 f(x)\left( \frac{1+x^2}{(1-x)_+} \right)d&x=\int^1_0 \frac{f(x)}{(1-x)_+}dx +\int^1_0 \frac{f(x)x^2}{(1-x)_+}dx \\
&=\int^1_0 \frac{f(x)-f(1)}{1-x}dx+\int^1_0 \frac{f(x)x^2-f(1)}{1-x}dx
\end{align*}
のように計算する必要がある.\par
さて,これが(20.6.46)を満たすことを見よう.まず$P_{ff'}$であるが
\begin{align*}
\int^1_0 x^{s-1}P_{ff'}(x)dx&=\delta_{ff'}\int^1_0 x^{s-1}\left[  \frac{4}{3}\left( \frac{1+x^2}{(1-x)_+} \right)+2\delta(1-x)  \right]dx \\
&=\delta_{ff'}\left[ \frac{4}{3}\int^1_0\frac{x^{s-1}}{(1-x)_+}dx+\frac{4}{3}\int^1_0\frac{x^{s+1}}{(1-x)_+}dx+2 \right] \\
&=\delta_{ff'}\left[ \frac{4}{3}\int^1_0\frac{x^{s-1}-1}{1-x}dx+\frac{4}{3}\int^1_0\frac{x^{s+1}-1}{1-x}dx+2 \right] \\
&=\delta_{ff'}\left[ -\frac{4}{3}\int^1_0\{1+x+x^2+\cdots +s^{s-2}\}dx-\frac{4}{3}\int^1_0\{1+x+x^2+\cdots x^{s}\}dx+2 \right] \\
&=\delta_{ff'}\left[- \frac{4}{3}\left\{1+\frac{1}{2}+\frac{1}{3}+\cdots +\frac{1}{s-1}\right\}-\frac{4}{3}\left\{1+\frac{1}{2}+\frac{1}{3}+\cdots +\frac{1}{s+1}\right\}+2 \right] \\
&=\delta_{ff'}\left[-\frac{8}{3}-\frac{8}{3}\sum^s_{t=2}\frac{1}{t}+\frac{4}{3}\frac{1}{s}-\frac{4}{3}\frac{1}{s+1}+2 \right] \\
&=\delta_{ff'}\left[-\frac{2}{3}+\frac{4}{3}\frac{1}{s(s+1)}-\frac{8}{3}\sum^2_{t=2}\frac{1}{t}\right] \\
&=-\frac{2}{3}\left[ 1-\frac{2}{s(s+1)}+4\sum^s_{t=2}\frac{1}{t} \right]\delta_{ff'} \\
&=-4\pi^2\frac{1}{8\pi^2}\frac{4}{3}\left[ 1-\frac{2}{s(s+1)}+4\sum^s_{t=2}\frac{1}{t} \right]\delta_{ff'}\\
&=-4\pi^2c_{ff'}(s)
\end{align*}
となる.次に$P_{f0}$は
\begin{align*}
\int^1_0 x^{s-1}P_{f0}(x)dx&=-4\int^1_0 x^{s-1}\left[x^2-x+\frac{1}{2}\right]dx \\
&=-4\left[\int^1_0x^{s+1}dx-\int^1_0 x^{s}dx+\frac{1}{2}\int^1_0 x^{s-1}dx\right] \\
&=-4\left[ \frac{1}{s+2}-\frac{1}{s+1}+\frac{1}{2s} \right]\\
&=-\frac{2(s^2+s+2)}{s(s+1)(s+2)} \\
&=-2\left[\frac{1}{s+2}+\frac{2}{s(s+1)(s+2)}\right] \\
&=-4\pi^2\frac{1}{\pi^2}\frac{1}{2}\left[\frac{1}{s+2}+\frac{2}{s(s+1)(s+2)}\right] =-4\pi^2 c_{f0}(s)
\end{align*}
となる.次に$P_{0f}$は
\begin{align*}
\int^1_0x^{s-1}P_{0f}(x)dx&=-\frac{2}{3}\int^1_0 x^{s-1}\left(\frac{2}{x}-2+x\right)dx \\
&=-\frac{2}{3}\left(2\int^1_0 x^{s-2}dx-2\int^1_0 x^{s-1}dx+\int^1_0x^sdx\right) \\
&=-\frac{2}{3}\left( \frac{2}{s-1}-\frac{2}{s}+\frac{1}{s+1} \right) \\
&=-\frac{2}{3}\left( \frac{1}{s+1}+\frac{2}{s(s-1)} \right) \\
&=-4\pi^2\frac{1}{8\pi^2}\frac{4}{3}\left( \frac{1}{s+1}+\frac{2}{s(s-1)} \right) \\
&=-4\pi^2 c_{0f}(s)
\end{align*}
となる.最後に$P_{00}$は
\begin{align*}
\int^1_0 x^{s-1}P_{00}(x)dx&=\int^1_0 x^{s-1}\left\{ 6\left[ \frac{1-x}{x}+x(1-x)+\frac{x}{(1-x)_+}+\frac{11}{12}\delta(1-x) \right]-\frac{N}{3}\delta(1-x) \right\}dx \\
&=6\left[\int^1_0x^{s-2}dx-\int^1_0x^{s-1}dx +\int^1_0 x^sdx -\int^1_0 x^{s+1}dx+\int^1_0 \frac{x^s}{(1-x)_+}dx+\frac{11}{12} \right]-\frac{N}{3} \\
&=6\left[\frac{1}{s-1}-\frac{1}{s}+\frac{1}{s+1}-\frac{1}{s+2}-\left(1+\frac{1}{2}+\frac{1}{3}+\cdots \frac{1}{s}\right)+\frac{11}{12}\right]-\frac{N}{3} \\
&=6\left[-\frac{1}{12}+\frac{1}{s(s-1)}+\frac{1}{(s+1)(s+2)}-\sum^s_{t=2}\frac{1}{t}  \right]-\frac{N}{3} \\
&=-4\pi^2\frac{1}{2\pi^2}\left\{ 3\left[ \frac{1}{12}-\frac{1}{s(s-1)}-\frac{1}{(s+1)(s+2)}+\sum^s_{t=2}\frac{1}{t} \right]+\frac{N}{6} \right\} \\
&=-4\pi^2 c_{00}(s)
\end{align*}
となる.\par
各$s$について,行列$c_{ij}(s)$には$(N-1)$重に縮退した固有値があることを示す.簡単のために(20.6.47)を$\alpha$,(20.6.48)を$\beta$,(20.6.49)を$\rho$,(20.6.50)の$\delta_{ff'}$の係数を$\sigma$とおくと$c_{ij}$行列は$(N+1)\times (N+1)$正方行列
\begin{align*}
c_{ij}=\left(
\begin{array}{ccccc}
c_{00}          & c_{0 f_1}             & c_{0 f_2}      &  \cdots      & c_{0 f_N} \\
c_{f_1 0}      & c_{f_1 f_1}            & c_{f_1 f_2}    &  \cdots      & c_{f_1 f_N}      \\
c_{f_2 0}      & c_{f_2 f_1}            &  c_{f_2 f_2}      & \cdots       &c_{f_2 f_N}     \\
\vdots        & \vdots              &    \vdots       &  \ddots     &         \\
c_{f_N 0}      & c_{f_N f_1}           &   c_{f_N f_2}    &                 & c_{f_N f_N} 
\end{array}
\right)=\left(
\begin{array}{ccccc}
\alpha      & \rho    & \rho      &  \cdots  & \rho \\
\beta     & \sigma  & 0          &  \cdots  & 0      \\
\beta      & 0          &  \sigma & \cdots   &0       \\
\vdots & \vdots  &   \vdots    &  \ddots  &         \\
\beta      & 0         &   0         &               & \sigma 
\end{array}
\right)
\end{align*}
と書ける.これの固有ベクトルのひとつは,$(0,v_{f_1},v_{f_2},\cdots ,v_{f_N})^t$というベクトルに,成分の和がゼロ$\sum_i v_{f_i}=0$という条件をつけたものである.実際これは
\begin{align*}
\left(
\begin{array}{ccccc}
\alpha      & \rho    & \rho      &  \cdots  & \rho \\
\beta      & \sigma  & 0          &  \cdots  & 0      \\
\beta      & 0          &  \sigma & \cdots   &0       \\
\vdots & \vdots  &   \vdots    &  \ddots  &         \\
\beta      & 0         &   0         &               & \sigma 
\end{array}
\right)\left(
\begin{array}{ccccc}
0 \\
v_{f_1} \\
v_{f_2} \\
\vdots \\
v_{f_N}
\end{array}
\right)
=\left(
\begin{array}{ccccc}
\rho(v_{f_1}+v_{f_2}+\cdots +v_{f_N}) \\
\sigma v_{f_1} \\
\sigma v_{f_2} \\
\vdots \\
\sigma v_{f_N}
\end{array}
\right)=\sigma \left(
\begin{array}{ccccc}
0 \\
v_{f_1} \\
v_{f_2} \\
\vdots \\
v_{f_N}
\end{array}
\right)
\end{align*}
となる.したがって固有ベクトルは$\sum_i v_{f_i}=0$という条件をもつ$\sum_i v_{f_i}\bm{e}_{f_i}$というベクトルで,その固有値は$\sigma$だとわかる.$N$重縮退のように見えるが,固有ベクトルに条件式が一つあるのでこれは$(N-1)$重縮退だ.このベクトルを,$\mc{O}_{sf}$が係数となるように基底を$\bm{e}_{f}\to \bm{w}_f$と変換すると
\begin{align*}
\sum_i v_{f_i}\bm{e}_{f_i}=\sum_i \mc{O}_{sf_i}\bm{w}_{f_i}
\end{align*}
となって,対応する条件式は,成分の和をとることで
\begin{align*}
\sum_j\left\{\sum_i v_{f_i}(\bm{e}_{f_i})_{f_j}\right\}&=\sum_i v_{f_i}\sum_j(\bm{e}_{f_i})_{f_j}=\sum_i v_{f_i}=0 \\
&=\sum_j\left\{\sum_i \mc{O}_{sf_i}(\bm{w}_{f_i})_{f_j}\right\}=\sum_i \mc{O}_{sf_i}\sum_j (\bm{w}_{f_i})_{f_j}
\end{align*}
$\mc{O}_{sf_i}$は独立であるから,$\sum_j (\bm{w}_{f_i})_{f_j}=0$が同値の条件であるとわかる.すなわち$c_{ij}$の固有ベクトルは(20.6.23)の演算子の線型結合$\sum_i \mc{O}_{sf_i}\bm{a}_{f_i}$で,$\mc{O}_{sf}$の係数は成分の和がゼロ$\sum_i (\bm{a}_{f_i})_f$だ.基底であるから$\bm{a}_f$はそれぞれ独立だ.そして(くどいようだが)その固有値は(20.6.50)の$\delta_{ff'}$の係数
\begin{align*}
c(s,随伴)=c_{ff'}(s)=\frac{1}{8\pi^2}C_3\left[ 1-\frac{2}{s(s+1)}+4\sum^s_{t=2}\frac{1}{t} \right]
\end{align*}
となり$(N-1)$重縮退である.\par
今の計算で何がわかるだろうか.(20.3.9)によれば,演算子積展開に現れる係数関数$U_{si}$はくりこみ因子を含み,OPE$\sum_i U_{si}\mc{O}_{si}$は,どのような演算子で展開するかによって$c_{ij}$の固有値が決まり,係数関数が計算できる.つまり今回の場合ならば
\begin{align*}
O'_{sf_i}=\sum_{f_j}a_{f_i f_j}\mc{O}_{sf_j} \quad \left(\sum_{f_i}{a_{f_if_j}}=0\right)
\end{align*}
という演算子で展開するならば,固有値の値が$c(s,随伴)$だと決まり,係数関数が計算できる,ということだ.\par
さらに,$SU(N)$の単位表現に属する固有演算子が二つある.これらは演算子(20.6.24)と,演算子(20.6.23)の$f$についての和,の二つの線型結合
\begin{align*}
\bm{a}\left[x\mc{O}_{s0}+y\sum_{f}\mc{O}_{sf}\right]=\left(
\begin{array}{ccccc}
a_0(x\mc{O}_{s0}+y\sum_{f}\mc{O}_{sf} )\\
a_{f_1}(x\mc{O}_{s0}+y\sum_{f}\mc{O}_{sf} ) \\
a_{f_2}(x\mc{O}_{s0}+y\sum_{f}\mc{O}_{sf}) \\
\vdots \\
a_{f_N}(x\mc{O}_{s0}+y\sum_{f}\mc{O}_{sf})
\end{array}
\right)
\end{align*}
で与えられる.(これらの演算子は$SU(N)$のもとで不変であることを思い出そう.よってこのベクトルは$SU(N)$のもとで単位表現に属する.)$\bm{a}$は$c_{ij}$の固有ベクトルとすれば全体で固有ベクトルである.これを知るためには,
\begin{align*}
\left(
\begin{array}{ccccc}
\alpha      & \rho    & \rho      &  \cdots  & \rho \\
\beta      & \sigma  & 0          &  \cdots  & 0      \\
\beta      & 0          &  \sigma & \cdots   &0       \\
\vdots & \vdots  &   \vdots    &  \ddots  &         \\
\beta      & 0         &   0         &               & \sigma 
\end{array}
\right)\left(
\begin{array}{ccccc}
a \\
b \\
b \\
\vdots \\
b
\end{array}
\right)
\end{align*}
を計算して,固有演算子である$x,y$を見つける必要がある.この式は実質
\begin{align*}
\left(
\begin{array}{cc}
\alpha & \rho N \\
\beta & \sigma
\end{array}
\right)\left(
\begin{array}{cc}
a \\
b
\end{array}
\right)=\left(
\begin{array}{cc}
c_{00(s)} & c_{0f}(s)N \\
c_{f0}(s) & c(s,随伴)
\end{array}
\right)\left(
\begin{array}{cc}
a \\
b
\end{array}
\right)
\end{align*}
であるから,固有演算子と固有値は$2\times 2$行列
\begin{align*}
c(s)_{\mathrm{singlet}}=\left(
\begin{array}{cc}
c_{00}(s) & c_{0f}(s)N \\
c_{f0}(s) & c(s,随伴)
\end{array}
\right)
\end{align*}
を対角化すれば得られる.(20.6.47)-(20.6.50)に$s=2$を代入すれば
\begin{align*}
c(2)_{\mathrm{singlet}}=\left(
\begin{array}{cc}
NC_2/6\pi^2 & NC_3/6\pi^2 \\
C_2/3\pi^2 & C_3/3\pi^2
\end{array}
\right)
\end{align*}
という形になる.これの固有値は
\begin{align*}
\det (c(s)_{\mathrm{singlet}}-\lambda I)&=\left(\frac{NC_2}{6\pi^2}-\lambda\right)\left(\frac{C_3}{3\pi^2}-\lambda\right)-\frac{NC_3}{6\pi^2}\frac{C_2}{6\pi^2} \\
&=\lambda^2-\lambda\left(\frac{NC_2}{6\pi^2}+\frac{C_3}{3\pi^2}\right)=0 \\
&\Leftrightarrow \lambda=0,\quad \frac{NC_2}{6\pi^2}+\frac{C_3}{3\pi^2}
\end{align*}
である.つまりこれはゼロ固有値を一つ持つ.
\begin{align*}
\mc{O}^{\mu\nu}_{20}&=\frac{1}{4}F^{\,\rho \{\mu}_{\alpha}F_{\alpha \,\,\rho}^{\,\nu\}} \\
&=\frac{1}{4}\left\{F_{\alpha}^{\rho \mu}F^{\nu}_{\alpha\,\rho}+F_{\alpha}^{\rho \nu}F^{\mu}_{\alpha\,\rho}-\frac{1}{2}\eta^{\mu\nu}F_{\alpha\, \rho \sigma}F^{\rho\sigma}_{\alpha} \right\}
\end{align*}
と
\begin{align*}
\sum_f \mc{O}^{\mu\nu}_{2f}&=\sum_f \frac{1}{2}\bar{\psi}_f \gamma^{\{\mu_1}D_{\leftrightarrow}^{\nu \}}\psi_f  \\
&=\frac{1}{4} \sum_f \left\{ \bar{\psi}_f\gamma^{\mu}D^{\nu}\psi_f +\bar{\psi}_f\gamma^{\nu}D^{\mu}\psi_f  -\frac{1}{2}\eta^{\mu\nu}\bar{\psi}_f\gamma^{\rho}D_{\rho}\psi_f \right\} +\mathrm{H.c.}
\end{align*}
の線型結合でエネルギー・運動量テンソルに等しいものに相当する.実際,QCDラグランジアンを通常の通り
\begin{align*}
\mc{L}=-\frac{1}{4}F_{\alpha\,\mu\nu}F^{\mu\nu}_{\alpha}-\sum_f \bar{\psi}_f\gamma^\mu D_\mu \psi_f
\end{align*}
ととると,(7.3.34)(7.4.11)を用いて計算されるエネルギー運動量テンソルと類似のものだと理解できる.(ここで計算するには複雑で面倒なので,QEDのエネルギー運動量テンソルで調べればすぐわかる.Peskinのp430等参照.トレース項がまさに(7.3.34)の$\delta^\mu_\nu$に比例する項から来るものだ.)$c_{ij}$の固有値がゼロであるから,このときは(20.3.7)よりくりこみ因子は不要となる.またもう一つの固有値として$NC_2/6\pi^2+ C_3/3\pi^2$がある.\par
さて,(20.6.42)を両辺$s$微分すると,
\begin{align*}
\mathrm{LHS} &=\frac{d}{ds}\left\{\int^1_0 dx\, x^{s-1}\mc{F}_i(x,q^2)\right\} \\
&=\int^1_0 dx \ln x \,x^{s-1}\mc{F}_i(x,q^2)<0 \quad \because \ln x<0 \quad(0<x<1) \\
\mathrm{RHS} &=\frac{d}{ds}\left\{\frac{1}{2}\sum_j \left[\left(g^2_q/g^2_{\ell} \right)^{8\pi^2c(s)/b} \right]_{ij}\braket{O_{sj}}\right\} \\
&= \frac{1}{2}\sum_{jk}\frac{8\pi^2}{b}c'_{ij}(s)\ln \left(g^2_q/g^2_{\ell} \right) \left[\left(g^2_q/g^2_{\ell} \right)^{8\pi^2c(s)/b} \right]_{jk}\braket{O_{sk}} \\
&=\sum_{j}\frac{8\pi^2}{b}c'_{ij}(s)\ln \left(g^2_q/g^2_{\ell} \right)\int^1_0 dx\, x^{s-1}\mc{F}_j(x,q^2)
\end{align*}
ここで$\mc{F}_i$は確率分布関数であるから正であり,また漸近的自由であるから$q\to\infty$で$g_q^2$はゼロに近づき$g^2_q<g^2_\ell$であるから$\ln \left(g^2_q/g^2_{\ell} \right)<0$である.左辺が負であるから,右辺も負である必要があり,したがって$c'_{ij}(s)>0$である必要がある.すなわち,ある与えられた$s$について$c_{ij}(s)$の最小固有値は,任意の$s'<s$の$c_{ij}(s')$の最小固有値よりも大きくなければならない.$s=2$については最小固有値はゼロだから,$s>2$の他の全ての固有値は正だと結論できる.実際,$s>2$の演算子は次元が4以上であるから,これらはくりこまれるので,くりこみ因子があり$c$はゼロでないとわかる.したがって$g^2_q\to 0$でエネルギー運動量テンソルのみの寄与が生き残るという極端な場合にのみ,ブヨルケンスケーリング則が厳密に満たされる.実際,このときは$s=2$のみ生き残り$c=0$であるから(20.6.42)より
\begin{align*}
\int^1_0 x \mc{F}_i(x,q^2)=\frac{1}{2}\braket{\mc{O}_{si}}
\end{align*}
となるので,その$i$についての和も定数となり(20.6.7)と同じ形式となる.


\newpage

\subsection{リノーマロン}
ダイソンは$n$次のダイアグラムの数は$n!$の程度に大きくなることに気付いた.\par
$\Rightarrow$($n\to\infty$)の極限でゼロにならないので,収束半径はゼロだ.(ダランベールの公式等参照)\par
$n$次の項が$n!$のように大きくなるベキ級数の収束性を改善するには,ボレル変換という方法がある.\par
$\Rightarrow$これで級数を収束させたり,少なくともその級数の振る舞いを改善して結合定数の(より広い範囲にわたって)漸近級数として使えるようにできる.

\vskip\baselineskip

ある与えられた級数
\begin{align*}
F(g)=\sum_n f_n g^n
\end{align*}
(結合定数のベキであるから,これがダイアグラムに対応)について,ボレル変換$B(z)\equiv B(F)(z)$
\begin{align*}
B(z)\equiv\sum_n \frac{f_n}{n!}z^n
\end{align*}
を考える.もし$f_n$が$n!$より速く大きくならなければ$\lim_{n\to\infty}(f_n/n!)=0$となりダランベールの公式等より級数$B(z)$は一般的にゼロでない収束半径を持つ.
\begin{align*}
\int^\infty_0 \exp(-z/g)z^ndz&=\int^\infty_0 (-g)\left\{ \exp(-z/g) \right\}'z^ndz \\
&=\left[(-g)\exp(-z/g)z^n\right]^\infty_0+gn\int^\infty_0 \exp(-z/g)z^{n-1}dz \\
&=gn \int^\infty_0 \exp(-z/g)z^{n-1}dz=g^2n(n-1)\int^\infty_0 \exp(-z/g)z^{n-2}dz \\
&=\cdots = g^i n(n-1)\cdots (n-i+1)\int^\infty_0 \exp(-z/g)z^{n-i}dz=\cdots \\
&=g^n n! \int^\infty_0 \exp(-z/g)dz=g^nn! [(-g)\exp(-z/g)]^\infty_0 \\
&=n! g^{n+1}
\end{align*}
を用いると,少なくとも形式的には
\begin{align*}
gF(g)=\sum_n f_n g^{n+1}&=\sum_n \frac{f_n}{n!}n!g^{n+1} \\
&=\sum_n \frac{f_n}{n!}\int^\infty_0\exp(-z/g)z^n dz \\
&=\int^\infty_0 \exp(-z/g)\left\{ \sum_n \frac{f_n}{n!}z^n \right\}dz \\
&=\int^\infty_0 \exp(-z/g)B(z)dz
\end{align*}
となる.

\vskip\baselineskip

$B(z)$が複素平面上のどこかに特異性をもてば,それによって(20.7.2)の収束半径が制限される.(一般の収束半径の話)
$\Rightarrow$しかし,\uwave{その特異性が正の実軸上になければ},これは乗り越えられない問題ではない!\par
(20.7.3)を用いて,$F(g)$を計算するには,積分範囲が$[0\sim\infty]$であるうえに$\exp(-z/g)$は$z\gg g$においてほぼゼロなので$F(g)$の計算にほぼ寄与しない.よって$B(z)$は$z$が$g$より小さい,あるいは$g$程度の大きさの実の正値について知れば良い!\par
$\Rightarrow$ $B(z)$の特異性が,全て原点から$g$よりずっと大きい場所にあるなら,$g$程度の距離までで(20.7.1)で展開できて,$B(z)$がわかる.もしいくつかの極$z_1,z_2\cdots$等が$g$と同じかそれより小さい絶対値をもっていても,$(z-z_1)(z-z_2)\cdots\times B(z)$に対するベキ級数を使って$B(z)$を$g$の大きさの$z$まで求められる.(しかし,そのためには$B(z)$の極がどこにあるかを知らなければならない) \par
ただし,もし$B(z)$の特異性が正の実軸上にある場合,$\exp(-z/g)$は$z\gg g$においてほぼゼロでも$B(z)$が特異性をもつので$z\gg g$の範囲の積分も大きく寄与し,無視できない.よって(20.7.3)を無効にしてしまい,タチが悪い.この積分の経路は特異点を避けるように曲げられるが,上を通るか下を通るかの不定性がある.

\vskip\baselineskip

ボレル変換$B(z)$には「インスタントン」として知られる,古典的な場の方程式(KG方程式のような)の解に伴う特異性と,OPEに関係した「リノーマロン」として知られる特異性が存在することを示す.QCDでは摂動級数の和をとるためのボレル変換を妨害するのはリノーマロンだ.\par
以下の経路積分で定義される関数$F(g)$を考える.
\begin{align*}
F(g)\equiv \int [d\phi]\exp (I[\phi,g])
\end{align*}
虚時間を用いているので,$\exp(iI)$ではない.これを(20.7.1)で展開するときの係数は
\begin{align*}
&F(g)=\sum_n f_n g^n \quad \Rightarrow \quad g^{-(i+1)}F(g)=\cdots \frac{f_{i-1}}{g^2}+\frac{f_i}{g}+f_{i+1}+\cdots \\
&\oint g^{-i-1}F(g)dg=2\pi i f_i \quad \because コーシーの積分公式 \\
\Rightarrow \quad & f_n=\frac{1}{2\pi i}\int [d\phi]\oint dg\, g^{-n-1}\exp(I[\phi,g]) \\
&\quad =\frac{1}{2\pi i}\int [d\phi]\oint dg\, \exp(I[\phi,g]-(n+1)\ln g) \quad \because x^a=\exp(a\ln x)
\end{align*}
ここで$\oint$は複素$g$空間で$g=0$の点を囲む閉曲線についての反時計まわりの一周積分を意味する.$n$が非常に大きいとき,この積分は指数関数の因数が$\phi,g$の両方について停留点
\begin{align*}
&\left. \frac{\delta}{\delta\phi(x)}\left[ I[\phi,g]-(n+1)\ln g \right]\right|_{\phi=\phi_n,g=g_n}=0 \quad \Rightarrow\quad \left.\frac{\delta I[\phi,g]}{\delta \phi(x)}\right|_{\phi=\phi_n}=0 \\
&\left.\frac{\partial}{\partial g}\left[ I[\phi,g]-(n+1)\ln g \right]\right|_{\phi=\phi_n g=g_n}=0 \quad \Rightarrow \quad \left.\frac{\partial I[\phi_n ,g ] }{\partial g}\right|_{g=g_n}=\frac{n+1}{g_n}
\end{align*}
から主要な寄与があるとするのが妥当だ.\par
たとえば,もし$I[\phi,g]$が質量ゼロのスカラー場の作用
\begin{align*}
I[\phi,g]=-\frac{1}{2}\int \partial_i \phi \partial_i \phi \,d^4x -\frac{g}{24} \int \phi^4 \,d^4x \quad(和iはユークリッド座標の方向1\sim4についてとる)
\end{align*}
と仮定すると,場の方程式(20.7.6)は
\begin{align*}
&\mc{L}=-\frac{1}{2}\partial_i \phi \partial_i \phi -\frac{g}{24} \phi^4 \quad として \quad \left.\frac{\delta I[\phi,g]}{\delta \phi(x)}\right|_{\phi=\phi_n}= \left[ \partial_i \left(\frac{\partial \mc{L}}{\partial(\partial_i \phi)}\right)-\frac{\partial \mc{L}}{\partial \phi} \right]_{\phi=\phi_n,g=g_n}=0
\end{align*}
であるから
\begin{align*}
&\partial_i \partial_i \phi_n\equiv \Box \phi_n =\frac{g_n}{6}\phi^3_n
\end{align*}
となる.すぐ後で$g_n$は負だとわかるので,この解は
\begin{align*}
\phi_n(x)=(-g_n)^{-1/2}\chi(x) \quad (\chi(x)は方程式\Box\chi=-\frac{1}{6}\chi^3の解で,gに独立)
\end{align*}
と書ける.実際,これは
\begin{align*}
\Box \phi_n&=(-g_n)^{-1/2}\Box \chi \\
&=-(-g_n)^{-1/2}\frac{1}{6}\chi^3 =-\frac{1}{6}(-g_n)^{-1/2}(-g_n)^{3/2}\phi_n^3 \\
&=\frac{g_n}{6}\phi^3_n
\end{align*}
で解とわかる.条件(20.7.7)は
\begin{align*}
-\frac{1}{24}\int d^4x \phi^4_n=\frac{n+1}{g_n}
\end{align*}
(20.7.10)を用いて
\begin{align*}
&-\frac{1}{24}\int d^4x (-g_n)^2\chi^4=\frac{n+1}{g_n} \\
\Rightarrow\quad & g_n=-\frac{1}{24(n+1)}\int d^4x\chi^4
\end{align*}
だとわかる.この停留点$\phi_n,g_n$では,作用(20.7.8)は
\begin{align*}
I[\phi_n,g_n]&=-\frac{1}{2}\int \partial_i \phi_n \partial_i \phi_n \,d^4x -\frac{g}{24} \int \phi^4_n \,d^4x \\
&=\frac{1}{2}\int \phi_n \Box \phi_n d^4x -\frac{g_n}{24}\int \phi^4_n d^4x \\
&=\frac{g_n}{12}\int \phi^4_n d^4x-\frac{g_n}{24}\int \phi^4_n d^4x \quad \because (20.7.9) \\
&=\frac{g_n}{24}\int \phi^4_n d^4x \\
&=-n-1\quad \because p45下から\ell 1
\end{align*}
となる.(20.7.5)をこの停留点で計算すると,$n\to\infty$で以下を得る.
\begin{align*}
f_n&\approx g^{-n-1}_n\exp(I[\phi_n,g_n]) \quad ((9.3.6)から出る定数因子があるが,無視) \\
&=\left(-\frac{1}{24}\int \chi^4 d^4x\right)^{-n-1}e^{-n-1} \quad\because (20.7.12)(20.7.13) \\
&=(n+1)^{n+1}\left(-\frac{e}{24}\int \chi^4 d^4x\right)^{-n-1} \\
&\approx n^n\left(-\frac{e}{24}\int \chi^4 d^4x\right)^{-n} \quad \because n\to\infty なので+1して良い\\
&\approx n!\left(-\frac{1}{24}\int \chi^4 d^4x\right)^{-n} \quad \because スターリングの公式n!\approx\sqrt{2\pi n}\left(\frac{n}{e}\right)^n
\end{align*}
最後の等式では,脚注にある通り$\sqrt{2\pi n}$因子も無視.したがって$B(z)$は
\begin{align*}
B(z)=\sum_n \frac{f_n}{n!}z^n\approx \sum_n \left(-\frac{1}{24}\int \chi^4 d^4x\right)^{-n}z^n=\sum_n \left(\frac{z}{z_1}\right)^n
\end{align*}
とできる.ここで
\begin{align*}
z_1\equiv -\frac{1}{24}\int \chi^4 d^4x
\end{align*}
だ.$z=z_1$でこの級数は発散し,$z=z_1$が主要な特異点だとわかる.$z_1$は負であるから,(20.7.3)の積分を実行する妨げにはならない.\par
極の位置(20.7.15)を計算するには,場の方程式(20.7.11)に解
\begin{align*}
\chi=\frac{4\sqrt{3}a}{r^2+a^2} \quad (r=(x_ix_i)^{1/2},aは任意のパラメータ)
\end{align*}
があることに気付けばいいようだ.とても気付けないが.実際これは
\begin{align*}
\Box \chi=\sum_{i=1}^4 \partial_i \partial_i \chi&=\sum_{i=1}^44\sqrt{3}a\left( \frac{8(x_i)^2}{(r^2+a^2)^3}-\frac{2}{(r^2+a^2)^2} \right) \\
&=4\sqrt{3}a\left( \frac{8r^2}{(r^2+a^2)^3}-\frac{8}{(r^2+a^2)^2} \right) \\
&=4\sqrt{3}a\left( \frac{8r^2}{(r^2+a^2)^3}-\frac{8(r^2+a^2)}{(r^2+a^2)^3} \right) \\
&=4\sqrt{3}a\frac{-8a^2}{(r^2+a^2)^3}=-32\sqrt{3}\frac{a^3}{(r^2+a^2)^2}=-\frac{1}{6}\chi^3
\end{align*}
であって解だ.この解は23.5節で論じる「インスタントン解」の初歩的な例になっている…らしい.幸運にも極の位置は$a$に依存しない.
\begin{align*}
z_1&=-\frac{1}{24}\int \chi^4 d^4x=-\frac{1}{24}\int^\infty_0 \frac{(4\sqrt{3}a)^4}{(r^2+a^2)^4}2\pi^2 r^3 dr  \quad \because 2巻p246よりd^4x\to2\pi^2r^3dr \\
&=-\frac{1}{24}\int^\infty_0 \frac{(4\sqrt{3}a)^4}{(r^2+a^2)^4}2\pi^2 \frac{r^2}{2} dr^2 \quad \because \frac{dr^2}{dr}=2rよりr^3dr=\frac{r^2}{2}dr^2 \\
&=-96\pi^2a^4\int^\infty_0 \frac{r^2}{(r^2+a^2)^4}dr^2 \\
&=-96\pi^2a^4 \frac{1}{6a^4} \quad\because \int^\infty_0 \frac{x}{(x+a)^4}dx=\frac{1}{6a^2} \\
&=-16\pi^2
\end{align*}
これにより,結合定数が$g\ll 16\pi^2$のエネルギーでは$B(z)$の摂動級数を(20.7.3)に使えることがわかる.もし$g/16^2$が1より大きくても,最初に述べた通り$B(z)$を$(z+16\pi^2)B(z)$の摂動級数から計算することができる.\par
量子色力学の現実の問題はこれと異なる種類の特異性である.これらはリノ―マロンと呼ばれる.リノ―マロンは20.3図のようなダイアグラムからくる.このときは$f_n$が$n!$より大きくなるから,(20.7.2)の級数$B(z)$に特異性が出てくる.これは変化する結合定数$g_\mu$を定義するのに用いられたくりこみ点の運動量よりはるかに小さな仮想運動量から生じるので,特に赤外リノーマロンと呼ばれるらしい.

\vskip\baselineskip

仮想運動量が小さいダイアグラムのせいで摂動論が使えなくなるのは何も新しいことではない.20.2節で見たように,演算子積展開の重要な点は,ファインマンダイアグラムのうち,\uwave{どの線も大きな運動量をもつ部分}を分離することにある.その部分は漸近的自由な理論では摂動論を用いて計算できる.ファインマンダイアグラムのうち,残された小さな運動量が流れる部分は摂動論では計算できない.


\newpage

\part{ゲージ対称性の自発的破れ}
\setcounter{section}{21}
\setcounter{subsection}{0}
\subsection{ユニタリー・ゲージ}
19章で見たように,大域的対称性の群$G$が部分群$H$に自発的に破れた理論では,各々の独立な破れた対称性毎に質量ゼロのNGボゾンが存在する.(NG定理) \par
$\Rightarrow$すなわち,スピンゼロの実場$\phi_n(x)$の質量行列$M^2_{nm}$(19.2節の最初を参照すれば,実表現のみを対象にしている.複素表現も実場にできることを思い出す.)は,$G$の各々の独立な対称性の生成子$t_\alpha$に対応して,ゼロ固有値と固有ベクトル$\sum_m (t_\alpha)_{nm}v_m$をもつ.(19.2節参照)ここで$v_m=\braket{\phi_m(0)}_{\VAC}$\par
これらのNGボゾンモード(成分)は,場に$G$変換$\gamma^{-1}(x)$
\begin{align*}
\tilde{\phi}_n(x)=\sum_m \gamma^{-1}_{nm}(x)\phi_m(x)
\end{align*}
を施して,新しい場$\tilde{\phi}(x)$がNGボゾン方向と直交させる,すなわち
\begin{align*}
\sum_{nm}\tilde{\phi}_n(x)(t_\alpha)_{nm}v_m=0
\end{align*}
とすることで消し去ることができる,ということを(19.6.4)で見た.(21.1.2)を満たすように場を回転(21.1.1)させた後では,NGボゾンモードは時空に依存するパラメータ$\gamma(x)$がNGボゾン場として再登場した.((19.6.12)参照,そのときはNGボゾン場として出てくるのは具体的に$\xi(x)$だった.)この手続きの重要な点は,時空的に定数の$\gamma(x)=\gamma$変換(21.1.1)のもとではラグランジアンは(大域$G$不変なので)不変なので,$\gamma(x)$依存性は全て\uwave{$\gamma(x)$に微分がかかる場合を除いて消える},ということだった.ここまでが19章の復習だ.

\vskip\baselineskip

他方,ラグランジアンが\uwave{定数の$G$変換}ばかりでなく,\uwave{時空点に依存する$G$変換}のもとでも不変ならば,変換(21.1.1)は理論の真の対称性であり,$\phi_n\to\tilde{\phi}_n$で置き換えるだけなので全ての$\gamma(x)$依存性はラグランジアンから消える!(つまり(19.5.6)のような新しい項は出てこない.)\par
$\Rightarrow$これは,ゲージ場自身に条件(ゲージ条件)を課して$A^\mu\to \tilde{A}^\mu$($\tilde{A}^\mu$はゲージ条件を満たすゲージ場)という再定義をする代わりに,$\phi\to\tilde{\phi}$($\tilde{\phi}$は条件(21.1.2)を満たすスカラー場)という再定義をすることに相当.\par
$\Rightarrow$(21.1.2)で再定義されるゲージは「ユニタリー・ゲージ」と呼ばれる.(8.2節参照)

\vskip\baselineskip

(21.1.2)より,ラグランジアンに現れる$\phi_n$はNGボゾンモードのない$\tilde{\phi}_n$に置き換えられるので,ユニタリーゲージではNGボゾン場は現れない.理論はゲージ不変なので,(ゲージ変換で自分の好きなゲージに移れるから)全てのゲージ条件において同様に物理的なNGボゾン場は存在しない!\par
$\Rightarrow$ベクトル・ボゾン($A^\mu_\alpha$の表す粒子)についてはどうだろうか?\par
もし$\phi_n$が正準規格化された基本的なスカラー場なら,ラグランジアンは
\begin{align*}
\mc{L}_\phi=-\frac{1}{2}\sum_n \mc{D}_\mu\phi_n \mc{D}^\mu\phi_n&=-\frac{1}{2}\sum_n \mc{D}_\mu\tilde{\phi}_n \mc{D}^\mu\tilde{\phi}_n \\
&=-\frac{1}{2}\sum_n\left\{ \partial_\mu \tilde{\phi}_n - i\sum_{m,\alpha}t^\alpha_{nm}A_{\alpha\mu}\tilde{\phi}_m \right\}^2
\end{align*}
の項を含む.ここで$t_\alpha$はゲージ群$G$の全ての生成子を動く.$\phi$と$\tilde{\phi}$は等価なので,今後チルダは省略し,この節では\uwave{常にユニタリーゲージ(21.1.2)を採用している}とする.対称性$G$は$\phi_n$の真空期待値$v_m$によってやぶれていると仮定しているので,粒子のスペクトルの性質を見るため,ずらした場
\begin{align*}
\phi_n=v_n+\phi'_n
\end{align*}
を定義する.(21.1.3)を$\phi'とA$について二次まで展開すると,
\begin{align*}
&\mc{L}_\phi=-\frac{1}{2}\sum_n\left\{ \partial_\mu \phi'_n - i\sum_{m,\alpha}t^\alpha_{nm}A_{\alpha\mu}(\phi'_m+v_m) \right\}^2 \\
\Rightarrow \quad &\mc{L}_{\phi,\mathrm{QUAD}}=-\frac{1}{2}\sum_n\left\{ \partial_\mu \phi'_n - i\sum_{m,\alpha}t^\alpha_{nm}A_{\alpha\mu}v_m \right\}^2
\end{align*}
(21.1.2)を微分すると,条件
\begin{align*}
0=\sum_{nm}\partial_\mu\phi_n(x)t^\alpha_{nm}v_m=\sum_{nm}\partial_\mu\phi'_n(x)t^\alpha_{nm}v_m
\end{align*}
を得るので,$\phi'$と$A$の交差項は消えて
\begin{align*}
\mc{L}_{\phi,\mathrm{QUAD}}&=-\frac{1}{2}\sum_n \partial_\mu\phi'_n(x)\partial^\mu \phi'_n(x)+i\sum_{nm\alpha}\uwave{\partial_\mu\phi'_n(x)t^\alpha_{nm}v_m}A^\mu_\alpha +\frac{1}{2}\sum_{nm\ell \alpha\beta} t^\alpha_{nm}v_mt^\beta_{n\ell}v_{\ell}A_{\alpha\mu}A_{\beta}^{\mu} \\
&=-\frac{1}{2}\sum_n \partial_\mu\phi'_n(x)\partial^\mu \phi'_n(x)+\frac{1}{2}\sum_{nm\ell \alpha\beta} t^\alpha_{nm}v_mt^\beta_{n\ell}v_{\ell}A_{\alpha\mu}A_{\beta}^{\mu} \\
&=-\frac{1}{2}\sum_n \partial_\mu\phi'_n(x)\partial^\mu \phi'_n(x)-\frac{1}{2}\sum_{\alpha\beta}\mu^2_{\alpha\beta}A_{\alpha\mu}A_{\beta}^{\mu}
\end{align*}
ここで
\begin{align*}
\mu^2_{\alpha\beta}\equiv -\sum_{nm\ell}t^\alpha_{nm}t^{\beta}_{n\ell}v_mv_{\ell}
\end{align*}
と定義した.これとヤン・ミルズラグランジアン$-\frac{1}{4}F_{\alpha\mu\nu}F_{\alpha}^{\mu\nu}$の二次の項を合わせると,ベクトル粒子は質量行列$\mu^2_{\alpha\beta}$をもつ.

\vskip\baselineskip

$\mu^2_{\alpha\beta}$の代数的な性質を見る.$t^\alpha_{nm}$は虚:$(t^{\alpha}_{nm})^*=-t^\alpha_{nm}$で反対称:$(t^{\alpha}_{nm})^T=t^\alpha_{mn}=-t^\alpha_{nm}$(したがってエルミート:$(t^\alpha_{nm})^\dagger=t^\alpha_{nm}$)なので,行列$\mu^2_{\alpha\beta}$は
\begin{align*}
\mu^2_{\alpha\beta}&=-\sum_{nm\ell}t^\alpha_{nm}t^{\beta}_{n\ell}v_mv_{\ell}=-\sum_{n}(t^\alpha v)_{n}(t^\beta v)_{n}  \\
&=\sum_{nm\ell}v_m t^\alpha_{mn} t^\beta_{n\ell}v_\ell
\end{align*}
などのように書き換えることができることに留意しておけば,
\begin{align*}
(\mu^2_{\alpha\beta})^*=-\sum_{nm\ell}\left(t^\alpha_{nm}t^\beta_{n\ell}v_m v_\ell\right)^*&=-\sum_{nm\ell}\left(t^\alpha_{nm}\right)^*(t^\beta_{n\ell})^*v_m v_\ell \quad \because vは実場 \\
&=-\sum_{nm\ell}t^\alpha_{nm}t^\beta_{n\ell}v_m v_\ell=\mu^2_{\alpha\beta}
\end{align*}
となって実がわかる.また
\begin{align*}
\mu^2_{\beta\alpha}=\sum_{nm\ell}v_mt^\beta_{mn}t^\alpha_{n\ell}v_\ell &=\sum_{nm\ell}v_\ell t^{\alpha}_{n\ell}t^\beta_{mn}v_m \quad \because t^\alpha_{nm} は成分なので入れ替えて良い \\
&=\sum_{nm\ell}v_\ell t^\alpha_{\ell n} t^\beta_{nm}v_m =\mu^2_{\alpha\beta}
\end{align*}
となって対称がわかる.また任意の非ゼロのベクトル$c_{\alpha}$に対して($it^\alpha$が実行列であるから)
\begin{align*}
\sum_{\alpha\beta}c_{\alpha}\mu^2_{\alpha\beta}c_{\beta}&=\sum_{n}\left\{\sum_{m\alpha} c_{\alpha} ( it^\alpha_{nm})v_m \right\} \left\{\sum_{\ell\beta}c_{\beta}( it^\beta_{n\ell}) v_\ell \right\} \\
&=\sum_{n}\left\{ \sum_{m\alpha}c_{\alpha} ( it^\alpha_{nm})v_m \right\}^2\geq0 \quad \because Gは破れていて(t^a v)_n\neq 0\quad(a は G/H の添え字)
\end{align*}
となって正行列がわかる.以上より$\mu^2_{\alpha\beta}$は実・対称・正行列だ.\par
また,生成子の実数の線型結合$t'^i=\sum_{\alpha}c^i_\alpha t^\alpha$で与えられる生成子が,破れていない群の生成子(例えば,$SU(2)\times U(1)\to U(1)_{em}においての,U(1)_{em}$の生成子$q$(21.3.9)のような)であるとき,破れていない群$H$について(19.6.2)と同様
\begin{align*}
0=\sum_{m}t'^i_{nm} v_m=\sum_{m}(c^i_\alpha t^\alpha)_{nm} v_m
\end{align*}
となる.この場合(21.1.7)より
\begin{align*}
\sum_{\beta}\mu^2_{\alpha\beta}c_\beta^i=\sum_{nm\ell\beta}v_m t^\alpha_{mn}(c^i_{\beta} t^{\beta}_{nm}v_m)=0
\end{align*}
すなわち,$A_{\alpha\mu}$を
\begin{align*}
A_{\alpha\mu}=\sum_i c^i_\alpha A'_{i\mu}+\cdots
\end{align*}
と展開すれば
\begin{align*}
\sum_\alpha t^\alpha_{nm}A_{\alpha\mu}&=\sum_{\alpha i}t^\alpha_{nm}c^{i}_{\alpha}A'_{i\mu}+\cdots \\
&=\sum_{i}t'^i_{nm} A'_{i\mu}+\cdots
\end{align*}
となり,$A'_{i\mu}$が破れていない生成子に結合するゲージ場となり,(21.1.6)の第二項目から発生する$A'_{i\mu}$の係数はゼロであることがわかる.よって$A'^{i\mu}$は質量ゼロとわかる.逆もまた正しい.(21.1.7)より任意の実定数$c^\beta_\alpha$について
\begin{align*}
\sum_{\alpha\beta}c^i_{\alpha}\mu^2_{\alpha\beta}c^i_{\beta}=\sum_{n}\left\{ \sum_{m\alpha}c^\gamma_{\alpha} ( it^\alpha_{nm})v_m \right\}^2\geq0
\end{align*}
($i$については和はとっていない.$t'^i$に結合するゲージ場の二次の項の振る舞いが知りたいからだ.)左辺がゼロになるのは$c^i_\alpha$が(21.1.8)を満たすときだけだ.\par
特に,破れていない対称性の生成子が$\sum_\alpha c_\alpha t^\alpha$の一個だけならば,一般のゲージ場は
\begin{align*}
A^\mu_\alpha=c_\alpha A^\mu+\cdots
\end{align*}
と書くことができる.($A^\mu$が所謂,後に出てくる質量ゼロの「電磁場」だ.)ここで「$\cdots$」は明確にゼロでない質量をもつ,ゲージ場の線型結合を表す.また$c_\alpha$は唯一の破れていない生成子$q$
\begin{align*}
q=\sum_\alpha c_\alpha t^\alpha
\end{align*}
の$t^\alpha$の係数なので,質量項$-\frac{1}{2}\sum_{\alpha\beta}\mu^2_{\alpha\beta}A^\mu_\alpha A_{\mu\beta}$のなかで場$A^\mu$の係数はゼロだ.ゲージ場のラグランジアンの運動項を展開したとき
\begin{align*}
-\frac{1}{4}\sum_\alpha(\partial_\mu A_{\alpha\nu}-\partial_\nu A_{\alpha\mu})^2 &=-\frac{1}{4}\sum_\alpha(\partial_\mu (c_\alpha A_\nu)-\partial_\nu (c_\alpha A_\mu)+\cdots)^2 \\
&=-\frac{1}{4}\sum_\alpha c_\alpha^2(\partial_\mu A_{\nu}-\partial_\nu A_{\mu})^2+\cdots
\end{align*}
となるが,第一項目の係数が正準値1になるためには,$c_\alpha$は
\begin{align*}
\sum_\alpha c^2_\alpha=1
\end{align*}
と規格化される必要がある.$q$は
\begin{align*}
\sum_{\alpha}t^\alpha A^\mu_\alpha=qA^\mu+\cdots
\end{align*}
の意味で$A^\mu$が結合する荷電だ.これは共変微分が
\begin{align*}
\mc{D}_\mu&=\partial_\mu-i\sum_\alpha t^\alpha A_{\alpha\mu}\\
&=\partial_\mu-iqA_\mu-\cdots
\end{align*}
の形となって,QEDの共変微分(8.1.21)と見比べれば第二項目の$q$が荷電であると理解できる.これらの一般的な結果は21.3節で電弱理論を調べる際に使う.

\vskip\baselineskip

以上の結果はここではゲージ群の実表現を形成するスカラー場について導いたが,複素表現に対して適切な形に書き換えることができる.19.6節で見たように,エルミートな生成子$T^\alpha$をもつゲージ群の表現に従って変換する複素スカラー場$\chi(x)$は,実場の組
\begin{align*}
\phi(x)=\left(
\begin{array}{cc}
\mr{Re}\chi(x) \\
\mr{Im}\chi(x)
\end{array}
\right)
\end{align*}
として記述できる.この場の組は生成子
\begin{align*}
it^\alpha=\left(
\begin{array}{cc}
-\mr{Im}T^\alpha & -\mr{Re}T^\alpha \\
\mr{Re}T^\alpha & -\mr{Im}T^\alpha
\end{array}
\right)
\end{align*}
をもつゲージ場の実表現をなす.19章の復習をしておくと,実際$\epsilon_\alpha$を微小パラメータとして
\begin{align*}
\chi(x)&=\mr{Re}\chi(x)+i\mr{Im}\chi(x)=(1,\ i)\left(
\begin{array}{cc}
\mr{Re}\chi(x) \\
\mr{Im}\chi(x)
\end{array}
\right)=(1,\ i)\phi(x)\\
\to \chi'(x)&=\left(1+i\sum_\alpha\epsilon_\alpha T^\alpha\right)\chi(x) \\
&=\left(\mr{Re}\chi(x)+i\mr{Im}\chi(x)\right)+i\sum_\alpha \epsilon_\alpha (\mr{Re}T^\alpha+i\mr{Im}T^\alpha)(\mr{Re}\chi(x)+i\mr{Im}\chi(x)) \\
&=(\mr{Re}\chi(x)+i\mr{Im}\chi(x))+i\sum_\alpha \epsilon_\alpha(\mr{Re}T^\alpha \mr{Re}\chi(x)-\mr{Im}T^\alpha \mr{Im}\chi(x)) \\
&\ \qquad\qquad\qquad\qquad\qquad -\sum_\alpha \epsilon_\alpha(\mr{Im}T^\alpha \mr{Re}\chi(x)+\mr{Re}T^\alpha \mr{Im}\chi(x)) \\
&=\left\{\mr{Re}\chi(x)-\sum_\alpha \epsilon_\alpha(\mr{Im}T^\alpha \mr{Re}\chi(x)+\mr{Re}T^\alpha \mr{Im}\chi(x))\right\} \\
&+i\left\{\mr{Im}\chi(x)+\sum_\alpha \epsilon(\mr{Re}T^\alpha \mr{Re}\chi(x)-\mr{Im}T^\alpha \mr{Im}\chi(x))\right\} \\
&=(1,\ i)\left[ \left(
\begin{array}{cc}
\mr{Re}\chi(x) \\
\mr{Im}\chi(x)
\end{array}
\right)+\sum_\alpha \epsilon_\alpha\left(
\begin{array}{cc}
-\mr{Im}T^\alpha & -\mr{Re}T^\alpha \\
\mr{Re}T^\alpha & -\mr{Im}T^\alpha
\end{array}
\right)\left(
\begin{array}{cc}
\mr{Re}\chi(x) \\
\mr{Im}\chi(x)
\end{array}
\right)
 \right] \\
&=(1,\ i)\left(1+i\sum_\alpha\epsilon_\alpha t^\alpha\right)\phi(x)
\end{align*}
となって,
\begin{align*}
&\chi(x)\to\chi'(x)=\left(1+i\sum_\alpha\epsilon_\alpha T^\alpha\right)\chi(x) \\
\Leftrightarrow \quad &\phi(x)\to \phi'(x)=\left(1+i\sum_\alpha\epsilon_\alpha t^\alpha\right)\phi(x)
\end{align*}
という対応が確認できる.(21.1.15)(21.1.16)を(21.1.7)に代入する.
\begin{align*}
\mr{Re}(XY)=\mr{Re}X\mr{Re}Y-\mr{Im}X\mr{Im}Y,\quad \mr{Im}(XY)=\mr{Re}X\mr{Im}Y+\mr{Im}X\mr{Re}Y
\end{align*}
に留意すると
\begin{align*}
\mu^2_{\alpha\beta}&=-\sum_{nm\ell}v_m(it^\alpha)_{mn}(it^\beta)_{n\ell}v_\ell \\
&=-(\mr{Re}\braket{\chi},\ \mr{Im}\braket{\chi})\left(
\begin{array}{cc}
-\mr{Im}T^\alpha & -\mr{Re}T^\alpha \\
\mr{Re}T^\alpha & -\mr{Im}T^\alpha
\end{array}
\right)\left(
\begin{array}{cc}
-\mr{Im}T^\beta & -\mr{Re}T^\beta \\
\mr{Re}T^\beta & -\mr{Im}T^\beta
\end{array}
\right)\left(
\begin{array}{cc}
\mr{Re}\braket{\chi} \\
\mr{Im}\braket{\chi}
\end{array}
\right) \\
&=-(\mr{Re}\braket{\chi},\ \mr{Im}\braket{\chi})\left(
\begin{array}{cc}
-\mr{Im}T^\alpha & -\mr{Re}T^\alpha \\
\mr{Re}T^\alpha & -\mr{Im}T^\alpha
\end{array}
\right)\left(
\begin{array}{cc}
-\mr{Im}T^\beta \mr{Re}\braket{\chi} -\mr{Re}T^\beta \mr{Im}\braket{\chi}\\
\mr{Re}T^\beta \mr{Re}\braket{\chi} -\mr{Im}T^\beta \mr{Im}\braket{\chi}
\end{array}
\right) \\
&=-(\mr{Re}\braket{\chi},\ \mr{Im}\braket{\chi})\left(
\begin{array}{cc}
-\mr{Im}T^\alpha & -\mr{Re}T^\alpha \\
\mr{Re}T^\alpha & -\mr{Im}T^\alpha
\end{array}
\right)\left(
\begin{array}{cc}
-\mr{Im}(T^\beta\braket{\chi}) \\
\mr{Re}(T^\beta\braket{\chi})
\end{array}
\right) \\
&=-(\mr{Re}\braket{\chi},\ \mr{Im}\braket{\chi})\left(
\begin{array}{cc}
\mr{Im}T^\alpha \mr{Im}(T^\beta\braket{\chi}) -\mr{Re}T^\alpha \mr{Re}(T^\beta\braket{\chi}) \\
-\mr{Re}T^\alpha \mr{Im}(T^\beta\braket{\chi}) -\mr{Im}T^\alpha \mr{Re}(T^\beta\braket{\chi})
\end{array}
\right)\\
&=-(\mr{Re}\braket{\chi},\ \mr{Im}\braket{\chi})\left(
\begin{array}{cc}
-\mr{Re}(T^\alpha T^\beta \braket{\chi}) \\
-\mr{Im}(T^\alpha T^\beta \braket{\chi})
\end{array}
\right)\\
&=-(\mr{Re}\braket{\chi}^\dagger,\ -\mr{Im}\braket{\chi}^\dagger)\left(
\begin{array}{cc}
-\mr{Re}(T^\alpha T^\beta \braket{\chi}) \\
-\mr{Im}(T^\alpha T^\beta \braket{\chi})
\end{array}
\right) \quad\because \mr{Re}(z^*)=\mr{Re}(z),\quad \mr{Im}(z^*)=-\mr{Im}(z) \\
&=\mr{Re}\braket{\chi}^\dagger \mr{Re}(T^\alpha T^\beta \braket{\chi}) -\mr{Im}\braket{\chi}^\dagger\mr{Im}(T^\alpha T^\beta \braket{\chi}) \\
&=\mr{Re}(\braket{\chi}^\dagger, T^\alpha T^\beta \braket{\chi}) \\
&=\frac{1}{2}\left\{ (\braket{\chi}^\dagger, T^\alpha T^\beta \braket{\chi}) + (\braket{\chi}^\dagger, T^\alpha T^\beta \braket{\chi})^* \right\} \quad \because \mr{Re}(z)=\frac{z+z^*}{2} \\
&=\frac{1}{2}\left\{ (\braket{\chi}^\dagger, T^\alpha T^\beta \braket{\chi}) + (\braket{\chi}^\dagger, T^\beta T^\alpha \braket{\chi}) \right\} \quad \because T^\alpha はエルミート \\
&=\frac{1}{2} (\braket{\chi}^\dagger, \{T^\alpha , T^\beta\} \braket{\chi})
\end{align*}
というベクトル・ボゾンの質量行列が得られる.

\vskip\baselineskip

ベクトル場のプロパゲータを見る.ヤン・ミルズラグランジアンの二次の項も含めて,全ラグランジアンの$A$について二次の項は
\begin{align*}
-&\frac{1}{4}\sum_\alpha (\partial_\lambda A_{\alpha\nu}-\partial_\nu A_{\alpha\lambda})^2-\frac{1}{2}\sum_{\alpha\beta}\mu^2_{\alpha\beta}A_{\alpha\lambda}A^\lambda_{\beta} \\
&=-\frac{1}{4}\sum_\alpha (\partial_\lambda A_{\alpha\nu}\partial^\lambda A^\nu_\alpha- \partial_\lambda A_{\alpha\nu}\partial^\nu A^\lambda_\alpha + \partial_\nu A_{\alpha\lambda}\partial^\nu A^\lambda_\alpha -\partial_\nu A_{\alpha\lambda}\partial^\lambda A^\nu_\alpha) -\frac{1}{2}\sum_{\alpha\beta}\mu^2_{\alpha\beta}A_{\alpha\lambda}A^\lambda_{\beta} \\
&=-\frac{1}{2}\sum_\alpha(\partial_\nu A_{\alpha\lambda}\partial^\nu A^\lambda_\alpha-\partial_\nu A_{\alpha\lambda}\partial^\lambda A^\nu_\alpha) -\frac{1}{2}\sum_{\alpha\beta}\mu^2_{\alpha\beta}A_{\alpha\lambda}A^\lambda_{\beta} \\
&=\frac{1}{2}\sum_\alpha(A_{\alpha\lambda}\partial_\nu \partial^\nu A^\lambda_\alpha-A_{\alpha\lambda}\partial_\nu \partial^\lambda A^\nu_\alpha) -\frac{1}{2}\sum_{\alpha\beta}\mu^2_{\alpha\beta}A_{\alpha\lambda}A^\lambda_{\beta} \\
&\qquad -\frac{1}{2}\sum_\alpha\partial_\nu(A_{\alpha\lambda} \partial^\nu A^\lambda_\alpha-A_{\alpha\lambda} \partial^\lambda A^\nu_\alpha) \qquad \leftarrow 全微分項 \\
&=\frac{1}{2}\sum_\alpha(A^\nu_{\alpha}\eta_{\nu\lambda} \Box A^\lambda_\alpha-A^\nu_{\alpha}\partial_\nu \partial_\lambda A^\lambda_\alpha) -\frac{1}{2}\sum_{\alpha\beta}\mu^2_{\alpha\beta}A_{\alpha\lambda}A^\lambda_{\beta}+[全微分項] \\
&=-\frac{1}{2}\sum_{\alpha\beta}A^\nu_\alpha \mc{D}_{\alpha\nu,\beta\lambda}(\partial)A^\lambda_\beta+[全微分項]
\end{align*}
ここで
\begin{align*}
\mc{D}_{\alpha\nu,\beta\lambda}(\partial)=-\delta_{\alpha\beta}[\eta_{\nu\lambda}\Box-\partial_\nu \partial_\lambda]+\mu^2_{\alpha\beta}\eta_{\nu\lambda}
\end{align*}
簡単のため全てのゲージ対称性は破れており,したがって$\mu^2_{\alpha\beta}$はゼロ固有値をもたない,とする.(9.4.15)からの議論と同様にして,運動量空間でのゲージ場のプロパゲータは
\begin{align*}
\Delta_{\alpha\nu,\beta\lambda}(k)&=(\mc{D}^{-1})_{\alpha\nu,\beta\lambda}(ik) \\
&=[(k^2+\mu^2)^{-1}(\eta_{\nu\lambda}+\mu^{-2}k_\nu k_\lambda)]_{\alpha\beta}
\end{align*}
となる.実際
\begin{align*}
\sum_{\beta\nu}\mc{D}_{\alpha\mu,\beta\nu}(ik)\Delta_{\beta,\gamma}^{\nu,\lambda}(k)&=\sum_{\beta\nu}\left[\delta_{\alpha\beta}[\eta_{\mu\nu}k^2-k_\mu k_\nu]+\mu^2_{\alpha\beta}\eta_{\mu\nu}\right] [(k^2+\mu^2)^{-1}(\eta^{\nu\lambda}+\mu^{-2}k^\nu k^\lambda)]_{\beta\gamma} \\
&=\sum_{\beta\nu}\left[\eta_{\mu\nu}[k^2+\mu^2]_{\alpha\beta} -\delta_{\alpha\beta}k_\mu k_\nu\right] [(k^2+\mu^2)^{-1}(\eta^{\nu\lambda}+\mu^{-2}k^\nu k^\lambda)]_{\beta\gamma} \\
&=\sum_{\nu}\eta_{\mu\nu}(\eta^{\nu\lambda}-\mu^{-2}k^\nu k^\lambda)_{\alpha\gamma}+[(k^2+\mu^2)^{-1}(k_\mu k^\lambda+\mu^{-2}k_\mu k^\lambda k^2)]_{\alpha\gamma} \\
&=(\delta^\lambda_\mu-\-\mu^{-2}k_\mu k^\lambda)_{\alpha\gamma}+[(k^2+\mu^2)^{-1}(\mu^2k_\mu k^\lambda+k_\mu k^\lambda k^2)\mu^{-2}]_{\alpha\gamma} \\
&=(\delta^\lambda_\mu-\mu^{-2}k_\mu k^\lambda)_{\alpha\gamma}+(k_\mu k^\lambda \mu^{-2})_{\alpha\gamma} \\
&=\delta^\lambda_\mu \delta_{\alpha\gamma}
\end{align*}
となって,逆行列であることがわかる.

\vskip\baselineskip

$A,C$がゲージ代数のある表現に属し,$B,D$が別の表現に属する場合,散乱過程$A+B\to C+D$の行列要素にはベクトルボゾン交換による,
\begin{align*}
S_{CD,AB}=i(2\pi)^4 \delta^4(p_A+p_B-p_C-p_D)\bra{C}J^\nu_{N\alpha}\ket{A}\Delta_{\alpha \nu ,\beta\lambda}\bra{D}J^\lambda_{N\beta}\ket{B}
\end{align*}
の寄与がある.ここで$k=p_A-p_C=p_D-p_B$で,$J^\nu_{N\alpha}$はゲージボゾンが結合するカレントだ.添え字$N$は,このゲージではカレントのNGボゾンの極を省略していることを意味する.(すなわち,カレントに対応する頂点が外線についていない,ということだ.3巻p236参照.よって(19.2.49)がこのカレントの行列要素の寄与となる.)このカレントは,ゲージ結合定数に比例するので((19.2.38)参照),ゲージ結合定数がゼロの極限で残る(21.1.21)の唯一の項は,この極限で発散する$\mu^{-2}$を含む項だ.すなわち,(21.1.20)を用いて
\begin{align*}
S_{CD,AB}&=i(2\pi)^4 \delta^4(p_A+p_B-p_C-p_D)\bra{C}{J_N}^\nu_\alpha\ket{A}[(k^2+\mu^2)^{-1}(\eta_{\nu\lambda}+\mu^{-2}k_\nu k_\lambda)]_{\alpha\beta}\bra{D}{J_N}^\lambda_\beta\ket{B} \\
&=i(2\pi)^4 \delta^4(p_A+p_B-p_C-p_D)\bra{C}{J_N}^\nu_{\alpha}\ket{A}[(k^2+\mu^2)^{-1}\mu^{-2}(\eta_{\nu\lambda}\mu^2+k_\nu k_\lambda)]_{\alpha\beta}\bra{D}{J_N}^\lambda_{\beta}\ket{B} \\
&\to i(2\pi)^4 \delta^4(p_A+p_B-p_C-p_D)k_\nu k_\lambda(\mu^{-2})_{\alpha\beta}\bra{C}{J_N}^\nu_{\alpha}\ket{A}\frac{1}{k^2}\bra{D}{J_N}^\lambda_{\beta}\ket{B} 
\end{align*}
となる.カレントのゲージ結合定数は,行列$\mu^2_{\alpha\beta}$のゲージ結合定数と相殺する.

\begin{figure}[H]
  \centering
\begin{tikzpicture}[decoration={markings, 
mark= at position -1cm with {\arrow[line width=0.5mm]{Stealth}}}]
\coordinate (b1) at (-2,2){};
\coordinate (b2) at (-2,-2){};
\coordinate (m1) at (0,1){};
\coordinate (a1) at (2,2){};
\coordinate (a2) at (2,-2){};
\coordinate (m2) at (0,-1){};
\draw[thick,postaction={decorate}](b1)node[left]{$e$}node[above right]{$p_A$}--(m1);
\draw[thick,postaction={decorate}](m1)--(a1)node[right]{$e$}node[above left]{$p_C$};

\draw[thick,postaction={decorate}](b2)node[left]{$\mu$}node[below right]{$p_B$}--(m2);
\draw[thick,postaction={decorate}](m2)--(a2)node[right]{$\mu$}node[below left]{$p_D$};

\begin{feynhand}
\propag[photon,thick](m1)--(m2);
\end{feynhand}

\node(q) at (-0.3,0){$k$};
    
\end{tikzpicture}
\end{figure}

(21.1.3)から,ゲージ場$A_\alpha^\mu$に結合するのは$J^\mu_\alpha=i\partial_\mu \phi'_n t^\alpha_{nm}\phi'_m$の形であることがわかる.$t^\alpha$がゲージ結合定数に比例しているのだ.\par
逆に,NGボゾンを交換する行列要素は,(19.2.49)より
\begin{align*}
S_{CD,AB}&=i(2\pi)^4 \delta^4(p_A+p_B-p_C-p_D)\bra{C}{J_N}^\nu_\alpha\ket{A}(iF^{-1}_{\alpha\gamma}k_\nu)\frac{1}{k^2}(-iF^{-1}_{\gamma\beta}k_\lambda)\bra{D}{J_N}^\lambda_\beta\ket{B} \\
&=i(2\pi)^4 \delta^4(p_A+p_B-p_C-p_D)k_\nu k_\lambda F_{\alpha\gamma}^{-1}F^{-1}_{\beta\gamma}\bra{C}{J_N}^\nu_\alpha\ket{A}\frac{1}{k^2}\bra{D}{J_N}^\lambda_\beta\ket{B}
\end{align*}
一方
\begin{align*}
\mu^2_{\alpha\beta}=\sum_n F_{\alpha\gamma}Z_{\gamma n} F_{\beta\delta}Z_{\delta n}=F_{\alpha\gamma}F_{\beta\gamma}
\end{align*}
となり,(21.1.22)と(21.1.24)は等しくなる.すなわちゲージ結合定数がゼロにおいて$S$行列要素は連続である.

\newpage

\subsection{くりこみ可能な$\xi$ゲージ}
一般のゲージでの理論のスカラー場についてのラグランジアンの運動項(21.1.3)は交差項
\begin{align*}
i\partial_\mu \phi'_n (t^\alpha)_{nm}A^\mu_\alpha v_m
\end{align*}
を含む.ユニタリーゲージではこの項はゲージ条件(21.1.2)の結果ゼロとなった.ここでは,15.5節と15.6節の方法に類似の別の方法を採用する.経路積分に汎関数$B[f]$(15.5.22)
\begin{align*}
B[f]=\exp \left( \frac{-i}{2\xi}\int d^4 x \sum_\alpha f_\alpha f_\alpha \right)
\end{align*}
を導入する.これはラグランジアンにゲージ固定項
\begin{align*}
\mc{L}_{gf}=-\frac{1}{2\xi}\sum_\alpha f_\alpha f_\alpha
\end{align*}
を付け加えるのと同等だ.15.5節では$f_\alpha=\partial_\mu A^\mu_\alpha$ととったが,ここではゲージ固定関数$f_\alpha$を
\begin{align*}
f_\alpha=\partial_\mu A^\mu_\alpha -i\xi(t_\alpha)_{nm}\phi'_n v_m
\end{align*}
にとる.これにより
\begin{align*}
-\frac{1}{2\xi}f_\alpha f_\alpha&=-\frac{1}{2\xi} \left[ (\partial_\mu A^\mu_\alpha)^2-2i\xi\phi'_n(t_\alpha)_{nm} v_m \partial_\mu A^\mu_\alpha -\xi^2 (\phi'_n(t_\alpha)_{nm} v_m)^2 \right] \\
&=-\frac{1}{2\xi} (\partial_\mu A^\mu_\alpha)^2-\frac{\xi}{2} (\phi'_n(it_\alpha)_{nm} v_m)^2+\uwave{i\phi'_n(t_\alpha)_{nm}  \partial_\mu A^\mu_\alpha}v_m
\end{align*}
交差項が(21.2.2)の交差項と合わさって全微分項となって消える!ここではユニタリーゲージは特殊な場合となっている.すなわち,$\xi\to\infty$でゲージ固有汎関数(21.2.1)は第二項目だけ残って,ユニタリーゲージ条件(21.1.2)
\begin{align*}
0=\sum_{nm}\tilde{\phi}_n(x)(t_\alpha)_{nm}v_m &=\sum_{nm}v_n (t_\alpha)_{nm}v_m +\sum_{nm}\phi'_n(x)(t_\alpha)_{nm}v_m \\
&=\sum_{nm}\phi'_n(x)(t_\alpha)_{nm}v_m \quad \because t_\alpha は反対称
\end{align*}
を満たす$\phi'$で無限に鋭いビークをもつ.もう一つの特殊な場合は,$\xi\to0$の極限んで,この場合はゲージ固定汎関数はランダウゲージ条件$\partial_\mu A^\mu_\alpha=0$を満たすゲージ場でビークをもつ.

\vskip\baselineskip

スカラー場のラグランジアンに,ゲージ不変性の条件(19.3.2)
\begin{align*}
\frac{\partial P(\phi)}{\partial \phi_n}(t_\alpha)_{nm}\phi_m=0
\end{align*}
を満たす4次の多項式項$-P(\phi)$も含める.もちろんゲージ場の項
\begin{align*}
\mc{L}_A=-\frac{1}{4}\sum_\alpha F^{\mu\nu}_\alpha F_{\alpha \mu\nu}
\end{align*}
も含める.すると,ゲージ場とスカラー場の全ラグランジアンは
\begin{align*}
\mc{L}_{A,\phi}=&\mc{L}_A +\mc{L}_\phi +\mc{L}_{gf} \\
=&-\frac{1}{4}\sum_\alpha F^{\mu\nu}_\alpha F_{\alpha \mu\nu}-\frac{1}{2} \sum_n\left( \partial_\mu \phi_n - i\sum_{m,\alpha}t^\alpha_{nm}A_{\alpha\mu}\phi_m \right)^2-P(\phi) -\frac{1}{2\xi}\sum_\alpha f_\alpha f_\alpha \\
=&-\frac{1}{4}\sum_\alpha F^{\mu\nu}_\alpha F_{\alpha \mu\nu}-\frac{1}{2}\sum_n \partial_\mu \phi'_n \partial^\mu \phi'_n +\uwave{i\sum_{\alpha n m}\partial_\mu \phi'_n (t_\alpha)_{nm}A^\mu_\alpha (\phi'_m+v_m)} \\
&+\frac{1}{2}\sum_{n\alpha\beta}(t^\alpha \phi)_n(t^\beta \phi)_m A^\mu_\alpha A^\nu_\beta-\frac{1}{2\xi}\sum_\alpha (\partial_\mu A^\mu_\alpha )(\partial_\nu A^\nu_\alpha) -P(\phi) \\
&+\frac{\xi}{2}\sum_{\alpha nm}(t^\alpha v)_n (t^\alpha v)_m \phi'_n \phi'_m+ \uwave{i\sum_{\alpha nm}\phi'_n(t^\alpha)_{nm}\partial_\mu A^\mu_\alpha v_m} \\
=&-\frac{1}{4}\sum_\alpha F^{\mu\nu}_\alpha F_{\alpha \mu\nu}+\frac{1}{2}\sum_{n\alpha\beta}(t^\alpha \phi)_n(t^\beta \phi)_m A^\mu_\alpha A^\nu_\beta \\
&-\frac{1}{2\xi}\sum_\alpha (\partial_\mu A^\mu_\alpha )(\partial_\nu A^\nu_\alpha) -\frac{1}{2}\sum_n \partial_\mu \phi'_n \partial^\mu \phi'_n \\
&+\frac{\xi}{2}\sum_{\alpha nm}(t^\alpha v)_n (t^\alpha v)_m \phi'_n \phi'_m-P(\phi) \\
&+i\sum_{\alpha n m}\partial_\mu \phi'_n (t_\alpha)_{nm}A^\mu_\alpha \phi'_m +[全微分項]
\end{align*}
となる.\par
15.6節で見たように,ゲージ固定汎関数$B[f]$を導入すると,$f_\alpha$のゲージ変換性に依存するラグランジアンをもつゴースト場$\omega_\alpha(x)$を導入する必要がある.一般のゲージ変換
\begin{align*}
\delta A^\mu_\alpha=&\partial^\mu \epsilon_\alpha-\sum_{\beta \gamma}C_{\alpha\beta\gamma}\epsilon_\beta A^\mu_\gamma \quad \because(15.1.9)参照 \\
\delta \phi_n =&\delta \phi'_n= i\sum_{\alpha m}\epsilon_\alpha (t^\alpha)_{nm}\phi_m
\end{align*}
のもとで$f_\alpha$は
\begin{align*}
f_\alpha [A_\epsilon ,\phi_\epsilon ,x]=&f_\alpha(x) +\Box \epsilon_\alpha(x) -\sum_{\beta\gamma}C_{\alpha\beta\gamma}\partial_\mu(\epsilon_\beta(x) A^\mu_\gamma(x))+\xi \sum_{n\beta}(t^\alpha v)_n \epsilon_\beta(x)(t^\beta \phi(x))_n \\
=& f_\alpha(x)+\int \Box \delta^4(x-y)\epsilon_\alpha(y)d^4y-\int \sum_{\beta\gamma}C_{\alpha\beta\gamma}\partial_\mu(\delta^4(x-y) A^\mu_\gamma(x))\epsilon_\beta(y)d^4y \\
&+\int \xi \sum_{n\beta}(t^\alpha v)_n \delta^4(x-y) (t^\beta \phi(x))_n \epsilon_\beta(y) d^4y
\end{align*}
となる.よって(15.5.3)(15.6.2)より汎関数微分は
\begin{align*}
\mc{F}_{\alpha x,\beta y}=&\left. \frac{\delta f_\alpha [A_\epsilon ,\phi_\epsilon ,x]}{\delta \epsilon_\beta(y)}\right|_{\epsilon=0} \\
=&\Box \delta^4(x-y)\delta_{\alpha\beta}-\sum_{\gamma}C_{\alpha\beta\gamma}\partial_\mu(\delta^4(x-y) A^\mu_\gamma(x))+\xi \sum_{n}(t^\alpha v)_n \delta^4(x-y) (t^\beta \phi)_n \\
I_{GH}=&\sum_{\alpha\beta}\int d^4 x d^4 y \,\omega^*_\alpha(x)\mc{F}_{\alpha x,\beta y}\omega_\beta(y) \\
=&\sum_\alpha \int d^4x \, \omega_\alpha^*\left[ \Box\omega_\alpha -\sum_{\beta\gamma}C_{\alpha\beta\gamma}\partial_\mu(\omega_\beta A^\mu_\gamma)+\xi\sum_{n\beta}(t^\alpha v)_n \omega_\beta (t_\gamma \phi)_n \right] \\
\mc{L}_{\omega}=& \sum_\alpha \omega_\alpha^*\left[ \Box\omega_\alpha -\sum_{\beta\gamma}C_{\alpha\beta\gamma}\partial_\mu(\omega_\beta A^\mu_\gamma)+\xi\sum_{n\beta}(t^\alpha v)_n \omega_\beta (t^\beta \phi)_n \right]
\end{align*}
となる.\par
もし理論がスピン1/2のフェルミオンを含むと,一般的なくりこみ可能項
\begin{align*}
\mc{L}_\psi&=-\bar{\psi}(\Slash{\partial}-i\Slash{A}_\alpha t_\alpha^{(\psi)}+m_0 +\Gamma_n \phi_n)\psi \\
&=-\bar{\psi}(\Slash{\mc{D}}+m_0+\Gamma_n \phi_n)\psi
\end{align*}
が存在する.ここで$t^{(\psi)}_\alpha$はフェルミオンに関するゲージ群の生成子の行列表現(結合定数の因子を含む)で,$m_0$と$\Gamma_n$はゲージ不変性条件を満たす定数行列だ.すなわち
\begin{align*}
\delta \mc{L}_\psi=0 
\end{align*}
であり,$t^{\psi}_\alpha$がエルミートであることと$\bar{\psi}=\psi^\dagger \gamma_4(\gamma_4=i\gamma_0)$であることを用いると
\begin{align*}
&-\delta\bar{\psi}m_0 \psi -\bar{\psi}m_0\delta\psi=0 \\
\Rightarrow \quad &-\psi^\dagger(-it^{(\psi)}_\alpha \epsilon_\alpha)\gamma_4 m_0 \psi-\psi^\dagger \gamma_4 m_0(it_\alpha^{(\psi)}\epsilon_\alpha)\psi=0 \\
\Rightarrow \quad &\psi^\dagger[t^{(\psi)}_\alpha , \gamma_4 m_0]i\epsilon_\alpha\psi=0 \quad \Rightarrow \quad [t^{(\psi)}_\alpha , \gamma_4 m_0]=0
\end{align*}
かつ
\begin{align*}
&-\delta\bar{\psi}\Gamma_n \phi_n \psi -\bar{\psi}\Gamma_n \phi_n \delta \psi -\bar{\psi}\Gamma_n \delta \phi_n \psi=0 \\
\Rightarrow \quad &-\psi^\dagger(-it^{(\psi)}_\alpha \epsilon_\alpha)\gamma_4 \Gamma_n \phi_n \psi-\psi^\dagger \gamma_4 \Gamma_n \phi_n(it_\alpha^{(\psi)}\epsilon_\alpha)\psi-\sum_m\psi^\dagger \gamma_4 \Gamma_n i\epsilon_\alpha (t^\alpha)_{nm}\phi_m \psi=0 \\
\Rightarrow \quad &\psi^\dagger i\epsilon_\alpha [t^{(\psi)}_\alpha ,\gamma_4 \Gamma_n]\phi_n \psi - \psi^\dagger i\epsilon_\alpha \sum_m \gamma_4 \Gamma_m (t^\alpha)_{mn}\phi_n \psi=0 \\
\Rightarrow \quad & [t^{(\psi)}_\alpha ,\gamma_4 \Gamma_n]-\sum_m (t^\alpha)_{mn}\gamma_4 \Gamma_m=0
\end{align*}
が不変性条件となる.一般に$m_0$と$\Gamma_n$は,1あるいは$\gamma_5$に比例する項の線型結合となる.\par
15.5節と15.6節で証明した一般定理(どちらかといえば15.7節だが)より,(21.2.6)(21.2.10)(21.2.11)の和で与えられるラグランジアンから計算される$S$行列は$\xi$に依らず,任意の$\xi$について,つまり$\xi\to\infty$でユニタリーゲージに相当する結果が得られる.任意のパラメータ$\xi$は(15.7.25)(15.7.26)よりBRST変換の核のみに存在し,物理的な計算結果には顔を出さないからだ.

\vskip\baselineskip

これら全ての場のプロパゲータを導くには,ラグランジアンの場について二次の部分
\begin{align*}
\mc{L}_{\mr{QUAD}}=&-\frac{1}{4}\sum_\alpha(\partial^\mu A_\alpha^\nu -\partial^\nu A_\alpha^\mu)(\partial^\nu A_\alpha^\mu -\partial^\mu A_\alpha^\nu) \\
&-\frac{1}{2}\sum_{\alpha\beta}\mu^2_{\alpha\beta}A^\mu_\alpha A_{\beta\mu} -\frac{1}{2\xi}\sum_\alpha (\partial_\mu A^\mu_\alpha )(\partial_\nu A^\nu_\alpha) \\
&-\frac{1}{2}\sum_n (\partial_\mu \phi'_n)( \partial^\mu \phi'_n)-\frac{1}{2}\sum_{nm}M^2_{nm}\phi'_n \phi'_m \\
&-\bar{\psi}(\Slash{\partial}+m)\psi-\partial_\mu \omega^*_\alpha \partial^\mu\omega_\alpha-\xi\sum_{\alpha\beta}\mu^2_{\alpha\beta}\omega^*_\alpha \omega_\beta +[全微分項]
\end{align*}
が必要だ.ここで$\mu^2_{\alpha\beta}$はベクトル・ボゾンの質量行列(21.1.7)だ.
\begin{align*}
\mu^2_{\alpha\beta}=-\sum_n (t^\alpha v)_n (t^\beta v)_n
\end{align*}
また$M^2_{nm}$と$m$は新たなスカラーとフェルミオンの質量行列だ.
\begin{align*}
M^2_{nm}=&\left. \frac{\partial^2 P(\phi)}{\partial \phi_n \partial\phi_m }\right|_{\phi=v}-\xi\sum_\alpha (t^\alpha v)_n(t^\alpha v)_m \\
m=& m_0+\sum_n \Gamma_n v_n
\end{align*}
(21.2.14)よりゴーストの質量はゲージの選び方に依存し,それに対応するベクトル・ボゾンの質量と$\sqrt{\xi}$との積に等しい.\par
この次数では真空期待値$v_n$はちょうど多項式ポテンシャル$P(\phi)$の最小点だ.
\begin{align*}
\left. \frac{\partial P(\phi)}{\partial \phi_n}\right|_{\phi=v}=0
\end{align*}
また,(19.2.6)
\begin{align*}
\sum_m \left. \frac{\partial^2 P(\phi)}{\partial \phi_n \partial \phi_m} \right|_{\phi=v}(t^\alpha v)_m=0
\end{align*}
が全ての$\alpha$でなりたつ.したがって,質量ゼロのNGボゾンモードの代わりに(21.2.16)のスカラー・ボゾンの質量行列$M_{nm}^2$はゼロでないベクトル・ボゾンの質量$\mu$と$\sqrt{\xi}$との積に等しい.すなわち,$\mu^2_{\alpha\beta}$が固有値$\mu^2$の固有ベクトル$c_\beta$をもてば,$\sum_\beta c_\beta t^\beta v$は$M^2$の固有値$\xi \mu^2$の固有ベクトルだ.実際
\begin{align*}
\sum_m M^2_{nm}\left(\sum_\beta c_\beta t^\beta v\right)_m=&\sum_m \left[ \left. \frac{\partial^2 P(\phi)}{\partial \phi_n \partial\phi_m }\right|_{\phi=v}-\xi\sum_\alpha (t^\alpha v)_n(t^\alpha v)_m \right]\sum_\beta c_\beta (t^\beta v)_m \\
=&-\xi \sum_{m\alpha\beta}(t^\alpha v)_n \uwave{(t^\alpha v)_m(t_\beta v)_m} c_\beta \quad \because (21.2.28) \\
=&\xi \sum_{\alpha\beta}(t^\alpha v)_n \mu^2_{\alpha\beta}c_\beta \quad \because (21.2.15) \\
=&\xi \mu^2 \sum_\alpha c_\alpha(t^\alpha v)_n=\xi\mu^2\left( \sum_\beta c_\beta t_\beta v \right)_n
\end{align*}
プロパゲータを計算する.ラグランジアンの自由粒子の部分が(部分積分のあとに)複素場$\zeta$について$-\zeta^\dagger \mc{D}(\partial)\zeta$あるいは実場$\zeta$について$-\frac{1}{2}\zeta^T \mc{D}(\partial)\zeta$のとき,プロパゲータは$\mc{D}(ik)$だ.\par
$A$についての$\mc{D}(\partial)$は(21.1.19)と同様に$A$の二次の項
\begin{align*}
\frac{1}{2}\sum_\alpha(A^\nu_{\alpha}\eta_{\nu\lambda} \Box A^\lambda_\alpha-A^\nu_{\alpha}\partial_\nu \partial_\lambda A^\lambda_\alpha) -\frac{1}{2}\sum_{\alpha\beta}\mu^2_{\alpha\beta}A_{\alpha\lambda}A^\lambda_{\beta} +\frac{1}{2\xi}\sum_\alpha A^\nu_\alpha \partial_\nu \partial_\lambda A^\lambda_\alpha+[全微分項]
\end{align*}
より
\begin{align*}
&\mc{D}_{\alpha \nu,\beta\lambda}(\partial)=-\delta_{\alpha\beta}\left[\eta_{\nu\lambda}\Box-\left(1-\frac{1}{\xi}\right)\partial_\nu\partial_\lambda\right]+\mu^2_{\alpha\beta}\eta_{\nu\lambda} \\
&\mc{D}_{\alpha\nu,\beta\lambda}(ik)=\delta_{\alpha\beta}\left[\eta_{\nu\lambda}k^2-\left(1-\frac{1}{\xi}\right)k_\nu k_\lambda\right]+\mu^2_{\alpha\beta}\eta_{\nu\lambda}=\left[ \eta_{\nu\lambda}(k^2+\mu^2)-\left( 1-\frac{1}{\xi} \right) k_\nu k_\lambda\right]_{\alpha\beta}
\end{align*}
この逆行列,プロパゲータ$\Delta(k)=\mc{D}^{-1}(ik)$は
\begin{align*}
\Delta_{\alpha\mu,\beta\nu}(k)=\left[ \frac{1}{k^2+\mu^2}\left( \eta_{\mu\nu}-\frac{(1-\xi)k_\mu k_\nu}{k^2+\xi \mu^2} \right) \right]
\end{align*}
で与えられる.実際これは
\begin{align*}
\mc{D}_{\alpha\mu,\beta\nu}(ik)\Delta_{\beta\ ,\gamma}^{\ \nu \ \ \lambda}(k)=&\left[ \eta_{\mu\nu}(k^2+\mu^2)-\left( 1-\frac{1}{\xi} \right) k_\mu k_\nu\right]_{\alpha\beta}\left[ \frac{1}{k^2+\mu^2}\left( \eta^{\nu\lambda}-\frac{(1-\xi)k^\nu k^\lambda}{k^2+\xi \mu^2} \right) \right]_{\beta\gamma}  \\
=&\eta_{\mu\nu}\left( \eta^{\nu\lambda}-\frac{(1-\xi)k^\nu k^\lambda}{k^2+\xi \mu^2} \right)_{\alpha\gamma} \\
&+\left(\frac{1-\xi}{\xi}\right)\left[ \frac{1}{k^2+\mu^2}\left( \eta^{\nu\lambda}-\frac{(1-\xi)k^\nu k^\lambda}{k^2+\xi \mu^2} \right) k_\mu k_\nu\right]_{\alpha\gamma} \\
=&\left(\delta^\lambda_\mu - \frac{(1-\xi)k_\mu k^\lambda}{k^2+\xi \mu^2} \right)_{\alpha\gamma}+\left(\frac{1-\xi}{\xi}\right)\left[ \frac{1}{k^2+\mu^2}\left( k_\mu k^\lambda-\frac{(1-\xi)k^2 k_\mu k^\lambda}{k^2+\xi \mu^2} \right) \right]_{\alpha\gamma} \\
=&\left(\delta^\lambda_\mu - \frac{(1-\xi)k_\mu k^\lambda}{k^2+\xi \mu^2} \right)_{\alpha\gamma}+\left(\frac{1-\xi}{\xi}\right)\left[ \frac{k_\mu k^\lambda}{k^2+\mu^2}\left( 1-\frac{(1-\xi)k^2}{k^2+\xi \mu^2} \right) \right]_{\alpha\gamma} \\
=&\left(\delta^\lambda_\mu - \frac{(1-\xi)k_\mu k^\lambda}{k^2+\xi \mu^2} \right)_{\alpha\gamma}+\left(\frac{1-\xi}{\xi}\right)\left[ \frac{k_\mu k^\lambda}{k^2+\mu^2}\left( \frac{k^2+\xi \mu^2-k^2+\xi k^2}{k^2+\xi \mu^2} \right) \right]_{\alpha\gamma} \\
=&\left(\delta^\lambda_\mu - \frac{(1-\xi)k_\mu k^\lambda}{k^2+\xi \mu^2} \right)_{\alpha\gamma}+ \left[\frac{(1-\xi)k_\mu k^\lambda}{k^2+\xi \mu^2}\right]_{\alpha\gamma} \\
=&\delta^\lambda_\mu \delta_{\alpha\gamma}
\end{align*}
となって逆だ.\par
$\phi$については
\begin{align*}
&\mc{D}_{nm}(\partial)=-\delta_{nm}\Box +M^2_{nm} \\
&\mc{D}_{nm}(ik)=\delta_{nm}k^2+M^2_{nm}=\delta_{nm}k^2+\left. \frac{\partial^2 P(\phi)}{\partial \phi_n \partial \phi_m }\right|_{\phi=v}-\xi\sum_\alpha (t^\alpha v)_n(t^\alpha v)_m
\end{align*}
となる.プロパゲータは
\begin{align*}
\Delta_{nm}(k)=\mc{D}_{nm}^{-1}(ik)=\left[\delta_{nm}k^2+\left. \frac{\partial^2 P(\phi)}{\partial \phi_n \partial\phi_m }\right|_{\phi=v}\right]^{-1}+\xi\sum_{\alpha\beta}(t^\alpha v)_n (t^\beta v)_m (k^2)^{-1}(k^2+\xi\mu^2)_{\alpha\beta}^{-1}
\end{align*}
で与えられる.これが実際に逆であることを確認するために
\begin{align*}
\sum_m (k^2+M^2)_{nm}(t^\beta v)_m=&k^2(t^\beta v)_n +\sum_m M^2_{nm}(t^\beta v)_m \\
=&\sum_\alpha (t^\alpha v)_n k^2 \delta_{\alpha\beta}+\xi \sum_\alpha(t^\alpha v)_n \mu^2_{\alpha\beta} \\
=&\sum_\alpha (t^\alpha v)_n (k^2+\xi\mu^2)_{\alpha\beta}
\end{align*}
であることと,
\begin{align*}
& \sum_m \left(\delta_{nm}k^2+\left. \frac{\partial^2 P(\phi)}{\partial \phi_n \partial\phi_m }\right|_{\phi=v}\right)(t^\alpha v)_m=k^2(t^\alpha v)_n \\
\Rightarrow \quad & \sum_m \left(\delta_{nm}k^2+\left. \frac{\partial^2 P(\phi)}{\partial \phi_n \partial\phi_m }\right|_{\phi=v}\right)^{-1}(t^\alpha v)_m=(k^2)^{-1}(t^\alpha v)_n \quad (逆行列の固有値は逆数)\\
\Rightarrow \quad & \sum_m(t^\alpha v)_m\left(\delta_{mn}k^2+\left. \frac{\partial^2 P(\phi)}{\partial \phi_m \partial\phi_n }\right|_{\phi=v}\right)^{-1}=(k^2)^{-1}(t^\alpha v)_n \quad(両辺転置)
\end{align*}
を用いると
\begin{align*}
&\sum_m \mc{D}_{nm}(ik)\Delta_{m\ell}(k) \\
=&\sum_m [\delta_{nm}k^2+M^2_{nm}]\left\{ \left[\delta_{m\ell}k^2+\left. \frac{\partial^2 P(\phi)}{\partial \phi_m \partial\phi_\ell }\right|_{\phi=v}\right]^{-1}+\xi\sum_{\alpha\beta}(t^\alpha v)_m (t^\beta v)_\ell (k^2)^{-1}(k^2+\xi\mu^2)_{\alpha\beta}^{-1} \right\} \\
=&\sum_m \left[\delta_{nm}k^2+\left. \frac{\partial^2 P(\phi)}{\partial \phi_n \partial\phi_m }\right|_{\phi=v}-\xi\sum_\alpha (t^\alpha v)_n(t^\alpha v)_m\right] \left[\delta_{m\ell}k^2+\left. \frac{\partial^2 P(\phi)}{\partial \phi_m \partial\phi_\ell }\right|_{\phi=v}\right]^{-1} \\
&+\xi \sum_{m\alpha\beta}  [\delta_{nm}k^2+M^2_{nm}](t^\alpha v)_m (t^\beta v)_\ell (k^2)^{-1}(k^2+\xi\mu^2)_{\alpha\beta}^{-1} \\
=&\delta_{n\ell} -\xi\sum_{m\alpha}(t^\alpha v)_n (t^\alpha v)_m\left(\delta_{m\ell}k^2+\left. \frac{\partial^2 P(\phi)}{\partial \phi_m \partial \phi_\ell }\right|_{\phi=v}\right)^{-1} \\
&+\xi\sum_{\alpha\beta\gamma}(t^\gamma v)_n (t^\beta v)_\ell (k^2+\xi\mu^2)_{\gamma\alpha}(k^2+\xi\mu^2)_{\alpha\beta}^{-1}(k^2)^{-1} \\
=&\delta_{n\ell} -\xi\sum_\alpha (t^\alpha v)_n (t^\alpha v)_\ell(k^2)^{-1}+\xi\sum_\beta (t^\beta v)_n (t^\beta v)_\ell(k^2)^{-1}=\delta_{n\ell}
\end{align*}
となって実際に逆であることがわかる.\par
$\psi$については,共役場は$\psi^\dagger$ではなく$\bar{\psi}$として
\begin{align*}
&\mc{D}(\partial)=\Slash{\partial}+m \\
&\mc{D}(ik)=i\Slash{k}+m
\end{align*}
このときプロパゲータは自明に
\begin{align*}
\Delta(k)=\mc{D}^{-1}(ik)=\frac{-i\Slash{k}+m}{k^2+m^2}
\end{align*}
となる.実際に逆であることは$\Slash{k}\Slash{k}=k^2$であることを用いればすぐにわかる.\par
$\omega$については
\begin{align*}
&\mc{D}_{\alpha\beta}(\partial)=-\delta_{\alpha\beta}\Box+\xi\mu^2_{\alpha\beta} \\
&\mc{D}_{\alpha\beta}(ik)=\delta_{\alpha\beta}k^2+\xi\mu^2_{\alpha\beta}=[k^2+\xi\mu^2]_{\alpha\beta}
\end{align*}
このときプロパゲータは自明に
\begin{align*}
\Delta_{\alpha\beta}(k)=\mc{D}^{-1}_{\alpha\beta}(ik)=[k^2+\xi\mu^2]^{-1}_{\alpha\beta}
\end{align*}
となる.

\newpage

\subsection{電弱理論}
最初に電子型レプトン場を考える.知られている限りでは,これらは電子場$e$の左手成分と右手成分
\begin{align*}
e_L=\frac{1}{2}(1+\gamma_5)e,\quad e_R=\frac{1}{2}(1-\gamma_5)e
\end{align*}
および純粋な左手成分の電子ニュートリノ場$\nu_{eL}$
\begin{align*}
\gamma_5 \nu_{eL}=+\nu_{eL}
\end{align*}
だけからなる.ゲージ群の任意の一つの表現に属する場は,全て同じローレンツ変換性を持たねばならない.\par
$\Rightarrow$ゲージ群の表現は,左手成分の2重項$(\nu_{eL},e_L)$と右手成分の1重項$e_R$に分かれる.\\
よってゲージ群は$SU(2)_L$と,左手と右手のそれぞれの場についての$U(1)$による,$SU(2)_L\times U(1)_L\times U(1)_R$が最も大きいものとなる.この群のもとで場は
\begin{align*}
\delta\left(
\begin{array}{cc}
\nu_e \\
e
\end{array}
\right)=i\left[ \underset{SU(2)_L}{\uwave{\vec{\epsilon}\cdot \vec{t}}}+\underset{U(1)_L}{\uwave{\epsilon_L t_L}}+ \underset{U(1)_R}{\uwave{\epsilon_R t_R }}\right]\left(
\begin{array}{cc}
\nu_e \\
e
\end{array}
\right)
\end{align*}
と変換される.ここで生成子$\vec{t},t_L,t_R$は
\begin{align*}
\vec{t}&=\frac{g}{4}(1+\gamma_5)\left\{ \left(
\begin{array}{cc}
0 & 1 \\
1 & 0 
\end{array}
\right),\left(
\begin{array}{cc}
0 & -i \\
i & 0
\end{array}
\right) , \left(
\begin{array}{cc}
1 & 0 \\
0 & -1
\end{array}
\right)\right\} \\
&=g\frac{1+\gamma_5}{2}\frac{\vec{\sigma}}{2} \quad (\vec{\sigma}はパウリ行列) \\
t_L &\propto(1+\gamma_5)\left(
\begin{array}{cc}
1 & 0 \\
0 & 1
\end{array}
\right),\quad t_R\propto (1-\gamma_5)
\end{align*}
で与えられる.($t_R$は$(1-\gamma_5)$によって1重項$e_R$のみに作用するので,行列は必要ない.)ここで$\vec{\sigma}/2$は$SU(2)$角運動量代数を構成することを思い出そう.ただし$g$は後に選ぶ定数だ.$t_L,t_R$の代わりに,その線型結合で与えられる新たな生成子
\begin{align*}
y&\equiv g' \left[ \frac{1+\gamma_5}{4}\left(
\begin{array}{cc}
1 & 0 \\
0 & 1
\end{array}
\right)+\frac{1-\gamma_5}{2} \right] \\
n_e &\equiv g'' \left[ \frac{1+\gamma_5}{2}\left(
\begin{array}{cc}
1 & 0 \\
0 & 1
\end{array}
\right)+\frac{1-\gamma_5}{2} \right]
\end{align*}
を考えるのが便利だ.ここで$g'$と$g''$は$g$と同様に後で選ぶ定数だ.\par
生成子$t_3$と$y$は,荷電$q$という形で現れる.($e$が素電荷の記号と電子場の記号とで混合しないように)
\begin{align*}
q\left(
\begin{array}{cc}
\nu_e \\
e
\end{array}
\right)=&\left(
\begin{array}{cc}
0\cdot \nu_e \\
-e\cdot e
\end{array}
\right)=e\left(
\begin{array}{cc}
0 & 0 \\
0 & -1
\end{array}
\right)\left(
\begin{array}{cc}
\nu_e \\
e
\end{array}
\right) \\
\frac{e}{g}t_3 -\frac{e}{g'}y=& e\frac{1+\gamma_5}{4}\left(
\begin{array}{cc}
1 & 0 \\
0 & -1
\end{array}
\right)-e\left[  \frac{1+\gamma_5}{4}\left(
\begin{array}{cc}
1 & 0 \\
0 & 1
\end{array}
\right)+\frac{1-\gamma_5}{2} \right] \\
=& -e \left[  \frac{1+\gamma_5}{2}\left(
\begin{array}{cc}
0 & 0 \\
0 & 1
\end{array}
\right)+\frac{1-\gamma_5}{2} \right] \\
\left( \frac{e}{g}t_3 -\frac{e}{g'}y \right)\left(
\begin{array}{cc}
\nu_e \\
e
\end{array}
\right)=&\left(
\begin{array}{cc}
0\cdot \nu_e \\
-e\cdot e
\end{array}
\right)=q\left(
\begin{array}{cc}
\nu_e \\
e
\end{array}
\right) \\
\Rightarrow \quad q=& \frac{e}{g}t_3 -\frac{e}{g'}y
\end{align*}
また$n_e$は電子型レプトン数
\begin{align*}
\frac{n_e}{g''}\left(
\begin{array}{cc}
\nu_e \\
e
\end{array}
\right)=\left(
\begin{array}{cc}
1\cdot \nu_e \\
1\cdot e
\end{array}
\right)
\end{align*}
に対応する.\par
(21.3.9)の両辺を電荷$e$で割ると,有名な式
\begin{align*}
Q=T_3+Y
\end{align*}
が導かれる.これは電磁相互作用と弱い相互作用について(つまり$SU(2)_L\times U(1)$対称な理論で)成り立つ保存量の関係式であり,$-y/g'$の値$Y$を普通,弱超電荷と呼ぶ.($Y/2$でも良い.)たとえば$(\nu_{eL} ,e_L)$に作用すると
\begin{align*}
-\frac{y}{g'}\left(
\begin{array}{cc}
\nu_{eL} \\
e_L
\end{array}
\right)=
\left(
\begin{array}{cc}
-1/2\cdot \nu_e \\
-1/2\cdot e
\end{array}
\right)
\end{align*}
となるから,$\nu_{eL}$の弱超電荷は$-1/2$,$e_L$は$-1/2$で,$e_R$は$-1$となる.$t_3/g$の値$T_3$は
\begin{align*}
\frac{t_3}{g}\left(
\begin{array}{cc}
\nu_{eL} \\
e_L
\end{array}
\right)=
\left(
\begin{array}{cc}
+1/2\cdot \nu_e \\
-1/2\cdot e
\end{array}
\right),\quad \frac{t_3}{g}e_R=0
\end{align*}
となって,$(\nu_{eL} , e_L)$について$(+1/2,-1/2)$で$e_R$については$0$となる.よって
\begin{align*}
e_L:&(Q,T_3,Y)=(-1,-1/2,-1/2) \\
e_R:&(Q,T_3,Y)=(-1,0,-1) \\
\nu_{eL}:&(Q,T_3,Y)=(0,+1/2,-1/2)
\end{align*}
というように量子数がわかる.今回は生成子$y$の形が特別にわかっているからこの関係は自明だが,スカラー場についての$y^{(\phi)}$やクォークについての$y^{(q)}$などは$SU(2)\times U(1)$対称性から,あるいは既知の電荷と角運動量から逆に保存量の関係式を用いて調べるしかない.

\vskip\baselineskip

ベータ崩壊のように,電荷を変える弱い相互作用と,電磁相互作用の両方を理論に入れたいので,$\vec{t}$と$y$に結合するゲージ場$\vec{A}^\mu$と$B^\mu$の存在を仮定する.さらに,残った一個の自由度に対応する,$t_L$の$t_R$の残った一個の線型結合($n_e$に選ぶことができる)に結合するゲージ場を含めたければ,そうしても良い.$n_e$に結合する質量ゼロのゲージ場によって作り出される長距離力については非常に厳しい制限がある.強さ$g''$で$n_e$に結合するゲージ場を理論に含めるためには,このゲージ対称性$n_e$は破れていると仮定しなければならない.\par
$\Rightarrow$しかしそのようなゲージ結合によって生成される弱い相互作用の実験的証拠はないので,$n_e$をゲージ群の生成子から除外することにする.\\
するとゲージ群は
\begin{align*}
G=SU(2)_L\times U(1)
\end{align*}
となり,生成子はそれぞれ(21.3.4)(21.3.7)で与えられる$\vec{t},y$だ.\par
結合定数$g,g'$は,生成子$\vec{t},y$に結合するゲージ場$\vec{A}^\mu,B^\mu$が正準的に規格化されるように(つまりそれぞれのゲージ場の運動項の係数が1になるように)調整される.これらのゲージ場と,電子型レプトン$\ell=(\nu_e , e)$のみを含む$G$不変でくりこみ可能($d<4$)な最も一般的なラグランジアンは(15.1.13)を用いて
\begin{align*}
\mc{L}_{YM}+\mc{L}_e=&-\frac{1}{4}\sum_\alpha F^{\mu\nu}_\alpha F_{\alpha\mu\nu}-\bar{\ell}(\Slash{\partial}-i\sum_\alpha \Slash{A}_\alpha t^\alpha)\ell \\
=&-\frac{1}{4}(\partial_\mu A_{i\nu} -\partial_\nu A_{i\mu} +g\epsilon_{ijk}A_{i\mu}A_{k\nu})^2 \quad \because (21.3.4)よりC_{ijk}=g\epsilon_{ijk}\\
&-\frac{1}{4}(\partial_\mu B_\nu-\partial_\nu B_\mu)^2 \quad \because U(1)ではC_{ijk}=0 \\
&-\bar{\ell}(\Slash{\partial}-i\Slash{A}_i t_i -i\Slash{B}y)\ell \\
=&\frac{1}{4}(\partial_\mu \vec{A}_\nu-\partial_\nu \vec{A}_\mu +g\vec{A}_\mu \times \vec{A}_\nu)^2 -\frac{1}{4}(\partial_\mu B_\nu-\partial_\nu B_\mu)^2 \\
&-\bar{\ell}(\Slash{\partial}-i\vec{\Slash{A}} \cdot\vec{t}_L -i\Slash{B}y)\ell
\end{align*}
で与えられる.($i$は$SU(2)$添え字.外積の定義を思い出すこと.) 

\vskip\baselineskip

もちろん,実際は$\vec{t}$と$y$に結合するゲージ場のうち,ただひとつの線型結合の電磁場$A^\mu$だけが質量ゼロだ.つまり,最終的に
\begin{align*}
-\bar{\ell}(\Slash{\partial}-i(-e)\Slash{A})\ell=-\bar{\ell}(\Slash{\partial}-i\Slash{A}q)\ell
\end{align*}
という項が現れた上で,このゲージ場$A^\mu$の質量項が現れてはいけない,という結果を目指す必要がある.この$q$は荷電であり,$U(1)$の生成子である.だから(21.1.11)より電荷(21.3.9)を生成子とする部分群\uwave{$U(1)_{em}$に$SU(2)_L\times U(1)$が自発的に破れている},と仮定しなければならない!\par

\vskip\baselineskip

決まった質量をもつスピン1の粒子に対応する正準規格化されたベクトル場は以下の場からなることが分かる.電荷$-e$で質量$m_W$の場(電荷は保存するので(21.3.20)で各項が電荷$\pm 0$となるようになっている.例えば$\bar{e}$は陽電子に対応して$+e$で,$e$は電子に対応して$-e$.$\nu_e$は中性なので$W$の電荷を$x$とすると第一項目より$\bar{e}W\nu_e\to +e+x+0\to x=-e$とわかる.)
\begin{align*}
W^\mu_-=\frac{1}{\sqrt{2}}(A^\mu_1+iA^\mu_2)
\end{align*}
電荷$+e$で同じ質量の別の場
\begin{align*}
W^\mu_+=\frac{1}{\sqrt{2}}(A^\mu_1-iA^\mu_2)
\end{align*}
(これらは複素場である.)また,電気的に中性で質量がそれぞれ$m_Z$とゼロの,$A_3^\mu$と$B^\mu$との直交する線型結合で与えられる二つの場
\begin{align*}
Z^\mu=\cos \theta A^\mu_3+\sin\theta B^\mu ,\quad A^\mu=-\sin \theta A^\mu_3+\cos\theta B^\mu
\end{align*}
つまり
\begin{align*}
&\left(
\begin{array}{cc}
Z^\mu \\
A^\mu
\end{array}
\right)=\left(
\begin{array}{cc}
\cos\theta & \sin \theta \\
-\sin\theta& \cos \theta
\end{array}
\right)\left(
\begin{array}{cc}
A^\mu_3 \\
B^\mu
\end{array}
\right) \quad \Rightarrow \quad  \left(
\begin{array}{cc}
A^\mu_3 \\
B^\mu
\end{array}
\right)=\left(
\begin{array}{cc}
\cos\theta & -\sin \theta \\
\sin\theta& \cos \theta
\end{array}
\right)\left(
\begin{array}{cc}
A^\mu \\
Z^\mu
\end{array}
\right) \\
\Rightarrow \quad & A^\mu_3=\cos\theta Z^\mu -\sin\theta A^\mu ,\quad B^\mu=\sin\theta Z^\mu+\cos\theta A^\mu
\end{align*}
となる.(わざわざ行列で書く必要はないが,これはただの回転をしただけだということを意識するため.)\par
一般的な結果(21.1.11)$\sim$(21.1.12)によると,破れていない対称性の生成子は,生成子$\vec{t},y$の線型結合で与えられ,その係数はゲージ場を展開(21.3.12)(21.3.13)(21.3.16)(21.3.17)したときの,対応する質量ゼロの場の展開係数と同じだ.
\begin{align*}
A^\mu_1&=\frac{1}{\sqrt{2}}(W^\mu_- + W^\mu_+) \quad \Rightarrow c_1=0 \\
A^\mu_2&=\frac{1}{\sqrt{2}}(W^\mu_- - W^\mu_+) \quad \Rightarrow c_2=0 \\
A^\mu_3&=\uwave{-\sin\theta A^\mu}+\cos\theta Z^\mu  \quad \Rightarrow c_3=-\sin\theta \\
B^\mu&=\uwave{\cos\theta A^\mu}+\sin\theta Z^\mu \quad \Rightarrow c_y=\cos\theta \\
\Rightarrow \quad & q=\sum_\alpha c_\alpha t^\alpha =-\sin\theta t_3 +\cos\theta y
\end{align*}
がわかる.これと(21.3.9)が等しいという仮定より
\begin{align*}
&q=\frac{e}{g}t_3-\frac{e}{g'}y=-\sin\theta t_3 +\cos\theta y \\
\Rightarrow \quad &g=-\frac{e}{\sin\theta} ,\quad g'=-\frac{e}{\cos\theta}
\end{align*}
が得られる.ついでに,$\sin^2\theta+\cos^2\theta=1$より
\begin{align*}
&\sin^2\theta+\cos^2\theta=e^2\left( \frac{1}{g'^2}+\frac{1}{g^2} \right)=1 \\
&e=\frac{gg'}{\sqrt{g^2+g'^2}} \quad \because e>0 , \\
&\quad \sin\theta=-\frac{e}{g}=-\frac{g'}{\sqrt{g^2+g'^2}},\quad \cos\theta=-\frac{e}{g'}=-\frac{g}{\sqrt{g^2+g'^2}}
\end{align*}
が得られる.\par
(21.3.11)の第三項目に含まれる,レプトンとゲージボゾンの間の完全な結合は,結合定数$g,g'$を用いて
\begin{align*}
i\mc{L}'_e=&-\overline{\left(
\begin{array}{cc}
\nu_e \\
e
\end{array}
\right)}\left[\sum_\alpha \Slash{A}_\alpha t_\alpha  \right]\left(
\begin{array}{cc}
\nu_e \\
e
\end{array}
\right)=-\overline{\left(
\begin{array}{cc}
\nu_e \\
e
\end{array}
\right)}\left[ \vec{\Slash{A}}\cdot \vec{t}+\Slash{B}y  \right]\left(
\begin{array}{cc}
\nu_e \\
e
\end{array}
\right) \\
=&-\overline{\left(
\begin{array}{cc}
\nu_e \\
e
\end{array}
\right)}\left[ \frac{g}{4}\left(
\begin{array}{cc}
0 & \Slash{A}_1-i\Slash{A}_2 \\
\Slash{A}_1+i\Slash{A}_2 & 0 
\end{array}
\right) (1+\gamma_5) +\Slash{A}_3 t_3 +\Slash{B}y \right]\left(
\begin{array}{cc}
\nu_e \\
e
\end{array}
\right) \\
=&-\overline{\left(
\begin{array}{cc}
\nu_e \\
e
\end{array}
\right)}\biggl[ \frac{g}{2}\frac{1}{\sqrt{2}}\left\{ \frac{1}{\sqrt{2}}(\Slash{A}_1+i\Slash{A}_2) \right\}\left(
\begin{array}{cc}
0 & 0 \\
1 & 0
\end{array}
\right)(1+\gamma_5) \\
&\qquad\qquad  +\frac{g}{2}\frac{1}{\sqrt{2}}\left\{ \frac{1}{\sqrt{2}}(\Slash{A}_1-i\Slash{A}_2) \right\}\left(
\begin{array}{cc}
0 & 1 \\
0 & 0
\end{array}
\right)(1+\gamma_5) \\
&\qquad\qquad +\left( \cos\theta \Slash{Z}-\sin\theta \Slash{A} \right)t_3 +\left( \sin\theta \Slash{Z}+\cos\theta \Slash{A} \right)y \biggr]\left(
\begin{array}{cc}
\nu_e \\
e
\end{array}
\right) \\
=&-\overline{\left(
\begin{array}{cc}
\nu_e \\
e
\end{array}
\right)}\biggl[ \frac{1}{\sqrt{2}}\Slash{W}_- \frac{g}{2}(1+\gamma_5)\frac{1}{2}(\sigma_1-i\sigma_2) \\
&\qquad\qquad +\frac{1}{\sqrt{2}}\Slash{W}_+ \frac{g}{2}(1+\gamma_5)\frac{1}{2}(\sigma_1+i\sigma_2) \\
&\qquad\qquad +\Slash{Z}(t_3\cos\theta + y\sin\theta)+\Slash{A}(-t_3 \sin\theta +y\cos\theta) \biggr]\left(
\begin{array}{cc}
\nu_e \\
e
\end{array}
\right) \\
=&-\overline{\left(
\begin{array}{cc}
\nu_e \\
e
\end{array}
\right)}\biggl[ \frac{1}{\sqrt{2}}\Slash{W}_- (t_1-it_2) +\frac{1}{\sqrt{2}}\Slash{W}_+ (t_1+it_2) \\
&\qquad\qquad +\Slash{Z}(t_3\cos\theta + y\sin\theta)+\Slash{A}\underset{(21.3.9)}{\uwave{(-t_3 \sin\theta +y\cos\theta)}} \biggr]\left(
\begin{array}{cc}
\nu_e \\
e
\end{array}
\right)
\end{align*}
各項を分けて調べると,第一項目は
\begin{align*}
&-\overline{\left(
\begin{array}{cc}
\nu_e \\
e
\end{array}
\right)}\frac{1}{\sqrt{2}}\Slash{W}_- (t_{1L}-it_{2L})\left(
\begin{array}{cc}
\nu_e \\
e
\end{array}
\right)=-\overline{\left(
\begin{array}{cc}
\nu_e \\
e
\end{array}
\right)} \frac{1}{\sqrt{2}}\Slash{W}_- \frac{g}{2}(1+\gamma_5)\left(
\begin{array}{cc}
0 & 0 \\
1 & 0
\end{array}
\right)\left(
\begin{array}{cc}
\nu_e \\
e
\end{array}
\right) \\
&=-(\nu_e^\dagger,e^\dagger)\gamma_4 \frac{1}{\sqrt{2}}\Slash{W}_- \frac{g}{2}(1+\gamma_5)\left(
\begin{array}{cc}
0 & 0 \\
1 & 0
\end{array}
\right)\left(
\begin{array}{cc}
\nu_e \\
e
\end{array}
\right) \\
&=-\frac{g}{\sqrt{2}}\left( \bar{e}\Slash{W}_- \left(\frac{1+\gamma_5}{2}\right)\nu_e \right)
\end{align*}
となる.(レプトン場の共役場はエルミート共役により行ベクトルになっていることに気を付ける.)最後の等式ではスピノル添え字と$SU(2)$添え字が別であるからガンマ行列を挟んで$2\times 2$行列を作用させることができる.同様に第二項目は
\begin{align*}
&-\overline{\left(
\begin{array}{cc}
\nu_e \\
e
\end{array}
\right)}\frac{1}{\sqrt{2}}\Slash{W}_+ (t_{1L}+it_{2L})\left(
\begin{array}{cc}
\nu_e \\
e
\end{array}
\right)=-\overline{\left(
\begin{array}{cc}
\nu_e \\
e
\end{array}
\right)} \frac{1}{\sqrt{2}}\Slash{W}_+ \frac{g}{2}(1+\gamma_5)\left(
\begin{array}{cc}
0 & 1 \\
0 & 0
\end{array}
\right)\left(
\begin{array}{cc}
\nu_e \\
e
\end{array}
\right) \\
&=-\frac{g}{\sqrt{2}}\left( \bar{\nu}_e\Slash{W}_+ \left(\frac{1+\gamma_5}{2}\right)e \right)
\end{align*}
となる.第三項目は$\frac{1}{2}(1-\gamma_5)\nu_e=0$に留意すると
\begin{align*}
&t_{3L}\cos\theta + y\sin\theta=-\frac{e}{g'}\frac{g}{4}(1+\gamma_5) \left(
\begin{array}{cc}
1 & 0 \\
0 & -1
\end{array}
\right)-\frac{e}{g}g'\left[ \frac{1+\gamma_5}{4}\left(
\begin{array}{cc}
1 & 0 \\
0 & 1
\end{array}
\right)+\frac{1-\gamma_5}{2} \right] \\
&=\frac{g^2}{\sqrt{g^2+g'^2}}\frac{1}{4}(1+\gamma_5) \left(
\begin{array}{cc}
1 & 0 \\
0 & -1
\end{array}
\right)-\frac{g'^2}{\sqrt{g^2+g'^2}}\left[ \frac{1+\gamma_5}{4}\left(
\begin{array}{cc}
1 & 0 \\
0 & 1
\end{array}
\right)+\frac{1-\gamma_5}{2} \right]
\end{align*}
であるから
\begin{align*}
-&\overline{\left(
\begin{array}{cc}
\nu_e \\
e
\end{array}
\right)} \Slash{Z}(t_3\cos\theta + y\sin\theta)\left(
\begin{array}{cc}
\nu_e \\
e
\end{array}
\right) \\
=& \overline{\left(
\begin{array}{cc}
\nu_e \\
e
\end{array}
\right)}\Slash{Z}\frac{g^2}{\sqrt{g^2+g'^2}}\frac{1}{4}(1+\gamma_5)\left(
\begin{array}{cc}
\nu_e \\
-e
\end{array}
\right) +\overline{\left(
\begin{array}{cc}
\nu_e \\
e
\end{array}
\right)}\Slash{Z}\frac{g'^2}{\sqrt{g^2+g'^2}}\frac{1}{4}(1+\gamma_5)\left(
\begin{array}{cc}
\nu_e \\
e
\end{array}
\right) \\
&+\overline{\left(
\begin{array}{cc}
\nu_e \\
e
\end{array}
\right)}\Slash{Z}\frac{g'^2}{\sqrt{g^2+g'^2}}\frac{1}{2}(1-\gamma_5)\left(
\begin{array}{cc}
\nu_e \\
e
\end{array}
\right) \\
=& \frac{1}{2}\frac{g^2+g'^2}{\sqrt{g^2+g'^2}}\bar{\nu}_e \Slash{Z}\left(\frac{1+\gamma_5}{2}\right)\nu_e-\frac{1}{2}\frac{g^2-g'^2}{\sqrt{g^2+g'^2}}\bar{e}\Slash{Z}\left(\frac{1+\gamma_5}{2}\right)e +\frac{g'^2}{\sqrt{g^2+g'^2}}\bar{e}\Slash{Z}\left(\frac{1-\gamma_5}{2}\right)e
\end{align*}
第四項目は(21.3.18)(21.3.9)より
\begin{align*}
-\overline{\left(
\begin{array}{cc}
\nu_e \\
e
\end{array}
\right)}\Slash{A}(-t_{3L}\sin\theta+y\cos\theta)\left(
\begin{array}{cc}
\nu_e \\
e
\end{array}
\right)=-\overline{\left(
\begin{array}{cc}
\nu_e \\
e
\end{array}
\right)}\Slash{A}q\left(
\begin{array}{cc}
\nu_e \\
e
\end{array}
\right)=+e(\bar{e}\Slash{A}e)
\end{align*}
以上からラグランジアンは
\begin{align*}
i\mc{L}'_e=&-\frac{g}{\sqrt{2}}\left( \bar{e}\Slash{W}_- \left(\frac{1+\gamma_5}{2}\right)\nu_e \right)-\frac{g}{\sqrt{2}}\left( \bar{\nu}_e\Slash{W}_+ \left(\frac{1+\gamma_5}{2}\right)e \right) \\
&+\frac{1}{2}\sqrt{g^2+g'^2}\bar{\nu}_e \Slash{Z}\left(\frac{1+\gamma_5}{2}\right)\nu_e-\frac{1}{2}\frac{g^2-g'^2}{\sqrt{g^2+g'^2}}\bar{e}\Slash{Z}\left(\frac{1+\gamma_5}{2}\right)e +\frac{g'^2}{\sqrt{g^2+g'^2}}\bar{e}\Slash{Z}\left(\frac{1-\gamma_5}{2}\right)e \\
&+e(\bar{e}\Slash{A}e)
\end{align*}
となる.\par
理論を完成させるためには,対称性の破れの機構について何らかの仮定が必要.具体的には,この機構により$W^\pm$と$Z^0$だけでなく,電子にも質量を与えたい!\par
$\Rightarrow$くりこみ可能な弱い結合の理論でこれを可能にする唯一の方法は,$\bar{\ell}_L$と$\ell_R$(および$\bar{\ell}_R$と$\ell_L$)に微分なしで結合するスカラー場を導入することだ.(Higgs機構)すると,$SU(2)_L\times U(1)$不変性より,スカラー場は$\ell_L$と同様$SU(2)_L$2重項だが,$y$の値,つまり$q$の値はずらしたものになる.\par
こうして「湯川」結合
\begin{align*}
\mc{L}_{\phi e}&=-G_e \overline{\left(
\begin{array}{cc}
\nu_e \\
e
\end{array}
\right)}_L \left(
\begin{array}{cc}
\phi^+ \\
\phi^0
\end{array}
\right)e_R-G_e \bar{e}_R({\phi ^+}^\dagger ,{\phi^0}^\dagger)\left(
\begin{array}{cc}
\nu_e \\
e
\end{array}
\right) \\
&=-G_e \overline{\left(
\begin{array}{cc}
\nu_e \\
e
\end{array}
\right)}_L \left(
\begin{array}{cc}
\phi^+ \\
\phi^0
\end{array}
\right)e_R+\mr{H.c.}
\end{align*}
を仮定する.(ラグランジアンは全体で実であるために,エルミート項が必要.)ここで$(\phi^+,\phi^0)$は$SU(2)$2重項で,それに作用する$SU(2)\times U(1)$生成子は行列
\begin{align*}
\vec{t}^{(\phi)}=&\frac{g}{2}\left\{ \left(
\begin{array}{cc}
0 & 1 \\
1 & 0 
\end{array}
\right),\left(
\begin{array}{cc}
0 & -i \\
i & 0
\end{array}
\right) , \left(
\begin{array}{cc}
1 & 0 \\
0 & -1
\end{array}
\right)\right\}=g\frac{\vec{\sigma}}{2} \\
y^{(\phi)}=&-\frac{g'}{2}\left(
\begin{array}{cc}
1 & 0 \\
0 & 1 
\end{array}
\right)
\end{align*}
で表され,電荷の行列は(21.3.9)より
\begin{align*}
q^{(\phi)}=\frac{e}{g}t_3^{(\phi)}-\frac{e}{g'}y^{(\phi)}=e\frac{1}{2}\left(
\begin{array}{cc}
1 & 0 \\
0 & -1
\end{array}
\right)+e\frac{1}{2}\left(
\begin{array}{cc}
1 & 0 \\
0 & 1
\end{array}
\right)=e\left(
\begin{array}{cc}
1 & 0 \\
0 & 0
\end{array}
\right)
\end{align*}
となる.(これに$SU(2)$2重項$(\phi_1,\phi_2)$に作用させると,$\phi_1$の電荷が$+e$で$\phi_2$の電荷がゼロとなることがわかる.したがって$\phi_1\to\phi^+ , \phi_2\to \phi^0$と表記している.)\par
ラグランジアン(21.3.21)が$SU(2)\times U(1)$不変であることを要請すれば生成子は自然と(21.3.22)(21.3.23)となることを確認しておく.
\begin{align*}
\delta\mc{L}_{\phi e}=&-G_e \delta \overline{\left(
\begin{array}{cc}
\nu_e \\
e
\end{array}
\right)}_L \left(
\begin{array}{cc}
\phi^+ \\
\phi^0
\end{array}
\right)e_R-G_e \overline{\left(
\begin{array}{cc}
\nu_e \\
e
\end{array}
\right)}_L \left(
\begin{array}{cc}
\phi^+ \\
\phi^0
\end{array}
\right)\delta e_R -G_e \overline{\left(
\begin{array}{cc}
\nu_e \\
e
\end{array}
\right)}_L \delta \left(
\begin{array}{cc}
\phi^+ \\
\phi^0
\end{array}
\right)e_R \\
=&-G_e  \left[i \left\{\vec{\epsilon}\cdot \vec{t}+\alpha y\right\} \left(
\begin{array}{cc}
\nu_{eL} \\
e_L
\end{array}
\right)\right]^\dagger \gamma_4 \left(
\begin{array}{cc}
\phi^+ \\
\phi^0
\end{array}
\right)e_R \\
&-G_e \overline{\left(
\begin{array}{cc}
\nu_e \\
e
\end{array}
\right)}_L \left(
\begin{array}{cc}
\phi^+ \\
\phi^0
\end{array}
\right)\left[i\left\{ \vec{\epsilon}\cdot \vec{t}+\alpha y \right\}e_R\right] \\
&-G_e \overline{\left(
\begin{array}{cc}
\nu_e \\
e
\end{array}
\right)}_L\left[i\left\{ \vec{\epsilon}\cdot \vec{t}^{(\phi)}+\alpha y^{(\phi)} \right\} \left(
\begin{array}{cc}
\phi^+ \\
\phi^0
\end{array}
\right)\right] e_R \\
=&-G_e  \left[i \left\{g\vec{\epsilon}\cdot \frac{\vec{\sigma}}{2}+\alpha \frac{g'}{2}\right\} \left(
\begin{array}{cc}
\nu_{eL} \\
e_L
\end{array}
\right)\right]^\dagger \gamma_4 \left(
\begin{array}{cc}
\phi^+ \\
\phi^0
\end{array}
\right)e_R \\
&-G_e \overline{\left(
\begin{array}{cc}
\nu_e \\
e
\end{array}
\right)}_L \left(
\begin{array}{cc}
\phi^+ \\
\phi^0
\end{array}
\right)\left[i \alpha g' e_R\right] \\
&-G_e \overline{\left(
\begin{array}{cc}
\nu_e \\
e
\end{array}
\right)}_L\left[i\left\{ \vec{\epsilon}\cdot \vec{t}^{(\phi)}+\alpha y^{(\phi)} \right\} \left(
\begin{array}{cc}
\phi^+ \\
\phi^0
\end{array}
\right)\right] e_R \\
=&-G_e \overline{\left(
\begin{array}{cc}
\nu_e \\
e
\end{array}
\right)}_L\left[-i \left\{g\vec{\epsilon}\cdot \frac{\vec{\sigma}}{2}+\alpha \frac{g'}{2}\right\}\right] \left(
\begin{array}{cc}
\phi^+ \\
\phi^0
\end{array}
\right) e_R-G_e \overline{\left(
\begin{array}{cc}
\nu_e \\
e
\end{array}
\right)}_L[ +i\alpha g'] \left(
\begin{array}{cc}
\phi^+ \\
\phi^0
\end{array}
\right) e_R \\
&-G_e \overline{\left(
\begin{array}{cc}
\nu_e \\
e
\end{array}
\right)}_L\left[i\left\{ \vec{\epsilon}\cdot \vec{t}^{(\phi)}+\alpha y^{(\phi)} \right\} \left(
\begin{array}{cc}
\phi^+ \\
\phi^0
\end{array}
\right)\right] e_R \\
=&-G_e \overline{\left(
\begin{array}{cc}
\nu_e \\
e
\end{array}
\right)}_L\left[-i g\vec{\epsilon}\cdot \frac{\vec{\sigma}}{2}+i\alpha \frac{g'}{2} \right] \left(
\begin{array}{cc}
\phi^+ \\
\phi^0
\end{array}
\right) e_R \\
&-G_e \overline{\left(
\begin{array}{cc}
\nu_e \\
e
\end{array}
\right)}_L\left[i\left\{ \vec{\epsilon}\cdot \vec{t}^{(\phi)}+\alpha y^{(\phi)} \right\} \left(
\begin{array}{cc}
\phi^+ \\
\phi^0
\end{array}
\right)\right] e_R
\end{align*}
不変であるためにはこれがゼロである必要があるから,$t^{(\phi)},y^{(\phi)}$が(21.3.22)(21.3.23)でなければならないと結論付けることができる.

\vskip\baselineskip

ラグランジアンにスカラー場とゲージ場を含むゲージ不変項を加える必要がある.$SU(2)\times U(1)$ゲージ不変性,ローレンツ不変性,くりこみ可能性を全て満たす最も一般的な形は
\begin{align*}
\mc{L}_\phi&=-\frac{1}{2}\mc{D}_\mu \phi^\dagger \mc{D}^\mu \phi -V(\phi) \\
&=-\frac{1}{2}\left| (\partial_\mu - i\vec{A}_\mu \cdot \vec{t}^{(\phi)} - iB_\mu y^{(\phi)}) \right|^2 -\frac{\mu^2}{2}\phi^\dagger \phi -\frac{\lambda}{4}(\phi^\dagger \phi)^2 
\end{align*}
ここで$\lambda>0$で
\begin{align*}
\phi=\left(
\begin{array}{cc}
\phi^+ \\
\phi^0
\end{array}
\right)
\end{align*}
$\mu^2<0$の場合,樹木近似の真空期待値がポテンシャルの停留点
\begin{align*}
&\left.\frac{\partial V(\phi)}{\partial \phi} \right|_{\phi=v}=0 \\
\Rightarrow \quad & \left[ \frac{\mu^2}{2}\phi^\dagger +\frac{\lambda}{2}\phi^\dagger (\phi^\dagger \phi) \right]_{\phi=v}=\frac{\lambda}{2}\left[\phi^\dagger\left\{\frac{\mu^2}{\lambda}+(\phi^\dagger \phi) \right\}\right]_{\phi=v}=0 \\
\Rightarrow \quad & \braket{\phi}^\dagger \braket{\phi}=\frac{-\mu^2}{\lambda}=\frac{|\mu^2|}{\lambda}=v^2
\end{align*}
に存在する.($\mu^2>0$なら$\braket{\phi}=0$となる.)生成子$q^{(\phi)}$は破れていないので,
\begin{align*}
\sum_{m}q^{(\phi)}_{nm}v_m =e\left(
\begin{array}{cc}
1 & 0 \\
0 & 0
\end{array}
\right)\left(
\begin{array}{cc}
\braket{\phi^+} \\
\braket{\phi^0}
\end{array}
\right)=0
\end{align*}
が要請され,したがって真空期待値$\braket{\phi^+}$はゼロとなり,$\braket{\phi^0}=v>0$となる.($v>0$の条件は決して自明ではなく,本当は$v=R\exp(i\theta)$として計算すべきである.標準模型に限り,かつスカラー2重項が一つだけの場合に\uwave{のみ},最終的に$\theta$依存性が消えることが分かっており,ここではそれを見越して$\theta=0$と置いている.今回以外の場合では$\theta$依存性が出てくる可能性が大きいため,何の断りもなく$\theta=0$とおくことはできない.)さて,$SU(2)\times U(1)$ゲージ変換をして常にユニタリーゲージに移ることができ,その条件式は(21.1.2)で与えらえる.今回の$\phi$は複素場であるから,3巻p286の下から2$\ell$の条件
\begin{align*}
\mr{Im}\left[\tilde{\phi}^\dagger t_\alpha^{(\phi)}\braket{\phi}\right]=\mr{Im}\left[ ({\phi^+}^\dagger,{\phi^0}^\dagger)t_\alpha^{(\phi)}\left(
\begin{array}{cc}
0 \\
v
\end{array}
\right) \right]=0
\end{align*}
を用いる必要がある.$t_\alpha^{(\phi)}=\vec{t}^{(\phi)},y^{(\phi)}$だから
\begin{align*}
t_1^{(\phi)}:\quad&\mr{Im}\left[\frac{g}{2} ({\phi^+}^\dagger,{\phi^0}^\dagger)\left(
\begin{array}{cc}
0 & 1 \\
1 & 0
\end{array}
\right)\left(
\begin{array}{cc}
0 \\
v
\end{array}
\right) \right]=\frac{gv}{2}\mr{Im}\left[ {\phi^+}^\dagger \right]=0 \quad \Rightarrow \quad \mr{Im}\phi^+ =0 \\
t_2^{(\phi)}:\quad &\mr{Im}\left[\frac{g}{2} ({\phi^+}^\dagger,{\phi^0}^\dagger)\left(
\begin{array}{cc}
1 & 0 \\
0 & -1
\end{array}
\right)\left(
\begin{array}{cc}
0 \\
v
\end{array}
\right) \right]=- \frac{gv}{2}\mr{Im}\left[{\phi^0}^\dagger \right]=0 \quad \Rightarrow \quad \mr{Im}\phi^0 =0 \\
t_3^{(\phi)}:\quad&\mr{Im}\left[-\frac{g'}{2} ({\phi^+}^\dagger,{\phi^0}^\dagger)\left(
\begin{array}{cc}
1 & 0 \\
0 & 1
\end{array}
\right)\left(
\begin{array}{cc}
0 \\
v
\end{array}
\right) \right]=-\frac{g'v}{2}\mr{Im}\left[{\phi^0}^\dagger \right]=0 \quad \Rightarrow \quad \mr{Im}\phi^0 =0 \\
y^{(\phi)}:\quad&\mr{Im}\left[\frac{g}{2} ({\phi^+}^\dagger,{\phi^0}^\dagger)\left(
\begin{array}{cc}
0 & -i \\
i & 0
\end{array}
\right)\left(
\begin{array}{cc}
0 \\
v
\end{array}
\right) \right]=-\frac{gv}{2}\mr{Im}\left[i{\phi^+}^\dagger \right]=0 \quad \Rightarrow \quad \mr{Re}\phi^+ =0
\end{align*}
よりユニタリーゲージでは$\phi^+=0$かつ$\phi^0$が実となる.\par
するとスカラーのラグランジアン(21.3.25)から,ベクトル中間子の質量項が分かる.$\phi^0(x)=v+\sigma(x)$として($\sigma(x)$はヒッグス場と呼ばれる)ベクトル場の二次の項のみを取り出すと
\begin{align*}
&-\frac{1}{2}\left| \left(\vec{A}_\mu \cdot \vec{t}^{(\phi)}+B_\mu y^{(\phi)} \right)\braket{\phi} \right|^2 =-\frac{1}{2}\left| \left(\frac{g}{2}\vec{A}_\mu \cdot \vec{\sigma}-\frac{g'}{2}B_\mu I \right)\left(
\begin{array}{cc}
0 \\
v
\end{array}
\right) \right|^2 \\
=&-\frac{1}{2}\left| \left\{\frac{g}{2}\left(
\begin{array}{cc}
0 & A_{1\mu}-iA_{2\mu} \\
A_{1\mu}+iA_{2\mu} & 0
\end{array}
\right) +\frac{g}{2}A_{3\mu} \left(
\begin{array}{cc}
1 & 0 \\
0 & -1
\end{array}
\right)-\frac{g'}{2}B_\mu \left(
\begin{array}{cc}
1 & 0 \\
0 & 1
\end{array}
\right) \right\}\left(
\begin{array}{cc}
0 \\
v
\end{array}
\right) \right|^2 \\
=&-\frac{1}{2}\left| \frac{g}{2}(A_{1\mu}-iA_{2\mu})\left(
\begin{array}{cc}
v \\
0
\end{array}
\right) +\frac{g}{2}A_{3\mu}\left(
\begin{array}{cc}
0 \\
-v
\end{array}
\right) -\frac{g'}{2}B_\mu \left(
\begin{array}{cc}
0 \\
v
\end{array}
\right) \right|^2 \\
=&-\frac{1}{2}\left| \frac{g}{2}\sqrt{2}W_{+\mu}\left(
\begin{array}{cc}
v \\
0
\end{array}
\right) -\frac{g}{2}(\cos\theta Z_\mu -\sin\theta A_\mu)\left(
\begin{array}{cc}
0 \\
v
\end{array}
\right) -\frac{g'}{2}(\sin\theta Z_\mu+\cos\theta A_\mu ) \left(
\begin{array}{cc}
0 \\
v
\end{array}
\right) \right|^2 \\
=&-\frac{1}{2}\left| \frac{g}{\sqrt{2}}W_{+\mu}\left(
\begin{array}{cc}
v \\
0
\end{array}
\right) -\left( \frac{g}{2}\cos\theta +\frac{g'}{2}\sin\theta \right)Z_{\mu}\left(
\begin{array}{cc}
0 \\
v
\end{array}
\right) +\underset{(21.3.19)よりゼロ}{\uwave{\left( \frac{g}{2}\sin\theta-\frac{g'}{2}\cos\theta \right)}}A_\mu \left(
\begin{array}{cc}
0 \\
v
\end{array}
\right) \right|^2 \\
=&-\frac{1}{2}\left| \frac{g}{\sqrt{2}}W_{+\mu}\left(
\begin{array}{cc}
v \\
0
\end{array}
\right) +\frac{e}{2}\frac{g^2+g'^2}{gg'}Z_{\mu}\left(
\begin{array}{cc}
0 \\
v
\end{array}
\right) \right|^2 \\
=&-\frac{1}{2}\left| \frac{g}{\sqrt{2}}W_{+\mu}\left(
\begin{array}{cc}
v \\
0
\end{array}
\right) +\frac{1}{2}\sqrt{g^2+g'^2}Z_{\mu}\left(
\begin{array}{cc}
0 \\
v
\end{array}
\right) \right|^2 \quad \because e=\frac{gg'}{\sqrt{g^2+g'^2}} \\
=&-\frac{v^2 g^2}{4}W_{+\mu}^\dagger W^\mu_+ -\frac{v^2 (g^2+g'^2)}{8}Z_\mu Z^\mu =-\frac{v^2 g^2}{4}W_{-\mu} W^\mu_+ -\frac{v^2 (g^2+g'^2)}{8}Z_\mu Z^\mu \\
\equiv& -m^2_W W_{-\mu}W^\mu_+ -\frac{1}{2}m^2_Z Z_\mu Z^\mu
\end{align*}
よって
\begin{align*}
m_W=\frac{v|g|}{2} ,\quad m_Z=\frac{v\sqrt{g^2+g'^2}}{2}
\end{align*}
となる.(ワインバーグ本文での式変形通りに計算するとこのようになるが,電荷行列$q^{(\phi)}$が破れていないので真空期待値$(0,v)$をかけるとゼロになることを用いることで$A^\mu$の質量が現れないという仕組みが根底にある.実際,$i\mc{L}'_e$を計算したときと同様の計算によって
\begin{align*}
-\frac{1}{2}\left| \left(\vec{A}_\mu \cdot \vec{t}^{(\phi)}+B_\mu y^{(\phi)} \right)\braket{\phi} \right|^2&=-\frac{1}{2}\left| \biggl\{ \frac{1}{\sqrt{2}}\Slash{W}_- (t_1^{(\phi)}-it_2^{(\phi)}) +\frac{1}{\sqrt{2}}\Slash{W}_+ (t_1^{(\phi)}+it_2^{(\phi)}) \right. \\
&\left. +\Slash{Z}(t_3^{(\phi)}\cos\theta + y^{(\phi)}\sin\theta)+\Slash{A}(-t_3^{(\phi)} \sin\theta +y^{(\phi)}\cos\theta)\biggr\}\left(
\begin{array}{cc}
0 \\
v
\end{array}
\right) \right|^2 \\
&=-\frac{1}{2}\left| \biggl\{ \frac{1}{\sqrt{2}}\Slash{W}_- (t_1^{(\phi)}-it_2^{(\phi)}) +\frac{1}{\sqrt{2}}\Slash{W}_+ (t_1^{(\phi)}+it_2^{(\phi)}) \right. \\
&\left. +\Slash{Z}(t_3^{(\phi)}\cos\theta + y^{(\phi)}\sin\theta)+\Slash{A}q^{(\phi)} \biggr\}\left(
\begin{array}{cc}
0 \\
v
\end{array}
\right) \right|^2
\end{align*}
となるからだ.あとは同じ計算で同様の結果が得られる.)実場は質量行列に係数$1/2$がつくことに注意.なぜなら質量行列はラグランジアンの場についての二階微分で与えられるが
\begin{align*}
&複素場: \phi^\dagger M^2 \phi \overset{\partial/\partial \phi}{\longrightarrow} \phi^\dagger M^2 \overset{\partial/\partial \phi^\dagger}{\longrightarrow} M^2 \\
&実場 : \frac{1}{2}\phi M^2 \phi \overset{\partial/\partial \phi}{\longrightarrow} \phi M^2 \overset{\partial/\partial \phi}{\longrightarrow} M^2
\end{align*}
として計算されるからだ.また(21.3.21)は(21.3.28)より
\begin{align*}
&-G_e \overline{\left(
\begin{array}{cc}
\nu_e \\
e
\end{array}
\right)}_L \left(
\begin{array}{cc}
0 \\
v
\end{array}
\right)e_R-G_e \bar{e}_R(0 ,v)\left(
\begin{array}{cc}
\nu_e \\
e
\end{array}
\right) \\
&=-G_e \bar{e}_L v e_R -G_e \bar {e}_R v e_L=-G_e v \bar{e} \frac{1-\gamma_5}{2}e -G_e v \bar{e} \frac{1+\gamma_5}{2}e \\
&=-G_e v \bar{e} e \equiv -m_e \bar{e}e
\end{align*}
となり,電子の質量$m_e =G_e v$がわかる.
がわかる.(誤植を訂正して)(21.3.20)と合わせると,$\bar{e}A e$項と$\bar{e}e$項は
\begin{align*}
-\bar{e}(\Slash{\partial}-i(-e)\Slash{A}+G_e v)e =-\bar{e}(\Slash{\mc{D}}_e+m_e )e
\end{align*}
となる.ここで$\mc{D}^\mu_e$はQEDにおける見慣れた共変微分だ.\par
$\Rightarrow$期待した通り,光子の質量はゼロで,$W^\pm$と$Z^0$と電子の質量が生じる!これが欲しかった結果だ!質量発生の起源は自発的対称性の破れに伴う(ゼロでない)真空期待値自体なのだとわかる.\par
また,(21.3.25)において$\phi=(0,v+\sigma(x))$と分解すると,ポテンシャル項において$\sigma(x)$の二次の項が
\begin{align*}
-\frac{\mu^2}{2}(v+\sigma)^2-\frac{\lambda}{4}(v+\sigma)^4=&-\frac{\mu^2}{2}(v^2+2v\sigma+\sigma^2)-\frac{\lambda}{4}\{v^4+4v^3\sigma+6v^2\sigma^2+4v\sigma^3+\sigma^4 \} \\
=&-\left(\frac{\mu^2}{2}+\frac{\lambda}{4}6v^2\right)\sigma^2+\cdots  \\
=&-\left( -\frac{\lambda}{2}v^2+\frac{3\lambda}{2}v^2 \right)\sigma^2+\cdots \quad \because v^2=-\mu^2/\lambda \\
=&-\lambda v^2\sigma^2+\cdots\equiv -\frac{m_{H}^2}{2}\sigma^2+\cdots
\end{align*}
となり,ヒッグス場$\sigma(x)$の質量は$m_H=v\sqrt{2\lambda}$であることがわかる.

\vskip\baselineskip

(21.3.20)より$W^\pm$に結合するカレントは
\begin{align*}
&\frac{g}{\sqrt{2}}\left(\bar{e}\Slash{W}_-\left(\frac{1+\gamma_5}{2}\right)\nu_e \right)+\frac{g}{\sqrt{2}}\left(\bar{\nu}_e\Slash{W}_+\left(\frac{1+\gamma_5}{2}\right) e \right) \\
&=W_{-\mu}J^\mu_e+W_{+\mu}{J^\mu_e}^\dagger
\end{align*}
ここで
\begin{align*}
J^\mu_e\equiv \frac{g}{\sqrt{2}}\left(\bar{e}\gamma^\mu \left(\frac{1+\gamma_5}{2}\right)\nu_e \right)
\end{align*}
となる.

\begin{figure}[H]
  \centering
\begin{tikzpicture}[decoration={markings, 
mark= at position -1cm with {\arrow[line width=0.5mm]{Stealth}}}]
\coordinate (a1) at (-2,-2){};
\coordinate (b2) at (2,3){};
\coordinate (m1) at (-1,0){};
\coordinate (b1) at (-2,2){};
\coordinate (a2) at (0,3){};
\coordinate (m2) at (0.5,1){};

\draw[thick,postaction={decorate}](b1)node[left]{$\bar{\nu}_\mu$}--(m1);
\draw[thick,postaction={decorate}](m1)--(a1)node[right]{$\mu^+$};

\draw[thick,postaction={decorate}](b2)node[above=2mm]{$e^+$}--(m2);
\draw[thick,postaction={decorate}](m2)--(a2)node[right]{$\bar{\nu}_e$};

\begin{feynhand}
\propag[photon,thick](m1)--(m2);
\end{feynhand}
\node(g) at (0,0){$W$};

\draw[very thick , ->](-3,-2)--(-3,3)node[above]{$t$};
\end{tikzpicture}
\end{figure}

この$J^\mu$で$\mu^+\to e^+ +\nu_e +\bar{\nu}_\mu$を考えると,$e\to \nu$という置き換えと$G_\mu=(m_\mu/m_e)G_e$で$J^\mu_\mu$が同様に得られるので,低エネルギー$(p^2\ll m^2_W)$の$e$型レプトンと$\mu$型レプトンの間での$W$交換は,
\begin{align*}
J^\mu_e \Delta_{\mu\nu}{J^\nu_\mu}^\dagger +\mr{H.c.}&=\left( \frac{g}{\sqrt{2}} \right)^2 \left(\bar{e}\gamma^\mu \left(\frac{1+\gamma_5}{2}\right)\nu_e \right)\frac{\eta_{\mu\nu}}{p^2+m_W^2}\left(\bar{\nu}_\mu\gamma^\nu \left(\frac{1+\gamma_5}{2}\right)\mu \right)+\mr{H.c.} \\
&\to \left( \frac{g}{\sqrt{2}} \right)^2\frac{1}{m^2_W} \left(\bar{e}\gamma^\mu \left(\frac{1+\gamma_5}{2}\right)\nu_e \right)\left(\bar{\nu}_\mu\gamma_\mu \left(\frac{1+\gamma_5}{2}\right)\mu \right)+\mr{H.c.}
\end{align*}
ここで分かりやすくするためファインマンゲージ$\xi=1$を用いた.ゲージによって物理的粒子の質量は変わらないのを21.2節で見たので,これは自由に行ってよい.これは低エネルギーで$\mu$中間子崩壊を良く記述することが知られている有効$V-A$理論の相互作用
\begin{align*}
\frac{G_F}{\sqrt{2}} \left(\bar{e}\gamma^\mu \left(1+\gamma_5\right)\nu_e \right)\left(\bar{\nu}_\mu\gamma_\mu \left(1+\gamma_5\right)\mu \right)+\mr{H.c.}
\end{align*}
と比較できる.ここで$G_F$は通常のフェルミ結合定数で,$\mu$中間子崩壊率より$G_F=1.16639(2)\times 10^{-5}$GeVが知られている.これらを比較し
\begin{align*}
\frac{g^2}{m_W^2}=4\sqrt{2}G_F
\end{align*}
がわかる.これより$G_F$が既知であるから(21.3.30)より真空期待値がわかる.
\begin{align*}
v=\frac{2m_W}{g}=\frac{1}{2^{1/4}\sqrt{G_F}}=247\mr{GeV}
\end{align*}
また(21.3.31)より
\begin{align*}
G_e=\frac{0.511\mr{MeV}}{247\mr{GeV}}=2.07\times 10^{-6}
\end{align*}
がわかる.これはとても小さい値だ.また(21.3.30)より
\begin{align*}
&m_Z=\frac{v\sqrt{g^2+g'^2}}{2}=\frac{v|g|}{2|\cos\theta|}=\frac{m_W}{|\cos\theta|}<m_W \\
&m_W=\frac{v|g|}{2}=\frac{ev}{2|\sin\theta|}=\frac{37.4\mr{MeV}}{|\sin\theta|} \\
&m_Z=\frac{m_W}{|\cos\theta|}=\frac{ev}{2|\sin\theta||\cos\theta|}=\frac{74.8\mr{MeV}}{|\sin2\theta|}
\end{align*}
がわかる.$\theta$はワインバーグ角と呼ばれる.(たびたび$\theta_W$と表記.)

\vskip\baselineskip

$m_W$と$m_Z$の値は$e$については従来の電荷の定義を用いて計算された.しかし,18.2節で説明したように,これは$E\gg m_e$のエネルギーでの過程での計算に用いるには正確には適切ではない.\par
$\Rightarrow$代わりに,対象としているエネルギースケールに匹敵する,変化するスケール$\mu$での電荷$e_\mu$を使う必要がある.\\
90GeV程度の$\mu$では有効微細構造定数$e^2_\mu/4\pi$は約$1/129$(3巻p172参照)なので,$m_Z,m_W$は
\begin{align*}
e_\mu=\sqrt{\frac{4\pi}{129}}=\sqrt{\frac{137}{129}}\sqrt{\frac{4\pi}{137}}=\sqrt{\frac{137}{129}}e
\end{align*}
で$\sqrt{137/129}$をかけなければならず
\begin{align*}
m_W=\frac{38.5 \mr{GeV}}{|\sin \theta|} ,\quad  m_Z=\frac{77.1  \mr{GeV}}{|\sin 2\theta|}
\end{align*}
となる.

\vskip\baselineskip

中性カレント($Z^0$に結合するカレント)の発見以前に,すでに電弱理論はハドロン・ハドロン間およびハドロン・レプトン間の弱い相互作用と電磁相互作用に拡張されていた.1960年代中頃までには,レプトン・ハドロン間で電荷が交換される弱い相互作用の過程は,低エネルギーでは有効ラグランジアン
\begin{align*}
\frac{G_F}{\sqrt{2}}[\bar{e}\gamma_\lambda (1+\gamma_5)\nu_e +\bar{\mu}\gamma_\lambda(1+\gamma_5)\nu_\mu]J^\lambda+\mr{H.c.}
\end{align*}
で良く記述されることが分かっていた.ここで$J^\lambda$はハドロン・カレントだ.このカレントは,クォークカレント
\begin{align*}
J^\lambda =\bar{u}\gamma^\lambda (1+\gamma_5)d\cos\theta_c+\bar{u}\gamma^\lambda (1+\gamma_5)s\sin\theta_c
\end{align*}
と同定できた.$\theta_c$はキャビボ角と呼ばれる,ワインバーグ角とは別の角度だ.第一世代のみを含む相互作用を考える場合は,このカレントは
\begin{align*}
J^\lambda=\bar{u}\gamma^\lambda (1+\gamma_5)d
\end{align*}
となるが,このときの相互作用はまさに(19.4.22)だ.実際(19.4.22)は
\begin{align*}
& V^\lambda_+=V^\lambda_1+iV^\lambda_2=i\overline{\left(
\begin{array}{cc}
u \\
d
\end{array}
\right)}\gamma^\lambda \left(
\begin{array}{cc}
0 & 1 \\
0 & 0
\end{array}
\right)\left(
\begin{array}{cc}
u \\
d
\end{array}
\right)=i\bar{u}\gamma^\lambda d \\
& A^\lambda_+=A^\lambda_1+iA^\lambda_2=i\overline{\left(
\begin{array}{cc}
u \\
d
\end{array}
\right)}\gamma^\lambda \gamma_5 \left(
\begin{array}{cc}
0 & 1 \\
0 & 0
\end{array}
\right)\left(
\begin{array}{cc}
u \\
d
\end{array}
\right)=i\bar{u}\gamma^\lambda \gamma_5 d \\
\Rightarrow \quad & V^\lambda_+ + A^\lambda_+=i\bar{u}\gamma^\lambda(1+\gamma_5)d \\
& \mc{L}_{\mr{wk}}=\frac{G_{\mr{wk}}}{\sqrt{2}}\bar{u}\gamma^\lambda (1+\gamma_5)d \sum_{\ell}\bar{\ell} \gamma_\lambda (1+\gamma_5)\nu_\ell +\mr{H.c.}
\end{align*}
となる.\par
第一世代$(\nu_e,e,u,d)$だけの場合はこれでよい.これを第二世代$(\nu_\mu,\mu,c,s)$,第三世代$(\nu_\tau,\tau ,t,b)$を含んだものに拡張したい.しかし,世代が違う粒子と相互作用しないのはレプトンに関してのみであり,クォークに関してはそう単純ではない.例えば,第二世代に属するクォークは$c$クォークと$s$クォークである.これらの粒子について対応する第一世代の$u$クォーク,$d$クォークと同じ相互作用を足し合わせると,弱い相互作用の荷電を変える遷移は$s$クォークと$c$クォークの遷移として表される.
\begin{align*}
J^\lambda=\bar{u}\gamma^\lambda (1+\gamma_5)d+\bar{c}\gamma^\lambda (1+\gamma_5)s
\end{align*}
しかし,$\Lambda$粒子から陽子への崩壊はクォークレベルでは$s$クォークから$u$クォークへの遷移であり,第一世代の相互作用に第二世代を単に足し合わせるだけでは記述できない.\par
$\Rightarrow$ここまで展開してきたゲージ理論を保持しつつ,現実を正しく記述するためには,弱い相互作用の電荷を変えるカレントについて\uwave{クォークの世代間混合}が必要だ.すなわち,フレーバーで区別されるクォークと,電弱理論の$SU(2)$2重項・1重項を構成するクォークは\uwave{同一ではない}と考える.\par
前者を$d,s$,後者$d',s'$で表すと,このときクォークの混合は
\begin{align*}
\left(
\begin{array}{cc}
d' \\
s'
\end{array}
\right)=\left(
\begin{array}{cc}
\cos\theta_c & \sin\theta_c \\
-\sin\theta & \cos\theta_c
\end{array}
\right)\left(
\begin{array}{cc}
d \\
s
\end{array}
\right)  \quad \Rightarrow \quad d'=d\cos\theta_c s\sin\theta_c ,\quad s'=-d\sin\theta_c+s\cos\theta_c
\end{align*}
と表される.もし第三世代までの混合をするなら
\begin{align*}
\left(
\begin{array}{ccc}
d' \\
s' \\
b'
\end{array}
\right)=V_{\mr{CKM}}\left(
\begin{array}{ccc}
d \\
s \\
b
\end{array}
\right) ,\quad V_{\mr{CKM}}=\left(
\begin{array}{ccc}
V_{ud} & V_{us} & V_{ub} \\
V_{cd} & V_{cs} & V_{cb} \\
V_{td} & V_{ts} & V_{tb} 
\end{array}
\right)
\end{align*}
となる.これはキャビボ・小林・益川行列と呼ばれるユニタリー行列だ.これは未知の行列だが,p86の行列で知られている.以上は$d,s,b$の混合を表し,$u,c,t$の間の混合はないとしているが,これは便宜的な表現であり,逆で混合させることもできる.また,電弱理論ではレプトンの質量行列は対角であり,レプトンの混合はないとしている.(理論によってはレプトンの混合も考えることがあり,Pontecorvo・牧・中川・坂田行列(PMNS行列)と呼ぶらしい.荷電レプトンは質量フレーバー固有状態とするが,ニュートリノに関してはPMNS行列により混合させることでニュートリノ振動を考えることができるらしい.)

\vskip\baselineskip

$u,d,s$クォークを考えたとき,混合状態$d'=d\sin\theta_c +s\sin\theta_c$は$u$と合わせて$SU(2)_+\times U(1)$左手2重項
\begin{align*}
&\mc{Q}_1=\left(
\begin{array}{cc}
u \\
d'
\end{array}
\right)_L=\frac{1+\gamma_5}{2}\left(
\begin{array}{cc}
u \\
d\cos\theta_c+s\sin\theta_c
\end{array}
\right) \\
&\delta \left(
\begin{array}{cc}
u \\
d'
\end{array}
\right)_L=i\left[ \vec{\epsilon}\cdot\vec{t}+\alpha y \right]\left(
\begin{array}{cc}
u \\
d'
\end{array}
\right)_L
\end{align*}
および右手成分の1重項
\begin{align*}
&u_R=\frac{1-\gamma_5}{2}u ,\quad d'_R=\frac{1-\gamma_5}{2}d' \\
&\delta u_R=i[\alpha y]u_R ,\quad \delta d'_R=i[\alpha y]d'_R
\end{align*}
を構成する.\par
クォークの電荷を$2e/3,-e/3$と与えるようにすると,$\vec{t}=g(1+\gamma_5)\vec{\sigma}/4$と$q=\frac{e}{g}t_3-\frac{e}{g'}y$より
\begin{align*}
&q\left(
\begin{array}{cc}
u_L \\
d'_L
\end{array}
\right)=e\left(\begin{array}{cc}
+2/3\cdot u_L \\
-1/3\cdot d'_L
\end{array}
\right) ,\quad q u_R=+\frac{2e}{3}u_R ,\quad q d'_R=-\frac{e}{3}d'_R \\
&t_3 \left(
\begin{array}{cc}
u_L \\
d'_L
\end{array}
\right)=\frac{g}{2}\left(
\begin{array}{cc}
1 & 0 \\
0 & -1
\end{array}
\right)\left(
\begin{array}{cc}
u_L \\
d'_L
\end{array}
\right)=\frac{g}{2}\left(
\begin{array}{cc}
+u_L \\
-d'_L
\end{array}
\right) ,\quad t_3 u_R=0 ,\quad t_3 d'_R=0 \\
\quad \Rightarrow \quad & \frac{e}{g'}y\left(
\begin{array}{cc}
u_L \\
d'_L
\end{array}
\right)=\left(\frac{e}{g'}t_3-q \right)\left(
\begin{array}{cc}
u_L \\
d'_L
\end{array}
\right)=\left(
\begin{array}{cc}
(+e/2-2e/3)u_L \\
(-e/2+e/3)d'_L
\end{array}
\right)=-\frac{e}{6}\left(
\begin{array}{cc}
u_L \\
d'_L
\end{array}
\right) \\
&\frac{e}{g'}y u_R=-\frac{2e}{3}u_R ,\quad \frac{e}{g'}y d'_R=\frac{e}{3}d'_R
\end{align*}
という風にクォーク場においての$y$の値が調整される.

\begin{table}[H]
  \centering
  \begin{tabular}{c|ccccccc}
    & & $q$ & $\dfrac{e}{g}t_3$ & $-\dfrac{e}{g'}y$ & 第一世代 & 第二世代 & 第三世代\\\hline
    & & & & & & & \\
    レプトン & 2重項 & \multirow{2}{*}{$\begin{pmatrix}0\\-1\end{pmatrix}$} & \multirow{2}{*}{$\begin{pmatrix}+\frac{1}{2}\\-\frac{1}{2}\end{pmatrix}$} & \multirow{2}{*}{$\begin{pmatrix}-\frac{1}{2}\\-\frac{1}{2}\end{pmatrix}$} & \multirow{2}{*}{$\begin{pmatrix}V_e\\e\end{pmatrix}_L$} & \multirow{2}{*}{$\begin{pmatrix}V_\mu\\\mu\end{pmatrix}_L$} & \multirow{2}{*}{$\begin{pmatrix}V_\tau\\\tau\end{pmatrix}_L$}\\
    & & & & & & & \\\vspace{-0.8cm}
    & & & & & & & \\
    & & & & & & & \\
    & 1重項 & $-1$ & $0$ & $-1$ & $e_R$ & $\mu_R$ & $\tau_R$\\
    & & & & & & & \\
    クォーク & 2重項 & \multirow{2}{*}{$\begin{pmatrix}+\frac{2}{3}\\-\frac{1}{3}\end{pmatrix}$} & \multirow{2}{*}{$\begin{pmatrix}+\frac{1}{2}\\-\frac{1}{2}\end{pmatrix}$} & \multirow{2}{*}{$\begin{pmatrix}+\frac{1}{6}\\+\frac{1}{6}\end{pmatrix}$} & \multirow{2}{*}{$\begin{pmatrix}u\\d'\end{pmatrix}_L$} & \multirow{2}{*}{$\begin{pmatrix}c\\s'\end{pmatrix}_L$} & \multirow{2}{*}{$\begin{pmatrix}t\\b'\end{pmatrix}_L$}\\
    & & & & & & & \\\vspace{-0.1cm}
    & & & & & & & \\
    & 1重項 & \multirow{2}{*}{$\begin{matrix}+\frac{2}{3}\\-\frac{1}{3}\end{matrix}$} & \multirow{2}{*}{$\begin{matrix}0\\0\end{matrix}$} & \multirow{2}{*}{$\begin{matrix}+\frac{2}{3}\\-\frac{1}{3}\end{matrix}$} &\multicolumn{3}{c}{$u_R\hspace{1.3cm}c_R\hspace{1.4cm}t_R$}\\\vspace{-0.6cm}
    & & & & & & & \\\vspace{0.2cm}
    & & & & & \multicolumn{3}{c}{$\tcboxmath[
      enhanced,frame hidden,interior hidden,size=minimal,
      overlay={
          \draw[thick,decorate,
              decoration={brace,         amplitude=6pt,raise=2pt}
          ] (frame.south east) -- (frame.south west)node[midway,below=8pt] {混合状態};
      }
  ]{d'_R\hspace{1.3cm}s'_R\hspace{1.4cm}b'_R}$}\\
  \multicolumn{1}{c}{}& & \multicolumn{3}{c}{$Q\hspace{0.4cm}=\hspace{0.4cm}T_3\hspace{0.4cm}+\hspace{0.4cm}\underset{弱超電荷}{Y}$} &&&
  \end{tabular}
\end{table}


\vskip\baselineskip

相互作用ラグランジアンは,クォーク2重項が(21.3.14)だけのとき
\begin{align*}
t_3\cos\theta +y \sin\theta& = t_3 \frac{1-\sin^2\theta}{\cos\theta}+y\sin\theta\\
&=t_3\frac{1}{\cos\theta}+ (-\sin\theta t_3+\cos\theta y)\frac{\sin\theta}{\cos\theta} \\ 
&=t_3\sec\theta +q\tan\theta \\
&=-t_3\frac{\sqrt{g^2+g'^2}}{g}+q\frac{g'}{g}
\end{align*}
を用いると
\begin{align*}
&\overline{\left(
\begin{array}{cc}
u_L \\
d'_L
\end{array}
\right)}\left[\sum_\alpha A_\alpha t^\alpha \right]\left(
\begin{array}{cc}
u_L \\
d'_L
\end{array}
\right)+\bar{u}_R \left[\sum_\alpha A_\alpha t^\alpha \right]u_R+\bar{d}'_R \left[\sum_\alpha A_\alpha t^\alpha \right]d'_R \\
=&\overline{\left(
\begin{array}{cc}
u_L \\
d'_L
\end{array}
\right)}\left[ \frac{1}{\sqrt{2}}\Slash{W}_- (t_1-it_2)+ \frac{1}{\sqrt{2}}\Slash{W}_+ (t_1+it_2) + \Slash{Z}(t_3\cos\theta+y\sin\theta)+\Slash{A} q \right]\left(
\begin{array}{cc}
u_L \\
d'_L
\end{array}
\right) \\
&+\bar{u}_R\left[ \Slash{Z}q\tan\theta +q\Slash{A} \right]u_R+\bar{d}'_R\left[ \Slash{Z}q\tan\theta +q\Slash{A} \right]d'_R \\
=&\overline{\left(
\begin{array}{cc}
u_L \\
d'_L
\end{array}
\right)}\left[ \frac{g}{\sqrt{2}}\Slash{W}_- \frac{1+\gamma_5}{2}\left(
\begin{array}{cc}
0 & 1 \\
0 & 0
\end{array}
\right)+\frac{g}{\sqrt{2}}\Slash{W}_+ \frac{1+\gamma_5}{2}\left(
\begin{array}{cc}
0 & 0 \\
1 & 0
\end{array}
\right)+\Slash{Z}(t_3\sec\theta +q\tan\theta )+\Slash{A}q \right]\left(
\begin{array}{cc}
u_L \\
d'_L
\end{array}
\right) \\
&+\bar{u}_R\left[ \Slash{Z}q\tan\theta +q\Slash{A} \right]u_R+\bar{d}'_R\left[ \Slash{Z}q\tan\theta +q\Slash{A} \right]d'_R \\
=&\frac{g}{\sqrt{2}}\bar{u}_L \Slash{W}_- \left(\frac{1+\gamma_5}{2}\right)d'_L+\frac{g}{\sqrt{2}}\bar{d}'_L \Slash{W}_- \left(\frac{1+\gamma_5}{2}\right)u_L \\
&+\frac{g}{2} \bar{u}_L \Slash{Z}\sec\theta \left(\frac{1+\gamma_5}{2}\right)u_L-\frac{g}{2} \bar{d}'_L \Slash{Z}\sec\theta \left(\frac{1+\gamma_5}{2}\right)d'_L \\
&+\frac{2e}{3}\bar{u}_L\Slash{Z}\tan\theta u_L-\frac{e}{3}\bar{d}'_L\Slash{Z}\tan\theta d'_L+\frac{2e}{3}\bar{u}_L\Slash{A}u_L-\frac{e}{3}\bar{d}'_L\Slash{A}d'_L \\
&+\frac{2e}{3}\bar{u}_R\Slash{Z}\tan\theta u_R-\frac{e}{3}\bar{d}'_R\Slash{Z}\tan\theta d'_R+\frac{2e}{3}\bar{u}_R\Slash{A}u_R-\frac{e}{3}\bar{d}'_R\Slash{A}d'_R \\
=&W_{-\mu}\left[ \frac{g}{\sqrt{2}}\bar{u} \gamma^\mu \left(\frac{1+\gamma_5}{2}\right)d'\right] +W_{+\mu}\left[\frac{g}{\sqrt{2}}\bar{d}' \gamma^\mu \left(\frac{1+\gamma_5}{2}\right)u\right] \\
&+\frac{g}{2} \bar{u} \Slash{Z}\sec\theta \left(\frac{1+\gamma_5}{2}\right)u-\frac{g}{2} \bar{d}' \Slash{Z}\sec\theta \left(\frac{1+\gamma_5}{2}\right)d' \\
&+\frac{2e}{3}\bar{u}\Slash{Z}\tan\theta u-\frac{e}{3}\bar{d}'\Slash{Z}\tan\theta d'+\frac{2e}{3}\bar{u}\Slash{A}u-\frac{e}{3}\bar{d}'\Slash{A}d'
\end{align*}
$W$ボゾン交換の相互作用は,$W$に結合するカレントを用いて低エネルギーにおいて
\begin{align*}
&\left(\frac{g}{\sqrt{2}}\right)^2\frac{1}{m_W^2}\left[\bar{e}\gamma^\lambda\left(\frac{1+\gamma_5}{2}\right)\nu_e+\bar{\mu}\gamma^\lambda\left(\frac{1+\gamma_5}{2}\right)\nu_\mu \right]\left[ \bar{u}\gamma_\lambda \left(\frac{1+\gamma_5}{2}\right)d' \right]+\mr{H.c.} \\
&=\frac{G_F}{\sqrt{2}}\left[\bar{e}\gamma^\lambda\left(1+\gamma_5\right)\nu_e+\bar{\mu}\gamma^\lambda\left(1+\gamma_5\right)\nu_\mu \right]\left[ \bar{u}\gamma_\lambda \left(1+\gamma_5\right)d\cos\theta_c + \bar{u}\gamma_\lambda \left(1+\gamma_5\right)s\sin\theta_c \right]+\mr{H.c.}
\end{align*}
となって(21.3.39)が出てくる.すなわちクォーク二重項によって相互作用が説明できた.\par
しかしこの相互作用の$Z$と結合する項を見ると,$d'=d\cos\theta_c +s\sin\theta_c$より$\bar{d}\gamma^\lambda(1+\gamma_5)s,\bar{s}\gamma^\lambda(1+\gamma_5)d$に比例する項が存在することに気付く.
\begin{align*}
\frac{g}{2} \bar{d}' \Slash{Z}\sec\theta \left(\frac{1+\gamma_5}{2}\right)d' =&Z_\lambda \left[\frac{g}{2} \bar{d} \gamma^\lambda \sec\theta \left(\frac{1+\gamma_5}{2}\right)d\cos^2\theta_c+\frac{g}{2} \bar{s} \gamma^\lambda \sec\theta \left(\frac{1+\gamma_5}{2}\right)s\sin^2\theta_c\right] \\
&+Z_\lambda \uwave{ \frac{g}{4}\sec\theta\sin\theta_c\cos\theta_c\left[\bar{d} \gamma^\lambda \left(1+\gamma_5\right)s +\bar{s} \gamma^\lambda \left(1+\gamma_5\right)d \right]}
\end{align*}
これはフレーバーが異なる場である$s$と$d$が結合し,中性カレントであるから,フレーバーを変える中性カレント(Flavor Changing Neutral Current,FCNC)と呼ぶ.$K^0$中間子はストレンジクォークと反ダウンクォークからなる中間子であるから,$K^0\to\bar{K}^0$振動はクォークレベルで見ると,$s+\bar{d}\to d+\bar{s}$という過程に相当する.もしこのFCNCが存在すると,この反応は4フェルミオン相互作用であるからおおざっぱに
\begin{align*}
(\bar{d}s)\frac{1}{m_Z^2}(\bar{s}d)
\end{align*}
となる.

\begin{figure}[H]
  \centering
\begin{tikzpicture}[decoration={markings, 
mark= at position -1cm with {\arrow[line width=0.5mm]{Stealth}}}]
\coordinate (a1) at (-2,2){};
\coordinate (b1) at (-2,-2){};
\coordinate (a2) at (2,2){};
\coordinate (b2) at (2,-2){};
\coordinate (m1) at (0,1){};
\coordinate (m2) at (0,-1){};

\draw[thick,postaction={decorate}](a1)node[left]{$d$}--(m1);
\draw[thick,postaction={decorate}](m2)--(b1)node[left]{$s$};
\draw[thick,postaction={decorate}](m1)--(a2)node[right]{$s$};
\draw[thick,postaction={decorate}](b2)node[right]{$d$}--(m2);

\begin{feynhand}
\propag[photon,thick](m1)--(m2);
\end{feynhand}
\draw(0,0)node[left]{$Z$};
\draw(3,0)node{$+$};

\end{tikzpicture}
\begin{tikzpicture}[decoration={markings, 
mark= at position -1cm with {\arrow[line width=0.5mm]{Stealth}}}]
\coordinate (a1) at (-2,2){};
\coordinate (b1) at (-2,-2){};
\coordinate (a2) at (2,2){};
\coordinate (b2) at (2,-2){};
\coordinate (m1) at (-1,0){};
\coordinate (m2) at (1,0){};

\draw[thick,postaction={decorate}](a1)node[left]{$d$}--(m1);
\draw[thick,postaction={decorate}](m1)--(b1)node[left]{$s$};
\draw[thick,postaction={decorate}](m2)--(a2)node[right]{$s$};
\draw[thick,postaction={decorate}](b2)node[right]{$d$}--(m2);

\begin{feynhand}
\propag[photon,thick](m1)--(m2);
\end{feynhand}
\draw(0,0)node[above]{$Z$};

\end{tikzpicture}
\end{figure}

しかし現実はこれで計算される確率では反応しない.ここで新たな二重項
\begin{align*}
\mc{Q}_2=\frac{1+\gamma_5}{2}\left(
\begin{array}{cc}
c\\
s'
\end{array}
\right)=\frac{1+\gamma_5}{2}\left(
\begin{array}{cc}
c\\
-d\sin\theta_c+s\cos\theta_c
\end{array}
\right)
\end{align*}
とその一重項を導入すると,そのラグランジアンは
\begin{align*}
&\overline{\left(
\begin{array}{cc}
u_L \\
d'_L
\end{array}
\right)}\left[\sum_\alpha A_\alpha t^\alpha \right]\left(
\begin{array}{cc}
u_L \\
d'_L
\end{array}
\right)+\bar{u}_R \left[\sum_\alpha A_\alpha t^\alpha \right]u_R+\bar{d}'_R \left[\sum_\alpha A_\alpha t^\alpha \right]d'_R \\
&+\overline{\left(
\begin{array}{cc}
c_L \\
s'_L
\end{array}
\right)}\left[\sum_\alpha A_\alpha t^\alpha \right]\left(
\begin{array}{cc}
c_L \\
s'_L
\end{array}
\right)+\bar{c}_R \left[\sum_\alpha A_\alpha t^\alpha \right]c_R+\bar{s}'_R \left[\sum_\alpha A_\alpha t^\alpha \right]s'_R \\
=&W_{-\mu}\left[ \frac{g}{\sqrt{2}}\bar{u} \gamma^\mu \left(\frac{1+\gamma_5}{2}\right)d'\right] +W_{+\mu}\left[\frac{g}{\sqrt{2}}\bar{d}' \gamma^\mu \left(\frac{1+\gamma_5}{2}\right)u\right] \\
&+W_{-\mu}\left[ \frac{g}{\sqrt{2}}\bar{c} \gamma^\mu \left(\frac{1+\gamma_5}{2}\right)s'\right] +W_{+\mu}\left[\frac{g}{\sqrt{2}}\bar{s}' \gamma^\mu \left(\frac{1+\gamma_5}{2}\right)c\right] \\
&+\frac{g}{2} \bar{u} \Slash{Z}\sec\theta \left(\frac{1+\gamma_5}{2}\right)u-\frac{g}{2} \bar{d}' \Slash{Z}\sec\theta \left(\frac{1+\gamma_5}{2}\right)d' \\
&+\frac{g}{2} \bar{c} \Slash{Z}\sec\theta \left(\frac{1+\gamma_5}{2}\right)c-\frac{g}{2} \bar{s}' \Slash{Z}\sec\theta \left(\frac{1+\gamma_5}{2}\right)s' \\
&+\frac{2e}{3}\bar{u}\Slash{Z}\tan\theta u-\frac{e}{3}\bar{d}'\Slash{Z}\tan\theta d' +\frac{2e}{3}\bar{c}\Slash{Z}\tan\theta c-\frac{e}{3}\bar{s}'\Slash{Z}\tan\theta s' \\
&+\frac{2e}{3}\bar{u}\Slash{A}u-\frac{e}{3}\bar{d}'\Slash{A}d'+\frac{2e}{3}\bar{c}\Slash{A}c-\frac{e}{3}\bar{s}'\Slash{A}s' \\
=&W_{-\mu}\left[ \frac{g}{\sqrt{2}}\left\{ \bar{u} \gamma^\mu \left(\frac{1+\gamma_5}{2}\right)d'+\bar{c} \gamma^\mu \left(\frac{1+\gamma_5}{2}\right)s' \right\}\right] \\
&+W_{+\mu}\left[\frac{g}{\sqrt{2}}\left\{ \bar{d}' \gamma^\mu \left(\frac{1+\gamma_5}{2}\right)u+\bar{s}' \gamma^\mu \left(\frac{1+\gamma_5}{2}\right)c \right\} \right] \\
&+\frac{g}{2} \bar{u} \Slash{Z}\sec\theta \left(\frac{1+\gamma_5}{2}\right)u+\frac{2e}{3}\bar{u}\Slash{Z}\tan\theta u-\frac{g}{2} \bar{d} \Slash{Z}\sec\theta \left(\frac{1+\gamma_5}{2}\right)d -\frac{e}{3}\bar{d}\Slash{Z}\tan\theta d \\
&+\frac{g}{2} \bar{c} \Slash{Z}\sec\theta \left(\frac{1+\gamma_5}{2}\right)c+\frac{2e}{3}\bar{c}\Slash{Z}\tan\theta c-\frac{g}{2} \bar{s} \Slash{Z}\sec\theta \left(\frac{1+\gamma_5}{2}\right)s -\frac{e}{3}\bar{s}\Slash{Z}\tan\theta s \\
&+\frac{2e}{3}\bar{u}\Slash{A}u-\frac{e}{3}\bar{d}\Slash{A}d+\frac{2e}{3}\bar{c}\Slash{A}c-\frac{e}{3}\bar{s}\Slash{A}s
\end{align*}
となり,ストレンジネスを変える項が相殺される.すぐに分かる通り,$Z$に結合するFCNCは存在しない.これにより通常の一次の大きさのフレーバーが変化するダイアグラムはなくなり,$K^0-\bar{K}^0$振動の確率は小さくなる.もちろんゼロではない.1ループの大きさでは,$W$交換による効果が存在するので,次のようなダイアグラムが存在することがわかる.これにより$K^0-\bar{K}^0$振動の反応率は実験値の大きさとなる.

\begin{tikzpicture}[decoration={markings, 
mark= at position -1cm with {\arrow[line width=0.5mm]{Stealth}}}]
\coordinate (a1) at (-2,2){};
\coordinate (a2) at (0,2){};
\coordinate (b1) at (-2,0){};
\coordinate (b2) at (0,0){};
\coordinate (c1) at (2,2){};
\coordinate (c2) at (4,2){};
\coordinate (d1) at (2,0){};
\coordinate (d2) at (4,0){};

\draw[thick,postaction={decorate}](a1)node[left]{$d$}--(a2);
\draw[thick,postaction={decorate}](b1)node[left]{$\bar{s}$}--(b2);
\draw[thick,postaction={decorate}](c1)--(c2)node[right]{$s$};
\draw[thick,postaction={decorate}](d1)--(d2)node[right]{$\bar{d}$};
\draw[thick,postaction={decorate}](a2)--(c1);
\draw[thick,postaction={decorate}](b2)--(d1);

\begin{feynhand}
\propag[photon,thick](a2)--(b2);
\propag[photon,thick](c1)--(d1);
\end{feynhand}
\draw(0,1)node[left]{$W$};
\draw(2,1)node[left]{$W$};
\draw(1,2)node[above]{$u,c$};
\draw(1,0)node[below]{$u,c$};
\draw(5,1)node{$+$};

\end{tikzpicture}
\begin{tikzpicture}[decoration={markings, 
mark= at position -1cm with {\arrow[line width=0.5mm]{Stealth}}}]
\coordinate (a1) at (-2,2){};
\coordinate (a2) at (0,2){};
\coordinate (b1) at (-2,0){};
\coordinate (b2) at (0,0){};
\coordinate (c1) at (2,2){};
\coordinate (c2) at (4,2){};
\coordinate (d1) at (2,0){};
\coordinate (d2) at (4,0){};

\draw[thick,postaction={decorate}](a1)node[left]{$d$}--(a2);
\draw[thick,postaction={decorate}](b1)node[left]{$\bar{s}$}--(b2);
\draw[thick,postaction={decorate}](c1)--(c2)node[right]{$s$};
\draw[thick,postaction={decorate}](d1)--(d2)node[right]{$\bar{d}$};
\draw[thick,postaction={decorate}](a2)--(b2);
\draw[thick,postaction={decorate}](d1)--(c1);

\begin{feynhand}
\propag[photon,thick](a2)--(c1);
\propag[photon,thick](b2)--(d1);
\end{feynhand}
\draw(0,1)node[left]{$u,c$};
\draw(2,1)node[left]{$u,c$};
\draw(1,2)node[above]{$W$};
\draw(1,0)node[below]{$W$};

\end{tikzpicture}

三世代を含んだラグランジアンを同様に計算しておこう.これによる重要な帰結は特にここでは出さないが,ひとつの到達点として残しておくことにする.
\begin{align*}
&\overline{\left(
\begin{array}{cc}
u_L \\
d'_L
\end{array}
\right)}\left[\sum_\alpha A_\alpha t^\alpha \right]\left(
\begin{array}{cc}
u_L \\
d'_L
\end{array}
\right)+\bar{u}_R \left[\sum_\alpha A_\alpha t^\alpha \right]u_R+\bar{d}'_R \left[\sum_\alpha A_\alpha t^\alpha \right]d'_R \\
&+\overline{\left(
\begin{array}{cc}
c_L \\
s'_L
\end{array}
\right)}\left[\sum_\alpha A_\alpha t^\alpha \right]\left(
\begin{array}{cc}
c_L \\
s'_L
\end{array}
\right)+\bar{c}_R \left[\sum_\alpha A_\alpha t^\alpha \right]c_R+\bar{s}'_R \left[\sum_\alpha A_\alpha t^\alpha \right]s'_R \\
&+\overline{\left(
\begin{array}{cc}
t_L \\
b'_L
\end{array}
\right)}\left[\sum_\alpha A_\alpha t^\alpha \right]\left(
\begin{array}{cc}
t_L \\
b'_L
\end{array}
\right)+\bar{t}_R \left[\sum_\alpha A_\alpha t^\alpha \right]t_R+\bar{b}'_R \left[\sum_\alpha A_\alpha t^\alpha \right]b'_R \\
=&W_{-\mu}\left[ \frac{g}{\sqrt{2}}\left\{ \bar{u} \gamma^\mu \left(\frac{1+\gamma_5}{2}\right)d'+\bar{c} \gamma^\mu \left(\frac{1+\gamma_5}{2}\right)s' + \bar{t}\gamma^\mu \left(\frac{1+\gamma_5}{2}\right)b' \right\}\right] \\
&+W_{+\mu}\left[\frac{g}{\sqrt{2}}\left\{ \bar{d}' \gamma^\mu \left(\frac{1+\gamma_5}{2}\right)u+\bar{s}' \gamma^\mu \left(\frac{1+\gamma_5}{2}\right)c+\bar{b}'\gamma^\mu \left(\frac{1+\gamma_5}{2}\right)t \right\} \right] \\
&+\frac{g}{2} \bar{u} \Slash{Z}\sec\theta \left(\frac{1+\gamma_5}{2}\right)u+\frac{2e}{3}\bar{u}\Slash{Z}\tan\theta u-\frac{g}{2} \bar{d} \Slash{Z}\sec\theta \left(\frac{1+\gamma_5}{2}\right)d -\frac{e}{3}\bar{d}\Slash{Z}\tan\theta d \\
&+\frac{g}{2} \bar{c} \Slash{Z}\sec\theta \left(\frac{1+\gamma_5}{2}\right)c+\frac{2e}{3}\bar{c}\Slash{Z}\tan\theta c-\frac{g}{2} \bar{s} \Slash{Z}\sec\theta \left(\frac{1+\gamma_5}{2}\right)s -\frac{e}{3}\bar{s}\Slash{Z}\tan\theta s \\
&+\frac{g}{2} \bar{t} \Slash{Z}\sec\theta \left(\frac{1+\gamma_5}{2}\right)t+\frac{2e}{3}\bar{t}\Slash{Z}\tan\theta t-\frac{g}{2} \bar{b} \Slash{Z}\sec\theta \left(\frac{1+\gamma_5}{2}\right)b -\frac{e}{3}\bar{b}\Slash{Z}\tan\theta b \\
&+\frac{2e}{3}\bar{u}\Slash{A}u-\frac{e}{3}\bar{d}\Slash{A}d+\frac{2e}{3}\bar{c}\Slash{A}c-\frac{e}{3}\bar{s}\Slash{A}s + \frac{2e}{3}\bar{t}\Slash{A}t-\frac{e}{3}\bar{b}\Slash{A}b
\end{align*}
$W$に結合するハドロン・カレントは
\begin{align*}
J^\lambda=&\bar{u}\gamma^\lambda (1+\gamma_5)d'+\bar{c}\gamma^\lambda (1+\gamma_5)s'+\bar{t}\gamma^\lambda (1+\gamma_5)b' \\
=&\overline{\left[
\begin{array}{cc}
u \\
c \\
t
\end{array}
\right]}\gamma^\lambda (1+\gamma_5)V_{\mr{CKM}}\left[
\begin{array}{cc}
d \\
s \\
b
\end{array}
\right]
\end{align*}
と表される.ここでラグランジアンを計算するのに用いる二重項は
\begin{align*}
&\left(\frac{1+\gamma_5}{2}\right)\left[
\begin{array}{cc}
u \\
d'
\end{array}
\right]=\left(\frac{1+\gamma_5}{2}\right)\left[
\begin{array}{cc}
u \\
V_{ud}d+V_{us}s+V_{ub}b
\end{array}
\right] \\
&\left(\frac{1+\gamma_5}{2}\right)\left[
\begin{array}{cc}
c \\
s'
\end{array}
\right]=\left(\frac{1+\gamma_5}{2}\right)\left[
\begin{array}{cc}
c \\
V_{cd}d+V_{cs}s+V_{cb}b
\end{array}
\right] \\
&\left(\frac{1+\gamma_5}{2}\right)\left[
\begin{array}{cc}
t \\
b'
\end{array}
\right]=\left(\frac{1+\gamma_5}{2}\right)\left[
\begin{array}{cc}
t \\
V_{td}d+V_{ts}s+V_{tb}b
\end{array}
\right]
\end{align*}
である.

\vskip\baselineskip

この2重項の構成が妥当であることを見る.そのために,以前と同じようにクォーク場についての質量項を構成してやろう.単純に考えれば次のように構成できる.
\begin{align*}
-\sum_{ijn}G_{ij}^n\overline{\left(
\begin{array}{cc}
U_{iL} \\
D_{iL}
\end{array}
\right)}\left(
\begin{array}{cc}
\phi^+_n \\
\phi^0_n
\end{array}
\right)D_{jR}-\sum_{ijn}H_{ij}^n\overline{\left(
\begin{array}{cc}
U_{iL} \\
D_{iL}
\end{array}
\right)}\left(
\begin{array}{cc}
\phi^+_n \\
\phi^0_n
\end{array}
\right)U_{jR}
\end{align*}
ここで$U_i$と$D_i(i=1,2,3)$はそれぞれ電荷$2e/3$と$-e/3$の三個の独立なクォーク場だ.以前と同じ議論をすれば,これが$SU(2)\times U(1)$不変であるためには自然に$\phi^+_n$の真空期待値はゼロとなり,$\phi^0_n$の真空期待値が有限の値をとる.すなわち,ダウンクォークの質量項はこれから生じるが,アップクォークについての質量項はゼロとなる.もちろんアップクォークの質量は有限であるから,これは適切ではない.(この模型では後者がそもそも電荷が保存しないのだが…)スカラー2重項の上下を単に入れ替えても,$SU(2)$不変性が破れてしまう.この問題の解決法は,通常のスカラー2重項と同じ$SU(2)$変換性をもつ
\begin{align*}
i\sigma_2 \left(
\begin{array}{cc}
{\phi^+}^* \\
{\phi^0}^*
\end{array}
\right)= \left(
\begin{array}{cc}
{\phi^0}^* \\
-{\phi^+}^*
\end{array}
\right)
\end{align*}
を用いることだ.ここで$\sigma_\alpha$は通常のパウリ行列だ.実際この2重項は$SU(2)$変換のもとで
\begin{align*}
\left(
\begin{array}{cc}
\phi^+ \\
\phi^0
\end{array}
\right)&\to \exp(i\theta_\alpha \sigma_\alpha)\left(
\begin{array}{cc}
\phi^+ \\
\phi^0
\end{array}
\right) \\
i\sigma_2 \left(
\begin{array}{cc}
{\phi^+}^* \\
{\phi^0}^*
\end{array}
\right)&\to i\sigma_2 \exp(-i\theta_\alpha \sigma^*_\alpha)\left(
\begin{array}{cc}
{\phi^+}^* \\
{\phi^0}^*
\end{array}
\right) \\
&=\sigma_2 \exp(-i\theta_\alpha \sigma^*_\alpha)\sigma_2 (i\sigma_2)\left(
\begin{array}{cc}
{\phi^+}^* \\
{\phi^0}^*
\end{array}
\right) \quad \because (\sigma_2)^2=1 \\
&=\exp(-i\theta_\alpha \sigma_2 \sigma^*_\alpha \sigma_2)i\sigma_2 \left(
\begin{array}{cc}
{\phi^+}^* \\
{\phi^0}^*
\end{array}
\right) \\
&=\exp(i\theta_\alpha \sigma_\alpha)i\sigma_2 \left(
\begin{array}{cc}
{\phi^+}^* \\
{\phi^0}^*
\end{array}
\right)
\end{align*}
となるからだ.二つ目の等号では,expを指数行列の定義にしたがって展開して考えれば容易にわかる.最後の等号ではパウリ行列の特殊性$\sigma_2\sigma^*_\alpha \sigma_2=-\sigma_\alpha$を用いた.
\begin{align*}
\sigma_2\sigma^*_1\sigma_2 &=\left(
\begin{array}{cc}
0 & -i \\
i & 0
\end{array}
\right)\left(
\begin{array}{cc}
0 & 1 \\
1 & 0
\end{array}
\right)\left(
\begin{array}{cc}
0 & -i \\
i & 0
\end{array}
\right)=-\left(
\begin{array}{cc}
0 & 1 \\
1 & 0
\end{array}
\right)=-\sigma_1 \\
\sigma_2\sigma^*_2\sigma_2 &=\left(
\begin{array}{cc}
0 & -i \\
i & 0
\end{array}
\right)\left(
\begin{array}{cc}
0 & i \\
-i & 0
\end{array}
\right)\left(
\begin{array}{cc}
0 & -i \\
i & 0
\end{array}
\right)=-\left(
\begin{array}{cc}
0 & -i \\
i & 0
\end{array}
\right)=-\sigma_2 \\
\sigma_2\sigma^*_3\sigma_2 &=\left(
\begin{array}{cc}
0 & -i \\
i & 0
\end{array}
\right)\left(
\begin{array}{cc}
1 & 0 \\
0 & -1
\end{array}
\right)\left(
\begin{array}{cc}
0 & -i \\
i & 0
\end{array}
\right)=-\left(
\begin{array}{cc}
1 & 0 \\
0 & -1
\end{array}
\right)=-\sigma_3
\end{align*}
これを用いることにより,クォークとスカラーの相互作用で最も一般的なものは
\begin{align*}
\mc{L}_\phi=-\sum_{ijn}G_{ij}^n\overline{\left(
\begin{array}{cc}
U_{iL} \\
D_{iL}
\end{array}
\right)}\left(
\begin{array}{cc}
\phi^+_n \\
\phi^0_n
\end{array}
\right)D_{jR}-\sum_{ijn}H_{ij}^n\overline{\left(
\begin{array}{cc}
U_{iL} \\
D_{iL}
\end{array}
\right)}\left(
\begin{array}{cc}
{\phi^0_n}^* \\
-{\phi^+_n}^*
\end{array}
\right)U_{jR}+\mr{H.c.}
\end{align*}
であるとわかる.ここではクォークのカラー添え字はあらわに書いていないが,クォークにはひとつにつき3つの自由度(カラー)が存在し,その添え字についても和をとっていることを言及しておく.カラー添え字とは,例えばアップクォーク$u$には三種類の$u^1,u^2,u^3$があり,$a=1,2,3$で$1=赤,2=緑,3=青$とすればクォークは$q^a$と書ける.これらクォークには$SU(3)$変換
\begin{align*}
q^a(x)\to q'^a(x)=\sum_{b=1}^3 U(x)^a_b q^b(x)
\end{align*}
というゲージ変換がある.一応カラー添え字も明確に書くと先ほどのラグランジアンは
\begin{align*}
\mc{L}_\phi=-\sum_{aijn}G_{ij}^n\overline{\left(
\begin{array}{cc}
U^a_{iL} \\
D^a_{iL}
\end{array}
\right)}\left(
\begin{array}{cc}
\phi^+_n \\
\phi^0_n
\end{array}
\right)D^a_{jR}-\sum_{aijn}H_{ij}^n\overline{\left(
\begin{array}{cc}
U^a_{iL} \\
D^a_{iL}
\end{array}
\right)}\left(
\begin{array}{cc}
{\phi^0_n}^* \\
-{\phi^+_n}^*
\end{array}
\right)U^a_{jR}+\mr{H.c.}
\end{align*}
となる.自明に,$SU(3)$変換のもとでこのラグランジアンは不変だ.$SU(2)\times U(1)$不変であることもすぐに確かめられる.(不変であるように構成したので自明であるが,第二項目のスカラーに関しての$U(1)$は複素共役の影響があることに留意すること.)実際,$SU(2)\times U(1)$変換を施すと,第一項目は
\begin{align*}
\delta \mc{L}_1=&-\sum_{ijn}G_{ij}^n\delta \overline{\left(
\begin{array}{cc}
U_{iL} \\
D_{iL}
\end{array}
\right)}\left(
\begin{array}{cc}
\phi^+_n \\
\phi^0_n
\end{array}
\right)D_{jR}-\sum_{ijn}G_{ij}^n\overline{\left(
\begin{array}{cc}
U_{iL} \\
D_{iL}
\end{array}
\right)}\left(
\begin{array}{cc}
\phi^+_n \\
\phi^0_n
\end{array}
\right)\delta D_{jR} \\
&-\sum_{ijn}G_{ij}^n\overline{\left(
\begin{array}{cc}
U_{iL} \\
D_{iL}
\end{array}
\right)}\delta\left(
\begin{array}{cc}
\phi^+_n \\
\phi^0_n
\end{array}
\right)D_{jR} \\
=&-\sum_{ijn}G_{ij}^n \overline{\left(
\begin{array}{cc}
U_{iL} \\
D_{iL}
\end{array}
\right)}\left[ -i\left(\vec{\epsilon}\cdot\vec{t}-\alpha\frac{g'}{6}\right) \right]\left(
\begin{array}{cc}
\phi^+_n \\
\phi^0_n
\end{array}
\right)D_{jR}-\sum_{ijn}G_{ij}^n\overline{\left(
\begin{array}{cc}
U_{iL} \\
D_{iL}
\end{array}
\right)}i\alpha\frac{g'}{3}\left(
\begin{array}{cc}
\phi^+_n \\
\phi^0_n
\end{array}
\right)D_{jR} \\
&-\sum_{ijn}G_{ij}^n\overline{\left(
\begin{array}{cc}
U_{iL} \\
D_{iL}
\end{array}
\right)}\left[ i\left(\vec{\epsilon}\cdot\vec{t}-\alpha\frac{g'}{2}\right) \right]\left(
\begin{array}{cc}
\phi^+_n \\
\phi^0_n
\end{array}
\right)D_{jR}\\
=&0
\end{align*}
同様に第二項目も
\begin{align*}
\delta \mc{L}_2=&-\sum_{ijn}H_{ij}^n\delta \overline{\left(
\begin{array}{cc}
U_{iL} \\
D_{iL}
\end{array}
\right)}\left(
\begin{array}{cc}
{\phi^0_n}^* \\
-{\phi^+_n}^*
\end{array}
\right)U_{jR}-\sum_{ijn}H_{ij}^n\overline{\left(
\begin{array}{cc}
U_{iL} \\
D_{iL}
\end{array}
\right)}\left(
\begin{array}{cc}
{\phi^0_n}^* \\
-{\phi^+_n}^*
\end{array}
\right)\delta U_{jR} \\
&-\sum_{ijn}H_{ij}^n\overline{\left(
\begin{array}{cc}
U_{iL} \\
D_{iL}
\end{array}
\right)}\delta\left(
\begin{array}{cc}
{\phi^0_n}^* \\
-{\phi^+_n}^*
\end{array}
\right)U_{jR} \\
=&-\sum_{ijn}H_{ij}^n\overline{\left(
\begin{array}{cc}
U_{iL} \\
D_{iL}
\end{array}
\right)}\left[-i\left( \vec{\epsilon}\cdot\vec{t}-\alpha\frac{g'}{6} \right)\right]\left(
\begin{array}{cc}
{\phi^0_n}^* \\
-{\phi^+_n}^*
\end{array}
\right)U_{jR} \\
&-\sum_{ijn}H_{ij}^n\overline{\left(
\begin{array}{cc}
U_{iL} \\
D_{iL}
\end{array}
\right)}\left[-i\alpha\frac{2g'}{3}\right]\left(
\begin{array}{cc}
{\phi^0_n}^* \\
-{\phi^+_n}^*
\end{array}
\right)U_{jR} \\
&-\sum_{ijn}H_{ij}^n\overline{\left(
\begin{array}{cc}
U_{iL} \\
D_{iL}
\end{array}
\right)}i\left( \vec{\epsilon}\cdot\vec{t}+\alpha\frac{g'}{2} \right)\left(
\begin{array}{cc}
{\phi^0_n}^* \\
-{\phi^+_n}^*
\end{array}
\right)U_{jR} \\
=&0
\end{align*}
となって不変であることが確かめられる.よってこのラグランジアンは$SU(3)\times SU(2)\times U(1)$不変だ.\par
中性スカラー場のゼロでない真空期待値により,以前と同様に真空期待値まわりで展開すれば,クォークの質量項
\begin{align*}
\mc{L}_m=-\sum_{ij}\bar{D}_{iL}m^D_{ij}D_{jR}-\sum_{ij}\bar{U}_{iL}m^U_{ij}U_{jR}+\mr{H.c.}
\end{align*}
が生じる.ここで
\begin{align*}
m^D_{ij}=\sum_n G^n_{ij}\braket{\phi^0_n},\quad m^U_{ij}=\sum_n H^n_{ij}\braket{\phi^0_n}^*
\end{align*}
と定義している.不変性からは行列$m^D_{ij},m^U_{ij}$にはなんの制限もかからず,特に複素非対角行列であってもよい.\par
$\Rightarrow$そのとき,パリティとフレーバーが保存しない(例えば$\bar{s}u$に比例した)項などが生じる. \\
このとき,新しいクォーク場
\begin{align*}
U'_R=A^U_R U_R , \quad U'_L=A^U_L U_L , \quad D'_R=A^D_R D_R , \quad D'_L=A^D_L D_L
\end{align*}
を導入することができる.ここで各$A$は,運動項
\begin{align*}
\mc{L}=-\sum_i \bar{U}_{i}\Slash{\partial}U_{i}-\sum_i \bar{D}_{i}\Slash{\partial}D_i
\end{align*}
の形を保つために,ユニタリーでなければならない,という条件だけがついた$3\times 3$行列(成分添え字は$i,j$)だ.すると質量項は
\begin{align*}
\mc{L}_m=&-\bar{D}_{L}m^D D_{R}-\bar{U}_{L}m^U U_{R}+\mr{H.c.} \\
=&-\bar{D}_{L}{A^D_L}^{\dagger}[A^L_R m^D {A^D_R}^\dagger] A^D_R D_{j R}-\bar{U}_{L}{A^U_L}^\dagger [A^U_L m^U {A^U_R}^\dagger] A^U_R U_{R}+\mr{H.c.} \\
=&-\bar{D}'_{L}{m^D}' D'_{R}-\bar{U}'_{L}{m^U}' U'_{R}+\mr{H.c.}
\end{align*}
となって同じ形のままだが,行列$m^U,m^D$はそれぞれ
\begin{align*}
{m^U}'=A^U_L m^U {A^U_R}^\dagger , \quad {m^D}'=A^D_L m^D {A^D_R}^\dagger
\end{align*}
で置き換えられる.\par
任意の行列$m$について,$AmB$が実対角行列になるようにユニタリー行列$A,B$を選ぶことが可能だ.極分解定理より,エルミート行列$H$とユニタリー行列$U$を用いて$m=HU$と極分解できる.$H$を対角化するユニタリー行列$S$を用いて$A=S^\dagger$,$B=U^\dagger S$ととれば,$AmB=S^\dagger HS$となり,これは対角化された上エルミートなので,成分は全て実だ.よって実対角行列となる.したがって$A^U_{L,R},A^D_{L,R}$は${m^U}',{m^D}'$が実で対角的になるように選ぶことができる.このとき
\begin{align*}
\mc{L}_m=&-\sum_{ij}\bar{U}'_{iL}{m^U_{ij}}'U'_{iR}-\sum_{ij}\bar{U}'_{iR}{m^U_{ij}}'U'_{jL} \quad \because {{m^U}'}^\dagger={m^U}' \\
&-\sum_{ij}\bar{D}'_{iL}{m^D_{ij}}'D'_{iR}-\sum_{ij}\bar{D}'_{iR}{m^D_{ij}}'D'_{jL} \quad \because {{m^D}'}^\dagger={m^D}' \\
=&-\sum_{ij}\overline{(U'_{iL}+U'_{iR})}{m^U_{ij}}'(U_{jL}+U_{jR})-\sum_{ij}\overline{(D'_{iL}+D'_{iR})}{m^D_{ij}}'(D_{jL}+D_{jR})
\end{align*}
となる.つまり,クォーク場$u,c,t$と$d,s,b$は,$U'_L+U'_R$と$D'_L+D'_R$の成分と同定すべきだ.\par
弱い相互作用の2重項は
\begin{align*}
Q_{iL}=\left(
\begin{array}{cc}
U_{Li} \\
D_{Li}
\end{array}
\right)=\left(
\begin{array}{cc}
({A^U_{L}}^{-1}U'_L)_i \\
({A^D_{L}}^{-1}D'_L)_i
\end{array}
\right)
\end{align*}
と書かれる.ここでも新たに
\begin{align*}
Q'_L=A^U_L Q_L=\left(
\begin{array}{cc}
U'_L \\
A^U_L {A^D_L}^{-1} D'_L
\end{array}
\right)
\end{align*}
ととることが可能だ.この上成分は決まった質量をもつ荷電$2e/3$のクォーク$U'_{Li}=u_L,c_L,t_L$なのであった.この場合,これらの2重項は(21.3.46)$\sim$(21.3.48)で
\begin{align*}
V_{\mr{CKM}}=A^U_L {A^D_L}^{-1}
\end{align*}
がキャビボ・小林・益川行列となる.

\vskip\baselineskip

ここで,便利のために$SU(3)\times SU(2)\times U(1)$の各表現リストを作っておく.$\lambda$はゲルマン行列に比例した$3\times 3$行列,$\sigma$はパウリ行列だ.(ゲルマン行列はカラー添え字に作用する.)$\theta^n$は$SU(n)$の生成子に結合するパラメータとする.
\begin{align*}
&\delta \left(
\begin{array}{cc}
U_{iL} \\
D_{iL}
\end{array}
\right)=i\left[ \theta^3_\alpha \lambda_\alpha +\frac{g}{2}\vec{\theta^2}\cdot \vec{\sigma}-\theta^1 \frac{g'}{6} \right]\left(
\begin{array}{cc}
U_{iL} \\
D_{iL}
\end{array}
\right) \\
&\delta U_{iR}=i\left[\theta^3_\alpha \lambda_\alpha -\theta^1 \frac{2}{3}g' \right]U_{iR},\quad \delta D_{iR}=i\left[\theta^3_\alpha \lambda_\alpha +\theta^1 \frac{1}{3}g' \right]D_{iR} \\
&\delta \left(
\begin{array}{cc}
\nu_{iL} \\
\ell_{iL}
\end{array}
\right)=i\left[\frac{g}{2}\vec{\theta^2}\cdot \vec{\sigma} +\theta^1 \frac{g'}{2} \right]\left(
\begin{array}{cc}
\nu_{iL} \\
\ell_{iL}
\end{array}
\right),\quad \delta\ell_{iR}=i[\theta^1 g'] \ell_{iR} 
\end{align*}
スカラー場については,
\begin{align*}
\delta \left(
\begin{array}{cc}
\phi^+ \\
\phi^0
\end{array}
\right)=i\left[ \frac{g}{2}\vec{\theta^2}\cdot \vec{\sigma} -\theta^1\frac{g'}{2} \right]\left(
\begin{array}{cc}
\phi^+ \\
\phi^0
\end{array}
\right),\quad \delta\left(
\begin{array}{cc}
{\phi^0}^* \\
-{\phi^+}^*
\end{array}
\right)=i\left[ \frac{g}{2}\vec{\theta^2}\cdot \vec{\sigma}+\theta^1\frac{g'}{2} \right]\left(
\begin{array}{cc}
{\phi^0}^* \\
-{\phi^+}^*
\end{array}
\right)
\end{align*}
となる.これらの変換の式は,各場の弱超電荷の値などからすぐにわかることだ.すぐに参照できるようにしているだけである.\par
一応後に使うので,スピノルのローレンツ変換についても復習しておこう.5.4節よりスピノルのローレンツ変換性は
\begin{align*}
&\psi\to D(\Lambda)\psi \\
&D(\Lambda)=\exp\left[ \frac{i}{2}\omega_{\mu\nu}J^{\mu\nu} \right] ,\quad J^{\mu\nu}=\frac{-i}{4}[\gamma^\mu,\gamma^\nu] \\
&\beta D^\dagger (\Lambda)\beta =D^{-1}(\Lambda) \quad \Rightarrow \quad \bar{\psi}\to \bar{\psi} D^{-1}(\Lambda)
\end{align*}
であるから,荷電共役行列$\mc{C}$と$D(\Lambda)$は
\begin{align*}
\mc{C}(D^{-1}(\Lambda))^T=\mc{C}\exp\left[ -\frac{i}{2}\omega_{\mu\nu}{J^{\mu\nu}}^T \right] =&\mc{C}\exp\left[ \frac{i}{2}\omega_{\mu\nu}\mc{C}J^{\mu\nu} \mc{C}^{-1}\right]\mc{C}^{-1}\mc{C} \quad \because (5.4.37) \\
=&\exp\left[ \frac{i}{2}\omega_{\mu\nu}J^{\mu\nu} \right]\mc{C}=D(\Lambda)\mc{C} \quad \because \mc{C}=-\mc{C}^\dagger=-\mc{C}^{-1}
\end{align*}
よって荷電共役スピノルのローレンツ変換性は
\begin{align*}
\psi^c&=-\xi^* \beta \mc{C} \psi^* =\xi^* \mc{C} (\bar{\psi})^T \quad \because \mc{C}=\gamma_2 \beta \\
{\psi^c}'&=\xi^*\mc{C}(\bar{\psi'})^T=\xi^* \mc{C}(\bar{\psi}D^{-1}(\Lambda))^T \\
&=\xi^*\mc{C}(D^{-1}(\Lambda))^T(\bar{\psi})^T=\xi^* D(\Lambda)\mc{C}(\bar{\psi})^T \\
&=D(\Lambda)\psi^c
\end{align*}
となって,通常のスピノルと同様の変換性となる.これにより(21.3.54)はローレンツ不変な相互作用とわかる.

\vskip\baselineskip

電弱理論の場の内容と,$SU(3)\times SU(2)\times U(1)$ゲージ対称性をもつ最も一般的な\uwave{くりこみ可能}なラグランジアンは自動的にバリオン数とレプトン数を保存する.これはゲージ相互作用と裸の質量項(今まで考えてきたような項)については明らかに正しい.なぜなら,クォーク,反クォーク,レプトン,反レプトンは全て$SU(3)\times SU(2)\times U(1)$の異なる表現に属する(変換性が違う)からだ.(反粒子は,共役場のこと.)スカラーが全て$SU(3)$に関して単位表現で,$SU(2)$の2重項であり,$U(1)$量子数$\pm 1/2$
\begin{align*}
\delta\left(
\begin{array}{cc}
\phi_1 \\
\phi_2
\end{array}
\right)=i\left[\vec{\epsilon}\cdot \vec{t}\pm \alpha \frac{g'}{2} \right]\left(
\begin{array}{cc}
\phi_1 \\
\phi_2
\end{array}
\right)
\end{align*}
をもつなら,スカラー・フェルミオン・反フェルミオンとのくりこみ可能な相互作用$(d\leq 4)$は,唯一,クォーク・反クォーク対(21.3.49)および,レプトン・反レプトン対(21.3.21)との相互作用である.\par
$\Rightarrow$これはもちろん,バリオン数とレプトン数を保存する.\par
これらの結果は標準模型が\uwave{くりこみ可能}だという過程に決定的に依っている.\par
$\Rightarrow$しかし,標準模型のラグランジアンには,次元$d>4$をもち,ある非常に大きな質量$M$の$4-d$次のベキで抑えられている\uwave{非くりこみ可能}な項が伴っていることが期待される.\\
くりこみ可能な標準模型の予言に対する主要な補正(例えば,ニュートリノの質量など)は,4を超える最も小さな次元をもつ項からくる.\par
標準模型のフェルミオンと,他の場から構成できる\uwave{次元5}でローレンツ不変な項は,せいぜいフェルミオン場について双一次(次元+3)で,2個のスカラー(次元+2)か,1個のスカラー(次元+1)と1個のゲージ共変微分(次元+1),あるいはスカラーは無しで2個のゲージ共変微分(それらの交換子(15.1.12),すなわち場の強度テンソルも含めて)かを含む.\par
$\Rightarrow$カラー$SU(3)$不変性より,そのような相互作用でのフェルミオン場は「クォーク・反クォーク双一次」か,「レプトン・反レプトン双一次」,「レプトン・レプトン対」,「反レプトン・反レプトン対」として表れ,そのすべての演算子はバリオン数を保存する.\par
$\Rightarrow$レプトン数の保存を破るには,2個のレプトン場の積(または2個の反レプトン場の積)を含まねばならない!\\
左手成分のレプトン2重項$(\nu_{iL},\ell_{iL})$は$U(1)$量子数1/2,右手成分のレプトン1重項$\ell_{iR}(i=e,\mu,\tau)$は$U(1)$量子数+1.一方,スカラー場$SU(2)$2重項(複素スカラー2重項があっても良い)の$(\phi^+,\phi^0)$は$U(1)$量子数-1/2をもつ.\par
$\Rightarrow$2つの左手成分レプトン2重項と2つスカラー2重項から,$U(1)$不変な\uwave{次元5}の相互作用を作ることができる.\par
スカラー2重項が$(\phi^+,\phi^0)$の1種類しかないとすると,そのような項で$SU(2)$不変性・ローレンツ不変性を満たすものは
\begin{align*}
\sum_{ij}f_{ij}(\overline{\ell^c_{iL}}\phi^+ -\overline{\nu^c_{iL}}\phi^0)(\ell_{jL}\phi^+ -\nu_{jL}\phi^0)
\end{align*}
だけだ.ここで$i,j$はレプトンのフレーバー添え字,$c$は荷電共役場を表す.\par
これは実際にローレンツ不変かつ$SU(3)\times SU(2)\times U(1)$不変だ.ローレンツ不変なことは既に見た.$SU(3)\times SU(2)\times U(1)$不変であることを示そう.荷電共役場のディラック共役場は
\begin{align*}
\overline{\psi^c}&=\left\{ -\xi^* \beta \mc{C} \psi^* \right\}^\dagger \beta=-\xi \psi^T \mc{C}^\dagger \beta\beta \\
&=\xi \psi^T \mc{C} \quad \because \mc{C}=-\mc{C}^\dagger=-\mc{C}^{-1}
\end{align*}
であるから,
\begin{align*}
(\overline{\ell^c_{iL}}\phi^+ -\overline{\nu^c_{iL}}\phi^0)&=\xi(\ell_{jL}^T \phi^+ -\nu_{jL}^T\phi^0)\mc{C} \\
&=\xi(\ell_{jL}\phi^+ -\nu_{jL}\phi^0)^T\mc{C}
\end{align*}
となる.この括弧の中身は(21.3.54)の二つ目の括弧の中身と同じだ.ここで
\begin{align*}
\ell_{jL}\phi^+ -\nu_{jL}\phi^0=-\left(
\begin{array}{cc}
{\phi^0}^* \\
-{\phi^+}^*
\end{array}
\right)^\dagger \left(
\begin{array}{cc}
\nu_{iL} \\
\ell_{iL}
\end{array}
\right)
\end{align*}
であることに留意すると,
\begin{align*}
\delta(\ell_{jL}\phi^+ -\nu_{jL}\phi^0)=&-\delta \left(
\begin{array}{cc}
{\phi^0}^* \\
-{\phi^+}^*
\end{array}
\right)^\dagger \left(
\begin{array}{cc}
\nu_{iL} \\
\ell_{iL}
\end{array}
\right)-\left(
\begin{array}{cc}
{\phi^0}^* \\
-{\phi^+}^*
\end{array}
\right)^\dagger \delta\left(
\begin{array}{cc}
\nu_{iL} \\
\ell_{iL}
\end{array}
\right) \\
=&-\left(
\begin{array}{cc}
{\phi^0}^* \\
-{\phi^+}^*
\end{array}
\right)^\dagger\left\{ -i\left[ \frac{g}{2}\vec{\theta^2}\cdot \vec{\sigma}+\theta^1\frac{g}{2} \right] \right\}\left(
\begin{array}{cc}
\nu_{iL} \\
\ell_{iL}
\end{array}
\right) \\
&-\left(
\begin{array}{cc}
{\phi^0}^* \\
-{\phi^+}^*
\end{array}
\right)^\dagger\left\{ i\left[ \frac{g}{2}\vec{\theta^2}\cdot \vec{\sigma}+\theta^1\frac{g}{2} \right] \right\}\left(
\begin{array}{cc}
\nu_{iL} \\
\ell_{iL}
\end{array}
\right) =0
\end{align*}
となって,したがって(21.3.54)は全体として$SU(3)\times SU(2)\times U(1)$不変であることがわかる.\par
電弱対称性の破れのスケールより,低いエネルギーでは,これは有効相互作用
\begin{align*}
\sum_{ij}f_{ij}\overline{\nu^c_{iL}}\nu_{jL}\braket{\phi^0}^2
\end{align*}
を生じる.次元解析より$f_{ij}$は$1/M$の次数で,多分小さな結合定数がかかっていると期待される.したがって真空期待値が(21.3.35)であるから,これはせいぜい$(300\mr{GeV})^2/M$程度の,レプトン数を保存しないニュートリノ質量を与える!今では実際に,ニュートリノ振動によってニュートリノが質量をもつことが分かっている.

\newpage

\subsection{動的に破れた対称性}
ゲージ結合がゼロの極限で,理論が\uwave{大域的対称性の群$G$}のもとで不変な理論に帰着し,その対称性は部分群$H$に自発的に破れている,と仮定する.19.6節の通り,この場合の理論はNGボゾン場$\xi_a$の組と他の物質場$\tilde{\psi}$で記述され,ラグランジアンが$\tilde{\psi}$と(19.6.14)(19.6.30)等で与えられる共変微分$D_{a\mu},D_\mu \tilde{\psi}$等だけから$H$不変であるように作られていれば,ラグランジアンは$G$不変なのであった.\par
$\Rightarrow$ここでゲージ結合の「スイッチを入れる」.ゲージ群$\mc{G}$は,理論の全対称性である群$G$の部分群だ.(局所変換はパラメータが$x$依存しなければ大域的変換となるから,ゲージ群$\mc{G}$は$G$の部分群だ.)また$G$が自発的に$H$に破れるとき,$\mc{G}$は,$\mc{G}$と$H$の共通部分に等しい部分群$\mc{H}$に自発的に破れなければならない.

\begin{figure}[H]
  \centering
  \begin{tikzpicture}
    \draw[thick](0,0)circle[radius=2];
    \draw[thick](-0.4,0.3)circle[x radius=1.1,y radius=1.4,rotate=-30];
    \draw[thick](0.4,-0.3)circle[x radius=1.1,y radius=1.4,rotate=-30];
    \draw[thick,dashed,rotate=-30](-0.1,-1.1)arc(233:115:1cm and 1.3cm)arc(52:-63:1cm and 1.3cm);
    \node(G) at (2.2,1){$G$};
    \node(H) at (-0.9,1){$H$};
    \node(g) at (1.3,0){$\mc{G}$};
    \node(h) at (0,0){$\mc{H}$};
  \end{tikzpicture}
\end{figure}

ゲージ群$\mc{G}$の生成子$\mc{T}_\alpha$は全体$G$の生成子$T_A$の結合で
\begin{align*}
\mc{T}_\alpha=\sum_A e_{\alpha A}T_A(=\sum_i e_{\alpha i}t_i +\sum_a e_{\alpha a}x_a)
\end{align*}
と表せる.ここで係数$e_{\alpha A}$,すなわちゲージ結合は非常に小さくとられる.添え字$A$は破れていない対称性$H$の生成子$t_i$と,破れた対称性$G/H$の生成子$x_a$の指標$i,a$についてとる.$x_a,t_i$の構造定数は結合定数を含まない.

\vskip\baselineskip

基礎にある理論で$G$不変性が線型実現(非線形実現については19.5節p263を参照)している場合,ゲージ場$\mc{A}_{\alpha\mu}$と他の場$\psi$との結合を導入するには,通常の微分をゲージ共変微分
\begin{align*}
\left(\partial_\mu -i\sum_\alpha \mc{T}_\alpha \mc{A}_{\alpha\mu} \right)\psi=&\left(\partial_\mu -i\sum_{\alpha A} e_{\alpha A}T_A \mc{A}_{\alpha\mu} \right)\psi \\
=&\left(\partial_\mu -i\sum_A T_A A_{A\mu} \right)\psi
\end{align*}
で置き換えればいい.ここで
\begin{align*}
A_{A\mu}=\sum_\alpha e_{\alpha A}\mc{A}_{\alpha\mu}
\end{align*}
と定義した.結果として得られる理論は,場が
\begin{align*}
\psi\to g\psi ,\quad \sum_A T_A A_{A\mu}\to g\left(\sum_A T_A A_{A\mu} \right)g^{-1}-i(\partial_\mu g)g^{-1}
\end{align*}
と変換する\uwave{形式的な}局所$G$変換のもとで不変だ.実際に
\begin{align*}
\left(\partial_\mu -i\sum_A T_A A_{A\mu} \right)\psi \to & \left(\partial_\mu -ig\sum_A T_A A_{A\mu}g^{-1}-(\partial_\mu g)g^{-1} \right)g\psi \\
=&(\partial_\mu g)\psi +g(\partial_\mu \psi)-ig\sum_A T_A A_{A\mu}\psi - (\partial_\mu g)\psi \\
=&g\left(\partial_\mu -i \sum_A T_A A_{A\mu}\right)\psi
\end{align*}
となる.\par
この不変性は純粋に形式的な不変性だ.なぜなら,変換(21.4.5)は一般に線型結合(21.4.3)の形を保持しない.つまり$g$が$T_A$を生成子とする局所$G$変換であるとき
\begin{align*}
\sum_A T_A A_{A\mu}\to g\left(\sum_A T_A A_{A\mu} \right)g^{-1} -i \underset{T_A に比例する}{\uwave{(\partial_\mu g)g^{-1}}} =\sum_A T_A A'_{A\mu}
\end{align*}
と書けるが,$(\partial_\mu g)g^{-1}$は$e_{\alpha A}$を含まないので
\begin{align*}
A_{A\mu}=\sum_{\alpha}e_{\alpha A}\mc{A}_{A\mu} \overset{G}{\longrightarrow} A'_{A\mu}=\sum_{\alpha}e_{\alpha A} \mc{A}'_{\alpha \mu}+(e_{\alpha A} を含まない項)\neq \sum_\alpha e_{\alpha A}\mc{A}'_{\alpha \mu}
\end{align*}
となって,(21.4.3)の形を保持しない.したがって局所$G$変換のもとで
\begin{align*}
\left(\partial_\mu -i\sum_A T_A A_{A\mu} \right)\psi \overset{G}{\longrightarrow} \left(\partial_\mu -i\sum_A T_A A'_{A\mu} \right)\psi' =g\left(\partial_\mu -i\sum_A T_A A_{A\mu} \right)\psi
\end{align*}
は成り立っているが
\begin{align*}
\left(\partial_\mu -i\sum_\alpha \mc{T}_\alpha \mc{A}_{\alpha\mu} \right)\psi \overset{G}{\longrightarrow} \left(\partial_\mu -i\sum_\alpha \mc{T}_\alpha \mc{A}'_{\alpha\mu} \right)\psi' \neq g\left(\partial_\mu -i\sum_\alpha \mc{T}_\alpha \mc{A}_{\alpha\mu} \right)\psi 
\end{align*}
であるということだ.もし$g$が$\mc{T}_\alpha$を生成子とする局所$\mc{G}$変換であるならば,$\mc{T}_\alpha$は$e_{\alpha A}$を含むので
\begin{align*}
\sum_A T_A A_{A\mu}\to g\left(\sum_A T_A A_{A\mu} \right)g^{-1} -i \underset{\mc{T}_\alpha に比例する}{\uwave{(\partial_\mu g)g^{-1}}} =\sum_A T_A A'_{A\mu}
\end{align*}
と書けるが,このときは
\begin{align*}
A_{A\mu}=\sum_{\alpha}e_{\alpha A}\mc{A}_{A\mu} \overset{G}{\longrightarrow} A'_{A\mu}=\sum_\alpha e_{\alpha A}\mc{A}'_{\alpha \mu}
\end{align*}
となって線型結合は保持される.この場合は
\begin{align*}
\left(\partial_\mu -i\sum_\alpha \mc{T}_\alpha \mc{A}_{\alpha\mu} \right)\psi \overset{G}{\longrightarrow} \left(\partial_\mu -i\sum_\alpha \mc{T}_\alpha \mc{A}'_{\alpha\mu} \right)\psi'=g\left(\partial_\mu -i\sum_\alpha \mc{T}_\alpha \mc{A}_{\alpha\mu} \right)\psi
\end{align*}
が成り立ち,保存されていることがわかる.大切なのは,$A_{A\mu}\to A'_{A\mu}$という変換と$\mc{A}_{\alpha\mu}\to\mc{A}'_{\alpha\mu}$という変換が$\mc{G}$変換のとき同値であるから$\mc{G}$は保存され,局所$G$変換のとき同値でないから$G$は保存されない,ということだ.この事実からわかるように,一般にゲージ結合は$G$を破っている.\par
$\Rightarrow$しかし,しばらくの間(21.4.3)については忘れる!!\\
$A_A^\mu$を拘束のない,古典的な外場として取り扱い,物質場,および物質場とゲージ場の相互作用ラグランジアンの構造をラグランジアンが局所変換(21.4.4)(21.4.5)のもとで不変だと要求して解析する.このようにして,ラグランジアンが真の局所対称部分群$\mc{G}$のもとで,さらに$e_{\alpha A}\to 0$ではもっと大きな大域的対称性の群$G$のもとで不変なこと,さらにカレントが破れた大域的対称性$G$のもとで正しく変換することの両方が保証される.その後で$A_A^\mu$を(21.4.3)の形に制限し,場$\mc{A}_\alpha^\mu$を量子場として扱い,この場のラグランジアンの適切な運動項を与える.

\vskip\baselineskip

対称群$G$が破れて部分群$H$になることの意味を明らかにするために,19.6節と同じように議論を進める.まず$\psi,A$を新しい場$\tilde{\psi},\tilde{A}$
\begin{align*}
\tilde{\psi}=\gamma^{-1}(\xi)\psi ,\quad \tilde{A}^\mu_A=\sum_B D_{AB}(\gamma^{-1}(\xi))A^\mu_B
\end{align*}
で置き換える.ここで$D(g)$は随伴表現
\begin{align*}
g T_A g^{-1}=\sum_B D_{BA}(g)T_B
\end{align*}
である.($\tilde{\psi}$についての置き換えは以前と同様だが,$\tilde{A}$に関しては若干天下り的である.なにがしたいかは,すぐ後にわかるはずだ.)これらのNGボゾンの自由度は,時空に依存するパラメータ$\xi_a$に再度表れる.そして$\gamma(\xi)$はこのパラメータに依存する19.6節p290(19.6.18)(19.6.20)(19.6.42)と同じ計算により,局所的$G$変換と大域的$G$変換の両方の場合の変換$g$のもとで,変換則(21.4.4)は変換
\begin{align*}
\xi^a\overset{G}{\longrightarrow} {\xi^a}'=f^a(\xi,g),\quad \tilde{\psi}\overset{G}{\longrightarrow}\tilde{\psi}'=h(\xi,g)\tilde{\psi}
\end{align*}
に置き換わる.ここで$h,f$は,$h$を破れていない部分群$H$の元として
\begin{align*}
g\gamma(\xi)=\gamma(f(\xi,g))h(\xi,g)
\end{align*}
で定義される.(右剰余類)\par
また,$\tilde{A}^\mu_A$の変換則も導出する必要がある.(21.4.5)より,これらの局所変換のもとで(21.4.2)のゲージ場の線型結合は
\begin{align*}
\sum_A T_A A_{A\mu}\to \sum_A T_A A'_{A\mu}=g\left[\sum_A T_A A_{A\mu}-ig^{-1}\partial_\mu g \right]g^{-1}
\end{align*}
と変換する.左と右からそれぞれ$\gamma^{-1}(\xi')$と$\gamma(\xi')$をかけ,(21.4.7)(21.4.8)(21.4.11)を用いると,
\begin{align*}
&i\gamma^{-1}(\xi')\sum_A T_A A'_{A\mu}\gamma(\xi')=\gamma^{-1}(\xi')g\left[i\sum_A T_A A_{A\mu}+g^{-1}\partial_\mu g \right]g^{-1}\gamma(\xi') \\
\Rightarrow \quad & i\sum_{AB}D_{BA}(\gamma^{-1}(\xi'))T_B A'_{A\mu}=h(\xi,g)\gamma^{-1}(\xi)\left[i\sum_A T_A A_{A\mu}+g^{-1}\partial_\mu g \right]\gamma(\xi)h^{-1}(\xi,g) \\
\Rightarrow \quad & i\sum_A T_A \tilde{A}'_{A\mu}=h(\xi,g)\left[ i\sum_{AB}D_{BA}(\gamma^{-1}(\xi))T_B A_{A\mu}+\gamma^{-1}(\xi)\left[g^{-1}\partial_\mu g\right]\gamma(\xi) \right]h^{-1}(\xi,g) 
\end{align*}
と書ける.こうして置き換えられたゲージ場の$G$変換がわかる.
\begin{align*}
\sum_A T_A \tilde{A}_{A\mu}=&\sum_{AB} D_{BA}(\gamma^{-1}(\xi))T_B A_{A\mu} \\
\overset{\xi\to \xi',A\to A'}{\longrightarrow}& \sum_{AB} D_{BA}(\gamma^{-1}(\xi'))T_B A'_{A\mu} =\sum_A T_A \tilde{A}'_{A\mu}
\end{align*}
非斉次項$g^{-1}\partial_\mu g$を相殺する方法を見るには,(21.4.11)を用いて
\begin{align*}
\gamma(\xi')=&g\gamma(\xi)h^{-1}(\xi,g) \\
\gamma^{-1}(\xi')\partial_\mu \gamma(\xi')=&h(\xi,g)\gamma^{-1}(\xi)g^{-1}\partial_\mu[g\gamma(\xi)h^{-1}(\xi,g)] \\
=&\uwave{h(\xi,g)\gamma^{-1}(\xi)[g^{-1}\partial_\mu g]\gamma(\xi)h^{-1}(\xi,g)} +h(\xi,g)[\gamma^{-1}(\xi)\partial_\mu \gamma(\xi)]h^{-1}(\xi,g) \\
&+h(\xi,g)\partial_\mu h^{-1}(\xi,g) \quad \Bigl(=-[\partial_\mu h(\xi,g)]h^{-1}(\xi,g) \Bigr)
\end{align*}
が得られることに留意する.最後の項の変形は,$1=h(\xi,g)h^{-1}(\xi,g)$を両辺微分すれば容易に得られる.これより,非斉次項を相殺するには(21.4.13)から(21.4.12)を引けば良いことに気付く.
\begin{align*}
&\uwave{\gamma^{-1}(\xi')\partial_\mu \gamma(\xi')-i\sum_A T_A \tilde{A}'_{A\mu}} \\
=&\underline{h(\xi,g)\gamma^{-1}(\xi)[g^{-1}\partial_\mu g]\gamma(\xi)h^{-1}(\xi,g)} +h(\xi,g)[\gamma^{-1}(\xi)\partial_\mu \gamma(\xi)]h^{-1}(\xi,g) -[\partial_\mu h(\xi,g)]h^{-1}(\xi,g) \\
&-h(\xi,g)i\sum_A T_A \tilde{A}_{A\mu}h^{-1}(\xi,g)-\underline{h(\xi,g)\gamma^{-1}(\xi)[g^{-1}\partial_\mu g]\gamma(\xi)h^{-1}(\xi,g)} \\
=&h(\xi,g)\uwave{\left[\gamma^{-1}\partial_\mu \gamma(\xi)-i\sum_A T_A \tilde{A}_{A\mu}\right] }h^{-1}(\xi,g)-[\partial_\mu h(\xi,g)]h^{-1}(\xi,g) 
\end{align*}
この形が(19.6.25)と同じであることに気付けるから,(19.6.26)(19.6.27)のように変換する(19.14)と同じように波線部分を生成子の線型結合で
\begin{align*}
i\sum_a x_a \mc{D}_{a\mu}+i\sum_i t_i \mc{E}_{i\mu}=\gamma^{-1}\partial_\mu \gamma(\xi)-i\sum_A T_A \tilde{A}_{A\mu}
\end{align*}
と書ける.もちろん,19.6節のときと同様の議論により,(21.4.14)の両辺を$x_a$と$t_i$で分解すれば$\mc{D},\mc{E}$の$G$変換性が
\begin{align*}
\sum_a x_a \mc{D}'_{a\mu}&=h(\xi,g)\left( \sum_a x_a \mc{D}_{a\mu} \right)h^{-1}(\xi,g) \\
\sum_i t_i \mc{E}'_{i\mu}&=h(\xi,g)\left(\sum_i t_i \mc{E}_{i\mu} \right)h^{-1}(\xi,g)+i [\partial_\mu h(\xi,g)]h^{-1}(\xi,g)
\end{align*}
であるとわかる.一応これを当てはめてみれば(21.4.14)の右辺は
\begin{align*}
&h(\xi,g)\left[i\sum_a x_a \mc{D}_{a\mu}+i\sum_i t_i \mc{E}_{i\mu}\right] h^{-1}(\xi,g)-[\partial_\mu h(\xi,g)]h^{-1}(\xi,g)  \\
=&i\sum_a x_a\mc{D}'_{a\mu} +i \left\{ h(\xi,g)\left(\sum_i t_i \mc{E}_{i\mu} \right)h^{-1}(\xi,g)+i [\partial_\mu h(\xi,g)]h^{-1}(\xi,g) \right\} \\
=&i\sum_a x_a \mc{D}'_{a\mu}+i\sum_i t_i \mc{E}'_{i\mu}=(21.4.14)左辺
\end{align*}
となって整合性が確認できる.\par
$\mc{E}'$を用いて,物質場の完全に共変な微分
\begin{align*}
\mc{D}_\mu \tilde{\psi}\equiv \partial_\mu \tilde{\psi}+i\sum_i t_i \mc{E}_{i\mu}\tilde{\psi}
\end{align*}
が定義できる!実際これは$G$変換のもとで
\begin{align*}
\mc{D}_\mu\tilde{\psi}=&\partial_\mu \tilde{\psi}+i\sum_i t_i \mc{E}_{i\mu}\tilde{\psi}\\
\overset{G}{\longrightarrow} \quad &\partial_\mu (h(\xi,g)\tilde{\psi})+i\sum_i t_i \mc{E}'_{i\mu}h(\xi,g)\tilde{\psi} \\
=&[\partial_\mu h(\xi,g)]\tilde{\psi}+h(\xi,g)[\partial_\mu \tilde{\psi}] \\
&+h(\xi,g)i\sum_i t_i \mc{E}_{i\mu}h^{-1}(\xi,g) h(\xi,g)\tilde{\psi}-[\partial_\mu h(\xi,g)]h^{-1}(\xi,g)h(\xi,g)\tilde{\psi} \\
=&h(\xi,g)\left(\partial_\mu \tilde{\psi}+i\sum_i t_i \mc{E}_{i\mu}\tilde{\psi} \right)=h(\xi,g)\mc{D}_\mu \tilde{\psi}
\end{align*}
と共変に変換される.この構成から,共変微分されるものが$X\overset{G}{\longrightarrow}h(\xi,g)X$と変換されるならば,それの共変微分はまた共変であることは明らかだ.したがって共変微分の共変微分$\mc{D}_\mu\mc{D}_\nu \tilde{\psi}$などの,もっと高次の微分も構成できる.\par
しかし,(21.4.12)の非斉次項があるため,$i\sum_A T_A \tilde{A}_{A\mu}\overset{G}{\longrightarrow}h(\xi,g)i\sum_A T_A \tilde{A}_{A\mu}$とは変換されない.したがって$\tilde{A}_{A\mu}$あるいは(21.4.19)のような$\tilde{A}_{A\mu}$の共変微分$\mc{D}_\mu \tilde{A}_{A\nu}$は共変ではない.そのため,それらはラグランジアンに自由に導入することはできない.\par
$\Rightarrow$しかし,局所的な$\mc{G}$変換と大域的な$G$変換の両方で共変的に変換する「回転」を構成することは容易だ.それは通常の場の強さテンソルと同様
\begin{align*}
\tilde{F}_{A\mu\nu}\equiv \sum_B D_{AB}(\gamma^{-1}(\xi))\left(\partial_\mu A_{B\nu}-\partial_\nu A_{B\mu} + \sum_{CB}C_{BCD}A_{C\mu}A_{D\nu} \right)
\end{align*}
とすればよい.ここで$C_{BCD}$は
\begin{align*}
[T_A,T_B]=iC_{CAB}T_C
\end{align*}
で与えられる.これが$G$変換のもとで共変であることを一応確認する.
\begin{align*}
\sum_A T_A \tilde{F}_{A\mu\nu}=&\sum_{AB} D_{AB}(\gamma^{-1}(\xi))T_A \left(\partial_\mu A_{B\nu}-\partial_\nu A_{B\mu} + \sum_{CB}C_{BCD}A_{C\mu}A_{D\nu} \right) \\
=&\sum_B \gamma^{-1}(\xi)T_B \gamma(\xi)\left(\partial_\mu A_{B\nu}-\partial_\nu A_{B\mu} + \sum_{CB}C_{BCD}A_{C\mu}A_{D\nu} \right) \\
=&\gamma^{-1}(\xi)\left\{ \partial_\mu\left(\sum_B T_B A_{B\nu}\right)-\partial_\nu \left(\sum_B T_B A_{B\mu}\right)+\sum_{BCD}C_{BCD}T_B A_{C\mu}A_{D\nu} \right\}\gamma(\xi) \\
=&\gamma^{-1}(\xi)\left\{ \partial_\mu\left(\sum_B T_B A_{B\nu}\right)-\partial_\nu \left(\sum_B T_B A_{B\mu}\right)-i\sum_{CD}\left[T_CA_{C\mu},T_DA_{D\nu}\right] \right\}\gamma(\xi) \\
=&\gamma^{-1}(\xi)\left\{ \partial_\mu\left(\sum_B T_B A_{B\nu}\right)-\partial_\nu \left(\sum_B T_B A_{B\mu}\right)-i\left[\sum_CT_CA_{C\mu},\sum_DT_DA_{D\nu}\right] \right\}\gamma(\xi) \\
\overset{G}{\longrightarrow} \sum_A T_A \tilde{F}'_{A\mu\nu}&=\gamma^{-1}(\xi')\left\{ \partial_\mu\left(\sum_B T_B A'_{B\nu}\right)-\partial_\nu \left(\sum_B T_B A'_{B\mu}\right)-i\left[\sum_CT_CA_{C\mu},\sum_DT_DA_{D\nu}\right] \right\}\gamma(\xi')
\end{align*}
括弧の中身を項別に計算すると
\begin{align*}
第一項目=&\partial_\mu\left\{ g\left( \sum_B T_B A_{B\nu} \right)g^{-1}-i(\partial_\nu g)g^{-1} \right\} \\
=&\color{blue}{\underline{\textcolor{black}{[\partial_\mu g]\left(\sum_B T_B A_{B\nu}\right)g^{-1}}}}\color{black}+g\left[\partial_\mu \left(\sum_B T_B A_{B\nu} \right) \right]g^{-1}-\color{blue}{\uwave{\textcolor{black}{g\left(\sum_B T_B A_{B\nu} \right)g^{-1}[\partial_\mu g]g^{-1}}}}\color{black}  \\
&-\color{red}{\underline{\textcolor{black}{i[\partial_\mu \partial_\nu g]g^{-1} }}}\color{black}+\color{green}{\uwave{\textcolor{black}{i[\partial_\nu g]g^{-1}[\partial_\mu g]g^{-1} }}}\color{black} \\
第二項目=&-\partial_\nu\left\{ g\left( \sum_B T_B A_{B\mu} \right)g^{-1}-i(\partial_\mu g)g^{-1} \right\} \\
=&-\color{black}{\underline{\textcolor{black}{[\partial_\nu g]\left(\sum_B T_B A_{B\mu}\right)g^{-1} }}}\color{black}-g\left[\partial_\nu \left(\sum_B T_B A_{B\mu} \right) \right]g^{-1}+\color{black}{\uwave{\textcolor{black}{g\left(\sum_B T_B A_{B\mu} \right)g^{-1}[\partial_\nu g]g^{-1} }}}\color{black} \\
&+\color{red}{\underline{\textcolor{black}{i[\partial_\nu \partial_\mu g]g^{-1} }}}\color{black}-\color{green}{\underline{\textcolor{black}{i[\partial_\mu g]g^{-1}[\partial_\nu g]g^{-1} }}} \\
第三項目=&-i\left[\left\{g\left(\sum_C T_C A_{C\mu}\right)g^{-1}-i[\partial_\mu g]g^{-1}\right\},\left\{g\left[\sum_D T_D A_{D\nu}\right]g^{-1}-i(\partial_\nu g)g^{-1}\right\}\right] \\
=&-i\left[ g\left(\sum_C T_C A_{C\mu} \right)g^{-1},g\left(\sum_D T_D A_{D\nu}\right)g^{-1} \right] -i\left[g\left(\sum_{C}T_C A_{C\mu}\right)g^{-1},-i[\partial_\nu g]g^{-1}\right] \\
&-i\left[-i[\partial_\nu g]g^{-1},g\left(\sum_D T_D A_{D\nu}\right)g^{-1}\right]-i\left[- i[\partial_\mu g]g^{-1},-i[\partial_\nu g]g^{-1}\right] \\
=&-ig\left[\sum_C T_C A_{C\mu},\sum_D T_D A_{D\nu} \right]g^{-1} \\
&+\underline{[\partial_\nu g]\left(\sum_C T_C A_{C\mu} \right)g^{-1}}-\uwave{g\left(\sum_C T_C A_{C\mu}\right)g^{-1}[\partial_\nu g]g^{-1}} \\
&-\color{blue}{\underline{\textcolor{black}{[\partial_\mu g]\left( \sum_D T_D A_{D\nu} \right)g^{-1} }}}\color{black}+\color{blue}{\uwave{\textcolor{black}{g\left(\sum_D T_D A_{D\nu}\right)g^{-1}[\partial_\mu g]g^{-1} }}} \\
&+\color{green}{\underline{\textcolor{black}{i[\partial_\mu g]g^{-1}[\partial_\nu g]g^{-1} }}}\color{black}-\color{green}{\uwave{\textcolor{black}{i[\partial_\nu g]g^{-1}[\partial_\mu g]g^{-1} }}}\color{black}
\end{align*}
したがって
\begin{align*}
\sum_A T_A \tilde{F}'_{A\mu\nu}=&\gamma^{-1}(\xi')g\left\{ \partial_\mu\left(\sum_B T_B A_{B\nu}\right)-\partial_\nu \left(\sum_B T_B A_{B\mu}\right)-i\sum_{CD}\left[T_CA_{C\mu},T_DA_{D\nu}\right] \right\}g^{-1}\gamma(\xi') \\
=&\gamma^{-1}(\xi')g\gamma(\xi)\sum_A T_A \tilde{F}_{A\mu\nu} \gamma^{-1}(\xi)g^{-1}\gamma(\xi') \\
=&\sum_A h(\xi,g)T_A \tilde{F}_{A\mu\nu} h^{-1}(\xi,g) \quad \because (21.4.11) \\
=&\sum_{AB}D_{AB}(h(\xi,g))T_B \tilde{F}_{A\mu\nu}=\sum_{AB}T_A \left\{D_{AB}(h(\xi,g))\tilde{F}_{B\mu\nu} \right\} \\
\Rightarrow \quad &\tilde{F}_{A\mu\nu} \overset{G}{\longrightarrow} \tilde{F}'_{A\mu\nu}=\sum_B D_{AB}(h(\xi,g))\tilde{F}_{B\mu\nu}
\end{align*}
が$\tilde{F}_{A\mu\nu}$の変換則となり,共変であることがわかる.したがって,ラグランジアンは,$\tilde{\psi},\mc{D}_{a\mu},\mc{D}_\mu\tilde{\psi},\tilde{F}_{A\mu\nu}$,およびもっと高次の共変微分の任意関数で大域的$H$不変性を満たすように構成すれば,\uwave{形式的な局所$G$変換}のもとで不変となる.\par
ここで現実に戻って,$A_A^\mu$を制限された形(21.4.3)の量子場として扱う.前述の通り,これにより今まで取り扱ってきた$A_{A\mu}\to A'_{A\mu}$変換は$\mc{A}_{\alpha\mu}\to\mc{A}'_{\alpha\mu}$変換となる.今までの議論が成り立つためにはこれらの変換が同値でなければならないので,局所$G$不変性は破れており,ゲージ結合を保つ真の不変群は局所$\mc{G}$群となる.これまでの量を$\mc{A}$で書き換えよう.(21.4.15)(21.4.20)は
\begin{align*}
\sum_a ix_a \mc{D}_{a\mu}+\sum_i it_i \mc{E}_{i\mu}=&\gamma^{-1}(\xi)\partial_\mu \gamma(\xi)-i\sum_A T_A \tilde{A}_{A\mu} \\
=&\gamma^{-1}(\xi)\partial_\mu \gamma(\xi)-i\sum_{AB}T_A D_{AB}(\gamma^{-1}(\xi))A_{B\mu} \quad \because (21.4.7)\\
=&\gamma^{-1}(\xi)\partial_\mu \gamma(\xi)-i\sum_{AB}T_A D_{AB}(\gamma^{-1}(\xi))e_{\alpha B}\mc{A}_{\alpha\mu} \quad \because (21.4.3)
\end{align*}
および
\begin{align*}
\tilde{F}_{A\mu\nu}=&\sum_B D_{AB}(\gamma^{-1}(\xi))\left\{\partial_\mu \left( \sum_\alpha e_{\alpha B}\mc{A}_{\alpha \nu} \right)-\partial_\nu \left(\sum_\alpha e_{\alpha B}\mc{A}_{\alpha \mu} \right)+\sum_{CD\gamma\delta}C_{BCD}e_{\gamma C}e_{\delta D}\mc{A}_{\gamma\mu}\mc{A}_{\delta\nu}\right\} \\
=&\sum_B D_{AB}(\gamma^{-1}(\xi))\left\{\partial_\mu \left( \sum_\alpha e_{\alpha B}\mc{A}_{\alpha \nu} \right)-\partial_\nu \left(\sum_\alpha e_{\alpha B}\mc{A}_{\alpha \mu} \right)+\sum_{\beta\gamma\delta}\mc{C}_{\beta\gamma\delta}e_{\beta B} \mc{A}_{\gamma\mu}\mc{A}_{\delta\nu}\right\} (21.4.25)\\
=&\sum_{B\alpha}D_{AB}(\gamma^{-1}(\xi))e_{\alpha B}\left(\partial_\mu \mc{A}_{\alpha\nu}-\partial_\nu \mc{A}_{\alpha\mu} +\sum_{\gamma\delta}\mc{C}_{\alpha\gamma\delta}\mc{A}_{\gamma\mu}\mc{A}_{\delta\nu} \right) \\
=&\sum_{B\alpha}D_{AB}(\gamma^{-1}(\xi))e_{\alpha B}\mc{F}_{\alpha\mu\nu}
\end{align*}
となる.ここで
\begin{align*}
\mc{F}_{\beta\mu\nu}\equiv \partial_\mu\mc{A}_{\beta\nu}-\partial_\nu\mc{A}_{\beta\mu}+\sum_{\gamma\delta}\mc{C}_{\beta\gamma\delta}\mc{A}_{\gamma\mu}\mc{A}_{\delta\nu}
\end{align*}
と定義した.また途中で,ゲージ群の構造定数$\mc{C}$と,$G$の構造定数$C$は
\begin{align*}
[T_A,T_B]=iC_{CAB}T_C ,\quad [\mc{T}_\alpha ,\mc{T}_\beta]=i\mc{C}_{\gamma\alpha\beta}\mc{T}_\gamma ,\quad \mc{T}_\alpha =\sum_A e_{\alpha A}T_A 
\end{align*}
より
\begin{align*}
[\mc{T}_\gamma,\mc{T}_\delta]=&i\sum_\beta \mc{C}_{\beta\gamma\delta}\mc{T}_\beta=i\sum_{\beta B}\mc{C}_{\beta\gamma\delta}e_{\beta B}T_B \\
=&\left[\sum_C e_{\gamma C}T_C,\sum_D e_{\delta D}T_D\right]=i\sum_{BCD}C_{BCD}e_{\gamma C} e_{\delta D}T_B \\
\Rightarrow \quad & \sum_{CD}C_{BCD}e_{\gamma C}e_{\delta D}=\sum_\beta \mc{C}_{\beta\gamma\delta}e_{\beta B}
\end{align*}
の関係があることを用いた.\par
ラグランジアンに,この場の運動項として通常の項
\begin{align*}
\mc{L}_\mc{A}=&-\frac{1}{4}\mr{Tr}\left[\sum_A T_A \tilde{F}_{A\mu\nu} \sum_B T_B \tilde{F}_B^{\mu\nu} \right] \\
=&-\frac{1}{4}\mr{Tr}\left[ \sum_{AC\alpha}D_{AC}(\gamma^{-1}(\xi))T_A e_{\alpha C} \mc{F}_{\alpha\mu\nu} \sum_{BD\beta} D_{BD}(\gamma^{-1}(\xi))T_B e_{\beta D} \mc{F}_\beta^{\mu\nu} \right] \\
=&-\frac{1}{4}\mr{Tr}\left[ \sum_{C\alpha} \gamma^{-1}(\xi)e_{\alpha C} T_C \mc{F}_{\alpha\mu\nu}\gamma(\xi) \sum_{D\beta} \gamma^{-1}(\xi)e_{\beta D}T_D \mc{F}_\beta^{\mu\nu}\gamma(\xi) \right] \\
=&-\frac{1}{4}\mr{Tr}\left[ \sum_{C\alpha} e_{\alpha C} T_C \mc{F}_{\alpha\mu\nu} \sum_{D\beta} e_{\beta D}T_D \mc{F}_\beta^{\mu\nu} \right] \quad \because トレースの巡回性\\
=&-\frac{1}{4}\mr{Tr}\left[ \sum_\alpha \mc{T}_\alpha \mc{F}_{\alpha\mu\nu} \sum_\beta \mc{T}_\beta \mc{F}_\beta^{\mu\nu} \right] \\
=&-\frac{1}{4}\sum_{\alpha\beta}\mr{Tr}[\mc{T}_\alpha \mc{T}_\beta]\mc{F}_{\alpha\mu\nu}\mc{F}_\beta^{\mu\nu} \\
=&-\frac{1}{4}\sum_{\alpha\beta}\delta_{\alpha\beta}\mc{F}_{\alpha\mu\nu}\mc{F}_\beta^{\mu\nu} \\
=&-\frac{1}{4}\sum_\alpha \mc{F}_{\alpha\mu\nu}\mc{F}_\alpha^{\mu\nu}
\end{align*}
を含める.ここで,$\mc{A}_{\alpha\mu\nu}$の線型変換に伴って$e_{\alpha A}$の線型変換をすることで,$\mr{Tr}[\mc{T}_\alpha \mc{T}_\beta]=\delta_{\alpha\beta}$となるようにしている.この項も明らかに$G$不変だ.なぜなら
\begin{align*}
\mc{L}'_\mc{A}=&-\frac{1}{4}\mr{Tr}\left[\sum_A T_A \tilde{F}'_{A\mu\nu} \sum_B T_B \tilde{F}_B^{\prime\mu\nu} \right] \\
=&-\frac{1}{4}\mr{Tr}\left[\sum_A h(\xi,g)T_A \tilde{F}_{A\mu\nu} h^{-1}(\xi,g)\sum_B h(\xi,g) T_B \tilde{F}_B^{\mu\nu} h^{-1}(\xi,g)\right] \\
=&-\frac{1}{4}\mr{Tr}\left[\sum_A T_A \tilde{F}_{A\mu\nu} \sum_B T_B \tilde{F}_B^{\mu\nu} \right] =\mc{L}_\mc{A}
\end{align*}
となるからだ.(くどいようだが,この不変性は局所$G$変換による$A\to A'$の変換のもとでの不変性であり,$\mc{A}$で書き換えた今,$\mc{A}\to\mc{A}'$の変換のもとでも同様に不変であるためにはやはりゲージ$\mc{G}$部分群における変換でなければならない.)$\mc{F}_{\alpha\mu\nu}$の線型項は(21.4.24)より$\partial_\mu \mc{A}_{\alpha\nu}-\partial_\nu \mc{A}_{\alpha\mu}$なので,(21.4.26)は$\mc{A}_{\alpha\mu}$を正準規格化されたベクトル場にする効果がある.以上よりラグランジアン密度は,大域的$H$変換のもとで不変なように$\tilde{\psi},\mc{D}_\mu \tilde{\psi},\mc{D}_{a\mu}$から構成された関数と,2個以上の$e_{\alpha A}$因子を持ち$\mc{G}$を保存する(が局所的$G$は保存しない)可能な項,の和で与えられる.

\vskip\baselineskip

次に,以上の構成要素から,どのような摂動論が構成できるかを見る.$e_{\alpha A}\to0$の極限でゲージボゾンは自発的対称性の破れを被る物質場との結合が切れる(共変微分(21.4.19)の$e_{\alpha A}\mc{A}_{\alpha\mu}$がなくなるので,$\mc{A}$と$\tilde{\psi}$の相互作用がなくなる)ので,質量ゼロとなる.\par
$\Rightarrow$したがって,とりあえず$e_{\alpha A}$が小さいときは,ゲージボゾンらの質量は$eM$の程度だと考えておく.ここで$e$は$e_{\alpha A}$の典型的な値で,$M$は自発的対称性の破れを生じる力学に典型的なエネルギースケールだ.\par
ここで,エネルギー(運動量)が$Q\lesssim eM$のゲージボゾンとNGボゾンを含んでいる,一般的なファインマンダイアグラムを考える.(ただし,$Q$より大きいエネルギー(運動量)を持つ全ての粒子や,$Q$より重い物質場粒子は有効場理論の結合定数への補正として埋め込まれている,とする.)ここでの摂動論は,$e$と$Q/M$についてのベキ展開だ.(12.1.6)より,$(Q/M)^{\nu_1}$と$e^{\nu_2}$のベキの総数$\nu=\nu_1+\nu_2$は
\begin{align*}
\nu_1&=2I_{NG}+2I_{\mc{A}}+\sum_i V_i (d_i-4)+4 \\
&=2(L+\sum_i V_i -1)+\sum_i V_i (d_i -4)+4 \quad \because トポロジー関係式 L-\sum_f I_f +\sum_i V_i =1 \\
&=\sum_i V_i (d_i -2)+2L+2 \\
\nu_2&=\sum_i V_i e_i \quad \because e_i は各頂点V_i での因子eの数 \\
\Rightarrow \quad \nu&=\nu_1+\nu_2 =\sum_i V_i(d_i+e_i -2)+2L+2
\end{align*}
となる.ここで$I_f$は種類$f$の粒子の内線の数,$d_i$は頂点$V_i$での微分の数だ.さてここで\\
・(21.4.3)の拘束があると,場$A_{A\mu}$または$\tilde{A}_{A\mu}$からは$e$の因子1個の寄与がある.\\
・(21.4.15)から,各NGボゾン共変微分$\mc{D}_{a\mu}$からは$(d_i+e_i)$に$+1$の寄与がある.\\
・(21.4.19)(21.4.15)から,その他の共変微分$\mc{D}_\mu$の各々からは$(d_i+e_i)$に別の$+1$の寄与がある.\\
に留意すれば,ラグランジアンの全ての可能な項は,$(d_i+e_i)\geq 2$を持つ,ということに気が付く.なぜなら全体で$G$不変であるために可能な項は,ヤン・ミルズ項(21.4.26)か,共変微分の二次以上などでなければならないからだ.そしてヤン・ミルズ項(21.4.26)は当然$d_i=2$を持つからだ.(物質場の二次の項$|\tilde{\psi}|^2$のような項は$(d_i+e_i)<2$を持つが,前述の通りこれは補正として扱うので除外される.)\par
$\Rightarrow$主要な項は$(d_i+e_i)=2$の相互作用で構成される樹木$(L=0)$ダイアグラムから来る.\\
唯一そのような相互作用は,NGボゾンの運動項
\begin{align*}
\mc{L}=-\frac{1}{2}\sum_{ab}F^2_{ab}\mc{D}_{a\mu}\mc{D}_b^\mu
\end{align*}
と,ヤン・ミルズ項(21.4.26),および$e_{\alpha A}$について2次($e_i=2$)で局所$G$対称性を破る非微分($d_i=0$)項だ.

\vskip\baselineskip

ここで少し脱線して,$\mc{D}_{a\mu},\mc{E}_{i\mu}$の具体的な形を19.6節と同様に求めておこう.
\begin{align*}
\gamma^{-1}(\xi)\partial_\mu \gamma(\xi)=&\exp\left[-i\sum_b x_b \xi_b\right]\partial_\mu \exp\left[i\sum_a x_a \xi_a\right] \\
=&\exp\left[-i\sum_b x_b \xi_b\right]i\sum_a x_a \partial_\mu \xi_a \exp \left[i\sum_c x_c \xi_c\right]
\end{align*}
ここでベーカー・キャンベル・ハウスドロフの公式
\begin{align*}
e^A B e^{-A}=B+[A,B]+\frac{1}{2!}\left[A,[A,B]\right]+\frac{1}{3!}\left[A,\left[A,[A,B]\right]\right]+\cdots
\end{align*}
を用いると(19.6.8)$\sim$(19.6.10)より
\begin{align*}
\gamma^{-1}\partial_\mu\gamma(\xi)=&i\sum_a x_a \partial_\mu \xi_a -\left[i\sum_b x_b \xi_b ,i\sum_a x_a\partial_\mu\xi_a\right]+\frac{1}{2}\left[i\sum_b x_b \xi_b \left[ i\sum_c x_c \xi_c , i\sum_a x_a \partial_\mu \xi_a \right]\right] +\cdots \\
=&i\sum_a x_a \partial_\mu \xi_a+\sum_{ab}[x_b,x_a]\xi_b \partial_\mu \xi_a-\frac{1}{2}\sum_{abc}i\left[x_b ,[x_c,x_a]\right]\xi_b\xi_c \partial_\mu \xi_a +O(\xi^3\partial_\mu \xi) \\
=&i\sum_a x_a \partial_\mu \xi_a+\sum_{abi} it_i C_{iba}\xi_b \partial_\mu \xi_a +\sum_{abc}ix_c C_{cba}\xi_b \partial_\mu \xi_a \\
&-\frac{1}{2}\sum_{abc} i\left[x_b,i\sum_i t_iC_{ica}+i\sum_d x_d C_{dca} \right]\xi_b \xi_c \partial_\mu \xi_a+O(\xi^3\partial_\mu\xi) \\
=&i\sum_a x_a \partial_\mu \xi_a+i\sum_{abi} t_i C_{iba}\xi_b \partial_\mu \xi_a +i\sum_{abc}x_c C_{cba}\xi_b \partial_\mu \xi_a \\
&-\frac{1}{2}\sum_{abci}[t_i ,x_b]C_{ica}\xi_b\xi_c\partial_\mu \xi_a+\frac{1}{2}\sum_{abcd}[x_b,x_d]C_{dca}\xi_b \xi_c \partial_\mu \xi_a+O(\xi^3\partial_\mu\xi) \\
=&i\sum_a x_a \partial_\mu \xi_a+i\sum_{abi} t_i C_{iba}\xi_b \partial_\mu \xi_a +i\sum_{abc}x_c C_{cba}\xi_b \partial_\mu \xi_a \\
&-\frac{1}{2}i\sum_{abcdi}x_dC_{dib}C_{ica}\xi_b\xi_c\partial_\mu \xi_a \\
&+\frac{1}{2}i\sum_{abcdi}t_iC_{ibd}C_{dca}\xi_b\xi_c\partial_\mu\xi_a+\frac{1}{2}\sum_{abcde}x_eC_{ebd}C_{dca}\xi_b\xi_c\partial_\mu\xi_a+O(\xi^3\partial_\mu\xi)
\end{align*}
ここで見やすいように添え字を入れ替えると
\begin{align*}
=&i\sum_a x_a \partial_\mu \xi_a+i\sum_{abi} t_i C_{iab}\xi_a \partial_\mu \xi_b +i\sum_{abc}x_a C_{abc}\xi_b \partial_\mu \xi_c \\
&+\frac{1}{2}i\sum_{abcdi}x_a C_{abi}C_{icd}\xi_b \xi_c\partial_\mu\xi_d \\
&+\frac{1}{2}i\sum_{abcdi}t_i C_{iab}C_{bcd}\xi_a \xi_c \partial_\mu\xi_d+\frac{1}{2}i\sum_{abcde}x_a C_{abc}C_{cde}\xi_b \xi_d \partial_\mu\xi_e+O(\xi^3\partial_\mu\xi)
\end{align*}
また
\begin{align*}
i\sum_A T_A \tilde{A}_{A\mu}=&i\sum_A \left\{\gamma^{-1}(\xi)T_A \gamma(\xi)\right\}A_{A\mu} \\
=&i\sum_a \left\{\gamma^{-1}(\xi)x_a \gamma(\xi)\right\}A_{a\mu}+i\sum_i \left\{\gamma^{-1}(\xi)t_i \gamma(\xi)\right\}A_{i\mu}
\end{align*}
も調べなくてはならない.第一項目は,先ほどの計算での$\partial_\mu \xi_a$を$A_{a\mu}$に置き換えたものと等しいので
\begin{align*}
&i\sum_a \left\{\gamma^{-1}(\xi)x_a \gamma(\xi)\right\}A_{a\mu}\\
=&i\sum_a x_a A_{a\mu}+i\sum_{abi} t_i C_{iab}\xi_a A_{b\mu} +i\sum_{abc}x_a C_{abc}\xi_b A_{c\mu} \\
&+\frac{1}{2}i\sum_{abcdi}x_a C_{abi}C_{icd}\xi_b \xi_cA_{d\mu} \\
&+\frac{1}{2}i\sum_{abcdi}t_i C_{iab}C_{bcd}\xi_a \xi_c A_{d\mu}+\frac{1}{2}i\sum_{abcde}x_a C_{abc}C_{cde}\xi_b \xi_d A_{e\mu}+O(\xi^3A)
\end{align*}
となる.第二項目は
\begin{align*}
&i\sum_i\left\{\gamma^{-1}(\xi)t_i \gamma(\xi)\right\}A_{i\mu} \\
=&i\sum_i\left\{t_i-i\sum_a[x_a,t_i]\xi_a-\frac{1}{2}\sum_{ab}[x_b,[x_a,t_i]]\xi_a \xi_b+O(\xi^3)\right\}A_{i\mu} \\
=&i\sum_i\left\{ t_i-\frac{1}{2}\sum_{ab}x_bC_{bia}\xi_a -\frac{1}{2}\sum_{abc}[x_b,x_c]iC_{cai}\xi_a\xi_b \right\}A_{i\mu}+O(\xi^3A) \\
=&i\sum_i\left\{ t_i-\sum_{ab}x_aC_{abi}\xi_b+\frac{1}{2}\sum_{abcj}t_j C_{jbc}C_{cai}\xi_a\xi_b +\frac{1}{2}\sum_{abcd}x_d C_{dbc}C_{cai}\xi_a \xi_b \right\}A_{i\mu}+O(\xi^3A) 
\end{align*}
添え字を入れ替えて
\begin{align*}
=&i\sum_i\left\{ t_i+\sum_{ab}x_aC_{abi}\xi_b+\frac{1}{2}\sum_{abcj}t_j C_{jab}C_{bci}\xi_a\xi_c +\frac{1}{2}\sum_{abcd}x_a C_{abc}C_{cdi}\xi_b \xi_d \right\}A_{i\mu}+O(\xi^3A) \\
=&i\sum_i t_iA_{i\mu}+i\sum_{ab}x_aC_{abi}\xi_bA_{i\mu}+i\frac{1}{2}\sum_{abcj}t_j C_{jab}C_{bci}\xi_a\xi_cA_{i\mu} +i\frac{1}{2}\sum_{abcd}x_a C_{abc}C_{cdi}\xi_b \xi_dA_{i\mu}+O(\xi^3A)
\end{align*}
となる.以上より
\begin{align*}
&i\sum_a x_a \mc{D}_{a\mu}+i\sum_i t_i \mc{E}_{i\mu}=\gamma^{-1}(\xi)\partial_\mu\gamma(\xi)-i\sum_A T_A \tilde{A}_{A\mu} \\
=&\underline{i\sum_a x_a \partial_\mu \xi_a}+\uwave{i\sum_{abi} t_i C_{iab}\xi_a \partial_\mu \xi_b} +\underline{i\sum_{abc}x_a C_{abc}\xi_b \partial_\mu \xi_c} \\
&+\underline{\frac{1}{2}i\sum_{abcdi}x_a C_{abi}C_{icd}\xi_b \xi_c\partial_\mu\xi_d} \\
&+\uwave{\frac{1}{2}i\sum_{abcdi}t_i C_{iab}C_{bcd}\xi_a \xi_c \partial_\mu\xi_d}+\underline{\frac{1}{2}i\sum_{abcde}x_a C_{abc}C_{cde}\xi_b \xi_d \partial_\mu\xi_e} \\
&-\underline{i\sum_a x_a A_{a\mu}}-\uwave{i\sum_{abi} t_i C_{iab}\xi_a A_{b\mu}} -\underline{i\sum_{abc}x_a C_{abc}\xi_b A_{c\mu}} \\
&-\underline{\frac{1}{2}i\sum_{abcdi}x_a C_{abi}C_{icd}\xi_b \xi_cA_{d\mu} }\\
&-\uwave{\frac{1}{2}i\sum_{abcdi}t_i C_{iab}C_{bcd}\xi_a \xi_c A_{d\mu}}-\underline{\frac{1}{2}i\sum_{abcde}x_a C_{abc}C_{cde}\xi_b \xi_d A_{e\mu}} \\
&-\uwave{i\sum_i t_iA_{i\mu}}-\underline{i\sum_{ab}x_aC_{abi}\xi_bA_{i\mu}}-\uwave{i\frac{1}{2}\sum_{abcj}t_j C_{jab}C_{bci}\xi_a\xi_cA_{i\mu}} -\underline{i\frac{1}{2}\sum_{abcd}x_a C_{abc}C_{cdi}\xi_b \xi_dA_{i\mu}} \\
&+O(\xi^3\partial_\mu\xi,\xi^3A) \\
=&i\sum_a x_a\Biggl[(\partial_\mu\xi_a -A_{a\mu})+\sum_{bc}C_{abc}\xi_b(\partial_\mu\xi_c-A_{c\mu})-\sum_{bc}C_{abi}\xi_bA_{i\mu} \\
&\qquad\qquad +\frac{1}{2}\sum_{bcd}\left(\sum_i C_{abi}C_{icd}+\sum_e C_{abe}C_{ecd} \right)\xi_b\xi_c(\partial_\mu\xi_d-A_{d\mu})-\frac{1}{2}\sum_{bcdi}C_{abd}C_{dci}\xi_b\xi_cA_{i\mu} \\
&\qquad \qquad +O(\xi^3\partial_\mu\xi,\xi^3A) \Biggr] \\
&+i\sum_i t_i\Biggl[-A_{i\mu}+\sum_{ab}C_{iab}\xi_a(\partial_\mu\xi_b-A_{b\mu}) \\
&\qquad\qquad +\frac{1}{2}\sum_{abcd}C_{iab}C_{bcd}\xi_a\xi_c(\partial_\mu\xi_d-A_{d\mu})-\frac{1}{2}\sum_{abcj}C_{iab}C_{bcj}\xi_a\xi_cA_{j\mu} \\
&\qquad\qquad +O(\xi^3\partial_\mu\xi,\xi^3A) \Biggr]
\end{align*}
したがって
\begin{align*}
\mc{D}_{a\mu}=&(\partial_\mu\xi_a -A_{a\mu})+\sum_{bc}C_{abc}\xi_b(\partial_\mu\xi_c-A_{c\mu})-\sum_{bc}C_{abi}\xi_bA_{i\mu} \\
&+\frac{1}{2}\sum_{bcd}\left(\sum_i C_{abi}C_{icd}+\sum_e C_{abe}C_{ecd} \right)\xi_b\xi_c(\partial_\mu\xi_d-A_{d\mu})-\frac{1}{2}\sum_{bcdi}C_{abd}C_{dci}\xi_b\xi_cA_{i\mu} \\
&+O(\xi^3\partial_\mu\xi,\xi^3A) \\
\mc{E}_{i\mu}=&-A_{i\mu}+\sum_{ab}C_{iab}\xi_a(\partial_\mu\xi_b-A_{b\mu}) \\
&+\frac{1}{2}\sum_{abcd}C_{iab}C_{bcd}\xi_a\xi_c(\partial_\mu\xi_d-A_{d\mu})-\frac{1}{2}\sum_{abcj}C_{iab}C_{bcj}\xi_a\xi_cA_{j\mu} \\
&+O(\xi^3\partial_\mu\xi,\xi^3A)
\end{align*}
という結果が得られる.これは少し興味深い結果だ!$\mc{D}_{a\mu}$と$\mc{E}_{i\mu}$の非対称性は,破れていない群$H$の生成子の添え字に対応するNGボゾン場が存在しないことと,(19.6.9)で見たように$C_{iaj}$がゼロであることから来る.特別に,$\xi_i=0,C_{iaj}=0$を導入するとこれらは統一的に
\begin{align*}
\tilde{\mc{D}}_{A\mu}=&\left\{
\begin{array}{ll}
\mc{D}_{a\mu} &(Aが破れた群G/Hの生成子の添え字a)\\
\mc{E}_{i\mu} &(Aが破れていない群Hの生成子の添え字i)
\end{array}
\right. \\
=&(\partial_\mu \xi_A-A_{A\mu})+\sum_{BC}C_{ABC}\xi_B (\partial_\mu\xi_C-A_{C\mu}) \\
&+\frac{1}{2}\sum_{BCDE}C_{ABE}C_{ECD}\xi_B\xi_C (\partial_\mu\xi_D-A_{D\mu})+O(\xi^3\partial_\mu\xi,\xi^3A)
\end{align*}
と書けることに気付く!ここで大文字の添え字についての和は,この節の最初から用いているように$i,a$についての和だ.ここまでの計算はベーカー・キャンベル・ハウスドロフの公式の3次までの項で打ち止めしていたが,高次の項に拡張する法則性はこの式から予想できるだろう.すなわち
\begin{align*}
\tilde{\mc{D}}_{A\mu}&=(\partial_\mu \xi_A-A_{A\mu})+\sum_{n=1}^\infty \frac{1}{n!}\sum_{\substack{B_1\sim B_n \\ C_1\sim C_n}}C_{AB_1C_1}\xi_{B_1}\prod^{n-1}_{i=1}\left[C_{C_i B_i C_{i+1}}\xi_{B_i}\right](\partial_\mu \xi_{C_n}-A_{C_n\mu}) \\
&=(\partial_\mu \xi_A-\sum_{\alpha}e_{\alpha A}\mc{A}_{\alpha\mu})+\sum_{n=1}^\infty \frac{1}{n!}\sum_{\substack{B_1\sim B_n \\ C_1\sim C_n}}C_{AB_1C_1}\xi_{B_1}\prod^{n-1}_{i=1}\left[C_{C_i B_i C_{i+1}}\xi_{B_i}\right](\partial_\mu \xi_{C_n}-\sum_{\alpha}e_{\alpha C_n}\mc{A}_{\alpha\mu})
\end{align*}
となる.これはなかなかに綺麗だ!

\vskip\baselineskip

場$\xi_a$の物理的な意味を知るには,$\mc{D}_{a\mu}$の線型項が
\begin{align*}
(\mc{D}_{a\mu})_{\mr{LIN}}=\partial_\mu \xi_a-\sum_{\alpha}e_{\alpha a}\mc{A}_{\alpha\mu}
\end{align*}
であることに気付けば良い.補遺で示すように,全ての$\alpha$について
\begin{align*}
\sum_{ab}F^2_{ab}\xi_a e_{\alpha b}=0
\end{align*}
を満たし,したがって(21.4.28)の交差項がゼロになる「ユニタリー・ゲージ」をとることが常に可能だ.\par
全ての破れた対称性がゲージ対称性$\mc{G}$である(つまり,ゲージ対称性以外の$G$は破れていない)\uwave{特別な場合}には,$\mc{G}$の生成子$\mc{T}_\alpha$と破れていない群$H$の生成子$t_i$の線型結合として
\begin{align*}
x_a=&\sum_\alpha c_{a\alpha}\mc{T}_\alpha +\sum_i c_{a i}t_i \\
=& \sum_{\alpha}c_{a \alpha}\left(\sum_{b}e_{\alpha b}x_b+ \sum_i e_{\alpha i}t_i \right)+\sum_i c_{a i}t_i \\
=& \sum_{\alpha b}c_{a\alpha}e_{\alpha b }x_b +\sum_{\alpha i}c_{a\alpha}e_{\alpha i}t_i+\sum_i c_{a i}t_i
\end{align*}
と書くことができ,係数比較すれば
\begin{align*}
&\sum_{\alpha b}c_{a\alpha}e_{\alpha b}x_b=x_a \\
\Rightarrow \quad &\sum_{\alpha}c_{a \alpha}e_{\alpha b}=\delta_{ab}
\end{align*}
となる.(21.4.30)の両辺で$c_{a\alpha}$を縮約すると,$F^2_{ab}$は正定値なので
\begin{align*}
\sum_{\alpha ab} F^2_{ab}\xi_a c_{c\alpha}e_{\alpha b}=&\sum_{ab}F^2_{ab}\xi_a \delta_{cb} \\
=&\sum_{a}F^2_{ac}\xi_a=0
\end{align*}
より$\xi_a=0$が得られる.(正定値であることと,全ての固有値は非ゼロの正であることは同値だということを思い出せば自明だ.)\par
$\Rightarrow$ユニタリー・ゲージでは,NGボゾンが\uwave{全く存在しない}!\\
より一般には,破れた対称性のうちゲージ対称性以外の部分は$\mc{G}$の部分集合でも$H$の部分集合でもないため,上記のように線型結合で$x_a$が記述できない.したがって$x_a$の添え字$a$がゲージ対称性以外の破れた群の生成子添え字である場合は$c_{a\alpha}$は存在せず,(21.4.30)の条件より,ゲージ対称性に\uwave{対応しない}NGボゾンだけが残る.\par
$\Rightarrow$残ったNGボゾンのうちいくつかは,ゲージ結合について2次の質量をもち,質量のあるNGボゾンであるからそれらは擬NGボゾンだ.\par
$\xi$がユニタリー・ゲージ条件(21.4.30)を満たすように選ぶと,ラグランジアン(21.4.28)の2次の部分は
\begin{align*}
(\mc{L}_{\xi})_{\mr{QUAD}}=&-\frac{1}{2}\sum_{ab}F^2_{ab}\partial_\mu \xi_a \partial^\mu \xi_b-\frac{1}{2}\sum_{ab\alpha\beta}F^2_{ab}e_{\alpha a}\mc{A}_{\alpha\mu}e_{\beta b}\mc{A}_{\beta}^\mu \\
=&-\frac{1}{2}\sum_{ab}F^2_{ab}\partial_\mu \xi_a \partial^\mu \xi_b-\frac{1}{2}\sum_{\alpha\beta}\mu^2_{\alpha\beta}\mc{A}_{\alpha\mu}\mc{A}_\beta^\mu
\end{align*}
となる.ここで
\begin{align*}
\mu^2_{\alpha\beta}=\sum_{ab}F^2_{ab}e_{\alpha a}e_{\beta b}
\end{align*}
と定義した.\par
ここから読み取れることとして,$\xi_a$は正準的に直交規格化された場$\pi_a$を用いて
\begin{align*}
\xi_a=\sum_b F^{-1}_{ab}\pi_b
\end{align*}
と表される.ここで$F^{-1}_{ab}$は正定値行列$F^2_{ab}$の正の平方根だ.19章の内容を思い出すと,これは$F^{-1}_{ab}$が低エネルギーのNGボゾンの放出・吸収に付随する$F^{-1}_\pi$に類似の因子だと理解できる.加えて,(21.4.26)で$\mc{A}_{\alpha\mu}$は正準規格化されたベクトル場と定義されたから,(21.4.31)より$\mu^2_{\alpha\beta}$がベクトル・ボゾンの質量行列の2乗だと理解できる.(21.4.32)はベクトル・ボゾンの質量行列の公式だ.実際,(21.1.7)に(21.4.1)を使えば,破れていない対称性の生成子$t_i$に関して$(t_i v)_n=0$であることを思い出せば
\begin{align*}
\mu^2_{\alpha\beta}=&-\sum_{nm\ell}(\mc{T}_\alpha)_{nm} (\mc{T}_\beta)_{n\ell} v_m v_\ell \\
=&-\sum_{nm\ell}\left[\sum_{a}e_{\alpha a}x_a +\sum_{i}e_{\alpha i}t_i \right]_{nm}\left[\sum_{b}e_{\beta b}x_b +\sum_{j}e_{\beta j}t_j \right]_{n\ell}v_m v_\ell \\
=&-\sum_{abnm\ell}(x_a)_{nm}(x_b)_{n\ell}v_m v_\ell e_{\alpha a}e_{\beta b} =\sum_{ab}F^2_{ab}e_{\alpha a}e_{\beta b} \\
\Rightarrow \quad F^2_{ab}=&-\sum_{nm\ell}(x_a)_{nm}(x_b)_{n\ell}v_m v_\ell
\end{align*}
で,(21.4.32)の特別な場合であることが容易にわかる.

\vskip\baselineskip

一般に$F^2_{ab}$行列はそのままでは計算できないが,破れていない部分群$H$のもとで運動項(21.4.28)が不変,すなわち(21.4.17)より
\begin{align*}
h(\xi,g)x_a h^{-1}(\xi,g)=&\exp\left[i\sum_i t_i \theta_i \right]x_a \exp\left[-i\sum_j t_j \theta_j \right] \\
=&x_a+\left[i\sum_i t_i \theta_i ,x_a \right]+\cdots \\
\Rightarrow \quad \delta \mc{D}_{a\mu}=&i\sum_i\theta_i[iC_{iab}]\mc{D}_{b\mu} \\
0=&\sum_{ab}\frac{\partial\mc{L}}{\partial \mc{D}_{a\mu}} i[iC_{iab}]\mc{D}_{b\mu} \quad \because (15.2.2)と同様 \\
=&\sum_{abc}F^2_{ab}\mc{D}_{b}^\mu C_{iac}\mc{D}_{c\mu} \\
=&\sum_{abc}\frac{1}{2}F^2_{ab}C_{iac}\mc{D}_b^\mu \mc{D}_{c\mu}+\sum_{abc}\frac{1}{2}F^2_{ac}C_{iab}\mc{D}_c^\mu \mc{D}_{b\mu} \\
=&\frac{1}{2}\sum_{abc}\left[F^2_{ab}C_{iac}+F^2_{ac}C_{iab} \right]\mc{D}_b^\mu\mc{D}_{c\mu} \\
\Rightarrow \quad & \sum_{a}\left[C_{ica}F^2_{ab}+C_{iba}F^2_{ac} \right] =0
\end{align*}
という条件から,ゲージ・ボゾンの質量(21.4.32)について有用な制限を課すことができる.\par
例として,電弱理論のゲージ群$SU(2)\times U(1)$が自発的に電磁理論のゲージ群$U(1)_{em}$に破れる場合を考える.3つの破れた生成子$x_a$は結合定数を含まない$SU(2)_L$の3つの生成子に,また1つの破れていない生成子$t$は因子$e$を含まない電荷$q$ととることができる.すなわち$x_a$と$t$はレプトン2重項にかかるときは
\begin{align*}
\vec{x}&=\frac{g}{4}(1+\gamma_5)\left\{ \left(
\begin{array}{cc}
0 & 1 \\
1 & 0 
\end{array}
\right),\left(
\begin{array}{cc}
0 & -i \\
i & 0
\end{array}
\right) , \left(
\begin{array}{cc}
1 & 0 \\
0 & -1
\end{array}
\right)\right\}=\frac{1+\gamma_5}{2}\frac{\vec{\sigma}}{2} \\
t&=-\left(
\begin{array}{cc}
0 & 0 \\
0 & 1
\end{array}
\right)
\end{align*}
で表される.するとゲージ群の生成子$\mc{T}_{\alpha}=\sum_A e_{\alpha A}T_A$は,(21.3.4)(21.3.9)より
\begin{align*}
&\frac{1}{e}q=\frac{1}{g}t_3-\frac{1}{g'}y\quad \Rightarrow \quad t=x_3-\frac{1}{g'}\mc{T}_y \\
\Rightarrow \quad &\vec{\mc{T}}=g\vec{x},\quad \mc{T}_y=g'(x_3-t) 
\end{align*}
となる.すなわち,ゲージ生成子$\mc{T}_\alpha=\sum_a e_{\alpha a}x_a+\sum_i e_{\alpha i}t_i$の係数$e_{\alpha A}$は
\begin{align*}
&\vec{\mc{T}}=g\vec{x}\quad \Rightarrow \quad e_{11}=e_{22}=e_{33}=g \\
&\mc{T}_y=g'x_3-g't \quad \Rightarrow \quad e_{y3}=g',e_{yt}=-g'
\end{align*}
となる.また$t$により,$\vec{x}$は$U(1)_{em}$により3元ベクトルとして3軸周りの回転を受ける.すなわち
\begin{align*}
h(\theta)x_a h^{-1}(\theta)=&\exp(i\theta t)x_a \exp(-i\theta t)=x_a+i\theta[t,x_a]+\cdots \\
t\vec{x}=&-\frac{1+\gamma_5}{4}\left\{ \left(
\begin{array}{cc}
0 & 0 \\
1 & 0 
\end{array}
\right),\left(
\begin{array}{cc}
0 & 0 \\
i & 0
\end{array}
\right) , \left(
\begin{array}{cc}
0 & 0 \\
0 & -1
\end{array}
\right)\right\} \\
\vec{x}t=&-\frac{1+\gamma_5}{4}\left\{ \left(
\begin{array}{cc}
0 & 1 \\
0 & 0 
\end{array}
\right),\left(
\begin{array}{cc}
0 & -i \\
0 & 0
\end{array}
\right) , \left(
\begin{array}{cc}
0 & 0 \\
0 & -1
\end{array}
\right)\right\} \\
[t,x_a]=&-\frac{1+\gamma_5}{4}\left\{ \left(
\begin{array}{cc}
0 & -1 \\
1 & 0 
\end{array}
\right),\left(
\begin{array}{cc}
0 & i \\
i & 0
\end{array}
\right) , \left(
\begin{array}{cc}
0 & 0 \\
0 & 0
\end{array}
\right)\right\}=\left\{ix_2,-ix_1,0\right\} \\
=&iC_{tab}x_b \\
\Rightarrow \quad &C_{t12}=1,C_{t21}=-1,C_{t3a}=0
\end{align*}
がわかる.ここで$F^2_{ab}$の条件式
\begin{align*}
\sum_d\left[C_{ibd}F^2_{dc}+C_{icd}F^2_{db}\right]=C_{tb1}F^2_{1c}+C_{tb2}F^2_{2c}+C_{tc1}F^2_{1b}+C_{tc2}F^2_{2b}=0
\end{align*}
を用いる.順番に調べると,$(b,c)=(1,1)$のとき
\begin{align*}
C_{t11}F^2_{11}+C_{t12}F^2_{21}+C_{t11}F^2_{11}+C_{t12}F^2_{21}=F^2_{21}+F^2_{21}=0\quad \Rightarrow \quad F^2_{21}=0
\end{align*}
$(b,c)=(2,2)$のとき
\begin{align*}
C_{t21}F^2_{12}+C_{t22}F^2_{22}+C_{t21}F^2_{12}+C_{t22}F^2_{22}=F^2_{12}+F^2_{12}=0\quad \Rightarrow \quad F^2_{12}=0
\end{align*}
$(b,c)=(3,1)$のとき
\begin{align*}
C_{t31}F^2_{11}+C_{t32}F^2_{21}+C_{t11}F^2_{13}+C_{t12}F^2_{23}=F^2_{23}=0 \quad \Rightarrow\quad F^2_{23}=0
\end{align*}
$(b,c)=(3,2)$のとき
\begin{align*}
C_{t31}F^2_{12}+C_{t32}F^2_{22}+C_{t21}F^2_{13}+C_{t22}F^2_{23}=F^2_{13}=0 \quad\Rightarrow\quad F^2_{13}=0
\end{align*}
同様にして$(b,c)=(1,3)$から$F^2_{32}=0$が,$(b,c)=(2,3)$から$F^2_{31}=0$がわかる.$(b,c)=(1,2)$のとき
\begin{align*}
C_{t11}F^2_{12}+C_{t12}F^2_{22}+C_{t21}F^2_{11}+C_{t22}F^2_{21}=F^2_{22}-F^2_{11}=0 \quad \Rightarrow \quad F^2_{11}=F^2_{22}=F^2_C
\end{align*}
がわかる.$(b,c)=(2,1)$からは同じ結果が出る.$F^2_{33}$についての関係式は出てこないが,$F^2_{ab}$は正定値であるという定義から$F^2_N$という非ゼロで正の値をとることがわかる.
\begin{align*}
F^2_{ab}=\left(
\begin{array}{ccc}
F^2_C & 0 & 0 \\
0 & F^2_C & 0 \\
0 & 0 & F^2_N
\end{array}
\right)
\end{align*}
(21.4.32)より,ゲージ・ボゾンの質量2乗行列はゼロでない成分
\begin{align*}
&\mu^2_{11}=F^2_{11}e^2_{11}=g^2F^2_C,\quad \mu^2_{22}=F^2_{22}e^2_{22}=g^2F^2_C \\
&\mu^2_{33}=F^2_{33}e^2_{33}=g^2F^2_N ,\quad \mu^2_{3y}=F^2_{33}e_{33}e_{y3}=gg'F^2_N=\mu^2_{y3} \\
&\mu^2_{yy}=F^2_{33}e^2_{y3}=g'^2F^2_N
\end{align*}
をもつ.
\begin{align*}
\mu^2_{\alpha\beta}=\left(
\begin{array}{cccc}
g^2F^2_C & 0 & 0 & 0 \\
0 & g^2F^2_C & 0 & 0 \\
0 & 0 & g^2F^2_N & gg'F^2_N \\
0 & 0 & gg'F^2_N & g'^2F^2_N
\end{array}
\right)
\end{align*}
ここで,(21.3.12)(21.3.13)(21.3.16)(21.3.17)(21.3.19)より
\begin{align*}
&W_-^\mu +W_+^\mu =\sqrt{2}A_1^\mu,\quad W_-^\mu - W_+^\mu =\sqrt{2}iA_2^\mu \\
&\frac{-g}{\sqrt{g^2+g'^2}}Z^\mu-\frac{-g'}{\sqrt{g^2+g'^2}}A^\mu=A_3^\mu , \quad \frac{-g}{\sqrt{g^2+g'^2}}Z^\mu+\frac{-g}{\sqrt{g^2+g'^2}}A^\mu=B^\mu
\end{align*}
であるから
\begin{align*}
\mc{A}_\alpha^\mu&=\left(
\begin{array}{cccc}
A_1^\mu \\
A_2^\mu \\
A_3^\mu \\
B^\mu
\end{array}
\right) \\
&=\frac{1}{\sqrt{2}}\left(
\begin{array}{cccc}
1 \\
-i \\
0 \\
0
\end{array}
\right)W_+^\mu+\frac{1}{\sqrt{2}}\left(
\begin{array}{cccc}
1 \\
i \\
0 \\
0
\end{array}
\right)W_-^\mu +\frac{-1}{\sqrt{g^2+g'^2}}\left(
\begin{array}{cccc}
0 \\
0 \\
g \\
g'
\end{array}
\right)Z^\mu+\frac{-1}{\sqrt{g^2+g'^2}}\left(
\begin{array}{cccc}
0 \\
0 \\
-g \\
g'
\end{array}
\right)A^\mu
\end{align*}
とできるが,ゲージ場に結合しているこれらのベクトルは$\mu^2_{\alpha\beta}$の固有ベクトルであることと,正規直交していることに気付けば
\begin{align*}
-\frac{1}{2}\sum_{\alpha\beta}\mu^2_{\alpha\beta}\mc{A}_{\alpha\mu} \mc{A}^\mu_{\beta} =&-\frac{1}{2}\left(A_{1\mu},A_{2\mu},A_{3\mu},B_\mu \right)\left(
\begin{array}{cccc}
g^2F^2_C & 0 & 0 & 0 \\
0 & g^2F^2_C & 0 & 0 \\
0 & 0 & g^2F^2_N & gg'F^2_N \\
0 & 0 & gg'F^2_N & g'^2F^2_N
\end{array}
\right)\left(
\begin{array}{cccc}
A_1^\mu \\
A_2^\mu \\
A_3^\mu \\
B^\mu
\end{array}
\right) \\
=&-g^2F^2_CW_+^\mu W_{-\mu}-\frac{1}{2}(g^2+g'^2)F^2_NZ^\mu Z_\mu+0A^\mu A_\mu
\end{align*}
したがってそれぞれの質量が
\begin{align*}
m^2_W=g^2F^2_C,\quad m^2_Z=(g^2+g'^2)F_N^2 ,\quad m^2_A =0
\end{align*}
となる.\par
さらに話を進めるためには$F^2_C$と$F^2_N$の関係式が必要だ.もしゲージ結合がゼロ$g=g'=0$の極限で,理論が$SU(2)\times U(1)$より\uwave{大きな大域的対称性の群$G$}のもとで不変で,その群は自発的に部分群$H$に破れているとする.(例えば19.4節のようなモデルであれば,$G=SU(2)\times SU(2),H=SU(2)$となる.)この$H$は$\vec{x}$を3元ベクトルとして回転させるとする.すなわち
\begin{align*}
h(\theta)x_ah^{-1}(\theta)&=\exp\left(i\sum_i \theta_i t_i\right)x_a \exp\left(-i\sum_i \theta_i t_i\right) \\
&=x_a+i\sum_i\theta_i[t_i,x_a]+\cdots \\
[t_i,x_a]&=iC_{iab}x_b
\end{align*}
であるとする.(今回は以前とは違い$C_{i3a}=0$ではない.)このとき$F^2_{ab}$の条件式より
\begin{align*}
&\sum_d\left[C_{ibd}F^2_{dc}+C_{icd}F^2_{db}\right]\\
&=C_{ib1}F^2_{1c}+C_{ib2}F^2_{2c}+C_{ib3}F^2_{3c}+C_{ic1}F^2_{1b}+C_{ic2}F^2_{2b}+C_{ic3}F^2_{3b}=0
\end{align*}
であり,$(b,c)=(1,3)$のとき
\begin{align*}
&C_{i11}F^2_{13}+C_{i12}F^2_{23}+C_{i13}F^2_{33}+C_{i31}F^2_{11}+C_{i32}F^2_{21}+C_{i33}F^2_{31}=C_{i13(}F^2_{33}-F^2_{11})=0 \\
\Rightarrow\quad&F^2_{11}=F^2_{33}
\end{align*}
となって,$F_{ab}$は$F^2_{ab}$の正の平方根だと定義していたので$F_N=F_C$となる.\par
このような対称性は全て「保護的な」対称性と呼ばれる.この結果として,$m_Z/m_W$は21.3節で議論した式
\begin{align*}
m_Z/m_W=\sqrt{\frac{g^2+g'^2}{g^2}}=\frac{1}{|\cos\theta|} \quad \because \cos\theta =\frac{-g}{\sqrt{g^2+g'^2}}
\end{align*}
が得られる.\par
例えば,ゲージ結合がゼロならば最も簡単な$SU(2)\times U(1)$電弱理論でのスカラー2重項$\phi$のラグランジアン(21.3.25)は
\begin{align*}
(\mc{L}_{\phi})_{g=g'=0}=-\frac{1}{2}(\partial_\mu \phi)^\dagger (\partial^\mu\phi)-\frac{\mu^2}{2}(\phi^\dagger\phi)-\frac{\lambda}{4}(\phi^\dagger\phi)^2
\end{align*}
となる.ここで
\begin{align*}
\phi^\dagger\phi=(\phi^{+*},\phi^{0*})\left(
\begin{array}{cc}
\phi^+\\
\phi^0
\end{array}
\right)=|\phi^+|^2+|\phi^0|^2=(\mr{Re}\phi^+)^2+(\mr{Im}\phi^+)^2+(\mr{Re}\phi^0)^2+(\mr{Im}\phi^0)^2
\end{align*}
とできるので,$(\partial_\mu \phi)^\dagger (\partial^\mu\phi)$も同様の処置によりラグランジアンは
\begin{align*}
(\mc{L}_{\phi})_{g=g'=0}=-\frac{1}{2}\partial_\mu \phi_n\partial^\mu\phi_n-\frac{\mu^2}{2}\phi_n\phi_n-\frac{\lambda}{4}(\phi_n\phi_n)^2
\end{align*}
と書ける.ここで
\begin{align*}
\phi_1=\mr{Im}\phi^+,\quad \phi_2=\mr{Re}\phi^+,\quad \phi_3=\mr{Im}\phi^0,\quad \phi_4=\mr{Re}\phi^0
\end{align*}
と定義した.このラグランジアン$(\mc{L}_\phi)_{g=g'=0}$は「偶然の」大域的$SO(4)\equiv SU(2)\times SU(2)$対称性の群のもとで自動的に不変だ!((19.5.1)参照)そしてこの群は今回$\mr{Re}\phi^0$の真空期待値によって自発的に破れているのであって,3巻19.2節p226~p227と同様にして近似的な破れていない保護的部分群$SO(3)$となる.すなわち(19.2.11)と同様に
\begin{align*}
\sum^4_{n=1}\phi_n\phi_n=|\mu^2|/\lambda
\end{align*}
が破れを引き起こすことになる.\par
この結果は,2個以上のスカラー2重項があっても成り立つ.なぜなら,この場合にはスカラーのラグランジアンの質量項$-(\mu^2/2)\phi_n\phi_n$と相互作用項$-(\lambda/4)(\phi_n\phi_n)^2$は一般に保護的対称性を保存しないが,上の質量の関係式を導くのに必要だったのはNGボゾンの運動項(21.4.28)だけだったからだ.この運動項は常に完全な$G=SO(4)$不変性があるのだった.大域的変換で一番大きい対称群$G$と,ゲージ結合がゼロでないときに保存されるゲージ群$\mc{G}$と破れていない群$\mc{H}$,そして保護的な対称群$H$の情報があれば,質量の関係式がNGボゾンの運動項から導かれるということだ!

\vskip\baselineskip

保護的対称性は他の種類の理論でも見ることができる.例えば,スカラー場は含まないがテクニカラー相互作用と呼ばれる新しい極めて強いベクトル・ゲージ相互作用をもつ理論を考える.この相互作用は「テクニクォーク」$U_r , D_r$からなる新しい$SU(2)\times U(1)$2重項$(U_r,D_r)$に作用する.ここで$r$はテクニカラー添え字だ.$U_r,D_r$の両方の右手成分と左手成分がテクニカラー・ゲージ群のもとで(カラー$SU(3)$と同様に)全て同じように変換する限り,\uwave{電弱結合がゼロの極限}でテクニクォーク2重項の右手成分と左手成分を独立に$SU(2)$変換する群$SU(2)_L\times SU(2)_R$のもとでラグランジアンは不変となる.\par
19.9節で述べている通り,テクニクォーク2重項の右手成分と左手成分の両方の同時$SU(2)$変換$SU(2)_V$は自発的に破れることはない.一方,ちょうど19.4節でのカラー相互作用により量子色力学の($m_u=m_d=0$での)カイラル$SU(2)_L\times SU(2)_R$対称性が$SU(2)_V$に自発的に破れるように,ここでもテクニカラー相互作用により$SU(2)_L\times SU(2)_R$が$SU(2)_V$に自発的に破れる,と考えるのが自然だ.破れていない$SU(2)_V$対称性のもとで,電弱生成子$\vec{x}$あるいは$\vec{\mc{T}}$は3元ベクトルとして回転し,この場合も$F_C=F_N$の関係式が導かれる.\par
$\Rightarrow$その結果,$W$と$Z$の質量の関係がうまく予言される.\par
今のところ,テクニカラー理論は三度死んだと言われ,失敗した理論となっているが,一部で研究されてはいる.

\newpage

\subsection*{補遺}
21.5節の前に,21.4節で用いられた「一般のユニタリーゲージ」についての命題の証明を追っておこう.\par
証明すべき命題は「NGボゾン場が(21.4.30)
\begin{align*}
\sum_{ab}F^2_{ab}\xi_a e_{\alpha b}=0
\end{align*}
を満たす「ユニタリー・ゲージ」を採用することが常に可能である.」だ.\par
これを示すため,まず全ての群について,指数関数によるパラメータ化を使うと,少なくとも1の有限な近傍では$G$のどの元$g$も
\begin{align*}
g=\exp\left( -i\sum_\alpha \theta_\alpha \mc{T}_\alpha \right)\exp\left( i\sum_\alpha \phi_a x_a \right)\exp\left( i\sum_i \mu_i t_i \right)
\end{align*}
と表せて,このとき$\phi_a$は全ての$\alpha$について線型の拘束条件
\begin{align*}
\sum_{ab}F^2_{ab}\phi_a e_{\alpha b}=0
\end{align*}
を満たすことを先に示そう.$g$が1に限りなく近いときは容易にわかる.そのような任意の$g$は,$\phi^0_a,\mu^0_i$を微小量として
\begin{align*}
g=1+i\sum_a\phi^0_ax_a+i\sum_i\mu^0_it_i
\end{align*}
と書ける.これは同等に
\begin{align*}
g=1+i\sum_a\phi_a x_a +i\sum_i\mu_i t_i -i\sum_\alpha \theta_\alpha \mc{T}_\alpha
\end{align*}
とも書ける.ここで$\theta_\alpha$は任意の微小量で,$\phi_a,\mu_i$は
\begin{align*}
\phi_a(\theta)\equiv \phi^0_a+\sum_\alpha \theta_\alpha e_{\alpha a},\quad \mu_i(\theta)\equiv \mu_i^0+\sum_\alpha \theta_\alpha e_{\alpha i}
\end{align*}
としている.$G$はコンパクト・リー群であるから,どのような$\phi^0_a$でも,正の量
\begin{align*}
\sum_{ab}F^2_{ab}\phi_a(\theta)\phi_b(\theta)
\end{align*}
が最小となる$\theta_\alpha$を選ぶことができる.そのような$\theta$を$\theta_0$とすると,この最小点$\theta_0$では(21.A.8)は停留するので,全ての$\alpha$について$\theta_\alpha$で微分するとゼロとなる.すなわち
\begin{align*}
\frac{\partial}{\partial \theta_\alpha}\left.\sum_{ab}F^2_{ab}\phi_a(\theta)\phi_b(\theta)\right|_{\theta=\theta_0}=\sum_{ab}F^2_{ab}e_{\alpha a}\phi_b(\theta_0)+\sum_{ab}F^2_{ab}\phi_a(\theta_0)e_{\alpha b}=2\sum_{ab}F^2_{ab}\phi_b(\theta_0)e_{\alpha b}=0
\end{align*}
となり,よってこの$\phi_a(\theta_0)$は(21.A.3)を満たすことが示せた.さて,$\phi,\mu,\theta$が微小ならば(21.A.5)と(21.A.2)は同じだとすぐ分かる.よって(21.A.2)の形の$g$の組は($\phi_a$は(21.A.3)を満たすとして),1に無限に近い全ての元$g$を表している.すると連続性より,少なくとも1のある有限な近傍では全ての$g$について上述の議論が適用でき,1の有限な近傍では$g$を(21.A.2)と表せて(21.A.3)を満たすようにできる.\par
次に,特定の群の要素
\begin{align*}
g=\gamma(\xi)=\exp\left(i\sum_a\xi_a x_a \right)
\end{align*}
を考え,これを(21.A.2)の形
\begin{align*}
\gamma(\xi)=&\exp\left(-i\sum_\alpha \theta_\alpha(\xi)\mc{T}_\alpha \right)\exp\left( i\sum_a\phi_a(\theta(\xi))x_a \right)\exp\left(i\sum_i\mu_i(\theta(\xi))t_i\right) \\
=&\exp\left(-i\sum_\alpha \theta_\alpha(\xi)\mc{T}_\alpha \right)\gamma(\phi(\xi))\exp\left(i\sum_i\mu_i(\xi)t_i\right)
\end{align*}
に書いてみる.ここで$\phi_a(\xi)$は(21.A.3)を満たすとする.これを少し書き換えると
\begin{align*}
\exp\left(i\sum_\alpha \theta_\alpha(\xi)\mc{T}_\alpha \right)\gamma(\xi)=\gamma(\phi(\xi))\exp\left(i\sum_i\mu_i(\xi)t_i\right)
\end{align*}
であり,(21.4.11)
\begin{align*}
g\gamma(\xi)=\gamma(\xi')h(g,\xi)
\end{align*}
と対応させれば,$\mc{T}_\alpha$を生成子とする変換(すなわちゲージ変換)$g=\exp\left(i\sum_\alpha \theta_\alpha(\xi)\mc{T}_\alpha \right)$により$\xi_a$が
\begin{align*}
\xi_a \to \xi'_a=\phi_a(\xi)
\end{align*}
に変換されたことを意味する!そしてこの$\xi'_a=\phi_a(\xi)$は(21.A.3)を満たすのだったから,ゲージ変換により$\xi_a$が(21.A.3)を満たすゲージに移り変わった,ということになる!ここでプライムを落とせば,(21.A.3)を書き直して(21.A.1)を満たすようにできたことになる.これが示したかったことだ.


\newpage

\subsection{電弱相互作用と強い相互作用の統一}
3巻15.2節p16で見たように,ゲージ理論はゲージ群$\mc{G}$の単純部分群あるいは$U(1)$部分群毎に独立な結合定数をもつ.\par
$\Rightarrow$したがってゲージ群$SU(2)\times U(1)$に基づく電弱相互作用は二つの独立な結合$g$と$g'$をもつ.自由なパラメータの数を減らすために,$SU(2)\times U(1)$ゲージ群が単純な$SU(3)$ゲージ群に埋め込まれており,その結果$g'=g\sqrt{3}$となる,という提案があった.$\leftarrow$しかしこれは実験的に否定された.\par
さらに量子色力学の出現により,理論家はゲージ群$SU(3)\times SU(2)\times U(1)$と相対するハメになった.($g_s,g,g'$の3つの定数が表れる.)このゲージを単純リー群に埋め込む理論は大統一理論(GUT)と呼ばれる.\par
色々な模型での$SU(3)\times SU(2)\times U(1)$結合定数の比は,個々の模型の詳細には依らない.これらの類の模型の特徴は,観測されているクォークとレプトンの世代が,各模型に含まれる唯一のフェルミオンであるか,少なくとも$SU(3)\times SU(2)\times U(1)$のもとで中性ではない唯一のフェルミオンだ,という点だ.\par
15.2節で示したように,どの単純コンパクト・リー群でも完全反対称な構造定数を持つ生成子$T_\alpha$の慣習的な選び方があり,それは各既約表現または可約表現$D$で規格化条件
\begin{align*}
\mr{Tr}\{T_\alpha T_\beta \}=N_D \delta_{\alpha\beta}
\end{align*}
を満たす,という選び方だ.\par
全ての左手成分のフェルミオンは次のように,$n_g$個の世代を持つと過程する.(右手成分は,以前とは違いディラック共役場で定義している点に注意せよ.)
\begin{align*}
\begin{array}{cccc}
\left(
\begin{array}{cc}
\nu_e \\
e
\end{array}
\right)_L & \left(
\begin{array}{cc}
\nu_\mu \\
\mu
\end{array}
\right)_L & \left(
\begin{array}{cc}
\nu_\tau \\
\tau
\end{array}
\right)_L & \cdots \\
\\
\bar{e}_R & \bar{\mu}_R & \bar{\tau}_R & \cdots \\
\\
\left(
\begin{array}{cc}
u \\
d
\end{array}
\right)_L & \left(
\begin{array}{cc}
c \\
s
\end{array}
\right)_L & \left(
\begin{array}{cc}
t \\
b
\end{array}
\right)_L & \cdots \\
\\
\bar{u}_R & \bar{c}_R & \bar{t}_R &\cdots \\
\\
\bar{d}_R & \bar{s}_R & \bar{b}_R &\cdots \\
\end{array}
\end{align*}
通常の場が$\psi\to\exp(i\theta_\alpha T_\alpha)\psi$と変換されるとき,共役場は
\begin{align*}
\bar{\psi}\to\bar{\psi}\exp(-i\theta_\alpha T^\dagger_\alpha)=\bar{\psi}\exp(i\theta_\alpha (-T^\dagger_\alpha))
\end{align*}
と変換されるから,以下では反フェルミオン場に対する生成子は$-T_\alpha^\dagger$であることに注意すること.$SU(3)$生成子$\frac{1}{2}g_s \lambda_3$
\begin{align*}
\frac{1}{2}g_s\lambda_3=\frac{1}{2}g_s\left(
\begin{array}{ccc}
1 & 0 & 0 \\
0 & -1 & 0 \\
0 & 0 & 0
\end{array}
\right)
\end{align*}
の固有値は,赤色クォーク2重項と緑色反クォーク1重項(ワインバーグ場の量子論本文では昔の流儀カラー添え字として赤・白・青を用いているが,ここでは慣例に従って赤・緑・青を用いる.)
\begin{align*}
\left(
\begin{array}{ccc}
q_L \\
0 \\
0
\end{array}
\right), \left(
\begin{array}{ccc}
0 \\
\bar{q}_R \\
0
\end{array}
\right)
\end{align*}
について$+\frac{1}{2}g_s$だ.これらは一つの世代につき,例えば第一世代ならば
\begin{align*}
\left(
\begin{array}{cc}
\left(\begin{array}{cc} u_L \\ 0 \end{array}\right) \\
0 \\
0 
\end{array}
\right) , \left(
\begin{array}{cc}
\left(\begin{array}{cc} 0 \\ d_L \end{array}\right) \\
0 \\
0 
\end{array}
\right), \left(
\begin{array}{cc}
0 \\
\bar{u}_R \\
0 
\end{array}
\right), \left(
\begin{array}{cc}
0 \\
\bar{d}_R \\
0 
\end{array}
\right)
\end{align*}
で,4つの独立な固有ベクトルがあることがわかり,全世代で$4n_g$個の固有ベクトルがあることがわかる(場はそれぞれ独立であるから,これらのベクトルは互いに独立といえる).また,白色クォーク2重項と赤色反クォーク1重項
\begin{align*}
\left(
\begin{array}{ccc}
0 \\
q_L \\
0
\end{array}
\right), \left(
\begin{array}{ccc}
\bar{q}_R \\
0 \\
0
\end{array}
\right)
\end{align*}
については$-\frac{1}{2}g_s$で,こちらも同様に4つの固有ベクトルであることがわかり,$4n_g$個の固有ベクトルが全世代にある.これら以外の全てのフェルミオンについてはゼロだから,この行列の二乗のトレースは,トレースは固有値の和と同値であることを用いれば
\begin{align*}
\mr{Tr}\left(\frac{1}{2}g_s\lambda_3\right)^2=4n_g\left(+\frac{1}{2}g_s\right)^2+4n_g\left(-\frac{1}{2}g_s\right)^2=2n_g g^2_s
\end{align*}
となる.\par
$SU(2)$生成子
\begin{align*}
t_3=g\left(\frac{1+\gamma_5}{2}\right)\frac{\sigma_3}{2}=\frac{g}{2}\left(\frac{1+\gamma_5}{2}\right)\left(
\begin{array}{cc}
1 & 0 \\
0 & -1
\end{array}
\right)
\end{align*}
の固有値は,電荷$+\frac{2}{3}$の赤色・緑色・青色クォークとニュートリノ
\begin{align*}
\left(
\begin{array}{cc}
q_{1L} \\
0
\end{array}
\right), \left(\begin{array}{cc}
q_{2L} \\
0
\end{array}\right),\left(\begin{array}{cc}
q_{3L} \\
0
\end{array}\right),\left(\begin{array}{cc}
\nu_{\ell L} \\
0
\end{array}\right)
\end{align*}
については$\frac{g}{2}$,電荷$-\frac{1}{3}$の赤色・緑色・青色クォークと荷電レプトン
\begin{align*}
\left(
\begin{array}{cc}
0 \\
q_{1L}
\end{array}
\right), \left(\begin{array}{cc}
0 \\
q_{1L}
\end{array}\right),\left(\begin{array}{cc}
0 \\
q_{3L}
\end{array}\right),\left(\begin{array}{cc}
0 \\
\ell_L
\end{array}\right)
\end{align*}
については$-\frac{g}{2}$,他の全てのフェルミオンについてはゼロだ.(左手反フェルミオンは$SU(2)_L$変換のもとで
\begin{align*}
\bar{\psi}_L\to \bar{\psi}_L\exp\left( -i\frac{g}{2}\left(\frac{1-\gamma_5}{2}\right)\vec{\theta}\cdot \vec{\sigma} \right)
\end{align*}
であるから$SU(2)_R$表現であり$SU(2)_L$表現でなく,$t_3$の固有値にならないので除外する.)したがって二乗のトレースは
\begin{align*}
\mr{Tr}\left(t_3\right)^2=[3n_g+n_g]\times \left(+\frac{1}{2}g\right)^2+[3n_g+n_g]\times \left(-\frac{1}{2}g\right)^2=2n_gg^2
\end{align*}
となる.\par
最後に$U(1)$生成子$y=t_3-q$の固有値は,21.3節にて作成した各弱超電荷リストを参照すれば容易にわかり,ニュートリノと荷電レプトンについて($2n_g$個)は$\frac{1}{2}g'$,荷電反レプトンについて($n_g$個)は$-g'$,クォークについて($SU(3)$3重項かつ$SU(2)$2重項だから$6n_g$個)は$-\frac{1}{6}g'$,電荷$-\frac{2}{3}$の反クォーク($3n_g$個)については$\frac{2}{3}g'$,電荷$\frac{1}{3}$の反クォーク($3n_g$個)については$-\frac{1}{3}g'$となる.したがって二乗のトレースは
\begin{align*}
\mr{Tr}(y)^2=&2n_g\left(\frac{1}{2}g'\right)^2+n_g\left(-g'\right)^2+6n_g\left( -\frac{1}{6}g' \right)^2+3n_g\left(\frac{2}{3}g'\right)^2+3n_g\left(-\frac{1}{3}g'\right)^2 \\
=&\left(\frac{1}{2}+1+\frac{1}{6}+\frac{4}{3}+\frac{1}{3}\right)n_gg'^2=\frac{10}{3}n_gg'^2
\end{align*}
となる.\par
さて,仮定より$SU(3)\times SU(2)\times U(1)$が一つの単純ゲージ群に埋め込まれているのであり,(21.5.1)が満たされているのだったからこれらの二乗のトレースは全て等しくなければならないのだった.したがって
\begin{align*}
g^2_s=g^2=\frac{5}{3}g'^2
\end{align*}
の関係が要請される.

\vskip\baselineskip

さて,この関係式は結合定数の観測値とはひどく異なっている.比$g'^2/g^2=3/5$は電弱混合角が$\sin^2\theta=g'^2/(g^2+g'^2)=\frac{3}{8}$となることを意味するが,一方実験値は$\sin^2\theta=0.231$だ.さらに,強い相互作用の結合$g_s^2$はもちろん$g^2$や$g'^2$よりはるかに大きい.\par
これを解決するには,今回の仮定に用いられているひとつの大きな単純ゲージ群が$SU(3)\times SU(2)\times U(1)$に自発的に破れるときに質量をもつゲージ・ボゾンの典型的な質量$M$に匹敵するエネルギー・スケールで測った結合についてのみこの関係式が成り立つと考えれば良い.我々が結合を測定するエネルギー$E$が$M$よりはるかに小さければ,$\ln(M/E)$に比例する大きな輻射補正がある.この輻射補正を計算する方法を我々は既に知っている!すなわち,くりこみ群の方法だ.\par
19章で述べたように,すぐ近くのエネルギー$\mu$と$\mu-d\mu$で測定した結合間の関係には大きな対数は存在しない.そしてこの関係を$M$から$E$まで積分することで大きな対数に出会うことなく$E\ll M$での結合を計算できるのだった.くりこみ群の方法を使うためには,結合がこのエネルギー領域全体に渡って小さく保たれさえすれば良いのだった.\par
さて,フェルミオンの世代数が$n_g$であるのだったから,(18.7.2)を用いよう.ここで注意するべきは,(17.5.35)(18.7.3)からは(18.7.2)の第二項は$-n_f/6$となるが,この$n_f$はそれぞれの$SU(N)$ゲージ群の\uwave{基本表現}($N$重項)のフェルミオンの数だった.しかしこれはフェルミオンの右手成分と左手成分がゲージ群の同じ表現に属する,という仮定のもとで計算されたものだった.(右手成分も$N$重項で,「右手の数$+$左手の数$=n_f$」が前提だった.)\par
$\Rightarrow$しかし今回,右手成分は$SU(2)1重項$なのであって基本表現ではない.したがって左手成分のフェルミオン
\begin{align*}
\left(
\begin{array}{cc}
\nu_{\ell L} \\
\ell_L
\end{array}
\right),\left(
\begin{array}{cc}
U_L \\
D_L
\end{array}
\right) \quad (と反フェルミオン\overline{\left(
\begin{array}{cc}
\nu_{\ell L} \\
\ell_L
\end{array}
\right)},\overline{\left(
\begin{array}{cc}
U_L \\
D_L
\end{array}
\right) })
\end{align*}
のみを数えることにすれば,今回の場合では$n_f=n_{fL}/2+n_{fR}/2\to n_{fL}/2$となって,$n_f$を$n_f/2$($n_f$は今回左手成分の基本表現の数)として扱わなければならず,(17.5.35)の$C_2$は1/2倍され$n_f/4$となり,(18.7.2)の第二項目は$-n_f/12$となる.\par
$SU(3)$については各世代あたり,2個の左手成分クォーク3重項$U_L,D_L$と2個の左手成分反クォーク3重項$\bar{U}_L\bar{D}_L$の4個があるから,$n_f=4n_g$で,$g_s$に対する(18.7.2)では$C_1=3,C_2/2=n_f/4=n_g$となる.$SU(2)$については3個(カラー添え字の個数)の左手成分クォーク2重項$(U_L,D_L)$と,1個の左手成分のレプトン2重項$(\nu_{\ell L},\ell_L)$の4個があるので,$n_f=4n_g$だ.(例によって左手成分反フェルミオンは$SU(2)_R$表現となってしまうから数えない.)よって$g$に対する(18.7.2)では$C_1=2,C_2/2=n_g$となる.$U(1)$については$C_2$の定義に立ち返り,(17.5.35)(21.5.4)により$g'$に対する(18.7.2)では$C_1=0,C_2/2=\frac{5}{3}n_g$がわかる.以上より(17.5.37)
\begin{align*}
\mu\frac{d}{d\mu}g_i(\mu)=-\frac{g_i^3(\mu)}{4\pi^2}\left(\frac{11}{12}C_1-\frac{1}{3}\left(\frac{C_2}{2}\right)\right) \quad (g_i=g_s,g',g)
\end{align*}
はそれぞれ
\begin{align*}
\mu\frac{d}{d\mu}g_s(\mu)&=-\frac{g_s^3(\mu)}{4\pi^2}\left(\frac{11}{4}-\frac{n_g}{3}\right) \\
\mu\frac{d}{d\mu}g(\mu)&=-\frac{g^3(\mu)}{4\pi^2}\left(\frac{11}{6}-\frac{n_g}{3}\right) \\
\mu\frac{d}{d\mu}g'(\mu)&=-\frac{g'^3(\mu)}{4\pi^2}\left(-\frac{5}{9}n_g\right)
\end{align*}
となる.(別の考え方として,ワインバーグ場の量子論脚注のように,$U(1)$のベータ関数(すなわちQEDのベータ関数)は(18.2.38)あるいは(18.6.12)で与えられるが,これを
\begin{align*}
\beta(e)=\frac{e^3}{12\pi^2}=\frac{e}{24\pi^2}[(e)^2+(-e)^2]
\end{align*}
と分解し,$-e$が電子の荷電に対応し$+e$が反電子の荷電に対応するのだから,今回は$g'/24\pi^2$を$U(1)$荷電の二乗和にかければ良いとわかる.つまり今回の$U(1)$での荷電の二乗和は(21.5.4)なので
\begin{align*}
\beta(g')=\frac{g'}{24\pi^2}\cdot \frac{10}{3}n_g g'^2=-\frac{g'^3}{4\pi^2}\left(-\frac{5}{9}n_g\right)
\end{align*}
となる.)\par
さて,解は(18.7.7)で与えられる.復習を兼ねてここで解いておこう.まず簡単のために$g^3(\mu)$の係数を$\alpha$とおいて
\begin{align*}
\mu\frac{d}{d\mu}g(\mu)=-\frac{g^3(\mu)}{4\pi^2}\left\{\frac{11}{12}C_1-\frac{1}{3}C_2\right\}=\alpha g^3(\mu)
\end{align*}
とする.この微分方程式は変数分離で簡単に解くことができて
\begin{align*}
& \int \frac{1}{g^3(\mu)}dg=\alpha\int \frac{1}{\mu}d\mu \\
\Rightarrow & -\frac{1}{2}g^{-2}(\mu)=\alpha \ln (\mu/\Lambda) \quad (\Lambda は積分定数) \\
&\frac{1}{g^2(\mu)}=-\alpha \ln(\mu^2/\Lambda^2)=\frac{1}{4\pi^2}\left\{\frac{11}{12}C_1-\frac{1}{3}C_2\right\}\ln(\mu^2/\Lambda^2)
\end{align*}
(18.7.7)はここから少し変形すれば得られた.ここでは$C_2\to C_2/2$として定積分を同様に実行すれば
\begin{align*}
\frac{1}{g^2(M)}-\frac{1}{g^2(\mu)}&=\frac{1}{4\pi^2}\left\{\frac{11}{12}C_1-\frac{1}{6}C_2\right\}\ln(M^2/\Lambda^2)-\frac{1}{4\pi^2}\left\{\frac{11}{12}C_1-\frac{1}{6}C_2\right\}\ln(\mu^2/\Lambda^2) \\
\frac{1}{g^2(\mu)}&=\frac{1}{g^2(M)}-\frac{1}{2\pi^2}\left\{\frac{11}{12}C_1-\frac{1}{6}C_2\right\}\ln\left(\frac{M}{\mu}\right)
\end{align*}
したがって(21.5.6)~(21.5.8)の解は
\begin{align*}
\frac{1}{g_s^2(\mu)}&=\frac{1}{g_s^2(M)}-\frac{1}{8\pi^2}\left(11-\frac{4n_g}{3}\right)\ln\left(\frac{M}{\mu}\right) \\
\frac{1}{g^2(\mu)}&=\frac{1}{g^2(M)}-\frac{1}{8\pi^2}\left(\frac{22}{3}-\frac{4n_g}{3}\right)\ln\left(\frac{M}{\mu}\right) \\
\frac{1}{g'^2(\mu)}&=\frac{1}{g'^2(M)}-\frac{1}{8\pi^2}\left(-\frac{20n_g}{9}\right)\ln\left(\frac{M}{\mu}\right)
\end{align*}
となる.また(21.5.5)はここでは
\begin{align*}
g^2_s(M)=g^2(M)=\frac{5}{3}g'^(M)
\end{align*}
の意味だと解釈すべきだ.(21.5.9)から(21.5.10)を引けば,(21.5.12)より
\begin{align*}
\frac{1}{g_s^2(\mu)}-\frac{1}{g^2(\mu)}=-\frac{1}{8\pi^2}\left(11-\frac{22}{3}\right)\ln\left(\frac{M}{\mu}\right)=-\frac{11}{24\pi^2}\ln\left(\frac{M}{\mu}\right)
\end{align*}
となり第一項目と$n_g$に比例する項が消える.同様に(21.5.10)から(21.5.11)の3/5倍を引けば
\begin{align*}
\frac{1}{g^2(\mu)}-\frac{3}{5g'^2(\mu)}&=\frac{1}{g^2(M)}-\frac{1}{\frac{5}{3}g'^2(M)}-\frac{1}{8\pi^2}\left(\frac{22}{3}-\frac{4n_g}{3}\right)\ln\left(\frac{M}{\mu}\right)-\frac{1}{8\pi^2}\left(-\frac{4n_g}{3}\right)\ln\left(\frac{M}{\mu}\right) \\
&=-\frac{11}{12\pi^2}\ln\left(\frac{M}{\mu}\right)
\end{align*}
が得られる.こうして結合(21.5.12)だけでなく,世代数$n_g$も消去できる.(別にラッキーなことではない.生成子の二乗のトレースがそれぞれ等しいと仮定しており,$C_2$が生成子の二乗のトレース由来のものだからこの計算結果はある意味当然だ.)\par
これら二つの式の比をとると,$\sin^2\theta\equiv g'^2/(g^2+g'^2)$についての式が,次のように求まる.
\begin{align*}
&2\left(\frac{1}{g^2_s(\mu)}-\frac{1}{g^2(\mu)}\right)=\frac{1}{g^2(\mu)}-\frac{3}{5g'^2(\mu)} \\
&2\frac{g^2(\mu)-g_s^2(\mu)}{g^2(\mu)g_s^2(\mu)}=\frac{5g'^2(\mu)-3g^2(\mu)}{5g^2(\mu)g'^2(\mu)} \\
&10[g^2(\mu)g'^2(\mu)-g'^2(\mu)g_s^2(\mu)]=5g'^2(\mu)g_s^2(\mu)-3g^2(\mu)g_s^2(\mu) \\
&15g'^2(\mu)g^2_s(\mu)-10g^2(\mu)g'^2(\mu)-3g^2_s(\mu)g^2(\mu)=0 \\
&18g'^2(\mu)g^2_s(\mu)-10g^2(\mu)g'^2(\mu)-3g^2_s(\mu)[g^2(\mu)+g'^2(\mu)]=0 \\
&6g_s^2(\mu)\left[\frac{g'^2(\mu)}{g^2(\mu)+g'^2(\mu)}\right]-\frac{10}{3}\left[\frac{g^2(\mu)g'^2(\mu)}{g^2(\mu)+g'^2(\mu)}\right]=0 \\
&\left[\frac{g'^2(\mu)}{g^2(\mu)+g'^2(\mu)}\right]=\frac{1}{6}+\frac{5}{9g^2_s(\mu)}\left[\frac{g^2(\mu)g'^2(\mu)}{g^2(\mu)+g'^2(\mu)}\right] \\
&\sin^2\theta=\frac{1}{6}+\frac{5e^2(\mu)}{9g^2_s(\mu)}
\end{align*}
少し煩雑に見えるが,やっていることは複雑ではない.ここで$\mu$は$\sin^2\theta$を測定するために使う過程の典型的なエネルギー,すなわち$\mu\approx m_Z$に等しいとすれば
\begin{align*}
\sin^2\theta=\frac{1}{6}+\frac{5e^2(m_Z)}{9g^2_s(m_Z)}
\end{align*}
が得られる.(21.5.6)~(21.5.8)は$m_Z$より大きいエネルギー領域でのみ使っており,自発的対称性の破れ$SU(2)\times U(1)\to U(1)_{em}$が起きるエネルギー領域をまたがずに使えるため,この破れによる影響を強く受けないことがこの手法を用いるメリットだ.また,(21.5.13)と(21.5.14)を組み合わせれば
\begin{align*}
(21.5.13)\times \frac{8}{5}&=\frac{8}{5g^2_s(\mu)}-\frac{8}{5g^2(\mu)}=-\frac{11}{15\pi^2}\ln\left(\frac{M}{\mu}\right) \\
(21.5.13)\times \frac{8}{5}+(21.5.14)&=\frac{8}{5g^2_s(\mu)}-\frac{3}{5}\left[\frac{1}{g^2(\mu)}+\frac{1}{g'^2(\mu)}\right]=-\frac{11}{\pi^2}\left(\frac{1}{12}+\frac{1}{15}\right)\ln\left(\frac{M}{\mu}\right) \\
\frac{8}{5g^2_s(\mu)}-\frac{3}{5e^2(\mu)}&=-\frac{33}{20\pi^2}\ln\left(\frac{M}{\mu}\right) \\
\frac{3}{5e^2(\mu)}\left[\frac{8e^2(\mu)}{3g^2_s(\mu)}-1\right]&=-\frac{33}{20\pi^2}\ln\left(\frac{M}{\mu}\right) \\
\ln\left(\frac{M}{\mu}\right)&=\frac{4\pi^2}{11e^2(\mu)}\left(1-\frac{8e^2(\mu)}{3g^2_s(\mu)}\right)
\end{align*}
再び$\mu\approx m_Z$ととれば
\begin{align*}
\ln\left(\frac{M}{m_Z}\right)&=\frac{4\pi^2}{11e^2(m_Z)}\left(1-\frac{8e^2(m_Z)}{3g^2_s(m_Z)}\right)
\end{align*}
が得られる.\par
18.2節で見たように,$\mu\approx m_Z$での$e(\mu)$の値は$e^2(m_Z)/4\pi^2=(128.87\pm 0.12)^{-1}$だ.これは真空偏極を使った従来の定義による電荷だ.\par
$\Rightarrow g_s,g',g$との比較に使うには,18.6節で述べられた修正された最小引き算を用いるのが良い,らしい.\\
その値は,$e^2(m_Z)/4\pi^2=(127.9\pm 0.1)^{-1}$となる,らしい.低エネルギーのデータから外挿した$g_s$の値は$g^2_s(m_Z)/4\pi=0.118\pm 0.006$で,一方$Z^0$のハドロンへの崩壊率から直接測定された値は$g^2_s/4\pi=0.120\pm 0.0025$だ.\par
$g^2_s(m_Z)/4\pi=0.118$と$e^2(m_Z)/4\pi=1/128$とすると,(21.5.15)(21.5.16)より
\begin{align*}
\sin^2\theta=\frac{1}{6}+\frac{5}{9}\times \frac{1}{0.118}\times \frac{1}{128}=\frac{3457}{16992}\approx 0.203
\end{align*}
と,($m_Z=91\mr{GeV}$として)
\begin{align*}
&\ln(M/m_Z)=\frac{128\pi}{11}\left(1-\frac{8}{3}\times \frac{1}{0.118}\times \frac{1}{128}\right) \\
&M=91\times \exp\left[\frac{128\pi}{11}\left(1-\frac{8}{3}\times \frac{1}{0.118}\times \frac{1}{128}\right)\right]\mr{GeV}\approx 1.1\times 10^{15}\mr{GeV}
\end{align*}
が得られる.

\vskip\baselineskip

21.3節で述べた通り,通常のエネルギーでの物理を記述する有効ラグランジアンで小さく抑えられた非くりこみ可能な項がバリオン数とレプトン数を保存すると期待する理由はない.\par
$\Rightarrow$したがって,$SU(3)\times SU(2)\times U(1)$を保存する4フェルミオン(クォーク3個とレプトン1個)相互作用の存在が期待される.その係数は次元解析から$M^{-2}$の程度だ.これに基づいて陽子の寿命が計算され,$10^{32}$年程度と見積もられた.\par
$\Rightarrow$そのような模型は,$M$程度の質量をもつゲージ・ボゾンの交換により生成される.\\
より一般には,バリオン数とレプトン数を保存しない過程が自然に小さく抑えられる理由が標準理論によって一度説明されてしまえば,バリオン数とレプトン数が厳密に保存するということを信じる理由はなくなった.

\vskip\baselineskip

(21.5.15)の予言が$\sin^2\theta$の測定値0.23に非常に近くなることを見た.しかし測定および計算の制度が良くなると両者が正確に一致しないことが明らかになった.





\newpage

\part{アノマリー}
\setcounter{section}{22}
\setcounter{subsection}{0}
\subsection{$\pi^0$崩壊の問題}
重い束縛された粒子を全て積分した後には,$\pi^0\to 2\gamma$の相互作用は以下のラグランジアンで与えられると考えられる.
\begin{align*}
\mc{L}_{\pi\gamma\gamma}=g\pi^0 \epsilon^{\mu\nu\rho\lambda}F_{\mu\nu}F_{\rho\lambda}
\end{align*}
19章で述べた通り,$\pi^0$は微分を必ず一つ持って表れるため,ヤンミルズ項$F_{\mu\nu}F^{\mu\nu}$との積では微分を3つ以上含んでしまう.したがって微分を3つ以上含まない有効ラグランジアンはこの形に限定され,部分積分によって$\pi^0$場の微分を外したものだと理解できる.西島和彦著「Fields and particles」あるいは「場の理論」の計算によると,これより
\begin{align*}
\Gamma(\pi^0\to2\gamma)=\frac{m^3_\pi g^2}{\pi}
\end{align*}
となるらしい.この反応は($\epsilon^{\mu\nu\rho\lambda}$によりカイラルを保存しないので)1ループを必ず含む.したがって単純に考えれば$g$は
\begin{align*}
g\approx \frac{\alpha}{2\pi F_\pi}=\frac{e^2}{8\pi^2F_\pi}
\end{align*}
となる.\par
カイラル不変性を課すと,崩壊率は(22.17)のように小さくなるが,実際の崩壊率は(22.1.3)から得られる値に近いと分かっている.すなわち,カイラル対称性を破る異常な何かがあると結論付けられる.これがアノマリーである.

\newpage

\subsection{測度の変換:可換アノマリー}
ここではミンコフスキー空間における経路積分を用いる.まず質量ゼロのスピン1/2複素フェルミオン場の列ベクトル$\psi_n(x)$についての任意の局所行列変換$\psi(x)\to U(x)\psi(x)$のもとでの測度のアノマリーを計算する.これらはフェルミオン変数で$\psi$と$\bar{\psi}$は独立な変数であるから,これらの変換は(連続的な添え字の縮約は積分であることを思い出して)
\begin{align*}
\psi_n(x)&\to U(x)_{nm}\psi_m(x)=\int d^4y \, U(x)_{nm}\delta^4(x-y)\cdot \psi_m(y)\equiv \mc{U}_{xn,ym}\psi_m(y) \\
\bar{\psi}_n(x)&\to\left[U(x)\psi(x)\right]^\dagger \gamma_4 \\
&=\psi^\dagger(x)\gamma_4 \gamma_4U(x)^\dagger \gamma_4 =\int d^4 \, y \bar{\psi}(y)\cdot \gamma_4 U(y)^\dagger \gamma_4 \delta^4(y-x) \\
&\equiv \bar{\psi}_m(y)\bar{\mc{U}}_{ym,xn}
\end{align*}
となって,測度は(9.5.38)の通り変換して
\begin{align*}
[d\psi][d\bar{\psi}]\to (\mr{Det}\mc{U}\mr{Det}\bar{\mc{U}})^{-1}[d\psi][d\bar{\psi}]
\end{align*}
となる.ここで$\gamma_4=i\gamma_0$で$\beta$と同じ行列だ.添え字$n,m$はフレーバーと,ディラックのスピンの添え字についてとる.(つまり,ガンマ行列の添え字も含めていることに注意.)\par
$\mr{Det}AB=\mr{Det}A\mr{Det}B$だから,$\mr{Det}\gamma_4=1$より
\begin{align*}
\mr{Det}\bar{\mc{U}}=\mr{Det}\gamma_4 \mr{Det}\mc{U}^\dagger \mr{Det}\gamma_4 =\mr{Det}\mc{U}^\dagger
\end{align*}
となるから,$\gamma_4$は行列式に影響しない.それにもかかわらずこれを計算に含めるのは,プロパゲータや行列式においてフェルミオンのモードについての和を正則化する必要があり,正則化された行列式には$\gamma_4$が実際に影響することがあるからだ.

\vskip\baselineskip

まず,$U(x)$がユニタリーな非カイラル的変換
\begin{align*}
U(x)=\exp[i\alpha(x)t]
\end{align*}
である場合を考える.ここで$t$は($\gamma_5$を含まないが必ずしもトレースレスでなくてもよい)エルミート行列だ.また$\alpha(x)$は任意の実関数だ.この場合は$U(x)$は$\gamma_4$と可換で$\gamma_4 U(x)^\dagger \gamma_4 =U(x)^\dagger$となり,かつ$U(x)^\dagger=U(x)^{-1}$だから
\begin{align*}
\bar{\mc{U}}\mc{U}&=\bar{\mc{U}}_{xn,ym}\mc{U}_{ym,z\ell}=\int d^4y\, U(x)^\dagger_{nm}\delta^4(x-y)U(y)_{m\ell}\delta^4(y-z) \\
&=\delta_{n\ell}\delta^4(x-z)
\end{align*}
となる.したがってこの変換では$\mr{Det}\mc{U}\mr{Det}\bar{\mc{U}}=1$なので,測度は不変だ.\par
$\Rightarrow$特に,ゲージ群自身の対称性において,$t$は非カイラル生成子の一つ$t_\alpha$になっていて,この対称性がアノマリーによって破られることはない.\par
次に,以下の局所カイラル変換を考える.
\begin{align*}
U(x)=\exp[i\gamma_5\alpha(x)t]
\end{align*}
この場合,$\mc{U}$は
\begin{align*}
\gamma_4 U(x)^\dagger \gamma_4=&\gamma_4 \exp[-i\gamma_5^\dagger \alpha(x)t]\gamma_4 \\
=&\exp[i\gamma_5 \alpha(x)t]=U(x)\quad \because (5.4.33), \gamma_4\gamma_5^\dagger \gamma_4=-\gamma_5
\end{align*}
より擬エルミート
\begin{align*}
\bar{\mc{U}}=\mc{U}
\end{align*}
がわかる.よって測度はカイラル変換のもとで不変ではなく,
\begin{align*}
[d\psi][d\bar{\psi}]\to (\mr{Det}\mc{U})^{-2}[d\psi][d\bar{\psi}]
\end{align*}
となる.以下ではこれを解析していく.\par
さて,微小局所カイラル変換の場合に特定する.(22.2.6)において$\alpha(x)$が微小とすると
\begin{align*}
U(x)&=1+i\gamma_5\alpha(x)t \\
\Rightarrow\quad \mc{U}_{xn,ym}&=\delta_{nm}\delta^4(x-y)+i\alpha(x)[\gamma_5t]_{nm}\delta^4(x-y) \\
[\mc{U}-1]_{xn,ym}&=i\alpha(x)[\gamma_5t]_{nm}\delta^4(x-y)
\end{align*}
となる.ここで恒等式$\mr{Det}M=\exp\mr{Tr}\ln M$を用いる.(これを証明するには,行列$A$のジョルダン標準形$J$とすると$A=PJP^{-1}$であるから,行列式が固有値の積と同値であることを用いれば,$A$の固有値を$\lambda_1,\lambda_2,\cdots \lambda_n$として
\begin{align*}
\mr{Det}e^A&=\mr{Det} e^{(PJ P^{-1})}=\mr{Det}(Pe^J P^{-1})=\mr{Det}e^J \\
&=e^{\lambda_1}e^{\lambda_2}\cdots e^{\lambda_n}=e^{\lambda_1+\lambda_2+\cdots \lambda_n} \\
&=\exp \mr{Tr} A
\end{align*}
となって,$\exp A=M$とおけば恒等式が示せる.)$\alpha(x)$が微小であるから,$x\to0$で$\ln(1+x)\to x$であることを用いることができて
\begin{align*}
[d\psi][d\bar{\psi}]\to& (\mr{Det}\mc{U})^{-2}[d\psi][d\bar{\psi}] \\
&=\exp \left\{-2 \mr{Tr} \ln [1+(\mc{U}-1)]\right\} [d\psi][d\bar{\psi}] \\
&=\exp \left\{ -2\mr{Tr}\left[i\alpha(x)(\gamma_5t)\delta^4(x-y)\right] \right\} \\
&=\exp \left\{ -2i \int d^4x \, \alpha(x) \mr{Tr}[\gamma_5 t] \delta^4(x-x) \right\} \\
&=\exp \left\{ i\int d^4x \, \alpha(x)\mc{A}(x) \right\}
\end{align*}
となる.ここで$\mc{A}$はアノマリー関数
\begin{align*}
\mc{A}(x)=-2\mr{Tr}[\gamma_5 t]\delta^4(x-x)
\end{align*}
だ.ここで$\mr{Tr}$はディラック添え字と種類添え字の両方についてトレースをとることを意味する.測度$[d\psi][d\bar{\psi}]$は経路積分において因子$\exp\left\{i\int d^4 x \mc{L}(x)\right\}$の重み付きで表れるのだったから,測度の変換則(22.2.10)の因子$\exp \left\{ i\int d^4x \, \alpha(x)\mc{A}(x) \right\}$は,あたかもラグランジアン密度$\mc{L}(x)$がこれらの変換で不変でなく,その代わりに$\mc{L}(x)\to \mc{L}(x)+\alpha (x)\mc{A}(x)$となるのと同じ影響をする.したがって,フェルミオンを積分した後の有効ラグランジアンを扱うときには,アノマリーも考慮に入れて,元の不変な項に加えて不変でない項を
\begin{align*}
\mc{L}_{eff}(x)\to \mc{L}_{eff}(x)+\alpha (x)\mc{A}(x)
\end{align*}
となるように入れておかなければならない.あとは,このアノマリー関数の計算だ.

\vskip\baselineskip

アノマリー関数を観察してみると,デルタ関数$\delta^4(x-x)$は無限大だがトレース$\mr{Tr}[\gamma_5 t]$は
\begin{align*}
\gamma_5=\left(
\begin{array}{cc}
1 & 0 \\
0 & -1
\end{array}
\right)
\end{align*}
によりゼロとなる.($t$は$\gamma_5$を含まないので$\gamma_5$のみでディラック添え字のトレースがとれる.)このままでは扱えないため,これらの量に意味を持たせるために正則化因子を導入する.これは,デルタ関数の引数がゼロとなる前に微分演算子$f(-\Slash{D}^2_x/M^2)$をデルタ関数に働かせることにょりゲージ不変に実行できる.(熱核による正則化)
\begin{align*}
\mc{A}(x)=-2[\mr{Tr}\left\{ \gamma_5 t f(-\Slash{D}^2_x/M^2) \right\} \delta^4(x-y)]_{y\to x}
\end{align*}
ここで$D_x$はゲージ場$A^\mu_\alpha(x)$があるもとでのディラック演算子だ.
\begin{align*}
(D_x)_\mu=\frac{\partial}{\partial x^\mu}-it_\alpha A_{\alpha \mu}(x)
\end{align*}
また$M$はある大きな質量で,後に無限大に極限をとる.また,$f(s)$は滑らかな関数で,$s$が$0から\infty$に行くにつれて$f(s)$は$1$から$0$に滑らかに落ちる,という条件さえ満たしていれば良い.(藤川の方法では$\exp(-[\Slash{D}_x/M)]^2$ととっていた.)
\begin{align*}
&f(0)=1,\quad f(\infty)=0 \\
&sf'(s)=0 \quad (s=0,s=\infty において)
\end{align*}
正則化関$f$はゲージ不変性を保つために$\Slash{\partial}$の関数にはとらず,また行列式だけでなくフェルミオン・プロパゲータ$\Slash{D}^{-1}$(このプロパゲータは通常のものではなく,完全なプロパゲータ(14.2.1)であることに注意.)も正則化する必要がある.したがって$D^\mu D_\mu$の関数にもとらないことに注意.(22.2.22)の通り,$\Slash{D}^2_x=D^2_x$ではないことにも注意だ.\par
(22.2.13)の表式を計算するために,デルタ関数のフーリエ積分表示を用いて,アノマリー関数を
\begin{align*}
\mc{A}(x)=&-2\int \frac{d^4 k}{(2\pi)^4}\left[\mr{Tr}\left\{\gamma_5 tf(-\Slash{D}^2_x/M^2) \right\}e^{ik(x-y)}\right]_{y\to x} \\
=&-2\int \frac{d^4 k}{(2\pi)^4}\mr{Tr}\left\{\gamma_5 tf(-[i\Slash{k}+\Slash{D}_x]^2/M^2) \right\}
\end{align*}
と書く.ここで
\begin{align*}
\Slash{D}_x^2 e^{ik(x-y)}=&\left\{ \Slash{\partial}_x-it_\alpha \Slash{A}_{\alpha}(x) \right\}\left\{ \Slash{\partial}_x-it_\alpha \Slash{A}_{\alpha}(x) \right\}e^{ik(x-y)} \\
=&\left\{ \Slash{\partial}_x-it_\alpha \Slash{A}_{\alpha}(x) \right\}\left\{i\Slash{k}-it_\alpha \Slash{A}_\alpha(x)\right\}e^{ik(x-y)} \\
=&\left\{ \Slash{\partial}_x-it_\alpha \Slash{A}_{\alpha}(x) \right\}i\Slash{k}e^{ik(x-y)}+\left\{ \Slash{\partial}_x-it_\alpha \Slash{A}_{\alpha}(x) \right\}(-it_\alpha \Slash{A}_\alpha(x))e^{ik(x-y)} \\
=&\left\{ i\Slash{k}-it_\alpha \Slash{A}_{\alpha}(x) \right\}i\Slash{k}e^{ik(x-y)} \\
&+i\Slash{k}(-it_\alpha \Slash{A}_\alpha(x))e^{ik(x-y)}+\left[ \Slash{\partial}(-it_\alpha \Slash{A}_\alpha(x))\right]e^{ik(x-y)}+(-it_\alpha \Slash{A}_\alpha(x))^2e^{ik(x-y)} \\
\overset{y\to x}{\longrightarrow} \quad &(i\Slash{k})^2+(-it_\alpha \Slash{A}_\alpha(x))i\Slash{k}+i\Slash{k}(-it_\alpha \Slash{A}_\alpha(x))+[\Slash{\partial}(-it_\alpha \Slash{A}_\alpha(x))]+(-it_\alpha \Slash{A}_\alpha(x))^2 \\
=&(i\Slash{k})^2+(\Slash{\partial}-it_\alpha \Slash{A}_\alpha(x))i\Slash{k}+i\Slash{k}(\Slash{\partial}-it_\alpha \Slash{A}_\alpha(x))+(\Slash{\partial}-it_\alpha \Slash{A}_\alpha(x))^2 \\
=& [i\Slash{k}+\Slash{D}_x]^2
\end{align*}
となることを用いた.ここでは微分演算子が作用する関数を省略しているのではなく$1$に作用していることに注意.極限をとった後の等号はどれも複雑だが,逆算してみれば成り立つことが確認できる.運動量$k^\mu$をスケール変換$k^\mu \to Mk^\mu$すると
\begin{align*}
\mc{A}(x)=&-2\int \frac{d(Mk)^4}{(2\pi)^4}\mr{Tr}\left\{\gamma_5 tf(-[iM\Slash{k}+\Slash{D}_x]^2/M^2) \right\} \\
=&-2M^4 \int \frac{d^4k}{(2\pi)^4}\mr{Tr}\left\{\gamma_5 tf(-[i\Slash{k}+\Slash{D}_x/M]^2) \right\}
\end{align*}
となる.切断関数の変数は
\begin{align*}
-\left[ i\Slash{k} +\frac{\Slash{D}}{M} \right]^2=&-(i\Slash{k})^2-i\Slash{k}=-(i\Slash{k})^2-i\Slash{k}\frac{\Slash{D}_x}{M}-\frac{\Slash{D}_x}{M}i\Slash{k}-\left(\frac{\Slash{D}_x}{M}\right)^2 \\
=&\frac{k^\mu k^\nu}{2} \left\{ \gamma_\mu ,\gamma_\nu \right\}-\frac{ik^\mu D_x^\nu}{M}\left\{ \gamma_\mu ,\gamma_\nu \right\}+\left(\frac{\Slash{D}_x}{M}\right)^2 \\
=&k^2-\frac{2ik\cdot D_x}{M}-\left(\frac{\Slash{D}_x}{M}\right)^2 \quad \because \left\{ \gamma_\mu ,\gamma_\nu \right\}=2\eta_{\mu\nu}
\end{align*}
と書ける.$M\to \infty$の極限では,(22.2.17)は$f(-[i\Slash{k}+\Slash{D}_x/M]^2)$の展開のなかで$1/M$の因子を5個以上持たない項と,(8.A.10)(8.A.11)(8.A.12)よりディラックのガンマ行列を最低4個持つ項のみから寄与を受ける.これにより,$\Slash{D}_x^2$について2次の項のみが残る.テイラー展開すれば
\begin{align*}
f\left(k^2-\frac{2ik\cdot D_x}{M}-\left(\frac{\Slash{D}_x}{M}\right)^2\right)=&f(k^2)+f'(k^2)\left(-\frac{2ik\cdot D_x}{M}-\left(\frac{\Slash{D}_x}{M}\right)^2\right) \\
&+f''(k^2)\left(-\frac{2ik\cdot D_x}{M}-\left(\frac{\Slash{D}_x}{M}\right)^2\right)^2+\cdots \\
=&f''(k^2)\frac{\Slash{D}^2_x}{M^4} +\cdots
\end{align*}
となるから
\begin{align*}
\mc{A}(x)=-\int \frac{d^4k}{(2\pi)^4}f''(k^2)\mr{Tr}\left\{\gamma_5 t \Slash{D}^4_x\right\}
\end{align*}
が残る.これは正則化質量$M$に独立だ.\par
$k$積分を計算するには,(11.2.10)で用いた方法を再び用いれば良い.すなわち,$k^0$積分経路をファインマンダイアグラムを計算する際と同様に回転させ,$k^0$を$ik^4$に置き換え,$k^4$は$-\infty$から$+\infty$まで動くとする.そして(11.2.10)の方法を用いて,4次元球面の面積は$2\pi^2$であるから$\kappa=\sqrt{k^2}$とすれば4次元極座標表示にできる.
\begin{align*}
\int d^4k\, f''(k^2)=i\int (d^4k)_E \, f''(k^2)=i\int^\infty_0 2\pi^2 \kappa^3 d\kappa f''(\kappa^2)
\end{align*}
(22.2.16)(22.2.15)を用いて繰り返し部分積分すると
\begin{align*}
\int d^4 k \, f''(k^2)=&i\int^\infty_0 2\pi^2 \kappa^3 d\kappa f''(\kappa^2) \\
=&i\pi^2 \int^\infty_0 ds\,s f''(s) \quad \because \kappa^2=s ,\,ds=2\kappa d\kappa \\
=&\left[i\pi^2 sf'(s)\right]^\infty_0-i\pi^2\int^\infty_0 dsf'(s)=-i\pi^2\int^\infty_0 dsf'(s) \quad \because (22.2.16) \\
=&-i\pi^2[f(\infty)-f(0)]=i\pi^2 \quad\because (22.2.15)
\end{align*}
となる.トレースを計算するには
\begin{align*}
\Slash{D}^2_x=&\frac{1}{4}\left\{ (D_x)^\mu,(D_x)^\nu \right\}\left\{ \gamma_\mu,\gamma_\nu \right\}+\frac{1}{4}[(D_x)^\mu,(D_x)^\nu][\gamma_\mu,\gamma_\nu] \\
=&\frac{1}{2}\left\{(D_x)^\mu,(D_x)^\nu\right\}\eta_{\mu\nu}-\frac{1}{4}it_\alpha F^{\mu\nu}_\alpha [\gamma_\mu,\gamma_\nu] \quad \because (15.1.12)\\
=&D_x^2-\frac{1}{4}it_\alpha F^{\mu\nu}_\alpha [\gamma_\mu,\gamma_\nu]
\end{align*}
と書けることを用いる.最初の等号に関しては
\begin{align*}
&\frac{1}{4}\left\{ (D_x)^\mu,(D_x)^\nu \right\}\left\{ \gamma_\mu,\gamma_\nu \right\}+\frac{1}{4}[(D_x)^\mu,(D_x)^\nu][\gamma_\mu,\gamma_\nu] \\
=&\frac{1}{4}((D_x)^\mu (D_x)^\nu+(D_x)^\nu(D_x)^\mu)(\gamma_\mu \gamma_\nu+\gamma_\nu \gamma_\mu) \\
&+\frac{1}{4}((D_x)^\mu (D_x)^\nu-(D_x)^\nu(D_x)^\mu)(\gamma_\mu \gamma_\nu-\gamma_\nu \gamma_\mu) \\
=&\frac{1}{4}(D_x)^\mu(D_x)^\nu\gamma_\mu\gamma_\nu +\frac{1}{4}(D_x)^\mu(D_x)^\nu\gamma_\nu\gamma_\mu +\frac{1}{4}(D_x)^\nu(D_x)^\mu\gamma_\mu\gamma_\nu+\frac{1}{4}(D_x)^\nu(D_x)^\mu\gamma_\nu\gamma_\mu \\
&+\frac{1}{4}(D_x)^\mu(D_x)^\nu\gamma_\mu\gamma_\nu -\frac{1}{4}(D_x)^\mu(D_x)^\nu\gamma_\nu\gamma_\mu -\frac{1}{4}(D_x)^\nu(D_x)^\mu\gamma_\mu\gamma_\nu+\frac{1}{4}(D_x)^\nu(D_x)^\mu\gamma_\nu\gamma_\mu \\
=& \frac{1}{2}(D_x)^\mu(D_x)^\nu\gamma_\mu\gamma_\nu+\frac{1}{2}(D_x)^\nu(D_x)^\mu\gamma_\nu\gamma_\mu=(D_x)^\mu(D_x)^\nu\gamma_\mu\gamma_\nu =\Slash{D}^2_x
\end{align*}
として確認できる.$(\Slash{D}^2_x)^2$でガンマ行列を4つ含むものは第二項の二次のみであり,これからは(8.A.12)より
\begin{align*}
\mr{tr_D}\left\{ \gamma_5[\gamma_\mu,\gamma_\nu][\gamma_\rho,\gamma_\sigma] \right\}=&\mr{tr_D}\left\{ \gamma_5(\gamma_\mu\gamma_\nu-\gamma_\nu\gamma_\mu)(\gamma_\rho\gamma_\sigma-\gamma_\sigma\gamma_\rho) \right\}  \\
=&\mr{tr_D}\left\{\gamma_5 \gamma_\mu\gamma_\nu \gamma_\rho \gamma_\sigma\right\}-\mr{tr_D}\left\{\gamma_5 \gamma_\mu\gamma_\nu\gamma_\sigma \gamma_\rho\right\}\\
&-\mr{tr_D}\left\{\gamma_5 \gamma_\nu\gamma_\mu \gamma_\rho \gamma_\sigma\right\}+\mr{tr_D}\left\{\gamma_5 \gamma_\mu\gamma_\nu \gamma_\sigma \gamma_\rho\right\} \\
=&4i\epsilon_{\mu\nu\rho\sigma}-4i\epsilon_{\mu\nu\sigma\rho}-4i\epsilon_{\nu\mu\rho\sigma}+4i\epsilon_{\nu\mu\sigma\rho}=16i\epsilon_{\mu\nu\rho\sigma}
\end{align*}
が生じる.ここで$\mr{tr_D}$はディラック添え字についてのみトレースをとることを意味する.したがってアノマリー関数は(22.2.19)に(22.2.21)~(22.2.23)を用いると
\begin{align*}
\mc{A}(x)=&-\frac{i\pi^2}{(2\pi)^4}\left(-\frac{i}{4}\right)^216i\epsilon_{\mu\nu\rho\sigma}F^{\mu\nu}_\alpha(x) F^{\rho\sigma}_\beta(x) \mr{tr}\left\{t_\alpha t_\beta t \right\} \\
=&-\frac{1}{16\pi^2}\epsilon_{\mu\nu\rho\sigma}F^{\mu\nu}_\alpha(x) F^{\rho\sigma}_\beta(x) \mr{tr}\left\{t_\alpha t_\beta t \right\}
\end{align*}
が得られる!ここで$\mr{tr}$は各種フェルミオンの種類添え字についてのみトレースをとることを意味する.$t$が単位行列だという特別な場合には,量(22.2.24)はチャーン・ポントリャーギン密度として知られている.\par
この結果はアノマリーをもつ対称性にともなうカレントを用いて表すことができる.簡単のため,定数の微小パラメータ$\alpha$の対称性変換$\psi(x)\to \psi(x)+it\gamma_5 \alpha \psi(x)$のもとで作用自身が不変だとする.その場合,(7.3.14)で示したように時空に\uwave{依存する}パラメータ$\alpha(x)$で変換を行うと,作用には$\delta I=\int d^4x J^\mu_5 (x)\partial_\mu \alpha(x)$だけの変化が生じる.(この項は作用の$\int d^4x J^\mu_\alpha (x)A_{\alpha\mu}(x)$から生じる.$\delta A_{\alpha\mu}=\partial_\mu \epsilon_\alpha+\cdots $だったのを思い出す.)ここで$J^\mu_5(x)$は,場についての任意の変化のもとで作用が停留するように場の演算子が動的な方程式を満たす時には保存されるカレントだ.すなわち,古典的作用の変化が停留する$\delta I=0$と仮定すると部分積分よりカレントは
\begin{align*}
\partial_\mu J^\mu_5(x)=0
\end{align*}
と保存されることがわかる.しかし,ここで$\delta \psi(x)=it\gamma_5 \alpha (x)\psi(x)$という変数変換を行うと(22.2.10)より
\begin{align*}
\delta \int [d\psi][d\bar{\psi}]e^{iI}=&\int [d\psi][d\bar{\psi}]\exp\left(i\int d^4x\,\left\{\alpha(x)\mc{A}(x)+J^\mu_5(x)\partial_\mu\alpha(x)\right\}+iI\right)-\int [d\psi][d\bar{\psi}]e^{iI} \\
=&i\int d^4x \int [d\psi][d\bar{\psi}][\mc{A}(x)\alpha(x)+J^\mu_5(x)\partial_\mu\alpha(x)]e^{iI}
\end{align*}
となる.(通常通りテイラー展開すれば導ける.)これは単に変数変換であるから,任意の$\alpha(x)$について経路積分に影響はない.したがってこの変化分が全微分であることが要請され,任意のゲージ場について($\mc{A}$は$\psi,\bar{\psi}$に依存しないので外に出せて)
\begin{align*}
&\int [d\psi][d\bar{\psi}]e^{iI}\partial_\mu J^\mu_5 =\mc{A}\int [d\psi][d\bar{\psi}]e^{iI}  \\
&\braket{\partial_\mu J^\mu_5}_A=\frac{\int[d\psi][d\bar{\psi}]e^{iI}\partial_\mu J^\mu_5}{\int[d\psi][d\bar{\psi}]e^{iI}}=\mc{A}=-\frac{1}{16\pi^2}\epsilon_{\mu\nu\rho\sigma}F^{\mu\nu}_\alpha F^{\rho\sigma}_\beta \mr{tr}\left\{t_\alpha t_\beta t \right\}
\end{align*}
ここで任意の演算子$\mc{O}$について,$\braket{\mc{O}}_A$は固定された背景場$A^\mu_\alpha(x)$のもとでの$\mc{O}$の量子平均
\begin{align*}
\braket{\mc{O}}_A=\frac{\int[d\psi][d\bar{\psi}]e^{iI}\mc{O}}{\int[d\psi][d\bar{\psi}]e^{iI}}
\end{align*}
だ.\par
表式(22.2.26)を保存条件に書き換えることもできる.ここで$\mr{tr}\{t_\alpha t_\beta t\}$が$\delta_{\alpha\beta}$に比例する特別の場合を考える.
\begin{align*}
\mr{tr}\{t_\alpha t_\beta t\}=N\delta_{\alpha\beta}
\end{align*}
つまりこの場合においては
\begin{align*}
\braket{\partial_\mu J^\mu_5}_A=-\frac{N}{16\pi^2}\epsilon_{\mu\nu\rho\sigma}F^{\mu\nu}_\alpha F^{\rho\sigma}_\alpha
\end{align*}
となる.ここでチャーン・サイモンズ類として知られるカレントを定義する.
\begin{align*}
G^\mu&\equiv 2\epsilon^{\mu\nu\lambda\rho}\left[A_{\gamma\nu}\partial_\lambda A_{\gamma\rho}+\frac{1}{3}C_{\alpha\beta\gamma}A_{\alpha\nu}A_{\beta\lambda}A_{\gamma\rho}\right] \\
&=\epsilon^{\mu\nu\lambda\rho}\left[A_{\gamma\nu}F_{\gamma\lambda\rho}-\frac{1}{3}C_{\alpha\beta\gamma}A_{\alpha\nu}A_{\beta\lambda}A_{\gamma\rho}\right]
\end{align*}
(二番目の等号は
\begin{align*}
G^\mu=&\epsilon^{\mu\nu\lambda\rho}\left[A_{\gamma\nu}F_{\gamma\lambda\rho}-\frac{1}{3}C_{\alpha\beta\gamma}A_{\alpha\nu}A_{\beta\lambda}A_{\gamma\rho}\right] \\
=&\epsilon^{\mu\nu\lambda\rho}\left[A_{\gamma\nu}\left( \partial_\lambda A_{\gamma\rho}-\partial_\rho A_{\gamma\lambda} +C_{\alpha\beta\gamma}A_{\alpha\lambda}A_{\beta\rho} \right)-\frac{1}{3}C_{\alpha\beta\gamma}A_{\alpha\nu}A_{\beta\lambda}A_{\gamma\rho}\right] \\
=&\epsilon^{\mu\nu\lambda\rho}\left[A_{\gamma\nu}\partial_\lambda A_{\gamma\rho}-A_{\gamma\nu}\partial_\rho A_{\gamma\lambda} +C_{\alpha\beta\gamma}A_{\alpha\lambda}A_{\beta\rho}A_{\gamma\nu}-\frac{1}{3}C_{\alpha\beta\gamma}A_{\alpha\nu}A_{\beta\lambda}A_{\gamma\rho}\right] \\
=&\epsilon^{\mu\nu\lambda\rho}\left[A_{\gamma\nu}\partial_\lambda A_{\gamma\rho}+A_{\gamma\nu}\partial_\lambda A_{\gamma\rho} +C_{\alpha\beta\gamma}A_{\alpha\nu}A_{\beta\lambda}A_{\gamma\rho}-\frac{1}{3}C_{\alpha\beta\gamma}A_{\alpha\nu}A_{\beta\lambda}A_{\gamma\rho}\right] \quad \because \epsilon^{\mu\nu\lambda\rho} の反対称性 \\
=& 2\epsilon^{\mu\nu\lambda\rho}\left[A_{\gamma\nu}\partial_\lambda A_{\gamma\rho}+\frac{1}{3}C_{\alpha\beta\gamma}A_{\alpha\nu}A_{\beta\lambda}A_{\gamma\rho}\right]
\end{align*}
として確認できる.)これを微分すると
\begin{align*}
\partial_\mu G^\mu=&2\epsilon^{\mu\nu\lambda\rho}\biggl[ (\partial_\mu A_{\gamma\nu} )(\partial_\lambda A_{\gamma\rho})+ A_{\gamma\nu}(\partial_\mu \partial_\lambda A_{\gamma\rho}) \\
&\quad +\frac{1}{3}C_{\alpha\beta\gamma}\left\{(\partial_\mu A_{\alpha\nu})A_{\beta\lambda}A_{\gamma\rho} +A_{\alpha\nu}(\partial_\mu A_{\beta\lambda})A_{\gamma\rho} +A_{\alpha\nu}A_{\beta\lambda}(\partial_\mu A_{\gamma\rho}) \right\} \biggr] \\
=&2\epsilon^{\mu\nu\lambda\rho}\biggl[ (\partial_\mu A_{\gamma\nu} )(\partial_\lambda A_{\gamma\rho}) \quad \because \epsilon^{\mu\nu\lambda\rho}の反対称性と微分の可換性 \\
&\quad +\frac{1}{3}C_{\alpha\beta\gamma}\left\{ (\partial_\mu A_{\alpha\nu})A_{\beta\lambda}A_{\gamma\rho} -(\partial_\mu A_{\alpha\lambda})A_{\beta\nu}A_{\gamma\rho} +(\partial_\mu A_{\alpha\rho})A_{\beta\lambda}A_{\gamma\nu}\right\}\biggr] \quad\because C_{\alpha\beta\gamma}の反対称性 \\
=&2\epsilon^{\mu\nu\lambda\rho}\biggl[ (\partial_\mu A_{\gamma\nu} )(\partial_\lambda A_{\gamma\rho}) \\
&\quad +\frac{1}{3}C_{\alpha\beta\gamma}\left\{ (\partial_\mu A_{\alpha\nu})A_{\beta\lambda}A_{\gamma\rho} +(\partial_\mu A_{\alpha\nu})A_{\beta\lambda}A_{\gamma\rho} +(\partial_\mu A_{\alpha\nu})A_{\beta\lambda}A_{\gamma\rho}\right\}\biggr] \quad\because \epsilon^{\mu\nu\lambda\rho}の反対称性 \\
=&2\epsilon^{\mu\nu\lambda\rho}\left[ (\partial_\mu A_{\alpha\nu} )(\partial_\lambda A_{\alpha\rho})+(\partial_\mu A_{\alpha\nu})C_{\alpha\beta\gamma}A_{\beta\lambda}A_{\gamma\rho}\right]
\end{align*}
となるが,一方
\begin{align*}
\epsilon^{\mu\nu\lambda\rho}F_{\alpha\mu\nu}F_{\alpha\lambda\rho}=&\epsilon^{\mu\nu\lambda\rho}(\partial_\mu A_{\alpha \nu}-\partial_\nu A_{\alpha \mu}+C_{\alpha\beta\gamma}A_{\beta\mu}A_{\gamma\nu})(\partial_\lambda A_{\alpha\rho}-\partial_\rho A_{\alpha\lambda}+C_{\alpha\beta\gamma}A_{\beta\lambda}A_{\gamma\rho}) \\
=&\epsilon^{\mu\nu\lambda\rho}(2\partial_\mu A_{\alpha\nu} +C_{\alpha\beta\gamma}A_{\beta\mu}A_{\gamma\nu})(2\partial_\lambda A_{\alpha\rho}+C_{\alpha\beta\gamma}A_{\beta\lambda}A_{\gamma\rho}) \quad \because \epsilon^{\mu\nu\lambda\rho}の反対称性 \\
=&\epsilon^{\mu\nu\lambda\rho}[4(\partial_\mu A_{\alpha\nu})(\partial_\lambda A_{\alpha\rho})+2(\partial_\mu A_{\alpha\nu})C_{\alpha\beta\gamma}A_{\beta\lambda}A_{\gamma\rho}+2(\partial_\lambda A_{\alpha\rho})C_{\alpha\beta\gamma}A_{\beta\mu}A_{\gamma\nu} \\
&+C_{\alpha\beta\gamma}C_{\alpha\delta\epsilon}A_{\beta\mu}A_{\gamma\nu}A_{\delta\lambda}A_{\epsilon\rho}] \\
=&\epsilon^{\mu\nu\lambda\rho}[4(\partial_\mu A_{\alpha\nu})(\partial_\lambda A_{\alpha\rho})+4(\partial_\mu A_{\alpha\nu})C_{\alpha\beta\gamma}A_{\beta\lambda}A_{\gamma\rho} ]
\end{align*}
となる.最後の等号ではヤコビの恒等式より
\begin{align*}
0=&\epsilon^{\mu\nu\lambda\rho}(C_{\alpha\beta\gamma}C_{\alpha\delta\epsilon}+C_{\alpha\delta\beta}C_{\alpha\gamma\epsilon}+C_{\alpha\gamma\delta}C_{\alpha\beta\epsilon})A_{\beta\mu}A_{\gamma\nu}A_{\delta\lambda}A_{\epsilon\rho} \\
=&\epsilon^{\mu\nu\lambda\rho}C_{\alpha\beta\gamma}C_{\alpha\delta\epsilon}A_{\beta\mu}A_{\gamma\nu}A_{\delta\lambda}A_{\epsilon\rho} \\
&+\epsilon^{\mu\nu\lambda\rho}C_{\alpha\delta\beta}C_{\alpha\gamma\epsilon}A_{\beta\mu}A_{\gamma\nu}A_{\delta\lambda}A_{\epsilon\rho}+\epsilon^{\mu\nu\lambda\rho}C_{\alpha\gamma\delta}C_{\alpha\beta\epsilon}A_{\beta\mu}A_{\gamma\nu}A_{\delta\lambda}A_{\epsilon\rho} \\
=&\epsilon^{\mu\nu\lambda\rho}C_{\alpha\beta\gamma}C_{\alpha\delta\epsilon}A_{\beta\mu}A_{\gamma\nu}A_{\delta\lambda}A_{\epsilon\rho}\\
&+\epsilon^{\mu\nu\lambda\rho}C_{\alpha\beta\gamma}C_{\alpha\delta\epsilon}A_{\gamma\mu}A_{\delta\nu}A_{\beta\lambda}A_{\epsilon\rho}+ \epsilon^{\mu\nu\lambda\rho}C_{\alpha\beta\gamma}C_{\alpha\delta\epsilon}A_{\delta\mu}A_{\beta\nu}A_{\gamma\lambda}A_{\epsilon\rho} \quad(\beta\gamma\delta 添え字の再定義)\\
=&\epsilon^{\mu\nu\lambda\rho}C_{\alpha\beta\gamma}C_{\alpha\delta\epsilon}A_{\beta\mu}A_{\gamma\nu}A_{\delta\lambda}A_{\epsilon\rho}\\
&+\epsilon^{\mu\nu\lambda\rho}C_{\alpha\beta\gamma}C_{\alpha\delta\epsilon}A_{\beta\lambda}A_{\gamma\mu}A_{\delta\nu}A_{\epsilon\rho}+ \epsilon^{\mu\nu\lambda\rho}C_{\alpha\beta\gamma}C_{\alpha\delta\epsilon}A_{\beta\nu}A_{\gamma\lambda}A_{\delta\mu}A_{\epsilon\rho}\quad (Aの並び替え) \\
=&\epsilon^{\mu\nu\lambda\rho}C_{\alpha\beta\gamma}C_{\alpha\delta\epsilon}A_{\beta\mu}A_{\gamma\nu}A_{\delta\lambda}A_{\epsilon\rho} \\
&+\epsilon^{\mu\nu\lambda\rho}C_{\alpha\beta\gamma}C_{\alpha\delta\epsilon}A_{\beta\mu}A_{\gamma\nu}A_{\delta\lambda}A_{\epsilon\rho}+\epsilon^{\mu\nu\lambda\rho}C_{\alpha\beta\gamma}C_{\alpha\delta\epsilon}A_{\beta\mu}A_{\gamma\nu}A_{\delta\lambda}A_{\epsilon\rho} \quad (\mu\nu\lambda 添え字の再定義) \\
=&3\epsilon^{\mu\nu\lambda\rho}C_{\alpha\beta\gamma}C_{\alpha\delta\epsilon}A_{\beta\mu}A_{\gamma\nu}A_{\delta\lambda}A_{\epsilon\rho}
\end{align*}
であるから最後の項がゼロとなることを用いた.したがって
\begin{align*}
\partial_\mu G^\mu=\frac{1}{2}\epsilon^{\mu\nu\lambda\rho}F_{\alpha\mu\nu}F_{\alpha\lambda\rho}
\end{align*}
となることがわかる.したがって(22.2.26)は
\begin{align*}
\braket{\partial_\mu J^\mu_5}_A&=-\frac{N}{8\pi^2}\partial_\mu G^\mu \\
\partial_\mu \left[\braket{J^\mu_5}_A +\frac{N}{8\pi^2}G^\mu\right]&=0 \\
\partial_\mu K^\mu &=0
\end{align*}
という保存条件に書き換えることができる.しかし,カレント$K^\mu$の保存則を用いて$\pi^0\to2\gamma$過程が抑えられると論じることはできない.$K^\mu$は保存されているが,(22.2.29)からわかるようにゲージ不変でないからだ.\par
アノマリーの表式(22.2.24)の導出を見れば,(22.2.13)において$f(-\Slash{D}^2_x/M^2)$の代わりに微分演算子$f(-\Slash{\partial}^2_x/M^2)$を使って計算すれば
\begin{align*}
\mc{A}(x)=&-2\int \frac{d^4 k}{(2\pi)^4}\left[\mr{Tr}\left\{\gamma_5 tf(-\Slash{\partial}^2_x/M^2) \right\}e^{ik(x-y)}\right]_{y\to x} \\
=&-2\int \frac{d^4 k}{(2\pi)^4}\mr{Tr}\left\{\gamma_5 tf(-(i\Slash{k})^2/M^2) \right\}\\
=&-2\int \frac{d^4 k}{(2\pi)^4}\mr{Tr}\left\{\gamma_5 tf(+k^2/M^2) \right\}=-2\int \frac{d^4 k}{(2\pi)^4}f(+k^2/M^2)\mr{Tr}\left\{\gamma_5 t \right\} \\
=&0
\end{align*}
となってアノマリー関数がゼロになることがわかる.このやり方の問題点は,前に述べた通り,正則化演算子がゲージ不変でなく,そのあめに$K^\mu$にゲージ不変でない項が生じることだ.フェルミオン・プロパゲータと行列式の正則化において,ゲージ不変かつカイラル不変な方法は存在しない!

\vskip\baselineskip

さて,アノマリーの話の原点となった問題に戻り,上の結果を使い$\pi^0\to 2\gamma$過程の実際の反応率を計算しよう.興味のある対称性は狩りクォーク場の中性カイラル変換(22.1.5)によって生成される.
\begin{align*}
&\left(
\begin{array}{cc}
u\\
d
\end{array}
\right)\to \exp(i\alpha\gamma_5\tau_3)\left(
\begin{array}{cc}
u\\
d
\end{array}
\right) \\
&\delta \left(
\begin{array}{cc}
u\\
d
\end{array}
\right)=i\alpha \gamma_5 \tau_3 \left(
\begin{array}{cc}
u\\
d
\end{array}
\right)=i\alpha \gamma_5 \left(
\begin{array}{cc}
1 & 0 \\
0 & -1
\end{array}
\right)\left(
\begin{array}{cc}
u\\
d
\end{array}
\right) \\
&\delta u=i\alpha \gamma_5 u ,\quad \delta d=-i\alpha \gamma_5 d
\end{align*}
純粋な量子色力学(クォークとグルーオンの相互作用のみについての理論)ではこの対称性はアノマリーを持たない.なぜなら$u,d$が色ゲージ群の同じ表現に属し,(22.2.33)の対称性のアノマリーへのグルーオン・グルーオン項へのそれらの寄与は相殺するからだ.つまり$t_\alpha$をゲージ群の生成子,$G^{\mu\nu}_\alpha$をグルーオン場のテンソルとして
\begin{align*}
\mc{A}(x)&=\mc{A}_u(x)+\mc{A}_d(x) \\
&=-\frac{1}{16\pi^2}\epsilon_{\mu\nu\rho\sigma}G^{\mu\nu}_\alpha G^{\rho\sigma}_\beta \mr{tr}\left\{t_\alpha t_\beta (+1)\right\}-\frac{1}{16\pi^2}\epsilon_{\mu\nu\rho\sigma}G^{\mu\nu}_\alpha G^{\rho\sigma}_\beta\mr{tr}\left\{t_\alpha t_\beta (-1)\right\} \\
&=0
\end{align*}
となって相殺する.($t_\alpha$は)一方,電磁場$A^\mu(x)$が存在すれば,この対称性はアノマリー
\begin{align*}
\mc{A}(x)=-\frac{1}{16\pi^2}\epsilon_{\mu\nu\rho\sigma}F^{\mu\nu} F^{\rho\sigma} \mr{tr}\left\{q^2 \tau_3\right\}
\end{align*}
をもつ.ここで$q$はクォークの電荷行列だ.もし,通常のように$N_c$個の電荷$2e/3$をもつ$u$クォークと,同じ個数の電荷$-e/3$をもつ$d$クォークがあるとすれば(つまりカラーの数が$N_c$個あるとすれば),トレースは固有値の和であるから
\begin{align*}
\mr{tr}\left\{q^2 \tau_3 \right\}=N_c \left(\frac{2e}{3}\right)(+1)+N_c\left(\frac{-e}{3}\right)(-1)=\frac{N_c e^2}{3}
\end{align*}
となる.したがってアノマリー関数は
\begin{align*}
\mc{A}(x)=-\frac{N_ce^2}{48\pi^2}\epsilon_{\mu\nu\rho\sigma}F^{\mu\nu}(x)F^{\rho\sigma}(x)
\end{align*}
だ.そのため有効ラグランジアンにはカイラル変換(22.2.33)のもとで(22.2.12)を与えるような項を含まなければならない.それは以下で与えられる.
\begin{align*}
\delta \mc{L}_{eff}(x)=\alpha \mc{A}(x)=-\frac{N_ce^2}{48\pi^2}\epsilon_{\mu\nu\rho\sigma}F^{\mu\nu}(x)F^{\rho\sigma}(x)\alpha
\end{align*}
さて,変換(22.2.33)の対称性がアノマリーによって完全に破れることで,(19.7節参照)$\tau_3$に結合するNGボゾンである$\pi^0$場は
\begin{align*}
g\gamma(\xi)&=\gamma(\xi')h(\xi,g) \\
\exp[i\gamma_5\tau_3\alpha]\exp[i\gamma_5 \tau_3 \xi_3(x)]&=\exp[i\gamma_5 \tau_3 \xi'_3(x)] \\
\xi'_3&=\xi_3+\alpha \\
{\pi^0}'&=\pi^0+\alpha F_\pi \quad \because \pi^0=F_\pi \xi_3 \\
\delta \pi^0&=\alpha F_\pi
\end{align*}
と変換する.ここで$F_\pi=184\mr{MeV}$は19章で導入した定数だ.これにより,有効ラグランジアンは以下の項を含まなければならない.
\begin{align*}
\frac{\pi^0\mc{A}(x)}{F_\pi}=-\frac{N_ce^2}{48\pi^2F_\pi}\epsilon_{\mu\nu\rho\sigma}F^{\mu\nu}(x)F^{\rho\sigma}(x)\pi^0(x)
\end{align*}
これは変換(22.2.33)のもとで不変であることがすぐに分かる.これを$\pi^0\to2\gamma$崩壊の有効ラグランジアンの一般的な表式(22.1.1)と比較すると,(22.1.1)の定数$g$は
\begin{align*}
g=\frac{N_ce^2}{48\pi^2F_\pi}
\end{align*}
という値をとらなければならないことがわかる.こうして,(22.1.2)のパイ中間子崩壊率は
\begin{align*}
\Gamma(\pi^0\to2\gamma)=\frac{N_c^2\alpha^2 m^3_\pi}{144\pi^3 F^2_\pi}=\left(\frac{N_c}{3}\right)^2\times 1.11\times 10^{16}\,\mr{s}^{-1}
\end{align*}
と予言される.観測された値は$\Gamma(\pi^0\to 2\gamma)=(1.19\pm 0.08)\times 10^{16}\,\mr{s}^{-1}$であり,これは$N_c=3$のときに限って理論値(22.2.39)とよく一致する.この計算の成功は,クォークに3つの色があるという証拠のもっとも初期のものの一つだ.

\vskip\baselineskip

アノマリーのより厳密な導出は,ウィック回転したユークリッド時空での経路積分を使って得られる.ユークリッド第4座標を$x_4=ix^0=-ix_0$で導入する.そしてそれに対応して,$\partial_4=\partial/\partial x_4=-i\partial/\partial x^0=-i\partial_0, \gamma_4\equiv i\gamma^0 , A_{\alpha 4}=iA_{\alpha}^0$とする.($x_4,A_{\alpha 4}$は実であるようにウィック回転していることに注意.また$\gamma_4^\dagger=\gamma_4$かつ$\gamma_i^\dagger=\gamma_i$よりガンマ行列はエルミートとなる.)また時空の微小体積は$(d^4x)_E$をユークリッド体積要素$(d^4x)_E=dx_1 dx_2 dx_3 dx_4$として$d^4x=-i(d^4x)_E$となる.ユークリッド時空では場$\psi(x)$とそのディラック共役場$\bar{\psi}(x)$は完全に独立として扱わなければならない.それらの局所カイラル変換は,これらが独立でないときに
\begin{align*}
\delta\psi(x)&=i\alpha(x)t \gamma_5 \psi(x) \\
\delta\bar{\psi}(x)&=[\delta \psi(x)]^\dagger \gamma_4=[i\alpha(x)t \gamma_5 \psi(x)]\gamma_4 \\
&=-i\alpha(x)\psi(x)^\dagger t\gamma_5^\dagger \gamma_4 \\
&=i\alpha(x)\bar{\psi}(x)t\gamma_5 \quad \because (5.4.33)
\end{align*}
と変換されるのだったから,$\delta\psi(x)=i\alpha(x)t \gamma_5 \psi(x),\delta\bar{\psi}(x)=i\alpha(x)\bar{\psi}(x)t\gamma_5$で定義される.そうすると,測度の変換は,アノマリー関数$\mc{A}(x)$を(22.2.11)として,前と同じく(22.2.10)で与えられる.
\begin{align*}
&[d\psi][d\bar\psi]\to\exp\left\{\int (d^4x)_E \alpha(x)\mc{A}(x)\right\}[d\psi][d\bar\psi] \\
&\mc{A}(x)=-2\mr{Tr}\left\{\gamma_5 t\right\}\delta^4(x-x)
\end{align*}
前と同様に正則化関数を導入すると,これは$\mc{A}(x)$について(22.2.13)と同様の表式を与える.
\begin{align*}
\mc{A}(x)=-2\lim_{M\to \infty}\left[\mr{Tr}\left\{\gamma_5 t f(-\Slash{D}^2/M^2)\right\}\delta^4(x-y)\right]_{y\to x}
\end{align*}
ユークリッド化した取り扱いの非常な有利な点は,$x_4$と$A_{\alpha 4}$が実としているために,表式(22.2.13)のディラック演算子$i\Slash{D}$がエルミートなことだ.
\begin{align*}
i\Slash{D}=&[i\partial_\mu+t_\alpha A_{\mu\alpha}]\gamma^\mu \\
=&[i\partial_0 +t_\alpha A_{0\alpha}]\gamma^0+[i\partial_i +t_\alpha A_{i \alpha}]\gamma^i \quad (i=1\sim 3 )\\
=&[-\partial_4+it_\alpha A_{4\alpha}](-i\gamma_4)+[i\partial_i +t_\alpha A_{i \alpha}]\gamma_i \quad (i=1\sim 3) \\
=&[i\partial_i +t_\alpha A_{i \alpha}]\gamma_i \quad (i=1\sim 4)\\
(i\Slash{D})^\dagger=&[(i\partial_i)^\dagger +t_\alpha^\dagger A_{i\alpha}^\dagger]\gamma^\dagger_i \\
=&[i\partial_i +t_\alpha A_{i\alpha}]\gamma_i \quad \because \gamma_i^\dagger =\gamma_i ,(i\partial_i)^\dagger =i\partial_i \\
=& i\Slash{D}
\end{align*}
ここで$i\partial_i$がエルミートであることは,運動量演算子であることを思い出せば理解できる.あるいは部分積分より
\begin{align*}
\int d^4x \psi^\dagger(x) \left(\frac{\partial}{\partial x_i} \psi(x)\right)=\int d^4x \left(-\frac{\partial}{\partial x_i} \psi(x)\right)^\dagger \psi(x)
\end{align*}
で,$\partial_i$のエルミート共役は$-\partial_i$だとわかるから,$i\partial_i$はエルミートだとわかる.したがって,ディラック演算子$i\Slash{D}$は正規直交スピノル固有関数$\varphi_\kappa(x)$をもつ.
\begin{align*}
&i\Slash{D}\varphi_\kappa=\lambda_\kappa \varphi_\kappa \\
&\int (d^4x)_E \varphi_\kappa(x)^\dagger \varphi_{\kappa'}(x)=\delta_{\kappa\kappa'}
\end{align*}
エルミート演算子の固有値は実であるから,$\lambda_\kappa$は実だ.また,この節でずっとそうしているように$t$は$i\Slash{D}$と可換だとしているから,
\begin{align*}
i\Slash{D}(t\varphi_\kappa)=t(i\Slash{D}\varphi_\kappa)=\lambda_\kappa (t\varphi_\kappa)
\end{align*}
となって$t\varphi_\kappa$も固有値$\lambda_\kappa$をもつ固有ベクトルであり$t\varphi_\kappa\propto \varphi_\kappa$とできる.この比例定数を$t_\kappa$とすれば$t\varphi_\kappa=t_\kappa \varphi_\kappa$を満たすように$\varphi_\kappa$を選ぶことができる.これらの固有関数は完全性条件
\begin{align*}
\sum_\kappa \varphi_\kappa(x)\varphi_\kappa^\dagger(y)=\sum_\kappa \braket{x|\varphi_\kappa}\braket{\varphi_\kappa|y}=\delta^4(x-y)1
\end{align*}
を満たす.ここで「1」は$4\times 4$単位行列だ.したがって,アノマリー関数はいまや明白に収束する和の極限として
\begin{align*}
\mc{A}(x)=&-2\lim_{M\to \infty}\mr{Tr}\left\{\gamma_5 t f(-\Slash{D}^2/M^2)\right\}\delta^4(x-x) \\
=&-2\lim_{M\to\infty}\mr{Tr}\left\{ \gamma_5 tf(-\Slash{D}^2/M^2)\sum_\kappa\varphi_\kappa(x)\varphi_\kappa^\dagger(x) \right\} \\
=&-2\lim_{M\to\infty}\sum_\kappa  t_\kappa f(\lambda_\kappa^2/M^2) \mr{Tr}\left\{ \gamma_5 \varphi_\kappa(x)\varphi_\kappa^\dagger(x) \right\} \\
=&-2\lim_{M\to\infty}\sum_\kappa  t_\kappa f(\lambda_\kappa^2/M^2) \left( \varphi_\kappa^\dagger(x) \gamma_5 \varphi_\kappa(x) \right)
\end{align*}
と書くことができる.\par
アノマリー関数について(22.2.24)の表式を導いたのと同様に再び導く.
\begin{align*}
\mc{A}(x)=&-2\lim_{M\to \infty}\left[\mr{Tr}\left\{\gamma_5 t f(-\Slash{D}^2/M^2)\right\}\delta^4(x-y)\right]_{y\to x} \\
=&-2\lim_{M\to \infty}\int \frac{(d^4 k)_E}{(2\pi)^4}\left[\mr{Tr}\left\{\gamma_5 t f(-\Slash{D}^2/M^2)\right\}e^{ik(x-y)}\right]_{y\to x} \quad (指数の肩の内積はユークリッド内積) \\
=&-2\lim_{M\to \infty}\int \frac{(d^4 k)_E}{(2\pi)^4}\mr{Tr}\left\{\gamma_5 t f(-[i\Slash{k} +\Slash{D}]^2/M^2)\right\} \\
=&-2\lim_{M\to \infty}M^4 \int \frac{(d^4 k)_E}{(2\pi)^4}\mr{Tr}\left\{\gamma_5 t f(-[i\Slash{k} +\Slash{D}/M]^2)\right\} \\
=&-2\int \frac{(d^4 k)_E}{(2\pi)^4}f''(k^2)\mr{Tr}\left\{\gamma_5 t \Slash{D}^4\right\} \quad \because (22.2.18)
\end{align*}
ここで$k$は既にユークリッド座標であるから,ウィック回転を介さずに(22.2.20)の計算を行うことができる.
\begin{align*}
\int (d^4 k)_Ef''(k^2)=\int^\infty_0 2\pi^2 \kappa^3 d\kappa f''(\kappa^2)
\end{align*}
(ウィック回転を介していないために虚数が表れていないことに注意せよ.)したがって(22.2.21)は
\begin{align*}
\int (d^4k)_E f''(k^2)=\pi^2
\end{align*}
となる.トレースを計算するには(22.2.22)と同様に
\begin{align*}
\Slash{D}^2=D^2-\frac{1}{4}it_\alpha F^{ij}_\alpha[\gamma_i,\gamma_j]
\end{align*}
と書き
\begin{align*}
\mr{tr_D}\left\{\gamma_5 [\gamma_i,\gamma_j][\gamma_k,\gamma_\ell]\right\}=16\epsilon^E_{ijk\ell}
\end{align*}
を用いると((22.2.23)において$\epsilon_{\mu\nu\rho\sigma}$により$\mu\sim\sigma$のどれのガンマ行列は必ずゼロ成分であるから,$\gamma_4=i\gamma^0=-i\gamma_0$を用いて全体に$-i$をかければこの等式が得られる.)
\begin{align*}
\mc{A}(x)=&-2\frac{\pi^2}{(2\pi)^4}\left(-\frac{i}{4}\right)^216\epsilon^E_{ijk\ell}F_{ij\alpha}F_{k\ell \beta}\mr{tr}\left\{t_\alpha t_\beta t\right\} \\
=&\frac{1}{16\pi^2}\epsilon^E_{ijk\ell}F_{ij\alpha}F_{k\ell \beta}\mr{tr}\left\{t_\alpha t_\beta t\right\}
\end{align*}
を示すことができる.(ユークリッド化による影響で,符号が(22.2.24)とは異なることに注意せよ.)\par
さて,固有値が$\lambda_\kappa\neq 0$である$i\Slash{D}$と$t$の任意の固有関数$\varphi_\kappa(x)$が与えられたとき,$\varphi_{\kappa^-}(x)\equiv \gamma_5 \varphi_\kappa(x)$で与えられる別の規格化された固有関数$\varphi_{\kappa^-}(x)$があって,$i\Slash{D}$の固有値は
\begin{align*}
i\Slash{D}\varphi_{\kappa^-}(x)&=i\Slash{D}\gamma_5\varphi_\kappa(x) \\
&=-\gamma_5 i\Slash{D}\varphi_\kappa(x) \\
&=-\lambda_\kappa \gamma_5 \varphi_\kappa(x)=-\lambda_\kappa \varphi_{\kappa^-}(x)
\end{align*}
であるから$-\lambda_\kappa$となり,$t$の固有値は相変わらず$t_\kappa$だ.さらに
\begin{align*}
\left(\varphi_\kappa^\dagger(x)\varphi_\kappa(x)\right)=&\left(\varphi_\kappa^\dagger(x)\gamma_5 \gamma_5\varphi_\kappa(x)\right) \\
=&\left((\gamma_5\varphi_\kappa(x))^\dagger \gamma_5 \varphi_\kappa(x)\right) \\
=&\left(\varphi_{\kappa^-}^\dagger(x)\varphi_{\kappa^-}(x)\right)
\end{align*}
だから,$\varphi_\kappa(x)$は規格直交基底であって異なる固有値に属するものは直交するので$\lambda_\kappa\neq 0$のとき
\begin{align*}
\left( \varphi_\kappa^\dagger(x) \gamma_5 \varphi_\kappa(x) \right)=\left( \varphi_\kappa^\dagger(x) \varphi_{\kappa^-}(x) \right)=0
\end{align*}
となる.したがって(22.2.44)の和において$\lambda_\kappa=0$の固有関数についての和のみが残る.これらは一般に対になっていない.むしろ,$\gamma_5$は$i\Slash{D}$と反可換だから,$i\Slash{D}$の固有値がゼロで\uwave{同時に}それぞれ$\gamma_5$の固有値(カイラリティ)$+1$と$-1$をもつ同時正規直交固有関数$\varphi_u$と$\varphi_v$
\begin{align*}
&i\Slash{D}\varphi_u=0 ,\quad \gamma_5 \varphi_u=+\varphi_u \\
&i\Slash{D}\varphi_u=0 ,\quad \gamma_5 \varphi_v=-\varphi_v
\end{align*}
に選ぶことができる.$M\to \infty$を実行して(22.2.15)$f(0)=1$を用いると
\begin{align*}
\mc{A}(x)=&-2\left[\sum_u t_u \left( \varphi_u^\dagger(x) \gamma_5 \varphi_u(x) \right)+\sum_v t_v \left( \varphi_v^\dagger(x) \gamma_5 \varphi_v(x) \right) \right] \\
=&-2\left[\sum_u t_u \left( \varphi_u^\dagger(x) \varphi_u(x) \right)-\sum_v t_v \left( \varphi_v^\dagger(x) \varphi_v(x) \right) \right]
\end{align*}
となる.さて,$\varphi_u$と$\varphi_v$は(22.2.42)のように規格化されているから,(22.2.47)の積分は以下を与える.
\begin{align*}
\int (d^4x)_E \mc{A}(x)=-2\left[\sum_u t_u -\sum_v t_v \right]
\end{align*}
ここで$u$と$v$についての和はそれぞれ,演算子$i\Slash{D}$の左手と右手成分のゼロ・モードについてとる.特に,$t$が単位行列の場合は,(22.2.45)を使って,これをゲージ場の汎関数と,ディラック演算子のカイラリティが決まったゼロ・モードの数との関係式として表すことができる!
\begin{align*}
-\frac{1}{32\pi^2}\int (d^4x)_E \epsilon^E_{ijk\ell}F_{ij\alpha}F_{k\ell \beta}\mr{tr}\left[t_\alpha t_\beta\right]=n_+-n_-
\end{align*}
ここで$n_\pm$は$i\Slash{D}$のゼロ・モードで,$\gamma_5$の固有値$\pm 1$をもつものの数だ.($n_++n_-=n_0$で,$n_0$がゼロ・モードの全体の数だ.)これが有名なアティヤ・シンガーの指数定理だ.(中原幹夫著「理論物理学のための幾何学とトポロジーII」参照)それは色々なことを示しているが,その中でも特に,ゲージ場の変化のもとで(22.2.49)の左辺の積分が滑らかに変化することができないことを示している!特に,変化分は整数に限られているので,この積分はゲージ場のトポロジーにのみ依存できる.

\newpage

\subsection{一般的な場合のアノマリーの直接計算}
一般のアノマリーを扱うには,全ての左手フェルミオン場を(区別がある場合には,反フェルミオン場も)一つの列ベクトル$\chi$に統一して記す.たとえば,もし$\psi$が全てのクォークとレプトンを含む(反クォークと反レプトンは含まない)列ベクトルとすると,
\begin{align*}
\chi \equiv \left[
\begin{array}{cc}
\frac{1}{2}(1+\gamma_5)\psi \\
\frac{1}{2}[\mc{C}(1-\gamma_5)\psi]^*
\end{array}
\right]= \left[
\begin{array}{cc}
\frac{1}{2}(1+\gamma_5)\psi \\
\frac{1}{2}(1+\gamma_5)\mc{C}\psi^*
\end{array}
\right] \quad\because \mc{C}^*=\mc{C},\gamma_5^*=\gamma_5, \mc{C}\gamma_5 =-\gamma_5 \mc{C}
\end{align*}
ここで,$\mc{C}$は
\begin{align*}
\mc{C}\gamma^T_\mu \mc{C}^{-1}=-\gamma_\mu ,\gamma ,\quad \mc{C}=\gamma_2\gamma_4 
\end{align*}
で定義された行列だ.(すなわち,(5.5.47)より,下成分は右手フェルミオンの荷電共役場を表しており,これで全ての左手フェルミオンを統一して記すことができている.)フェルミオン数(バリオン数か,バリオン数引くレプトン数)を保存する微小ゲージ変換
\begin{align*}
\delta\psi=i\theta_\alpha\left[\frac{1}{2}(1+\gamma_5)t^L_\alpha +\frac{1}{2}(1-\gamma_5)t^R_\alpha \right]\psi
\end{align*}
のもとで,この列ベクトルは
\begin{align*}
\delta\chi=&\left[
\begin{array}{cc}
\frac{1}{2}(1+\gamma_5)\delta\psi \\
\frac{1}{2}[\mc{C}(1-\gamma_5)\delta\psi]^*
\end{array}
\right] \\
=&\left[
\begin{array}{cc}
\frac{1}{2}(1+\gamma_5)i\theta_\alpha t^L_\alpha \psi \\
\frac{1}{2}[\mc{C}(1-\gamma_5)i\theta_\alpha t^R_\alpha \psi]^*
\end{array}
\right] \\
=&\left[
\begin{array}{cc}
i\theta_\alpha t^L_\alpha\frac{1}{2}(1+\gamma_5)\psi \\
-i\theta_\alpha t^{R*}_\alpha \frac{1}{2}[\mc{C}(1-\gamma_5)\psi]^*
\end{array}
\right] \\
=&i\theta_\alpha \left[
\begin{array}{cc}
t^L_\alpha & 0 \\
0 & -t^{R*}_\alpha
\end{array}
\right]\left[
\begin{array}{cc}
\frac{1}{2}(1+\gamma_5)\delta\psi \\
\frac{1}{2}[\mc{C}(1-\gamma_5)\delta\psi]^*
\end{array}
\right]=i\theta_\alpha T_\alpha \chi
\end{align*}
という変換を受ける.ここで
\begin{align*}
T_\alpha =\left[
\begin{array}{cc}
t^L_\alpha & 0 \\
0 & -t^{R*}_\alpha
\end{array}
\right]=\left[
\begin{array}{cc}
t^L_\alpha & 0 \\
0 & -(t^{R}_\alpha)^T
\end{array}
\right] \quad \because t^R_\alpha はエルミート
\end{align*}
だ.ここではフェルミオン数を保存する理論に話を限らない.そのため,$T_\alpha$はゲージ代数の任意のエルミート表現であり,必ずしも(22.3.4)のブロック対角形をしていない.まず質量ゼロのフェルミオンのみを考察し,後でフェルミオンの質量の影響を論じる.

\vskip\baselineskip

ここで問題となるのは1ループ3点関数
\begin{align*}
\Gamma^{\mu\nu\rho}_{\alpha\beta\gamma}(x,y,z)\equiv \braket{T\left\{ j^\mu_\alpha (x),j^\nu_\beta(y),j^\rho_\gamma(z) \right\} }_{\VAC}
\end{align*}
だ.ここで$j^\mu_\alpha$はフェルミオン的カレントで自由場
\begin{align*}
j^\mu_\alpha =-i\bar{\chi}T_\alpha \gamma^\mu \chi
\end{align*}
を使って計算される.

\begin{figure}[H]
\centering
\begin{tikzpicture}[decoration={markings, 
mark= at position -1cm with {\arrow[line width=0.5mm]{Stealth}}}, scale=0.5]

\coordinate (a1) at (-4,0){};
\coordinate (a2) at (-2,0){};
\coordinate (b1) at ({2},{2*sqrt(3)}){};
\coordinate (c1) at ({2},{-2*sqrt(3)}){};
\coordinate (b2) at ({1},{sqrt(3)}){};
\coordinate (c2) at ({1},{-sqrt(3)}){};

\draw[thick,postaction={decorate}](b2)--(a2);
\draw[thick,postaction={decorate}](c2)--(b2);
\draw[thick,postaction={decorate}](a2)--(c2);

\begin{feynhand}
\propag[photon,thick](a2)--(a1);
\propag[photon,thick](b2)--(b1);
\propag[photon,thick](c2)--(c1);
\end{feynhand}

\draw(a1)node[above right]{$j^\mu_\alpha(x)$};
\draw(b1)node[right]{$j^\nu_\beta(y)$};
\draw(c1)node[right]{$j^\rho_\gamma(z)$};

\end{tikzpicture}
\begin{tikzpicture}[decoration={markings, 
mark= at position -1cm with {\arrow[line width=0.5mm]{Stealth}}}, scale=0.5]
\coordinate (a1) at (-4,0){};
\coordinate (a2) at (-2,0){};
\coordinate (b1) at ({2},{2*sqrt(3)}){};
\coordinate (c1) at ({2},{-2*sqrt(3)}){};
\coordinate (b2) at ({1},{sqrt(3)}){};
\coordinate (c2) at ({1},{-sqrt(3)}){};

\draw[thick,postaction={decorate}](b2)--(a2);
\draw[thick,postaction={decorate}](c2)--(b2);
\draw[thick,postaction={decorate}](a2)--(c2);

\begin{feynhand}
\propag[photon,thick](a2)--(a1);
\propag[photon,thick](b2)--(b1);
\propag[photon,thick](c2)--(c1);
\end{feynhand}

\draw(a1)node[above right]{$j^\mu_\alpha(x)$};
\draw(b1)node[right]{$j^\rho_\gamma(z)$};
\draw(c1)node[right]{$j^\nu_\beta(y)$};

\end{tikzpicture}
\end{figure}

\noindent 図の二つのファインマン・ダイアグラムは以下を与える.
\begin{align*}
\Gamma^{\mu\nu\rho}_{\alpha\beta\gamma}(x,y,z)=&-(-i)^3\mr{Tr}[S(x-y)T_\beta \gamma^\nu P_L S(y-z)T_\gamma \gamma^\rho P_L S(z-x)T_\alpha \gamma^\mu P_L] \\
&-(-i)^3\mr{Tr}[S(x-z)T_\gamma \gamma^\rho P_L S(z-y)T_\beta \gamma^\nu P_L S(z-x)T_\alpha \gamma^\mu P_L] \\
=&-i\mr{Tr}[S(x-y)T_\beta \gamma^\nu P_L S(y-z)T_\gamma \gamma^\rho P_L S(z-x)T_\alpha \gamma^\mu P_L] \\
&-i\mr{Tr}[S(x-z)T_\gamma \gamma^\rho P_L S(z-y)T_\beta \gamma^\nu P_L S(z-x)T_\alpha \gamma^\mu P_L]
\end{align*}
$(-i)$の因子はカレントから生じる.マイナスはフェルミオンループの存在から生じる.(1巻368pを参照.)ここで$P_L$は左手フェルミオン場への射影演算子
\begin{align*}
P_L=\left(\frac{1+\gamma_5}{2}\right)
\end{align*}
で,また$S(x-y)$は質量ゼロのフェルミオン場のプロパゲータ
\begin{align*}
S(x-y)=\frac{-i}{(2\pi)^4}\int d^4p\,\left(\frac{-i\Slash{p}}{p^2-i\epsilon}\right)e^{ip(x-y)}
\end{align*}
だ.(これは$\braket{T\left\{\chi(x),\bar{\chi}(y)\right\}}$ではなく,22.3節の最後の議論における$\braket{T\left\{\Psi(x),\bar{\Psi}(y)\right\}}$であることに注意.)因子を集めると,(22.3.7)は
\begin{align*}
&\Gamma^{\mu\nu\rho}_{\alpha\beta\gamma}(x,y,z)= \\
&-i\mr{Tr}\biggl[\frac{-i}{(2\pi)^4}\int d^4 q_1 \left(\frac{-i\Slash{q}_1}{q^2_1-i\epsilon} \right)e^{iq_1\cdot (x-y)}T_\beta \gamma^\nu \frac{1+\gamma_5}{2} \\
&\qquad \times \frac{-i}{(2\pi)^4}\int d^4 q_2 \left(\frac{-i\Slash{q}_2}{q^2_2-i\epsilon} \right)e^{iq_2\cdot (y-z)}T_\gamma \gamma^\rho \frac{1+\gamma_5}{2} \\
&\qquad \times \frac{-i}{(2\pi)^4}\int d^4 q_3 \left(\frac{-i\Slash{q}_3}{q^2_3-i\epsilon} \right)e^{iq_3\cdot (z-x)}T_\alpha \gamma^\mu \frac{1+\gamma_5}{2} \biggr] \\
&-i\mr{Tr}\biggl[\frac{-i}{(2\pi)^4}\int d^4 q_1 \left(\frac{-i\Slash{q}_1}{q^2_1-i\epsilon} \right)e^{iq_1\cdot (x-z)}T_\beta \gamma^\nu \frac{1+\gamma_5}{2} \\
&\qquad \times \frac{-i}{(2\pi)^4}\int d^4 q_2 \left(\frac{-i\Slash{q}_2}{q^2_2-i\epsilon} \right)e^{iq_2\cdot (z-y)}T_\gamma \gamma^\rho \frac{1+\gamma_5}{2} \\
&\qquad \times \frac{-i}{(2\pi)^4}\int d^4 q_3 \left(\frac{-i\Slash{q}_3}{q^2_3-i\epsilon} \right)e^{iq_3\cdot (y-x)}T_\alpha \gamma^\mu \frac{1+\gamma_5}{2} \biggr] \\
=&\frac{i}{(2\pi)^{12}}\biggl\{\int d^4 q_1 d^4 q_2 d^4 q_3 e^{i(q_1-q_3)\cdot x}e^{i(q_2-q_1)\cdot y}e^{i(q_3-q_2)\cdot z} \\
& \qquad \times \mr{tr}\left[\frac{\Slash{q}_1}{q^2_1-i\epsilon} \gamma^\nu \frac{\Slash{q}_2}{q^2_2-i\epsilon}\gamma^\rho\frac{\Slash{q}_3}{q^2_3-i\epsilon}\gamma^\mu \frac{1+\gamma_5}{2}\right]\mr{tr}[T_\beta T_\gamma T_\alpha ] \\
&+ \int d^4 q_1 d^4 q_2 d^4 q_3e^{i(q_1-q_3)\cdot x}e^{i(q_2-q_1)\cdot y}e^{i(q_3-q_2)\cdot z} \\
& \qquad \times \mr{tr}\left[\frac{\Slash{q}_1}{q^2_1-i\epsilon} \gamma^\rho \frac{\Slash{q}_2}{q^2_2-i\epsilon}\gamma^\nu \frac{\Slash{q}_3}{q^2_3-i\epsilon}\gamma^\mu \frac{1+\gamma_5}{2}\right]\mr{tr}[T_\gamma T_\beta T_\alpha ]\biggr\}
\end{align*}
ここで,第一項目では変数変換
\begin{align*}
q_1^\mu \to k_1^\mu=p^\mu-q_1^\mu+a^\mu,\quad q_2^\mu \to p^\mu=q_2^\mu-a^\mu ,\quad q^\mu_3\to k_2=-p^\mu+q_3^\mu-a^\mu
\end{align*}
を,第二項目では
\begin{align*}
q_1^\mu \to k_2^\mu=p^\mu-q_1^\mu+b^\mu ,\quad q_2^\mu \to p^\mu=q_2^\mu-b^\mu ,\quad q^\mu_3\to k_1^\mu=-p^\mu+q_3^\mu-b^\mu
\end{align*}
を行うと(体積要素から出るマイナスは積分領域のマイナスと打ち消し合って)
\begin{align*}
\Gamma^{\mu\nu\rho}_{\alpha\beta\gamma}(x,y,z)&=\frac{i}{(2\pi)^{12}}\int d^4 k_1 d^4k_2 e^{-i(k_1+k_2)\cdot x}e^{ik_1\cdot y}e^{ik_2\cdot z}\int d^4p \\
&\times \biggl\{\mr{tr}\left[\frac{\Slash{p}-\Slash{k}_1+\Slash{a}}{(p-k_1+a)^2-i\epsilon}\gamma^\nu \frac{\Slash{p}+\Slash{a}}{(p+a)^2-i\epsilon}\gamma^\rho \frac{\Slash{p}+\Slash{k}_2+\Slash{a}}{(p+k_2+a)^2-i\epsilon}\gamma^\mu \frac{1+\gamma_5}{2}\right]\mr{tr}[T_\beta T_\gamma T_\alpha] \\
&+\mr{tr}\left[\frac{\Slash{p}-\Slash{k}_2+\Slash{b}}{(p-k_2+b)^2-i\epsilon}\gamma^\rho \frac{\Slash{p}+\Slash{b}}{(p+b)^2-i\epsilon}\gamma^\nu \frac{\Slash{p}+\Slash{k}_1+\Slash{b}}{(p+k_1+b)^2-i\epsilon}\gamma^\mu \frac{1+\gamma_5}{2}\right]\mr{tr}[T_\gamma T_\beta T_\alpha]  \biggr\}
\end{align*}
となる.
\begin{align*}
\Slash{k}_1+\Slash{k}_2=(\Slash{p}+\Slash{k}_2+\Slash{a})-(\Slash{p}-\Slash{k}_1+\Slash{a})=(\Slash{p}+\Slash{k}_1+\Slash{b})-(\Slash{p}-\Slash{k}_2+\Slash{b})
\end{align*}
を用いると,トレースの巡回性と$\gamma_5\gamma^\mu=-\gamma^\mu \gamma_5$と$\Slash{k}\Slash{k}=k^2$より
\begin{align*}
&\frac{\partial}{\partial x^\mu}\Gamma^{\mu\nu\rho}_{\alpha\beta\gamma}(x,y,z) \\
=&\frac{1}{(2\pi)^{12}}\int d^4k_1 d^4k_2(k_{1\mu} +k_{2\mu})e^{-i(k_1+k_2)\cdot x}e^{ik_1\cdot y}e^{ik_2\cdot z}\int d^4p \\
&\times \biggl\{\mr{tr}\left[\frac{\Slash{p}-\Slash{k}_1+\Slash{a}}{(p-k_1+a)^2-i\epsilon}\gamma^\nu \frac{\Slash{p}+\Slash{a}}{(p+a)^2-i\epsilon}\gamma^\rho \frac{\Slash{p}+\Slash{k}_2+\Slash{a}}{(p+k_2+a)^2-i\epsilon}\gamma^\mu \frac{1+\gamma_5}{2}\right]\mr{tr}[T_\beta T_\gamma T_\alpha] \\
&+\mr{tr}\left[\frac{\Slash{p}-\Slash{k}_2+\Slash{b}}{(p-k_2+b)^2-i\epsilon}\gamma^\rho \frac{\Slash{p}+\Slash{b}}{(p+b)^2-i\epsilon}\gamma^\nu \frac{\Slash{p}+\Slash{k}_1+\Slash{b}}{(p+k_1+b)^2-i\epsilon}\gamma^\mu \frac{1+\gamma_5}{2}\right]\mr{tr}[T_\gamma T_\beta T_\alpha]  \biggr\} \\
=&\frac{1}{(2\pi)^{12}}\int d^4k_1 d^4k_2e^{-i(k_1+k_2)\cdot x}e^{ik_1\cdot y}e^{ik_2\cdot z}\int d^4p \\
&\times \biggl\{\mr{tr}\left[\frac{\Slash{p}-\Slash{k}_1+\Slash{a}}{(p-k_1+a)^2-i\epsilon}\gamma^\nu \frac{\Slash{p}+\Slash{a}}{(p+a)^2-i\epsilon}\gamma^\rho \frac{\Slash{p}+\Slash{k}_2+\Slash{a}}{(p+k_2+a)^2-i\epsilon}(\Slash{k}_1+\Slash{k}_2) \frac{1+\gamma_5}{2}\right]\mr{tr}[T_\beta T_\gamma T_\alpha] \\
&+\mr{tr}\left[\frac{\Slash{p}-\Slash{k}_2+\Slash{b}}{(p-k_2+b)^2-i\epsilon}\gamma^\rho \frac{\Slash{p}+\Slash{b}}{(p+b)^2-i\epsilon}\gamma^\nu \frac{\Slash{p}+\Slash{k}_1+\Slash{b}}{(p+k_1+b)^2-i\epsilon}(\Slash{k}_1+\Slash{k}_2) \frac{1+\gamma_5}{2}\right]\mr{tr}[T_\gamma T_\beta T_\alpha]  \biggr\} \\
=&\frac{1}{(2\pi)^{12}}\int d^4k_1 d^4k_2e^{-i(k_1+k_2)\cdot x}e^{ik_1\cdot y}e^{ik_2\cdot z}\int d^4p \\
&\times \biggl\{\mr{tr}\left[\frac{\Slash{p}-\Slash{k}_1+\Slash{a}}{(p-k_1+a)^2-i\epsilon}\gamma^\nu \frac{\Slash{p}+\Slash{a}}{(p+a)^2-i\epsilon}\gamma^\rho \frac{\Slash{p}+\Slash{k}_2+\Slash{a}}{(p+k_2+a)^2-i\epsilon}(\Slash{p}+\Slash{k}_2+\Slash{a}) \frac{1+\gamma_5}{2}\right]\mr{tr}[T_\beta T_\gamma T_\alpha] \\
&-\mr{tr}\left[\frac{\Slash{p}-\Slash{k}_1+\Slash{a}}{(p-k_1+a)^2-i\epsilon}\gamma^\nu \frac{\Slash{p}+\Slash{a}}{(p+a)^2-i\epsilon}\gamma^\rho \frac{\Slash{p}+\Slash{k}_2+\Slash{a}}{(p+k_2+a)^2-i\epsilon}(\Slash{p}-\Slash{k}_1+\Slash{a}) \frac{1+\gamma_5}{2}\right]\mr{tr}[T_\beta T_\gamma T_\alpha] \\
&+\mr{tr}\left[\frac{\Slash{p}-\Slash{k}_2+\Slash{b}}{(p-k_2+b)^2-i\epsilon}\gamma^\rho \frac{\Slash{p}+\Slash{b}}{(p+b)^2-i\epsilon}\gamma^\nu \frac{\Slash{p}+\Slash{k}_1+\Slash{b}}{(p+k_1+b)^2-i\epsilon}(\Slash{p}+\Slash{k}_1+\Slash{b}) \frac{1+\gamma_5}{2}\right]\mr{tr}[T_\gamma T_\beta T_\alpha] \\
&-\mr{tr}\left[\frac{\Slash{p}-\Slash{k}_2+\Slash{b}}{(p-k_2+b)^2-i\epsilon}\gamma^\rho \frac{\Slash{p}+\Slash{b}}{(p+b)^2-i\epsilon}\gamma^\nu \frac{\Slash{p}+\Slash{k}_1+\Slash{b}}{(p+k_1+b)^2-i\epsilon}(\Slash{p}-\Slash{k}_2+\Slash{b}) \frac{1+\gamma_5}{2}\right]\mr{tr}[T_\gamma T_\beta T_\alpha]  \biggr\} \\
=&\frac{1}{(2\pi)^{12}}\int d^4k_1 d^4k_2e^{-i(k_1+k_2)\cdot x}e^{ik_1\cdot y}e^{ik_2\cdot z}\int d^4p \\
&\times \biggl\{\mr{tr}\left[\frac{\Slash{p}-\Slash{k}_1+\Slash{a}}{(p-k_1+a)^2-i\epsilon}\gamma^\nu \frac{\Slash{p}+\Slash{a}}{(p+a)^2-i\epsilon}\gamma^\rho \frac{1+\gamma_5}{2}\right]\mr{tr}[T_\beta T_\gamma T_\alpha] \\
&-\mr{tr}\left[\gamma^\nu \frac{\Slash{p}+\Slash{a}}{(p+a)^2-i\epsilon}\gamma^\rho \frac{\Slash{p}+\Slash{k}_2+\Slash{a}}{(p+k_2+a)^2-i\epsilon} \frac{1-\gamma_5}{2}\right]\mr{tr}[T_\beta T_\gamma T_\alpha] \\
&+\mr{tr}\left[\frac{\Slash{p}-\Slash{k}_2+\Slash{b}}{(p-k_2+b)^2-i\epsilon}\gamma^\rho \frac{\Slash{p}+\Slash{b}}{(p+b)^2-i\epsilon}\gamma^\nu \frac{1+\gamma_5}{2}\right]\mr{tr}[T_\gamma T_\beta T_\alpha] \\
&-\mr{tr}\left[\gamma^\rho \frac{\Slash{p}+\Slash{b}}{(p+b)^2-i\epsilon}\gamma^\nu \frac{\Slash{p}+\Slash{k}_1+\Slash{b}}{(p+k_1+b)^2-i\epsilon} \frac{1-\gamma_5}{2}\right]\mr{tr}[T_\gamma T_\beta T_\alpha]  \biggr\} \\
=&\frac{1}{(2\pi)^{12}}\int d^4k_1 d^4k_2e^{-i(k_1+k_2)\cdot x}e^{ik_1\cdot y}e^{ik_2\cdot z}\int d^4p \\
&\times \biggl\{\mr{tr}\left[\frac{\Slash{p}-\Slash{k}_1+\Slash{a}}{(p-k_1+a)^2-i\epsilon}\gamma^\nu \frac{\Slash{p}+\Slash{a}}{(p+a)^2-i\epsilon}\gamma^\rho \frac{1+\gamma_5}{2}\right]\mr{tr}[T_\beta T_\gamma T_\alpha] \\
&-\mr{tr}\left[ \frac{\Slash{p}+\Slash{a}}{(p+a)^2-i\epsilon}\gamma^\rho \frac{\Slash{p}+\Slash{k}_2+\Slash{a}}{(p+k_2+a)^2-i\epsilon} \gamma^\nu\frac{1+\gamma_5}{2}\right]\mr{tr}[T_\beta T_\gamma T_\alpha] \\
&+\mr{tr}\left[\frac{\Slash{p}-\Slash{k}_2+\Slash{b}}{(p-k_2+b)^2-i\epsilon}\gamma^\rho \frac{\Slash{p}+\Slash{b}}{(p+b)^2-i\epsilon}\gamma^\nu \frac{1+\gamma_5}{2}\right]\mr{tr}[T_\gamma T_\beta T_\alpha] \\
&-\mr{tr}\left[ \frac{\Slash{p}+\Slash{b}}{(p+b)^2-i\epsilon}\gamma^\nu \frac{\Slash{p}+\Slash{k}_1+\Slash{b}}{(p+k_1+b)^2-i\epsilon} \gamma^\rho\frac{1+\gamma_5}{2}\right]\mr{tr}[T_\gamma T_\beta T_\alpha]  \biggr\}
\end{align*}
となる.\par
この時点で,3点関数を群の指標$\alpha\beta\cdots$について対称な部分と反対称な部分に分けると便利だ.
\begin{align*}
&\mr{tr}[T_\alpha T_\beta]=N\delta_{\alpha\beta} \\
&D_{\alpha\beta\gamma}=\frac{1}{2}\mr{tr}[\{T_\beta , T_\gamma \}T_\alpha]
\end{align*}
という量を定義すると
\begin{align*}
\mr{tr}[T_\beta T_\gamma T_\alpha]-\mr{tr}[T_\gamma T_\beta T_\alpha]=& iC_{\beta\gamma\delta}\mr{tr}[T_\delta T_\alpha] \\
=&iC_{\beta\gamma\delta}N\delta_{\delta \alpha}=iNC_{\beta \gamma \alpha} \\
\mr{tr}[T_\beta T_\gamma T_\alpha]+\mr{tr}[T_\gamma T_\beta T_\alpha]=&\mr{tr}[\{T_\beta , T_\gamma \}T_\alpha]=2D_{\beta \gamma \alpha}
\end{align*}
前者は$C_{\alpha\beta\gamma}$が完全反対称であったことを思い出せば自明に完全反対称だ.$D_{\beta\gamma\alpha}$は$\beta\gamma$については反交換子より対称であることがわかり,またトレースの巡回性より
\begin{align*}
2D_{\alpha\gamma\beta}=&\mr{tr}[T_\alpha T_\gamma T_\beta]+\mr{tr}[T_\gamma T_\alpha T_\beta] \\
=&\mr{tr}[T_\gamma T_\beta T_\alpha]+\mr{tr}[T_\beta T_\gamma T_\alpha]=2D_{\beta\gamma\alpha}
\end{align*}
が示せるから
\begin{align*}
D_{\beta\alpha\gamma}=D_{\alpha\beta\gamma}=D_{\gamma\beta\alpha}
\end{align*}
となって完全対称であることがわかる.よってこれらの完全対称な量と完全反対称な量によって
\begin{align*}
&\mr{tr}[T_\beta T_\gamma T_\alpha]=D_{\alpha\beta\gamma}+\frac{1}{2}iNC_{\alpha\beta\gamma} \\
&\mr{tr}[T_\gamma T_\beta T_\alpha]=D_{\alpha\beta\gamma}-\frac{1}{2}iNC_{\alpha\beta\gamma}
\end{align*}
と書ける.群の指標に関して反対称な項は一般にゼロではない.しかしこれはなんら対称性の破れを表すものではない.行列要素(22.3.5)の発散を形式的に計算する際には時間順序積の$\theta$関数の時間微分から以下の寄与がある.(20.4節の計算を見直すと良い)
\begin{align*}
&\left[ \frac{\partial}{\partial x^\mu}\Gamma^{\mu\nu\rho}_{\alpha\beta\gamma}(x,y,z) \right]_{\mr{formal}}=\frac{\partial}{\partial x^\mu}\braket{T\{ j^\mu_\alpha(x),j^\nu_\beta(y),j^\rho_\gamma(z) \}}_{\VAC} \\
=&\frac{\partial}{\partial x^\mu}\langle j^\mu_\alpha(x)\theta(x^0-y^0)j^\nu_\beta(y)\theta(y^0-z^0)j^\rho_\gamma(z)+ j^\nu_\beta (y)\theta(y^0-x^0)j^\mu_\alpha(x)\theta(x^0-z^0)j^\rho_\gamma(z) \\
&\qquad+j^\nu_\beta(y)\theta(y^0-z^0)j^\rho_\gamma(z)\theta(z^0-x^0)j^\mu_\alpha(x) \\
&\qquad +j^\mu_\alpha(x)\theta(x^0-z^0)j^\rho_\gamma(z)\theta(z^0-y^0)j^\nu_\beta(y)+ j^\rho_\gamma (z)\theta(z^0-x^0)j^\mu_\alpha(x)\theta(x^0-y^0)j^\nu_\beta(y)  \\
&\qquad +j^\rho_\gamma(z)\theta(z^0-y^0)j^\nu_\beta(y)\theta(y^0-x^0)j^\mu_\alpha(x) \rangle_{\VAC} \\
=&\langle j^0_\alpha(x)\delta(x^0-y^0)j^\nu_\beta(y)\theta(y^0-z^0)j^\rho_\gamma(z) \\
&-j^\nu_\beta (y)\delta(y^0-x^0)j^0_\alpha(x)\theta(x^0-z^0)j^\rho_\gamma(z) +j^\nu_\beta (y)\theta(y^0-x^0)j^0_\alpha(x)\delta(x^0-z^0)j^\rho_\gamma(z) \\
&-j^\nu_\beta(y)\theta(y^0-z^0)j^\rho_\gamma(z)\delta(z^0-x^0)j^0_\alpha(x) \\
&+j^0_\alpha(x)\delta(x^0-z^0)j^\rho_\gamma(z)\theta(z^0-y^0)j^\nu_\beta(y) \\
&-j^\rho_\gamma (z)\delta(z^0-x^0)j^0_\alpha(x)\theta(x^0-y^0)j^\nu_\beta(y)+j^\rho_\gamma (z)\theta(z^0-x^0)j^0_\alpha(x)\delta(x^0-y^0)j^\nu_\beta(y) \\
&-j^\rho_\gamma(z)\theta(z^0-y^0)j^\nu_\beta(y)\delta(y^0-x^0)j^0_\alpha(x) \rangle_{\VAC} \\
=&\langle \delta(x^0-y^0)[j^0_\alpha (\mathbf{x},y^0),j^\nu_\beta(\mathbf{y},y^0)]\theta(y^0-z^0)j^\rho_\gamma(z) +\delta(x^0-z^0)j^\nu_\beta(y) \theta(y^0-z^0)[j^0_\alpha(\mathbf{x},z^0),j^\rho_\gamma(\mathbf{z},z^0)] \\
&+\delta(x^0-z^0)[j^0_\alpha (\mathbf{x},z^0),j^\rho_\gamma(\mathbf{z},z^0)]\theta(z^0-y^0)j^\nu_\beta(y)+\delta(x^0-y^0)j^\rho_\gamma(z) \theta(z^0-y^0)[j^0_\alpha(\mathbf{x},y^0),j^\nu_\beta(\mathbf{y},y^0)]\rangle_{\VAC}
\end{align*}
ここで(22.3.6)より,$\bar{\chi}=\chi^\dagger \gamma_4,\gamma_4 =i\gamma^0$を用いれば
\begin{align*}
[j^0_\alpha (\mathbf{x},y^0),j^\nu_\beta(\mathbf{y},y^0)]=&(-i)^2\bar{\chi}(\mathbf{x})T_\alpha \gamma^0 \chi(\mathbf{x})\bar{\chi}(\mathbf{y})T_\beta \gamma^\nu \chi(\mathbf{y}) \\
&-(-i)^2\bar{\chi}(\mathbf{y})T_\beta \gamma^\nu \chi(\mathbf{y})\bar{\chi}(\mathbf{x})T_\alpha \gamma^0 \chi(\mathbf{x}) \\
=&-\bar{\chi}(\mathbf{x})T_\alpha \gamma^0 \chi(\mathbf{x})\chi^\dagger(\mathbf{y})i\gamma^0T_\beta \gamma^\nu \chi(\mathbf{y}) \\
&+\bar{\chi}(\mathbf{y})T_\beta \gamma^\nu \chi(\mathbf{y})\chi^\dagger(\mathbf{x})(-i)T_\alpha \chi(\mathbf{x}) \\
=&i\delta^3(\mathbf{x}-\mathbf{y})\left\{ \bar{\chi}(\mathbf{x})T_\alpha T_\beta \gamma^\nu \chi(\mathbf{y}) -\bar{\chi}(\mathbf{y}) T_\beta T_\alpha \gamma^\nu \chi(\mathbf{x}) \right\} \\
=&i\delta^3(\mathbf{x}-\mathbf{y})\bar{\chi}(\mathbf{y})[T_\alpha,T_\beta]\gamma^\nu \chi(\mathbf{y}) \\
=&-C_{\alpha\beta\delta}\delta^3(\mathbf{x}-\mathbf{y})\bar{\chi}(\mathbf{y})T_\delta\gamma^\nu \chi(\mathbf{y})=-i\delta^3(\mathbf{x}-\mathbf{y})C_{\alpha\beta\delta}j^\nu_\delta(y)
\end{align*}
が示せる.(ここで$\chi(\mathbf{y},y^0)$等における$y^0$は簡単のため省略した.)したがって
\begin{align*}
&\left[ \frac{\partial}{\partial x^\mu}\Gamma^{\mu\nu\rho}_{\alpha\beta\gamma}(x,y,z) \right]_{\mr{formal}} \\
=&\langle -iC_{\alpha\beta\delta} \delta^4(x-y)j^\nu_\delta(y) \theta(y^0-z^0) j^\rho_\gamma(z) -iC_{\alpha\gamma\delta}\delta^4(x-z)j^\nu_\beta(y) \theta(y^0-z^0)j^\rho_\delta(z) \\
&-iC_{\alpha\beta\delta}\delta^4(x-z)j^\rho_\delta(z) \theta(z^0-y^0)j^\nu_\beta(y)-iC_{\alpha\gamma\delta}\delta^4(x-y)j^\rho_\gamma(z) \theta(z^0-y^0)j^\nu_\delta(y)\rangle_{\VAC} \\
=&-iC_{\alpha\beta\delta}\delta^4(x-y)\braket{T\{j^\nu_\delta(y),j^\rho_\gamma(z)\}}_{\VAC}-iC_{\alpha\gamma\delta}\delta^4(x-z)\braket{T\{j^\nu_\beta(y),j^\rho_\delta(z) \}}_{\VAC}
\end{align*}
となる.(22.3.11)の反対称項がこれを再現することを示そう.(22.3.11)は
\begin{align*}
&\frac{\partial}{\partial x^\mu}\Gamma^{\mu\nu\rho}_{\alpha\beta\gamma}(x,y,z) \\
=&\frac{1}{(2\pi)^{12}}\int d^4k_1 d^4k_2e^{-i(k_1+k_2)\cdot x}e^{ik_1\cdot y}e^{ik_2\cdot z}\int d^4p \\
&\times \biggl\{ \left[D_{\alpha\beta\gamma}+\frac{1}{2}iNC_{\alpha\beta\gamma}\right]\mr{tr}\left[\frac{\Slash{p}-\Slash{k}_1+\Slash{a}}{(p-k_1+a)^2-i\epsilon}\gamma^\nu \frac{\Slash{p}+\Slash{a}}{(p+a)^2-i\epsilon}\gamma^\rho \frac{1+\gamma_5}{2}\right] \\
&-\left[D_{\alpha\beta\gamma}+\frac{1}{2}iNC_{\alpha\beta\gamma}\right]\mr{tr}\left[ \frac{\Slash{p}+\Slash{a}}{(p+a)^2-i\epsilon}\gamma^\rho \frac{\Slash{p}+\Slash{k}_2+\Slash{a}}{(p+k_2+a)^2-i\epsilon} \gamma^\nu\frac{1+\gamma_5}{2}\right] \\
&+\left[D_{\alpha\beta\gamma}-\frac{1}{2}iNC_{\alpha\beta\gamma}\right]\mr{tr}\left[\frac{\Slash{p}-\Slash{k}_2+\Slash{b}}{(p-k_2+b)^2-i\epsilon}\gamma^\rho \frac{\Slash{p}+\Slash{b}}{(p+b)^2-i\epsilon}\gamma^\nu \frac{1+\gamma_5}{2}\right] \\
&-\left[D_{\alpha\beta\gamma}-\frac{1}{2}iNC_{\alpha\beta\gamma}\right]\mr{tr}\left[ \frac{\Slash{p}+\Slash{b}}{(p+b)^2-i\epsilon}\gamma^\nu \frac{\Slash{p}+\Slash{k}_1+\Slash{b}}{(p+k_1+b)^2-i\epsilon} \gamma^\rho\frac{1+\gamma_5}{2}\right]  \biggr\} \\
=&\frac{1}{(2\pi)^{12}}D_{\alpha\beta\gamma}\int d^4k_1 d^4k_2e^{-i(k_1+k_2)\cdot x}e^{ik_1\cdot y}e^{ik_2\cdot z} \\
&\times \int d^4p \biggl\{ \mr{tr}\left[\frac{\Slash{p}-\Slash{k}_1+\Slash{a}}{(p-k_1+a)^2-i\epsilon}\gamma^\nu \frac{\Slash{p}+\Slash{a}}{(p+a)^2-i\epsilon}\gamma^\rho \frac{1+\gamma_5}{2}\right]  \\
&\qquad \quad -\mr{tr}\left[ \frac{\Slash{p}+\Slash{a}}{(p+a)^2-i\epsilon}\gamma^\rho \frac{\Slash{p}+\Slash{k}_2+\Slash{a}}{(p+k_2+a)^2-i\epsilon} \gamma^\nu\frac{1+\gamma_5}{2}\right] \\
&\qquad \quad +\mr{tr}\left[\frac{\Slash{p}-\Slash{k}_2+\Slash{b}}{(p-k_2+b)^2-i\epsilon}\gamma^\rho \frac{\Slash{p}+\Slash{b}}{(p+b)^2-i\epsilon}\gamma^\nu \frac{1+\gamma_5}{2}\right] \\
&\qquad\quad -\mr{tr}\left[ \frac{\Slash{p}+\Slash{b}}{(p+b)^2-i\epsilon}\gamma^\nu \frac{\Slash{p}+\Slash{k}_1+\Slash{b}}{(p+k_1+b)^2-i\epsilon} \gamma^\rho\frac{1+\gamma_5}{2}\right]  \biggr\} \\
&+\frac{i}{2(2\pi)^{12}}NC_{\alpha\beta\gamma}\int d^4k_1 d^4k_2e^{-i(k_1+k_2)\cdot x}e^{ik_1\cdot y}e^{ik_2\cdot z} \\
&\times \int d^4p\biggl\{ \mr{tr}\left[\frac{\Slash{p}-\Slash{k}_1+\Slash{a}}{(p-k_1+a)^2-i\epsilon}\gamma^\nu \frac{\Slash{p}+\Slash{a}}{(p+a)^2-i\epsilon}\gamma^\rho \frac{1+\gamma_5}{2}\right]  \\
&\qquad \quad -\mr{tr}\left[ \frac{\Slash{p}+\Slash{a}}{(p+a)^2-i\epsilon}\gamma^\rho \frac{\Slash{p}+\Slash{k}_2+\Slash{a}}{(p+k_2+a)^2-i\epsilon} \gamma^\nu\frac{1+\gamma_5}{2}\right] \\
&\qquad \quad -\mr{tr}\left[\frac{\Slash{p}-\Slash{k}_2+\Slash{b}}{(p-k_2+b)^2-i\epsilon}\gamma^\rho \frac{\Slash{p}+\Slash{b}}{(p+b)^2-i\epsilon}\gamma^\nu \frac{1+\gamma_5}{2}\right] \\
&\qquad\quad +\mr{tr}\left[ \frac{\Slash{p}+\Slash{b}}{(p+b)^2-i\epsilon}\gamma^\nu \frac{\Slash{p}+\Slash{k}_1+\Slash{b}}{(p+k_1+b)^2-i\epsilon} \gamma^\rho\frac{1+\gamma_5}{2}\right]  \biggr\}
\end{align*}
と分解される.ここで$D_{\alpha\beta\gamma}$に比例する項を
\begin{align*}
&\left[ \frac{\partial}{\partial x^\mu}\Gamma^{\mu\nu\rho}_{\alpha\beta\gamma}(x,y,z) \right]_{\mr{anom}} \\
=&\frac{1}{(2\pi)^{12}}D_{\alpha\beta\gamma}\int d^4k_1 d^4k_2e^{-i(k_1+k_2)\cdot x}e^{ik_1\cdot y}e^{ik_2\cdot z} \\
&\times \int d^4p \biggl\{ \mr{tr}\left[\frac{\Slash{p}-\Slash{k}_1+\Slash{a}}{(p-k_1+a)^2-i\epsilon}\gamma^\nu \frac{\Slash{p}+\Slash{a}}{(p+a)^2-i\epsilon}\gamma^\rho \frac{1+\gamma_5}{2}\right]  \\
&\qquad \quad -\mr{tr}\left[ \frac{\Slash{p}+\Slash{a}}{(p+a)^2-i\epsilon}\gamma^\rho \frac{\Slash{p}+\Slash{k}_2+\Slash{a}}{(p+k_2+a)^2-i\epsilon} \gamma^\nu\frac{1+\gamma_5}{2}\right] \\
&\qquad \quad +\mr{tr}\left[\frac{\Slash{p}-\Slash{k}_2+\Slash{b}}{(p-k_2+b)^2-i\epsilon}\gamma^\rho \frac{\Slash{p}+\Slash{b}}{(p+b)^2-i\epsilon}\gamma^\nu \frac{1+\gamma_5}{2}\right] \\
&\qquad\quad -\mr{tr}\left[ \frac{\Slash{p}+\Slash{b}}{(p+b)^2-i\epsilon}\gamma^\nu \frac{\Slash{p}+\Slash{k}_1+\Slash{b}}{(p+k_1+b)^2-i\epsilon} \gamma^\rho\frac{1+\gamma_5}{2}\right]  \biggr\}
\end{align*}
として,また$C_{\alpha\beta\gamma}$に比例する項を
\begin{align*}
&\left[ \frac{\partial}{\partial x^\mu}\Gamma^{\mu\nu\rho}_{\alpha\beta\gamma}(x,y,z) \right]'_{\mr{formal}} \\
=&\frac{i}{2(2\pi)^{12}}NC_{\alpha\beta\gamma}\int d^4k_1 d^4k_2e^{-i(k_1+k_2)\cdot x}e^{ik_1\cdot y}e^{ik_2\cdot z} \\
&\times \int d^4p\biggl\{ \mr{tr}\left[\frac{\Slash{p}-\Slash{k}_1+\Slash{a}}{(p-k_1+a)^2-i\epsilon}\gamma^\nu \frac{\Slash{p}+\Slash{a}}{(p+a)^2-i\epsilon}\gamma^\rho \frac{1+\gamma_5}{2}\right]  \\
&\qquad \quad -\mr{tr}\left[ \frac{\Slash{p}+\Slash{a}}{(p+a)^2-i\epsilon}\gamma^\rho \frac{\Slash{p}+\Slash{k}_2+\Slash{a}}{(p+k_2+a)^2-i\epsilon} \gamma^\nu\frac{1+\gamma_5}{2}\right] \\
&\qquad \quad -\mr{tr}\left[\frac{\Slash{p}-\Slash{k}_2+\Slash{b}}{(p-k_2+b)^2-i\epsilon}\gamma^\rho \frac{\Slash{p}+\Slash{b}}{(p+b)^2-i\epsilon}\gamma^\nu \frac{1+\gamma_5}{2}\right] \\
&\qquad\quad +\mr{tr}\left[ \frac{\Slash{p}+\Slash{b}}{(p+b)^2-i\epsilon}\gamma^\nu \frac{\Slash{p}+\Slash{k}_1+\Slash{b}}{(p+k_1+b)^2-i\epsilon} \gamma^\rho\frac{1+\gamma_5}{2}\right]  \biggr\}
\end{align*}
とおく.一方,(22.3.13)からは次のファインマンダイアグラムから生じる項がある.
\begin{figure}[H]
  \centering
  \begin{tikzpicture}[decoration={markings, 
    mark= at position 1.46cm with {\arrow[line width=0.5mm]{Stealth}}}]
    \draw[thick,decorate,decoration={snake,amplitude=.5mm,segment length=4mm,post length=1mm}](0,0)--(2.5,0);
    \draw[thick,decorate,decoration={snake,amplitude=.5mm,segment length=4mm,post length=1mm}](6,0)--(3.5,0);
    \filldraw[thick,fill=white](3,0)circle[x radius=1,y radius=0.7];
    \draw[thick,postaction={decorate}](2,0)arc(180:80:1cm and 0.7cm);
    \draw[thick,postaction={decorate}](4,0)arc(0:-100:1cm and 0.7cm);
   
    
  \end{tikzpicture}
  
\end{figure}
したがって,
\begin{align*}
&\left[ \frac{\partial}{\partial x^\mu}\Gamma^{\mu\nu\rho}_{\alpha\beta\gamma}(x,y,z) \right]_{\mr{formal}} \\
=&-iC_{\alpha\beta\delta}\delta^4(x-y)\left\{-(-i)^2\mr{Tr}\left[S(y-z)T_\delta \gamma^\nu S(z-y)T_\gamma \gamma^\rho P_L\right] \right\} \\
&-iC_{\alpha\gamma\delta}\delta^4(x-z)\left\{-(-i)^2\mr{Tr}\left[S(y-z)T_\beta \gamma^\nu S(z-y) T_\delta \gamma^\rho P_L \right]\right\} \\
=&-iC_{\alpha\beta\delta}\delta^4(x-y)\int d^4q_1 d^4q_2 e^{iq_1\cdot( y-z)}e^{iq_2\cdot (z-y)}\mr{Tr}\left[\frac{-i}{(2\pi)^4}\frac{-i\Slash{q_1}}{q^2_1-i\epsilon}T_\delta \gamma^\nu \frac{-i}{(2\pi)^4}\frac{-i\Slash{q_2}}{q^2_2-i\epsilon}T_\gamma \gamma^\rho \frac{1+\gamma_5}{2}\right] \\
&-iC_{\alpha\gamma\delta}\delta^4(x-z)\int d^4q_1 d^4q_2 e^{iq_1\cdot (y-z)}e^{iq_2\cdot (z-y)} \mr{Tr}\left[\frac{-i}{(2\pi)^4}\frac{-i\Slash{q_1}}{q^2_1-i\epsilon}T_\beta \gamma^\nu \frac{-i}{(2\pi)^4}\frac{-i\Slash{q_2}}{q^2_2-i\epsilon}T_\delta \gamma^\rho \frac{1+\gamma_5}{2}\right] \\
=&-iC_{\alpha\beta\delta}\delta^4(x-y)\frac{1}{(2\pi)^8}\int d^4q_1 d^4q_2 e^{i(q_1-q_2)\cdot y}e^{i(q_2-q_1)\cdot z}\mr{tr}\left[\frac{\Slash{q_1}}{q^2_1-i\epsilon}\gamma^\nu \frac{\Slash{q_2}}{q^2_2-i\epsilon}\gamma^\rho \frac{1+\gamma_5}{2} \right]\mr{tr}[T_\delta T_\gamma] \\
&-iC_{\alpha\gamma\delta}\delta^4(x-z)\frac{1}{(2\pi)^8}\int d^4q_1 d^4q_2 e^{i(q_1-q_2)\cdot y}e^{i(q_2-q_1)\cdot z}\mr{tr}\left[\frac{\Slash{q_1}}{q^2_1-i\epsilon}\gamma^\nu \frac{\Slash{q_2}}{q^2_2-i\epsilon}\gamma^\rho \frac{1+\gamma_5}{2} \right]\mr{tr}[T_\beta T_\delta] \\
=&-iNC_{\alpha\beta\gamma}\delta^4(x-y)\frac{1}{(2\pi)^8}\int d^4q_1 d^4q_2 e^{i(q_1-q_2)\cdot y}e^{i(q_2-q_1)\cdot z}\mr{tr}\left[\frac{\Slash{q_1}}{q^2_1-i\epsilon}\gamma^\nu \frac{\Slash{q_2}}{q^2_2-i\epsilon}\gamma^\rho \frac{1+\gamma_5}{2} \right] \\
&+iNC_{\alpha\beta\gamma}\delta^4(x-z)\frac{1}{(2\pi)^8}\int d^4q_1 d^4q_2 e^{i(q_1-q_2)\cdot y}e^{i(q_2-q_1)\cdot z}\mr{tr}\left[\frac{\Slash{q_1}}{q^2_1-i\epsilon}\gamma^\nu \frac{\Slash{q_2}}{q^2_2-i\epsilon}\gamma^\rho \frac{1+\gamma_5}{2} \right]
\end{align*}
ここで$\mr{tr}[T_\alpha T_\beta]=N\delta_{\alpha\beta}$を用いた.第一項目は
\begin{align*}
&-iNC_{\alpha\beta\gamma}\delta^4(x-y)\frac{1}{(2\pi)^8}\int d^4q_1 d^4q_2 e^{i(q_1-q_2)\cdot y}e^{i(q_2-q_1)\cdot z}\mr{tr}\left[\frac{\Slash{q_1}}{q^2_1-i\epsilon}\gamma^\nu \frac{\Slash{q_2}}{q^2_2-i\epsilon}\gamma^\rho \frac{1+\gamma_5}{2} \right] \\
=&-iNC_{\alpha\beta\gamma}\delta^4(x-y)\frac{1}{2(2\pi)^8}\int d^4q_1 d^4q_2 e^{i(q_1-q_2)\cdot y}e^{i(q_2-q_1)\cdot z}\mr{tr}\left[\frac{\Slash{q_1}}{q^2_1-i\epsilon}\gamma^\nu \frac{\Slash{q_2}}{q^2_2-i\epsilon}\gamma^\rho \frac{1+\gamma_5}{2} \right] \\
&-iNC_{\alpha\beta\gamma}\delta^4(x-y)\frac{1}{2(2\pi)^8}\int d^4q_1 d^4q_2 e^{i(q_1-q_2)\cdot y}e^{i(q_2-q_1)\cdot z}\mr{tr}\left[\frac{\Slash{q_1}}{q^2_1-i\epsilon}\gamma^\nu \frac{\Slash{q_2}}{q^2_2-i\epsilon}\gamma^\rho \frac{1+\gamma_5}{2} \right]  \\
=&-iNC_{\alpha\beta\gamma}\delta^4(x-y)\frac{1}{2(2\pi)^8}\int d^4k_2 d^4p e^{-ik_2\cdot y}e^{ik_2\cdot z} \mr{tr}\left[\frac{\Slash{p}+\Slash{a}}{(p+a)^2-i\epsilon}\gamma^\nu \frac{\Slash{p}+\Slash{k}_2+\Slash{a}}{(p+k_2+a)^2-i\epsilon}\gamma^\rho \frac{1+\gamma_5}{2} \right] \\
&-iNC_{\alpha\beta\gamma}\delta^4(x-y)\frac{1}{2(2\pi)^8}\int d^4k_2 d^4p e^{-ik_2\cdot y}e^{ik_2\cdot z}\mr{tr}\left[\frac{\Slash{p}-\Slash{k}_2+\Slash{b}}{(p-k_2+b)^2-i\epsilon}\gamma^\nu \frac{\Slash{p}+\Slash{b}}{(p+b)^2-i\epsilon}\gamma^\rho \frac{1+\gamma_5}{2} \right] \\
=&-iNC_{\alpha\beta\gamma}\frac{1}{2(2\pi)^{12}}\int d^4k_1 e^{-ik_1\cdot (x-y)}\int d^4k_2 d^4p e^{-ik_2\cdot x}e^{ik_2\cdot z}\qquad \because \delta^4(x-y)e^{ikx}=\delta^4(x-y)e^{iky} \\
&\times \mr{tr}\left[\frac{\Slash{p}+\Slash{a}}{(p+a)^2-i\epsilon}\gamma^\nu \frac{\Slash{p}+\Slash{k}_2+\Slash{a}}{(p+k_2+a)^2-i\epsilon}\gamma^\rho \frac{1+\gamma_5}{2} \right] \\
&-iNC_{\alpha\beta\gamma}\frac{1}{2(2\pi)^{12}}\int d^4k_1 e^{-ik_1\cdot (x-y)}\int d^4k_2 d^4p e^{-ik_2\cdot y}e^{ik_2\cdot z}\\
&\times \mr{tr}\left[\frac{\Slash{p}-\Slash{k}_2+\Slash{b}}{(p-k_2+b)^2-i\epsilon}\gamma^\nu \frac{\Slash{p}+\Slash{b}}{(p+b)^2-i\epsilon}\gamma^\rho \frac{1+\gamma_5}{2} \right] \\
=&-iNC_{\alpha\beta\gamma}\frac{1}{2(2\pi)^{12}}\int d^4k_1 d^4k_2\, e^{-i(k_1+k_2)\cdot x}e^{ik_1\cdot y}e^{ik_2\cdot z}\int d^4p \\
&\times \mr{tr}\left[\frac{\Slash{p}+\Slash{a}}{(p+a)^2-i\epsilon}\gamma^\nu \frac{\Slash{p}+\Slash{k}_2+\Slash{a}}{(p+k_2+a)^2-i\epsilon}\gamma^\rho \frac{1+\gamma_5}{2} \right] \\
&-iNC_{\alpha\beta\gamma}\frac{1}{2(2\pi)^{12}}\int d^4k_1 d^4k_2\, e^{-i(k_1+k_2)\cdot x}e^{ik_1\cdot y}e^{ik_2\cdot z}\int d^4p \\
&\times\mr{tr}\left[\frac{\Slash{p}-\Slash{k}_2+\Slash{b}}{(p-k_2+b)^2-i\epsilon}\gamma^\nu \frac{\Slash{p}+\Slash{b}}{(p+b)^2-i\epsilon}\gamma^\rho \frac{1+\gamma_5}{2} \right]
\end{align*}
となり,第二項目も同様にして
\begin{align*}
&+iNC_{\alpha\beta\gamma}\delta^4(x-z)\frac{1}{(2\pi)^8}\int d^4q_1 d^4q_2 e^{i(q_1-q_2)\cdot y}e^{i(q_2-q_1)\cdot z}\mr{tr}\left[\frac{\Slash{q_1}}{q^2_1-i\epsilon}\gamma^\nu \frac{\Slash{q_2}}{q^2_2-i\epsilon}\gamma^\rho \frac{1+\gamma_5}{2} \right] \\
=&+iNC_{\alpha\beta\gamma}\delta^4(x-z)\frac{1}{2(2\pi)^8}\int d^4q_1 d^4q_2 e^{i(q_1-q_2)\cdot y}e^{i(q_2-q_1)\cdot z}\mr{tr}\left[\frac{\Slash{q_1}}{q^2_1-i\epsilon}\gamma^\nu \frac{\Slash{q_2}}{q^2_2-i\epsilon}\gamma^\rho \frac{1+\gamma_5}{2} \right] \\
&+iNC_{\alpha\beta\gamma}\delta^4(x-z)\frac{1}{2(2\pi)^8}\int d^4q_1 d^4q_2 e^{i(q_1-q_2)\cdot y}e^{i(q_2-q_1)\cdot z}\mr{tr}\left[\frac{\Slash{q_1}}{q^2_1-i\epsilon}\gamma^\nu \frac{\Slash{q_2}}{q^2_2-i\epsilon}\gamma^\rho \frac{1+\gamma_5}{2} \right] \\
=&+iNC_{\alpha\beta\gamma}\delta^4(x-z)\frac{1}{2(2\pi)^8}\int d^4q_1 d^4q_2 e^{-i(q_1-q_2)\cdot y}e^{-i(q_2-q_1)\cdot z}\mr{tr}\left[\frac{-\Slash{q_1}}{q^2_1-i\epsilon}\gamma^\nu \frac{-\Slash{q_2}}{q^2_2-i\epsilon}\gamma^\rho \frac{1+\gamma_5}{2} \right] \\
&+iNC_{\alpha\beta\gamma}\delta^4(x-z)\frac{1}{2(2\pi)^8}\int d^4q_1 d^4q_2 e^{-i(q_1-q_2)\cdot y}e^{-i(q_2-q_1)\cdot z}\mr{tr}\left[\frac{-\Slash{q_1}}{q^2_1-i\epsilon}\gamma^\nu \frac{-\Slash{q_2}}{q^2_2-i\epsilon}\gamma^\rho \frac{1+\gamma_5}{2} \right] \\
=&+iNC_{\alpha\beta\gamma}\delta^4(x-z)\frac{1}{2(2\pi)^8}\int d^4k_1 d^4p e^{ik_1\cdot y}e^{-ik_1 \cdot z}\mr{tr}\left[\frac{\Slash{p}-\Slash{k}_1+\Slash{a}}{(p-k_1+a)^2-i\epsilon}\gamma^\nu \frac{\Slash{p}+\Slash{a}}{(p+a)^2-i\epsilon}\gamma^\rho \frac{1+\gamma_5}{2} \right] \\
&+iNC_{\alpha\beta\gamma}\delta^4(x-z)\frac{1}{2(2\pi)^8}\int d^4k_1 d^4p e^{ik_1 \cdot y}e^{-ik_1\cdot z}\mr{tr}\left[\frac{\Slash{p}+\Slash{b}}{(p+b)^2-i\epsilon}\gamma^\nu \frac{\Slash{p}+\Slash{k}_1+\Slash{b}}{(p+k_1+b)^2-i\epsilon}\gamma^\rho \frac{1+\gamma_5}{2} \right] \\
=&+iNC_{\alpha\beta\gamma}\frac{1}{2(2\pi)^{12}}\int d^4k_2 e^{-ik_2\cdot(x-z)} \int d^4k_2 d^4p e^{ik_1\cdot y}e^{-ik_1 \cdot x} \qquad \because \delta^4(x-z)e^{ikz}=\delta^4(x-z)e^{ikx}\\
&\times \mr{tr}\left[\frac{\Slash{p}-\Slash{k}_1+\Slash{a}}{(p-k_1+a)^2-i\epsilon}\gamma^\nu \frac{\Slash{p}+\Slash{a}}{(p+a)^2-i\epsilon}\gamma^\rho \frac{1+\gamma_5}{2} \right] \\
&+iNC_{\alpha\beta\gamma}\frac{1}{2(2\pi)^{12}}\int d^4k_2 e^{-ik_2\cdot(x-z)}\int d^4k_1 d^4p e^{ik_1 \cdot y}e^{-ik_1\cdot x} \\
&\times \mr{tr}\left[\frac{\Slash{p}+\Slash{b}}{(p+b)^2-i\epsilon}\gamma^\nu \frac{\Slash{p}+\Slash{k}_1+\Slash{b}}{(p+k_1+b)^2-i\epsilon}\gamma^\rho \frac{1+\gamma_5}{2} \right] \\
=&+iNC_{\alpha\beta\gamma}\frac{1}{2(2\pi)^{12}}\int d^4k_1d^4k_2 e^{-i(k_1+k_2)\cdot x}  e^{ik_1\cdot y}e^{ik_2 \cdot z} \int d^4p \\
&\times \mr{tr}\left[\frac{\Slash{p}-\Slash{k}_1+\Slash{a}}{(p-k_1+a)^2-i\epsilon}\gamma^\nu \frac{\Slash{p}+\Slash{a}}{(p+a)^2-i\epsilon}\gamma^\rho \frac{1+\gamma_5}{2} \right] \\
&+iNC_{\alpha\beta\gamma}\frac{1}{2(2\pi)^{12}}\int d^4k_1d^4k_2 e^{-i(k_1+k_2)\cdot x}  e^{ik_1\cdot y}e^{ik_2 \cdot z} \int d^4p \\
&\times \mr{tr}\left[\frac{\Slash{p}+\Slash{b}}{(p+b)^2-i\epsilon}\gamma^\nu \frac{\Slash{p}+\Slash{k}_1+\Slash{b}}{(p+k_1+b)^2-i\epsilon}\gamma^\rho \frac{1+\gamma_5}{2} \right]
\end{align*}
まとめると
\begin{align*}
&\left[ \frac{\partial}{\partial x^\mu}\Gamma^{\mu\nu\rho}_{\alpha\beta\gamma}(x,y,z) \right]_{\mr{formal}} \\
=&\frac{i}{2(2\pi)^{12}}NC_{\alpha\beta\gamma}\int d^4k_1 d^4k_2e^{-i(k_1+k_2)\cdot x}e^{ik_1\cdot y}e^{ik_2\cdot z} \\
&\times \int d^4p\biggl\{ \mr{tr}\left[\frac{\Slash{p}-\Slash{k}_1+\Slash{a}}{(p-k_1+a)^2-i\epsilon}\gamma^\nu \frac{\Slash{p}+\Slash{a}}{(p+a)^2-i\epsilon}\gamma^\rho \frac{1+\gamma_5}{2}\right]  \\
&\qquad \quad -\mr{tr}\left[ \frac{\Slash{p}+\Slash{a}}{(p+a)^2-i\epsilon}\gamma^\rho \frac{\Slash{p}+\Slash{k}_2+\Slash{a}}{(p+k_2+a)^2-i\epsilon} \gamma^\nu\frac{1+\gamma_5}{2}\right] \\
&\qquad \quad -\mr{tr}\left[\frac{\Slash{p}-\Slash{k}_2+\Slash{b}}{(p-k_2+b)^2-i\epsilon}\gamma^\rho \frac{\Slash{p}+\Slash{b}}{(p+b)^2-i\epsilon}\gamma^\nu \frac{1+\gamma_5}{2}\right] \\
&\qquad\quad +\mr{tr}\left[ \frac{\Slash{p}+\Slash{b}}{(p+b)^2-i\epsilon}\gamma^\nu \frac{\Slash{p}+\Slash{k}_1+\Slash{b}}{(p+k_1+b)^2-i\epsilon} \gamma^\rho\frac{1+\gamma_5}{2}\right]  \biggr\} \\
=&\left[ \frac{\partial}{\partial x^\mu}\Gamma^{\mu\nu\rho}_{\alpha\beta\gamma}(x,y,z) \right]'_{\mr{formal}}
\end{align*}
となって,(22.3.11)の反対称項が(22.3.13)を再現することが示せた.したがってアノマリーは(22.3.11)の対称部分
\begin{align*}
&\left[ \frac{\partial}{\partial x^\mu}\Gamma^{\mu\nu\rho}_{\alpha\beta\gamma}(x,y,z) \right]_{\mr{anom}} \\
&=\frac{1}{(2\pi)^{12}}D_{\alpha\beta\gamma}\int d^4k_1 d^4k_2e^{-i(k_1+k_2)\cdot x}e^{ik_1\cdot y}e^{ik_2\cdot z} \\
&\times \int d^4p \biggl\{ \mr{tr}\left[\frac{\Slash{p}-\Slash{k}_1+\Slash{a}}{(p-k_1+a)^2-i\epsilon}\gamma^\nu \frac{\Slash{p}+\Slash{a}}{(p+a)^2-i\epsilon}\gamma^\rho \frac{1+\gamma_5}{2}\right]  \\
&\qquad \quad -\mr{tr}\left[ \frac{\Slash{p}+\Slash{a}}{(p+a)^2-i\epsilon}\gamma^\rho \frac{\Slash{p}+\Slash{k}_2+\Slash{a}}{(p+k_2+a)^2-i\epsilon} \gamma^\nu\frac{1+\gamma_5}{2}\right] \\
&\qquad \quad +\mr{tr}\left[\frac{\Slash{p}-\Slash{k}_2+\Slash{b}}{(p-k_2+b)^2-i\epsilon}\gamma^\rho \frac{\Slash{p}+\Slash{b}}{(p+b)^2-i\epsilon}\gamma^\nu \frac{1+\gamma_5}{2}\right] \\
&\qquad\quad -\mr{tr}\left[ \frac{\Slash{p}+\Slash{b}}{(p+b)^2-i\epsilon}\gamma^\nu \frac{\Slash{p}+\Slash{k}_1+\Slash{b}}{(p+k_1+b)^2-i\epsilon} \gamma^\rho\frac{1+\gamma_5}{2}\right]  \biggr\} \\
&=\frac{1}{(2\pi)^{12}}D_{\alpha\beta\gamma}\int d^4k_1 d^4k_2e^{-i(k_1+k_2)\cdot x}e^{ik_1\cdot y}e^{ik_2\cdot z} \\
&\times \int d^4p \biggl\{ \mr{tr}\left[\gamma^\kappa \gamma^\nu \gamma^\lambda \gamma^\rho \frac{1+\gamma_5}{2}\right]\left[\frac{p_\kappa-k_{1\kappa}+a_\kappa}{(p-k_1+a)^2-i\epsilon}\frac{p_\lambda+a_\lambda}{(p+a)^2-i\epsilon} \right]  \\
&\qquad \qquad -\mr{tr}\left[\gamma^\kappa \gamma^\rho \gamma^\lambda \gamma^\nu\frac{1+\gamma_5}{2}\right]\left[ \frac{p_\kappa+a_\kappa}{(p+a)^2-i\epsilon} \frac{p_\lambda+k_{2\lambda}+a_\lambda }{(p+k_2+a)^2-i\epsilon}\right] \\
&\qquad \qquad +\mr{tr}\left[\gamma^\kappa \gamma^\rho \gamma^\lambda \gamma^\nu \frac{1+\gamma_5}{2}\right]\left[\frac{p_\kappa-k_{2\kappa}+b_\kappa}{(p-k_2+b)^2-i\epsilon} \frac{p_\lambda+b_\lambda }{(p+b)^2-i\epsilon}\right] \\
&\qquad\qquad -\mr{tr}\left[\gamma^\kappa \gamma^\nu \gamma^\lambda \gamma^\rho \frac{1+\gamma_5}{2}\right]\left[ \frac{p_\kappa+b_\kappa}{(p+b)^2-i\epsilon} \frac{p_\lambda+k_{1\lambda}+b_\lambda}{(p+k_1+b)^2-i\epsilon} \right]  \biggr\} \\
&=\frac{1}{(2\pi)^{12}}D_{\alpha\beta\gamma}\int d^4k_1 d^4k_2e^{-i(k_1+k_2)\cdot x}e^{ik_1\cdot y}e^{ik_2\cdot z} \\
&\times \int d^4p \biggl\{ \mr{tr}\left[\gamma^\kappa \gamma^\nu \gamma^\lambda \gamma^\rho \frac{1+\gamma_5}{2}\right] \\
&\qquad \qquad \times \left[\frac{(p-k_1+a)_\kappa}{(p-k_1+a)^2-i\epsilon}\frac{(p+a)_\lambda}{(p+a)^2-i\epsilon}- \frac{(p+b)_\kappa}{(p+b)^2-i\epsilon} \frac{(p+k_1+b)_\lambda}{(p+k_1+b)^2-i\epsilon} \right]  \\
&\qquad \qquad +\mr{tr}\left[\gamma^\kappa \gamma^\rho \gamma^\lambda \gamma^\nu\frac{1+\gamma_5}{2}\right] \\
&\qquad \qquad \times \left[\frac{(p-k_2+b)_\kappa}{(p-k_2+b)^2-i\epsilon} \frac{(p+b)_\lambda }{(p+b)^2-i\epsilon}- \frac{(p+a)_\kappa}{(p+a)^2-i\epsilon} \frac{(p+k_2+a)_\lambda }{(p+k_2+a)^2-i\epsilon}\right] \biggr\} \\
&=\frac{1}{(2\pi)^{12}}D_{\alpha\beta\gamma}\int d^4k_1 d^4k_2e^{-i(k_1+k_2)\cdot x}e^{ik_1\cdot y}e^{ik_2\cdot z} \\
&\times \int d^4p \biggl\{ \mr{tr}\left[\gamma^\kappa \gamma^\nu \gamma^\lambda \gamma^\rho \frac{1+\gamma_5}{2}\right] \\
&\times \left[\frac{(p+(a-b-k_1)+b)_\kappa}{(p+(a-b-k_1)+b)^2-i\epsilon}\frac{(p+(a-b-k_1)+(b+k_1))_\lambda}{(p+(a-b-k_1)+(b+k_1))^2-i\epsilon}- \frac{(p+b)_\kappa}{(p+b)^2-i\epsilon} \frac{(p+k_1+b)_\lambda}{(p+k_1+b)^2-i\epsilon} \right]  \\
&+\mr{tr}\left[\gamma^\kappa \gamma^\rho \gamma^\lambda \gamma^\nu\frac{1+\gamma_5}{2}\right] \\
&\times \left[\frac{(p+(b-a-k_2)+a)_\kappa}{(p+(b-a-k_2)+a)^2-i\epsilon} \frac{(p+(b-a-k_2)+(a+k_2))_\lambda }{(p+(b-a-k_2)+(a+k_2))^2-i\epsilon}- \frac{(p+a)_\kappa}{(p+a)^2-i\epsilon} \frac{(p+k_2+a)_\lambda }{(p+k_2+a)^2-i\epsilon}\right] \biggr\} \\
\end{align*}
から生じる.ここで
\begin{align*}
f_{\kappa\lambda}(p,c,d)\equiv \frac{(p+c)_\kappa (p+d)_\lambda}{[(p+c)^2-i\epsilon][(p+d)^2-i\epsilon]}
\end{align*}
とすると
\begin{align*}
&\left[ \frac{\partial}{\partial x^\mu}\Gamma^{\mu\nu\rho}_{\alpha\beta\gamma}(x,y,z) \right]_{\mr{anom}} \\
=&\frac{1}{(2\pi)^{12}}D_{\alpha\beta\gamma}\int d^4k_1 d^4k_2e^{-i(k_1+k_2)\cdot x}e^{ik_1\cdot y}e^{ik_2\cdot z} \\
&\int d^4p \biggl\{ \mr{tr}\left[\gamma^\kappa \gamma^\nu \gamma^\lambda \gamma^\rho \frac{1+\gamma_5}{2}\right] \left[f_{\kappa\lambda}(p+(a-b-k_1),b,b+k_1)-f_{\kappa\lambda}(p,b,b+k_1) \right]  \\
&\qquad \quad +\mr{tr}\left[\gamma^\kappa \gamma^\rho \gamma^\lambda \gamma^\nu\frac{1+\gamma_5}{2}\right] \left[f_{\kappa\lambda}(p+(b-a-k_2),a,a+k_2) -f_{\kappa\lambda}(p,a,a+k_2) \right] \\
\end{align*}
ここで
\begin{align*}
I_{\kappa,\lambda}(k,c,d)\equiv \int d^4 p[f_{\kappa\lambda}(p+k,c,d)-f_{\kappa\lambda}(p,c,d)]
\end{align*}
とすると
\begin{align*}
&\left[ \frac{\partial}{\partial x^\mu}\Gamma^{\mu\nu\rho}_{\alpha\beta\gamma}(x,y,z) \right]_{\mr{anom}} \\
=&\frac{1}{(2\pi)^{12}}D_{\alpha\beta\gamma}\int d^4k_1 d^4k_2e^{-i(k_1+k_2)\cdot x}e^{ik_1\cdot y}e^{ik_2\cdot z} \\
&\times \biggl\{ \mr{tr}\left[\gamma^\kappa \gamma^\nu \gamma^\lambda \gamma^\rho \frac{1+\gamma_5}{2}\right] I_{\kappa\lambda}(a-b-k_1,b,b+k_1)  \\
&+\mr{tr}\left[\gamma^\kappa \gamma^\rho \gamma^\lambda \gamma^\nu\frac{1+\gamma_5}{2}\right] I_{\kappa\lambda}(b-a-k_2,a,a+k_2) \biggr\}
\end{align*}
となる.これらの積分を計算するには,関数$f_{\kappa\lambda}(p+k,c,d)$の$k$についてのベキ展開を考える.
\begin{align*}
f_{\kappa\lambda}(p+k,c,d)=&\sum_{n=0}^\infty \frac{1}{n!}k^{\mu_1}\cdots k^{\mu_n}\frac{\partial^n f_{\kappa\lambda}(p,c,d)}{\partial p^{\mu_1}\cdots \partial p^{\mu_n}} \\
=&f_{\kappa\lambda}(p,c,d)+k^\mu \frac{\partial f_{\kappa\lambda}(p,c,d)}{\partial p^{\mu}}+\frac{1}{2} k^\mu k^\nu \frac{\partial^2 f_{\kappa\lambda}(p,c,d)}{\partial p^\mu \partial p^\nu}+ \cdots
\end{align*}
ゼロ次の項は明らかに(22.3.16)でゼロとなる.(22.3.16)の他の項は全て$p$に関する微分の積分だから,4次元版Stokesの定理より,ウィック回転の後には大きな4次元球の表面積分と書くことができる.その球の半径をいま仮に$P$としよう.そうすると,
\begin{align*}
\int d^4p k^{\mu_1}\cdots k^{\mu_n}\frac{\partial^n f_{\kappa\lambda}(p,c,d)}{\partial p^{\mu_1}\cdots \partial p^{\mu_n}}=i\int P^3 n_{\mu_1}\left\{k^{\mu_1}\cdots k^{\mu_n}\frac{\partial^{(n-1)} f_{\kappa\lambda}(p,c,d)}{\partial p^{\mu^2}\cdots \partial p^{\mu_n}}\right\} dS^3
\end{align*}
$f$は$P^{-2}$として振る舞うことが(22.3.17)からわかるから,右辺の,$f$の$n-1$階微分は$P^{-2-(n-1)}$として振る舞い,半径$P$の4次元球の表面積は$P^3$と振る舞うから,全体は$P^{2-n}$と振る舞う.したがって,$P\to \infty$において寄与できるのは$n=1$か$n=2$の項のみだ.
\begin{align*}
I_{\kappa\lambda}(k,c,d)=k^\mu\int d^4p \frac{\partial f_{\kappa\lambda}(p,c,d)}{\partial p^{\mu}}+\frac{1}{2} k^\mu k^\nu \int d^4p \frac{\partial^2 f_{\kappa\lambda}(p,c,d)}{\partial p^\mu \partial p^\nu}
\end{align*}
これを計算する.Stokesの定理よりこれらの積分は(SrednickiのChapter75参照)
\begin{align*}
\int d^4p \frac{\partial f_{\kappa\lambda}(p,c,d)}{\partial p^\mu}=&i\lim_{p\to \infty}\int d\Omega_4 p^2p_\mu f_{\kappa\lambda}(p,c,d) \\
\int d^4p \frac{\partial^2 f_{\kappa\lambda}(p,c,d)}{\partial p^\mu \partial p^\nu}=&i\lim_{p\to \infty}\int d\Omega_4 p^2p_\mu \frac{\partial f_{\kappa\lambda}(p,c,d)}{\partial p^\nu} 
\end{align*}
となることを用いる.ここで$\Omega_4$は4次元立体角で,全立体角は$\Omega_4=2\pi^2$となる.$f$の微分は
\begin{align*}
\frac{\partial f_{\kappa\lambda}(p,c,d)}{\partial p^{\nu}}=&\frac{\partial}{\partial p^\nu}\frac{(p+c)_\kappa (p+d)_\lambda}{[(p+c)^2-i\epsilon][(p+d)^2-i\epsilon]} \\
=&\eta_{\nu\kappa}\frac{(p+d)_\lambda}{[(p+c)^2-i\epsilon][(p+d)^2-i\epsilon]}+\eta_{\nu\lambda}\frac{(p+c)_\kappa}{[(p+c)^2-i\epsilon][(p+d)^2-i\epsilon]} \\
&-2\frac{(p+c)_\nu(p+c)_\kappa (p+d)_\lambda}{[(p+c)^2-i\epsilon]^2[(p+d)^2-i\epsilon]} \\
&-2\frac{(p+d)_\nu(p+c)_\kappa (p+d)_\lambda}{[(p+c)^2-i\epsilon][(p+d)^2-i\epsilon]^2}
\end{align*}
となる.(11.2.4)(11.3.2)のファインマンのパラメータ技法を用いると,
\begin{align*}
&\frac{1}{[(p+c)^2-i\epsilon][(p+d)^2-i\epsilon]}=\frac{1}{[p'^2-i\epsilon][(p'-q)^2-i\epsilon]} \quad(p'=p+c,q=c-d) \\
=&\int^1_0 dx\left[ (p'^2-i\epsilon)(1-x) +((p'-q)^2-i\epsilon)x \right]^{-2} \\
=&\int^1_0 dx\left[ (p'-qx)^2-i\epsilon +q^2x(1-x) \right]^{-2} \\
=&\int^1_0 dx\left[ (p+c-(c-d)x)^2-i\epsilon +(c-d)^2x(1-x) \right]^{-2} \\
&\frac{1}{[(p+c)^2-i\epsilon]^2[(p+d)^2-i\epsilon]}=\frac{1}{[p'^2-i\epsilon]^2[(p'-q)^2-i\epsilon]} \\
=&2\int^1_0 dx \int^x_0 dy \left[ (p'^2-i\epsilon)(1-x)+(p'^2-i\epsilon)(x-y)+((p'-q)^2-i\epsilon)y \right]^{-3} \\
=&2\int^1_0 dx \int^x_0 dy \left[(p'^2-i\epsilon)(1-x)+((p'-q)^2-i\epsilon)x \right ]^{-3} \\
=&2\int^1_0 dx \left[(p'-qx)^2-i\epsilon +q^2x(1-x)  \right]^{-3} x \\
=&2\int^1_0 dx \left[(p+c-(c-d)x)^2-i\epsilon +(c-d)^2x(1-x) \right]^{-3} x\\
&\frac{1}{[(p+c)^2-i\epsilon][(p+d)^2-i\epsilon]^2}=\frac{1}{[p'^2-i\epsilon][(p'-q)^2-i\epsilon]^2} \\
=&2\int^1_0 dx \int^x_0 dy \left[ (p'^2-i\epsilon)(1-x)+((p'-q)^2-i\epsilon)(x-y)+((p'-q)^2-i\epsilon)y \right]^{-3} \\
=&2\int^1_0 dx \int^x_0 dy \left[ (p'^2-i\epsilon)(1-x)+((p'-q)^2-i\epsilon)x \right]^{-3} \\
=&2\int^1_0 dx \left[(p+c-(c-d)x)^2-i\epsilon +(c-d)^2x(1-x) \right]^{-3} x
\end{align*}
ここで$p\to p'=p-c+(c-d)x$と変数変換しウィック回転を施し,Stokesの定理と,角度平均をとる際に(11.2.8)(11.2.9)の処方を用いれば
\begin{align*}
&\int d^4p \frac{\partial f_{\kappa\lambda}(p,c,d)}{\partial p^\mu} =i\int d^4p_E \frac{\partial f_{\kappa\lambda}(p,c,d)}{\partial p^\mu}\\
=&i\int d^4p'_E \frac{\partial f_{\kappa\lambda}(p',c,d)}{\partial p'^\mu}=i\lim_{p\to\infty}\int d\Omega_4 p'^2 p'_\mu f_{\kappa\lambda}(p',c,d) \\
=&i\lim_{p\to\infty}\int^1_0 dx \int d\Omega_4\frac{ [p-c+(c-d)x ]^2[p-c+(c-d)x]_\mu[p+(c-d)x]_\kappa [p-(c-d)(1-x)]_\lambda }{\left[ p^2 +(c-d)^2x(1-x) \right]^2} \\
=&i\lim_{p\to\infty}\int^1_0 dx\int d\Omega_4 [p^2-2p\cdot(c-(c-d)x)+(c-(c-d)x)^2] \\
&\qquad \times [p-c+(c-d)x]_\mu[p+(c-d)x]_\kappa [p-(c-d)(1-x)]_\lambda \\
&\qquad \times \left[ p^2 +(c-d)^2x(1-x) \right]^{-2} \\
=&i\lim_{p\to\infty}\int^1_0 dx \int d\Omega_4 \\
&\times [ -p^2p_\mu p_\kappa(c-d)_\lambda(1-x) + p^2p_\mu(c-d)_\kappa p_\lambda x -p^2[c-(c-d)x]_\mu p_\kappa p_\lambda  \\
&-2p\cdot [c-(c-d)x] p_\mu p_\kappa p_\lambda]\times  \left[ p^2 +(c-d)^2x(1-x) \right]^{-2} \\
=&i\Omega_4\lim_{p\to\infty}\int^1_0 dx \\
&\times \left[-\frac{(p^2)^2}{4}\eta_{\mu\kappa}(c-d)_\lambda (1-x)+\frac{(p^2)^2}{4}\eta_{\mu\lambda}(c-d)_\kappa x -\frac{(p^2)^2}{4}\eta_{\kappa\lambda}[c-(c-d) x]_\mu  \right. \\
&\qquad \left. -2\frac{(p^2)^2}{24}[c-(c-d)x]^\rho\{\eta_{\rho\mu}\eta_{\kappa\lambda}+\eta_{\rho\kappa}\eta_{\mu\lambda}+\eta_{\rho\lambda}\eta_{\mu\kappa} \} \right]\left[ p^2 +(c-d)^2x(1-x) \right]^{-2} \\
=&2\pi^2i\int^1_0 dx \left[-\frac{1}{4}\eta_{\mu\kappa}(c-d)_\lambda(1-x)+\frac{1}{4}\eta_{\mu\lambda}(c-d)_\kappa x+ \frac{1}{4}\eta_{\kappa\lambda}(c-d)_\mu x-\frac{1}{4}\eta_{\kappa\lambda}c_\mu \right. \\
&\qquad \left.-\frac{1}{12}\left\{(c-(c-d)x)_\mu\eta_{\kappa\lambda}+(c-(c-d)x)_\kappa\eta_{\mu\lambda}+(c-(c-d)x)_\lambda\eta_{\mu\kappa} \right\} \right] \\
=&2\pi^2i\left[-\frac{1}{8}\eta_{\mu\kappa}(c-d)_\lambda+\frac{1}{8}\eta_{\mu\lambda}(c-d)_\kappa - \frac{1}{8}\eta_{\kappa\lambda}(c+d)_\mu \right. \\
&\qquad \left.-\frac{1}{24}\{(c+d)_\mu\eta_{\kappa\lambda}+(c+d)_\kappa\eta_{\mu\lambda}+(c+d)_\lambda\eta_{\mu\kappa} \} \right] \\
=&\pi^2i\left[-\frac{1}{4}\eta_{\mu\kappa}c_\lambda +\frac{1}{4}\eta_{\mu\kappa}d_\lambda +\frac{1}{4}\eta_{\mu\lambda}c_\kappa -\frac{1}{4}\eta_{\mu\lambda}d_\kappa -\frac{1}{4}\eta_{\kappa\lambda}(c+d)_\mu\right. \\
&\qquad \left.-\frac{1}{12}\{(c+d)_\mu\eta_{\kappa\lambda}+(c+d)_\kappa\eta_{\mu\lambda}+(c+d)_\lambda\eta_{\mu\kappa} \} \right] \\
=&i\pi^2\left[-\frac{1}{3}\eta_{\mu\kappa}c_\lambda +\frac{1}{6}\eta_{\mu\kappa}d_\lambda +\frac{1}{6}\eta_{\mu\lambda}c_\kappa-\frac{1}{3}\eta_{\mu\lambda}d_\kappa -\frac{1}{3}\eta_{\kappa\lambda}(c+d)_\mu\right]
\end{align*}
となる.したがって
\begin{align*}
k^\mu\int d^4p \frac{\partial f_{\kappa\lambda}(p,c,d)}{\partial p^\mu}=i\pi^2\left[\frac{1}{6}k_\lambda c_\kappa+\frac{1}{6}k_\kappa d_\lambda-\frac{1}{3}k_\lambda d_\kappa-\frac{1}{3}k_\kappa c_\lambda -\frac{1}{3}\eta_{\kappa\lambda}k\cdot(c+d)\right]
\end{align*}
となる.これは係数が少し違う…どこかでミスがあるかもしれない.$I_{\kappa\lambda}(k,c,d)$の第二項目は
\begin{align*}
&\int d^4p \frac{\partial^2 f_{\kappa\lambda}(p,c,d)}{\partial p^\mu \partial p^\nu}=i\int d^4p_E \frac{\partial}{\partial p^\mu}\left\{\frac{\partial f_{\kappa\lambda}(p,c,d)}{\partial p^\nu}\right\} \\
=&i\int d^4p'_E \frac{\partial}{\partial p'^\mu}\left\{\frac{\partial f_{\kappa\lambda}(p',c,d)}{\partial p'^\nu}\right\} =i\int d\Omega_4 p'^2p'_\mu \left\{\frac{\partial f_{\kappa\lambda}(p',c,d)}{\partial p'^\nu} \right\}\\
=&i\lim_{p\to \infty} \int^1_0 dx \int d\Omega_4 (p-c+(c-d)x)^2(p-c+(c-d)x)_\mu \\
&\qquad \biggl\{ \frac{\eta_{\nu\kappa}(p-(c-d)(1-x))_\lambda +\eta_{\nu\lambda}(p+(c-d)x)_\kappa }{\left[ p^2+(c-d)^2x(1-x) \right]^2} \\
&\qquad -4\frac{(p+(c-d)x)_\nu(p+(c-d)x)_\kappa (p-(c-d)(1-x))_\lambda}{\left[p^2+(c-d)^2x(1-x) \right]^3}x \biggr\} \\
&\qquad -4\frac{(p-(c-d)(1-x))_\nu(p+(c-d)x)_\kappa (p-(c-d)(1-x))_\lambda}{\left[p^2+(c-d)^2x(1-x) \right]^3}x \biggr\} \\
=&i\lim_{p\to \infty}\int^1_0 dx \int d\Omega_4 \biggl\{ \frac{\eta_{\nu\kappa}p^2p_\mu p_\lambda+\eta_{\nu\lambda}p^2p_\mu p_\kappa}{\left[ p^2+(c-d)^2x(1-x) \right]^2} \\
&\qquad \qquad - 8\frac{p^2p_\mu p_\nu p_\kappa p_\lambda }{\left[p^2+(c-d)^2x(1-x) \right]^3} x +O(p^{-1}) \biggr\} \\
=&i\Omega_4\lim_{p\to \infty}\int^1_0 dx \biggl\{ \frac{[\eta_{\nu\kappa}\eta_{\mu\lambda}+\eta_{\nu\lambda}\eta_{\mu\kappa}](p^2)^2}{4\left[ p^2+(c-d)^2x(1-x) \right]^2}- 8\frac{(p^2)^3(\eta_{\mu\nu}\eta_{\kappa\lambda}+\eta_{\mu\kappa}\eta_{\nu\lambda}+\eta_{\mu\lambda}\eta_{\nu\kappa}) }{24\left[p^2+(c-d)^2x(1-x) \right]^3} x\biggr\} \\
=&i\Omega_4\int^1_0 dx \left\{ \frac{1}{4}(\eta_{\nu\kappa}\eta_{\mu\lambda}+\eta_{\nu\lambda}\eta_{\mu\kappa})-\frac{1}{3}(\eta_{\mu\nu}\eta_{\kappa\lambda}+\eta_{\mu\kappa}\eta_{\nu\lambda}+\eta_{\mu\lambda}\eta_{\nu\kappa})x \right\} \\
=&2\pi^2 i \left\{ \frac{1}{4}(\eta_{\nu\kappa}\eta_{\mu\lambda}+\eta_{\nu\lambda}\eta_{\mu\kappa})-\frac{1}{6}(\eta_{\mu\nu}\eta_{\kappa\lambda}+\eta_{\mu\kappa}\eta_{\nu\lambda}+\eta_{\mu\lambda}\eta_{\nu\kappa}) \right\} \\
=&i\pi^2\left\{ \frac{1}{6}\eta_{\nu\kappa}\eta_{\mu\lambda}+\frac{1}{6}\eta_{\nu\lambda}\eta_{\mu\kappa} -\frac{1}{3}\eta_{\mu\nu}\eta_{\kappa\lambda}\right\}
\end{align*}
となる.したがって
\begin{align*}
\frac{1}{2} k^\mu k^\nu \int d^4p \frac{\partial^2 f_{\kappa\lambda}(p,c,d)}{\partial p^\mu \partial p^\nu}=&\frac{1}{2}\left\{ \frac{1}{6}k_\kappa k_\lambda +\frac{1}{6}k_\lambda k_\kappa -\frac{1}{3}\eta_{\kappa\lambda}k^2 \right\} \\
=&\frac{1}{6}i\pi^2\left\{k_\kappa k_\lambda -\eta_{\kappa\lambda}k^2\right\}
\end{align*}
となる.この結果から分かる通り(22.3.19)は誤植を含んでいて,正しい結果は
\begin{align*}
I_{\kappa\lambda}(k,c,d)=\frac{1}{6}i\pi^2\left[k_\kappa k_\lambda +2k_\lambda c_\kappa +2k_\kappa d_\lambda -k_\lambda d_\kappa -k_\kappa c_\lambda -\eta_{\kappa\lambda} k\cdot (k+c+d) \right]
\end{align*}
となる…と思われる.\par

\vskip\baselineskip

(追記)この計算は以下のようにして得られることが分かった.\par
$f$には$f_{\kappa\lambda}(p;c,d)=f_{\lambda\kappa}(p;d,c)$という対称性があることに注意すると$I_{\kappa\lambda}$は以下の形に限られる.
\begin{align*}
I_{\kappa\lambda}(k;c,d)=A(k_\lambda c_\kappa+k_\kappa d_\lambda)+B(k_\lambda d_\kappa +k_\kappa c_\lambda)+C \eta_{\kappa\lambda} k\cdot (c+d)+D\eta_{\kappa\lambda} k^2 +E k_\kappa k_\lambda
\end{align*}
Stokesの定理により
\begin{align*}
\int d^4p \frac{\partial f_{\kappa\lambda}(p,c,d)}{\partial p^\mu}=&i\lim_{p\to \infty}\int d\Omega_4 p^2p_\mu f_{\kappa\lambda}(p,c,d) \\
=&i 2\pi^2 \lim_{p\to \infty} p_\mu p^2 f^o_{\kappa\lambda}(p;c,d) \\
\int d^4p \frac{\partial^2 f_{\kappa\lambda}(p,c,d)}{\partial p^\mu \partial p^\nu}=&i\lim_{p\to \infty}\int d\Omega_4 p^2p_\mu \frac{\partial f_{\kappa\lambda}(p,c,d)}{\partial p^\nu} \\
=&i2\pi^2 \lim_{p\to \infty } p_\mu p^2  \frac{\partial f^e_{\kappa\lambda}(p,c,d)}{\partial p^\nu}
\end{align*}
となる.ここで$f^o,f^e$は$f$の$p\to -p$による反対称(odd)部分と対称(even)部分である.これは角度平均をとる際に(11.2.8)(11.2.9)の処方を用いると,$p$について偶数次のみの項が残ることに拠る.\par
まず,$f_{\kappa\lambda}(p;c,d)$の反対称部分を計算して,$A,B,C$を求める.簡単のため$d=0$とすると
\begin{align*}
\frac{1}{2} \left( \frac{(p+c)_\kappa p_\lambda }{(p+c)^2 p^2}- \frac{(-p+c)_\kappa (-p)_\lambda}{(-p+c)^2 p^2} \right)=&\frac{(-p+c)^2(p+c)_\kappa p_\lambda +(p+c)^2 (-p+c)_\kappa p_\lambda}{2p^2 (p+c)^2 (-p+c)^2} \\
=&p_\lambda \frac{[(-p+c)^2-(p+c)^2]p_\kappa+[(-p+c)^2+(p+c)^2]c_\kappa}{2p^2 (p+c)^2 (-p+c)^2} \\
=&p_\lambda \frac{-2(p\cdot c)p_\kappa +(p^2+c^2) c_\kappa}{p^2 (p+c)^2 (-p+c)^2}
\end{align*}
であるから
\begin{align*}
\int d^4p \frac{\partial f_{\kappa\lambda}(p,c,0)}{\partial p^\mu}=&i 2\pi^2 \lim_{p\to \infty} p_\mu p^2 f^o_{\kappa\lambda}(p;c,0) \\
=&i2\pi^2 \lim_{p\to \infty }p_\mu p_\lambda \frac{-2(p\cdot c)p_\kappa +(p^2+c^2) c_\kappa}{(p+c)^2 (-p+c)^2} \\
=&i2\pi^2 \lim_{p\to \infty }\left( \frac{-2p_\mu p_\lambda p_\kappa p_\rho}{p^4}c^\rho + \frac{p_\mu p_\lambda}{p^2}c_\kappa \right) \\
=& i2\pi^2 \left( -\frac{1}{12}(\eta_{\mu\lambda}\eta_{\kappa\rho}+\eta_{\mu\kappa}\eta_{\lambda\rho}+\eta_{\mu\rho}\eta_{\kappa\lambda})c^\rho +\frac{1}{4} \eta_{\mu\lambda}c_\kappa \right) \\
=& i\pi^2 \left( -\frac{1}{6}(\eta_{\mu\kappa}\eta_{\lambda\rho}+\eta_{\mu\rho}\eta_{\kappa\lambda})c^\rho +\frac{1}{3} \eta_{\mu\lambda}c_\kappa \right)  \\
\therefore \quad k^\mu \int d^4p \frac{\partial f_{\kappa\lambda}(p;c,0)}{\partial p^\mu}=&-\frac{i\pi^2 }{6}(k_\kappa c_\lambda+\eta_{\kappa\lambda}k\cdot c)+\frac{i\pi^2}{3} k_\lambda c_\kappa
\end{align*}
したがって$A=-i\pi^2/6,B=C=i\pi^2/3$を得る.\par
次に,$f_{\kappa\lambda}(p;c.d)$の対称部分を計算して,$D,E$を求める.これらは$c,d$に依らないので$c=d=0$としてよい.このとき,$f$は偶関数となるから
\begin{align*}
\frac{\partial }{\partial p^\nu}f^e_{\kappa\lambda}(p;0,0)=&\frac{\partial}{\partial p^\nu} \frac{p_\kappa p_\lambda }{p^4} \\
=&-4\frac{p_\nu p_\kappa p_\lambda}{p^6}+\eta_{\nu\kappa}\frac{p_\lambda}{p^4}+\eta_{\nu\lambda}\frac{p_\kappa}{p^4} \\
\int d^4p \frac{\partial^2 f_{\kappa\lambda}(p,0,0)}{\partial p^\mu \partial p^\nu}=&i2\pi^2 \lim_{p\to \infty } p_\mu p^2  \frac{\partial f^e_{\kappa\lambda}(p,0,0)}{\partial p^\nu} \\
=&i2\pi^2\lim_{p\to \infty } \left[-4\frac{p_\mu p_\nu p_\kappa p_\lambda}{p^4}+\eta_{\nu\kappa}\frac{p_\mu p_\lambda}{p^2}+\eta_{\nu\lambda}\frac{p_\mu p_\kappa}{p^2} \right] \\
=&i2\pi^2 \left[-\frac{1}{6}(\eta_{\mu\nu}\eta_{\kappa\lambda}+\eta_{\mu\kappa}\eta_{\nu\lambda}+\eta_{\mu\lambda}\eta_{\nu\kappa})+\frac{1}{4}\eta_{\nu\kappa}\eta_{\mu\lambda}+\frac{1}{4}\eta_{\nu\lambda}\eta_{\mu\kappa}\right] \\
=&\frac{i\pi^2}{6}(\eta_{\nu\kappa}\eta_{\mu\lambda}+\eta_{\nu\lambda}\eta_{\mu\kappa})-\frac{i\pi^2}{3}\eta_{\mu\nu}\eta_{\kappa\lambda} \\
\therefore \quad \frac{1}{2}k^\mu k^\nu \int d^4p \frac{\partial^2 f_{\kappa\lambda}(p,c,d)}{\partial p^\mu \partial p^\nu}=&\frac{i\pi^2}{3}(-\eta_{\kappa\lambda}k^2+k_\kappa k_\lambda)
\end{align*}
したがって$D=-i\pi^2/3,E=i\pi^2/3$を得る.\par
以上より
\begin{align*}
I_{\kappa\lambda}(k;c,d)=\frac{1}{6}i\pi^2\left[k_\kappa k_\lambda +2k_\lambda c_\kappa +2k_\kappa d_\lambda -k_\lambda d_\kappa -k_\kappa c_\lambda -\eta_{\kappa\lambda} k\cdot (k+c+d) \right]
\end{align*}
となる.これが示したかった.


\vskip\baselineskip

さて,(22.3.15)において射影行列$\frac{1}{2}(1+\gamma_5)$の$1$と$\gamma_5$のトレースの項を別個に考察しなければならない.$1$から発生する項は,$\mr{tr}[\gamma^\kappa \gamma^\nu \gamma^\lambda \gamma^\rho]$を含む.これは(8.A.6)により$\kappa$と$\lambda$について,また$\nu$と$\rho$について対称だ.したがって積分は以下の組み合わせで表れる.
\begin{align*}
&\mr{tr}\left[\gamma^\kappa \gamma^\nu \gamma^\lambda \gamma^\rho \right] I_{\kappa\lambda}(a-b-k_1,b,b+k_1)  \\
&+\mr{tr}\left[\gamma^\kappa \gamma^\rho \gamma^\lambda \gamma^\nu \right] I_{\kappa\lambda}(b-a-k_2,a,a+k_2) \\
=&\frac{1}{2}\mr{tr}\left[\gamma^\kappa \gamma^\nu \gamma^\lambda \gamma^\rho \right]\left\{ I_{\kappa\lambda}(a-b-k_1,b,b+k_1)+I_{\lambda\kappa }(a-b-k_1,b,b+k_1)\right\} \\
&+\frac{1}{2}\mr{tr}\left[\gamma^\kappa \gamma^\rho \gamma^\lambda \gamma^\nu \right]\left\{ I_{\kappa\lambda}(b-a-k_2,a,a+k_2)+ I_{\lambda\kappa}(b-a-k_2,a,a+k_2)\right\} \\
=&\frac{1}{2}\mr{tr}\left[\gamma^\kappa \gamma^\nu \gamma^\lambda \gamma^\rho \right]\bigl\{ I_{\kappa\lambda}(a-b-k_1,b,b+k_1)+I_{\lambda\kappa }(a-b-k_1,b,b+k_1) \\
&\qquad +I_{\kappa\lambda}(b-a-k_2,a,a+k_2)+ I_{\lambda\kappa}(b-a-k_2,a,a+k_2)\bigr\} \\
\Rightarrow \quad & I_{\kappa\lambda}(a-b-k_1,b,b+k_1)+I_{\lambda\kappa }(a-b-k_1,b,b+k_1) \\
&\quad +I_{\kappa\lambda}(b-a-k_2,a,a+k_2)+ I_{\lambda\kappa}(b-a-k_2,a,a+k_2)
\end{align*}
(22.3.19)を使うと,
\begin{align*}
&I_{\kappa\lambda}(k,c,d)+I_{\lambda\kappa}(k,c,d) \\
=&\frac{1}{6}i\pi^2\left[k_\kappa k_\lambda +2k_\lambda c_\kappa +2k_\kappa d_\lambda -k_\lambda d_\kappa -k_\kappa c_\lambda -\eta_{\kappa\lambda} k\cdot (k+c+d) \right] \\
&+\frac{1}{6}i\pi^2\left[k_\lambda k_\kappa +2k_\kappa c_\lambda +2k_\lambda d_\kappa -k_\kappa d_\lambda -k_\lambda c_\kappa -\eta_{\kappa\lambda} k\cdot (k+c+d) \right] \\
=&\frac{1}{6}i\pi^2\left[2k_\kappa k_\lambda +k_\lambda (c_\kappa+d_\kappa) + k_\kappa(c_\lambda +d_\lambda)-2\eta_{\kappa\lambda}k\cdot(k+c+d) \right] \\
=&\frac{1}{6}i\pi^2\left[k_\lambda (k_\kappa+c_\kappa+d_\kappa) + k_\kappa(k_\lambda +c_\lambda +d_\lambda)-2\eta_{\kappa\lambda}k\cdot(k+c+d) \right]
\end{align*}
であるから
\begin{align*}
&I_{\kappa\lambda}(a-b-k_1,b,b+k_1)+I_{\lambda\kappa }(a-b-k_1,b,b+k_1) \\
& \quad +I_{\kappa\lambda}(b-a-k_2,a,a+k_2)+ I_{\lambda\kappa}(b-a-k_2,a,a+k_2) \\
=&\frac{1}{6}i\pi^2\left[(a-b-k_1)_\lambda (a+b)_\kappa + (a-b-k_1)_\kappa(a+b)_\lambda-2\eta_{\kappa\lambda}(a-b-k_1)\cdot(a+b) \right] \\
+&\frac{1}{6}i\pi^2\left[(b-a-k_2)_\lambda (a+b)_\kappa + (b-a-k_2)_\kappa(a+b)_\lambda-2\eta_{\kappa\lambda}(b-a-k_2)\cdot(a+b) \right] 
\end{align*}
となる.これが消えるのは任意の定数ベクトルを
\begin{align*}
a=-b
\end{align*}
と選んだときのみであることがわかる.このように選ぶと3つ全部のカレントについて非カイラル的なアノマリーを避けることができる.これは(22.3.11)の導出まで戻って,今度は$y^\nu$で微分すると
\begin{align*}
\Slash{k}_1=(\Slash{p}+\Slash{a})-(\Slash{p}-\Slash{k}_1+\Slash{a})=(\Slash{p}+\Slash{k}_1+\Slash{b})-(\Slash{p}+\Slash{b})
\end{align*}
を用いて
\begin{align*}
&\frac{\partial}{\partial y^\nu}\Gamma^{\mu\nu\rho}_{\alpha\beta\gamma}(x,y,z) \\
=&\frac{-1}{(2\pi)^{12}}\int d^4k_1 d^4k_2 \, k_{1\nu} e^{-i(k_1+k_2)\cdot x}e^{ik_1\cdot y}e^{ik_2\cdot z}\int d^4p \\
&\times \biggl\{\mr{tr}\left[\frac{\Slash{p}-\Slash{k}_1+\Slash{a}}{(p-k_1+a)^2-i\epsilon}\gamma^\nu \frac{\Slash{p}+\Slash{a}}{(p+a)^2-i\epsilon}\gamma^\rho \frac{\Slash{p}+\Slash{k}_2+\Slash{a}}{(p+k_2+a)^2-i\epsilon}\gamma^\mu \frac{1+\gamma_5}{2}\right]\mr{tr}[T_\beta T_\gamma T_\alpha] \\
&+\mr{tr}\left[\frac{\Slash{p}-\Slash{k}_2+\Slash{b}}{(p-k_2+b)^2-i\epsilon}\gamma^\rho \frac{\Slash{p}+\Slash{b}}{(p+b)^2-i\epsilon}\gamma^\nu \frac{\Slash{p}+\Slash{k}_1+\Slash{b}}{(p+k_1+b)^2-i\epsilon}\gamma^\mu \frac{1+\gamma_5}{2}\right]\mr{tr}[T_\gamma T_\beta T_\alpha]  \biggr\} \\
=&\frac{-1}{(2\pi)^{12}}\int d^4k_1 d^4k_2e^{-i(k_1+k_2)\cdot x}e^{ik_1\cdot y}e^{ik_2\cdot z}\int d^4p \\
&\times \biggl\{\mr{tr}\left[\frac{\Slash{p}-\Slash{k}_1+\Slash{a}}{(p-k_1+a)^2-i\epsilon} \Slash{k}_1 \frac{\Slash{p}+\Slash{a}}{(p+a)^2-i\epsilon}\gamma^\rho \frac{\Slash{p}+\Slash{k}_2+\Slash{a}}{(p+k_2+a)^2-i\epsilon}\gamma^\mu \frac{1+\gamma_5}{2}\right]\mr{tr}[T_\beta T_\gamma T_\alpha] \\
&+\mr{tr}\left[\frac{\Slash{p}-\Slash{k}_2+\Slash{b}}{(p-k_2+b)^2-i\epsilon}\gamma^\rho \frac{\Slash{p}+\Slash{b}}{(p+b)^2-i\epsilon}\Slash{k}_1 \frac{\Slash{p}+\Slash{k}_1+\Slash{b}}{(p+k_1+b)^2-i\epsilon}\gamma^\mu \frac{1+\gamma_5}{2}\right]\mr{tr}[T_\gamma T_\beta T_\alpha]  \biggr\} \\
=&\frac{-1}{(2\pi)^{12}}\int d^4k_1 d^4k_2e^{-i(k_1+k_2)\cdot x}e^{ik_1\cdot y}e^{ik_2\cdot z}\int d^4p \\
&\times \biggl\{\mr{tr}\left[\frac{\Slash{p}-\Slash{k}_1+\Slash{a}}{(p-k_1+a)^2-i\epsilon}(\Slash{p}+\Slash{a}) \frac{\Slash{p}+\Slash{a}}{(p+a)^2-i\epsilon}\gamma^\rho \frac{\Slash{p}+\Slash{k}_2+\Slash{a}}{(p+k_2+a)^2-i\epsilon}\gamma^\mu \frac{1+\gamma_5}{2}\right]\mr{tr}[T_\beta T_\gamma T_\alpha] \\
&-\mr{tr}\left[\frac{\Slash{p}-\Slash{k}_1+\Slash{a}}{(p-k_1+a)^2-i\epsilon}(\Slash{p}-\Slash{k}_1+\Slash{a}) \frac{\Slash{p}+\Slash{a}}{(p+a)^2-i\epsilon}\gamma^\rho \frac{\Slash{p}+\Slash{k}_2+\Slash{a}}{(p+k_2+a)^2-i\epsilon}\gamma^\mu \frac{1+\gamma_5}{2}\right]\mr{tr}[T_\beta T_\gamma T_\alpha] \\
&+\mr{tr}\left[\frac{\Slash{p}-\Slash{k}_2+\Slash{b}}{(p-k_2+b)^2-i\epsilon}\gamma^\rho \frac{\Slash{p}+\Slash{b}}{(p+b)^2-i\epsilon}(\Slash{p}+\Slash{k}_1+\Slash{b}) \frac{\Slash{p}+\Slash{k}_1+\Slash{b}}{(p+k_1+b)^2-i\epsilon}\gamma^\mu \frac{1+\gamma_5}{2}\right]\mr{tr}[T_\gamma T_\beta T_\alpha] \\
&-\mr{tr}\left[\frac{\Slash{p}-\Slash{k}_2+\Slash{b}}{(p-k_2+b)^2-i\epsilon}\gamma^\rho \frac{\Slash{p}+\Slash{b}}{(p+b)^2-i\epsilon}(\Slash{p}+\Slash{b}) \frac{\Slash{p}+\Slash{k}_1+\Slash{b}}{(p+k_1+b)^2-i\epsilon}\gamma^\mu \frac{1+\gamma_5}{2}\right]\mr{tr}[T_\gamma T_\beta T_\alpha]  \biggr\} \\
=&\frac{-1}{(2\pi)^{12}}\int d^4k_1 d^4k_2e^{-i(k_1+k_2)\cdot x}e^{ik_1\cdot y}e^{ik_2\cdot z}\int d^4p \\
&\times \biggl\{\mr{tr}\left[\frac{\Slash{p}-\Slash{k}_1+\Slash{a}}{(p-k_1+a)^2-i\epsilon}\gamma^\rho \frac{\Slash{p}+\Slash{k}_2+\Slash{a}}{(p+k_2+a)^2-i\epsilon}\gamma^\mu \frac{1+\gamma_5}{2}\right]\mr{tr}[T_\beta T_\gamma T_\alpha] \\
&-\mr{tr}\left[ \frac{\Slash{p}+\Slash{a}}{(p+a)^2-i\epsilon}\gamma^\rho \frac{\Slash{p}+\Slash{k}_2+\Slash{a}}{(p+k_2+a)^2-i\epsilon}\gamma^\mu \frac{1+\gamma_5}{2}\right]\mr{tr}[T_\beta T_\gamma T_\alpha] \\
&+\mr{tr}\left[\frac{\Slash{p}-\Slash{k}_2+\Slash{b}}{(p-k_2+b)^2-i\epsilon}\gamma^\rho \frac{\Slash{p}+\Slash{b}}{(p+b)^2-i\epsilon}\gamma^\mu \frac{1+\gamma_5}{2}\right]\mr{tr}[T_\gamma T_\beta T_\alpha] \\
&-\mr{tr}\left[\frac{\Slash{p}-\Slash{k}_2+\Slash{b}}{(p-k_2+b)^2-i\epsilon}\gamma^\rho \frac{\Slash{p}+\Slash{k}_1+\Slash{b}}{(p+k_1+b)^2-i\epsilon}\gamma^\mu \frac{1+\gamma_5}{2}\right]\mr{tr}[T_\gamma T_\beta T_\alpha]  \biggr\} \\
=&\frac{-1}{(2\pi)^{12}}\int d^4k_1 d^4k_2e^{-i(k_1+k_2)\cdot x}e^{ik_1\cdot y}e^{ik_2\cdot z}\int d^4p \\
&\times \biggl\{\mr{tr}\left[ \frac{\Slash{p}+\Slash{k}_2+\Slash{a}}{(p+k_2+a)^2-i\epsilon}\gamma^\mu \frac{\Slash{p}-\Slash{k}_1+\Slash{a}}{(p-k_1+a)^2-i\epsilon}\gamma^\rho \frac{1+\gamma_5}{2}\right]\mr{tr}[T_\beta T_\gamma T_\alpha] \\
&-\mr{tr}\left[ \frac{\Slash{p}+\Slash{a}}{(p+a)^2-i\epsilon}\gamma^\rho \frac{\Slash{p}+\Slash{k}_2+\Slash{a}}{(p+k_2+a)^2-i\epsilon}\gamma^\mu \frac{1+\gamma_5}{2}\right]\mr{tr}[T_\beta T_\gamma T_\alpha] \\
&+\mr{tr}\left[ \frac{\Slash{p}-\Slash{k}_2+\Slash{b}}{(p-k_2+b)^2-i\epsilon}\gamma^\rho \frac{\Slash{p}+\Slash{b}}{(p+b)^2-i\epsilon}\gamma^\mu \frac{1+\gamma_5}{2}\right]\mr{tr}[T_\gamma T_\beta T_\alpha] \\
&-\mr{tr}\left[ \frac{\Slash{p}+\Slash{k}_1+\Slash{b}}{(p+k_1+b)^2-i\epsilon}\gamma^\mu \frac{\Slash{p}-\Slash{k}_2+\Slash{b}}{(p-k_2+b)^2-i\epsilon}\gamma^\rho \frac{1+\gamma_5}{2}\right]\mr{tr}[T_\gamma T_\beta T_\alpha]  \biggr\} 
\end{align*}
となってアノマリー部分は
\begin{align*}
&\left[ \frac{\partial}{\partial y^\nu}\Gamma^{\mu\nu\rho}_{\alpha\beta\gamma}(x,y,z) \right]_{\mr{anom}} \\
=&\frac{-1}{(2\pi)^{12}}D_{\alpha\beta\gamma}\int d^4k_1 d^4k_2e^{-i(k_1+k_2)\cdot x}e^{ik_1\cdot y}e^{ik_2\cdot z}\int d^4p \\
&\times \biggl\{\mr{tr}\left[ \frac{\Slash{p}+\Slash{k}_2+\Slash{a}}{(p+k_2+a)^2-i\epsilon}\gamma^\mu \frac{\Slash{p}-\Slash{k}_1+\Slash{a}}{(p-k_1+a)^2-i\epsilon}\gamma^\rho \frac{1+\gamma_5}{2}\right] \\
&-\mr{tr}\left[ \frac{\Slash{p}+\Slash{a}}{(p+a)^2-i\epsilon}\gamma^\rho \frac{\Slash{p}+\Slash{k}_2+\Slash{a}}{(p+k_2+a)^2-i\epsilon}\gamma^\mu \frac{1+\gamma_5}{2}\right] \\
&+\mr{tr}\left[ \frac{\Slash{p}-\Slash{k}_2+\Slash{b}}{(p-k_2+b)^2-i\epsilon}\gamma^\rho \frac{\Slash{p}+\Slash{b}}{(p+b)^2-i\epsilon}\gamma^\mu \frac{1+\gamma_5}{2}\right] \\
&-\mr{tr}\left[ \frac{\Slash{p}+\Slash{k}_1+\Slash{b}}{(p+k_1+b)^2-i\epsilon}\gamma^\mu \frac{\Slash{p}-\Slash{k}_2+\Slash{b}}{(p-k_2+b)^2-i\epsilon}\gamma^\rho \frac{1+\gamma_5}{2}\right] \biggr\} \\
=&\frac{-1}{(2\pi)^{12}}D_{\alpha\beta\gamma}\int d^4k_1 d^4k_2e^{-i(k_1+k_2)\cdot x}e^{ik_1\cdot y}e^{ik_2\cdot z} \\
&\times \int d^4p \biggl\{\mr{tr}\left[\gamma^\kappa \gamma^\mu \gamma^\lambda \gamma^\rho \frac{1+\gamma_5}{2}\right] \\
&\qquad \qquad \times \left[ \frac{(p+k_2+a)_\kappa}{(p+k_2+a)^2-i\epsilon} \frac{(p-k_1+a)_\lambda }{(p-k_1+a)^2-i\epsilon}-\frac{(p+k_1+b)_\kappa}{(p+k_1+b)^2-i\epsilon} \frac{(p-k_2+b)_\lambda }{(p-k_2+b)^2-i\epsilon} \right]  \\
&\qquad \qquad + \mr{tr}\left[\gamma^\kappa \gamma^\rho \gamma^\lambda \gamma^\mu \frac{1+\gamma_5}{2}\right] \\
&\qquad \qquad \times \left[ \frac{(p-k_2+b)_\kappa}{(p-k_2+b)^2-i\epsilon} \frac{(p+b)_\lambda}{(p+b)^2-i\epsilon}-\frac{(p+a)_\kappa}{(p+a)^2-i\epsilon}\frac{(p+k_2+a)_\lambda}{(p+k_2+a)^2-i\epsilon} \right] \biggr\} \\
=&\frac{-1}{(2\pi)^{12}}D_{\alpha\beta\gamma}\int d^4k_1 d^4k_2e^{-i(k_1+k_2)\cdot x}e^{ik_1\cdot y}e^{ik_2\cdot z} \\
&\times \int d^4p \biggl\{\mr{tr}\left[\gamma^\kappa \gamma^\mu \gamma^\lambda \gamma^\rho \frac{1+\gamma_5}{2}\right] \\
&\times \biggl[ \frac{(p+(a-b-k_1+k_2)+b+k_1)_\kappa}{(p+(a-b-k_1+k_2)+b+k_1)^2-i\epsilon} \frac{(p+(a-b-k_1+k_2)+b-k_2)_\lambda }{(p+(a-b-k_1+k_2)+b-k_2)^2-i\epsilon} \\
&\qquad -\frac{(p+b+k_1)_\kappa}{(p+b+k_1)^2-i\epsilon} \frac{(p+b-k_2)_\lambda }{(p+b-k_2)^2-i\epsilon} \biggr]  \\
&+ \mr{tr}\left[\gamma^\kappa \gamma^\rho \gamma^\lambda \gamma^\mu \frac{1+\gamma_5}{2}\right] \\
&\times \biggl[ \frac{(p+(b-a-k_2)+a)_\kappa}{(p+(b-a-k_2)+a)^2-i\epsilon} \frac{(p+(b-a-k_2)+a+k_2)_\lambda}{(p+(b-a-k_2)+a+k_2)^2-i\epsilon} \\
&\qquad -\frac{(p+a)_\kappa}{(p+a)^2-i\epsilon}\frac{(p+a+k_2)_\lambda}{(p+a+k_2)^2-i\epsilon} \biggr] \biggr\} \\
=&\frac{-1}{(2\pi)^{12}}D_{\alpha\beta\gamma}\int d^4k_1 d^4k_2e^{-i(k_1+k_2)\cdot x}e^{ik_1\cdot y}e^{ik_2\cdot z} \\
&\times \int d^4p\biggl\{\mr{tr}\left[\gamma^\kappa \gamma^\mu \gamma^\lambda \gamma^\rho \frac{1+\gamma_5}{2}\right][f_{\kappa\lambda}(p+(a-b-k_1+k_2),b+k_1,b-k_2)-f_{\kappa\lambda}(p,b+k_1,b-k_2)] \\
&\qquad +\mr{tr}\left[\gamma^\kappa \gamma^\rho \gamma^\lambda \gamma^\mu \frac{1+\gamma_5}{2}\right][f_{\kappa\lambda}(p+(b-a-k_2),a,a+k_2)-f_{\kappa\lambda}(p,a,a+k_2)] \\
=&\frac{-1}{(2\pi)^{12}}D_{\alpha\beta\gamma}\int d^4k_1 d^4k_2e^{-i(k_1+k_2)\cdot x}e^{ik_1\cdot y}e^{ik_2\cdot z} \\
&\times \biggl\{ \mr{tr}\left[\gamma^\kappa \gamma^\mu \gamma^\lambda \gamma^\rho \frac{1+\gamma_5}{2}\right] I_{\kappa\lambda}(a-b-k_1+k_2,b+k_1,b-k_2) \\
&\qquad +\mr{tr}\left[\gamma^\kappa \gamma^\rho \gamma^\lambda \gamma^\mu \frac{1+\gamma_5}{2}\right]I_{\kappa\lambda}(b-a-k_2,a,a+k_2) \biggr\}
\end{align*}
となる.これは(22.3.15)における第一項目で$a\to a'=a-k_1,b\to b'=b+k_1,k_1\to k'_1=-k_1-k_2 $と置き換えたものと等しい.$a=-b$を仮定すれば$a'=-b'$も保証され,非カイラル項は打ち消し合わされる.$z^\rho$で微分した場合も同様に
\begin{align*}
&\frac{\partial}{\partial z^\rho}\Gamma^{\mu\nu\rho}_{\alpha\beta\gamma}(x,y,z) \\
=&\frac{-1}{(2\pi)^{12}}\int d^4k_1 d^4k_2 \, k_{2\rho} e^{-i(k_1+k_2)\cdot x}e^{ik_1\cdot y}e^{ik_2\cdot z}\int d^4p \\
&\times \biggl\{\mr{tr}\left[\frac{\Slash{p}-\Slash{k}_1+\Slash{a}}{(p-k_1+a)^2-i\epsilon}\gamma^\nu \frac{\Slash{p}+\Slash{a}}{(p+a)^2-i\epsilon}\gamma^\rho \frac{\Slash{p}+\Slash{k}_2+\Slash{a}}{(p+k_2+a)^2-i\epsilon}\gamma^\mu \frac{1+\gamma_5}{2}\right]\mr{tr}[T_\beta T_\gamma T_\alpha] \\
&+\mr{tr}\left[\frac{\Slash{p}-\Slash{k}_2+\Slash{b}}{(p-k_2+b)^2-i\epsilon}\gamma^\rho \frac{\Slash{p}+\Slash{b}}{(p+b)^2-i\epsilon}\gamma^\nu \frac{\Slash{p}+\Slash{k}_1+\Slash{b}}{(p+k_1+b)^2-i\epsilon}\gamma^\mu \frac{1+\gamma_5}{2}\right]\mr{tr}[T_\gamma T_\beta T_\alpha]  \biggr\} \\
=&\frac{-1}{(2\pi)^{12}}\int d^4k_1 d^4k_2\, e^{-i(k_1+k_2)\cdot x}e^{ik_1\cdot y}e^{ik_2\cdot z}\int d^4p \\
&\times \biggl\{\mr{tr}\left[\frac{\Slash{p}-\Slash{k}_1+\Slash{a}}{(p-k_1+a)^2-i\epsilon}\gamma^\nu \frac{\Slash{p}+\Slash{a}}{(p+a)^2-i\epsilon}\Slash{k}_2 \frac{\Slash{p}+\Slash{k}_2+\Slash{a}}{(p+k_2+a)^2-i\epsilon}\gamma^\mu \frac{1+\gamma_5}{2}\right]\mr{tr}[T_\beta T_\gamma T_\alpha] \\
&+\mr{tr}\left[\frac{\Slash{p}-\Slash{k}_2+\Slash{b}}{(p-k_2+b)^2-i\epsilon}\Slash{k}_2 \frac{\Slash{p}+\Slash{b}}{(p+b)^2-i\epsilon}\gamma^\nu \frac{\Slash{p}+\Slash{k}_1+\Slash{b}}{(p+k_1+b)^2-i\epsilon}\gamma^\mu \frac{1+\gamma_5}{2}\right]\mr{tr}[T_\gamma T_\beta T_\alpha]  \biggr\} \\
=&\frac{-1}{(2\pi)^{12}}\int d^4k_1 d^4k_2\, e^{-i(k_1+k_2)\cdot x}e^{ik_1\cdot y}e^{ik_2\cdot z}\int d^4p \\
&\times \biggl\{\mr{tr}\left[\frac{\Slash{p}-\Slash{k}_1+\Slash{a}}{(p-k_1+a)^2-i\epsilon}\gamma^\nu \frac{\Slash{p}+\Slash{a}}{(p+a)^2-i\epsilon}(\Slash{p}+\Slash{k}_2+\Slash{a}) \frac{\Slash{p}+\Slash{k}_2+\Slash{a}}{(p+k_2+a)^2-i\epsilon}\gamma^\mu \frac{1+\gamma_5}{2}\right]\mr{tr}[T_\beta T_\gamma T_\alpha] \\
&-\mr{tr}\left[\frac{\Slash{p}-\Slash{k}_1+\Slash{a}}{(p-k_1+a)^2-i\epsilon}\gamma^\nu \frac{\Slash{p}+\Slash{a}}{(p+a)^2-i\epsilon}(\Slash{p}+\Slash{a}) \frac{\Slash{p}+\Slash{k}_2+\Slash{a}}{(p+k_2+a)^2-i\epsilon}\gamma^\mu \frac{1+\gamma_5}{2}\right]\mr{tr}[T_\beta T_\gamma T_\alpha] \\
&+\mr{tr}\left[\frac{\Slash{p}-\Slash{k}_2+\Slash{b}}{(p-k_2+b)^2-i\epsilon}(\Slash{p}+\Slash{k}_2) \frac{\Slash{p}+\Slash{b}}{(p+b)^2-i\epsilon}\gamma^\nu \frac{\Slash{p}+\Slash{k}_1+\Slash{b}}{(p+k_1+b)^2-i\epsilon}\gamma^\mu \frac{1+\gamma_5}{2}\right]\mr{tr}[T_\gamma T_\beta T_\alpha] \\
&-\mr{tr}\left[\frac{\Slash{p}-\Slash{k}_2+\Slash{b}}{(p-k_2+b)^2-i\epsilon}(\Slash{p}-\Slash{k}_2+\Slash{b}) \frac{\Slash{p}+\Slash{b}}{(p+b)^2-i\epsilon}\gamma^\nu \frac{\Slash{p}+\Slash{k}_1+\Slash{b}}{(p+k_1+b)^2-i\epsilon}\gamma^\mu \frac{1+\gamma_5}{2}\right]\mr{tr}[T_\gamma T_\beta T_\alpha]  \biggr\} \\
=&\frac{-1}{(2\pi)^{12}}\int d^4k_1 d^4k_2\, e^{-i(k_1+k_2)\cdot x}e^{ik_1\cdot y}e^{ik_2\cdot z}\int d^4p \\
&\times \biggl\{\mr{tr}\left[\frac{\Slash{p}-\Slash{k}_1+\Slash{a}}{(p-k_1+a)^2-i\epsilon}\gamma^\nu \frac{\Slash{p}+\Slash{a}}{(p+a)^2-i\epsilon}\gamma^\mu \frac{1+\gamma_5}{2}\right]\mr{tr}[T_\beta T_\gamma T_\alpha] \\
&-\mr{tr}\left[\frac{\Slash{p}-\Slash{k}_1+\Slash{a}}{(p-k_1+a)^2-i\epsilon}\gamma^\nu \frac{\Slash{p}+\Slash{k}_2+\Slash{a}}{(p+k_2+a)^2-i\epsilon}\gamma^\mu \frac{1+\gamma_5}{2}\right]\mr{tr}[T_\beta T_\gamma T_\alpha] \\
&+\mr{tr}\left[\frac{\Slash{p}-\Slash{k}_2+\Slash{b}}{(p-k_2+b)^2-i\epsilon} \gamma^\nu \frac{\Slash{p}+\Slash{k}_1+\Slash{b}}{(p+k_1+b)^2-i\epsilon}\gamma^\mu \frac{1+\gamma_5}{2}\right]\mr{tr}[T_\gamma T_\beta T_\alpha] \\
&-\mr{tr}\left[ \frac{\Slash{p}+\Slash{b}}{(p+b)^2-i\epsilon}\gamma^\nu \frac{\Slash{p}+\Slash{k}_1+\Slash{b}}{(p+k_1+b)^2-i\epsilon}\gamma^\mu \frac{1+\gamma_5}{2}\right]\mr{tr}[T_\gamma T_\beta T_\alpha]  \biggr\} \\
=&\frac{-1}{(2\pi)^{12}}\int d^4k_1 d^4k_2\, e^{-i(k_1+k_2)\cdot x}e^{ik_1\cdot y}e^{ik_2\cdot z}\int d^4p \\
&\times \biggl\{\mr{tr}\left[\frac{\Slash{p}-\Slash{k}_1+\Slash{a}}{(p-k_1+a)^2-i\epsilon}\gamma^\nu \frac{\Slash{p}+\Slash{a}}{(p+a)^2-i\epsilon}\gamma^\mu \frac{1+\gamma_5}{2}\right]\mr{tr}[T_\beta T_\gamma T_\alpha] \\
&-\mr{tr}\left[ \frac{\Slash{p}+\Slash{k}_2+\Slash{a}}{(p+k_2+a)^2-i\epsilon}\gamma^\mu \frac{\Slash{p}-\Slash{k}_1+\Slash{a}}{(p-k_1+a)^2-i\epsilon}\gamma^\nu \frac{1+\gamma_5}{2}\right]\mr{tr}[T_\beta T_\gamma T_\alpha] \\
&+\mr{tr}\left[ \frac{\Slash{p}+\Slash{k}_1+\Slash{b}}{(p+k_1+b)^2-i\epsilon}\gamma^\mu \frac{\Slash{p}-\Slash{k}_2+\Slash{b}}{(p-k_2+b)^2-i\epsilon} \gamma^\nu \frac{1+\gamma_5}{2}\right]\mr{tr}[T_\gamma T_\beta T_\alpha] \\
&-\mr{tr}\left[ \frac{\Slash{p}+\Slash{b}}{(p+b)^2-i\epsilon}\gamma^\nu \frac{\Slash{p}+\Slash{k}_1+\Slash{b}}{(p+k_1+b)^2-i\epsilon}\gamma^\mu \frac{1+\gamma_5}{2}\right]\mr{tr}[T_\gamma T_\beta T_\alpha]  \biggr\}
\end{align*}
これのアノマリー部分は
\begin{align*}
&\left[ \frac{\partial}{\partial y^\nu}\Gamma^{\mu\nu\rho}_{\alpha\beta\gamma}(x,y,z) \right]_{\mr{anom}} \\
=&\frac{-1}{(2\pi)^{12}}D_{\alpha\beta\gamma}\int d^4k_1 d^4k_2\, e^{-i(k_1+k_2)\cdot x}e^{ik_1\cdot y}e^{ik_2\cdot z}\int d^4p \\
&\times \biggl\{\mr{tr}\left[\frac{\Slash{p}-\Slash{k}_1+\Slash{a}}{(p-k_1+a)^2-i\epsilon}\gamma^\nu \frac{\Slash{p}+\Slash{a}}{(p+a)^2-i\epsilon}\gamma^\mu \frac{1+\gamma_5}{2}\right] \\
&-\mr{tr}\left[ \frac{\Slash{p}+\Slash{k}_2+\Slash{a}}{(p+k_2+a)^2-i\epsilon}\gamma^\mu \frac{\Slash{p}-\Slash{k}_1+\Slash{a}}{(p-k_1+a)^2-i\epsilon}\gamma^\nu \frac{1+\gamma_5}{2}\right] \\
&+\mr{tr}\left[ \frac{\Slash{p}+\Slash{k}_1+\Slash{b}}{(p+k_1+b)^2-i\epsilon}\gamma^\mu \frac{\Slash{p}-\Slash{k}_2+\Slash{b}}{(p-k_2+b)^2-i\epsilon} \gamma^\nu \frac{1+\gamma_5}{2}\right] \\
&-\mr{tr}\left[ \frac{\Slash{p}+\Slash{b}}{(p+b)^2-i\epsilon}\gamma^\nu \frac{\Slash{p}+\Slash{k}_1+\Slash{b}}{(p+k_1+b)^2-i\epsilon}\gamma^\mu \frac{1+\gamma_5}{2}\right] \biggr\} \\
=&\frac{-1}{(2\pi)^{12}}D_{\alpha\beta\gamma}\int d^4k_1 d^4k_2e^{-i(k_1+k_2)\cdot x}e^{ik_1\cdot y}e^{ik_2\cdot z} \\
&\times \int d^4p \biggl\{\mr{tr}\left[\gamma^\kappa \gamma^\nu \gamma^\lambda \gamma^\mu \frac{1+\gamma_5}{2}\right] \\
&\qquad \qquad \times \left[ \frac{(p-k_1+a)_\kappa}{(p-k_1+a)^2-i\epsilon} \frac{(p+a)_\lambda }{(p+a)^2-i\epsilon}-\frac{(p+b)_\kappa}{(p+b)^2-i\epsilon} \frac{(p+k_1+b)_\lambda }{(p+k_1+b)^2-i\epsilon} \right]  \\
&\qquad \qquad + \mr{tr}\left[\gamma^\kappa \gamma^\mu \gamma^\lambda \gamma^\nu \frac{1+\gamma_5}{2}\right] \\
&\qquad \qquad \times \left[ \frac{(p+k_1+b)_\kappa}{(p+k_1+b)^2-i\epsilon} \frac{(p-k_2+b)_\lambda}{(p-k_2+b)^2-i\epsilon}-\frac{(p+k_2+a)_\kappa}{(p+k_2+a)^2-i\epsilon}\frac{(p-k_1+a)_\lambda}{(p-k_1+a)^2-i\epsilon} \right] \biggr\} \\
=&\frac{-1}{(2\pi)^{12}}D_{\alpha\beta\gamma}\int d^4k_1 d^4k_2e^{-i(k_1+k_2)\cdot x}e^{ik_1\cdot y}e^{ik_2\cdot z} \\
&\times \int d^4p \biggl\{\mr{tr}\left[\gamma^\kappa \gamma^\nu \gamma^\lambda \gamma^\mu \frac{1+\gamma_5}{2}\right] \\
&\times \biggl[ \frac{(p+(a-b-k_1)+b)_\kappa}{(p+(a-b-k_1)+b)^2-i\epsilon} \frac{(p+(a-b-k_1)+b+k_1)_\lambda }{(p+(a-b-k_1)+b+k_1)^2-i\epsilon} \\
&\qquad -\frac{(p+b)_\kappa}{(p+b)^2-i\epsilon} \frac{(p+b+k_1)_\lambda }{(p+b+k_1)^2-i\epsilon} \biggr]  \\
&+ \mr{tr}\left[\gamma^\kappa \gamma^\mu \gamma^\lambda \gamma^\nu \frac{1+\gamma_5}{2}\right] \\
&\times \biggl[ \frac{(p+(b-a+k_1-k_2)+a+k_2)_\kappa}{(p+(b-a+k_1-k_2)+a+k_2)^2-i\epsilon} \frac{(p+(b-a+k_1-k_2)+a-k_1)_\lambda}{(p+(b-a+k_1-k_2)+a-k_1)^2-i\epsilon} \\
&\qquad -\frac{(p+a+k_2)_\kappa}{(p+a+k_2)^2-i\epsilon}\frac{(p+a-k_1)_\lambda}{(p+a-k_1)^2-i\epsilon} \biggr] \biggr\} \\
=&\frac{-1}{(2\pi)^{12}}D_{\alpha\beta\gamma}\int d^4k_1 d^4k_2e^{-i(k_1+k_2)\cdot x}e^{ik_1\cdot y}e^{ik_2\cdot z} \\
&\times \int d^4p\biggl\{\mr{tr}\left[\gamma^\kappa \gamma^\nu \gamma^\lambda \gamma^\mu \frac{1+\gamma_5}{2}\right][f_{\kappa\lambda}(p+(a-b-k_1),b,b+k_1)-f_{\kappa\lambda}(p,b,b+k_1)] \\
&\qquad +\mr{tr}\left[\gamma^\kappa \gamma^\mu \gamma^\lambda \gamma^\nu \frac{1+\gamma_5}{2}\right][f_{\kappa\lambda}(p+(b-a+k_1-k_2),a+k_2,a-k_1)-f_{\kappa\lambda}(p,a+k_2,a-k_1)] \\
=&\frac{-1}{(2\pi)^{12}}D_{\alpha\beta\gamma}\int d^4k_1 d^4k_2e^{-i(k_1+k_2)\cdot x}e^{ik_1\cdot y}e^{ik_2\cdot z} \\
&\times \biggl\{ \mr{tr}\left[\gamma^\kappa \gamma^\nu \gamma^\lambda \gamma^\mu \frac{1+\gamma_5}{2}\right] I_{\kappa\lambda}(a-b-k_1,b,b+k_1) \\
&\qquad +\mr{tr}\left[\gamma^\kappa \gamma^\mu \gamma^\lambda \gamma^\nu \frac{1+\gamma_5}{2}\right]I_{\kappa\lambda}(b-a+k_1-k_2,a+k_2,a-k_1) \biggr\}
\end{align*}
となるが,これは(22.3.15)の第二項目で$a\to a''=a+k_2,b\to b''=b-k_2,k''_2=-k_1-k_2$と置き換えたものと等しい.$a=-b$を仮定すれば$a''=-b''$も保証される.

\vskip\baselineskip

$\gamma_5$を含むトレース項が残った.これは(8.A.12)より
\begin{align*}
\mr{tr}\left[\gamma^\kappa \gamma^\nu \gamma^\lambda \gamma^\rho \gamma_5 \right]=-4i\epsilon^{\kappa\nu\lambda\rho}
\end{align*}
となって完全反対称だ.(マイナスは$\gamma_0=-\gamma^0$より生じる.$\kappa,\nu,\lambda,\rho$のどれか一つは0を取るからだ.)これと$b=-a$を(22.3.15)に使うと
\begin{align*}
&I_{\kappa\lambda}(a-b-k_1,b,b+k_1)=I_{\kappa\lambda}(2a-k_1,-a,-a+k_1) \\
=&\frac{1}{6}i\pi^2[ (2a-k_1)_\kappa (2a-k_1)_\lambda-2a_\kappa(2a-k_1)_\lambda  +2(2a-k_1)_\kappa (-a+k_1)_\lambda \\
&- (-a+k_1)_\kappa (2a-k_1)_\lambda + (2a-k_1)_\kappa a_\lambda  ] \\
=&\frac{1}{6}i\pi^2[-k_{1\kappa}(2a-k_1)_\lambda -(-a+k_1)_\kappa(2a-k_1)_\lambda +2(2a-k_1)_\kappa (-a+k_1)_\lambda +(2a-k_1)_\kappa a_\lambda ]\\
=&\frac{1}{6}i\pi^2[(a-2k_1)_\kappa(2a-k_1)_\lambda +(2a-k_1)_\kappa (-a+2k_1)_\lambda ]\\
=&\frac{1}{6}i\pi^2[-a_\kappa k_{1\lambda}-4k_{1\kappa}a_{\lambda}+k_{1\kappa}a_\lambda +4a_\kappa k_{1\lambda}] \\
=&\frac{1}{2}i\pi^2[a_\kappa k_{1\lambda}- k_{1\kappa}a_\lambda]
\end{align*}
と,対称性より
\begin{align*}
&I_{\kappa\lambda}(b-a-k_2,a,a+k_2)=I_{\kappa\lambda}(2b-k_2,-b,-b+k_2) \\
=&\frac{1}{2}i\pi^2[b_\kappa k_{2\lambda}- k_{2\kappa}b_\lambda] \\
=&\frac{1}{2}i\pi^2[-a_\kappa k_{2\lambda}+ k_{2\kappa}a_\lambda]
\end{align*}
を用いると
\begin{align*}
&\left[ \frac{\partial}{\partial x^\mu}\Gamma^{\mu\nu\rho}_{\alpha\beta\gamma}(x,y,z) \right]_{\mr{anom}} \\
=&\frac{-2i}{(2\pi)^{12}}D_{\alpha\beta\gamma}\int d^4k_1 d^4k_2e^{-i(k_1+k_2)\cdot x}e^{ik_1\cdot y}e^{ik_2\cdot z} \\
&\times \biggl\{ \epsilon^{\kappa\nu\lambda\rho} I_{\kappa\lambda}(a-b-k_1,b,b+k_1) +\epsilon^{\kappa\rho\lambda\nu} I_{\kappa\lambda}(b-a-k_2,a,a+k_2) \biggr\} \\
=&\frac{\pi^2}{2(2\pi)^{12}}D_{\alpha\beta\gamma}\int d^4k_1 d^4k_2e^{-i(k_1+k_2)\cdot x}e^{ik_1\cdot y}e^{ik_2\cdot z} \\
&\times \biggl\{ \epsilon^{\kappa\nu\lambda\rho} [a_\kappa k_{1\lambda}- k_{1\kappa}a_\lambda] +\epsilon^{\kappa\rho\lambda\nu} [-a_\kappa k_{2\lambda}+ k_{2\kappa}a_\lambda] \biggr\} \\
=&\frac{2}{(2\pi)^{12}}D_{\alpha\beta\gamma}\int d^4k_1 d^4k_2e^{-i(k_1+k_2)\cdot x}e^{ik_1\cdot y}e^{ik_2\cdot z}\pi^2\epsilon^{\kappa\nu\lambda\rho}a_{\kappa}(k_1+k_2)_\lambda
\end{align*}
となる.\par
$J^\mu_\alpha(x)$のカレントでは,$a\propto k_1+k_2$として$\epsilon^{\kappa\nu\lambda\rho}$の完全反対称によりアノマリー(22.3.22)を消去できる.これは可能だが,アノマリーが全て消えるわけではない.アノマリーは$(\partial/\partial y^\nu)\Gamma^{\mu\nu\rho}_{\alpha\beta\gamma}(x,y,z)$か$(\partial/\partial z^\rho)\Gamma^{\mu\nu\rho}_{\alpha\beta\gamma}(x,y,z)$に現れる.以前見たように,$(\partial/\partial y^\nu)\Gamma^{\mu\nu\rho}_{\alpha\beta\gamma}(x,y,z)$では(22.3.15)の第一項目の$I_{\kappa\lambda}(a-b-k_1,b,b+k_1)$で$a\to a'=a-k_1,b\to b'=b+k_1,k_1\to k'_1=-k_1-k_2$と置き換えたものと等しいのであったから,
\begin{align*}
&I_{\kappa\lambda}(a'-b'-k_1,b',b'+k'_1)=I_{\kappa\lambda}(2a'-k_1,-a',-a'+k_1) \\
=&\frac{1}{2}i\pi^2[a'_\kappa k'_{1\lambda}- k'_{1\kappa}a'_\lambda] \\
=&\frac{1}{2}i\pi^2[(a-k_1)_\kappa (-k_1-k_2)_{1\lambda}- (-k_1-k_2)_{1\kappa}(a-k_1)_\lambda] \\
=&\frac{1}{2}i\pi^2[-(a-k_1)_\kappa (k_1+k_2)_{1\lambda}+(k_1+k_2)_{1\kappa}(a-k_1)_\lambda] \\
=&\frac{1}{2}i\pi^2[-a_\kappa (k_1+k_2)_{1\lambda}+k_{1\kappa}k_{2\lambda}+(k_1+k_2)_{1\kappa}a_\lambda-k_{2\kappa}k_{1\lambda}] \\
\end{align*}
を用いて
\begin{align*}
&\left[ \frac{\partial}{\partial y^\nu}\Gamma^{\mu\nu\rho}_{\alpha\beta\gamma}(x,y,z) \right]_{\mr{anom}} \\
=&\frac{2i}{(2\pi)^{12}}D_{\alpha\beta\gamma}\int d^4k_1 d^4k_2e^{-i(k_1+k_2)\cdot x}e^{ik_1\cdot y}e^{ik_2\cdot z} \\
&\times \biggl\{ \epsilon^{\kappa\mu\lambda\rho} I_{\kappa\lambda}(a'-b'-k'_1,b',b'+k'_1) +\epsilon^{\kappa\rho\lambda\mu} I_{\kappa\lambda}(b-a-k_2,a,a+k_2) \biggr\} \\
=&\frac{-\pi^2}{(2\pi)^{12}}D_{\alpha\beta\gamma}\int d^4k_1 d^4k_2e^{-i(k_1+k_2)\cdot x}e^{ik_1\cdot y}e^{ik_2\cdot z} \\
&\times \biggl\{ \epsilon^{\kappa\mu\lambda\rho} [-a_\kappa (k_1+k_2)_{1\lambda}+k_{1\kappa}k_{2\lambda}+(k_1+k_2)_{1\kappa}a_\lambda-k_{2\kappa}k_{1\lambda}] \\
& +\epsilon^{\kappa\rho\lambda\mu} [-a_\kappa k_{2\lambda}+ k_{2\kappa}a_\lambda] \biggr\} \\
=&\frac{-\pi^2}{(2\pi)^{12}}D_{\alpha\beta\gamma}\int d^4k_1 d^4k_2e^{-i(k_1+k_2)\cdot x}e^{ik_1\cdot y}e^{ik_2\cdot z} \\
&\times \biggl\{ \epsilon^{\kappa\mu\lambda\rho} [-2a_\kappa (k_1+k_2)_{1\lambda}+2k_{1\kappa}k_{2\lambda}] +2\epsilon^{\kappa\mu\lambda\rho} a_\kappa k_{2\lambda} \biggr\} \\
=&\frac{-2}{(2\pi)^{12}}D_{\alpha\beta\gamma}\int d^4k_1 d^4k_2e^{-i(k_1+k_2)\cdot x}e^{ik_1\cdot y}e^{ik_2\cdot z} \pi^2 \epsilon^{\kappa\rho\lambda\mu}(a+k_1)_\kappa k_{2\lambda}
\end{align*}
となるが,これが消えるのは$a+k_1\propto k_2$のときだ.同様に,$(\partial/\partial z^\rho)\Gamma^{\mu\nu\rho}_{\alpha\beta\gamma}(x,y,z)$の場合も,今度は(22.3.15)の第二項目で$a\to a''=a+k_2,b\to b''=b-k_2,k''_2=-k_1-k_2$と置き換えたものと等しいのであったから
\begin{align*}
&I_{\kappa\lambda}(b''-a''-k''_2,a'',a''+k''_2)=I_{\kappa\lambda}(2b''-k''_2,-b'',-b''+k''_2) \\
=&\frac{1}{2}i\pi^2[-a''_\kappa k''_{2\lambda}+ k''_{2\kappa}a''_\lambda] \\
=&\frac{1}{2}i\pi^2[-(a+k_2)_\kappa (-k_1-k_2)_{\lambda}+ (-k_1-k_2)_{\kappa}(a+k_2)_\lambda] \\
=&\frac{1}{2}i\pi^2[(a+k_2)_\kappa (k_1+k_2)_{\lambda}- (k_1+k_2)_{\kappa}(a+k_2)_\lambda] \\
=&\frac{1}{2}i\pi^2[a_\kappa (k_1+k_2)_{\lambda}+k_{2\kappa}k_{1\lambda}- (k_1+k_2)_{\kappa}a_\lambda-k_{1\kappa}k_{2\lambda}]
\end{align*}
を用いて
\begin{align*}
&\left[ \frac{\partial}{\partial z^\rho}\Gamma^{\mu\nu\rho}_{\alpha\beta\gamma}(x,y,z) \right]_{\mr{anom}} \\
=&\frac{2i}{(2\pi)^{12}}D_{\alpha\beta\gamma}\int d^4k_1 d^4k_2e^{-i(k_1+k_2)\cdot x}e^{ik_1\cdot y}e^{ik_2\cdot z} \\
&\times \biggl\{ \epsilon^{\kappa\nu\lambda\mu} I_{\kappa\lambda}(a-b-k_1,b,b+k_1) +\epsilon^{\kappa\mu\lambda\nu} I_{\kappa\lambda}(b''-a''-k''_2,a'',a''+k''_2) \biggr\} \\
=&\frac{-\pi^2}{(2\pi)^{12}}D_{\alpha\beta\gamma}\int d^4k_1 d^4k_2e^{-i(k_1+k_2)\cdot x}e^{ik_1\cdot y}e^{ik_2\cdot z} \\
&\times \biggl\{ \epsilon^{\kappa\nu\lambda\mu}[a_\kappa k_{1\lambda}- k_{1\kappa}a_\lambda] \\
& +\epsilon^{\kappa\mu\lambda\nu} [a_\kappa (k_1+k_2)_{\lambda}+k_{2\kappa}k_{1\lambda}- (k_1+k_2)_{\kappa}a_\lambda-k_{1\kappa}k_{2\lambda}] \biggr\} \\
=&\frac{-\pi^2}{(2\pi)^{12}}D_{\alpha\beta\gamma}\int d^4k_1 d^4k_2e^{-i(k_1+k_2)\cdot x}e^{ik_1\cdot y}e^{ik_2\cdot z} \\
&\times \biggl\{ -2\epsilon^{\kappa\mu\lambda\nu}a_\kappa k_{1\lambda}+\epsilon^{\kappa\mu\lambda\nu}[2a_\kappa (k_1+k_2)_{\lambda}+2k_{2\kappa}k_{1\lambda}] \biggr\} \\
=&\frac{-2}{(2\pi)^{12}}D_{\alpha\beta\gamma}\int d^4k_1 d^4k_2e^{-i(k_1+k_2)\cdot x}e^{ik_1\cdot y}e^{ik_2\cdot z} \pi^2 \epsilon^{\kappa\mu\lambda\nu}(a-k_2)_\kappa k_{1\lambda}
\end{align*}
となるが,これが消えるのは$a-k_2\propto k_1$のときだ.$a^\mu$をこれらの条件のどれか二つを満たすように選び,アノマリーがカレントのどれか二つから取り除かれるようにすることができる.しかし$k_1$と$k_2$が平行でなければ,三つの条件$a\propto k_1+k_2,a+k_1\propto k_2,a-k_2\propto k_1$を同時に満たすことは不可能だ.(たとえば,最初の二つの条件は$a=-k_1-k_2$を意味するが,これは三番目の条件に反する.)したがって,どのカレントがアノマリーを持つかを決める自由度はある程度存在するものの,$D_{\alpha\beta\gamma}$がゼロでなければカレント$J^\mu_\alpha(x),J^\nu_\beta(y),J^\rho_\gamma(z)$のどれか最低一つはアノマリーを持つ.これが上の計算の主要な結果だ.

\vskip\baselineskip

非常に重要な問題として,$J^\mu_\alpha(x)$が大域的対称性のカレントであり,$J^\nu_\beta(y)$と$J^\rho_\gamma(z)$がゲージ対称性のカレント,つまりゲージ場が結合するカレントである場合がある.(そのような問題は前節扱った.そのときは$J^\mu_\alpha(x)$が大域的カイラル対称性のカレントであった.)このような場合は$a^\mu$を$J^\nu_\beta(y)$や$J^\rho_\gamma(z)$ではなく,$J^\mu_\alpha(x)$のみにアノマリーが生じるように選ばなければならない.既に見たように,このためには$a+k_1\propto k_2$と$a-k_2\propto k_1$が満たされなければならず
\begin{align*}
a=k_2-k_1
\end{align*}
という唯一の結果が得られる.この$a^\mu$の値を採用して,アノマリー(22.3.22)は
\begin{align*}
&\left[ \frac{\partial}{\partial x^\mu}\Gamma^{\mu\nu\rho}_{\alpha\beta\gamma}(x,y,z) \right]_{\mr{anom}} \\
=&\frac{2}{(2\pi)^{12}}D_{\alpha\beta\gamma}\int d^4k_1 d^4k_2e^{-i(k_1+k_2)\cdot x}e^{ik_1\cdot y}e^{ik_2\cdot z} \pi^2\epsilon^{\kappa\nu\lambda\rho}(k_2-k_1)_{\kappa}(k_1+k_2)_\lambda \\
=&\frac{-1}{(2\pi)^{12}}D_{\alpha\beta\gamma}\int d^4k_1 d^4k_2e^{-i(k_1+k_2)\cdot x}e^{ik_1\cdot y}e^{ik_2\cdot z}4\pi^2\epsilon^{\kappa\nu\lambda\rho}k_{1\kappa}k_{2\lambda} \\
=&\frac{1}{(2\pi)^{12}}D_{\alpha\beta\gamma}\int d^4k_1 d^4k_2\frac{\partial}{\partial y^\kappa}e^{ik_1\cdot (y-x)}\frac{\partial}{\partial z^\rho}e^{ik_2\cdot (z-x)}4\pi^2\epsilon^{\kappa\nu\lambda\rho} \\
=&\frac{1}{(2\pi)^{4}}D_{\alpha\beta\gamma}\frac{\partial \delta^4(y-x)}{\partial y^\kappa}\frac{\partial\delta^4(z-x)}{\partial z^\rho}4\pi^2\epsilon^{\kappa\nu\lambda\rho} \\
=&\frac{1}{4\pi^2}D_{\alpha\beta\gamma}\epsilon^{\kappa\nu\lambda\rho}\frac{\partial \delta^4(y-x)}{\partial y^\kappa}\frac{\partial\delta^4(z-x)}{\partial z^\rho}
\end{align*}
となる.\par
この結果をカレント$J^\nu_\beta$と$J^\rho_\gamma$に結合するゲージ場があるときのカレント$J^\mu_\alpha$の真空期待値を使って表す.
\begin{align*}
\braket{J^\mu_\alpha(x)}&=\int[d\psi][d\bar{\psi}]J^\mu_\alpha(x) \exp\left(i\int d^4 x' \mc{L}(x')\right) \\
&=\int[d\psi][d\bar{\psi}]\sum_{N=0}^\infty\frac{1}{N!}\int d^4x_1 d^4x_2\cdots d^4x_N \, T\left\{ J^\mu_\alpha(x),i\mc{L}(x_1),i\mc{L}(x_2),\cdots,i\mc{L}(x_N) \right\}  \\
&=\sum_{N=0}^\infty\frac{1}{N!}\int d^4x_1 d^4x_2\cdots d^4x_N \braket{ T\left\{ J^\mu_\alpha(x),i\mc{L}(x_1),i\mc{L}(x_2),\cdots,i\mc{L}(x_N) \right\}}_{\VAC}
\end{align*}
このとき,物質場の共変微分から生じるゲージ場と結合するカレントから,三角ダイアグラムによる以下の寄与が存在する.
\begin{align*}
\braket{J^\mu_\alpha(x)}_{\triangle}=&i^2\frac{1}{2!}\int d^4y d^4z \braket{ T\left\{ J^\mu_{\alpha}(x),J^\nu_\beta(y)A^\beta_\nu(y),J^\rho_\gamma(z)A^\gamma_\rho(z) \right\} }_{\VAC} \\
=&-\frac{1}{2}\int d^4y d^4z \braket{T\left\{ J^\mu_\alpha(x),J^\nu_\beta(y),J^\rho_\gamma(z) \right\}}A^\beta_\nu(y)A^\gamma_\rho(z) \\
=&-\frac{1}{2}\int d^4y d^4z \Gamma^{\mu\nu\rho}_{\alpha\beta\gamma}(x,y,z) A^\beta_\nu(y)A^\gamma_\rho(z)
\end{align*}
(22.3.24)を用いると,これはアノマリーな発散
\begin{align*}
\left[\braket{\partial_\mu J^\mu_\alpha(x)}_\triangle \right]_{\mr{anom}}=&-\frac{1}{2}\int d^4y d^4z \left[\frac{\partial}{\partial x^\mu}\Gamma^{\mu\nu\rho}_{\alpha\beta\gamma}(x,y,z)\right] A^\beta_\nu(y)A^\gamma_\rho(z) \\
=&-\frac{1}{8\pi^2} \int d^4y d^4z D_{\alpha\beta\gamma} \epsilon^{\kappa\nu\lambda\rho}\frac{\partial \delta^4(y-x)}{\partial y^\kappa}\frac{\partial \delta^4(z-x)}{\partial z^\rho} A^\beta_\nu(y)A^\gamma_\rho(z) \\
=&-\frac{1}{8\pi^2} \int d^4y d^4z D_{\alpha\beta\gamma} \epsilon^{\kappa\nu\lambda\rho}\frac{\partial A^\beta_\nu(y)}{\partial y^\kappa}\frac{\partial A^\gamma_\rho(z)}{\partial z^\rho} \delta^4(y-x)  \delta^4(z-x) \\
&=-\frac{1}{8\pi^2}D_{\alpha\beta\gamma}\epsilon^{\kappa\nu\lambda\rho}\partial_\kappa A^\beta_\nu(x) \partial_\lambda A^\gamma_\rho(x)
\end{align*}
を持つことがわかる.他に,四角形ダイアグラムや五角形ダイアグラムからもアノマリーを生じる.ゲージ不変性から,それらダイアグラムの和全体は以下のゲージ不変な結果にならなければならない.
\begin{align*}
\left[\braket{\partial_\mu J^\mu_\alpha(x)} \right]_{\mr{anom}}=-\frac{1}{32\pi^2}D_{\alpha\beta\gamma}\epsilon^{\kappa\nu\lambda\rho}F^\beta_{\kappa\nu}(x)F^\gamma_{\lambda\rho}(x)
\end{align*}
(係数が$1/8\pi^2$から$1/32\pi^2$になっているのは,(22.3.27)を展開し$\epsilon^{\kappa\nu\lambda\rho}$の反対称性を用いたときに(22.3.26)を再現するようにだ.実際
\begin{align*}
&-\frac{1}{32\pi^2}D_{\alpha\beta\gamma}\epsilon^{\kappa\nu\lambda\rho}F^\beta_{\kappa\nu}(x)F^\gamma_{\lambda\rho}(x) \\
=&-\frac{1}{32\pi^2}D_{\alpha\beta\gamma}\epsilon^{\kappa\nu\lambda\rho}\left( \partial_\kappa A^\beta_\nu(x) -\partial_\nu A^\beta_\kappa(x)+C_{\beta\delta\epsilon}A^\delta_\kappa(x)A^\epsilon_\nu(x) \right) \\
&\qquad \times\left( \partial_\lambda A^\gamma_\rho(x) -\partial_\rho A^\gamma_\lambda(x)+C_{\gamma\delta'\epsilon'}A^{\delta'}_\lambda(x)A^{\epsilon'}_\rho(x) \right) \\
=&-\frac{1}{32\pi^2}D_{\alpha\beta\gamma}\epsilon^{\kappa\nu\lambda\rho}\left( 2\partial_\kappa A^\beta_\nu(x)+C_{\beta\delta\epsilon}A^\delta_\kappa(x)A^\epsilon_\nu(x) \right) \left( 2\partial_\lambda A^\gamma_\rho(x)+C_{\gamma\delta'\epsilon'}A^{\delta'}_\lambda(x)A^{\epsilon'}_\rho(x) \right) \\
=&-\frac{1}{8\pi^2}D_{\alpha\beta\gamma}\epsilon^{\kappa\nu\lambda\rho}\partial_\kappa A^\beta_\nu(x) \partial_\lambda A^\gamma_\rho(x)+\cdots
\end{align*}
となる.)\par
確認のためにフェルミオン数を保存する理論を考え,その生成子$T_\alpha$が(22.3.4)の形だとする.アノマリー(22.3.26)の定数$D_{\alpha\beta\gamma}$は以下で与えられる.
\begin{align*}
D_{\alpha\beta\gamma}=&\frac{1}{2}\mr{tr}\left[\left\{T_\alpha,T_\beta\right\} T_\gamma \right] \\
=&\frac{1}{2}\mr{tr}\left[ T_\alpha T_\beta T_\gamma \right]+\frac{1}{2}\mr{tr}\left[ T_\beta T_\alpha T_\gamma \right] \\
=&\frac{1}{2}\mr{tr}\left[ t_\alpha^L t_\beta^L t_\gamma^L \right]-\frac{1}{2}\mr{tr}\left[ (t_\alpha^R)^T (t_\beta^R)^T (t_\gamma^R)^T \right]+\frac{1}{2}\mr{tr}\left[ t_\beta^L t_\alpha^L t_\gamma^L \right]-\frac{1}{2}\mr{tr}\left[ (t_\beta^R)^T (t_\alpha^R)^T (t_\gamma^R)^T \right] \\
=&\frac{1}{2}\mr{tr}\left[ t_\alpha^L t_\beta^L t_\gamma^L \right]-\frac{1}{2}\mr{tr}\left[ t_\gamma^R t_\beta^R t_\alpha^R \right]+\frac{1}{2}\mr{tr}\left[ t_\beta^L t_\alpha^L t_\gamma^L \right]-\frac{1}{2}\mr{tr}\left[ t_\gamma^R t_\alpha^R t_\beta^R \right]  \\
=&\frac{1}{2}\mr{tr}\left[\left\{t^L_\alpha,t^L_\beta\right\} t^L_\gamma \right]-\frac{1}{2}\mr{tr}\left[\left\{t^R_\alpha,t^R_\beta\right\} t^R_\gamma \right]
\end{align*}
特に,22.2節でゲージ場の$t^L_\beta=t^R_\beta \equiv t_\beta$として(すなわち,(22.3.2)より$\delta \psi=i\theta_\beta t_\beta \psi$という変換),また$t_\gamma$についても同様にとったベクトルカレント$J^\nu_\beta$と$J^\rho_\gamma$(22.2節では$J^\nu_\alpha$と$J^\rho_\beta$と呼んだ)との相互作用による$t^L=-t^R\equiv t$(すなわち$\delta \psi=i\theta \gamma_5 t\psi$)の軸性ベクトルカレント$J^\mu_5$の発散を計算した.したがって,この場合には
\begin{align*}
D_{\alpha\beta\gamma}=&\frac{1}{2}\mr{tr}\left[\left\{t^L_\alpha,t^L_\beta\right\} t^L_\gamma \right]-\frac{1}{2}\mr{tr}\left[\left\{t^R_\alpha,t^R_\beta\right\} t^R_\gamma \right] \\
=&\frac{1}{2}\mr{tr}\left[\left\{t,t_\beta \right\} t_\gamma \right]-\frac{1}{2}\mr{tr}\left[\left\{(-t),t_\beta\right\} t_\gamma \right] \\
=&\mr{tr}\left[\left\{t,t_\beta \right\} t_\gamma \right]=\mr{tr}\left[\left\{t_\beta , t_\gamma \right\} t \right]\quad \because 完全対称性
\end{align*}
で置き換えられ,(22.3.27)は
\begin{align*}
\left[\braket{\partial_\mu J^\mu_5(x)} \right]_{\mr{anom}}=&-\frac{1}{32\pi^2}\mr{tr}\left[\left\{t_\beta , t_\gamma \right\} t \right] \epsilon^{\kappa\nu\lambda\rho}F^\beta_{\kappa\nu}(x)F^\gamma_{\lambda\rho}(x) \\
=&-\frac{1}{32\pi^2}\mr{tr}\left[t_\beta t_\gamma t \right] \epsilon^{\kappa\nu\lambda\rho}F^\beta_{\kappa\nu}(x)F^\gamma_{\lambda\rho}(x) -\frac{1}{32\pi^2}\mr{tr}\left[ t_\gamma t_\beta t \right] \epsilon^{\kappa\nu\lambda\rho}F^\beta_{\kappa\nu}(x)F^\gamma_{\lambda\rho}(x) \\
=&-\frac{1}{32\pi^2}\mr{tr}\left[t_\beta t_\gamma t \right] \epsilon^{\kappa\nu\lambda\rho}F^\beta_{\kappa\nu}(x)F^\gamma_{\lambda\rho}(x) -\frac{1}{32\pi^2}\mr{tr}\left[t_\gamma t_\beta t \right] \epsilon^{\lambda\rho\kappa\nu}F^\gamma_{\lambda\rho}(x)F^\beta_{\kappa\nu}(x) \\
=&-\frac{1}{32\pi^2}\mr{tr}\left[t_\beta t_\gamma t \right] \epsilon^{\kappa\nu\lambda\rho}F^\beta_{\kappa\nu}(x)F^\gamma_{\lambda\rho}(x) -\frac{1}{32\pi^2}\mr{tr}\left[t_\beta t_\gamma t \right] \epsilon^{\kappa\nu\lambda\rho}F^\beta_{\kappa\nu}(x)F^\gamma_{\lambda\rho}(x) \\
=&-\frac{1}{16\pi^2}\mr{tr}\left[t_\beta t_\gamma t \right] \epsilon^{\kappa\nu\lambda\rho}F^\beta_{\kappa\nu}(x)F^\gamma_{\lambda\rho}(x)
\end{align*}
となり,(22.2.26)と一致する.

\vskip\baselineskip

どんなゲージ場も$J^\mu_\alpha(x),J^\nu_\beta(y),J^\rho_\gamma(z)$のいずれかのカレントとも結合していない場合は,ずらすベクトル場$a^\mu$の選び方は便宜的なものでしかない.なぜなら,ゲージ場が結合するカレントからアノマリーが生じることは禁止されるが,全てのカレントがそうでないならばどのカレントからアノマリーが生じても問題はないからだ.もし,全部ではなく幾つかのカレントが「自発的に破れない」対称性に伴っていれば,破れていない対称性に対応したカレントにアノマリーがないように$a^\mu$を選んで,破れていない対称性を自明に保つ.\par
たとえば,量子色力学(あるいはそれと同等の理論)では,カイラル$SU(3)\times SU(3)$のような大域的対称性の生成子$T_\alpha$が(22.3.4)の形となり,全てのカレントは破れていない対称性に相当する$t^R_\alpha=t^L_\alpha$の(つまり$\delta \psi=i\theta_\alpha t_\alpha \psi$のときの)ベクトルカレントか,破れている対称性に相当する$t^R_\alpha=-t^L_\alpha$の(つまり$\delta \psi=i\theta_\alpha \gamma_5 t_\alpha \psi$)軸性ベクトルカレントか,だ.\par
(22.3.28)から,三つのカレントが全てベクトルカレントである場合は
\begin{align*}
D_{\alpha\beta\gamma}=&\frac{1}{2}\mr{tr}[\{ t^L_\alpha,t^L_\beta \} t^L_\gamma ]-\frac{1}{2}\mr{tr}[\{ t^L_\alpha,t^L_\beta \} t^L_\gamma ] \\
=&\frac{1}{2}\mr{tr}[\{ t^L_\alpha,t^L_\beta \} t^L_\gamma ]-\frac{1}{2}\mr{tr}[\{ t^L_\alpha,t^L_\beta \} t^L_\gamma ] \quad \because t^L_\alpha=t^R_\alpha \\
=&0
\end{align*}
となってアノマリーは生じない.一つのベクトルカレントと二つの軸性カレントの場合も,$\alpha,\beta$が軸性カレントの添え字で$\gamma$がベクトルカレントの添え字だとすると
\begin{align*}
D_{\alpha\beta\gamma}=&\frac{1}{2}\mr{tr}[\{ t^L_\alpha,t^L_\beta \} t^L_\gamma ]-\frac{1}{2}\mr{tr}[\{ t^L_\alpha,t^L_\beta \} t^L_\gamma ] \\
=&\frac{1}{2}\mr{tr}[\{ t^L_\alpha,t^L_\beta \} t^L_\gamma ]-\frac{1}{2}\mr{tr}[\{ (-t^L_\alpha),(-t^L_\beta) \} t^L_\gamma ] \quad \because t^L_\alpha=-t^R_\alpha , t^L_\beta=-t^R_\beta , t^L_\gamma=t^R_\gamma \\
=&0
\end{align*}
となってアノマリーは生じない.(ベクトルカレントに対応する添え字が$\alpha,\beta,\gamma$のどれであっても,それぞれの項は完全対称であるからどの場合でもゼロとなる.)一方,前に計算したような,二つのベクトルカレントと一つの軸性カレントの場合はアノマリーを持つ.
\begin{align*}
D_{\alpha\beta\gamma}=&\frac{1}{2}\mr{tr}[\{ t^L_\alpha,t^L_\beta \} t^L_\gamma ]-\frac{1}{2}\mr{tr}[\{ t^L_\alpha,t^L_\beta \} t^L_\gamma ] \\
=&\frac{1}{2}\mr{tr}[\{ t^L_\alpha,t^L_\beta \} t^L_\gamma ]-\frac{1}{2}\mr{tr}[\{ (-t^L_\alpha),t^L_\beta \} t^L_\gamma ] \quad \because t^L_\alpha=-t^R_\alpha , t^L_\beta=t^R_\beta , t^L_\gamma=t^R_\gamma \\
=&\mr{tr}[\{ t^L_\alpha,t^L_\beta \} t^L_\gamma ] \neq 0
\end{align*}
さらに,軸性カレントが三つの場合もアノマリーを持つ.
\begin{align*}
D_{\alpha\beta\gamma}=&\frac{1}{2}\mr{tr}[\{ t^L_\alpha,t^L_\beta \} t^L_\gamma ]-\frac{1}{2}\mr{tr}[\{ t^L_\alpha,t^L_\beta \} t^L_\gamma ] \\
=&\frac{1}{2}\mr{tr}[\{ t^L_\alpha,t^L_\beta \} t^L_\gamma ]-\frac{1}{2}\mr{tr}[\{ (-t^L_\alpha),(-t^L_\beta) \} (-t^L_\gamma) ] \quad \because t^L_\alpha=-t^R_\alpha , t^L_\beta=-t^R_\beta , t^L_\gamma=-t^R_\gamma \\
=&\mr{tr}[\{ t^L_\alpha,t^L_\beta \} t^L_\gamma ]\neq 0
\end{align*}
軸性カレント一つとベクトルカレント二つの場合には,アノマリーがベクトルカレントの保存則を妨げないように$a^\mu$を選ぶ.つまり,今回計算したように,もし$J^\mu_\alpha(x)$が軸性カレントで,$J^\nu_\beta(y),J^\rho_\gamma(z)$が二つのベクトルカレントならば(22.3.23)と同じように$a^\mu=k_2^\mu-k_1^\mu$ととり,アノマリーが(22.3.24)で与えられるようにする.一方,三つの軸性ベクトルカレントの場合には,そのどれもアノマリーを持たないことを要請する理由はない.そのため,三つのカレントに対するアノマリーが対称的になるようにして$a^\mu$を与えるのが自然だ.ローレンツ不変性から,$\alpha,\beta$を定数として$a=\alpha k_1+\beta k_2$を試してみることにする.それぞれのアノマリーは
\begin{align*}
\left[ \frac{\partial}{\partial x^\mu}\Gamma^{\mu\nu\rho}_{\alpha\beta\gamma}(x,y,z) \right]_{\mr{anom}} &=\frac{2}{(2\pi)^{12}}D_{\alpha\beta\gamma}\int d^4k_1 d^4k_2e^{-i(k_1+k_2)\cdot x}e^{ik_1\cdot y}e^{ik_2\cdot z}\pi^2\epsilon^{\kappa\nu\lambda\rho}a_{\kappa}(k_1+k_2)_\lambda \\
&=\frac{2}{(2\pi)^{12}}D_{\alpha\beta\gamma}\int d^4k_1 d^4k_2e^{-i(k_1+k_2)\cdot x}e^{ik_1\cdot y}e^{ik_2\cdot z}\pi^2\epsilon^{\kappa\nu\lambda\rho}(\alpha-\beta)k_{1\kappa}k_{2\lambda} \\
\left[ \frac{\partial}{\partial y^\nu}\Gamma^{\mu\nu\rho}_{\alpha\beta\gamma}(x,y,z) \right]_{\mr{anom}} &=\frac{-2}{(2\pi)^{12}}D_{\alpha\beta\gamma}\int d^4k_1 d^4k_2e^{-i(k_1+k_2)\cdot x}e^{ik_1\cdot y}e^{ik_2\cdot z} \pi^2 \epsilon^{\kappa\rho\lambda\mu}(a+k_1)_\kappa k_{2\lambda} \\
&=\frac{-2}{(2\pi)^{12}}D_{\alpha\beta\gamma}\int d^4k_1 d^4k_2e^{-i(k_1+k_2)\cdot x}e^{ik_1\cdot y}e^{ik_2\cdot z} \pi^2 \epsilon^{\kappa\rho\lambda\mu}(\alpha+1)k_{1\kappa} k_{2\lambda} \\
\left[ \frac{\partial}{\partial z^\rho}\Gamma^{\mu\nu\rho}_{\alpha\beta\gamma}(x,y,z) \right]_{\mr{anom}} &=\frac{-2}{(2\pi)^{12}}D_{\alpha\beta\gamma}\int d^4k_1 d^4k_2e^{-i(k_1+k_2)\cdot x}e^{ik_1\cdot y}e^{ik_2\cdot z} \pi^2 \epsilon^{\kappa\mu\lambda\nu}(a-k_2)_\kappa k_{1\lambda} \\
&=\frac{-2}{(2\pi)^{12}}D_{\alpha\beta\gamma}\int d^4k_1 d^4k_2e^{-i(k_1+k_2)\cdot x}e^{ik_1\cdot y}e^{ik_2\cdot z} \pi^2 \epsilon^{\kappa\rho\lambda\mu}(1-\beta)k_{1\kappa} k_{2\lambda}
\end{align*}
であったが,これらが等しくなるという要請より$\alpha=-\beta=-1/3$となる.したがって
\begin{align*}
a=\frac{1}{3}(k_2-k_1)
\end{align*}
となる.これを(22.3.22)に使い(22.3.23)と比較すると,三つの軸性ベクトルカレントのアノマリーは一つの軸性ベクトルカレントと二つのベクトルカレントのものの3分の1だということがわかる.

\vskip\baselineskip

カレントの発散には,さらに22.2図のダイアグラムからのアノマリーも含まれる.(22.3.26)の通り,三角ダイアグラムさえ$\braket{\partial_\mu J^\mu_\alpha(x)}=0$となるような保存カレントを与えないので,ここではゲージ不変性はなんの指針にもならない.強い相互作用の$SU(3)\times SU(3)$カイラル対称性の全アノマリーは,ベクトルカレントが保存され,軸性カレントがそれらのカレントについて対称なように運動量をずらすベクトル$a^\mu$を選んで,バーディーンによって計算された.量子色力学では,この$SU(3)\times SU(3)$(これはクォークの色ではなくフレーバーに働く)はゲージ化されていないが,このアノマリーをベクトル場$V^\mu_a(x)$の8重項と軸性ベクトル場$A^\mu_a$の8重項に強く結合する仮想的なゲージ場の汎関数$\Gamma[V,A]$におけるゲージ不変性の破れとして表すと便利だ.また,非カイラル$SU(3)$とカイラル$SU(3)$の微小ゲージ変換は(15.1.17)よりそれぞれ
\begin{align*}
t_a(V_a^\mu+\gamma_5 A_a^\mu)_\epsilon =& \exp(it_b\epsilon_b)t_a(V_a^\mu+\gamma_5 A_a^\mu)\exp(-it_b \epsilon_b) -i[\partial^\mu \exp(it_b \epsilon_b)]\exp(-it_b \epsilon_b) \\
=&t_a(V_a^\mu+\gamma_5 A_a^\mu)+[it_c\epsilon_c,t_b](V_b^\mu+\gamma_5 A_b^\mu)+t_b\partial_\mu \epsilon_b \quad \because \mr{BCH}公式\\
=&t_a(V_a^\mu+\gamma_5 A_a^\mu)+f_{abc}t_a \epsilon_c (V_b^\mu+\gamma_5 A_b^\mu)+t_a\partial_\mu \epsilon_a \\
\therefore \delta(V_a^\mu+\gamma_5 A_a^\mu)=&\partial^\mu \epsilon_a+f_{abc} \epsilon_c (V_b^\mu+\gamma_5 A_b^\mu) \\
t_a(V_a^\mu+\gamma_5 A_a^\mu)_{5}=&\exp(i\gamma_5 t_b\epsilon_b)t_a(V_a^\mu+\gamma_5 A_a^\mu)\exp(-i\gamma_5 t_b \epsilon_b) -i[\partial^\mu \exp(i\gamma_5 t_b \epsilon_b)]\exp(-i\gamma_5 t_b \epsilon_b) \\
=&t_a(V_a^\mu+\gamma_5 A_a^\mu)+[i\gamma_5 t_c \epsilon_c , t_b](V_b^\mu+\gamma_5 A_b^\mu)+\gamma_5 t_a \partial \epsilon_a \quad \because \mr{BCH}公式\\
=&t_a(V_a^\mu+\gamma_5 A_a^\mu)+\gamma_5 f_{abc}t_a \epsilon_c (V_b^\mu+\gamma_5 A_b^\mu)+\gamma_5 t_a \partial \epsilon_a \\
\therefore \delta_5 (V_a^\mu+\gamma_5 A_a^\mu)=&\partial^\mu (\gamma_5 \epsilon_a) +\gamma_5 f_{abc} \epsilon_c (V_b^\mu+\gamma_5 A_b^\mu)
\end{align*}
となる.ここで$\delta$は生成子$t_a$による非カイラルゲージ変換を表し,$\delta_5$は生成子$\gamma_5t_a$によるカイラルゲージ変換を表す.$\gamma_5$の係数比較により
\begin{align*}
&\delta V^\mu_a=\partial^\mu \epsilon_a+f_{abc}\epsilon_c V^\mu_b,\quad \delta A^\mu_a=f_{abc}\epsilon_c A^\mu_b \\
&\delta_5 V^\mu_a=f_{abc}\epsilon_c A^\mu_b,\quad \delta_5 A^\mu_a=\partial^\mu \epsilon_a+f_{abc}\epsilon_c V^\mu_b
\end{align*}
となる.したがって微小ゲージ変換演算子は
\begin{align*}
\int d^4x i\epsilon_a(x) \mc{Y}_a(x)=&\int d^4x\, \delta V_{a\mu}(x) \frac{\delta}{\delta V_{a\mu}(x)}+\int d^4x \, \delta A_{a\mu}(x)\frac{\delta}{\delta A_{a\mu}(x)} \\
=&\int d^4x \left[\partial_\mu \epsilon_a(x)+f_{abc}\epsilon_c(x) V_{b\mu}(x)\right] \frac{\delta}{\delta V_{a\mu}(x)} +\int d^4x \, f_{abc}\epsilon_c(x) A_{b\mu}(x)\frac{\delta}{\delta A_{a\mu}(x)} \\
=&\int d^4x \left[-\epsilon_a(x)\frac{\partial}{\partial x^\mu}\frac{\delta}{\delta V_{a\mu}(x)}-\epsilon_a(x)f_{abc}V_{b\mu}(x)\frac{\delta}{\delta V_{c\mu}(x)}-\epsilon_a(x)f_{abc}A_{b\mu}(x)\frac{\delta}{\delta A_{c\mu}(x)} \right] \\
\therefore i\mc{Y}_a(x)=&-\frac{\partial}{\partial x^\mu}\frac{\delta}{\delta V_{a\mu}(x)}-f_{abc}V_{b\mu}(x)\frac{\delta}{\delta V_{c\mu}(x)}-f_{abc}A_{b\mu}(x)\frac{\delta}{\delta A_{c\mu}(x)} \\
\int d^4x i\epsilon_a(x) \mc{X}_a(x)=&\int d^4x \, \delta_5 V_{a\mu}(x) \frac{\delta}{\delta V_{a\mu}(x)}+\int d^4x \delta_5 A_{a\mu}(x)\frac{\delta}{\delta A_{a\mu}(x)} \\
=&\int d^4x \, f_{abc}\epsilon_c(x) A_{b\mu}(x) \frac{\delta}{\delta V_{a\mu}(x)} +\int d^4x \left[\partial_\mu \epsilon_a(x)+f_{abc}\epsilon_c(x) V_{b\mu}(x)\right]\frac{\delta}{\delta A_{a\mu}(x)} \\
=&\int d^4x \left[-\epsilon_a(x)\frac{\partial}{\partial x^\mu}\frac{\delta}{\delta A_{a\mu}(x)}-\epsilon_a(x)f_{abc}V_{b\mu}(x)\frac{\delta}{\delta A_{c\mu}(x)}-\epsilon_a(x)f_{abc}A_{b\mu}(x)\frac{\delta}{\delta V_{c\mu}(x)}\right] \\
\therefore i\mc{X}_a(x)=&-\frac{\partial}{\partial x^\mu}\frac{\delta}{\delta A_{a\mu}(x)}-f_{abc}V_{b\mu}(x)\frac{\delta}{\delta A_{c\mu}(x)}-f_{abc}A_{b\mu}(x)\frac{\delta}{\delta V_{c\mu}(x)}
\end{align*}
となる.ここで$f_{abc}$は$SU(3)$の構造定数だ.(これが実際に微小ゲージ変換演算子であることを確かめたければ,これを$V^\nu_b(y)$などに作用させてみれば良い.たとえば
\begin{align*}
\left[\int d^4x i\epsilon_a(x) \mc{Y}_a(x)\right] V^\nu_b(y)=&\int d^4x \, \delta V_{a\mu}(x) \frac{\delta V^\nu_b(y)}{\delta V_{a\mu}(x)}+\int d^4x \,\delta A_{a\mu}(x)\frac{\delta V^\nu_b(y)}{\delta A_{a\mu}(x)} \\
=&\int d^4x \,\delta V_{a\mu}(x)\eta^{\mu\nu}\delta^a_b \delta^4(x-y) +0 \\
=&\delta V^\nu_b(y)
\end{align*}
となる.)これらの演算子を$\Gamma[V,A]$に作用することは$\delta \Gamma[V,A],\delta_5 \Gamma[V,A]$を計算することであり,これはすなわち(22.2.12)における$\mc{A}(x)$に等しい.内線の運動量の添え字は,ベクトルカレントがアノマリーを持たないように選ばれている.
\begin{align*}
\mc{Y}_a\Gamma[V,A]=0
\end{align*}
そうすると,軸性ベクトルカレントにアノマリーが現れる.(以下の表式は,参考文献8のバーディーンの論文での(45)式を参照.)
\begin{align*}
\mc{X}_a\Gamma[V,A]=\frac{i}{32\pi^2}\epsilon^{\mu\nu\rho\sigma}\mr{Tr}\biggl\{\lambda_a\biggl[ V_{\mu\nu}V_{\rho\sigma}+\frac{1}{3}A_{\mu\nu}A_{\rho\sigma}-\frac{32}{3}A_\mu A_\nu A_\rho A_\sigma \\
+\frac{8}{3}i\left( A_\mu A_\nu V_{\rho\sigma}+A_\mu V_{\rho\sigma}A_\nu +V_{\rho\sigma}A_\mu A_\nu \right) \biggr]  \biggr\}
\end{align*}
ここで$\lambda_a$は(19.7.2)で与えられる$SU(3)$行列で
\begin{align*}
&V_\mu\equiv \frac{1}{2}\lambda_a V_{a\mu},\quad A_\mu\equiv \frac{1}{2}\lambda_a A_{a\mu} \\
&V_{\mu\nu}=\partial_\mu V_\nu -\partial_\nu V_\mu-i[V_\mu,V_\nu]-i[A_\mu,A_\nu] \\
&A_{\mu\nu}=\partial_\mu A_\nu-\partial_\nu A_\mu-i[V_\mu,A_\nu]-i[A_\mu,V_\nu]
\end{align*}
である.すでに説明したように(22.3.34)右辺第二項に伴う因子1/3は$AVV$と$AAA$のダイアグラムでの$a^\mu$の選び方が異なることによる.

\vskip\baselineskip

対称性が全て自発的に破れる場合のアノマリーについては,異なるカレントを区別するように$a^\mu$を選ぶ理由は全くない.その代わりに,フェルミオンの内線について,それについたゲージ・ボゾンについて対称なように添え字を付けるのが自然だ.既にみたように,これは$a=\frac{1}{3}(k_1-k_2)$と選んで三角ダイアグラムを計算することを意味する.これにより,(22.3.26)で与えられるものの三分の一が三角アノマリーに与えられる.四角形と五角形のダイアグラムを含めると,この結果は
\begin{align*}
\left[\braket{D_\mu J^\mu_\alpha }\right]_{\mr{anom}}=-\frac{1}{24\pi^2}\epsilon^{\kappa\nu\lambda\rho}\mr{Tr}\left\{T_\alpha \left[\partial_\kappa A_\nu \partial_\lambda A_\rho -\frac{1}{2} i\partial_\kappa A_\nu A_\lambda A_\rho +\frac{1}{2}iA_\kappa \partial_\nu A_\lambda A_\rho -\frac{1}{2}iA_\kappa A_\nu \partial_\lambda A_\rho \right]\right\}
\end{align*}
となる.ここで$A^\mu=A^\mu_\alpha T_\alpha$だ.(この結果は中原「理論物理学のための幾何学とトポロジー」を参照するといい.)

\vskip\baselineskip

(22.3.22)を導いた議論は,摂動論の任意の次数で使えて,一般に表面積分として表せる運動量積分からアノマリーが生じることが示せる.その結果,(22.3.18)で見たようにアノマリーに寄与するカレントの発散ダイアグラムはフェルミオンループを回る運動量についての積分の次元が(運動量のベキで)ゼロ,あるいは正のものだけが残る.ループのフェルミオンが仮想的ゲージボゾンと相互作用すると,フェルミオンループの運動量積分の次元が下がる.(例えば,今回三角形ダイアグラムを計算したときは(22.3.17)の次元は$-2$だったが,四角形ダイアグラムを計算するとフェルミオンプロパゲータがひとつ増えるために$-3$の次元に下がる.)そのためアノマリーへの寄与が無くなることがある.したがってそのような輻射補正からアノマリーへの寄与はない.(例えば,前の計算では六角形ダイアグラム以降は寄与しなかった.)フェルミオンループについているゲージボゾンと,他のゲージボゾンや他のフェルミオンループとの相互作用は,アノマリーに影響する.(例えば,ゲージボゾンの外線に真空偏極のときのようなループがついているとき.)しかし,これはゲージ場をくりこめば影響を消せるので,$\epsilon^{\kappa\nu\lambda\rho}F^\beta_{\kappa\nu}(x)F^\gamma_{\lambda\rho}(x)$のような演算子をくりこむ効果しかない.同じ理由で,問題の対称性を破らないフェルミオンの質量は,この質量の因子を取り出すと運動量積分の次元が下がるのでアノマリーを変えない.\par
一般の種類の理論では,ラグランジアン密度の質量項は
\begin{align*}
\mc{L}_{\mr{mass}}=-\sum_{nn',\sigma\sigma'}\chi_{\sigma n}\epsilon_{\sigma\sigma'}M_{nn'}\chi_{\sigma' n'}+\mr{h.c.}
\end{align*}
の形をとる.ここで$\sigma$はローレンツ群の$(\frac{1}{2},0)$表現の2成分スピノル指標,$\epsilon_{\sigma\sigma'}$はローレンツ不変性に必要な$\epsilon_{\frac{1}{2},-\frac{1}{2}}=+1$の反対称行列,$M$は対称質量行列だ.(これがローレンツ不変であることは,表記が違ってわかりにくいが九後ゲージの1章における点付きスピノルと点なしスピノルの話と同じことであることに気付けばわかる.)$\mc{L}_{\mr{mass}}$がゲージ不変性を保つためには(22.3.3)より
\begin{align*}
\delta \mc{L}_{\mr{mass}}=&-\sum_{nn',\sigma\sigma'}\delta \chi_{\sigma n}\epsilon_{\sigma\sigma'}M_{nn'}\chi_{\sigma' n'} -\sum_{nn',\sigma\sigma'}\chi_{\sigma n}\epsilon_{\sigma\sigma'}M_{nn'}\delta \chi_{\sigma' n'}+\mr{h.c.} \\
=&-\sum_{nn',\sigma\sigma'}i\epsilon_\alpha \chi_{\sigma n} T^{\mr{T}}_\alpha \epsilon_{\sigma\sigma'}M_{nn'}\chi_{\sigma' n'}-\sum_{nn',\sigma\sigma'}\chi_{\sigma n}\epsilon_{\sigma\sigma'}M_{nn'}i\epsilon_\alpha T_\alpha \chi_{\sigma' n'}+\mr{h.c.} \\
=&-\sum_{nn',\sigma\sigma'}i\epsilon_\alpha \chi_{\sigma n}\epsilon_{\sigma\sigma'}\left(T_\alpha^{\mr{T}}M_{nn'}+M_{nn'}T_\alpha\right)\chi_{\sigma' n'}+\mr{h.c.}=0
\end{align*}
であるから,質量行列は
\begin{align*}
-T^{\mr{T}}_\alpha M=MT_\alpha
\end{align*}
を満たさなければならない.指標$n$はゲージ群の既約表現自体を表す指標$r$と,それぞれの既約表現内の成分を表す指標$s$に置き換えることができる.(深非弾性散乱での(20.6.19)あたりの議論と同様.)ある既約表現$r$の場がゲージ変換によって別の既約表現$r'$に移り変わることはないから,$T_\alpha$は異なる既約表現同士を結びつける効果を持たない.したがって
\begin{align*}
(T_\alpha)_{rs,r's'}=\delta_{rr'}(T^{(r)}_\alpha)_{s,s'}
\end{align*}
となって,また
\begin{align*}
M_{rs,r's'}=(M^{rr'})_{ss'}
\end{align*}
と書くことができる.すると,(22.3.240)は以下のように書き直せる.
\begin{align*}
-T_\alpha^{(r)\mr{T}}M^{(r,r')}=M^{(r,r')}T_\alpha^{(r')}
\end{align*}
ここで$r$や$r'$について和はとっていない.この変形が分かりにくれば,変換(22.3.3)が以下の形になることに留意して(22.3.40)の導出をやり直した方がいいかもしれない.
\begin{align*}
\delta \chi_{\sigma,sr}=&\sum_{rs,r's'}i\epsilon_\alpha \delta_{rr'}(T_\alpha^{(r)})_{s,s'}\chi_{\sigma,s'r'} \\
=&\sum_{ss'}i\epsilon_\alpha (T_\alpha^{(r)})_{s,s'}\chi_{\sigma,s'r}
\end{align*}
これにより
\begin{align*}
\delta \mc{L}_{\mr{mass}}=&-\sum_{\sigma,\sigma'}\sum_{rr',ss'}\delta \chi_{\sigma rs}\epsilon_{\sigma\sigma'}M^{(rr')}_{ss'}\chi_{\sigma' ,r's'} -\sum_{\sigma\sigma'}\sum_{rr',ss'}\chi_{\sigma,rs}\epsilon_{\sigma\sigma'}M^{(rr')}_{ss'}\delta \chi_{\sigma', r's'}+\mr{h.c.} \\
=&-\sum_{\sigma\sigma'}\sum_{rr',ss's''}i\epsilon_\alpha \chi_{\sigma,s''r} (T^{(r)\mr{T}}_\alpha)_{s'',s} \epsilon_{\sigma\sigma'}M^{(rr')}_{ss'} \chi_{\sigma' s'r'} \\
&-\sum_{\sigma\sigma'}\sum_{rr',ss's''}\chi_{\sigma,sr}\epsilon_{\sigma\sigma'}M^{(rr')}_{ss'}i\epsilon_\alpha (T^{(r')}_\alpha)_{s's''} \chi_{\sigma' r's''}+\mr{h.c.} \\
=&-\sum_{\sigma\sigma'}\sum_{rr',ss's''}i\epsilon_\alpha \chi_{\sigma, rs}\epsilon_{\sigma\sigma'}\left((T_\alpha^{(r)\mr{T}})_{s,s''}M^{(rr')}_{s''s'}+M^{(rr')}_{ss''}(T_\alpha^{(r')})_{s'',s'}\right)\chi_{\sigma', s'r'}+\mr{h.c.}=0
\end{align*}
より
\begin{align*}
-T_\alpha^{(r)\mr{T}}M^{(r,r')}=M^{(r,r')}T_\alpha^{(r')}
\end{align*}
となる.ここでシューアの補題とは,ふたつの既約表現$D_{(r)}:\chi_r \to \chi'_r$と$D'_{(r')}:\chi_{r'} \to \chi'_{r'}$と,線形写像に対応する質量行列$M^{(rr')}:\chi_{r'}\to(M\chi)_{r}$との間に
\begin{align*}
D_{(r)}M^{(rr')}=M^{(rr')}D'_{(r')}
\end{align*}
が成り立つとき,$D_{(r)}$と$D'_{(r')}$が同型でない(すなわち相似変換で関係していない)ならば$M^{(rr')}=0$,また同型である(相似変換で関係している)ならば$M^{(rr')}$は同型写像すなわち正則行列となる,という補題だ.今回の場合は,ゲージ不変性よりゲージ変換$D_{(r)}$によって
\begin{align*}
M^{(rr')}=&D_{(r)}^{\mr{T}}M^{(rr')}D_{(r')} \\
=&M^{(rr')}+i\epsilon_{\alpha}[M^{(r,r')}T_\alpha^{(r')}+T_\alpha^{(r)\mr{T}}M^{(rr')}]
\end{align*}
となって(22.3.42)が導かれるのであったから,シューアの補題を適用することができ(i)$M^{(r,r')}=0$か(ii)$-T^{(r)\mr{T}}_\alpha$と$T^{(r')}_\alpha$が相似変換で関係している.
\begin{align*}
D_{(r)}&=1+i\epsilon_\alpha T^{(r)}_\alpha \\
=&U[D^{\mr{T}}_{(r')}]^{-1}U^{-1} \quad(U は相似変換行列)\\
=&1-i\epsilon_\alpha U T^{(r')T}_\alpha U^{-1} \\
\therefore &\quad T^{(r)}_\alpha=U [- T_\alpha^{(r')\mr{T}}]U^{-1}
\end{align*}
したがってこのとき,アノマリー定数(22.3.12)への寄与は
\begin{align*}
D^{(r)}_{\alpha\beta\gamma}=&\frac{1}{2}\mr{tr}[\{T_\alpha^{(r)} ,T^{(r)}_\beta \} T^{(r)}_\gamma] \\
=&\frac{1}{2}\mr{tr}[T_\alpha^{(r)} T^{(r)}_\beta T^{(r)}_\gamma]+\frac{1}{2}\mr{tr}[T^{(r)}_\beta T^{(r)}_\alpha T^{(r)}_\gamma] \\
=&\frac{1}{2}\mr{tr}[U[- T_\alpha^{(r')\mr{T}}]U^{-1} U[-T^{(r')}_\beta]U^{-1} U[-T^{(r')}_\gamma]U^{-1}] \\
&+\frac{1}{2}\mr{tr}[U[-T^{(r')}_\beta]U^{-1} U[- T_\alpha^{(r')\mr{T}}]U^{-1} U[-T^{(r')}_\gamma]U^{-1}] \\
=&-\frac{1}{2}\mr{tr}[T_\alpha^{(r')\mr{T}} T^{(r')\mr{T}}_\beta T^{(r')\mr{T}}_\gamma]-\frac{1}{2}\mr{tr}[T^{(r')\mr{T}}_\beta T_\alpha^{(r')\mr{T}} T^{(r')\mr{T}}_\gamma ] \\
=&-\frac{1}{2}\mr{tr}[T^{(r')}_\gamma T_\alpha^{(r')} T^{(r')}_\beta ]-\frac{1}{2}\mr{tr}[T^{(r')}_\gamma T^{(r')}_\beta T_\alpha^{(r')}  ]=-\frac{1}{2}\mr{tr}[\{T^{(r')}_\alpha,T^{(r')}_\beta\}T^{(r')}_\gamma] \\
=&-D^{(r')}_{\alpha\beta\gamma}
\end{align*}
と関係している.したがって,($r=r'$のとき)$D_{\alpha\beta\gamma}=0$となってアノマリーはゼロとなるか,もしくは($r\neq r'$のとき)二つの既約表現の間で$D_{\alpha\beta\gamma}^{(r)}+D_{\alpha\beta\gamma}^{(r')}=0$となって相殺する.したがって,ある対称性の与えられた集合のアノマリーはそれらの対称性のもとで質量を持つことが許されるフェルミオンがあっても,それに影響されない.


\newpage

\subsection{アノマリーを持たないゲージ理論}
ここまで,アノマリーの一般的なカレント$J^\mu_\alpha$の保存に対する影響を計算してきた.このカレントがそれ自身ゲージ場に結合していると,ゲージ不変性よりアノマリーが無いことが要請される.前の節ではアノマリーは(22.3.12)で定義される完全に対称な定数因子$D_{\alpha\beta\gamma}$に比例していなければらなず,このためにゲージカレントについてはアノマリーが消えるために
\begin{align*}
D_{\alpha\beta\gamma}\equiv \frac{1}{2}\mr{tr}[\{T_\alpha ,T_\beta \} T_\gamma]=0
\end{align*}
となっていなければならない.ここで$T_\alpha$は全ての左手フェルミオンと左手反フェルミオン場に対するゲージ代数の表現,また「$\mr{tr}$」は以前と同様にこれらのフェルミオンと反フェルミオンの種類についてのトレース和を意味する.もしフェルミオン場が群の適切な既約または可約な表現になっていれば,任意のゲージ群においてこの条件を満たすことができる.さらにいくつかのゲージ群では,フェルミオンがその群の任意の表現になっていても(22.4.1)が満たされる.\par
もし左手フェルミオン(と反フェルミオン)場がゲージ代数の表現であり,その複素共役がそれ自身に同値ならば,(つまり後に言う「広い意味での実表現」ならば)
\begin{align*}
(iT_\alpha)^*=S(iT_\alpha)S^{-1}
\end{align*}
となり,あるいは書き換えて($T_\alpha$を常にエルミートととるから)
\begin{align*}
-iT^*_\alpha&=-iT_\alpha^{\mr{T}}=iST_\alpha S^{-1} \quad \because T^\dagger_\alpha=T_\alpha \\
T_\alpha^{\mr{T}}&=-ST_\alpha S^{-1}
\end{align*}
となる.これを(22.3.12)に代入すると,22.3節最後の計算と同様にして$D_{\alpha\beta\gamma}=-D_{\alpha\beta\gamma}$が得られゼロとなる.ここで(22.4.2)を満たす表現は実か擬実だ.\par

\vskip\baselineskip

表現$(\pi,V)$とその複素表現$(\bar{\pi},\bar{V})$が同値であるとき,$(\pi,V)$は広い意味での実表現と呼ぶ.「広い意味」とつけた理由はすぐに分かる.$\kappa:V\to \bar{V}$を,成分全てに複素共役をとる自然な写像とする.これは反ユニタリである.
\begin{align*}
\kappa (cv)=c^* \kappa v
\end{align*}
また当たり前だが
\begin{align*}
\kappa \pi(g)=\bar{\pi}(g)\kappa
\end{align*}
だ.さて,$(\pi,V)$が広い意味で実表現だとする.すると連続な線形写像$M:\bar{V}\to V$が存在して
\begin{align*}
\pi(g)M=M\bar{\pi}(g)
\end{align*}
という相似変換が成り立っていることになる.$M$はユニタリである.これに両辺右から$\kappa$をかけて,$M$と$\kappa$を組み合わせた
\begin{align*}
\tau=M\kappa:V\to V
\end{align*}
という反ユニタリ写像を定義すると
\begin{align*}
&\pi(g)M\kappa =M\bar{\pi}(g)\kappa = M \kappa \pi(g) \\
\Rightarrow \quad & \pi (g)\tau =\tau \pi(g)
\end{align*}
となる.いま,$V$がさらに既約だとする.すると$\tau^2:V\to V$はユニタリかつ
\begin{align*}
\tau^2 \pi(g)=\pi(g)\tau^2
\end{align*}
を満たす.よってシューアの補題より$\tau^2=\lambda \mr{id}$となる.ただし$\lambda$は絶対値$1$の複素数となる.ここで
\begin{align*}
\lambda \tau=(\tau)^2\tau=\tau(\tau)^2=\tau \lambda
\end{align*}
となるが,$\tau$は反ユニタリであるから$\lambda\tau =\tau \lambda^*$となるから,$\lambda=\lambda^*$となる.したがって$\lambda=\pm 1$となる.$\tau^2=+1$の場合,任意の$v\in V$を
\begin{align*}
v=w+w' \quad \mr{where} ,w=\frac{v+\tau v}{2} ,w'=\frac{v-\tau v}{2}
\end{align*}
として,実部分
\begin{align*}
W=\left\{ w\in V |w=\tau w \right\}
\end{align*}
と純虚部分
\begin{align*}
W'=\left\{ w'\in V |w'=-\tau w' \right\}
\end{align*}
に分けることができる.そうすると,$\pi(g):V\to V$かつ$\tau \pi(g)=\pi(g)\tau$より,$\pi(g):W\to W$となる.これは実ベクトルから実ベクトルへの写像となり,行列表示は全成分が実表現となる.したがって狭義の実表現とは,$\tau^2=+1$となる表現である.対して$\tau^2=-1$になる表現を擬実表現と呼ぶ.\par
まとめると,自然な反ユニタリ写像$\tau :V\to V$が存在し,$\pi(g)$と交換するならば広い意味での実表現であり,$\tau^2=+1$ならば実,$\tau^2=-1$ならば擬実である.\par
擬実表現の例としては,21.3節で示したように,$SU(2)$二重項$(\phi_1,\phi_2)$に対して
\begin{align*}
\tau :\left(
\begin{array}{cc}
\phi_1 \\
\phi_2
\end{array}
\right) \to i\sigma_2\left(
\begin{array}{cc}
\phi_1^* \\
\phi_2^*
\end{array}
\right)
\end{align*}
を定めると,$\exp(i\theta_\alpha t_\alpha)\tau =\tau \exp(i\theta_\alpha t_\alpha)$を満たし,これは$\tau^2=-1$を満たすので擬実表現となる.\par
補足すると,$j=\tau$と定めると虚数単位$i$と合わせて$i^2=j^2=-1$となり,また$\tau$が反ユニタリーであるから$ij=-ji$を満たし,$k=ij$を定義すると$i,j,k$は四元数の関係式を満たす.このことから擬実表現は四元数表現とも呼ぶことがある.

\vskip\baselineskip

実表現の場合には相似変換$T'_\alpha=RT_\alpha R^{-1}$でこの表現を$T'_\alpha$が虚で反対称の表現に変換できる.擬実表現の場合はこれは不可能だ.(たとえば,$SU(2)$の三次元既約表現は$SO(3)$に同型であるから自明に実で,$SU(2)$の二次元表現は前述の通り擬実だ.)ゲージ代数が実か擬実な表現しか持たなければ,アノマリーは存在しないということだ.そのような代数とは,$SO(2n+1)$,$n\geq 2$の$SO(4n)$,$n\geq 3$の$USp(2n)$,$G_2,F_4,E_7,E_8$とそれらの直和の全て,らしい.他の代数でも,いくつかの表現は実でも擬実でもないが$D_{\alpha\beta\gamma}$がゼロとなる表現のみを持っていて,それらは$SO(2)\equiv U(1)$と$SO(6)\equiv SU(4)$を除いた$SO(4n+2)$,$E_6$,それら同士あるいは上のどれかの代数との直和となる.したがってアノマリーは,$n\geq 3$の$SU(n)$か$U(1)$因子を持つゲージ代数だ.標準理論はゲージ群$SU(3)\times SU(2)\times U(1)$に基づいているから,理論にアノマリーが無いようにするためにクォークやレプトンの間で相殺が起こるようにしなければならない.\par
ここで$T_\alpha,T_\beta,T_\gamma$が$G=SU(3)\times SU(2)\times U(1)$の生成子の全てをとるとき$D_{\alpha\beta\gamma}$がゼロとなるかをチェックする.$T_\alpha,T_\beta,T_\gamma$の積が$G$のもとで不変になる生成子の組み合わせだけを考えれば良い.$SU(3)$の生成子がゼロ,二つ,三つの場合には不変量を作ることができる.なぜなら$SU(3)$の生成子は$SU(3)$のもとで$8$重項となり,二つの場合には
\begin{align*}
8\times 8=1+8+8+10+\bar{10}+27
\end{align*}
で1重項を含み,三つの場合にも
\begin{align*}
8\times 8\times 8&=(1+8+8+10+\bar{10}+27)\times 8 \\
=&(1\times 8)+ (8\times 8)+(8\times 8)+\cdots \\
=&8+1+8+8+10+\bar{10}+27 +\cdots
\end{align*}
となって,これも1重項を含む.ただし一つの場合はゲルマン行列$\lambda_a$に対して$\mr{tr}(\lambda_a)=0$により$D_{\alpha\beta\gamma}$はゼロとなる.(他の$SU(2),U(1)$の生成子は作用する箇所が違うので,$SU(3)$添え字のみのトレースでゼロとなる.)また同様に,$SU(2)$生成子がゼロ,二つ,三つの場合にも不変量が作ることができる.なぜなら,$SU(2)$生成子は$SU(2)$のもとで$3$重項となり,二つの場合には
\begin{align*}
3\times 3=1+3+5
\end{align*}
となって,1重項を含む.三つの場合にも
\begin{align*}
3\times 3\times 3=&(1+3+5)\times 3 \\
=&3 +1+3+5 +(5\times 3) \\
=&1+3+3+3+5+5+7
\end{align*}
となって,これも1重項を含む.ただし一つの場合は$SU(2)$生成子$t_i$に対して$\mr{tr}(t_i)=0$により$D_{\alpha\beta\gamma}$はゼロとなる.$U(1)$生成子については任意の個数で良い.したがって,以下の組み合わせのみチェックすれば良い.

\vskip\baselineskip

(1)$SU(3)-SU(3)-SU(3)$\par
この場合,$u_L,d_L$が$3$表現,$u^*_R,d^*_R$が$\bar{3}$表現,他のフェルミオンが1表現であるから,左手フェルミオン(と反フェルミオン)が$SU(3)$の$3+3+\bar{3}+\bar{3}+1+1+1$表現になり,これは実なので$D_{\alpha\beta\gamma}$はゼロとなる.

\vskip\baselineskip

(2)$SU(3)-SU(3)-U(1)$\par
トレースは固有値の和に等しく,$SU(3)$の固有ベクトルになれるのは$u_L,d_L,u^*_R,d^*_R$のみである.したがって$D_{\alpha\beta\gamma}=\frac{1}{2}\mr{tr}[\{\lambda_a,\lambda_b\}y/g']$は,それらの$U(1)$固有値の和に比例する.すなわち
\begin{align*}
\sum_{3,\bar{3}}y/g'=-\frac{1}{6}-\frac{1}{6}+\frac{2}{3}-\frac{1}{3}=0
\end{align*}
となって,$D_{\alpha\beta\gamma}$はゼロとなる.

\vskip\baselineskip

(3)$SU(2)-SU(2)-SU(2)$\par
この場合は,前述の通り$SU(2)$が実か擬実の表現しか持たないのでアノマリーは存在しない.

\vskip\baselineskip

(4)$SU(2)-SU(2)-U(1)$\par
この場合は,$SU(3)-SU(3)-U(1)$のときと同様に,今度は$SU(2)$生成子の固有ベクトルになれるのは$SU(2)$二重項のみとなるので,$D_{\alpha\beta\gamma}$は以下に比例する.(カラーが3つあるので,その分足し合わせなければならない.)
\begin{align*}
\sum_{\mr{doublet}}y/g'=3\left(-\frac{1}{6}\right)+\frac{1}{2}=0
\end{align*}
したがって$D_{\alpha\beta\gamma}$はゼロとなる.

\vskip\baselineskip

(5)$U(1)-U(1)-U(1)$\par
この場合,全ての左手フェルミオンが固有ベクトルとなるから,$D_{\alpha\beta\gamma}$は以下に比例する.(今度は$U(1)$生成子が三つだから,固有値は三乗になる.)
\begin{align*}
\sum(y/g')^3=6\left(-\frac{1}{6}\right)^3+3\left(+\frac{2}{3}\right)^3+3\left(-\frac{1}{3}\right)^3+2\left(+\frac{1}{2}\right)^3+(-1)^3=0
\end{align*}
したがって$D_{\alpha\beta\gamma}$はゼロとなる.

\vskip\baselineskip

以上により,標準模型のゲージ対称性については全てのアノマリーが相殺することがわかる.この結果は,大統一理論に関する議論でも述べた通り$SU(3)\times SU(2)\times U(1)$が$SO(10)$に埋め込めることから簡潔に理解できる.たまたま,1世代分のクォーク・レプトン・反クォーク・反レプトン,さらにそれに($SU(3)\times SU(2)\times U(1)$)の一重項,つまり$y/g'=0$の項を追加すると$SO(10)$の16次元表現(基本スピノル表現)が完全に形成され,$SO(10)$は前述の通りアノマリーが存在しない.この一重項はアノマリーに寄与しないので,標準模型のゲージ対称性にはアノマリーが存在しないと分かる.\par
計算しなければならないアノマリーはもうひとつある.フェルミオンはどの種類も重力と同じように相互作用する.カレント$\bar{\chi}T\gamma^\mu \chi$の期待値について,重力外場のもとでフェルミオンループを持つダイアグラムを計算すると以下に比例するアノマリー$\braket{\partial_\mu (\bar{\chi}T\gamma^\mu \chi)}$が得られるらしい.
\begin{align*}
\mr{tr}\{T \}\epsilon^{\mu\nu\rho\sigma}R_{\mu\nu\kappa\lambda}R_{\rho\sigma}^{\quad \kappa \lambda}
\end{align*}
(22.3.3)のようなゲージ対称性を重力が破らないためには,このアノマリーがゼロになる,つまり
\begin{align*}
\mr{tr}\{T_\alpha \}=0
\end{align*}
を満たさなければならない.純粋なゲージアノマリーのように,これは(22.4.2)を満たすゲージ群の生成子についてはゼロになる.
\begin{align*}
\mr{tr}\{T_\alpha\}&=\mr{tr}\{T_\alpha^{\mr{T}}\}=-\mr{tr}\{ST_\alpha S^{-1}\}=-\mr{tr}\{T_\alpha \} \\
\therefore &\quad \mr{tr}\{T_\alpha\}=0
\end{align*}
したがって,ゲージ群の実か擬実の表現をなすフェルミオンからの寄与はこのアノマリーにはない.よってこのアノマリーはゲージ対称性を破ることによって質量を得るフェルミオンのみを考慮に入れて計算してやれば良い.また,この条件は$SU(2),SU(3)$の生成子については定義より明らかに満たされていることが分かるだろう.よって「$重力子-重力子-SU(2)$」「$重力子-重力子-SU(2)$」の場合にはアノマリーは存在せず,「$重力子-重力子-U(1)$」の場合のみを確かめれば良い.これは同様に以下に比例する.
\begin{align*}
\sum (y/g')=6\left( -\frac{1}{6} \right)+3\left(\frac{2}{3}\right) +3\left(-\frac{1}{3}\right)+2\left(\frac{1}{2}\right)+(-1)=0
\end{align*}
したがって,標準模型のカレントには重力アノマリーは存在しない.

\vskip\baselineskip

アノマリーがゼロとなる要請は現実的な理論を作る上の指針となる.たとえば,21章で作った表のような,$SU(3)\times SU(2)$の多重項ハイパーチャージは元々実験から得られた.しかし,なぜこれらの弱ハイパーチャージ(および対応する各多重項の電荷)が観測された値をとるか不思議だ.多重項$(u_L,d_L),u_R^*,d^*_R,(\nu_L,e_L),e^*_R$にそれぞれ,任意の弱ハイパーチャージ$a,b,c,d,e$を与えたとしよう.アノマリーが相殺する条件から \par
\noindent $SU(3)-SU(3)-U(1)$
\begin{align*}
\sum_{3,\bar{3}} y=2a+b+c=0
\end{align*}
$SU(2)-SU(2)-U(1)$
\begin{align*}
\sum_{\mr{doublet}}y=3a+d=0
\end{align*}
$U(1)-U(1)-U(1)$
\begin{align*}
\sum y^3=6a^3+3b^3+3c^3+2d^3+e^3=0
\end{align*}
$重力子-重力子-U(1)$
\begin{align*}
\sum y=6a+3b+3c+2d+e=0
\end{align*}
となる.最初の式より$b+c=-2a$,二番目の式より$d/a=-3$がすぐに得られ,四番目の式より
\begin{align*}
6a+3(-2a)+-6a+e=0
\end{align*}
で$e/a=6$を得る.三番目の式より
\begin{align*}
&6a^3+3(b^3+c^3)+2(-3a)^3+(6a)^3=0 \\
&b^3+c^3=-56a^3 \\
&=(b+c)(b^2-bc+c^2)=-2a(b^2-bc+c^2)
\end{align*}
ここで$b/a=x,c/a=y$とすると,解くべき方程式は
\begin{align*}
&x+y=-2 \\
&x^2-xy+y^2=28
\end{align*}
の連立方程式となる.上の式を下の式に代入して整理すると$x^2+2x-8=0$となって,$x=2,-4$となる.$x,y$は対称性があるので,$b/a,c/a$のどちらかが$2$で,どちらかが$-4$となる.$u^*_R,d^*_R$が交換する可能性を除くと,$d/a$の方を$-4$に決めて
\begin{align*}
b/a=-4,\quad c/a=2,\quad d/a=-3,\quad e/a=6
\end{align*}
という解を得る.あるいは$a=0$とすることで$b=-c$以外全てゼロの
\begin{align*}
b=-c,\quad a=d=e=0
\end{align*}
という解も得られる.前者の解を$U(1)$,後者を$U(1)'$とする.これらの解は排他的だ.つまり,$U(1)$と$U(1)'$の両方が局所対称性だとすることはできない.なぜなら,もしどちらも局所対称性であるとすると,$(-4)+(-1)^2(+2)\neq 0$に比例する$U(1)'-U(1)'-U(1)$アノマリーと,$(-4)^2-(+2)^2\neq 0$に比例する$U(1)'-U(1)-U(1)$アノマリーが生じるからだ.$U(1)$生成子は$a=-\frac{1}{6}g'$とすれば標準電弱理論の弱ハイパーチャージであり,$U(1)'$対称性は自然界に観測されるどのような対称性にも似ていない.この計算より,標準理論での$y$の値,すなわち電荷の与え方について論理的な説明が得られた.これより,クォークとレプトンの一世代内で全てのゲージアノマリーが相殺しなければならないならば,ゲージボゾンが弱ハイパーチャージ$y$以外のどんな$U(1)$量子数($U(1)'$のような)とも結合するのは不可能なことがわかる.\par
一方,標準模型で$SU(3)\times SU(2)\times U(1)$のゲージボゾンが,知られているクォークとレプトンにのみ結合すると仮定するのは妥当だ.しかし,他の$U(1)'$ゲージボゾンが他の発見されていない$SU(3)\times SU(2)\times U(1)$について中性なフェルミオンや,知られているクォークとレプトンに結合しているかもしれない.多重項$(u_L,d_L),u_R^*,d^*_R,(\nu_L,e_L),e^*_R$の$U(1)'$量子数$y'$をそれぞれ改めて$a',b',c',d',e'$としよう.$SU(3)\times SU(2)\times U(1)$について中性のフェルミオンに対しては何も知らないから,$U(1)'-U(1)'-U(1)'$や$重力子-重力子-U(1)'$アノマリーが消える要請をしても,中性フェルミオンが何個存在しているか分からず,条件式を立てることができない.残りのアノマリーがゼロとなる条件から以下がわかる.\par
\noindent $SU(3)-SU(3)-U(1)'$
\begin{align*}
\sum_{3,\bar{3}}y'=2a'+b'+c'=0
\end{align*}
$SU(2)-SU(2)-U(1)'$
\begin{align*}
\sum_{\mr{doublet}}y'=3a'+d'=0
\end{align*}
$U(1)-U(1)-U(1)'$
\begin{align*}
\sum (y/a)^2 y'=6(1)^2a'+3(-4)^2b'+3(+2)^2c'+2(-3)^2d'+(+6)^2e'=0
\end{align*}
$U(1)-U(1)'-U(1)'$
\begin{align*}
\sum (y/a)y'^2=6(1)a'^2+3(-4)b'^2+3(+2)c'^2+2(-3)d'^2+(+6)e'^2=0
\end{align*}
これを解くと,(もはや面倒なのでwolfram alphaに入れて,このままでは条件不足なので$b'=c'$を仮定すると)
\begin{align*}
a'=e'/3,\quad b'=c'=-e/3,\quad d=-e
\end{align*}
となる.$e=1$とすると
\begin{align*}
a=1/3,\quad b=-1/3 ,\quad c=-1/3, \quad d=-1,\quad e=1
\end{align*}
となる.これが多重項のそれぞれの値に対応していると考えると,これはまさに$B-L$の数だ.つまり,量子数$B-L$はカイラルアノマリーや重力アノマリーによって破れず,保存する.通常の物体は巨視的な$B-L$の値を持つから,もし$B-L$が局所対称性であり(つまり結合するゲージ場が存在し),その結合定数が$e$より何桁も小さい,ということがなければ,より大きな対称性から$SU(3)\times SU(2)\times U(1)$に破れる際に自発的に破れていなければならない.この対称性の破れの典型的スケール$F$は,電弱相互作用のスケールより大きくなければならないが,何桁もかなり大きい必然性はない.したがって,$B-L$カレントに結合する中性ベクトルボゾン($Z^0$より重い)が標準模型に付け加わるのは,ありえる.

\newpage

\setcounter{subsection}{5}
\subsection{無矛盾条件}
どんな対称性のアノマリーに現れる数係数も理論の物質の内容(どの群の表現か,など)に依存する.一方,アノマリーの形は理論の詳細にはあまり依存しないようだ.これはヴェス-ズミノの無矛盾条件によって決まっている.\par
大域的対称性のアノマリーに興味があるときにでも,無矛盾条件を導くには全ての対称性カレントがゲージ場に結合し,非可換対称性ではゲージ場同士も(ヤンミルズ項で)結合し,そのためにこれらの対称性がラグランジアン密度の局所対称性になると想定すると便意らしい.対応するゲージ結合定数を微小にすることで,いつでも対称性が大域的な場合に戻ってくることができる.この形式では,アノマリーを除いては背景ゲージ$A_{\alpha \mu}(x)$の有効作用$\Gamma[A]$はゲージ場に対する微小ゲージ変換
\begin{align*}
A_{\beta\mu}(y)\to A_{\beta\mu}(y)-i\int d^4 x\epsilon_\alpha(x)\mc{T}_\alpha(x)A_{\beta\mu}(y)
\end{align*}
のもとで不変だ.ここで(15.1.9)を再現するように(22.3.31)(22.3.32)の導出と同様にして
\begin{align*}
-i\mc{T}_\alpha(x)=-\frac{\partial}{\partial x^\mu}\frac{\delta}{\delta A_{\alpha\mu}(x)}-C_{\alpha\beta\gamma}A_{\beta\mu}(x)\frac{\delta}{\delta A_{\gamma\mu}(x)}
\end{align*}
ととらねばならない.アノマリーを考慮すると,$\mc{T}_\alpha (x)$は作用は消すが有効作用$\Gamma[A]$を消さず
\begin{align*}
\mc{T}_\alpha (x)\Gamma[A]=G_\alpha [x;A]
\end{align*}
となる.$G_\alpha [x;A]$はアノマリーの影響を表す.(22.6.2)はまたカレントの期待値の共変発散の表式として書くこともできる.
\begin{align*}
iG_\alpha [x;A]&=i\mc{T}_\alpha (x)\Gamma[A]=\left\{\frac{\partial}{\partial x^\mu}\frac{\delta}{\delta A_{\alpha\mu}(x)}+C_{\alpha\beta\gamma}A_{\beta\mu}(x)\frac{\delta}{\delta A_{\gamma\mu}(x)}\right\}\Gamma[A] \\
&=\frac{\partial}{\partial x^\mu} \frac{\delta}{\delta A_{\alpha\mu}(x)}\Gamma[A]+iA_{\beta\mu}(t_\beta)_{\gamma\alpha}\frac{\delta}{\delta A_{\gamma\mu}(x)}\Gamma[A] \\
&=\frac{\partial}{\partial x^\mu} \frac{\delta}{\delta A_{\alpha\mu}(x)}\Gamma[A]-iA_{\beta\mu}(t_\beta)_{\alpha\gamma}\frac{\delta}{\delta A_{\gamma\mu}(x)}\Gamma[A] \\
&=D_\mu \frac{\delta}{\delta A_{\alpha\mu}(x)}\Gamma[A] =-D_\mu \braket{J^\mu_\alpha(x)} \\
\therefore &\quad D_\mu \braket{J^\mu_\alpha(x)}=-iG_\alpha [x;A]
\end{align*}
ここで(16.1.7)より
\begin{align*}
\braket{J^\mu_\alpha(x)}\equiv -\frac{\delta}{\delta A_{\alpha\mu}(x)}\Gamma[A]
\end{align*}
であり$D_\mu$は$(t_\beta)_{\gamma\alpha}=-iC_{\alpha\beta\gamma}$の随伴表現(15.1.6)でのゲージ共変微分(15.1.10)(あるいは(17.4.13)の方がわかりやすいだろうか)だ.\par
(22.6.1)を二回作用させることで
\begin{align*}
\mc{T}_\beta(y) \mc{T}_\alpha (x)A_{\gamma\mu}(z)=&i\mc{T}_\beta (y)\left[ \frac{\partial}{\partial x^\nu}\frac{\delta}{\delta A_{\alpha\nu}(x)}+C_{\alpha\delta\epsilon}A_{\delta\nu}(x)\frac{\delta}{\delta A_{\epsilon\nu}(x)} \right]A_{\gamma\mu}(z) \\
=&i\mc{T}_\beta(y)\left[\frac{\partial}{\partial x^\mu}\delta^4(z-x)\delta_{\alpha\gamma}+C_{\alpha\delta\gamma}A_{\delta\mu}(x)\delta^4(z-x)\right] \\
=&-C_{\alpha\delta\gamma}\left[\frac{\partial}{\partial y^\rho}\frac{\delta}{\delta A_{\beta\rho}(y)}+C_{\beta\eta\theta}A_{\eta\rho}(y)\frac{\delta}{\delta A_{\theta\rho}(y)} \right]\delta^4(z-x)A_{\delta\mu}(x) \\
=&-C_{\alpha\beta\gamma}\delta^4(z-x)\frac{\partial}{\partial y^\mu}\delta^4(x-y)-C_{\alpha\delta\gamma}C_{\beta\eta\delta}A_{\eta\mu}(y)\delta^4(x-y)\delta^4(x-z) \\
\mc{T}_\alpha(x) \mc{T}_\beta (y)A_{\gamma\mu}(z)=&C_{\alpha\beta\gamma}\delta^4(z-y)\frac{\partial}{\partial x^\mu}\delta^4(y-x)-C_{\beta\delta\gamma}C_{\alpha\eta\delta}A_{\eta\mu}(x)\delta^4(x-y)\delta^4(x-z) \\
[\mc{T}_\alpha(x),\mc{T}_\beta(y)]A_{\gamma\mu}(z)=&C_{\alpha\beta\gamma}\left[\delta^4(z-y)\frac{\partial}{\partial x^\mu}\delta^4(y-x)+\delta^4(z-x)\frac{\partial}{\partial y^\mu}\delta^4(x-y)\right] \\
&+[C_{\alpha\eta\delta}C_{\delta\beta\gamma}+C_{\alpha\gamma\delta}C_{\delta\eta\beta}]A_{\eta\mu}(x)\delta^4(x-y)\delta^4(x-z) \\
=&C_{\alpha\beta\gamma}\left[\delta^4(z-y)\frac{\partial}{\partial x^\mu}\delta^4(y-x)+\delta^4(z-x)\frac{\partial}{\partial y^\mu}\delta^4(x-y)\right]  \\
&-C_{\alpha\beta\delta}C_{\delta\gamma\eta}A_{\eta\mu}(x)\delta^4(x-y)\delta^4(x-z)
\end{align*}
一方
\begin{align*}
iC_{\alpha\beta\delta}\delta^4(x-y)\mc{T}_\delta(x)A_{\gamma\mu}(z)=&C_{\alpha\beta\delta}\delta^4(x-y)\left[\frac{\partial}{\partial x^\nu}\frac{\delta}{\delta A_{\delta\nu}(x)}+C_{\delta\eta\epsilon}A_{\eta\nu}(x)\frac{\delta}{\delta A_{\epsilon\nu}(x)}\right]A_{\gamma\mu}(z) \\
=&C_{\alpha\beta\gamma}\delta^4(x-y)\frac{\partial}{\partial x^\mu}\delta^4(z-x)+C_{\alpha\beta\delta}C_{\delta\eta\gamma}A_{\eta\mu}(x)\delta^4(x-y)\delta^4(x-z) \\
=&C_{\alpha\beta\gamma}\delta^4(x-y)\frac{\partial}{\partial x^\mu}\delta^4(z-x)-C_{\alpha\beta\delta}C_{\delta\gamma\eta}A_{\eta\mu}(x)\delta^4(x-y)\delta^4(x-z)
\end{align*}
ここで
\begin{align*}
&\delta^4(z-y)\frac{\partial}{\partial x^\mu}\delta^4(y-x)+\delta^4(z-x)\frac{\partial}{\partial y^\mu}\delta^4(x-y) \\
&=\frac{\partial}{\partial x^\mu}[\delta^4(z-y)\delta^4(y-x)]-\delta^4(z-x)\frac{\partial}{\partial x^\mu}\delta^4(x-y) \\
&=\frac{\partial}{\partial x^\mu}[\delta^4(z-x)\delta^4(y-x)]-\delta^4(z-x)\frac{\partial}{\partial x^\mu}\delta^4(x-y) \\
&=\delta^4(x-y)\frac{\partial}{\partial x^\mu}\delta^4(z-x)
\end{align*}
より交換関係
\begin{align*}
[\mc{T}_\alpha(x),\mc{T}_\beta(y)]=iC_{\alpha\beta\gamma}\delta^4(x-y)\mc{T}_\gamma(x)
\end{align*}
が得られる.((22.6.1)左辺のマイナス符号はこの交換関係を得るために必要なものだった.)(22.6.2)と(22.6.5)より一般の無矛盾条件
\begin{align*}
\mc{T}_\alpha(x)G_\beta[y;A]-\mc{T}_\beta(y)G_\alpha[x;A]=iC_{\alpha\beta\gamma}\delta^4(x-y)G_\gamma[x;A]
\end{align*}
が導かれる.\par
(22.3.31)(22.3.32)で定義されるゲージ変換演算子
\begin{align*}
-i\mc{Y}_a(x)=&-\frac{\partial}{\partial x^\mu}\frac{\delta}{\delta V_{a\mu}(x)}-f_{abc}V_{b\mu}(x)\frac{\delta}{\delta V_{c\mu}(x)}-f_{abc}A_{b\mu}(x)\frac{\delta}{\delta A_{c\mu}(x)} \\
-i\mc{X}_a(x)=&-\frac{\partial}{\partial x^\mu}\frac{\delta}{\delta A_{a\mu}(x)}-f_{abc}V_{b\mu}(x)\frac{\delta}{\delta A_{c\mu}(x)}-f_{abc}A_{b\mu}(x)\frac{\delta}{\delta V_{c\mu}(x)}
\end{align*}
は,交換関係
\begin{align*}
[\mc{Y}_a(x),\mc{Y}_b(y)]=if_{abc}\delta^4(x-y)\mc{Y}_c(x) \\
[\mc{Y}_a(x),\mc{X}_b(y)]=if_{abc}\delta^4(x-y)\mc{X}_c(x) \\
[\mc{X}_a(x),\mc{X}_b(y)]=if_{abc}\delta^4(x-y)\mc{Y}_c(x)
\end{align*}
を満たす.導出は(22.6.5)と同様にできるだろう.ここで$f_{abc}$は$SU(3)$の構造定数だ.$\mc{Y}_a$で生成される$SU(3)$部分群は自発的に破れていないから,フェルミオン運動量に関する積分を$\mc{Y}_a$で生成されるゲージ変換のもとでの不変性が保たれるように取り扱うと便利だ.こうすると
\begin{align*}
\mc{Y}_a (x)\Gamma=0
\end{align*}
となり,代わりにゼロでないアノマリー
\begin{align*}
\mc{X}_a(x)\Gamma=G_a(x)
\end{align*}
が残る.すると,自明でない無矛盾条件は
\begin{align*}
&\mc{Y}_a(x)G_b(y)=i\delta^4(x-y)f_{abc}G_c(x) \\
&\mc{X}_a(x)G_b(y)-\mc{X}_b(y)G_a(x)=0
\end{align*}
になる.上の第一の条件は,単に$G_a(x)$は通常の非カイラル$SU(3)$変換のもとで8重項として変換されることを意味し,第二の条件は$G_a(x)$に他の強い拘束を課す.バーディーン表式(22.3.34)はこの条件を満たしている.\par
一般のゲージ理論で全てのカレントを対象に扱う場合のアノマリーとして(22.3.28)を引用していたが,この表式のゲージ場についての二次より高い次数の項の係数は導出しなかった.ここで,これらの高次項が無矛盾条件(22.6.6)によって決まることを示そう.この目的のために,そしてさらに一般化をするために,ヴェス-ズミノ無矛盾条件をBRST変換のもとでの不変性条件として再構成するのが便利だ.一般のゲージ場でゴースト場$\omega_\alpha$を導入して,ベキ零のBRST演算子$s$を以下のように書く.((15.7.8)(15.7.10)参照)
\begin{align*}
sA_{\alpha\mu}=\partial_\mu \omega_\alpha +C_{\alpha\beta\gamma}A_{\beta\mu}\omega_\gamma \\
s\omega_\alpha =-\frac{1}{2}C_{\alpha\beta\gamma}\omega_\beta \omega_\gamma
\end{align*}
ここで$s$は分配則$s(AB)=(sA)B\pm A(sB)$を満たし,第二項目の符号は$A$がフェルミオン的ならば負,そうでなければ正とする.またアノマリー関数$G_\alpha[x;A]$の代わりに汎関数
\begin{align*}
G[\omega,A]=\int \omega_\alpha(x)G_\alpha[x;A]d^4x
\end{align*}
を用いる.これをBRST変換すると($\omega_\alpha$がフェルミオン的なので$G_\alpha$をBRST変換する際は負符号がつくことを忘れずに)
\begin{align*}
sG[\omega,A]=&\int \left\{s\omega_\alpha(x)\right\} G_\alpha[x;A]d^4x -\int \omega_\alpha(x)\left\{sG_\alpha[x;A]\right\}d^4x \\
=&-\frac{1}{2}C_{\alpha\beta\gamma}\int d^4x \omega_\beta (x)\omega_\gamma(x)G_\alpha[x;A] \\
&-\int d^4x \omega_\alpha(x) \int d^4y \, sA_{\beta\mu}(y)\frac{\delta G_\alpha[x;A]}{\delta A_{\beta\mu}(y)} \\
=&-\frac{1}{2}C_{\alpha\beta\gamma}\int d^4x \omega_\beta (x)\omega_\gamma(x)G_\alpha[x;A] \\ 
&-\int d^4x \omega_\alpha(x) \int d^4y \, \left[\frac{\partial \omega_\beta(y)}{\partial y^\mu}+C_{\beta\gamma\delta}A_{\gamma\mu}(y)\omega_\delta(y)\right]\frac{\delta G_\alpha[x;A]}{\delta A_{\beta\mu}(y)} \\
=&-\frac{1}{2}C_{\alpha\beta\gamma}\int d^4x \omega_\beta (x)\omega_\gamma(x)G_\alpha[x;A] \\
&+\int d^4x \omega_\alpha(x) \int d^4y \, \left[\omega_\beta(y)\frac{\partial }{\partial y^\mu}+\omega_\gamma(y)C_{\gamma\delta\beta}A_{\gamma\mu}(y)\right]\frac{\delta G_\alpha[x;A]}{\delta A_{\beta\mu}(y)} \\
=&-\frac{1}{2}C_{\alpha\beta\gamma}\int d^4x \omega_\beta (x)\omega_\gamma(x)G_\alpha[x;A] \\
&+\int d^4x \int d^4y \omega_\alpha(x)\omega_\beta(y) \left[\frac{\partial }{\partial y^\mu}\frac{\delta G_\alpha[x;A]}{\delta A_{\beta\mu}(y)}+C_{\beta\gamma\delta}A_{\gamma\mu}(y)\frac{\delta G_\alpha[x;A]}{\delta A_{\delta\mu}(y)}\right] \\
=&-\frac{1}{2}C_{\alpha\beta\gamma}\int d^4x \int d^4y \delta^4(x-y)\omega_\alpha (x)\omega_\beta(y)G_\gamma[x;A] \\
&+i\int d^4x \int d^4y \omega_\alpha(x)\omega_\beta(y)i\mc{T}_\beta(y)G_\alpha[x;A] \\
=&\int d^4x \int d^4y \omega_\alpha(x)\omega_\beta(y)\left[-\frac{1}{2}C_{\alpha\beta\gamma}\delta^4(x-y)G_\gamma[x;A]+i\mc{T}_\beta(y)G_\alpha[x;A]\right]
\end{align*}
ゴースト場は互いに反可換だから,
\begin{align*}
sG[\omega,A]=&-\frac{1}{2}C_{\alpha\beta\gamma}\int d^4x \int d^4y \delta^4(x-y)\omega_\alpha (x)\omega_\beta(y)G_\gamma[x;A] \\
&+i\int d^4x \int d^4y \omega_\alpha(x)\omega_\beta(y)i\mc{T}_\beta(y)G_\alpha[x;A] \\
=&-\frac{1}{2}C_{\alpha\beta\gamma}\int d^4x \int d^4y \delta^4(x-y)\omega_\alpha (x)\omega_\beta(y)G_\gamma[x;A] \\
&+i\frac{1}{2}\int d^4x \int d^4y \omega_\alpha(x)\omega_\beta(y)i\mc{T}_\beta(y)G_\alpha[x;A] \\
&+i\frac{1}{2}\int d^4x \int d^4y \omega_\beta(y)\omega_\alpha(x)i\mc{T}_\alpha(x)G_\beta[y;A] \quad \because x\leftrightarrow y,\alpha\leftrightarrow \beta \\
=&\int d^4x \int d^4y \omega_\alpha(x)\omega_\beta(y)\left[-\frac{1}{2}C_{\alpha\beta\gamma}\delta^4(x-y)G_\gamma[x;A]+i\frac{1}{2}\mc{T}_\beta(y)G_\alpha[x;A]-i\frac{1}{2}\mc{T}_\alpha(y)G_\beta[x;A]\right] \\
=&\frac{1}{2}i\int d^4x \int d^4y \omega_\alpha(x)\omega_\beta(y)\left[iC_{\alpha\beta\gamma}\delta^4(x-y)G_\gamma[x;A]+\mc{T}_\beta(y)G_\alpha[x;A]-\mc{T}_\alpha(y)G_\beta[x;A]\right] 
\end{align*}
この括弧の中を見てみよう.無矛盾条件(22.6.6)が成立するならばこの積分はゼロであり,逆にこの積分がゼロであるならば(22.6.6)は成り立っていなければならない!すなわち,無矛盾条件(22.6.6)は$G[\omega,A]$が全てのゴースト場$\omega_\alpha(x)$についてBRST不変
\begin{align*}
sG[\omega,A]=0
\end{align*}
と同値であることがわかる!\par
ここで,アノマリー$G[\omega,A]$が局所的汎関数$F[A]$にBRST演算子$s$を作用させたものと書ける可能性を考える.
\begin{align*}
G[\omega,A]=sF[A]
\end{align*}
(汎関数$F[A]$はゴースト場から独立だ.なぜなら,アノマリー汎関数$G[\omega,A]$は(22.6.8)より既にゴースト場について一次だが,演算子$s$はゴースト場の因子を一つ加えるからだ.もし$F$がゴースト場について一次以上ならば,$sF$はゴースト場について二次以上の項が現れ矛盾する.)BRST演算子は$s^2=0$を満たすから,このようなアノマリーは無矛盾条件$sG=0$を明らかに満たす.もし$F[A]$がゲージ場の局所的汎関数(ある与えられた点の場のと場の微分の関数,の積分の意)ならば,それを作用から差し引いてアノマリーを相殺できる.すなわち,BRST変換はパラメータを$\epsilon_\alpha(x)=\theta \omega_\alpha(x)$としたゲージ変換と見なせるから,測度からは$i\int d^4x \theta \omega_\alpha(x)G_\alpha [x;A]$が生じる((22.2.25)あたりを見ると分かりやすい).$\delta_\theta =\theta s$に気を付けると
\begin{align*}
&\delta_\theta \int[d\psi][d\bar{\psi}]e^{i(S[\psi,A]-F[A])} \\
= & \int [d\psi][d\bar{\psi}]\left\{i\delta_\theta S[\psi,A]-i\delta_\theta F[A]+i\int d^4x\theta \omega_\alpha(x) G_\alpha[x;A] \right\}e^{i(S[\psi,A]-F[A])} \\
=&\int [d\psi][d\bar{\psi}]\left\{i\delta_\theta S[\psi,A]-i\theta sF[A]+i\theta G[\omega,A] \right\}e^{i(S[\psi,A]-F[A])} \\
=& i\int [d\psi][d\bar{\psi}]\delta_\theta S[\psi,A]e^{i(S[\psi,A]-F[A])}=0
\end{align*}
と,$\delta_\theta S=\int d^4x J^\mu_\alpha(x)\partial_\mu(\theta\omega_\alpha(x))$より$\braket{\partial_\mu J_\alpha^\mu}=0$となる.これより,(22.6.11)とできる分のアノマリーは局所汎関数によって相殺できる.同じことはアノマリーの中の,局所汎関数にBRST演算子$s$を作用させた形に書けるような任意の項にも当てはまる.つまり,そのような項はそれ自身で無矛盾条件を満たし,作用に局所的な項を加えることで相殺できる.したがって,我々の興味のあるアノマリーは,無矛盾条件(22.6.10)を満たしゴースト数が1の局所汎関数$G[\omega,A]$から,ゴースト数ゼロの局所汎関数に$s$が作用した形の部分を全て引き去ったものだ.ベキ零の演算子の通常の用語に従えば,そのような汎関数の同値類は$s$演算子のゴースト数1のコホモロジー$H[\omega,A]$を成すという.
\begin{align*}
&Z[\omega,A]=\{G[\omega,A]|\mr{gh}(G[\omega,A])=1,G[\omega,A]\in \mr{Ker}(s) \} \\
&B[\omega,A]=\{G[\omega,A]|G[\omega,A]=sF[A]\in \mr{Im}(s),\mr{gh}(F[A])=0\} \\
&H[\omega,A]=Z[\omega,A]/B[\omega,A]
\end{align*}
これは局所密度自身を使って表すこともできる.アノマリー(もしくはアノマリーの中の任意の項)を$G[\omega,A]=\int d^4x \mc{G}(x)$と書くことができる.ここで$\mc{G}(x)$は時空の点$x$でのゲージ場とゴースト場,およびそれらの微分についてのベキ級数だ.$sG=0$という条件は以下の表式と同値だ.
\begin{align*}
s\mc{G}(x)=\partial_\mu \mc{J}^\mu(x)
\end{align*}
ここで,$\mc{J}^\mu(x)$は場と場の微分のある関数だ.同様に,作用に局所的な項を付け加えて相殺できる$\mc{G}$の中の項は,微分を除いて$s\mc{F}$の形のものだ.さらに,もし$\mc{G}=\partial_\mu \mc{H}^\mu+\cdots $と書ける場合,微分の項はアノマリーに寄与しない.したがって,我々に興味のあるアノマリーは,無矛盾条件(22.6.12)を満たしゴースト数が1の局所関数から,ゴースト数がゼロのある局所関数に$s$が作用した分だけ差し引き,さらにそれから微分だけ差し引いたものだ.これは,微分の文だけ差し引いた局所関数の空間でゴースト数が1の$s$のコホモロジーとして知られていて,通常$H^1(s|d)$と書かれるらしい.\par
代数的方法を使ってBRST演算子$s$のコホモロジーが調べられている.これより一般のゲージ理論でのアノマリーの形が分かる.この節の最初に述べた通り,この手法はアノマリーの形を決めるものであり,定数係数は理論に含まれる物質の内容に依存しているためこの方法ではわからない.これは22.2節と22.3節の方法で計算しなければならない.一般のゲージ理論でゲージ場について2次のアノマリー項は,その定数係数も既に計算したので,ここでは無矛盾条件(22.6.12)を使って場について高次の項を計算する.\par
22.3節で見たように,全てのカレントを対称に扱うと,場について二次の項は(22.3.26)の表式の1/3だ.質量次元4の演算子もあり,ヴェス-ズミノ無矛盾条件(22.6.6)は同じ次元の演算子を関係付けるので,この条件を満たすためには次元が同じで場について二次以上の項のみ足せば良い.したがって,以下の形の無矛盾条件の解を探す.
\begin{align*}
G_\alpha[x;A]=&i[\braket{D_\mu J^\mu_\alpha}]_{\mr{anom}} \\
=&-\frac{i}{24\pi^2}\epsilon^{\kappa\nu\lambda\rho}\mr{Tr}\biggl\{T_\alpha \Bigl[\partial_\kappa A_\nu \partial_\lambda A_\rho +ic_1 \partial_\kappa A_\nu A_\lambda A_\rho  \\
& +ic_2 A_\kappa \partial_\nu A_\lambda A_\rho +ic_3 A_\kappa A_\nu \partial_\lambda A_\rho -c_4 A_\kappa A_\nu A_\lambda A_\rho \Bigr]\biggl\}
\end{align*}
ここで$A_\mu \equiv A_{\alpha\mu}T_\alpha$で,$c_i(i=1,\cdots, 4)$は決めるべき定数だ.\par
努力を大幅に軽減するために,これを微分形式の言葉を使って書くと良い.c数のパラメータの集合$dx^\mu$を導入し,これは互いに,またゴースト場$\omega_\alpha$を含む全てのフェルミオン的な場と反可換とする.
\begin{align*}
dx^\mu dx^\nu=-dx^\nu dx^\mu,\quad \omega_\alpha dx^\mu=-dx^\mu \omega_\alpha
\end{align*}
このとき,$dx^\mu$はBRST演算子$s$と反可換だ.これは以下からすぐにわかる.
\begin{align*}
(sA_{\alpha\mu})dx^\mu=&\left\{\partial_\mu \omega_\alpha +C_{\alpha\beta\gamma}A_{\beta\mu}\omega_\gamma\right\}dx^\mu \\
=&-dx^\mu\left\{\partial_\mu \omega_\alpha +C_{\alpha\beta\gamma}A_{\beta\mu}\omega_\gamma\right\} =-dx^\mu(sA_{\alpha\mu})
\end{align*}
$dx^\kappa dx^\nu dx^\lambda dx^\rho$は完全反対称だから,それは以下のように書ける.
\begin{align*}
dx^\kappa dx^\nu dx^\lambda dx^\rho=\epsilon^{\kappa\nu\lambda\rho}d^4x ,\quad d^4x =dx^0dx^1dx^2dx^3
\end{align*}
また外微分
\begin{align*}
d \equiv dx^\mu \frac{\partial}{\partial x^\mu}
\end{align*}
も導入する.これは微分が可換なためにベキ零
\begin{align*}
&d^2=dx^\mu dx^\nu \frac{\partial^2}{\partial x^\mu \partial x^\nu}=-dx^\nu dx^\mu \frac{\partial^2}{\partial x^\nu \partial x^\mu}=-d^2 \\
\therefore &\quad d^2=0
\end{align*}
であり,また$s$と反可換
\begin{align*}
&ds=dx^\mu \frac{\partial}{\partial x^\mu} s=-sdx^\mu\frac{\partial}{\partial x^\mu}=-sd \\
\therefore &\quad  ds+sd=0
\end{align*}
だ.最後に,反可換量
\begin{align*}
A\equiv iA_\mu dx^\mu =iA_{\alpha\mu}T_\alpha dx^\mu ,\quad \omega \equiv i\omega_\alpha T_\alpha 
\end{align*}
を導入する.これらの記法を用いれば,(22.6.13)は
\begin{align*}
G[\omega,A]=&\int \omega_\alpha(x) G_\alpha[x;A]d^4x \\
=&-\frac{i}{24\pi^2}\int \omega_\alpha \epsilon^{\kappa\nu\lambda\rho}\mr{Tr}\biggl\{T_\alpha \Bigl[\partial_\kappa A_\nu \partial_\lambda A_\rho +ic_1 \partial_\kappa A_\nu A_\lambda A_\rho  \\
& +ic_2 A_\kappa \partial_\nu A_\lambda A_\rho +ic_3 A_\kappa A_\nu \partial_\lambda A_\rho -c_4 A_\kappa A_\nu A_\lambda A_\rho \Bigr]\biggl\} d^4x \\
=&-\frac{1}{24\pi^2}\int \mr{Tr}\biggl\{i\omega_\alpha T_\alpha \biggl[\left(dx^\kappa \frac{\partial}{\partial x^\kappa} A_\nu dx^\nu \right) \left(dx^\lambda \frac{\partial}{\partial x^\lambda} A_\rho dx^\rho \right) \\
&+ic_1A_\kappa dx^\kappa \left(dx^\nu \frac{\partial}{\partial x^\nu} A_\lambda dx^\lambda \right)A_\rho dx^\rho +ic_3A_\kappa dx^\kappa A_\nu dx^\nu \left(dx^\lambda \frac{\partial}{\partial x^\lambda} A\rho dx^\rho \right) \\
&-c_4 A_\kappa dx^\kappa A_\nu dx^\nu A_\lambda dx^\lambda A_\rho dx^\rho \biggr]\biggl\} \\
=&\frac{1}{24\pi^2}\int \mr{Tr}\biggl\{\omega \Bigl[(dA)^2 +c_1 A(dA)A+c_3 A^2(dA)+c_4A^4 \Bigr]\biggl\}
\end{align*}
となる.これはわかりやすい!\par
無矛盾条件(22.6.10)を満たすために,BRST変換則(22.6.7)(22.6.8)が
\begin{align*}
sA=&i\left\{\partial_\mu \omega_\alpha +C_{\alpha\beta\gamma}A_{\beta\mu}\omega_\gamma\right\}T_\alpha dx^\mu \\
=&-idx^\mu \partial_\mu \omega_\alpha T_\alpha - iC_{\alpha\beta\gamma}T_\alpha A_{\beta\mu}dx^\mu \omega_\gamma \\
=&-d\omega-[T_\beta,T_\gamma]A_{\beta\mu}dx^\mu \omega_\gamma \\
=&-d\omega-(A_{\beta\mu}T_\beta dx^\mu \omega_\gamma T_\gamma+\omega_\gamma T_\gamma A_{\beta\mu}T_\beta dx^\mu )\quad \because 反可換性 \\
=&-d\omega+\{A,\omega\} \\
s\omega=&-i\frac{1}{2} C_{\alpha\beta\gamma}T_\alpha \omega_\beta \omega_\gamma \\
=&-\frac{1}{2}[T_\beta,T_\gamma]\omega_\beta \omega_\gamma \\
=&-\frac{1}{2}[\omega_\beta T_\beta \omega_\gamma T_\gamma +\omega_\gamma T_\gamma \omega_\beta T_\beta ] \\
=&\omega^2
\end{align*}
と書けることに着目する.すると,(22.6.17)の$c_4$の項のBRST変換は
\begin{align*}
s\mr{Tr}[\omega A^4]=&\mr{Tr}[\omega^2 A^4 -\omega \{A,\omega\}A^3+\omega A\{A,\omega\}A^2 -\omega A^2 \{A,\omega\}A+\omega A^3\{A,\omega\} ]+[\omega d\omega A^3 項] \\
=&\mr{Tr}[\omega^2 A^4-\omega A\omega A^3 -\omega^2 A^4 +\omega A^2 \omega A^2 +\omega A \omega A^3 \\
&-\omega A^3\omega A-\omega A^2 \omega A^2 +\omega A^4 \omega +\omega A^3\omega A]+[\omega d\omega A^3 項] \\
=&\mr{Tr}[\omega^2 A^4]+[\omega d\omega A^3 項]
\end{align*}
となる.$\mr{Tr}[\omega^2 A^4]$に比例する$sG$への寄与は他の項から生じないので,無矛盾条件(22.6.10)は$c_4=0$でのみ満たすことができる.$c_4$のもとで,他の項のBRST変換を実行する.(ここで$d^2=0,sd+ds=0$に注意する.また,トレースの巡回性を用いるときは,$\omega_\alpha,A_{\alpha\mu},T_\alpha$などには巡回性を用いることができるが,$dx^\mu$だけは行列ではなく添え字の足し合わせには関係ないので巡回性を用いることができない.他の$dx^\nu$や$\omega$と反交換させながら注意して移動させること.)
\begin{align*}
s(dA)=&-d(sA)=-d(-d\omega+\{A,\omega\})=-d\{A,\omega\} \\
=&-d(A\omega)-d(\omega A)=-(dA)\omega+A(d\omega)-(d\omega)A+\omega(dA)
\end{align*}
となるから
\begin{align*}
s\mr{Tr}[\omega(dA)^2]=&\mr{Tr}[s\omega (dA)^2-\omega (sdA)(dA)-\omega (dA)(sdA)] \\
=&\mr{Tr}[\omega^2(dA)^2+\omega d\{A,\omega\} (dA)+\omega (dA)d\{A,\omega\}] \\
=&\mr{Tr}[-\omega^2(dA)^2+\omega (dA) \omega (dA)-\omega A (d\omega) (dA)+\omega (d\omega) A (dA)-\omega^2 (dA)^2 \\
&+\omega (dA)^2 \omega -\omega(dA)A(d\omega) +\omega(dA)(d\omega) A-\omega (dA)\omega (dA)] \\
=&\mr{Tr}[-\omega^2(dA)^2-\omega A (d\omega) (dA)+\omega (d\omega) A (dA) -\omega(dA)A(d\omega) +\omega(dA)(d\omega) A]
\end{align*}
となる.また$c_1$の項は
\begin{align*}
s\mr{Tr}[\omega(dA)A^2]=&\mr{Tr}[s\omega (dA)A^2-\omega(sdA)A^2-\omega(dA)(sA)A+\omega(dA)A(sA)] \\
=&\mr{Tr}[\omega^2 (dA)A^2+\omega d\{A,\omega \}A^2-\omega(dA)(-d\omega+\{A,\omega\})A+\omega(dA)A(-d\omega+\{A,\omega\})] \\
=&\mr{Tr}[\omega^2 (dA)A^2+\omega (dA)\omega A^2-\omega A(d\omega) A^2+\omega (d\omega)A^3- \omega^2(dA)A^2 \\
&+\omega (dA)(d\omega)A -\omega (dA)A\omega A -\omega(dA)\omega A^2 \\
&-\omega (dA)A(d\omega)+\omega(dA)A^2\omega +\omega(dA)A\omega A] \\
=&\mr{Tr}[-\omega A(d\omega) A^2+\omega (d\omega)A^3 +\omega (dA)(d\omega)A -\omega (dA)A(d\omega)+\omega(dA)A^2\omega]
\end{align*}
となる.$c_2$の項からは
\begin{align*}
s\mr{Tr}[\omega A(dA)A]=&\mr{Tr}[s\omega A(dA)A-\omega(sA)(dA)A+\omega A(sdA)A + \omega A(dA)(sA) ] \\
=&\mr{Tr}[\omega^2 A(dA)A-\omega (-d\omega +\{A,\omega \} )(dA)A-\omega A d\{A,\omega\}A+\omega A(dA)(-d\omega+\{\omega,A\})]\\
=&\mr{Tr}[\omega^2A(dA)A+\omega(d\omega)(dA)A-\omega A\omega (dA)A-\omega^2 A(dA)A \\
&-\omega A (dA)\omega A +\omega A^2(d\omega)A - \omega A(d\omega)A^2+\omega A\omega (dA)A \\
&-\omega A(dA)(d\omega)+\omega A(dA)\omega A+\omega A(dA)A\omega] \\
=&\mr{Tr}[\omega(d\omega)(dA)A+\omega A^2(d\omega)A-\omega A(d\omega)A^2-\omega A(dA)(d\omega)+\omega A(dA)A\omega]
\end{align*}
となる.$c_3$の項からは
\begin{align*}
s\mr{Tr}[\omega A^2(dA)]=&\mr{Tr}[s\omega A^2(dA)-\omega (sA)A(dA)+\omega A(sA)(dA)-\omega A^2(sdA)] \\
=&\mr{Tr}[\omega^2 A^2(dA)-\omega (-d\omega+\{A,\omega \})A(dA)+\omega A(-d\omega+\{A,\omega \})(dA)+\omega A^2d\{A,\omega\}] \\
=&\mr{Tr}[\omega^2A^2(dA)+\omega(d\omega)A(dA)-\omega A\omega A(dA)-\omega^2 A^2(dA) \\
&-\omega A(d\omega)(dA)+\omega A^2\omega (dA)+\omega A\omega A(dA) \\
&+\omega A^2(dA)\omega -\omega A^3 (d\omega)+\omega A^2(d\omega)A-\omega A^2\omega (dA)] \\
=&\mr{Tr}[\omega(d\omega)A(dA)-\omega A(d\omega)(dA)+\omega A^2(dA)\omega -\omega A^3 (d\omega)+\omega A^2(d\omega)A]
\end{align*}
となる.これらを合わせると
\begin{align*}
sG=&\frac{1}{24\pi^2}\int \mr{Tr}\Bigl\{ -\omega^2(dA)^2-\omega A(d\omega)(dA)+\omega(d\omega)A(dA)-\omega (dA)A(d\omega)+\omega(dA)(d\omega)A \\
&+c_1[\omega(dA)(d\omega)A-\omega(dA)A(d\omega)] \\
&+c_2[\omega(d\omega)(dA)A-\omega A(dA)(d\omega)] \\
&+c_3[\omega(d\omega)A(dA)-\omega A(d\omega)(dA)] \\
&+c_1[-\omega A(d\omega)A^2+\omega(d\omega)A^3+\omega(dA)A^2\omega] \\
&+c_2[\omega A^2(d\omega)A-\omega A(d\omega)A^2+\omega A(dA)A\omega] \\
&+c_3[\omega A^2(dA)\omega-\omega A^3(d\omega)+\omega A^2(d\omega)A]\Bigr\}
\end{align*}
となる.被積分関数がゼロと仮定する必要はなく,なんらかの局所関数の全微分の形になっていてその積分が消えるとするだけで良い.この条件は微分を一つだけ含む項と,二つ含む項のそれぞれで別々に満たされる必要がある.なぜなら微分の数が異なる項の間では相殺が起こらないからだ.微分を一つだけ含む項では,$+c_1=-c_2=+c_3\equiv c$とすると
\begin{align*}
\mr{Tr}\Bigl\{&[-\omega A(d\omega)A^2+\omega(d\omega)A^3+\omega(dA)A^2\omega] \\
&-[\omega A^2(d\omega)A-\omega A(d\omega)A^2+\omega A(dA)A\omega] \\
&+[\omega A^2(dA)\omega-\omega A^3(d\omega)+\omega A^2(d\omega)A]\Bigr\} \\
=&\mr{Tr}\Bigl\{ \omega(d\omega)A^3+\omega(dA)A^2\omega -\omega A(dA)A\omega +\omega A^2(dA)\omega -\omega A^3(d\omega) \Bigr\} \\
=&\mr{Tr}\Bigl\{\omega(d\omega)A^3+\omega^2(dA)A^2-\omega^2A(dA)A+\omega^2 A^2(dA)-(d\omega)\omega A^3 \Bigr\} \\
=&-\mr{Tr}\Bigl\{d(\omega^2 A^3)\Bigr\}
\end{align*}
となり全微分$d\mc{F}$の形になることが分かる.残りの項は,もし$c=-1/2$ならば
\begin{align*}
\mr{Tr}\Bigl\{&-\omega^2(dA)^2-\omega A(d\omega)(dA)+\omega(d\omega)A(dA)-\omega (dA)A(d\omega)+\omega(dA)(d\omega)A \\
&-\frac{1}{2}[\omega(dA)(d\omega)A-\omega(dA)A(d\omega)]\\
&+\frac{1}{2}[\omega(d\omega)(dA)A-\omega A(dA)(d\omega)]\\
&-\frac{1}{2}[\omega(d\omega)A(dA)-\omega A(d\omega)(dA)] \Bigr\} \\
=\mr{Tr}\Bigl\{&-\omega^2(dA)^2-\frac{1}{2}\omega A(d\omega)(dA)+\frac{1}{2}\omega(d\omega)A(dA)-\frac{1}{2}\omega(dA)A(d\omega)+\frac{1}{2}\omega(dA)(d\omega)A \\
&+\frac{1}{2}\omega(d\omega)(dA)A-\frac{1}{2}\omega A(dA)(d\omega) \Bigr\}
\end{align*}
ここで
\begin{align*}
d(\omega^2 A (dA))=&(d\omega)\omega A(dA)-\omega(d\omega)A(dA)+\omega^2(dA)^2 \\
d(\omega^2 (dA) A)=&(d\omega)\omega(dA)A-\omega(d\omega)(dA)A+\omega^2(dA)^2 \\
d(\omega (dA) \omega A)=&(d\omega)(dA)\omega A-\omega (dA)(d\omega)A+\omega(dA)\omega(dA) \\
d(\omega A \omega (dA))=&(d\omega)A\omega(dA)-\omega(dA)\omega(dA)+\omega A(d\omega dA)
\end{align*}
を足し合わせると
\begin{align*}
&d\mr{Tr}\Bigl\{\omega^2 A(dA)+\omega^2(dA)A+\frac{1}{2}\omega(dA)\omega A+\frac{1}{2}\omega A\omega (dA)\Bigr\} \\
=&\mr{Tr}\Bigl\{2\omega^2(dA)^2-\omega(d\omega)A(dA)+(d\omega)\omega (dA)A-\omega(d\omega)(dA)A+(d\omega)\omega A(dA) \\
&-\frac{1}{2}\omega(dA)(d\omega)A+\frac{1}{2}\omega A(d\omega)(dA)-\frac{1}{2}\omega(dA)(d\omega)A+\frac{1}{2}\omega A(d\omega)(dA)\Bigr\} \\
=&\mr{Tr}\Bigl\{2\omega^2(dA)^2-\omega(d\omega)A(dA)+(d\omega)\omega (dA)A-\omega(d\omega)(dA)A+(d\omega)\omega A(dA) \\
&+\omega A(d\omega)(dA)-\omega(dA)(d\omega)A\Bigr\} \\
=&-2\mr{Tr}\Bigl\{-\omega^2(dA)^2-\frac{1}{2}\omega A(d\omega)(dA)+\frac{1}{2}\omega(d\omega)A(dA)-\frac{1}{2}\omega(dA)A(d\omega)+\frac{1}{2}\omega(dA)(d\omega)A \\
&+\frac{1}{2}\omega(d\omega)(dA)A-\frac{1}{2}\omega A(dA)(d\omega) \Bigr\}
\end{align*}
となり,これは残りの項と同じであるから,全微分で書けることが分かる.
\begin{align*}
sG=\frac{1}{24\pi^2}\int d \mr{Tr}\left\{-\frac{1}{2}[\omega^2 A(dA)+\omega^2(dA)A+\frac{1}{2}\omega(dA)\omega A+\frac{1}{2}\omega A \omega (dA)]-\omega^2 A^3\right\}
\end{align*}
以上より,$c=-1/2=c_1=-c_2=c_3,c_4=0$を(22.6.13)(22.6.17)に用いると
\begin{align*}
&-iG_\alpha=[\braket{D_\mu J^\mu_\alpha}]_{\mr{anom}} \\
=&-\frac{1}{24\pi^2}\epsilon^{\kappa\nu\lambda\rho}\mr{Tr}\biggl\{T_\alpha \Bigl[\partial_\kappa A_\nu \partial_\lambda A_\rho -\frac{i}{2} \partial_\kappa A_\nu A_\lambda A_\rho +\frac{i}{2} A_\kappa \partial_\nu A_\lambda A_\rho -\frac{i}{2} A_\kappa A_\nu \partial_\lambda A_\rho \Bigr]\biggl\}
\end{align*}
となり,これは(22.3.38)だ.また,この結果はしばしばより簡潔に
\begin{align*}
G[\omega,A]=&\frac{1}{24\pi^2}\int \mr{Tr}\left\{\omega\left[(dA)^2-\frac{1}{2}(dA)A^2+\frac{1}{2}A(dA)A-\frac{1}{2}A^2(dA)\right]\right\} \\
=&\frac{1}{24\pi^2}\int \mr{Tr}\left\{\omega\left[(dA^2)-\frac{1}{2}d(A^3)\right]\right\} \\
=&\frac{1}{24\pi^2}\int \mr{Tr}\left\{\omega d\left[A(dA)-\frac{1}{2}A^3\right]\right\}
\end{align*}
あるいは,場の強度の二形式
\begin{align*}
F\equiv & \frac{1}{2}iT_\alpha F_{\alpha\mu\nu}dx^\mu dx^\nu \\
=&\frac{1}{2}iT_\alpha (\partial_\mu A_{\alpha\nu}-\partial_\nu A_{\alpha\mu}+C_{\alpha\beta\gamma}A_{\beta\mu}A_{\gamma\nu})dx^\mu dx^\nu \\
=&\frac{1}{2}i(2dx^\mu \partial_\mu A_{\alpha\nu}T_\alpha dx^\nu -i[T_\beta,T_\gamma]A_{\beta\mu}dx^\mu A_{\gamma\nu}dx^\nu) \\
=&\frac{1}{2}i(2dx^\mu \partial_\mu A_{\alpha\nu}T_\alpha dx^\nu -i(A_{\beta\mu}T_\beta dx^\mu A_{\gamma\nu}T_\gamma dx^\nu)-i(A_{\gamma\nu}T_\gamma dx^\nu A_{\beta\mu}T_\beta dx^\mu)) \\
=&dA-A^2
\end{align*}
を用いて
\begin{align*}
G[\omega,A]=\frac{1}{24\pi^2}\int \mr{Tr}\left\{\omega d\left[AF+\frac{1}{2}A^3\right]\right\}
\end{align*}
と書ける.アノマリーは必ずしも(22.6.13)の形にする必要はないので,(22.6.20)は$G[\omega,A]$についての唯一の結果ではない.

\vskip\baselineskip

無矛盾条件の解を構成するには,ストラ-ズミノ降下方程式として知られるエレガントな代数方程式がある.この方法を任意の偶数次元時空で説明するのは4次元時空でと同じくらい簡単なので,以下では時空の次元を$2n$とする.まず,$2n$個の時空座標に少なくとも二つの余分な変数を加え,$(2n+2)$形式$\mr{Tr}F^{n+1}$が意味を持つと考える.ここで(22.6.21)より
\begin{align*}
dF=-d(A^2)=-(dA)A+A(dA)=[A,F]
\end{align*}
に注意すると,$\mr{Tr}F^{n+1}$は閉形式だ.
\begin{align*}
d\mr{Tr}F^{n+1}=&\mr{Tr}\left\{(dF)F^n\right\}+\mr{Tr}\left\{F(dF)F^{n-1}\right\}+\cdots +\mr{Tr}\left\{F^n(dF)\right\} \\
=&(n+1)\mr{Tr}\left\{(dF)F^n \right\} \\
=&\mr{Tr}\left\{[A,F]F^n\right\}+\mr{Tr}\left\{F[A,F]F^{n-1}\right\}+\cdots +\mr{Tr}\left\{F^n[A,F]\right\} \\
=&\mr{Tr}\left\{[A,F^{n+1}]\right\}=0
\end{align*}
ポアンカレの定理より,拡張した時空が単連結ならば,その時空上の任意の閉形式は完全形式,すなわち$(2n+1)$形式$\Omega_{2n+1}$が存在して(これはチャーン・サイモンズ形式として知られている)
\begin{align*}
\mr{Tr}F^{n+1}=d\Omega_{2n+1}
\end{align*}
となっている.さらに,ゲージ変換のもとで(22.6.21)は$F\to gFg^{-1}$と変換し,$\mathrm{Tr}F^{n+1}$は明白にゲージ不変で,かつゲージ場のみに依存するから,それはBRST不変だ.(ゲージ変換はBRST変換の一種であることに気付けば良い.)
\begin{align*}
s\mr{Tr}F^{n+1}=0
\end{align*}
(22.6.15)より
\begin{align*}
sd+ds=0
\end{align*}
であるから,これにより$s\Omega_{2n+1}$もまた閉形式となる.
\begin{align*}
d(s\Omega_{2n+1})=-s(d\Omega_{2n+1})=-s\mr{Tr}F^{n+1}=0
\end{align*}
BRST演算子$s$がゴースト数を1つ増やす演算子であることと,(22.6.24)より$\Omega_{2n+1}$はゴースト数ゼロであることに注意する.ポアンカレの定理を再び使うと,閉形式$s\Omega_{2n+1}$は$\omega_\alpha$について1次の$2n$形式$\Omega^1_{2n}$が存在し
\begin{align*}
s\Omega_{2n+1}=d\Omega^1_{2n}
\end{align*}
となる.さらに,$s$はベキ零だから,
\begin{align*}
d(s\Omega^1_{2n})=-s(d\Omega^1_{2n})=-s(s\Omega_{2n+1})=0
\end{align*}
となって,$s\Omega^1_{2n}$もまた閉形式だから,ゴースト場について2次の$(2n-1)$形式$\Omega^2_{2n-1}$も存在して
\begin{align*}
s\Omega^1_{2n}=d\Omega^2_{2n-1}
\end{align*}
となっている.したがって,$\Omega^1_{2n}$はそれ自身BRST不変ではない(つまり$s\Omega^1_{2n}\neq 0$)が,$\Omega^1_{2n}$の$2n$次元時空についての積分はBRST不変だ.
\begin{align*}
s\int_{\mr{spacetime}} \Omega^1_{2n}=\int_{\mr{spacetime}} d\Omega^2_{2n-1}=0
\end{align*}
これより,アノマリー汎関数$G[\omega,A]$の候補$\int \Omega^1_{2n}$を二つの一次微分方程式$d\Omega_{2n+1}=\mr{Tr}F^{n+1}$と$d\Omega^1_{2n}=s\Omega_{2n+1}$を積分して求めることができる.これらの方程式の一般解(唯一の解ではないが)は以下となるらしい.
\begin{align*}
\Omega_{2n+1}&=(n+1)\int^1_0 dt \mr{Tr}\left\{ AF^n_t \right\} \\
\Omega^1_{2n}&=-(n+1)\int^1_0 dt(1-t)\mr{Tr}\sum^{n-1}_{r=0}\left\{ \omega d(F^r_t A F^{n-1-r}_t) \right\}
\end{align*}
ここで,$F_t=tF+(t-t^2)A^2=tdA-t^2A^2$だ.($t:[0,1]$で$F_t:[0,F]$を走る.)積分(22.6.30)を$n=2$で計算すると,
\begin{align*}
\Omega^1_4=&-3\int^1_0 dt(1-t)\mr{Tr}\left\{ \omega d(AF_t+F_t A) \right\} \\
=&-3\int^1_0 dt(1-t)\mr{Tr}\left\{ \omega d[A(tF+t(1-t)A^2)+(tF+t(1-t)A^2)A ]\right\} \\
=&-3\int^1_0 dt \mr{Tr}\left[t(1-t)\omega d(AF)+t(1-t)^2\omega d(A^3)+t(1-t)\omega d(FA)+t(1-t)^2\omega d(A^3)\right] \\
=&-3\mr{Tr}\left[\frac{1}{6}\omega d(AF)+\frac{1}{12}\omega d(A^3)+\frac{1}{6}\omega d(FA)+\frac{1}{12}\omega d(A^3)\right] \\
=&-\frac{1}{2}\mr{Tr}\left[ \omega d(AF+FA+A^3) \right]=-\frac{1}{2}\mr{Tr}\left[\omega d(A(dA)+(dA)A-A^3)\right] \\
=&-\frac{1}{2}\mr{Tr}\left[\omega d(2A(dA)-A^3)\right]=-\frac{1}{4}\mr{Tr}\left[ \omega d\left[A(dA)-\frac{1}{2}A^3\right] \right]
\end{align*}
となって,(22.6.20)は$G[\omega,A]$について4次元時空の場合は$\int \Omega^1_4$に比例する結果を与える.\par
この降下を続けて,他の有用な結果を導くことができる.とくに(22.6.27)と$s$のベキ零性から,
\begin{align*}
d(s\Omega^2_{2n-1})=-s(d\Omega^2_{2n-1})=-s(s\Omega^1_{2n})=0
\end{align*}
であり,再びポアンカレの定理から$s\Omega^2_{2n-1}$が$d\Omega^3_{2n-2}$の形になり,$\Omega^2_{2n-1}$の$2n-1$空間座標積分(例えば$n=2$ならば,3次元空間座標積分)はBRST不変だと分かる.
\begin{align*}
s\int_{\mr{space}} \Omega^2_{2n-1}=\int_{\mr{space}} d\Omega^3_{2n-2}=0
\end{align*}
そのような,ゴースト場について二次のBRST不変な汎関数はいわゆるシュウィンガー項の候補となる.

\vskip\baselineskip

この節で,ここまでに与えたアノマリーの解析は1ループのアノマリーについてのみ厳密に適用できる.量子ゲージ場が結合するカレントに1ループ・アノマリーがある理論は(つまり,破れていないのにアノマリーで破れているというのは)矛盾しているので,それ以上高次を調べる必要はない.しかし,その逆は正しくない.つまり量子ゲージ場の理論が1ループの次数でアノマリーを持たない場合には,高次ループでもアノマリーがないことを示す必要がある.また,量子色力学のカイラル対称性のような大域的対称性にアノマリーを持つ理論の場合はどうせ破れている対称性なのだから矛盾は起きないが,そのような理論でもアノマリーに高次の補正があるかどうか調べる必要はある.\par
BRST変換は場に非線形に作用するから,アノマリーが無くても量子有効作用$\Gamma[\omega,A]$が1ループ近似を越えてBRST不変と期待する必然性はない.17.1節で見たように,この近似を越えるとゲージ場とゴースト場だけでなく,それらの反場も考慮する必要がある.\par
作用に反場を含めてアノマリーを調べると,これがコホモロジー形式で表させることもわかることを見る.この問題の解析はバタリン・ビルコビスキーがマスター方程式と呼んだもののジン・ジュスタン版(17.1.10)にもとづいている.有効作用がBRST不変ならば,すなわちアノマリーがなければ17.1節の議論を繰り返して$(\Gamma,\Gamma)=0$だ.\footnote{(17.1.10)での反括弧は$\chi$と$K$の組によるものだが,(15.9.1)にて$\chi^\ddagger_n=K_n+\delta\Psi/\delta \chi^n$とおけば(17.1.4)にすることができてジン・ジュスタン方程式が導出できる.$\chi^\ddagger$と$K$は反正準変換で関係しているので$\chi$と$\chi^\ddagger$の組での反括弧に置き換えてよい.}これにより,$S$をゼロ次の作用,$\Gamma_1$を量子有効作用への1ループの寄与として,1ループの次数で($S$ と$\Gamma_1$ は共にボゾン的であるから(15.9.15)より反括弧が交換できて)
\begin{align*}
(\Gamma,\Gamma)=(S+\Gamma_1,S+\Gamma_1)=(S,S)+2(S,\Gamma_1)=0
\end{align*}
となり,マスター方程式(15.9.14)より第一項目はゼロとなり,よって$(S,\Gamma_1)=0$となる.\footnote{今回はあらわに書いていないが,$\Gamma_1$には$\hbar$が係数としてついていることに意識すること.17.2節の脚注で説明している通り,ループ数を数えるパラメータとして$\hbar$が使える.この式で$(\Gamma_1,\Gamma_1)$の項が出ていないのは,それが$\hbar$について二次であり,2ループの次数だからだ.}アノマリーがあれば,その代わりに
\begin{align*}
(S,\Gamma_1)=G_1
\end{align*}
となる.ここで$G_1$は場と反場のある汎関数で,$S$と$\Gamma_1$がゴースト数ゼロなので,$G_1$はゴースト数1だ.作用は古典的なマスター方程式$(S,S)=0$を満たすと仮定するから,(15.9.22)より(22.6.35)の反括弧演算はベキゼロだ.したがって
\begin{align*}
(S,G_1)=(S,(S,\Gamma_1))=0
\end{align*}
が成り立つ.しかし,もしゴースト数ゼロのある局所汎関数$F_1$が存在して$G_1=(S,F_1)$が成り立っているならば,作用$\Gamma_1$から項$F_1$を差し引いて,1ループの次数でアノマリーを相殺
\begin{align*}
(S,\Gamma_1-F_1)=0
\end{align*}
させることができる.したがって,候補となるアノマリーは$(S,G_1)=0$の意味で閉形式であるが,ゴースト数1の局所汎関数を用いて$G_1=(S,F_1)$と書くことはできないという意味で完全ではないゴースト数1の局所汎関数$G_1$だ.言い換えると,候補となるアノマリーとは,場とその反場の局所関数の空間における,反括弧演算$X\mapsto (S,X)$のゴースト数1のコホモロジーに相当する.\par
これはまさに,この節の前半で得た結果で,BRST演算子$s$を反括弧$(S,\cdots)$に入れ替えたものだ.(22.6.35)は(15.9.16)(15.9.17)より
\begin{align*}
(S,\Gamma_1)=&\frac{\delta_R S}{\delta \chi^n}\frac{\delta_L \Gamma_1}{\delta \chi^\ddagger_n}-\frac{\delta_R \Gamma_1}{\delta \chi^n}\frac{\delta_L S}{\delta \chi^\ddagger_n} \\
=&-s\chi^\ddagger_n\frac{\delta_L \Gamma_1}{\delta \chi^\ddagger_n}-\frac{\delta_R \Gamma_1}{\delta \chi^n}s\chi_n=-2s\Gamma_1=G_1
\end{align*}
を与える.これは以前の議論と同等だ.しかし,(22.6.35)にもとづいた解析は高次に拡張できるという利点がある.\par
これを見るため,反括弧演算$X\mapsto(S,X)$がゴースト数1の局所汎関数の空間において,コホモロジーが空であって(つまり全てのアノマリーがBRST変換の像として書けるとして),上で述べたように$G_1=(S,\Gamma_1)=0$となるように作用を再定義できると仮定する.2ループでマスター方程式を破るアノマリーは
\begin{align*}
&(S+\Gamma_1+\Gamma_2,S+\Gamma_1+\Gamma_2) \\
&=(S,S)+2(S,\Gamma_1)+2(S,\Gamma_2)+(\Gamma_1,\Gamma_1)+2(\Gamma_1,\Gamma_2)+(\Gamma_2,\Gamma_2)
\end{align*}
の$\hbar$の二次の項を取り出して
\begin{align*}
(\Gamma_1,\Gamma_1)+2(S,\Gamma_2)=G_2
\end{align*}
となる関数$G_2$で表される.しかし仮定より$(S,\Gamma_1)=0,(S,S)=0$だから,そのような任意の$G_2$はヤコビ恒等式(15.9.21)と(15.9.22)より
\begin{align*}
(S,G_2)=&(S,(\Gamma_1,\Gamma_1))+2(S,(S,\Gamma_2))=0
\end{align*}
が成り立つ.これは,コホモロジーが空だという仮定により$G_2=(S,F_2)$と表せることを意味する.ここで$F_2$はゴースト数ゼロの局所汎関数だ.したがって,この次数ではアノマリーは$F_2$を作用から差し引くことで
\begin{align*}
&(S+\Gamma_1+\Gamma_2-F_2,S+\Gamma_1+\Gamma_2-F_2)=0
\end{align*}
となって相殺できる.\par
この議論は全次数に拡張できる.
\begin{align*}
(\Gamma,\Gamma)=&(S+\Gamma_1+\cdots +\Gamma_N,S+\Gamma_1+\cdots +\Gamma_N) \\
=&(S,S) +(S,\Gamma_1)+(\Gamma_1,S) \\
&+ (S,\Gamma_2) \\
&+(\Gamma_1,\Gamma_1)+(\Gamma_2,S) \\
&+\cdots \\
&+(S,\Gamma_N)+(\Gamma_1,\Gamma_{N-1})+\cdots (\Gamma,L,\Gamma_{N-L})+\cdots (\Gamma_N,S) \\
=&\sum^N_{M=0}\sum^M_{L=0}(\Gamma_L ,\Gamma_{M-L})
\end{align*}
マスター方程式でアノマリーを$N-1$次まで相殺し
\begin{align*}
0=G_M=\sum^M_{L=0}(\Gamma_L,\Gamma_{M-L})=2(S,\Gamma_M)+\sum^{M-1}_{L=1}(\Gamma_L,\Gamma_{M-L})
\end{align*}
が全ての$M<N$で成り立っているとする.(ただし$M=1$のときは第二項目はゼロだ.)反括弧$(\Gamma,\Gamma)$の$N$次の項は同様に
\begin{align*}
G_N=2(S,\Gamma_N)+\sum^{N-1}_{M=1}(\Gamma_M,\Gamma_{N-M})
\end{align*}
だから,ヤコビ恒等式(15.9.21)が,(ボゾン演算子三つの場合については全ての項の符号は負であるから)
\begin{align*}
0=-(S,(\Gamma_M,\Gamma_{N-M}))-(\Gamma_{N-M},(S,\Gamma_M))-(\Gamma_M,(\Gamma_{N-M},S))
\end{align*}
であることを用いて,上の表式全体を$S$と反括弧演算してやると
\begin{align*}
(S,G_N)=&\sum^{N-1}_{M=1}\left(S,(\Gamma_M,\Gamma_{N-M})\right) \\
=&-\sum^{N-1}_{M=1}(\Gamma_{N-M},(S,\Gamma_M))-\sum^{N-1}_{M=1}(\Gamma_M,(S,\Gamma_{N-M}))\\
=&-2\sum^{N-1}_{M=1}(\Gamma_{N-M},(S,\Gamma_M))
\end{align*}
ここで$(S,\Gamma_M)$についての仮定を用いれば
\begin{align*}
(S,G_N)=&\sum^{N-1}_{M=1}\sum^{M-1}_{L=1}(\Gamma_{N-M},(\Gamma_L,\Gamma_{M-L})) \\
=&\sum^{N-1}_{M=2}\sum^{M-1}_{L=1}(\Gamma_{N-M},(\Gamma_L,\Gamma_{M-L}))
\end{align*}
を得る.ここで$M=1$でゼロになることを用いた.この反括弧の中の$\Gamma$の添え字の和は$(N-M)+L+(M-L)=N$であることに気付く.これを用いれば,より対称な形に書き換えることができる.
\begin{align*}
(S,G_N)=\sum^{N-2}_{M_1=1}\sum^{N-2}_{M_2=1}\sum^{N-2}_{M_3=1}\delta_{N,M_1+M_2+M_3}(\Gamma_{M_1},(\Gamma_{M_2},\Gamma_{M_3}))
\end{align*}
$M_1,M_2,M_3$の範囲は同じだから,この和の二重反括弧をこれらの添え字の$3!$個の置換についての和と書くことができる.
\begin{align*}
(S,G_N)=&\frac{1}{3!}\sum^{N-2}_{M_1=1}\sum^{N-2}_{M_2=1}\sum^{N-2}_{M_3=1}\delta_{N,M_1+M_2+M_3}(\Gamma_{M_1},(\Gamma_{M_2},\Gamma_{M_3})) \\
&+\frac{1}{3!}\sum^{N-2}_{M_1=1}\sum^{N-2}_{M_3=1}\sum^{N-2}_{M_2=1}\delta_{N,M_1+M_2+M_3}(\Gamma_{M_1},(\Gamma_{M_3},\Gamma_{M_2})) \\
&+\frac{1}{3!}\sum^{N-2}_{M_2=1}\sum^{N-2}_{M_1=1}\sum^{N-2}_{M_3=1}\delta_{N,M_1+M_2+M_3}(\Gamma_{M_2},(\Gamma_{M_1},\Gamma_{M_3})) \\
&+\frac{1}{3!}\sum^{N-2}_{M_2=1}\sum^{N-2}_{M_3=1}\sum^{N-2}_{M_1=1}\delta_{N,M_1+M_2+M_3}(\Gamma_{M_2},(\Gamma_{M_3},\Gamma_{M_1})) \\
&+\frac{1}{3!}\sum^{N-2}_{M_3=1}\sum^{N-2}_{M_1=1}\sum^{N-2}_{M_2=1}\delta_{N,M_1+M_2+M_3}(\Gamma_{M_3},(\Gamma_{M_1},\Gamma_{M_2})) \\
&+\frac{1}{3!}\sum^{N-2}_{M_3=1}\sum^{N-2}_{M_2=1}\sum^{N-2}_{M_1=1}\delta_{N,M_1+M_2+M_3}(\Gamma_{M_3},(\Gamma_{M_2},\Gamma_{M_1})) \\
=&\frac{1}{3!}\sum^{N-2}_{M_1=1}\sum^{N-2}_{M_2=1}\sum^{N-2}_{M_3=1}\delta_{N,M_1+M_2+M_3}\\
&\quad \times \left[(\Gamma_{M_1},(\Gamma_{M_2},\Gamma_{M_3}))+(\Gamma_{M_3},(\Gamma_{M_1},\Gamma_{M_2}))+\Gamma_{M_2},(\Gamma_{M_3},\Gamma_{M_1}))\right] \\
+&\frac{1}{3!}\sum^{N-2}_{M_1=1}\sum^{N-2}_{M_2=1}\sum^{N-2}_{M_3=1}\delta_{N,M_1+M_2+M_3}\\
&\quad\times\left[(\Gamma_{M_1},(\Gamma_{M_3},\Gamma_{M_2}))+(\Gamma_{M_2},(\Gamma_{M_1},\Gamma_{M_3}))+\Gamma_{M_3},(\Gamma_{M_2},\Gamma_{M_1}))\right]
\end{align*}
これはヤコビ恒等式(15.9.21)によりゼロとなるから,$(S,G_N)=0$という結論に達する.仮定したようにコホモロジーが空ならば,$G_N=(S,F_N)$となる局所汎関数$F_N$が存在することになるから,$F_N$を作用から差し引くことでアノマリーが$N$次で相殺される.数学的帰納法により,コホモロジーが空ならば任意の次数のアノマリーが相殺できることが証明できた.これが証明したいことだった.\par
純粋に代数的な方法を用いて,バーニッヒ,ブラント,ヘネーによって半単純ゲージ群の(4次元時空の)ヤンミルズ理論では反括弧$X\mapsto (S,X)$のコホモロジーは,(ゴースト数1の局所汎関数の空間において)ゲージ群の各単純部分群に一つずつの(22.6.20)の形の項の線型結合のみからなることを証明した.この線型結合の係数は未知だ.これにより,理論の物質の内容に全く依らず,1ループの次数でアノマリー$G_1$が自動的に$(S,G_1)=0$を満たすときは,半単純ゲージ群のアノマリーは(22.6.20)の形の項の線型結合でなければならないことがわかる.その各単純群の定数係数は理論の物質の内容を考慮した詳細な計算で決定される.\par
さらに,22.4節では(22.6.20)のトレースがどのようなフェルミオン場についても自動的にゼロとなるようなゲージ群が存在することを見た.(それらは,$n\leq 3$の$SU(n)$群を因子として持たない半単純群ゲージ群だった.)そのような場合には反括弧演算$X\mapsto (S,X)$のゴースト数1のコホモロジーは空らしい.既に見たように,そのような理論では摂動論の任意の次数でアノマリーが存在しないことを意味する.つまり,標準理論のようなゲージ理論ではアノマリーは1ループ近似を越えても存在しない.

\vskip\baselineskip

別の意味でも,アノマリーは反括弧演算のコホモロジーに関係する.スラブノフ・テイラー恒等式(16.4.6)を導く際に,測度$\prod_{n,x}d\chi^n(x)$は問題の対称性変換のもとで不変だと仮定した.17.1節では対称性変換$\chi^n\to \chi^n+\theta \delta S/\delta \chi^\ddagger_n$((15.9.16)参照)のスラブノフ・テイラー恒等式からジン・ジュスタン方程式を導いたから,17.1節で与えたジン・ジュスタン方程式は$\prod_{n,x}d\chi^n(x)$がこの変換のもとで不変でなければ,言い換えると$\Delta S=0$でなければ,成立しない.


\newpage


\subsection{アノマリーとゴールドストンボゾン}
閉じ込められた質量ゼロのフェルミオンを持つある基本理論で線型に実現されていて,破れている大域的対称性群$G$を考える.その例としては3種の質量ゼロのクォークを持つ量子色力学の大域的カイラル対称性$SU(3)\times SU(3)$がある.この基本理論に「架空のゲージ場」を導入し,可能なアノマリーを除いては大域的対称性$G$が局所的になるようにする.この局所対称性はアノマリーで破られるだろう.なぜなら,この拡張した対称性はこの世界においては純粋に大域的で,大域的対称性は必ずしもアノマリーを持たない局所対称性に拡張できなくても良いからだ.しかし,これらのアノマリーは適切な質量ゼロの傍観フェルミオンを加えることによって消去することができる.このように導入したゲージ結合が十分小さくて,傍観フェルミオンがこの非常に弱いゲージ相互作用のみを持つ限り,これらの変更において理論のダイナミクスはあまり変わることはないはずだ.\par
次に,閉じ込められたフェルミオンが観測されないほどの低いエネルギーでの有効場の理論を考える.この理論の自由度は,質量ゼロの架空ゲージボゾンと傍観フェルミオン,そしてゴールドストンボゾンの集合だけだ.このゴールドストンボゾンは破れた対称性の独立なものについて一つずつあり,その場を$\xi_a$と書く.基本理論はゲージ不変かつアノマリーがない,と仮定したから,有効場の理論でも同じことが成り立っていなければならない.しかし傍観フェルミオンは,基本理論のアノマリーを相殺するだけのアノマリーを生じているのであったのだから,それを打ち消すためにはゴールドストンボゾンによるアノマリーと一致していなければならない.したがってゴールドストンボゾンの有効ゲージ理論は,基本理論の閉じ込められたフェルミオンが生じるアノマリーに等しい,架空的な局所対称性のアノマリーを持たなければならない.つまり,(22.6.2)の代わりに,架空のゲージ場とゴールドストンボゾンの有効作用$\Gamma[\xi,A]$は以下の条件を満たさなければならない.
\begin{align*}
\mc{T}_\beta(x)\Gamma[\xi,A]=G_\beta[x;A]
\end{align*}
ここで$G_\beta[x;A]$は基本理論の時点で既に生じていたアノマリー関数であり,つまりこれにはゴールドストンボゾンは含まれていない.また$\mc{T}_\beta$はいまやゲージ場とゴールドストンボゾン場の両方に作用するゲージ群$G$の生成子だ.$\mc{T}_\beta$の添え字$\beta$は,破れていない対称性部分群$H$の独立な生成子$\mc{Y}_i$の集合の添え字$i$と,破れた対称性の生成子$\mc{X}_a$の添え字$a$の値をとる.書く$\mc{X}_a$について一つずつゴールドストンボゾン$\xi_a$が存在する.\par

\vskip\baselineskip

(22.7.1)は,現実の弱く結合するゲージ場とゴールドストンボゾンの相互作用を調べることにも使えるらしい.たとえば,基本理論がクォークに結合した電弱ゲージ場を含む場合には,「架空の」ゲージ場と呼んだものを電弱ゲージ場とする.そのような場合には,「傍観」フェルミオンも実際に存在し,閉じ込められたフェルミオンのループによって生じる,現実の弱く結合したゲージ場のゲージ対称性のアノマリーを相殺しなければならない.これは,22.4節でレプトンがクォークによって生じる電弱アノマリーを相殺することに相当する.

\vskip\baselineskip

さて,(22.7.1)から得られる結論を論じる.この目的のために,ゲージ対称性の生成子$\mc{T}_\beta(x)$を計算するには,一般の群の線型変換
\begin{align*}
A_{\alpha\mu}(x)\to&A'_{\alpha\mu}(x)= A_{\alpha\mu}(x)-i\int d^4x \zeta_\beta(y) \mc{T}^A_\beta(y) A_{\alpha\mu}(x) \\
=&\left[1-i\int d^4x \zeta_\beta(y) \mc{T}^A_\beta(y) \right]A_{\alpha\mu}(x)=\exp\left(-i\int \zeta_\beta(y)\mc{T}^A_\beta(y)d^4x\right) A_{\alpha\mu}(x) \\
\xi_a(x)\to& \xi'_a(x)=\xi_a-i\int d^4x \zeta_\beta(y) \mc{T}^\xi_\beta(y)\xi_a(x) \\
=&\exp\left(-i\int \zeta_\beta(y)\mc{T}^\xi_\beta(y)d^4x\right)\xi_a(x) \\
\therefore g=& \exp\left(-i\int \zeta_\beta(x)\mc{T}_\beta(x)d^4x\right)
\end{align*}
のもとでゴールドストンボゾンゴールドストンボゾン場$\xi_a(x)$は(19.6.18)によって与えられる場$\xi'_a(x)$に変換され,ゲージ場$A^\mu_\alpha(x)$はゲージ変換された場$A'^\mu_\alpha(x)$に変換されることに気付くことだ.これにより
\begin{align*}
\mc{T}_\beta(x)=\mc{T}_\beta^A(x)+\mc{T}_\beta^\xi(x)
\end{align*}
となる(まぁ当たり前だと思うが).ここで$\mc{T}_\beta^A(x)$はゲージ場に働き,(22.6.1)で与えられる(導出は(22.3.31)(22.3.32)のときと同様).
\begin{align*}
-i\mc{T}^A_\beta=-\frac{\partial}{\partial x^\mu}\frac{\delta}{\delta A_{\beta\mu}(x)}-C_{\beta\gamma\alpha} A_{\gamma\mu}(x) \frac{\delta}{\delta A_{\alpha\mu}(x)}
\end{align*}
ここで$C_{\alpha\beta\gamma}$はゲージ群$G$の完全反対称構造定数だ.また$\mc{T}_\beta^\xi(x)$は(19.6.17)の微小極限で与えられる.これは$\gamma(\xi)=\exp(i\xi_a X_a)$の指数パラメータ表示(19.6.12)では以下となる.
\begin{align*}
&g\gamma(\xi(x))=\gamma(\xi'(x))h(\xi,g) \\
&\exp(i\Lambda_\beta(x)T_\beta)\exp(i\xi_a(x)X_a)=\exp\left(-i\int \Lambda_\beta(y)\mc{T}_\beta(y)d^4y\right)\left\{\exp(i\xi_a(x)X_a)\right\} \exp(i\theta_i(x) Y_i) \\
&(1+i\Lambda_\beta(x)T_\beta)\exp(i\xi_a(x)X_a)=\left(1-i\int \Lambda_\beta(y)\mc{T}_\beta(y)d^4y \right)\exp(i\xi_a(x)X_a)(1+i\theta_i(x)Y_i) \\
&i\Lambda_\beta(x)T_\beta\exp(i\xi_a(x)X_a)=-i\int \Lambda_\beta(y)\mc{T}_\beta(y)d^4y\exp(i\xi_a(x)X_a) +\exp(i\xi_a(x)X_a)i\theta_i(x)Y_i \\
&\Lambda_\beta(x)T_\beta\exp(i\xi_a(x)X_a)+\int \Lambda_\beta(y)\mc{T}_\beta(y)d^4y\exp(i\xi_a(x)X_a)=\exp(i\xi_a(x)X_a)\theta_i(x)Y_i \\
&\exp(-i\xi_a(x)X_a)\left[\Lambda_\beta(x)T_\beta+\int \Lambda_\beta(y)\mc{T}_\beta(y)d^4y\right]\exp(i\xi_a(x)X_a)=\theta_i(x)Y_i
\end{align*}
((22.7.4)の形式にしたければ,デルタ関数などを用いて積分を外して$\Lambda_\beta(x)$で全体を割れば良いと思うが,誤植があるのは明らかだし今後の証明においてはこの形式のまま使った方が便利なので,このままにしておく.重要なのは,左辺のこの和が全体として破れていない対称性の生成子に比例するという事実だ.)ここで$T_\beta$は任意の表現で$G$の生成子を表す行列だ.通常通り$T_\beta$は,それぞれ破れた対称性と破れていない対称性の生成子$X_a,Y_i$の集合に分けられる.\par
アノマリー関数$G_\beta[x;A]$については無矛盾条件(22.6.6)
\begin{align*}
\mc{T}_\alpha(x)G_\beta[y;A]-\mc{T}_\beta(y)G_\alpha[x;A]=iC_{\alpha\beta\gamma}\delta^4(x-y)G_\gamma[x;A]
\end{align*}
と,破れていない対称性にはアノマリーがないこと
\begin{align*}
\mc{T}_i(x)\Gamma[\xi,A]=G_i[x;A]=0
\end{align*}
のみを仮定する.破れていない対称性部分群の生成子について,トレース$\mr{Tr}[T_i\{T_j ,T_k \}]$がゼロとなるかぎり,作用に局所汎関数を加えて(22.7.6)が満たされるようにすることが常に可能らしい.\par
(22.7.5)と(22.7.6)の仮定のもとでは,アノマリーを持つスラブノフ・テイラー恒等式(16.4.6)(22.7.1)の解を見つけることが「常に」可能だ.
\begin{align*}
\Gamma[\xi,A]=-i\int^1_0 dt \int \xi_b(y)G_b[y;A_{-t\xi}]d^4y
\end{align*}
ここで$[A_{-t\xi}(x)]_\mu$は,$A_\mu\equiv T_\beta A_{\beta\mu}$に$\Lambda_a=-t\xi_a$と$\Lambda_i=0$のゲージ変換(15.1.17)を働かせた結果だ.
\begin{align*}
[A_{-t\xi}(x)]_\mu=&\exp(i\Lambda_\alpha(x)T_\alpha)\left[A_\mu(x)+i\partial_\mu \right]\exp(-i\Lambda_\alpha(x)T_\alpha) \\
=&\exp(-it\xi_a(x) X_a)A_\mu(x)\exp(it\xi_a(x)X_a)-i\left[\partial_\mu \exp(-it\xi_a(x)X_a)\right]\exp(it\xi_a(x)X_a)
\end{align*}
作用(22.7.7)が(22.7.1)を満たす証明の概略は以下となる.局所的な生成子$\mc{T}_\beta(x)$を使う代わりに任意の関数$\eta_\beta(x)$を導入し
\begin{align*}
\mc{T}[\eta]=\int d^4x \eta_\beta(x)\mc{T}_\beta(x)
\end{align*}
と定義する.(22.7.7)両辺に$\mc{T}[\eta]$を作用させたとき,振る舞いが気になるのは右辺に生じる$\mc{T}[\eta]\xi_b(x)$と$\mc{T}[\eta]G_b[y,A]$だ.$\mc{T}[\eta]\xi_b(x)$を計算するには,行列
\begin{align*}
\eta_{-t\xi}(x)&\equiv \exp(-it\xi_a(x)X_a)\left[\eta(x)+\mc{T}[\eta]\right]\exp(it\xi_a(x)X_a) \\
&\equiv [\eta_{-t\xi}(x)]_\beta T_\beta
\end{align*}
を導入する.ここで$\eta(x)\equiv \eta_\beta(x)T_\beta$だ.これを$t$微分すると(少し複雑だが単純に積の微分だ.)
\begin{align*}
\frac{\partial}{\partial t}\eta_{-t\xi}(x)=&-i\xi_b X_b e^{-it\xi_a X_a}\left[\eta(x)+\mc{T}[\eta]\right]e^{it\xi_a X_a} \\
&+ e^{-it\xi_a X_a}\left[\eta(x)+\mc{T}[\eta]\right]\Bigl\{i\xi_b X_b e^{it\xi_a X_a}\Bigr\} \\
=&-i\xi_b X_b\eta_{-t\xi}(x) \\
&+i e^{-it\xi_a X_a}\eta(x)\xi_b X_b e^{it\xi_a X_a} \\
&+ie^{-it\xi_a X_a}\mc{T}[\eta]\Bigl\{ e^{it\xi_a X_a}\xi_b X_b\Bigr\} \\
=&-i\xi_b X_b\eta_{-t\xi}(x) \\
&+ie^{-it\xi_a X_a}\eta(x)\xi_b X_b e^{it\xi_a X_a} \\
&+ie^{-it\xi_a X_a}e^{it\xi_a X_a}\Bigl\{\mc{T}[\eta]\xi_b\Bigr\}X_b \\
&+ie^{-it\xi_a X_a}\Bigl\{\mc{T}[\eta]e^{it\xi_a X_a}\Bigr\}\xi_b X_b \\
=&-i\xi_b X_b\eta_{-t\xi}(x) \\
&+ie^{-it\xi_a X_a}\eta(x)e^{it\xi_a X_a}\xi_b X_b \\
&+i\Bigl\{\mc{T}[\eta]\xi_b \Bigr\}X_b \\
&+ie^{-it\xi_a X_a}\Bigl\{\mc{T}[\eta]e^{it\xi_a X_a}\Bigr\}\xi_b X_b \\
=&-i\xi_b X_b\eta_{-t\xi}(x) +i\eta_{-t\xi}(x)\xi_b X_b +i\Bigl\{\mc{T}[\eta]\xi_b \Bigr\}X_b \\
=&-i[\xi_a X_a,\eta_{-t\xi}(x)]+i\Bigl\{\mc{T}[\eta]\xi_b \Bigr\}X_b 
\end{align*}
三つ目の等号では$\mc{T}[\eta]$の積のライプニッツ則を用いていることに注意.よって
\begin{align*}
&\frac{\partial}{\partial t}\eta_{-t\xi}(x)=\frac{\partial}{\partial t}[\eta_{-t\xi}(x)]_\beta T_\beta \\
=&-i[\xi_a X_a,[\eta_{-t\xi}(x)]_\gamma T_\gamma]+i\left(\mc{T}[\eta]\xi_b \right)X_b \\
=&\xi_a [\eta_{-t\xi}(x)]_\gamma C_{a\gamma \beta}T_\beta+i\left(\mc{T}[\eta]\xi_b \right)X_b
\end{align*}
より,$X_b$の係数比較より
\begin{align*}
\mc{T}[\eta]\xi_b(x)=-i\frac{\partial}{\partial t}[\eta_{-t\xi}(x)]_b+i C_{a\gamma b}\xi_a(x)[\eta_{-t\xi}(x)]_\gamma
\end{align*}
が得られる.$\mc{T}[\eta]G_b[y,A]$を求めるには,ゲージ場に$\mc{T}[\eta]$を作用させ
\begin{align*}
\mc{T}[\eta][A_{-t\xi}(x)]_\mu=\left(\mc{T}^A[\eta_{-t\xi}]A_\mu(x)\right)_{A\to A_{-t\xi}}
\end{align*}
を示す必要がある.右辺の$A\to A_{-t\xi}$は,$\mc{T}^A$を$A_\mu$に作用させた後に$A_\mu$を$[A_{-t\xi}]_\mu$にする操作である.これを示すのはかなり労力がいるが,右辺から左辺を示すのが楽だろう.以下で示す.\par
まず(22.7.9)(22.7.3)より
\begin{align*}
\left(\mc{T}^A[\eta_{-t\xi}]A_\mu \right)_{A\to A_{-t\xi}}=&\left(\int d^4 y [\eta_{-t\xi}(y)]_\beta \mc{T}^A_\beta(y) A_{\gamma \mu}T_\gamma \right)_{A\to A_{-t\xi}} \\
=&i\Bigl\{\partial _\mu [\eta_{-t\xi}(x)]_\gamma +C_{\gamma\beta\alpha}[\eta_{-t\xi}(x)]_\alpha A_{\beta\mu}(x)\Bigr\}_{A\to A_{-t\xi}}T_\gamma \\
=&\Bigl\{ i\partial_\mu \eta_{-t\xi}(x)+[T_\beta,T_\alpha][\eta_{-t\xi}(x)]_\alpha A_{\beta\mu}(x) \Bigr\}_{A\to A_{-t\xi}} \\
=&\Bigl\{i\partial_\mu \eta_{-t\xi}(x)+[A_{\mu}(x),\eta_{-t\xi}(x)]\Bigr\}_{A\to A_{-t\xi}} \\
=&i\partial_\mu \eta_{-t\xi}(x)+[[A_{-t\xi}(x)]_\mu,\eta_{-t\xi}]
\end{align*}
となり,(22.7.8)(22.7.10)より
\begin{align*}
=&i\partial_\mu\Bigl\{ e^{-it\xi_a X_a}[\eta(x)+\mc{T}[\eta]]e^{it\xi_a X_a} \Bigr\} \\
&+\Bigl[ e^{-it\xi_a X_a}A_\mu(x)e^{it\xi_a X_a},e^{-it\xi_a X_a}[\eta(x)+\mc{T}[\eta]]e^{it\xi_a X_a} \Bigr] \\
&+\Bigl[-i[\partial_\mu e^{-it\xi_a X_a}]e^{it\xi_a X_a},e^{-it\xi_a X_a}[\eta(x)+\mc{T}[\eta]]e^{it\xi_a X_a} \Bigr] \\
=&i\partial_\mu\Bigl\{ e^{-it\xi_a X_a}\eta(x)e^{it\xi_a X_a} \Bigr\}+i\partial_\mu\Bigl\{ e^{-it\xi_a X_a}\mc{T}[\eta]e^{it\xi_a X_a} \Bigr\} \\
&+e^{-it\xi_a X_a}A_\mu(x)[\eta(x)+\mc{T}[\eta]]e^{it\xi_a X_a} \\
&-e^{-it\xi_a X_a}[\eta(x)+\mc{T}[\eta]]e^{it\xi_a X_a}e^{-it\xi_a X_a}A_\mu(x)e^{it\xi_a X_a} \\
&-i[\partial_\mu e^{-it\xi_a X_a}][\eta(x)+\mc{T}[\eta]]e^{it\xi_a X_a} \\
&+ie^{-it\xi_a X_a}[\eta(x)+\mc{T}[\eta]]e^{it\xi_a X_a}[\partial_\mu e^{-it\xi_a X_a}]e^{it\xi_a X_a}
\end{align*}
ここで注意するべきことは,$\mc{T}[\eta]$が作用しているのは後ろに隣接している$\exp(it\xi_a X_a)$のみで,そのさらに後ろにはこの時点では例外なく作用していない,ということだ.複雑になってきたので,各項に分けて計算する.第四項目を
\begin{align*}
&-e^{-it\xi_a X_a}[\eta(x)+\mc{T}[\eta]]e^{it\xi_a X_a}e^{-it\xi_a X_a}A_\mu(x)e^{it\xi_a X_a} \\
=&-e^{-it\xi_a X_a}\eta(x)A_\mu(x)e^{it\xi_a X_a} \\
&-e^{-it\xi_a X_a}\Bigl\{\mc{T}[\eta]e^{it\xi_a X_a}\Bigr\}e^{-it\xi_a X_a}A_\mu(x)e^{it\xi_a X_a} \\
=&-e^{-it\xi_a X_a}\eta(x)A_\mu(x)e^{it\xi_a X_a} \\
&-e^{-it\xi_a X_a}\mc{T}[\eta]\Bigl\{e^{it\xi_a X_a}e^{-it\xi_a X_a }A_\mu(x)e^{it\xi_a X_a}\Bigr\} \\
&+e^{-it\xi_a X_a}e^{it\xi_a X_a}\mc{T}[\eta]\Bigl\{e^{-it\xi_a X_a}A_\mu(x)e^{it\xi_a X_a}\Bigr\} \\
=&-e^{-it\xi_a X_a}\eta(x)A_\mu(x)e^{it\xi_a X_a} \\
&-e^{-it\xi_a X_a}\mc{T}[\eta]\Bigl\{A_\mu(x)e^{it\xi_a X_a}\Bigr\} \\
&+\mc{T}[\eta]\Bigl\{e^{-it\xi_a X_a}A_\mu(x)e^{it\xi_a X_a}\Bigr\}
\end{align*}
ここまで変形する.次に第六項目は
\begin{align*}
&ie^{-it\xi_a X_a}[\eta(x)+\mc{T}[\eta]]e^{it\xi_a X_a}[\partial_\mu e^{-it\xi_a X_a}]e^{it\xi_a X_a} \\
=&ie^{-it\xi_a X_a}\eta(x)e^{it\xi_a X_a}[\partial_\mu e^{-it\xi_a X_a}]e^{it\xi_a X_a} \\
&+ie^{-it\xi_a X_a}\Bigl\{\mc{T}[\eta]e^{it\xi_a X_a}\Bigr\}[\partial_\mu e^{-it\xi_a X_a}]e^{it\xi_a X_a} \\
=&-ie^{-it\xi_a X_a}\eta(x)[\partial_\mu e^{it\xi_a X_a}] \\
&+ie^{-it\xi_a X_a}\mc{T}[\eta]\Bigl\{e^{it\xi_a X_a}[\partial_\mu e^{-it\xi_a X_a}]e^{it\xi_a X_a}\Bigr\} \\
&-ie^{-it\xi_a X_a}e^{it\xi_a X_a}\mc{T}[\eta]\Bigl\{[\partial_\mu e^{-it\xi_a X_a}]e^{it\xi_a X_a}\Bigr\} \\
=&-ie^{-it\xi_a X_a}\eta(x)[\partial_\mu e^{it\xi_a X_a}] \\
&-ie^{-it\xi_a X_a}\mc{T}[\eta][\partial_\mu e^{it\xi_a X_a}] \\
&-i\mc{T}[\eta]\Bigl\{[\partial_\mu e^{-it\xi_a X_a}]e^{it\xi_a X_a}\Bigr\}
\end{align*}
ここまで変形する.ここで用いたのは以下の関係式だ.
\begin{align*}
0=&\partial_\mu 1=\partial_\mu [e^{it\xi_a X_a}e^{-it\xi_a X_a}] \\
=&[\partial_\mu e^{it\xi_a X_a}]e^{-it\xi_a X_a}+e^{it\xi_a X_a}[\partial_\mu e^{-it\xi_a X_a}] \\
\therefore \quad & e^{it\xi_a X_a}[\partial_\mu e^{-it\xi_a X_a}]e^{it\xi_a X_a}=-[\partial_\mu e^{it\xi_a X_a}]
\end{align*}
ここまでの変形を用いることによって,元の式は
\begin{align*}
&\left(\mc{T}^A[\eta_{-t\xi}]A_\mu \right)_{A\to A_{-t\xi}} \\
=&i\partial_\mu\Bigl\{ e^{-it\xi_a X_a}\eta(x)e^{it\xi_a X_a} \Bigr\}+i\partial_\mu\Bigl\{ e^{-it\xi_a X_a}\mc{T}[\eta]e^{it\xi_a X_a} \Bigr\} \\
&+e^{-it\xi_a X_a}A_\mu(x)\eta(x)e^{it\xi_a X_a}+e^{-it\xi_a X_a}A_\mu(x)\mc{T}[\eta]e^{it\xi_a X_a} \\
&-e^{-it\xi_a X_a}\eta(x)A_\mu(x)e^{it\xi_a X_a} \\
&-e^{-it\xi_a X_a}\mc{T}[\eta]\Bigl\{A_\mu(x)e^{it\xi_a X_a}\Bigr\} \\
&+\mc{T}[\eta]\Bigl\{e^{-it\xi_a X_a}A_\mu(x)e^{it\xi_a X_a}\Bigr\} \\
& -i[\partial_\mu e^{-it\xi_a X_a}]\eta(x)e^{it\xi_a X_a}-i[\partial_\mu e^{-it\xi_a X_a}]\mc{T}[\eta]e^{it\xi_a X_a}  \\
&-ie^{-it\xi_a X_a}\eta(x)[\partial_\mu e^{it\xi_a X_a}] \\
&-ie^{-it\xi_a X_a}\mc{T}[\eta][\partial_\mu e^{it\xi_a X_a}] \\
&-i\mc{T}[\eta]\Bigl\{[\partial_\mu e^{-it\xi_a X_a}]e^{it\xi_a X_a}\Bigr\}
\end{align*}
もう少しだ.第七項目と第十二項目以外が全て相殺すればゴールとなる.さて,第一項目が
\begin{align*}
&i\partial_\mu\Bigl\{ e^{-it\xi_a X_a}\eta(x)e^{it\xi_a X_a} \Bigr\} \\
=&i[\partial_\mu e^{-it\xi_a X_a}]\eta(x)e^{it\xi_a X_a} +ie^{-it\xi_a X_a}\eta(x)[\partial_\mu e^{it\xi_a X_a}] \\
&+ie^{-it\xi_a X_a}[\partial_\mu \eta(x)]e^{it\xi_a X_a}
\end{align*}
であり,第三項目と第五項目が
\begin{align*}
&e^{-it\xi_a X_a}A_\mu(x)\eta(x)e^{it\xi_a X_a}-e^{-it\xi_a X_a}\eta(x)A_\mu(x)e^{it\xi_a X_a} \\
=&e^{-it\xi_a X_a}[A_\mu(x),\eta(x)]e^{it\xi_a X_a}
\end{align*}
とまとめられ,第四項目と第六項目が
\begin{align*}
&e^{-it\xi_a X_a}A_\mu(x)\mc{T}[\eta]e^{it\xi_a X_a} -e^{-it\xi_a X_a}\mc{T}[\eta]\{A_\mu(x)e^{it\xi_a X_a}\} \\
=&-e^{-it\xi_a X_a}\{\mc{T}[\eta]A_\mu(x)\}e^{it\xi_a X_a} \\
=&-e^{-it\xi_a X_a}\{i\partial_\mu \eta(x)+[A_\mu,\eta(x)]\}e^{it\xi_a X_a}
\end{align*}
とまとめられることを用いると,第一項目と第三項目・第五項目と第四項目・第六項目と第八項目と第十項目は相殺してゼロとなる.残りの第二項目と第九項目と第十一項目はライプニッツ則で明らかに相殺する.したがって結局残るのは
\begin{align*}
&\left(\mc{T}^A[\eta_{-t\xi}]A_\mu \right)_{A\to A_{-t\xi}} \\
=&\mc{T}[\eta]\Bigl\{e^{-it\xi_a X_a}A_\mu(x)e^{it\xi_a X_a}\Bigr\}-i\mc{T}[\eta]\Bigl\{[\partial_\mu e^{-it\xi_a X_a}]e^{it\xi_a X_a}\Bigr\} \\
=&\mc{T}[\eta][A_{-t\xi}(x)]_\mu
\end{align*}
となって,(22.7.12)が示される.\par
(22.7.12)と無矛盾条件(22.7.5)を用いれば
\begin{align*}
\mc{T}[\eta]G_b[y;A_{-t\xi}]=&\Bigl(\mc{T}^A[\eta_{-t\xi}] G_b[y;A]\Bigr)_{A\to A_{-t\xi}} \\
=&\int d^4x[\eta_{-t\xi}(x)]_\gamma \left(\mc{T}^A_\gamma(x)G_b[y,A]\right)_{A\to A_{-t\xi}} \\
=&\int d^4x[\eta_{-t\xi}(x)]_\gamma \left(\mc{T}^A_b(y)G_\gamma[x,A]+iC_{\gamma b \alpha }\delta^4(x-y)G_\alpha[x;A]\right)_{A\to A_{-t\xi}} \\
=&\int d^4x[\eta_{-t\xi}(x)]_\gamma \left(\mc{T}^A_b(y)G_\gamma[x,A]\right)_{A\to A_{-t\xi}}+iC_{\gamma b\alpha}[\eta_{-t\xi}(x)]_\gamma G_\alpha[y;A_{-t\xi}] \\
=&\int d^4x[\eta_{-t\xi}(x)]_a \left(\mc{T}^A_b(y)G_a[x,A]\right)_{A\to A_{-t\xi}}+iC_{\gamma ba}[\eta_{-t\xi}(x)]_\gamma G_a[y;A_{-t\xi}]
\end{align*}
最後の等号では(22.7.6)を用いた.ここまで計算した結果を用いると,(22.7.7)の両辺に$\mc{T}[\eta]$を作用させると
\begin{align*}
\mc{T}[\eta]\Gamma[\xi,A]=&-i\mc{T}[\eta]\int^1_0 dt \int \xi_b(y) G_b[y;A_{-t\xi}]d^4y \\
=&-i\int^1_0 dt \int d^4y\Bigl\{\mc{T}[\eta]\xi_b(y)\Bigr\} G_b[y;A_{-t\xi}]d^4y-i\int^1_0 dt \int \xi_b(y)\Bigl\{ \mc{T}[\eta]G_b[y;A_{-t\xi}]\Bigr\} \\
=&-i\int^1_0 dt\int d^4y\left\{-i\frac{\partial}{\partial t}[\eta_{-t\xi}(y)]_b+i C_{a\gamma b}\xi_a(y)[\eta_{-t\xi}(y)]_\gamma\right\}G_b[y;A_{-t\xi}] \\
&-i\int^1_0 dt\int d^4y\xi_b(y)\biggl\{\int d^4x[\eta_{-t\xi}(x)]_a \left(\mc{T}^A_b(y)G_a[x,A]\right)_{A\to A_{-t\xi}} \\
&\qquad+iC_{\gamma ba}[\eta_{-t\xi}(x)]_\gamma G_a[y;A_{-t\xi}]\biggr\} \\
=&-\int^1_0 dt \int d^4y \frac{\partial}{\partial t}[\eta_{-t\xi}(y)]_b G_b[y;A_{-t\xi}]+\int^1_0 dt \int d^4y C_{a\gamma b}\xi_a(y) [\eta_{-t\xi}(y)]_\gamma G_b[y;A_{-t\xi}] \\
&-i\int^1_0 dt \int d^4y \int d^4x \xi_b(y)[\eta_{-t\xi}(x)]_a \left(\mc{T}^A_b(y)G_a[x,A]\right)_{A\to A_{-t\xi}} \\
&-\int^1_0 dt \int d^4y C_{a\gamma b}\xi_a(y) [\eta_{-t\xi}(y)]_\gamma G_b[y;A_{-t\xi}] \\
=&-\int^1_0 dt \int d^4y\frac{\partial}{\partial t}[\eta_{-t\xi}(y)]_b G_b[y;A_{-t\xi}] \\
&-i\int^1_0 dt \int d^4x  [\eta_{-t\xi}(x)]_a \left(\int d^4y\xi_b(y)\mc{T}^A_b(y)G_a[x,A]\right)_{A\to A_{-t\xi}} \\
=&\int^1_0 dt \int d^4y\left\{ -\frac{\partial}{\partial t}[\eta_{-t\xi}(y)]_b G_b[y;A_{-t\xi}]-i[\eta_{-t\xi}(y)]_a \left(\mc{T}^A[\xi]G_a[y,A]\right)_{A\to A_{-t\xi}} \right\}
\end{align*}
となる.また
\begin{align*}
i\Bigl( \mc{T}^A[\xi]A_\mu (x) \Bigr)_{A\to A_{-t\xi}}=&-(\partial_\mu \xi_\alpha +C_{\alpha \beta \gamma}\xi_\gamma A_{\beta\mu})_{A\to A_{-t\xi}} T_\alpha \\
=&(-\partial_\mu \xi_\alpha T_\alpha +i[T_\beta,T_\gamma]\xi_\gamma A_{\beta\mu})_{A\to A_{-t\xi}} \\
=&(-\partial_\mu \xi_a X_a +i[A_\mu,\xi_aX_a])_{A\to A_{-t\xi}} \quad \because \xi_i=0 \\
=&-\partial_\mu \xi_a X_a +i[[A_{-t\xi}]_\mu,\xi_aX_a] \\
=&-\partial_\mu \xi_a X_a +i[e^{-it\xi_a X_a}A_\mu e^{it\xi_a X_a}-i[\partial_\mu e^{-it\xi_a X_a}]e^{it\xi_a X_a},\xi_aX_a]\\
=&-\partial_\mu \xi_a X_a +i[e^{-it\xi_a X_a}A_\mu e^{it\xi_a X_a},\xi_aX_a]+[[\partial_\mu e^{-it\xi_a X_a}]e^{it\xi_a X_a},\xi_aX_a]
\end{align*}
一方
\begin{align*}
\frac{\partial}{\partial t}[A_{-t\xi}(x)]_\mu=&-i\xi_a X_a e^{it\xi_a X_a}A_\mu e^{it\xi_a X_a}+e^{-it\xi_a X_a}A_\mu e^{it\xi_a X_a}i\xi_a X_a \\
&-i\left[\partial_\mu \{ -i\xi_a X_a e^{-it\xi_a X_a} \} \right]e^{it\xi_a X_a}-i\left[\partial_\mu e^{-it\xi_a X_a} \right](i\xi_a X_a)e^{it\xi_a X_a} \\
=&i[e^{-it\xi_a X_a}A_\mu e^{it\xi_a X_a},\xi_aX_a] \\
&-\partial_\mu \xi_a X_a-\xi_a X_a [\partial_\mu e^{-it\xi_a X_a}]e^{it\xi_a X_a}+[\partial_\mu e^{-it\xi_a X_a}]e^{it\xi_a X_a}\xi_a X_a
\end{align*}
であるから
\begin{align*}
\frac{\partial}{\partial t}[A_{-t\xi}(x)]_\mu=i\Bigl( \mc{T}^A[\xi]A_\mu (x) \Bigr)_{A\to A_{-t\xi}}
\end{align*}
が示される.したがって
\begin{align*}
\mc{T}[\eta]\Gamma[\xi,A]=&\int^1_0 dt \int d^4y\left\{ -\frac{\partial}{\partial t}[\eta_{-t\xi}(y)]_b G_b[y;A_{-t\xi}]-i[\eta_{-t\xi}(y)]_b \left(\mc{T}^A[\xi]G_b[y,A]\right)_{A\to A_{-t\xi}} \right\} \\
=&\int^1_0 dt \int d^4y\left\{ -\frac{\partial}{\partial t}[\eta_{-t\xi}(y)]_b G_b[y;A_{-t\xi}]-[\eta_{-t\xi}(y)]_b \frac{\partial}{\partial t}G_b[y;A_{-t\xi}] \right\} \\
=&-\int^1_0 dt \int d^4y \frac{\partial}{\partial t}\Bigl\{ [\eta_{-t\xi}(y)]_b G_b[y;A_{-t\xi}] \Bigr\} \\
=&-\int d^4y \biggl[[\eta_{-t\xi}(y)]_b G_b[y;A_{-t\xi}] \biggr]^1_0
\end{align*}
となる.$t=1$では(22.7.10)は
\begin{align*}
\eta_{-\xi}(x)&\equiv \exp(-i\xi_a(x)X_a)\left[\eta(x)+\mc{T}[\eta]\right]\exp(i\xi_a(x)X_a)
\end{align*}
となるが,これは(22.7.4)の誤植を訂正した式により,破れていない対称性部分群$H$の生成子の線型結合だと分かるので,任意の破れた対称性の生成子$X_b$の係数$[\eta_{-\xi}(y)]_b$はゼロとなる!また,(22.7.10)と(22.7.8)から$t=0$では$\eta_{-t\xi}(y)=\eta(y)$かつ$[A_{-t\xi}(y)]_\mu=A_\mu(y)$となる.したがって(22.7.16)より
\begin{align*}
\mc{T}[\eta]\Gamma[\xi,A]=\int d^4y [\eta_0(y)]_b G_b[y;A_0]=\int d^4y \eta_b(y)G_b[y;A]
\end{align*}
(22.7.6)(22.7.9)よりこれは
\begin{align*}
\mc{T}_\beta(y)\Gamma[\xi,A]=G_\beta[y;A]
\end{align*}
と同等の結果であることがわかる!これで証明が完成した!\par
この解は唯一の解ではない.しかしこれは$\xi=0$でゼロになるという条件のもとでは唯一の解となる.これを見るには,ゲージ変換によって
\begin{align*}
\exp\left(-i \int \eta_\beta (x)\mc{T}_\beta(x)d^4x \right)\Gamma[\xi,A]=\Gamma[\xi',A']
\end{align*}
となることに注目すれば良い.
\begin{align*}
\exp(z)=1+\int^1_0 dt \exp(zt)z
\end{align*}
を用いると(22.7.1)より
\begin{align*}
\Gamma[\xi',A']=&\Gamma[\xi,A]+\int^1_0 dt \exp\left(-it \int \eta_\beta (x)\mc{T}_\beta(x)d^4x \right)\left[ -i \int \eta_\beta (y)\mc{T}_\beta(y)d^4y \right] \Gamma[\xi,A] \\
=&\Gamma[\xi,A]-i\int^1_0 dt \exp\left(-it \int \eta_\beta (x)\mc{T}^A_\beta(x)d^4x \right)\int \eta_b(y)G_b[y;A]d^4y
\end{align*}
となる.もし$\eta_a=-\xi_a$かつ$\eta_i=0$とすると
\begin{align*}
&g\gamma(\xi)=\gamma(\xi')h \\
&\exp(-i\xi_a X_a +0)\exp(i\xi_a X_a)=\exp(i\xi'_a X_a) \\
&\xi'_a=0
\end{align*}
となり,この場合は仮定により$\Gamma[\xi',A']$はゼロとなる.汎関数演算子$\exp(it \int \xi_a(x)\mc{T}^A_a(x)d^4x)$は,単にゲージ場に対しゲージパラメータ$\Lambda_\beta=-t\xi_a(x)$のゲージ変換(15.1.17)を行うだけであることを念頭におくと,以下の表式を得る.
\begin{align*}
\Gamma[\xi,A]=&-i\int^1_0 dt \exp\left(it \int \xi_b (x)\mc{T}^A_b(x)d^4x \right)\int \xi_b(y)G_b[y;A]d^4y \\
=&-i\int^1_0 dt \int \xi_b(y) \exp \left(-i \int (-t\xi_b (x))\mc{T}^A_b(x)d^4x \right)G_b[y;A]d^4y \\
=&-i\int^1_0 dt \int \xi_b(y) G_b[y;A_{-t\xi}]d^4y
\end{align*}
つまり,$\xi=0$では$\Gamma[\xi,A]=0$の条件のもとでは(22.7.7)は唯一の形であることがわかる!

\vskip\baselineskip

(22.7.7)は擬スカラーのゴールドストンボゾン8重項の電弱相互作用を調べるのに使うことができるらしい.ゲージ場が実際に無くても,ゴールドストンボゾン自身の相互作用については重要な結論を導くことができるらしい.$A=0$の場合,(22.7.8)は純ゲージ場となる.
\begin{align*}
[A_{-t\xi}(x)]_\mu=&-i\left[\partial_\mu \exp(-it\xi_a(x)X_a)\right]\exp(it\xi_a(x)X_a) \\
=&-i[\partial_\mu V(t\xi(x))]V^{-1}(t\xi(x))
\end{align*}
(一般に$A=-i(\partial_\mu g)g^{-1}$の形のゲージ場を純ゲージ場と呼ぶ.)ここで
\begin{align*}
V(t\xi(x))\equiv \exp(-it \xi_a X_a)
\end{align*}
と定義した.$A_{\alpha\mu}(x)=0$では,アノマリー項はゲージ場の関数であるから消えて
\begin{align*}
\mc{T}_\beta(x)\Gamma[\xi,0]=0
\end{align*}
となる.したがって,(22.7.21)を(22.7.7)に使うと,ゴールドストンボゾン場$\xi_a(x)$の$G$不変局所汎関数を得る.ただし,それは一般には19.6節で構成したような$\xi_a(x)$とその微分の$G$不変な関数の時空座標についての積分ではない.\par
最も単純な例は,対称性が完全に破れている場合だ.この場合は条件(22.7.6)は空で,アノマリーの対称形(22.3.38)を使える.
\begin{align*}
G_a[x;A]=-\frac{i}{24\pi^2}\epsilon^{\kappa\nu\lambda\rho}\mr{Tr}\left\{X_a \left[\partial_\kappa A_\nu \partial_\lambda A_\rho -\frac{1}{2} i\partial_\kappa A_\nu A_\lambda A_\rho +\frac{1}{2}iA_\kappa \partial_\nu A_\lambda A_\rho -\frac{1}{2}iA_\kappa A_\nu \partial_\lambda A_\rho \right]\right\}
\end{align*}
ここで$X_a$は理論の左手フェルミオン(反フェルミオンとの差異が重要なときには反フェルミオンも含む)によって与えられる群の生成子の特定の表現だ.($T_\alpha$ではないのは,仮定(22.7.6)より$T_i$に比例するアノマリーはないとしているからだ.)例えば,フェルミオン数を保存する場合には(22.3.4)で与えられる.このアノマリー形の場合には,(22.7.24)に(22.7.21)を使って,(22.7.24)のトレース内の項で$\epsilon^{\kappa\nu\lambda\rho}$と縮約をとっても生き残る項は,全て以下に比例することが分かる.
\begin{align*}
\mr{Tr}\left\{ X_a (\partial_\kappa V)V^{-1} (\partial_\nu V) V^{-1} (\partial_\lambda V) V^{-1} (\partial_\rho V) V^{-1} \right\}
\end{align*}
なぜなら,例えば(22.7.24)で出てくる微分
\begin{align*}
\partial_\kappa [A_{-t\xi}]_\nu=&-i\partial_\kappa \{(\partial_\nu V)V^{-1}\} \\
=&-i(\partial_\kappa \partial_\nu V)V^{-1}-i(\partial_\nu V)(\partial_\kappa V^{-1}) \\
=&-i(\partial_\kappa \partial_\nu V)V^{-1}+i(\partial_\nu V)V^{-1}(\partial_\kappa V)V^{-1}
\end{align*}
の第一項目は$\epsilon^{\kappa\nu\lambda\rho}$の反対称性によりゼロとなるからだ.よって第二項目のみが残り,(22.7.24)の各項でこの操作をすれば
\begin{align*}
&\epsilon^{\kappa\nu\lambda\rho}\mr{Tr}\left\{ X_a \partial_\kappa [A_{-t\xi}]_\nu \partial_\lambda [A_{-t\xi}]_\rho \right\} \\
=&-\epsilon^{\kappa\nu\lambda\rho}\mr{Tr}\Bigl\{ X_a (\partial_\nu V)V^{-1}(\partial_\kappa V)V^{-1}(\partial_\rho V)V^{-1}(\partial_\lambda V)V^{-1} \Bigr\} \\
=&-\epsilon^{\kappa\nu\lambda\rho}\mr{Tr}\Bigl\{ X_a (\partial_\kappa V)V^{-1}(\partial_\nu V)V^{-1}(\partial_\lambda V)V^{-1}(\partial_\rho V)V^{-1} \Bigr\}
\end{align*}
(22.7.24)第一項目の係数は$-1$となる.
\begin{align*}
&\epsilon^{\kappa\nu\lambda\rho}\mr{Tr}\left\{ X_a \left[-\frac{1}{2}i \partial_\kappa [A_{-t\xi}]_\nu [A_{-t\xi}]_\lambda [A_{-t\xi}]_\rho \right] \right\} \\
=&-\frac{1}{2}\epsilon^{\kappa\nu\lambda\rho}\mr{Tr}\Bigl\{X_a(\partial_\nu V)V^{-1}(\partial_\kappa V)V^{-1}(\partial_\lambda V)V^{-1}(\partial_\rho V)V^{-1}\Bigr\} \\
=&\frac{1}{2}\epsilon^{\kappa\nu\lambda\rho}\mr{Tr}\Bigl\{X_a(\partial_\kappa V)V^{-1}(\partial_\nu V)V^{-1}(\partial_\lambda V)V^{-1}(\partial_\rho V)V^{-1}\Bigr\}
\end{align*}
第二項目は$+1/2$となる.
\begin{align*}
&\epsilon^{\kappa\nu\lambda\rho}\mr{Tr}\left\{ X_a \left[\frac{1}{2}i [A_{-t\xi}]_\kappa \partial_\nu [A_{-t\xi}]_\lambda [A_{-t\xi}]_\rho \right] \right\} \\
=&\frac{1}{2}\epsilon^{\kappa\nu\lambda\rho}\mr{Tr}\Bigl\{X_a(\partial_\kappa V)V^{-1}(\partial_\lambda V)V^{-1}(\partial_\nu V)V^{-1}(\partial_\rho V)V^{-1}\Bigr\} \\
=&-\frac{1}{2}\epsilon^{\kappa\nu\lambda\rho}\mr{Tr}\Bigl\{X_a(\partial_\kappa V)V^{-1}(\partial_\nu V)V^{-1}(\partial_\lambda V)V^{-1}(\partial_\rho V)V^{-1}\Bigr\}
\end{align*}
第三項目は$-1/2$となる.
\begin{align*}
&\epsilon^{\kappa\nu\lambda\rho}\mr{Tr}\left\{ X_a \left[-\frac{1}{2}i [A_{-t\xi}]_\kappa [A_{-t\xi}]_\nu \partial_\lambda [A_{-t\xi}]_\rho \right] \right\} \\
=&-\frac{1}{2}\epsilon^{\kappa\nu\lambda\rho}\mr{Tr}\Bigl\{X_a(\partial_\kappa V)V^{-1}(\partial_\nu V)V^{-1}(\partial_\rho V)V^{-1}(\partial_\lambda V)V^{-1}\Bigr\} \\
=&\frac{1}{2}\epsilon^{\kappa\nu\lambda\rho}\mr{Tr}\Bigl\{X_a(\partial_\kappa V)V^{-1}(\partial_\nu V)V^{-1}(\partial_\lambda V)V^{-1}(\partial_\rho V)V^{-1}\Bigr\}
\end{align*}
第四項目は$+1/2$となる.これらを足し合わせると
\begin{align*}
G_a[x;A]=&-\frac{i}{24\pi^2}\epsilon^{\kappa\nu\lambda\rho}\mr{Tr}\left\{X_a \left[\partial_\kappa A_\nu \partial_\lambda A_\rho -\frac{1}{2} i\partial_\kappa A_\nu A_\lambda A_\rho +\frac{1}{2}iA_\kappa \partial_\nu A_\lambda A_\rho -\frac{1}{2}iA_\kappa A_\nu \partial_\lambda A_\rho \right]\right\} \\
=&\frac{i}{48\pi^2}\epsilon^{\kappa\nu\lambda\rho}\mr{Tr}\left\{X_a (\partial_\kappa V)V^{-1} (\partial_\nu V) V^{-1} (\partial_\lambda V) V^{-1} (\partial_\rho V) V^{-1}\right\}
\end{align*}
となる.これを(22.7.7)に代入すると
\begin{align*}
\Gamma[\xi,0]=&i\int^1_0 dt \int \xi_b(y)G_b[y;A_{-t\xi}]d^4y \\
=&\frac{1}{48\pi^2}\epsilon^{\kappa\nu\lambda\rho} \int d^4y\xi_a (y)\int^1_0 dt \mr{Tr}\biggl\{X_a \Bigl[\partial_\kappa V \Bigl(t\xi(y)\Bigr) \Bigr]V^{-1}\Bigl(t\xi(y)\Bigr) \Bigl[\partial_\nu V\Bigl(t\xi(y)\Bigr)\Bigr] V^{-1}\Bigl(t\xi(y)\Bigr) \\
&\qquad \qquad \times  \Bigl[\partial_\lambda V\Bigl(t\xi(y)\Bigr)\Bigr] V^{-1}\Bigl(t\xi(y)\Bigr) \Bigl[\partial_\rho V\Bigl(t\xi(y)\Bigr)\Bigr] V^{-1}\Bigl(t\xi(y)\Bigr)\biggr\}
\end{align*}
約束したように,これは19.6節で構成したような,場と場の微分の不変関数の4次元積分ではない.たとえば,ゴールドストンボゾン場が弱いとき,これは
\begin{align*}
\Gamma[\xi,0]=&\frac{1}{48\pi^2}\epsilon^{\kappa\nu\lambda\rho} \int d^4y\int^1_0 dt \mr{Tr}\Bigl\{X_a \xi_a(y)(-it\partial_\kappa \xi_b X_b)(-it\partial_\nu \xi_c X_c)(-it \partial_\lambda \xi_d X_d)(-it\partial_\rho \xi_e X_e)\Bigr\} \\
&+O(\xi^6) \\
=&\frac{1}{48\pi^2}\epsilon^{\kappa\nu\lambda\rho} \mr{Tr}\left\{X_a X_b X_c X_d X_e\right\} \int^1_0 t^4 dt \int d^4y \xi_a \partial_\kappa \xi_b \partial_\nu \xi_c \partial_\lambda \xi_d \partial_\rho \xi_e +O(\xi^6) \\
=&\frac{1}{240\pi^2}\epsilon^{\kappa\nu\lambda\rho} \mr{Tr}\left\{X_a X_b X_c X_d X_e\right\} \int d^4y \xi_a \partial_\kappa \xi_b \partial_\nu \xi_c \partial_\lambda \xi_d \partial_\rho \xi_e +O(\xi^6)
\end{align*}
となる.19.6節で構成したように,ゴールドストンボゾン場は共変微分から発生するから,(22.7.26)のような形の最低次の項は持てない.

\vskip\baselineskip

実用上さらに重要な場合として,質量ゼロの$u,d,s$クォークを持つ量子色力学の$SU(3)\times SU(3)$カイラル対称性が,ゲルマン・ネーマンの対角$SU(3)$部分群に自発的に破れる例を考える.22.3節で述べたように,破れていない対称性である対角$SU(3)$部分群のベクトルカレントがアノマリーを持っていないように扱わなければならないため,この場合に対応するアノマリーとしてはバーディーン形(22.3.34)となる.
\begin{align*}
G_a[V,A]=-\frac{in}{16\pi^2}\epsilon^{\mu\nu\rho\sigma}\mr{Tr}\biggl\{t_a\biggl[ V_{\mu\nu}V_{\rho\sigma}+\frac{1}{3}A_{\mu\nu}A_{\rho\sigma}-\frac{32}{3}A_\mu A_\nu A_\rho A_\sigma \\
+\frac{8}{3}i\left( A_\mu A_\nu V_{\rho\sigma}+A_\mu V_{\rho\sigma}A_\nu +V_{\rho\sigma}A_\mu A_\nu \right) \biggr]  \biggr\}
\end{align*}
ここで$t_a$は(19.7.2)で定義されるゲルマン行列$\lambda_a$の半分$t_a=\lambda_a/2$だ.また$V_\mu,A_\mu,V_{\mu\nu},A_{\mu\nu}$は(22.3.35)-(22.3.37)と同様に定義される.そして$n$は各フレーバーのクォークの種類(色)の数だ.ここでは群の生成子を表す行列に小文字の$t$を使っているが,これはこのトレース$\mr{Tr}$では左手クォークについてのみ和をとり,左手反クォークについては和をとっていないからだ.(いわゆる,(22.3.4)の小文字の$t$かな.)場の強度テンソルはゲージ変換に対して$F\to g^{-1}F g$と変換するので,(22.7.21)のような純ゲージ場については自明にゼロになる.なぜなら,ゲージ変換前のゲージ場がゼロなのだから,ゼロから構成された場の強度テンソルはゼロで,ゲージ変換してもこれは変わらないからだ.すると生き残るのは第三項目だけで
\begin{align*}
G_a[V,A]=\frac{2in}{3\pi^2}\epsilon^{\mu\nu\rho\sigma}\mr{Tr}\{t_a [A_{-t\xi}]_\mu [A_{-t\xi}]_\nu [A_{-t\xi}]_\rho [A_{-t\xi}]_\sigma \}
\end{align*}
だ.これを(22.7.7)に使うと
\begin{align*}
\Gamma[\xi,0]=\frac{2n}{3\pi^2}\epsilon^{\mu\nu\rho\sigma}\int^1_0 dt \int d^4x \mr{Tr}\{\xi_a t_a [A_{-t\xi}]_\mu [A_{-t\xi}]_\nu [A_{-t\xi}]_\rho [A_{-t\xi}]_\sigma \}
\end{align*}
ここで注意すべきは,この$A_{-t\xi}$は(22.7.21)と同じではなくて,ベクトル場$V_\mu$と軸性ベクトル場$A_\mu$に分けた$A_\mu$だ.$[A_{-t\xi}]_\mu$が知りたいので,(22.7.21)を
\begin{align*}
[V_{-t\xi}(x)]_\mu +\gamma_5 [A_{-t\xi}(x)]_\mu =-i[\partial_\mu \exp(-it\xi_a \gamma_5 t_a)]\exp(it\xi_a \gamma_5 t_a)
\end{align*}
と書く.$(1+\gamma_5)/2$をかけて$\gamma_5$に比例しない項を集めれば
\begin{align*}
&\frac{1+\gamma_5}{2}[V_{-t\xi}(x)]_\mu+\frac{1+\gamma_5}{2}[A_{-t\xi}(x)]_\mu=-\frac{1+\gamma_5}{2}i[\partial_\mu \exp(-it\xi_a t_a)]\exp(+it\xi_a t_a) \\
\therefore \quad &\frac{1}{2}[V_{-t\xi}(x)]_\mu+\frac{1}{2}[A_{-t\xi}(x)]_\mu=-\frac{1}{2}i[\partial_\mu \exp(-it\xi_a t_a)]\exp(+it\xi_a t_a)
\end{align*}
$(1-\gamma_5)/2$をかけて同様に
\begin{align*}
&\frac{1-\gamma_5}{2}[V_{-t\xi}(x)]_\mu-\frac{1-\gamma_5}{2}[A_{-t\xi}(x)]_\mu=-\frac{1-\gamma_5}{2}i[\partial_\mu \exp(+it\xi_a t_a)]\exp(-it\xi_a t_a) \\
\therefore \quad &\frac{1}{2}[V_{-t\xi}(x)]_\mu-\frac{1}{2}[A_{-t\xi}(x)]_\mu=-\frac{1}{2}i[\partial_\mu \exp(+it\xi_a t_a)]\exp(-it\xi_a t_a)
\end{align*}
前者から後者を引けば,$[V_{-t\xi}(x)]_\mu$を消すことができて
\begin{align*}
[A_{-t\xi}(x)]_\mu=&-\frac{1}{2}i[\partial_\mu \exp(-it\xi_a t_a)]\exp(+it\xi_a t_a)+\frac{1}{2}i[\partial_\mu \exp(+it\xi_a t_a)]\exp(-it\xi_a t_a) \\
=&\frac{1}{2}i\exp(-it\xi_a t_a)[\partial_\mu \exp(+it\xi_a t_a)]+\frac{1}{2}i[\partial_\mu \exp(+it\xi_a t_a)]\exp(-it\xi_a t_a) \\
=&\frac{1}{2}i\exp(it\xi_a t_a)\exp(-2it\xi_a t_a)\biggl\{[\partial_\mu \exp(+it\xi_a t_a)]\exp(it\xi_a t_a)\biggr\}\exp-(it\xi_a t_a) \\
&+\frac{1}{2}i\exp(it\xi_a t_a)\exp(-2it\xi_a t_a)\biggl\{\exp(it\xi_a t_a)[\partial_\mu \exp(+it\xi_a t_a)]\biggr\}\exp(-it\xi_a t_a) \\
=&\frac{1}{2}i\exp(it\xi_a t_a)\exp(-2it\xi_a t_a)[\partial_\mu \exp(2it\xi_a t_a )]\exp(-it\xi_a t_a) \\
=&\frac{1}{2}i\exp(it\xi_a t_a)U^{-1}\Bigl(t\xi(x)\Bigr)\Bigl[\partial_\mu U\Bigl(t\xi(x)\Bigr)\Bigr]\exp(-it\xi_a t_a)
\end{align*}
となる.ここで
\begin{align*}
U\Bigl(t\xi \Bigr) \equiv \exp(2it\xi_a t_a )
\end{align*}
だ.これを(22.7.28)に使うと,
\begin{align*}
\Gamma[\xi,0]=&\frac{n}{24\pi^2}\epsilon^{\mu\nu\rho\sigma}\int^1_0 dt \int d^4x \mr{Tr}\biggl\{\xi_a t_a U^{-1}\Bigl(t\xi(x)\Bigr)\Bigl[\partial_\mu U\Bigl(t\xi(x)\Bigr)\Bigr] U^{-1}\Bigl(t\xi(x)\Bigr)\Bigl[\partial_\nu U\Bigl(t\xi(x)\Bigr)\Bigr] \\
&\times U^{-1}\Bigl(t\xi(x)\Bigr)\Bigl[\partial_\rho U\Bigl(t\xi(x)\Bigr)\Bigr] U^{-1}\Bigl(t\xi(x)\Bigr)\Bigl[\partial_\sigma U\Bigl(t\xi(x)\Bigr)\Bigr] \biggr\}
\end{align*}
となる.これは便利な5次元形にすることができる.$t$を5番目の座標として,$\xi_a(x,t)\equiv t\xi_a(x)$と定義する.すると
\begin{align*}
\Gamma[\xi,0]=&-\frac{in}{240\pi^2}\epsilon^{ijk\ell m}\int d^5z \mr{Tr}\biggl\{ U^{-1}\Bigl(\xi(z)\Bigr)\Bigl[\partial_i U\Bigl(\xi(z)\Bigr)\Bigr] U^{-1}\Bigl(\xi(z)\Bigr)\Bigl[\partial_j U\Bigl(\xi(z)\Bigr)\Bigr]  \\
&\times U^{-1}\Bigl(\xi(z)\Bigr)\Bigl[\partial_k  U\Bigl(\xi(z)\Bigr)\Bigr] U^{-1}\Bigl(\xi(z)\Bigr)\Bigl[\partial_\ell U\Bigl(\xi(z)\Bigr)\Bigr] U^{-1}\Bigl(\xi(z)\Bigr)\Bigl[\partial_m U\Bigl(\xi(z)\Bigr)\Bigr] \biggr\}
\end{align*}
と表せる.ここで$i,j$等は$1,2,3,0,5$の値をとる.$i=1,2,3,0$については$z^i=x^i$で,$z^5=t$だ.また$z^5$積分は$0\leq z^5 \leq 1$の領域についてする.5つの添え字$i,j,k,\ell,m$のどれでも5の値をとる可能性があることを勘定に入れて,余分な因子1/5を入れてある.このように表せる理由は,外微分形式(22.6.14)を使うと見やすい.
\begin{align*}
\Gamma[\xi,0]=&-\frac{in}{240\pi^2}\int \mr{Tr}\biggl\{ U^{-1}\Bigl(\xi(z)\Bigr)\Bigl[dz^i \partial_i U\Bigl(\xi(z)\Bigr)\Bigr] U^{-1}\Bigl(\xi(z)\Bigr)\Bigl[dz^j\partial_j U\Bigl(\xi(z)\Bigr)\Bigr]  \\
&\times U^{-1}\Bigl(\xi(z)\Bigr)\Bigl[dz^k \partial_k  U\Bigl(\xi(z)\Bigr)\Bigr] U^{-1}\Bigl(\xi(z)\Bigr)\Bigl[dz^\ell \partial_\ell U\Bigl(\xi(z)\Bigr)\Bigr] U^{-1}\Bigl(\xi(z)\Bigr)\Bigl[dz^m\partial_m U\Bigl(\xi(z)\Bigr)\Bigr] \biggr\} \\
=&-\frac{in}{48\pi^2}\int \mr{Tr}\biggl\{ U^{-1}\Bigl(t\xi(x)\Bigr)\Bigl[dt \frac{\partial}{\partial t} U\Bigl(t\xi(x)\Bigr)\Bigr] U^{-1}\Bigl(t\xi(x)\Bigr)\Bigl[dx^\mu \partial_\mu U\Bigl(t\xi(x)\Bigr)\Bigr]  \\
&\times U^{-1}\Bigl(t\xi(x)\Bigr)\Bigl[dx^\nu \partial_\nu  U\Bigl(t\xi(x)\Bigr)\Bigr] U^{-1}\Bigl(t\xi(x)\Bigr)\Bigl[dx^\rho \partial_\rho U\Bigl(t\xi(x)\Bigr)\Bigr] U^{-1}\Bigl(t\xi(x)\Bigr)\Bigl[dx^\sigma\partial_\sigma U\Bigl(t\xi(x)\Bigr)\Bigr] \biggr\}
\end{align*}
ここで(22.7.31)
\begin{align*}
U^{-1}\Bigl(t\xi(x)\Bigr)\Bigl[ \frac{\partial}{\partial t} U\Bigl(t\xi(x)\Bigr)\Bigr]=2i\xi_a t_a
\end{align*}
を使えば,元の形式と同じであることが分かる.$z^5=0$においては,他の$z^i$のどの成分$z^\mu$の値によらず$\xi_a(z)=0\times \xi_a(x)=0$は固定値ゼロをとるので,これらの$z^i=$の値を一つの点に同定して,(22.7.33)の積分領域を5次元球体と考えることができる.その4次元境界$z^5=1$は通常の時空だ.(19.8節参照)このようにして(22.7.33)は(19.8.1)と(19.8.3)で与えられるWZW作用の特別な場合となる.これはやはり整数$n$に比例する.\par
唯一の違いは,$n$はここでは色の数と特定されていることだ.19.8節では積分(19.8.3)が5次元球体の時空境界での$\xi_a(z)$の値のみに依存することを見た.したがって(22.7.33)を導く際に(19.8.2)は$\xi_a(x,t)=t\xi_a(x)$のみならず$\xi_a(x)$の5次元球体の内部への任意の接続$\xi_a(x) \to \xi_a(x,t)$の仕方について当てはまる.

\newpage

\part{拡がりのある場の配位}
\setcounter{section}{23}
\setcounter{subsection}{0}
\subsection{トポロジーの有用性}
全ての可能な場の配位の空間は,様々な場のある汎関数$S$が有限だという条件によって自明でないトポロジーが与えられることがしばしば起こるらしい.(この「配位」という言葉は説明がされておらずかなり不明瞭だが,強いて言うなら,写像$\phi:x\to\phi(x)$の全てを集めた空間の元を配位と呼ぶ.)場の古典論では,この$S$はポテンシャルエネルギーだ.有限な摂動でポテンシャルエネルギーが無限大となる配位を生成することはできない.古典統計力学では$S$はハミルトニアンで,(ユークリッド時空で定式化された)場の量子論では$S$はユークリッド作用か,またはそれに比例する.\par
二つの場の配位は,$S$が無限大の禁止された配位を経ずに一方を他方に連続的に変形できるならトポロジー的に同値だという.これは明らかに反射律・対称律・推移律を満たすので同値関係だ.よって全ての場の配位の集合を,それぞれが同じトポロジーの配位からなる同値類に分類できる.\par
\textgt{(a)スカーミオン等}\par
連続な大域的対称性の群$G$が部分群$H$へ自発的に破れるとき,それに伴う実ゴールドストンボゾン場$\pi_a$を考える.19章,特に19.6節で見たように,最低次はゴールドストン場の微分について二次の項となるのだった.したがって次元$d>2$のユークリッド空間でのこれらのゴールドストンボゾンのポテンシャルエネルギーは
\begin{align*}
S[\pi]=\int d^d x\left[\frac{1}{2} \sum_{ab}g_{ab}(\pi)\partial_i \pi_a \partial_i \pi_b +\cdots \right]
\end{align*}
の形をとる.ここで$g_{ab}$は正定値行列で,「$+\cdots$」はゴールドストン場$\pi$の微分について高次の可能な項を表す.あるいは,これは$d$次元ユークリッド時空でのゴールドストンボゾン場の作用(19.6.46)に負符号をつけたものだと見なすこともできる.(なぜ負符号か?ミンコフスキー内積は$(-,+,+,+)$の慣習をとっているから負であるから負符号がある必要があったが,今回のミンコフスキー内積は正であるから,負符号がいらない.)\par
有限の$S$を持つ場の配位は$\partial_i \pi_a(\mathbf{x})$が無限遠$\mathbf{x}\to \infty$で$|\mathbf{x}|^{-d/2}$(ここで$|\mathbf{x}|\equiv \sqrt{x_i x_i}$)より速くゼロに近づかなければならない.なぜなら,$d^dx$が$|\mathbf{x}|^d$の振る舞いをするから,もし$\partial_i \pi_a(\mathbf{x})$がもし$|\mathbf{x}|\to \infty$で$|\mathbf{x}|^{-d/2}$よりも大きい振る舞いをするならば,$\partial_i \pi_a \partial_i \pi_b$は$|\mathbf{x}|^{-d}$より大きい振る舞いをし,積分全体は$\mathbf{x} \to \infty$で発散してしまうことになるからだ.よって,$\pi_a(\mathbf{x})$は$\mathbf{x}\to \infty$で漸近的に定数$\pi_{a\infty}$に近づき,この定数の差は$|\mathbf{x}|^{1-(d/2)}$より速くゼロに近づかなければならない.任意の点のゴールドストンボゾン場$\pi_a$は糖質空間つまり剰余類空間$G/H$を構成し,$G$変換により任意の場の値を別の値に変換できる.したがって大域的$G$変換により漸近的極限$\pi_{a\infty}$が特定の値,例えば$\pi_{a\infty}=0$を持つように調整することが常に可能だ.したがって場$\pi_a(\mathbf{x})$は,球面$r=\infty$を1点と見なした全$d$次元空間から全ての場の値の多様体$G/H$への写像$\pi_{a}:\mathbb{R}^d\cup \{\infty \}\to G/H$を表す.\par
さて,無限遠の$(d-1)$次元球面を1点と見なした$d$次元ユークリッド空間$\mathbb{R}^d\cup \{\infty \}$は,どちらからも連続的に写像できるという意味でトポロジー的には$d$次元球面$S^d$(すなわち,$(d+1)$次元ユークリッド空間に埋め込まれた球面)に同相だ.(1点コンパクト化)\footnote{直感的には,$d=2$の場合を考えると分かりやすい.二次元平面$\mathbb{R}^2$の中心に3次元球面$S^2$を配置し,$S^2$の北極点ともう一つの点を指定すれば,その二点を通る直線と$\mathbb{R}^2$の交点は1対1対応がつく.無限遠点は指定する点を北極点とすれば良いので,$\mathbb{R}^2$上の無限遠$r=\infty$の$S^1$を北極点1点と見なした$\mathbb{R}^2\cup \{\infty\}$は完全に$S^2$と同相となる.}したがって$\mathbf{x}\to\infty$でゼロに近づく場$\pi(\mathbf{x})$は$S^d$から場の変数の多様体$G/H$へのトポロジー的に異なる写像に基づいて分類できる.ここで無限遠点はゼロへと写像される.そのような$S^d$の1点が$\mc{M}$のある固定点(今回はゼロ)に移されるトポロジー的に異なる写像$S^d\to \mc{M}$の分類の集合は$\pi_d(\mc{M})$,すなわち多様体$\mc{M}$の$d$次元ホモトピー群と呼ばれる.これらのホモトピー群(およびそれらの群の構造)は次節で説明する.多様体$\mc{M}$が線形空間$V\simeq \mathbb{R}^n$ならホモトピー群$\pi_d(\mc{M})$は($\mathbf{x}\to \infty$で定数に近づく任意の場の配位$\pi(\mathbf{x})$は,場があらゆる場所でその定数値をとる配位に連続的に変形できるという意味で)自明\footnote{$\mathbb{R}^n$における任意のループが1点に可縮なことを考えれば明らかだ}だが,ゴールドストンボゾン場の多様体$\mc{M}=G/H$はしばしば自明でないホモトピー群を持つ.量子色力学が関係する$SU(2)\otimes SU(2)$が$SU(2)$に破れる場合や$SU(3)\otimes SU(3)$が$SU(3)$に破れる場合には多様体$G/H$はそれぞれ$SU(2)$あるいは$SU(3)$に等しく,そのときのホモトピー$\pi_3(SU(2)),\pi_3(SU(3))$は自明ではない.$d=3$でポテンシャルエネルギーの極小点でのトポロジー的に自明でない場はスカーミオンと呼ばれる.陽子のようなバリオンはある意味で中間子のみの理論でのスカーミオンと見なすことができるらしい.\par
汎関数(23.1.1)は,被積分関数に$\partial_i \pi_a$の高次ベキを含む項が含まれないとスカーミオンの停留点を持たない.デリックの定理により,そのような項が無ければどのようなトポロジー的に自明でない場の配位も$S$の連続値をとり,その連続値は$\pi$
が特異性を持つ下限$S=0$にまで達する.\par
デリックの定理は以下の通りだ.任意の場の配位$\pi_a(\mathbf{x})$について同じトポロジーを持った別の配位
\begin{align*}
\pi_a^R(\mathbf{x})\equiv \pi_a(\mathbf{x}/R)
\end{align*}
を導入することができることに留意する.ここで$R$は任意の実で正のスケール因子だ.すると(23.1.1)で明示した項について
\begin{align*}
S[\pi]&=\int d^dx R^{-d}\left[ \frac{1}{2}\sum_{ab}g_{ab}(\pi)R^{2}\partial_i \pi_a(\mathbf{x}/R)\partial_i \pi_b(\mathbf{x}/R) \right]\\
&=R^{2-d}S[\pi^R] \\
\therefore &\quad S[\pi^R]=R^{d-2}S[\pi]
\end{align*}
が成り立つ.$d>2$の場合,$R\to 0$で$R^{d-2}\to 0$であるから,$S[\pi^R]$の連続値は下限$S=0$まで拡がっている.しかし$S[\pi]>0$である.なぜなら$S[\pi]$は至るところの$\mathbf{x}$で$\pi(\mathbf{x})$が定数の場合にのみゼロだが,これは$\pi$がトポロジー的に自明でないと仮定したので不可能だからだ.($g_{ab}$は正定値だから,当然$S$は非負だ.)したがってこの下限$S=0$は$R=0$でだけ到達できるようになるが,$\pi^R$は非自明な配位にも拘わらず$S$がゼロなのだから特異性を持つようになってしまう.\par
ゴールドストンボゾン場の配位は$S$に高次微分項を含めることで安定化させることができる.例えば$S[\pi]=T[\pi]+D[\pi]$,ただし
\begin{align*}
T[\pi]=&\int d^d x\left[\frac{1}{2} \sum_{ab}g_{ab}(\pi)\partial_i \pi_a \partial_i \pi_b +\cdots \right] \\
D[\pi]=&\int d^dx f_{abcd}(\pi) \left[\partial_i \pi_a \partial_i \pi_b\right] \left[\partial_j \pi_c \partial_j\pi_d\right]\geq 0
\end{align*}
ととれば,前と同じように$T[\pi^R]=R^{d-2}T[\pi]$だが$D[\pi^R]=R^{d-4}D[\pi]$となるので,$2<d<4$ならば$S[\pi^R]=R^{d-2}T[\pi]+R^{d-4}D[\pi]$は有限の$R$で最小値をとる.特に物理的に興味深い$d=3$の場合にはそうだ.

\begin{figure}[H]
  \centering
\begin{tikzpicture}[scale=1.5]
\draw (0,2)[left]node{$S[\pi^R]$};
\draw (4,0)[right]node{$R$};
\draw[->] (-0.2,0) -- (4,0);
\draw[->] (0,-0.2) -- (0,2.5);
\draw [smooth,samples=100,domain=0.3:3] plot({\x},{\x+1/\x-1.5});
\end{tikzpicture}
\end{figure}
スカーミオンの理論の問題点は,$D[\pi]$のような高次微分項を作用に含めなければならない,という点ではない.19章で何度も述べたように,そのような項はどんな有効場理論の作用でも予想される.問題は,他の無限個の高次微分項を排除する理論的根拠がないことだ.上で述べたような,異なる数の微分を持つ項の間によって安定する配位については,そのような無限個の項は一般に同じ程度の大きさになってしまい,その足し合わせは実質的に計算不可能だ.






























\end{document}