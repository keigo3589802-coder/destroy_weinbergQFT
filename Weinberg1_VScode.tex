\documentclass[dvipdfmx]{jsarticle}
\let\headfont=\gtfamily
\usepackage[dvips]{graphicx}
\usepackage{amsmath}
\usepackage{mathrsfs} % 花文字\mathscr{M}, 筆記体\mathcal{M}, 黒板文字\mathbb{M},ドイツ文字\mathfrak{M}
\usepackage{bm} %太文字
\usepackage{amssymb}
\usepackage{latexsym}
\usepackage{braket}
\usepackage{tikz}
\usepackage{tikz-feynhand}
\usepackage{ulem}
\usepackage{tensor}
\usepackage{bigdelim}
\usepackage{multirow}
\usepackage{tcolorbox}
\usepackage{here}
\tcbuselibrary{theorems,skins}
\usetikzlibrary{decorations}
\usepackage{color}
\usepackage{physics}

\usetikzlibrary{intersections, calc, arrows.meta}
 \usetikzlibrary{patterns}

\newfont{\bg}{cmr9 scaled\magstep4}
\newcommand{\bigzerol}{\smash{\lower1.0ex\hbox{\bg 0}}}
\newcommand{\bigzerou}{%
   \smash{\hbox{\bg 0}}}
\newcommand{\mcO}{\mathcal{O}}
\newcommand{\VAC}{\mathrm{VAC}}
\newcommand{\Slash}[1]{{\ooalign{\hfil/\hfil\crcr$#1$}}} %ファインマンのスラッシュ記号
\renewcommand{\mc}{\mathcal}


% \textrm{Roman デフォルト}
% \textgt{Gothic 和文ゴシック体}*専門用語に
% \textbf{Boldface 太字}*専門用語(英語)に
% \textit{Italic 斜体}
% \textsl{Slanted ローマンを傾けただけ}
% \textsf{Sans Serif サンセリフ体}
% \texttt{Typewriter タイプライタ体、等幅}
% \textsc{Small Caps 小文字が大文字に}

\setlength{\textwidth}{\fullwidth}
\setlength{\textheight}{44\baselineskip}
\addtolength{\textheight}{\topskip}
\setlength{\voffset}{-0.6in}

\allowdisplaybreaks[4]

\makeatletter
  \renewcommand{\theequation}
  {\arabic{section}.\arabic{equation}}
  \@addtoreset{equation}{section}
 \makeatother

\title{\vspace{-1cm}\Huge{WeinbergQFT1}}
\author{坂井 啓悟(Sakai Keigo)}
\date{}
\begin{document}


\maketitle

\part*{まえがき}
これはWeinberg著「The Quantum Theory of Field」の日本語訳「ワインバーグ場の量子論」(以下,WeinbergQFTだったり単にWeinbergという)を読んだりゼミをする中で行った計算を全部書き留めておいて,復習にかける時間を最小限にしたり,同期や後輩たちがもしこの本を勉強に使おうと思ったときにドヤ顔で知ったかぶりをするためのノート.というのは本音で,建前としては,よく「Weinbergは初学者にはオススメできない本」と言われることが多いけど自分で読んでみたところ(少なくとも他の本に比べてそんなこと言われるほど)そうは思わなかったので,そのような悪評(?)を覆すつもりで行間を一行たりとも残さないつもりで周りに布教するためのノート.\par
行間埋めの程度としては,この説明なら高校生の頃の自分が理解できたかな?ぐらいの意識でやることにしてる.まあやっぱり実際に高校生が読んで理解できるかは別だと思うけど.行間がある式や難しい文だけピックアップして説明だけ書いていこうとしたけど,写経もかねて(全部自分の言葉で説明するってのは流石にダルいなぁって言い訳だけど)基本的に本文と似た形式で書くことにした.ただ式番号だけはガチでダルいので,このノートでは基本的に式番号は使ってない.このノート中で式番号が出たら,それはWeibergQFT本文中の式番号を参照していると思ってほしい.そういう意味でこのノートは全く自己完結してないので,もし利用するならWeinbergQFTと並べて読んでほしい.行間だけ埋めてても面白くないので,自分の興味ある$+\alpha$くらいの話は書いていこうと思う.もし初学者がこのノートを見るときは(そんなことがあれば嬉しいけど),そういう戯言は無視してほしい.\par
計算や理解が間違ってたり,そもそも行間埋めの式が長ったるすぎたり可読性が終わってり,まぁ色々あると思うから,もし苦情があれば keigo@eken.phys.nagoya-u.ac.jp までお知らせしてくれると助かる.\par
昔はネット上に「ワインバーグ場の量子論を学ぶ人のために」っていう,日本語版1巻と2巻だけだけど,結構たくさん行間を埋めてくれている神みたいなサイトがあったんだよねぇ.前にURLが無効になってたから多分消えちゃったんだけど.

\newpage

\setcounter{part}{1}
\part{相対論的量子力学}
\setcounter{section}{2}
\subsection{量子力学}
i)物理的状態はヒルベルト空間$\mc{H}$内の射線で表される.ヒルベルト空間は一種の複素ベクトル空間だ.つまり,もし$\Phi$と$\Psi$がこの空間$\mc{H}$のベクトル(しばしば「状態ベクトル」と呼ばれる)なら,任意の複素数$\xi,\eta\in \mathbb{C}$を用いて作られる$\xi\Phi+\eta\Psi$もこの空間$\mc{H}$のベクトルだ.\par
また,ヒルベルト空間にはノルム,内積が定義されている(ヒルベルト空間は完備な内積空間で,内積空間はノルム空間の一種だからだ.完備性とは,直感的に言えば「極限をとってもその空間内で必ず収まる」性質である.詳しい定義などは黒田成俊「関数解析」など参照.直接使う場面は特にないので知らなくても良い).つまり,任意のベクトルの対に対して以下を満たす複素数$(\Phi,\Psi)$が定義されている.
\begin{align*}
&(\Phi,\Psi)=(\Psi,\Phi)^* \\
&(\Phi,\xi_1\Psi_1+\xi_2\Psi_2)=\xi_1(\Phi,\Psi_1)+\xi_2(\Phi,\Psi_2) \quad \forall \xi_1,\xi_2 \in \mathbb{C}\\
&(\eta_1 \Phi_1+\eta_2 \Phi_2,\Psi)=\eta^*_1(\Phi_1,\Psi)+\eta^*_2(\Phi_2,\Psi) \quad \forall \eta_1,\eta_2 \in \mathbb{C}
\end{align*}
(最後の式は上二つの公理から導出可能.)ノルム$(\Psi,\Psi)$は正定値条件を満たす.これは一般に$(\Psi,\Psi)\geq0$で,等号条件は$\Psi=0$だ.\par
射線とは,規格化された(つまり$(\Psi,\Psi)=1$)ベクトルの集合であり,$\xi$を$|\xi|=1$を満たす任意の複素数として,$\Psi'=\xi\Psi$が成り立てば,$\Psi$と$\Psi'$は同じ射線に属するという.(つまり同値類である.同値関係$\Psi'\sim \Psi$を関係$\Psi'=\xi \Psi(|\xi|=1)$で定めれば,商空間$\mc{H}/\sim$が定まり,射線$\mc{R}$とは商空間$\mc{H}/\sim$の元となる集合である.$\Psi$の属する射線$\mc{R}$とは同値類(射影空間)$\mc{R}=[\Psi]=\left\{\Psi' |\Psi'=\xi \Psi,|\xi|=1\right\}$である.我々はヒルベルト空間の元であるベクトルそのものを観測するのではなく,系がどのような状態であるかという射線を観測するだけであるから,理論から計算される観測可能量はこの射線の代表元の選び方に依っていてはいけない.)


\vskip\baselineskip


ii)観測量は\uwave{エルミート演算子}で表現される.これらは,ヒルベルト空間から自分自身への写像$\Psi\to A\Psi$であり
\begin{align*}
A(\xi\Psi+\eta\Phi)=\xi A\Psi+\eta A \Phi
\end{align*}
という意味で線形写像だ.(線形写像の定義が$f(a\bm{v}+b\bm{w})=af(\bm{v})+bf(\bm{w})$だったのを思い出す.)そして実条件$A^\dagger=A$を満たす.ここで,任意の線形写像演算子$A$に対してその共役は
\begin{align*}
(\Phi,A^\dagger \Psi)=(A\Phi,\Psi)=(\Psi,A\Phi)^*
\end{align*}
で定義される.また,$A\Psi$が$\Psi$の関数として連続である,という仮定もある.ある射影表現$\mc{R}$で表される状態は,この表現に属するベクトル$\Psi$が$A$の固有値$\alpha$をもつ固有状態
\begin{align*}
A\Psi=\alpha\Psi
\end{align*}
であれば,演算子$A$で表されている観測量は$\alpha$という唯一決まった値をもつ.さて
\begin{align*}
(\Psi,A\Psi)=&(\Psi,\alpha\Psi)=\alpha(\Psi,\Psi) \\
=&(\Psi,A^\dagger \Psi)=(\Psi,A\Psi)^*=\alpha^*(\Psi,\Psi) \quad \because (2.1.5) \\
\therefore \quad \alpha=&\alpha^*
\end{align*}
となってエルミート演算子の固有値は実数だ.さらに
\begin{align*}
A\Phi=\beta\Phi ,\quad A\Psi=\alpha \Psi \quad (\alpha\neq\beta)
\end{align*}
とすると
\begin{align*}
(\Phi,A\Psi)=&\alpha(\Phi,\Psi) \\
=&(\Phi,A^\dagger)=(\Psi,A\Phi)^*=\beta^*(\Psi,\Phi)^*=\beta(\Phi,\Psi) \quad \because\beta は実数 \\
\therefore \quad & (\alpha- \beta)(\Phi,\Psi)=0
\end{align*}
$\alpha\neq\beta$より$(\Phi,\Psi)=0$となる.したがってエルミート演算子に対して異なる固有値をもつ状態同士は直交する.

\vskip\baselineskip

iii)もし系が射線$\mc{R}$で表される状態にあり,この系が互いに直交する射線$\mc{R}_1,\mc{R}_2,\cdots$で表される異なる状態のひとつにあるかを実験で調べたとき,この系を$\mc{R}_n$の状態で見出す確率は
\begin{align*}
P(\mc{R}\to \mc{R}_n)=|(\Psi,\Psi_n)|^2
\end{align*}
で与えられる.(つまり,射線ごとにエネルギーや運動量などのエルミート演算子の固有値が異なったラベル付けがされており,それらの射線に属するベクトル$\Psi_1\in \mc{R}_1,\Psi_2\in \mc{R}_2,\cdots $は上の議論から互いに直交$(\Psi_n,\Psi_m)=0$する.未観測の射線$\mc{R} (\ni \Psi)$から,観測によってエネルギーや運動量やスピンなどが確定されたとき,あるエネルギー$E_n$や運動量$p_n$やスピン$\sigma_n$などの実数量でラベル付けされた一つの射線$\mc{R}_n (\ni \Psi_n)$へ固定される確率が$\mc{P}(\mc{R}\to \mc{R}_n)$である.この確率は$\mc{R},\mc{R}_n$の代表元$\Psi,\Psi_n$の選び方に依らない.実際,射線$\mc{R},\mc{R}_n$に属する別の元$\Psi',\Psi_n'$を選んでも$\Psi'=\alpha \Psi,\Psi_n'=\beta \Psi_n(|\alpha|=|\beta|=1)$であるから
\begin{align*}
|(\Psi',\Psi_n')|^2=|\alpha^*\beta(\Psi,\Psi_n)|^2=|(\Psi,\Psi_n)|^2
\end{align*}
となり同じ確率を与える.)状態ベクトル$\Psi_n$が完全性をなすとすると,
\begin{align*}
\Psi=\sum_n \Psi_n f_n
\end{align*}
と一意的に展開できる.(添え字$n$は離散的かもしれないし連続的かもしれない.連続ならば和は積分となる.)その係数は両辺に$\Psi_m$でスカラー積をとって
\begin{align*}
(\Psi_m,\Psi)=&\sum_n (\Psi_m,\Psi_n)f_n=f_m \quad \because (\Psi_m,\Psi_n)=0,(\Psi_n,\Psi_n)=1\\
\therefore \quad \Psi=&\sum_n \Psi_n(\Psi_n ,\Psi)
\end{align*}
となるから
\begin{align*}
1=(\Psi,\Psi)=\sum_n(\Psi,\Psi_n)(\Psi_n , \Psi)=\sum_n|(\Psi,\Psi_n)|^2=\sum_n P(\mc{R}\to \mc{R}_n)
\end{align*}
となり,全確率は1となることがわかる.

\newpage

\subsection{対称性}
対称性変換とは実験の結果を変えない我々の視点の変化のことだ.観測者$O$が射線$\mc{R}$か$\mc{R}_1$か$\mc{R}_2$か$\cdots$で表される状態にある系を見ると,それと同等な観測者$O'$は同じ系を見たとき,それぞれ射線$\mc{R}'$か$\mc{R}_1'$か$\mc{R}'_2$か$\cdots$で表される異なった状態を観測する.しかし,この二人の観測者は同じ確率
\begin{align*}
P(\mc{R}\to \mc{R}_n)=P(\mc{R}'\to \mc{R}'_n)
\end{align*}
を見出すはずだ.補遺Aで示すウィグナーの定理によると,このような$\mc{R}\to\mc{R}'$の任意の変換\footnote{対称性変換$T$は集合$\mc{R}$を$\mc{R}'$に対応させているが,これは$\mc{R}$の各元から$\mc{R}'$の各元への対応を定めるわけではない.それを与えるのはヒルベルト空間への作用素$U$であり,$T$はどちらかというと商空間$\mc{H}/\sim$の元$\mc{R}$を,商空間$\mc{H}/\sim$の元$\mc{R}'$に対応させる写像を与えているから,$T$の集合論的な正しい表記は$T:\mc{H}/\sim \to \mc{H}/\sim, \mc{R}\mapsto \mc{R}'$である.}に対して,ヒルベルト空間の演算子$U$
\begin{align*}
\begin{array}{ccccc}
T : &\mc{R}                     & \longrightarrow & T\mc{R}&=\mc{R}'                     \\ \\[-4pt]
 &\rotatebox{90}{$\in$} &                 & \rotatebox{90}{$\in$}& \\ \\[-4pt]
U : &\Psi                     & \longmapsto     & U\Psi&
\end{array}
\end{align*}
が定義できて,もし$\Psi$が射線$\mc{R}$に属するならば,$U\Psi$は射線$\mc{R}'$に属し,さらに$U$はユニタリーかつ線型
\begin{align*}
(U\Phi,U\Psi)=(\Phi,\Psi), \quad U(\xi\Phi+\eta\Psi)=\xi U\Phi+\eta U\Psi
\end{align*}
かもしくは反ユニタリーかつ反線型
\begin{align*}
(U\Phi,U\Psi)=(\Phi,\Psi)^*, \quad U(\xi\Phi+\eta\Psi)=\xi^* U\Phi+\eta^* U\Psi
\end{align*}
に\uwave{できる}.\par
線型演算子$L$の共役演算子$L^\dagger$は
\begin{align*}
(\Phi,L^\dagger \Psi) \equiv (L\Phi,\Psi)
\end{align*}
で与えられる.この条件は,反線型演算子には満たすことができない.なぜなら(2.2.6)の右辺は$\Phi$について線型だが,左辺は$\Phi$について反線型だからだ.実際この右辺と左辺を別々に反線型演算子の場合で計算すると
\begin{align*}
\mathrm{RHS}&=(L(\xi_1\Phi_1+\xi_2\Phi_2),\Psi)=(\xi_1^*L\Phi_1+\xi_2^*L\Phi_2,\Psi) \\
&=\xi_1(L\Phi_1,\Psi)+\xi_2(\Phi_2,\Psi) \\
\mathrm{LHS}&=(\xi_1\Phi_1+\xi_2\Phi_2,\Psi)=\xi^*_1(\Phi_1,L^\dagger \Psi)+\xi^*_2(\Phi_2,L^\dagger \Psi)
\end{align*}
となって,これらは等しくない.反線型演算子$A$の共役$A^\dagger$は,上の代わりに
\begin{align*}
(\Phi,A^\dagger\Psi)\equiv (A\Phi,\Psi)^\dagger=(\Psi,A\Phi)
\end{align*}
と定義する.この定義を用いれば,ユニタリー性と反ユニタリー性は共に以下のように書ける.
\begin{align*}
U^\dagger=U^{-1}
\end{align*}
実際,
\begin{align*}
線型演算子:(\Phi,\Psi)&=(\Phi,U^{-1}U\Psi)=(\Phi,U^\dagger U\Psi)=(U\Phi,U\Psi) \\
反線型演算子:(\Phi,\Psi)&=(\Phi,U^{-1}U\Psi)=(\Phi,U^\dagger U\Psi)=(U\Phi,U\Psi)^*
\end{align*}
となって,ユニタリー性(2.2.2)と反ユニタリー性(2.2.4)が再現される.\par
常に恒等演算子$U=1$で表される自明な対称性$\mc{R}\to\mc{R}$が存在する.この演算子は当然ユニタリーで線型だ.連続性から(ローレンツ変換のような)対称性変換が,(角度,距離,速度のような)パラメータを連続的に変えて恒等変換にできるときは,それらの対称性変換は線型でユニタリーだ.\par
特に,恒等変換から微小しか離れていない対称性変換は,1から微小しか離れていない線型ユニタリー演算子で表される.
\begin{align*}
U=1+i\epsilon t
\end{align*}
ここで$\epsilon$は実数の微小量だ.これがユニタリーで線型であるためには,$t$は線型で
\begin{align*}
U^\dagger U=(1-i\epsilon t^\dagger)(1+i\epsilon t)=1+i\epsilon(t-t^\dagger)+O(\epsilon^2)
\end{align*}
が1だからエルミート$t^\dagger=t$でなければならない.したがって,何らかの観測量になりうる.実際,物理に現れる角運動量や運動量など,ほとんどの観測量はこのような対称性変換から発生する.例えば全運動量は空間座標の並進の生成子であり,ハミルトニアンは時間の並進の生成子であり,全角運動量は空間の回転の生成子である.


\vskip\baselineskip


対称性変換の集合は,以下の性質を満たす\par
(1)$T_1$が射線$\mc{R}_n$を$\mc{R}'_n$に変換し,別の変換$T_2$が$\mc{R}'_n$を$\mc{R}''_n$に変換するなら,これらを両方とも行うと$T_1T_2$と書ける別の対称性変換ができて,これは$\mc{R}_n$を$\mc{R}''_n$に変換する.(二項演算で閉じている)
\begin{align*}
T_1 :\mc{R}_n \to \mc{R}_n' , T_2:\mc{R}_n' \to \mc{R}_n'' \quad \Rightarrow \quad T_2 T_1 : \mc{R}_n \to \mc{R}''_n
\end{align*}\par
(2)対称性変換$T$が射線$\mc{R}_n$を$\mc{R}'_n$に変換するとき,そこには$T^{-1}$と書く逆が存在し,$\mc{R}'_n$を$\mc{R}_n$に変換する. (逆元の存在)
\begin{align*}
T:\mc{R}_n \to \mc{R}_n' \quad \Rightarrow T^{-1}: \mc{R}'_n \to \mc{R}_n
\end{align*}\par
(3)恒等変換$T=1$があり,これは射線を変えない.(単位元の存在)
\begin{align*}
T=1: \mc{R}_n \to \mc{R}_n
\end{align*}\par
(4)$T_1:\mc{R}_n\to \mc{R}_n'$と$T_2:\mc{R}_n'\to \mc{R}''_n $の合成をした$T_2T_1:\mc{R}_n \to \mc{R}_n''$と,$T_3:\mc{R}_n''\to \mc{R}_n'''$をさらに合成させた$T_3(T_2T_1):\mc{R}_n\to \mc{R}'''_n$が,$T_1$と,$T_2$と$T_3$を合成した$T_2T_3$との合成$(T_3T_2)T_1:\mc{R}_n\to \mc{R}'''_n$が同じ対称性変換となっている.(結合則)
\begin{align*}
T_3(T_2T_1)=(T_3T_2)T_1
\end{align*}
これは群の公理を満たしているから,したがって対称性変換の集合は群を構成する.\par
これらの対称性変換に相当するユニタリーか反ユニタリーな演算子$U(T)$はこの群構造を反映する性質をもっている.ただ,対称性変換自身とは異なり,$U(T)$は射線ではなく,ヒルベルト空間のベクトル自体に作用する点で,色々な事情が複雑になっている.その代表的な例が以下でみる射影表現である.\par
$T_1$が$\mc{R}_n$を$\mc{R}'_n$に変換するとき,$U(T_1)$は射線$\mc{R}_n$に属するベクトル$\Psi_n$に作用すると,射線$\mc{R}'_n$に属するベクトル$U(T_1)\Psi_n$を与える.同様に,$T_2$が$\mc{R}'_n$を$\mc{R}''_n$に変換するなら,$U(T_1)\Psi_n$に作用すると,ベクトル$U(T_2)U(T_1)\Psi$は射線$\mc{R}_n''$に属する.しかし$U(T_2T_1)\Psi$もまたこの射線に属するから,これらのベクトルは互いに位相$\phi(T_2 ,T_1)$を除いて一致することになる.
\begin{align*}
U(T_2)U(T_1)\Psi_n=e^{i\phi_n(T_2,T_1)}U(T_2T_1)\Psi_n
\end{align*}
さらに,一つの重要な例外を除いて$U(T)$の線型性(または反線型性)から,この位相は状態$\Psi_n$に依存しないことがわかる.これは以下のように証明される.いま互いに比例関係にない二つのベクトル$\Psi_A,\Psi_B$を考える.このとき(2.2.10)を状態$\Psi_{AB}=\Psi_A+\Psi_B$に施すと
\begin{align*}
&U(T_1)U(T_2)\Psi_{AB}=e^{i\phi_{AB}}\Psi_{AB}=e^{i\phi_{AB}}(\Psi_A+\Psi_B) \\
=&U(T_1)U(T_2)(\Psi_A+\Psi_B)=U(T_1)U(T_2)\Psi_A+U(T_1)U(T_2)\Psi_B=e^{i\phi_A}U(T_2T_1)\Psi_A+e^{i\phi_B}U(T_2T_1)\Psi_B \\
\Rightarrow \quad & e^{i\phi_{AB}}U(T_2T_1)(\Psi_A+\Psi_B)=e^{i\phi_A}U(T_2T_1)\Psi_A+e^{i\phi_B}U(T_2T_1)\Psi_B
\end{align*}
を得る.どんなユニタリー(か反ユニタリー)な演算子にも,逆(つまり共役)演算子があり,それもユニタリーか反ユニタリーである.(2.2.11)に左から$U^{-1}(T_2T_1)$を施すと,
\begin{align*}
e^{\pm i\phi_{AB}}(\Psi_A+\Psi_B)=e^{\pm i\phi_A}\Psi_A+e^{\pm i\phi_B}\Psi_B
\end{align*}
上下の符号は,それぞれ$U(T_2T_1)$がユニタリーか反ユニタリーの場合だ.$\Psi_A,\Psi_B$は線型独立なので,
\begin{align*}
&\left[e^{\pm i\phi_{AB}}-e^{\pm i\phi_A}\right]\Psi_A+\left[e^{\pm i\phi_{AB}}-e^{\pm i\phi_B}\right]\Psi_B=0 \\
\Leftrightarrow \quad & e^{i\phi_A}=e^{i\phi_{AB}}=e^{i\phi_B}
\end{align*}
が満たされるときのみ可能だ.したがって,(2.2.10)の位相は状態ベクトル$\Psi_n$に依らないことが示せた.\par
このため,この式は演算子の関係式として書ける.
\begin{align*}
U(T_2)U(T_1)=e^{i\phi(T_2,T_1)}U(T_2T_1)
\end{align*}
$\phi=0$の場合は,$U(T)$は対称性変換の群表現を与える,という.一般的な位相$\phi(T_2,T_1)$の場合,射影表現(位相を除いて表現になっている)である.リー群の構造だけからは,物理的な状態ベクトルが通常の表現を与えるか射影表現を与えるか分からない.しかし,2.7節でみるように,この群が固有の射影表現をもつかどうかはわかる.\par
(2.2.14)を導く議論の例外は,系の状態を$\Psi_A+\Psi_B$で表せられる状態におけないときに発生する.例えば,系を全角運動量が整数の状態と半整数の状態の重ね合わせにおくことはできないと広く信じられている.なぜならばその二つは回転変換に対して異なる変換を受け,したがってもしそのような重ね合わせ(例えば$\ket{\psi}=\frac{1}{\sqrt{2}}(\ket{0,0}+\ket{\frac{1}{2},\frac{1}{2})}$のような状態)が作り出すことができるならば,系の回転$SO(3)$対称性が破れていることになる.(例えば$2\pi$回転すると$\ket{1/2,1/2}$の方だけ符号が変化する.)しかし我々は系の回転に対して真に対称性があると広く信じているから,このような重ね合わせはそもそも作り出せないと考える.このような場合,異なる類の状態の間に「超選択則」があるという.このとき位相$\phi(T_2,T_1)$は,$U(T_2)U(T_1)$と$U(T_2,T_1)$がそのどちらの類の状態に作用するかに依存することもできる.2.7節でこれらのいそうと射影表現についてさらに論じることにする.そこでわかるように,射影表現をもつどんな対称群も(物理的な意味を変更することなしに)その表現が常に$\phi=0$の非射影表現に拡大することができる.2.7節まで,単にこれがすでに実行されていて(2.2.14)で$\phi=0$となっているとする.(つまり,$SO(3,1)$のユニタリー射影表現ではなく,厳密にはその普遍被覆群である$SL(2,\mathbb{C})$の非射影なユニタリー表現を扱っている.)


\vskip\baselineskip

連結リー群と呼ばれる種類の群は,物理で特に重要な役割をする.その変換$T(\theta)$が有限個の実の連続パラメータの集合,例えば$\theta^a$で記述され,どの元も単位元と群の内部の経路で結ばれている群をいう.この場合,群の積は,
\begin{align*}
T(\bar{\theta})T(\theta)=T(f(\bar{\theta},\theta))
\end{align*}
という形をとる.ここで$f^a(\bar{\theta},\theta)$は$\bar{\theta}と \theta$の関数だ.($f$全体が$T$のパラメータなので,実である必要がある.)$\theta^a=0$を単位元の座標とすると
\begin{align*}
&T(f(\bar{\theta},0))=T(\bar{\theta})T(0)=T(\bar{\theta})I=T(\bar{\theta}) \quad \Rightarrow \quad f^a(\bar{\theta},0)=\bar{\theta}^a \\
&T(f(0,\theta))=T(0)T(\theta)=IT(\theta)=T(\theta) \quad \Rightarrow \quad f^a(0,\theta)=\theta^a \\
\Rightarrow \quad & f^a(0,\theta)=f^a(\theta,0)=\theta^a
\end{align*}
でなければならない.\par
このような連続群の変換は物理的ヒルベルト空間上では(反ユニタリーではなく)ユニタリー演算子$U(T(\theta))$で表現される((2.2.9)を参照).リー群ではこれらの演算子は少なくとも単位元の有界な近傍でベキ級数
\begin{align*}
U(T(\theta))=1+i\theta^a t_a +\frac{1}{2}\theta^b\theta^c t_{bc}+\cdots
\end{align*}
で表される.ここで$t_a ,$は$\theta$に依存しないエルミート演算子だ.$t_{bc}=t_{cb}$はエルミートとは\uwave{限らない}.($\theta^a$は可換な実数パラメータだから,$\theta^a \theta^b t_{bc}=\theta^c \theta^b t_{cb}=\theta^b\theta^c t_{cb}$なので,$t_{bc}=t_{cb}$がわかる.エルミートでないことは後の結果(2.2.21)を見れば明白.)$U(T(\theta))$はこの変換群の通常の(射影でない)表現,つまり
\begin{align*}
U(T(\bar{\theta}))U(T(\theta))=U(T(f(\bar{\theta},\theta)))
\end{align*}
であるとする.この条件を$\theta^a$と$\bar{\theta}^a$についてベキ展開するとどうなるか見てみる.(2.2.16)にしたがうと,$f^a(\bar{\theta},\theta)$を二次まで展開すると
\begin{align*}
f^a(\bar{\theta},\theta)=\theta^a + \bar{\theta}^a+f^a_{\ bc}\bar{\theta}^b\theta^c +
\end{align*}
となる.$f^a$は実かつ$\theta^a$は実パラメータであるから,$f^a_{\ bc}$は実係数だ.もし$\theta^2$や$\bar{\theta}^2$の次数の項があると(2.2.16)が成立しない.このとき,(2.2.18)は以下のようになる.
\begin{align*}
\mathrm{LHS}&=\left[ 1+i\bar{\theta}^a t_a +\frac{1}{2}\bar{\theta}^b\bar{\theta}^c t_{bc}+\cdots \right]\left[ 1+i\theta^a t_a +\frac{1}{2}\theta^b\theta^c t_{bc}+\cdots \right] \\
&=1+i(\theta^a +\bar{\theta}^a)t_a+\frac{1}{2}(\theta^b\theta^c+\bar{\theta}^b\bar{\theta}^c)t_{bc}-\bar{\theta}^b\theta^c t_b t_c\cdots \\
\mathrm{RHS}&=1+i(\theta^a+\bar{\theta}^a+f^a_{\ bc}\bar{\theta}^b\theta^c+\cdots )t_a \\
&\qquad +\frac{1}{2}(\theta^b+\bar{\theta}^b+f^b_{\ de}\bar{\theta}^d\theta^e+\cdots)(\theta^c+\bar{\theta}^b+f^c_{\ de}\bar{\theta}^d\theta^e+\cdots)t_{bc}+\cdots \\
&=1+i(\theta^a +\bar{\theta}^a)t_a+\frac{1}{2}(\theta^b\theta^c+\bar{\theta}^b\bar{\theta}^c)t_{bc}+ i f^a_{\ bc} t_a \bar{\theta}^b  \theta^c+ \frac{1}{2}(\theta^b \bar{\theta}^c t_{bc}+ \bar{\theta}^b \theta^c t_{bc})+\cdots \\
&=1+i(\theta^a +\bar{\theta}^a)t_a+\frac{1}{2}(\theta^b\theta^c+\bar{\theta}^b\bar{\theta}^c)t_{bc}+ (i f^a_{\ bc} t_a+t_{bc}) \bar{\theta}^b  \theta^c +\cdots 
\end{align*}
$1,\theta.\bar{\theta},\theta^2,\bar{\theta}^2$の次数の項は自動的に一致するが,$\bar{\theta}\theta$の項からは以下の自明でない条件が得られる.
\begin{align*}
t_{bc}=-t_b t_c -i f^{a}_{\ bc}t_a
\end{align*}
これは,群の構造,つまり$f(\bar{\theta},\theta)$がわかって,その二次の項の係数$f^{a}_{\ bc}$を知れば,$U(T(\theta))$の一次の項に現れる生成子$t_{a}$を用いて二次の項が求められることを示している.
しかし$t_{bc}$は$b,c$について対称であるから,そのための無矛盾条件が存在する.これは(2.2.21)より
\begin{align*}
&t_{bc}=-t_b t_c -i f^{a}_{\ bc}t_a \\
=&t_{cb}=-t_c t_b -i f^{a}_{\ cb}t_a \\
\Rightarrow \quad & [t_b , t_c] \equiv t_b t_c - t_c t_b  \\
&=-if^{a}_{\ bc}t_a + if^{a}_{\ cb}t_a=i(-f^{a}_{\ bc}+f^{a}_{\ cb})t_a\equiv i C^a_{\ bc}t_a
\end{align*}
となる.ここで$C^a_{\ bc}$は構造定数と呼ばれる実定数だ.
\begin{align*}
\tensor{C}{^a_b_c}:=-\tensor{f}{^a_b_c}+\tensor{f}{^a_c_b}
\end{align*}
この交換関係はリー代数と呼ばれる.$t_{bc}$についての展開を(2.2.17)に代入すると
\begin{align*}
U(T(\theta))=1+i\left(\theta^a-\frac{1}{2}\tensor{f}{^a_b_c}\theta^b \theta^c\right) t_a -\frac{1}{2}\theta^b\theta^c t_b t_c+\cdots
\end{align*}
となり,$\theta$の二次までの項は$f^a(\bar{\theta},\theta)$の関数と生成子$t_a$だけで書くことができる.2.7節で,交換関係(2.2.22)がこの過程を続けるために必要な唯一の条件であることを実質上証明する.つまり,$U(T(\theta))$の完全なベキ級数は群の二項演算を決める$f(\bar{\theta},\theta)$と,一次項,つまり生成子$t_a$を知っている限り(2.2.21)のような関係式を無限に続けることで得られる.これは必ずしも$U(T(\theta))$が$t_a$から全ての$\theta^a$に対して一意に決まることは意味しない.しかし$U(T(\theta))$は少なくとも座標$\theta^a=0$の単位元の有界な近傍で$\theta,\bar{\theta}$と$f(\theta,\bar{\theta})$がこの近傍にあるなら,(2.2.15)が満たされる,という意味で一意に決まる.全ての$\theta^a$については2.7節.\par
(ところで,上の$U(T(\theta))$展開の展開を見ると
\begin{align*}
U(T(\theta))=&\exp(i\Omega^a(\theta)t_a) \\
\Omega^a(\theta)=&\theta^a -\frac{1}{2}\tensor{f}{^a_b_c}\theta^b \theta^c+\mc{O}(\theta^3)
\end{align*}
という形で指数形で書けそうな予感がする.展開についての一般項を求めることはできないのだろうか?同じ流れで次の次数の演算子$t_{abc}$についても同様のことをしてみようとしたが,爆発的に項が増えて手に負えなくなったので諦めた.指数写像の数学的な性質を考えると$\theta$の小さいところ(単位元近く)で全単射になるが,遠くへ行くとそうではなさそうだから,あまり意味はなさそうではある.)


\vskip\baselineskip

これから何度も出会う,特別で重要な場合がある.これは関数$f(\theta,\bar{\theta})$が(多分,座標$\theta$の部分集合について)単に加法的,つまり
\begin{align*}
f^a(\theta,\bar{\theta})=\theta^a +\bar{\theta}^a
\end{align*}
となる場合だ.このときには(2.2.19)の$f^{a}_{\ bc}$も構造定数$C^a_{\ bc}$もゼロだ.したがって,生成子は全て可換だ.
\begin{align*}
[t_b,t_c]=0
\end{align*}
このような群は可換群と呼ばれる.この場合は$U(T(\theta))$が全ての$\theta^a$について容易に計算できる.
\begin{align*}
U(T(\theta))&=U\left( T\left(\frac{\theta}{N}+ \frac{\theta}{N}+\cdots + \frac{\theta}{N} \right)\right) \quad (N個の和)\\
&=U\left(T\left( \frac{\theta}{N} \right)\right)\times U\left(T\left( \frac{\theta}{N} \right)\right) \times\cdots \times U\left(T\left( \frac{\theta}{N} \right)\right) \quad (N個の積) \\
&=\left[U\left(T\left( \frac{\theta}{N} \right)\right)\right]^N
\end{align*}
となる.$N\to\infty$として$U(T(\theta/N))$の一次項のみを残しておくと
\begin{align*}
U(T(\theta))=\lim_{N\to\infty}\left[ 1+i\frac{\theta^a}{N}t_a \right]^N=\exp (i\theta^a t_a)
\end{align*}
となる.ここで指数行列は
\begin{align*}
e^A\equiv& \sum_{n=0}^\infty \frac{A^n}{n!} \\
=&I+A+\frac{1}{2!}A^2+ \frac{1}{3!}A^3+\cdots 
\end{align*}
で定義される.上の等式の極限$N\to \infty$で指数行列に等しくなることは本来自明ではない.したがって以下でこれを証明する.\par
$A$を任意の行列(無限次元の場合は有界線形作用素であることを課す)とする.
\begin{align*}
f_n(A)=\left(I+\frac{A}{n}\right)^n
\end{align*}
とおく.示すべきは
\begin{align*}
\forall \epsilon>0 ,\exists N \geq 1 \quad  \mathrm{s.t.} \quad n\geq N \Rightarrow \left\Vert  f_n(A)-\sum_{k=0}^\infty \frac{A^k}{k!}  \right\Vert <\epsilon 
\end{align*}
である.ここで$||\cdot ||$は作用素ノルムであり,有界作用素$A$に対して
\begin{align*}
||A||\equiv \sup_{u\neq 0} \frac{||Au||}{||u||}=\sup_{||u||=1}||Au||
\end{align*}
で定義されて,次の性質がなりたつ.(証明はしない)
\begin{align*}
||A+B||\leq ||A||+||B||,\quad ||AB||\leq ||A||||B||
\end{align*}
二項定理より
\begin{align*}
\left(I+\frac{A}{n}\right)^n=&\sum_{k=0}^n \binom{n}{k}\left(\frac{A}{n}\right)^k \\
=&\sum_{k=0}^n \frac{n!}{k!(n-k)!}\left(\frac{A}{n}\right)^k
\end{align*}
となるから
\begin{align*}
u_{n,k}(A)\equiv \frac{n!}{k!(n-k)!}\left(\frac{A}{n}\right)^k
\end{align*}
とおけば
\begin{align*}
f_n(A)=\sum_{k=0}^n u_{n,k}(A)
\end{align*}
であり,一方$k$を固定したとき
\begin{align*}
u_{n,k}(A)=&\frac{n!}{k!(n-k)!}\left(\frac{A}{n}\right)^k \\
=&\frac{A^k}{n^k \cdot k!}\cdot n(n-1)(n-2)\cdots (n-k+1) \\
=&\frac{A^k}{k!}1\left(1-\frac{1}{n}\right)\left(1-\frac{2}{n}\right)\cdots \left(1-\frac{k-1}{n}\right) \\
=&\frac{A^k}{k!} \prod_{m=1}^{k-1}\left(1-\frac{m}{n}\right) \\
\to& \frac{A^k}{k!} \quad (n\to \infty )
\end{align*}
となる.ここで$u_k(A)\equiv A^k/k!$とおけば,$n\to \infty $で$u_{n,k}(A)\to u_k(A)$である.また$||u_{n,k}(A)||<||u_k(A)||$もわかる.$e^x$のマクローリン展開は収束半径が無限大であるから,級数和
\begin{align*}
\sum_{k=0}^\infty \frac{||A||^k}{k!}=e^{||A||}
\end{align*}
は収束し,Cauchyの収束判定法により,任意の$\epsilon>0$に対して,次を満たす$n_0$をとれる.
\begin{align*}
n\geq n_0 \quad \Rightarrow \quad \sum_{k=n_0+1}^n \frac{||A||^k}{k!}<\frac{\epsilon}{3}
\end{align*}
したがって
\begin{align*}
\left\Vert f_n(A)-\sum_{k=0}^{n_0} u_{n,k} (A) \right\Vert =&\left\Vert \sum_{k=0}^n u_{n,k}(A) - \sum_{k=0}^{n_0} u_{n,k} (A) \right\Vert =\left\Vert \sum_{k=n_0+1}^{n} u_{n,k} (A) \right\Vert \\
\leq & \sum_{k=n_0+1}^{n} \left\Vert u_{n,k} (A) \right\Vert < \sum_{k=n_0+1}^{n} \left\Vert u_k (A)\right\Vert \quad \because ||u_{n,k}(A)||<||u_k(A)|| \\
=&\sum_{k=n_0+1}^{n} \frac{||A^k||}{k!} \leq \sum_{k=n_0+1}^{n} \frac{||A||^k}{k!} <\frac{\epsilon}{3}
\end{align*}
がなりたつ.また$u_{n,k}(A)$は$u_k(A)$に収束するから,任意の$\epsilon'$に対して次を満たす$n_1$がとれる.
\begin{align*}
n \geq n_1 \quad \Rightarrow \quad \left\Vert \sum^{n_0}_{k=0}(u_{n,k}(A)-u_k(A)) \right\Vert <\epsilon'
\end{align*}
$\epsilon'$は任意であるから,$\epsilon'=\epsilon/3$としてかまわない.したがって$N=\max \{n_0,n_1\}$とすれば,$n\geq N$ならば
\begin{align*}
\left\Vert f_n(A) -\sum_{k=0}^\infty \frac{A^k}{k!} \right\Vert=&\left\Vert f_n(A) -\sum_{k=0}^\infty u_k(A) \right\Vert \\
=&\left\Vert f_n(A) -\sum_{k=0}^{n_0} u_{n,k}(A) + \sum^{n_0}_{k=0}u_{n,k}(A)-\sum^{n_0}_{k=0}u_k(A)-\sum^\infty_{k=n_0+1}\frac{A^k}{k!} \right\Vert \\
\leq & \left\Vert f_n(A) -\sum_{k=0}^{n_0} u_{n,k}(A) \right\Vert+\left\Vert \sum^{n_0}_{k=0}u_{n,k}(A) -\sum^{n_0}_{k=0}u_k(A) \right\Vert +\sum_{k=n_0+1}^\infty \frac{||A||^k}{k!} \\
<&\frac{\epsilon}{3}+\frac{\epsilon}{3}+\frac{\epsilon}{3} =\epsilon
\end{align*}
以上より証明が完了した.\par
非有界線形作用素の場合はこの証明を用いることができないが,ハミルトニアンや運動量演算子などの非有界線形作用素についても同様の式が成り立っている.並進演算子や時間発展演算子がいい例だ.これの証明はスペクトル分解定理を用いる.大雑把な説明なので,スペクトル分解定理の各記号の意味や使い方の詳細は黒田「関数解析」など参照.\par
自己共役作用素$A$に対して,Borel測度$E(\lambda)$を用いた分解が存在し
\begin{align*}
A=\int^\infty_{-\infty} \lambda dE(\lambda)
\end{align*}
と書ける(スペクトル分解定理).任意のBorel可測関数$f$に対して$f(A)$を
\begin{align*}
f(A):=\int^\infty_{-\infty} f(\lambda) dE(\lambda)
\end{align*}
と定義できる.したがって$f(\lambda)=\exp(\lambda)$を代入することで
\begin{align*}
\exp(A):=\int^\infty_{-\infty} e^\lambda dE(\lambda)
\end{align*}
が定義できる.任意の複素数$\lambda$について
\begin{align*}
e^\lambda=\lim_{n\to \infty }\left(1+\frac{\lambda}{n}\right)^n
\end{align*}
が成り立つから(これの証明は上の有界作用素の証明を,複素数にしてノルムを絶対値に変更すれば同様に示せる)
\begin{align*}
\exp(A):=&\int^\infty_{-\infty} e^\lambda dE(\lambda) \\
=&\int^\infty_{-\infty} \lim_{n\to \infty}\left(1+\frac{\lambda}{n}\right)^n dE(\lambda) \\
=&\lim_{n\to \infty}\int^\infty_{-\infty} \left(1+\frac{\lambda}{n}\right)^n dE(\lambda) \\
=&\lim_{n\to \infty}\left(1+\frac{A}{n}\right)^n
\end{align*}
が得られる.これで証明ができた.


\newpage

\subsection{量子論的ローレンツ変換}
アインシュタインの(特殊)相対性原理によれば,「慣性」系の間の同等性が成立する.$x^\mu$をある慣性系の座標($x^1,x^2,x^3$をデカルト空間座標,光速度を$c=1$として$x^0=t$を時間座標)とする.\par
他の慣性系では,$x'^\mu$は
\begin{align*}
\eta_{\mu\nu}dx'^\mu dx'^\nu=\eta_{\mu\nu}dx^\mu dx^\nu
\end{align*}
または,それと同等な
\begin{align*}
\eta_{\mu\nu}\frac{\partial x'^\mu}{\partial x^\rho}\frac{\partial x'^\nu}{\partial x^\sigma}=\eta_{\rho\sigma}
\end{align*}
を満たさなければならない($dx'^\mu=\frac{\partial x'^\mu}{\partial x^\rho}dx^\rho$として代入して係数比較すれば得られる).ここで$\eta_{\mu\nu}$はミンコフスキー計量
\begin{align*}
\eta_{\mu\nu}=\left\{
\begin{array}{lll}
-1 \quad & (\mu=\nu=0) \\
+1 & (\mu=\nu=1,2,3) \\
0 & (\mathrm{otherwise})
\end{array}
\right.
\end{align*}
である.\par
これらの変換は,光速がすべての慣性系で同じだという特別の性質をもつ(アインシュタインの光速度不変の原理).\par
速さ$1$の光波は$|d\bm{x}/dt|=1$を満たす.したがって
\begin{align*}
\eta_{\mu\nu}dx^\mu dx^\nu=&d\bm{x}^2-dt^2 \\
=&\left(\left|\frac{d\bm{x}}{dt}\right|^2-1\right)dt^2=0 \\
=\eta_{\mu\nu}dx'^\mu dx'^\nu=&\left(\left|\frac{d\bm{x}'}{dt'}\right|^2-1\right)dt'^2 \\
\Leftrightarrow \left|\frac{d\bm{x}'}{dt'}\right|=&1
\end{align*}
(2.3.2)を満たすどんな座標変換も$x^\mu \to x'^\mu$も線形,つまり,$a^\mu$を任意の定数,$\tensor{\Lambda}{^\mu_\nu}$を定数行列として
\begin{align*}
x'^\mu=\tensor{\Lambda}{^\mu_\nu}x^\nu+a^\mu
\end{align*}
と書ける.(これはアインシュタインの相対性原理により,ある慣性系における等速直線運動が別の慣性系においても等速直線運動に見えるためには$x'$は$x$の一次式で表されていなければならないという要請による.)ここで(2.3.2)より,
\begin{align*}
\eta_{\mu\nu}\frac{\partial x'^\mu}{\partial x^\rho}\frac{\partial x'^\nu}{\partial x^\sigma}=&\eta_{\mu\nu}(\tensor{\Lambda}{^\mu_\alpha}\delta^\alpha_\rho)(\tensor{\Lambda}{^\nu_\beta}\delta^\beta_\sigma)=\eta_{\mu\nu}\tensor{\Lambda}{^\mu_\rho}\tensor{\Lambda}{^\nu_\sigma}=\eta_{\rho\sigma} \\
\therefore \quad \eta_{\mu\nu}\tensor{\Lambda}{^\mu_\rho}\tensor{\Lambda}{^\nu_\sigma}=&\eta_{\rho\sigma} 
\end{align*}
を満たす.

\vskip\baselineskip

行列$\eta_{\mu\nu}$は逆を持つので,それを$\eta^{\mu\nu}$と書く.(成分は$\eta^{\mu\nu}=\eta_{\mu\nu}$である.)(2.3.5)の右から$\eta^{\sigma\tau}\tensor{\Lambda}{^\kappa_\tau}$をかけると
\begin{align*}
(\mathrm{LHS})=&\eta_{\mu\nu}\tensor{\Lambda}{^\mu_\rho}\tensor{\Lambda}{^\nu_\sigma}\eta^{\sigma\tau}\tensor{\Lambda}{^\kappa_\tau}=\eta_{\mu\nu}\tensor{\Lambda}{^\mu_\rho}\left( \tensor{\Lambda}{^\nu_\sigma} \tensor{\Lambda}{^\kappa_\tau} \eta^{\sigma\tau} \right) \\
(\mathrm{RHS})=&\eta_{\rho\sigma}\eta^{\sigma\tau}\tensor{\Lambda}{^\kappa_\tau}=\delta^\tau_\rho \tensor{\Lambda}{^\kappa_\tau}=\tensor{\Lambda}{^\kappa_\rho}=\delta^\kappa_\mu \tensor{\Lambda}{^\mu_\tau}=\eta_{\mu\nu}\eta^{\nu\kappa} \tensor{\Lambda}{^\mu_\rho} \\
\therefore \quad & \eta_{\mu\nu}\tensor{\Lambda}{^\mu_\rho}\left( \tensor{\Lambda}{^\nu_\sigma} \tensor{\Lambda}{^\kappa_\tau} \eta^{\sigma\tau} \right) =\eta_{\mu\nu} \tensor{\Lambda}{^\mu_\rho} (\eta^{\nu\kappa})
\end{align*}
行列$\eta_{\mu\nu} \tensor{\Lambda}{^\mu_\rho}$の逆を左からかけると
\begin{align*}
\tensor{\Lambda}{^\nu_\sigma} \tensor{\Lambda}{^\kappa_\tau} \eta^{\sigma\tau} =\eta^{\nu\kappa}
\end{align*}
となる.\par
これらの変換は群をなす.ローレンツ変換(2.3.4)を最初に行って,第二のローレンツ変換$x'^\mu\to x''^\mu$を次に行うと
\begin{align*}
x''^\mu=&\tensor{\bar{\Lambda}}{^\mu_\rho}x'^\rho +\bar{a}^\mu \\
=&\tensor{\bar{\Lambda}}{^\mu_\rho}(\tensor{\Lambda}{^\rho_\nu}x^\nu+a^\rho) +\bar{a}^\mu \\
=&(\tensor{\bar{\Lambda}}{^\mu_\rho}\tensor{\Lambda}{^\rho_\nu})x^\nu+(\tensor{\bar{\Lambda}}{^\mu_\rho}a^\rho +\bar{a}^\mu)
\end{align*}
これは$x^\mu\to x''^\mu$のローレンツ変換の形になっている.もし$\tensor{\Lambda}{^\mu_\nu}$と$\tensor{\bar{\Lambda}}{^\mu_\nu}$が両方とも(2.3.5)を満たせば,$\tensor{\bar{\Lambda}}{^\mu_\rho}\tensor{\Lambda}{^\rho_\nu}$も(2.3.5)を満たす.実際
\begin{align*}
\eta_{\mu\nu}(\tensor{\bar{\Lambda}}{^\mu_\rho}\tensor{\Lambda}{^\rho_\alpha})(\tensor{\bar{\Lambda}}{^\nu_\sigma}\tensor{\Lambda}{^\sigma_\beta} )=&(\eta_{\mu\nu}\tensor{\bar{\Lambda}}{^\mu_\rho}\tensor{\bar{\Lambda}}{^\nu_\sigma})\tensor{\Lambda}{^\rho_\alpha}\tensor{\Lambda}{^\sigma_\beta} \\
=&\eta_{\rho\sigma}\tensor{\Lambda}{^\rho_\alpha}\tensor{\Lambda}{^\sigma_\beta} \\
=&\eta_{\alpha\beta}
\end{align*}
であるから,(2.3.7)はローレンツ変換になっている.したがって変換$T(\Lambda,a):x^\mu \mapsto x'^\mu$は物理的状態に施されたとき,合成則
\begin{align*}
T(\bar{\Lambda},\bar{a})T(\Lambda,a)=T(\bar{\Lambda}\Lambda,\bar{\Lambda}a+\bar{a})
\end{align*}
を満たす.\par
(2.3.5)は行列表記で$\Lambda^T \eta \Lambda=\eta$と書けるから,この両辺の行列式をとると,($\det AB=(\det A)( \det B)$より)
\begin{align*}
(\det \Lambda) (\det \eta) (\det \Lambda)=&\det \eta \\
\therefore \quad (\det \Lambda)^2 =&1\neq 0 \quad \because \det{\eta}=-1
\end{align*}
よって$\tensor{\Lambda}{^\mu_\nu}$には逆$\tensor{(\Lambda^{-1})}{^\nu_\rho}$があり,(2.3.5)の右からかければ
\begin{align*}
(\mathrm{LHS})=&\eta_{\mu\nu}\tensor{\Lambda}{^\mu_\rho}\tensor{\Lambda}{^\nu_\sigma}\tensor{(\Lambda^{-1})}{^\sigma_\lambda} =\eta_{\mu\nu}\tensor{\Lambda}{^\mu_\rho}\delta^\nu_\lambda=\eta_{\mu\lambda}\tensor{\Lambda}{^\mu_\rho} \\
(\mathrm{RHS})=& \eta_{\rho\sigma}\tensor{(\Lambda^{-1})}{^\sigma_\lambda}
\end{align*}
さらに両辺に$\eta^{\alpha\rho}$を左からかければ
\begin{align*}
\eta^{\alpha\rho}\eta_{\mu\lambda}\tensor{\Lambda}{^\mu_\rho}=&\eta^{\alpha\rho} \eta_{\rho\sigma}\tensor{(\Lambda^{-1})}{^\sigma_\lambda} \\
\therefore \quad  \tensor{\Lambda}{_\lambda^\rho}\equiv &\eta_{\lambda\mu}\tensor{\Lambda}{^\mu_\rho}\eta^{\sigma\rho}=\tensor{(\Lambda^{-1})}{^\sigma_\lambda}
\end{align*}
であるとわかる.変換$T(\Lambda,a)$の逆は,(2.3.8)より
\begin{align*}
\bar{\Lambda}\Lambda=1 \quad \therefore &\, \bar{\Lambda}=\Lambda^{-1} \\
\bar{\Lambda}a+\bar{a}=\Lambda^{-1} a+\bar{a}=0 \quad \therefore &\, \bar{a}=-\Lambda^{-1}a
\end{align*}
よって$[T(\Lambda,a)]^{-1}=T(\Lambda^{-1},-\Lambda^{-1}a)$がわかる.恒等変換は当然$T(1,0)$である.以上により,ローレンツ変換は群の公理を満たしていることが分かった.

\vskip\baselineskip

前節の議論によれば,変換$T(\Lambda,a)$は物理的ヒルベルト空間のベクトルにユニタリー変換を引き起こす.
\begin{align*}
\Psi \mapsto U(T(\Lambda,a))\Psi \equiv U(\Lambda,a)\Psi
\end{align*}
演算子$U$は合成則
\begin{align*}
U(\bar{\Lambda},\bar{a})U(\Lambda,a)=&U(T(\bar{\Lambda},\bar{a}))U(T(\Lambda,a)) \\
=&U(T(\bar{\Lambda}\Lambda,\bar{\Lambda}a+\bar{a})) \\
=& U(\bar{\Lambda}\Lambda,\bar{\Lambda}a+\bar{a})
\end{align*}
を満たす.変換$T(\Lambda,a)$の全体がなす群は,非斉次ローレンツ群,またはポアンカレ群と呼ばれる.(半直積群)\par
$a^\mu=0$の変換は
\begin{align*}
T(\bar{\Lambda},0)T(\Lambda,0)=T(\bar{\Lambda}\Lambda,0)
\end{align*}
を満たすので,明らかに部分群をなす.これは斉次ローレンツ群と呼ばれる.\par
また,(2.3.9)より$\det \Lambda=+1$か$\det \Lambda=-1$である.$\det \Lambda=+1$の変換は明らかに斉次または非斉次ローレンツ群の部分群をなす.($\det \Lambda=-1$の元の集合は,二つの元$\bar{\Lambda},\Lambda$の積が$\det (\bar{\Lambda}\Lambda)=+1$になるので群をなさない.)\par
また,(2.3.5)の$00$成分から
\begin{align*}
-1=&-\tensor{\Lambda}{^0_0}\tensor{\Lambda}{^0_0}+\tensor{\Lambda}{^1_0}\tensor{\Lambda}{^1_0}+\tensor{\Lambda}{^2_0}\tensor{\Lambda}{^2_0}+\tensor{\Lambda}{^3_0}\tensor{\Lambda}{^3_0} \\
(\tensor{\Lambda}{^0_0})^2=&1+\tensor{\Lambda}{^i_0}\tensor{\Lambda}{^i_0}  \quad (\geq 1)
\end{align*}
(2.3.6)から同様にして
\begin{align*}
(\tensor{\Lambda}{^0_0})^2=1+\tensor{\Lambda}{^0_i}\tensor{\Lambda}{^0_i} \quad (\geq 1)
\end{align*}
も得る.よって$\tensor{\Lambda}{^0_0}\geq +1$か$\tensor{\Lambda}{^0_0}\leq -1$であることがわかる.$\tensor{\Lambda}{^0_0}\geq +1$の変換は部分群をなすことを見る.\par
部分群をなすことを示すには,$\tensor{\Lambda}{^\mu_\nu},\tensor{\bar{\Lambda}}{^\mu_\nu}$が$\tensor{\Lambda}{^0_0}\geq 1,\tensor{\bar{\Lambda}}{^0_0}\geq 1$を満たすと仮定したときに$\tensor{(\bar{\Lambda} \Lambda)}{^0_0}\geq 1$がなりたつことをみればよい.行列の積から
\begin{align*}
\tensor{(\bar{\Lambda} \Lambda)}{^0_0}=\tensor{\bar{\Lambda}}{^0_0}\tensor{\Lambda}{^0_0}+\tensor{\bar{\Lambda}}{^0_1}\tensor{\Lambda}{^1_0}+\tensor{\bar{\Lambda}}{^0_2}\tensor{\Lambda}{^2_0}+\tensor{\bar{\Lambda}}{^0_3}\tensor{\Lambda}{^3_0}
\end{align*}
が成り立つ.しかし(2.3.13)より3元ベクトル$(\tensor{\Lambda}{^1_0},\tensor{\Lambda}{^2_0},\tensor{\Lambda}{^3_0})$はノルム$\sqrt{(\tensor{\Lambda}{^0_0})^2-1}$をもち,同様に$(\tensor{\bar{\Lambda}}{^0_1},\tensor{\bar{\Lambda}}{^0_2},\tensor{\bar{\Lambda}}{^0_3})$もノルムが$\sqrt{(\tensor{\bar{\Lambda}}{^0_0})^2-1}$なので,コーシー・シュワルツの不等式より
\begin{align*}
\left| \tensor{\bar{\Lambda}}{^0_1}\tensor{\Lambda}{^1_0}+\tensor{\bar{\Lambda}}{^0_2}\tensor{\Lambda}{^2_0}+\tensor{\bar{\Lambda}}{^0_3}\tensor{\Lambda}{^3_0} \right| \leq \sqrt{(\tensor{\Lambda}{^0_0})^2-1} \sqrt{(\tensor{\bar{\Lambda}}{^0_0})^2-1}
\end{align*}
という制限がかかる.よって
\begin{align*}
\tensor{(\bar{\Lambda} \Lambda)}{^0_0}=&\tensor{\bar{\Lambda}}{^0_0}\tensor{\Lambda}{^0_0}+\tensor{\bar{\Lambda}}{^0_1}\tensor{\Lambda}{^1_0}+\tensor{\bar{\Lambda}}{^0_2}\tensor{\Lambda}{^2_0}+\tensor{\bar{\Lambda}}{^0_3}\tensor{\Lambda}{^3_0} \\
\geq & \tensor{\bar{\Lambda}}{^0_0}\tensor{\Lambda}{^0_0}-\left| \tensor{\bar{\Lambda}}{^0_1}\tensor{\Lambda}{^1_0}+\tensor{\bar{\Lambda}}{^0_2}\tensor{\Lambda}{^2_0}+\tensor{\bar{\Lambda}}{^0_3}\tensor{\Lambda}{^3_0} \right|  \\
 \geq & \tensor{\bar{\Lambda}}{^0_0}\tensor{\Lambda}{^0_0} - \sqrt{(\tensor{\Lambda}{^0_0})^2-1} \sqrt{(\tensor{\bar{\Lambda}}{^0_0})^2-1}
\end{align*}
一行目の等号成立条件は絶対値の中身がマイナスのときだ.最後の式が$1$以上であることを示せればよい.ここで
\begin{align*}
&\left( \tensor{\bar{\Lambda}}{^0_0}\tensor{\Lambda}{^0_0}-1\right)^2 -\left( \sqrt{(\tensor{\Lambda}{^0_0})^2-1} \sqrt{(\tensor{\bar{\Lambda}}{^0_0})^2-1} \right)^2 \\
=& \left( \bar{\tensor{\Lambda}{^0_0}}\tensor{\Lambda}{^0_0} \right)^2-2\bar{\tensor{\Lambda}{^0_0}}\tensor{\Lambda}{^0_0} - \left( \bar{\tensor{\Lambda}{^0_0}}\tensor{\Lambda}{^0_0} \right)^2+\left(\tensor{\Lambda}{^0_0}\right)^2 + \left(\tensor{\bar{\Lambda}}{^0_0}\right)^2-1 \\
=&\left(\tensor{\Lambda}{^0_0}\right)^2 + \left(\tensor{\bar{\Lambda}}{^0_0}\right)^2-2\bar{\tensor{\Lambda}{^0_0}}\tensor{\Lambda}{^0_0} = \left(\tensor{\bar{\Lambda}}{^0_0}-\tensor{\Lambda}{^0_0}\right)^2\geq 0 \\
\therefore \quad &\left( \tensor{\bar{\Lambda}}{^0_0}\tensor{\Lambda}{^0_0}-1\right)^2 \geq  \left( \sqrt{(\tensor{\Lambda}{^0_0})^2-1} \sqrt{(\tensor{\bar{\Lambda}}{^0_0})^2-1} \right)^2
\end{align*}
であることと,$\tensor{\bar{\Lambda}}{^0_0},\tensor{\Lambda}{^0_0} \geq 1$より$\tensor{\bar{\Lambda}}{^0_0}\tensor{\Lambda}{^0_0}-1 \geq 0, (\tensor{\Lambda}{^0_0})^2-1 \geq 0, (\tensor{\bar{\Lambda}}{^0_0})^2-1 \geq0$であることを用いると
\begin{align*}
\tensor{\bar{\Lambda}}{^0_0}\tensor{\Lambda}{^0_0}-1 \geq \sqrt{(\tensor{\Lambda}{^0_0})^2-1} \sqrt{(\tensor{\bar{\Lambda}}{^0_0})^2-1} \\
\tensor{\bar{\Lambda}}{^0_0}\tensor{\Lambda}{^0_0} - \sqrt{(\tensor{\Lambda}{^0_0})^2-1} \sqrt{(\tensor{\bar{\Lambda}}{^0_0})^2-1} \geq 1
\end{align*}
以上より$\tensor{(\bar{\Lambda} \Lambda)}{^0_0} \geq 1$がわかる.これが示したかったことだ.\par
$\det \Lambda=+1$と$\tensor{\Lambda}{^0_0}\geq +1$のローレンツ変換がなす部分群は.固有順時ローレンツ群$SO(3,1)$(あるいは$O^\uparrow_+(3,1)$)と呼ばれる.パラメータを連続的に変化させて,$\det \Lambda =+1$から$\det \Lambda=-1$に飛んだり,$\tensor{\Lambda}{^0_0}\geq +1$から$\tensor{\Lambda}{^0_0}\leq -1$に飛ぶことはできないので,単位元(単位行列)から連続的にパラメータを変えて得られるローレンツ変換は単位元と同じ符号の$\det \Lambda$と$\tensor{\Lambda}{^0_0}$をもつ.したがって,それらは固有順時ローレンツ群$SO(3,1)$に属する.\footnote{数学の文脈では$SO(3,1)$を,$O(3,1)$の部分群で$\det \Lambda=1$を満たす固有ローレンツ変換全体の集合であるとして定義している場合がある.その場合,固有順時ローレンツ変換の集合は$SO_0(3,1),SO^+(3,1),O_+^{\uparrow}(3,1)$などと書くことが多い.毎回「$+$」を書くのが面倒なので,以下では$SO(3,1)$と書いたら固有順時ローレンツ変換の集合であるとする.}\par
ここで一応,連続的に変化させて飛ぶことができないことを形式的に示しておく.$t\in[0,1]$をパラメータとして,$\Lambda(t)\in O(3,1)$を,$O(3,1)$内の2点を結ぶ任意の連続的な経路で,$t$に関して連続関数になっているものとする.
\begin{align*}
\tilde{\Lambda} :[0,1]\to O(3,1) \quad \mathrm{s.t.} \quad \tilde{\Lambda}(0)=\Lambda_1,\tilde{\Lambda}(1)=\Lambda_2 \quad (\Lambda_1,\Lambda_2\in O(3,1))
\end{align*}
ここで,$\Lambda_1,\Lambda_2$を$\det\Lambda_1=+1,\det \Lambda_2=-1$を満たすものだとし,それにもかかわらず,そこに連続的な経路が存在すると仮定しよう.$\det \Lambda$は行列式の定義
\begin{align*}
\det A=\sum_{\sigma_n\in S_n}\mathrm{sgn}(\sigma) A_{\sigma_n(1) 1}A_{\sigma_n(2)2}\cdots A_{\sigma_n(n)n}
\end{align*}
より,行列$\Lambda$の行列要素からなる多項式になっており,したがって行列$\tilde{\Lambda}(t)$の各要素が$t$に関して連続関数になっているならば$\det \tilde{\Lambda}(t)$も$t$に関して連続関数である.さらに任意の$\Lambda\in O(3,1)$は実行列だから$\det \tilde{\Lambda}(t)$は$t$に関して実関数になっている.以上より,中間値の定理から,閉区間$[0,1]$上で定義される連続な実関数$\det \tilde{\Lambda}(t)$は$+1=\det\tilde{\Lambda}(0)$と$-1=\det\tilde{\Lambda}(1)$の間$0<t<1$で$\det \tilde{\Lambda}(t)=0$を少なくとも一度通らなければいけない.しかし任意の$\tensor{\Lambda}{^\mu_\nu}\in O(3,1)$は既に述べた性質から$\det\Lambda=\pm1$であるから,そのような点$\tilde{\Lambda}(t)\in O(3,1)$は存在せず,矛盾となる.したがってそのような連続的な経路は存在しない.同様にして$\tensor{(\Lambda_1)}{^0_0}\leq -1$となる$\Lambda_1$と,$\tensor{(\Lambda_2)}{^0_0}\geq 1$となる$\Lambda_2$を結ぶ連続的な経路$\tilde{\Lambda}(t)$も存在しないことがわかる.これは$t$について連続な$\tilde{\Lambda}(t)\in O(3,1)$が再び存在すると仮定し,$\tensor{(\tilde{\Lambda}(t))}{^0_0}$も$t$について連続な実関数なことと中間値の定理から,再び$0<t<1$で$\tensor{(\tilde{\Lambda}(t))}{^0_0}=0$を満たすものが存在することがわかるが,そのような$\Lambda$は$O(3,1)$に存在しないため矛盾となり,やはり連続経路が存在しないことがわかる.


\vskip\baselineskip


どんなローレンツ変換も固有で順時であるか,または固有順時ローレンツ群の要素と離散的変換$\mc{P},\mc{T}$か$\mc{PT}$の積になっている.ここで$\mc{P}$は空間反転で,$\mc{T}$は時間反転
\begin{align*}
\mc{P}=\left(
\begin{matrix}
1 &      &     & \\
   & -1 &     & \\
   &      & -1 & \\
   &     &      & -1
\end{matrix}
\right) , \quad \mc{T}=\left(
\begin{matrix}
-1 &      &     & \\
   & 1 &     & \\
   &      & 1 & \\
   &     &      & 1
\end{matrix}
\right)
\end{align*}
である.実際,任意の固有だが順時でないローレンツ変換$\Lambda$は$\det \Lambda =+1$で$\tensor{\Lambda}{^0_0}\leq -1$であるが,新しく行列$\Lambda'=\mc{PT} \Lambda$を定義すると,$\det \Lambda'=+1,\tensor{(\Lambda')}{^0_0}\geq +1$を満たすので,この$\Lambda'$は固有順時ローレンツ変換になっている.$\mc{P},\mc{T}$は自明な逆行列を持つので,$\Lambda=\mc{PT}\Lambda'$と書けて,固有だが順時でないローレンツ変換は固有順時ローレンツ群の要素と離散的変換$\mc{P},\mc{T}$の積になっていることが確かめられる.同様に固有でないが順時なローレンツ変換$\Lambda$に対して$\Lambda'=\mc{P} \Lambda$とおけばこれは固有順時ローレンツ変換になっている.固有でも順時でもないローレンツ変換$\Lambda$に対しては$\Lambda'=\mc{T} \Lambda$とおけばこれも固有順時ローレンツ変換になっている.

\vskip\baselineskip


連結成分にも言及しておく.一般に,ある元$g\in G$から連続的な経路で結ぶことができる全ての元を集めてできる集合
\begin{align*}
C_g :=\{g'\in G|\exists f\in C^0:[0,1]\to G \, \mathrm{s.t.} \, f(0)=g ,f(1)=g'\}
\end{align*}
を$g\in G$の連結成分と呼ぶ.単位元$1$の連結成分は$SO(3,1)$である.
\begin{align*}
SO(3,1)=C_1=\{\Lambda \in O(3,1)|\det \Lambda =+1,\tensor{\Lambda}{^0_0}\geq +1\}
\end{align*}
残りの連結成分は,$\mc{P}$の連結成分
\begin{align*}
C_\mc{P}=\{\Lambda \in O(3,1)| \det \Lambda =-1,\tensor{\Lambda}{^0_0}\geq +1 \}
\end{align*}
と,$\mc{T}$の連結成分
\begin{align*}
C_\mc{T}=\{\Lambda \in O(3,1)|\det \Lambda =-1,\tensor{\Lambda}{^0_0}\leq -1\}
\end{align*}
と,$\mc{PT}$の連結成分
\begin{align*}
C_\mc{PT}=\{\Lambda \in O(3,1)|\det \Lambda =+1,\tensor{\Lambda}{^0_0}\leq -1\}
\end{align*}
で構成される.(本当はそれぞれの空間は
\begin{align*}
C'_\mc{P}=&\{\mc{P}\Lambda| \Lambda \in SO(3,1) \} \\
C'_\mc{T}=&\{\mc{T}\Lambda |\Lambda \in SO(3,1)\} \\
C'_\mc{PT}=&\{\mc{PT}\Lambda |\Lambda \in SO(3,1)\}
\end{align*}
と書くべきだが,実は同値であることは,この段落の一番上の議論を使えばすぐに示せる.例えば$C'_{\mc{P}}\subset C_{\mc{P}},C_{\mc{P}}\subset C'_{\mc{P}}$を示して$C_{\mc{P}}=C_{\mc{P}}'$となる.)$SO(3,1)$が連結であることから,これ以上に連結成分は存在しないことはわかる.したがって$O(3,1)$の連結成分は4つである.ただし注意すべきは,これらは所謂「弧状連結」(連続的な経路で任意の二点を結ぶことができる)の意味で連結なのであり,「単連結」(空間内の任意のループが1点可縮)の意味ではない.後に2.7節にて,$SO(3,1)$が単連結ではなく二重連結であることを示す.

\vskip\baselineskip


直感的に$O(3,1)$の連結成分は,$1$と連結な固有順時ローレンツ変換$SO(3,1)$と$\mc{P},\mc{T}$と連結な$C_{\mc{P}},C_{\mc{T}}$,そして$\mc{PT}$と連結な$C_{\mc{PT}}$から構成され,4つである.しかし厳密にはそれ以上存在しないことも証明する必要がある.このためには,$SO(3,1)$が弧状連結であることを示せば十分だ.これを証明するのは若干手間であるからいくつか準備をする.


\vskip\baselineskip


まず,$SO(3)$が弧状連結であることを示したい.これにより任意の$SO(3)$の元が単位元からのパスでつなぐことができる.\par
数学的帰納法を使うのが一番手っ取り早いと思う.$SO(1)=\{1\}$であるから,$SO(1)$は自明に弧状連結である.$SO(n-1)$が弧状連結であることを用いて,$SO(n)$が弧状連結であることを証明できれば完了である.$R \in SO(n)$ならば,$R=\{\bm{v}_1,\bm{v}_2,\cdots ,\bm{v}_n\}$と列ベクトル表示したとき,$R^T R=I$より$\{\bm{v}_k\}$は$\mathbb{R}^n$の正規直交基底をなす.この行列に回転$C\in SO(n)$を作用させて,次のように変形する.
\begin{align*}
CR=\{\bm{e}_1,C\bm{v}_2,\cdots ,C\bm{v}_n\}
\end{align*}
ただしここで$\bm{e}_1$は$(1,0,\cdots,0)^T$である.$C$が回転$SO(n)$の元であることから,$\bm{e}_1,C\bm{v}_1,\cdots ,C\bm{v}_n$も$\mathbb{R}^n$の正規直交基底をなす.(ノルムが1である任意の$\bm{v}\in \mathbb{R}^n$に対して$C\bm{v}=\bm{e}_1$となるような連続的な回転移動$C\in SO(n)$は必ず存在する.これは後で示す.)$C$は連続的な回転移動であるから,$t\in [0,1]$でパラメトライズして
\begin{align*}
C(t) \in SO(n) ,\quad C(0)=I,C(1)=C
\end{align*}
となるようにできる.移動途中の状態$C(t)R$は全て$SO(n)$の元となり,よって$R$と$CR$は$SO(n)$内の弧で連結できる.$CR$が$SO(n)$の点であることから
\begin{align*}
I=CR(CR)^T=\{\bm{e}_1,C\bm{v}_2,\cdots ,C\bm{v}_n\} \left(
\begin{matrix}
1 & \cdots \\
(C\bm{v}_2)_1 & \cdots \\
\vdots & \cdots \\
(C\bm{v}_n)_1 & \cdots \\ 
\end{matrix}
\right)
\end{align*}
という形にできる.$11$成分を見てみると
\begin{align*}
1=1+\sum_{i=2}^n\left[(C\bm{v}_i)_1\right]^2
\end{align*}
となっているから,これが成り立つためには全ての$i$について1成分目はゼロ$(C\bm{v}_i)_1=0$である.したがって
\begin{align*}
CR=\left(
\begin{matrix}
1& 0 \\
0 & R'
\end{matrix}
\right)
\end{align*}
という形となる.$SO(n)$の条件から$R'$は$SO(n-1)$の元であることがすぐにわかる.仮定より$R'$は$SO(n-1)$内の弧で$I$と連結できる.以上より$R\sim CR \sim I$と連結できることが示され,よって任意の$R_1,R_2\in SO(n)$は$I$を介して$SO(n)$内の弧で連結でき,証明が完了する.特に$SO(3)$も弧状連結である.証明完了

\vskip\baselineskip


上の証明にて途中で用いた補題を証明する.すなわち,任意のノルムが1の$\bm{x}\in \mathbb{R}^n$に対して$C\bm{x}=\bm{e}_1=(1,0,\cdots,0)^T$となる$C\in SO(n)$があり,それは単位元と連続的にパスがある.これを証明するためには$n$次元極座標空間を考える必要があるから,まず$n$次元ユークリッド空間から極座標へ移行する手順を解説する.\par
$\bm{x}=(x_1,x_2,\cdots ,x_n)=\sum_{i=1}^n x_i \bm{e}_i$を極座標表示することを目指す.これは次の手順でまとめられる.\par
1.$\bm{x}$を$x_1$軸と$x_2x_3\cdots x_n$超平面($x_1=0$)に射影する.\par
2.$x_1=0$への射影を,$x_2$軸と$x_3x_4\cdots x_n$超平面($x_2=0$)に射影する.\par
3.上と同様の手順を繰り返す.\par
\noindent
実際に行ってみる.手順1は
\begin{align*}
x_1=r_1 \cos\theta_1,r_2=r_1\sin\theta_1 \quad (0\leq r_1 ,0 \leq \theta_1 \leq \pi)
\end{align*}
と書ける.ここで$r_1$は$\bm{r}$の大きさ
\begin{align*}
r_1=\sqrt{x_1^2+x_2^2+\cdots +x_n^2}
\end{align*}
で,$r_2$は$\bm{r}$の$x_2x_3\cdots x_n$超平面への射影の大きさ
\begin{align*}
r_2=\sqrt{x_2^2+x_3^2+\cdots +x_n^2}
\end{align*}
である.この射影に対しても同じことをする.すなわち
\begin{align*}
x_2=r_2 \cos\theta_2 ,r_3=r_2\sin\theta_2 \quad (0\leq r_2 ,0 \leq \theta_2 \leq \pi)
\end{align*}
これを繰り返す.
\begin{align*}
&x_3=r_3 \cos\theta_3 ,r_4=r_3\sin\theta_3 \quad (0\leq r_3 ,0 \leq \theta_3 \leq \pi) \\
&\cdots  \\
&x_k=r_k \cos\theta_k ,r_{k+1}=r_k\sin\theta_k \quad (0\leq r_k ,0 \leq \theta_k \leq \pi  ) \\
&\cdots \\
&x_{n-1}=r_{n-1} \cos\theta_{n-1} ,r_{n}=r_{n-1}\sin\theta_{n-1} \quad (0\leq r_{n-1} ,0\leq \theta_{n-1} \leq 2\pi)
\end{align*}
最後の$r_n$は,$x_n$軸への射影であるから,$x_n$座標そのものをあらわす.そのため$r_n\geq 0$とは限らないから,$\theta_{n-1}$の範囲だけは$0$から$2\pi$である.以上をまとめると
\begin{align*}
\left\{
\begin{aligned}
&x_1=r\cos\theta_1 \\
&x_2=r\sin\theta_1\cos\theta_2 \\
&\vdots \\
&x_k=r\sin\theta_1\cdots \sin\theta_{k-1}\cos\theta_k \\
&\vdots \\
&x_{n-1}=r\sin\theta_1\cdots \sin\theta_{n-2}\cos\theta_{n-1} \\
&x_n=r\sin\theta_1\cdots \sin\theta_{n-2}\sin\theta_{n-1}
\end{aligned}
\right.
\end{align*}
簡潔に書くと
\begin{align*}
x_k=\left\{
\begin{array}{lll}
r\cos\theta_1  & (k=1)\\
r\cos\theta_k\prod^{k-1}_{i=1}\sin\theta_{i} \quad & (k=2,\cdots , n-1) \\
r\prod^{n-1}_{i=1}\sin\theta_i & (k=n)
\end{array}
\right.
\end{align*}
となる.これで極座標に移行することができた.これはHurwitz-Schurの球面座標と呼ばれているらしい.\par
具体的に$\mathbb{R}^3$で球座標を導く.まず
\begin{align*}
z=r\cos\theta ,r'=r\sin\theta \quad (0\leq r ,0\leq \theta \leq \pi)
\end{align*}
となる.ここで$r=\sqrt{x^2+y^2+z^2}$であり$r'=\sqrt{x^2+y^2}$である.次に
\begin{align*}
x=r'\cos\phi ,r''=r'\sin\phi \quad (0\leq \phi \leq 2\pi)
\end{align*}
となる.$r''$は$y$軸への射影であるから,$y$座標そのものである;$r''=y$.そのため,$r''\geq 0$とは限らないので$\phi$は$0$から$2\pi$をとる.以上より
\begin{align*}
z=r\cos\theta,x=r\sin\theta\cos\phi ,y=r\sin\theta\sin\phi \quad (0\leq \theta \leq \pi ,0\leq \phi \leq 2\pi)
\end{align*}
と極座標変換ができた.\par


\vskip\baselineskip

$(\bm{e}_i,\bm{e}_j)$平面だけを角度$\theta$だけ回転させる変換を$r_{ij}(\theta)$と書くと,それが引き起こす座標変換は
\begin{align*}
r_{ij}(\theta):&\bm{x}=\sum_{i}x_i \bm{e}_i \mapsto r_{ij}(\theta)\bm{x}=\sum_i x'_i \bm{e}_i \\
&\left(
\begin{matrix}
x_i \\
x_j
\end{matrix}
\right) \mapsto \left(
\begin{matrix}
x_i' \\
x_j'
\end{matrix}
\right)=\left(
\begin{matrix}
\cos\theta & -\sin\theta \\
\sin\theta & \cos\theta
\end{matrix}
\right),\quad x_k \mapsto x_k'=x_k \quad (k\neq i,j)
\end{align*}
と書ける.これを用いて,ノルムが1の任意のベクトル$\bm{x}\in \mathbb{R}^n$(つまり$S^{n-1}$の任意の点$\bm{x}$)は,回転
\begin{align*}
r_{1,2}(-\theta_1)r_{2,3}(-\theta_2)\cdots r_{n-1,n}(-\theta_{n-1})
\end{align*}
によって北極点$\bm{e}_1=(1,0,\cdots ,0)^T$に送ることができる.ここで$\theta_1,\cdots ,\theta_{n-1}$はHurwitz-Schurの球面座標に現れる角度変数である.証明は簡単.実際,この写像の表現行列は
\begin{align*}
C=&\left(
\begin{matrix}
\cos \theta_1 & \sin\theta_1 &0       & \cdots \\
-\sin\theta_1 & \cos\theta_1 &0        & \cdots \\
0            & 0            & 1       & \cdots \\
\vdots       & \vdots       & \vdots  & \ddots 
\end{matrix}
\right)\left(
\begin{matrix}
1      & 0            & 0            &\cdots \\
0      &\cos \theta_2 & \sin\theta_2 &\cdots \\
0      &-\sin\theta_2  & \cos\theta_2 &\cdots \\
\vdots &\vdots        & \vdots       &\ddots \\
\end{matrix}
\right)\cdots \left(
\begin{matrix}
\ddots  &\vdots &                 & \vdots          \\
\cdots  &1      & 0               & 0                \\
\cdots  &0      &\cos\theta_{n-1} & \sin\theta_{n-1}\\
\cdots  &0      &-\sin\theta_{n-1} & \cos\theta_{n-1}
\end{matrix}
\right) \\
=:&C_{1,2}(\theta_1)C_{2,3}(\theta_2)\cdots C_{n-1,n}(\theta_{n-1}) \\
&\left(
\begin{matrix}
x_1 \\
x_2 \\
\vdots \\
x_n
\end{matrix}
\right)\to \left(
\begin{matrix}
x_1' \\
x_2' \\
\vdots \\
x_n'
\end{matrix}
\right)=C\left(
\begin{matrix}
x_1 \\
x_2 \\
\vdots \\
x_n
\end{matrix}
\right)
\end{align*}
であるから,上のHurwitz-Schurの球面座標を用いて順々に作用させていけば,最初の$C_{n-1,n}(\theta_{n-1})$で$x'_n=0$となり,次に$C_{n-2,n-1}(\theta_{n-2})$によって$x'_{n-1}=0$となり…を繰り返し,最後$x_1'=1$だけが残る.この$C$は部分行列が$SO(2)$となっている行列の積で構成されているから明らかに$\det C=1$であり,さらに$C^T C=1$を満たすから$SO(n)$の元である.この$C$に対して
\begin{align*}
C(t)=C_{1,2}(t\theta_1)C_{2,3}(t\theta_2)\cdots C_{n-1,n}(t\theta_{n-1})
\end{align*}
とすれば,$C(0)=1$かつ$C(1)=C$である連続な経路であり,任意の$t$で$C(t)\in SO(n)$である.これで補題の証明が終わった.


\vskip\baselineskip


以上の準備のもと,$SO(3,1)$が弧状連結であることを示す.任意の$SO(3,1)$の元$\Lambda$はローレンツブースト$B$と回転$R\in SO(3)$で$\Lambda=BR$の形で書けることを示す.任意の$\Lambda\in SO(3,1)$に対して,純粋に時間的なベクトル$k^\mu=(0,0,0,1)$と書くと
\begin{align*}
\bar{\Lambda}=L^{-1}(\Lambda k)\Lambda
\end{align*}
は$k^\mu$を不変に保つ.ここで$L(\Lambda k)$はブースト(2.5.24)で
\begin{align*}
L(\mathbf{v})=\left(
\begin{matrix}
1+(\sqrt{1+|\mathbf{v}|^2}-1)\frac{\mathbf{v} \hat{\mathbf{v}}^T}{|\mathbf{v}|^2} & \mathbf{v} \\
\mathbf{v}^T & \sqrt{1+|\mathbf{v}|^2}
\end{matrix}
\right)
\end{align*}
あり,これは基準速度$k$から$(L(\mathbf{v})k)^\mu=v^\mu$となるように選んだ基準ブーストである.(このように書ける理由は,p95のウィグナー回転にて$p=k$として$L(k)=1$であることを使えばすぐわかる.実際$\Lambda :k\to \Lambda k $としてから$L^{-1}(\Lambda k):\Lambda k\to k$としているからその合成は$k^\mu$を不変とする.さらに$L(\mathbf{v})$が固有順時ローレンツ変換$SO(3,1)$の元であることは,2.5節のノートで$L(p)$が$SO(3,1)$の元であることを示すから,そこで$\mathbf{p},M\to \mathbf{v},1$と置き換えればわかる.そこのノートを見てもらうことにして,今は細かい説明を省く.)時間成分しかないベクトルが不変であることから,この行列$\bar{\Lambda}$は空間成分のみを変換するような行列であり,したがって$\bar{\Lambda}$は空間回転の元であることが推測できる.実際に,以下でそうであることを示そう.$\bar{\Lambda}$は$k^\mu=(0,0,0,1)$を不変にするから,なんらかの3成分ベクトル$\bm{u}$と$3\times 3$行列$R$を用いて一般的に
\begin{align*}
\bar{\Lambda}=\left(
\begin{matrix}
R & 0 \\
\bm{u}^T & 1
\end{matrix}
\right)
\end{align*}
という形で書ける.しかし$L,\Lambda$は$SO(3,1)$の元であるから,$\bar{\Lambda}=L^{-1}(\Lambda k)\Lambda$も$SO(3,1)$の元であり,したがって関係式$\bar{\Lambda}^T\eta\bar{\Lambda}=\eta$を満たす.
\begin{align*}
\bar{\Lambda}^T\eta\bar{\Lambda}=&\left(
\begin{matrix}
R^T & \bm{u} \\
0 & 1
\end{matrix}
\right)\left(
\begin{matrix}
1 & 0\\
0 & -1
\end{matrix}
\right)\left(
\begin{matrix}
R & 0 \\
\bm{u}^T & 1
\end{matrix}
\right) \\
=&\left(
\begin{matrix}
R^T & \bm{u} \\
0 & 1
\end{matrix}
\right)\left(
\begin{matrix}
R & 0\\
-\bm{u}^T & -1
\end{matrix}
\right) \\
=&\left(
\begin{matrix}
R^T R-\bm{u}^T\bm{u} & -\bm{u} \\
-\bm{u}^T & -1
\end{matrix}
\right) \\
=\eta=&\left(
\begin{matrix}
1 & 0\\
0 & -1
\end{matrix}
\right)
\end{align*}
よって$\bm{u}=0,R^T R=1$となる.$\det \bar{\Lambda}=+1$より$\det R=+1$もわかり,したがって$R\in SO(3)$の元となり$\bar{\Lambda}$は純粋に空間回転であることが示された.以上より任意の元$\Lambda \in SO(3,1)$は$L(\Lambda k) \bar{\Lambda}$と書くことができる.任意の$SO(3)$の元は単位元とのパスがあることは既に示したから,$R(0)=1$かつ$R(1)=R$となるような連続な$R(t)\in SO(3)$が存在する.さらに任意のブーストも単位元とのパスがある.実際
\begin{align*}
L(t\mathbf{v})=\left(
\begin{matrix}
1+(\sqrt{1+|t\mathbf{v}|^2}-1)\frac{\hat{\mathbf{v}} \hat{\mathbf{v}}^T}{|\mathbf{v}|^2} & t\mathbf{v} \\
t\mathbf{v}^T & \sqrt{1+|t\mathbf{v}|^2}
\end{matrix}
\right)
\end{align*}
は任意の$t\in [0,1]$に対して$SO(3,1)$の元であることが,$L(\mathbf{v})$が$SO(3,1)$の元であることと全く同様にして示せて,$L(0)=I$であることもすぐわかる.したがって
\begin{align*}
\Lambda(t):=&L(t\Lambda k)\bar{\Lambda}(t) \\
=&\left(
\begin{matrix}
1+(\sqrt{1+|t\mathbf{v}|^2}-1)\frac{\hat{\mathbf{v}} \hat{\mathbf{v}}^T}{|\mathbf{v}|^2} & t\mathbf{v} \\
t\mathbf{v}^T & \sqrt{1+|t\mathbf{v}|^2}
\end{matrix}
\right)\left(
\begin{matrix}
R(t) & 0 \\
0 & 1
\end{matrix}
\right) \quad (\mathbf{v}=(\Lambda k)^i=\tensor{\Lambda}{^i_0})
\end{align*}
と定めれば,$\Lambda(0)=1$かつ$\Lambda(1)=\Lambda$になるようなパスが構成でき,任意の$t$で$\Lambda(t)\in SO(3,1)$である.したがって任意の$\Lambda\in SO(3,1)$も単位元とのパスが存在し,$SO(3,1)$は弧状連結であることが証明できた.\par
($L(p)$のブロック行列の形は$4\times 4$行列であることをダイレクトに使っていないことに留意する.したがってこの議論をD次元ローレンツ変換$SO(D-1,1)$に拡張するのは容易である.$SO(D-1)$の弧状連結性を用いて$SO(D-1,1)$は弧状連結であることが全く同様にして証明できる.)


\newpage

\subsection{ポアンカレ代数}
2.2節でみたように,どんな対称性リー群の情報も,ほぼ単位元の近傍の元の性質に含まれている.非斉次ローレンツ群については,単位元は$\tensor{\Lambda}{^\mu_\nu}=\delta^\mu_\nu,a^\mu=0$の変換であるから,$\tensor{\omega}{^\mu_\nu}$と$\epsilon^\mu$のどちらも微小量として
\begin{align*}
\tensor{\Lambda}{^\mu_\nu}=\delta^\mu_\nu +\tensor{\omega}{^\mu_\nu},\quad a^\mu =\epsilon^\mu
\end{align*}
となるような変換の性質を調べよう.このときローレンツ変換の条件は(2.3.5)は
\begin{align*}
\eta_{\rho\sigma}=&\eta_{\mu\nu}\left(\delta^\mu_\nu +\tensor{\omega}{^\mu_\rho}\right)\left( \delta^\nu_\sigma +\tensor{\omega}{^\nu_\sigma} \right) \\
=&\eta_{\sigma\rho}+\omega_{\sigma\rho}+\omega_{\rho\sigma}+\mc{O}(\omega^2)
\end{align*}
となる.ここで
\begin{align*}
\omega_{\sigma\rho}\equiv \eta_{\mu\sigma}\tensor{\omega}{^\mu_\rho},\quad \tensor{\omega}{^\mu_\rho}=\eta^{\mu\sigma}\omega_{\sigma\rho}
\end{align*}
という記法を使っている.ローレンツ変換の条件(2.3.5)で$\omega$について一次の項のみをとると
\begin{align*}
\omega_{\mu\nu}=-\omega_{\nu\mu}
\end{align*}
四次元の任意の反対称テンソル(2階)は$(4\times 3)/2=6$個の独立な条件を持っているので,$\epsilon^\mu$の四個の成分と合わせて,非斉次ローレンツ変換は$6+4=10$個の独立なパラメータをもつ.$U(1,0)$はどんな射線もそれ自身に変換するので,恒等演算子に比例しなければならない.(比例なのは,$|\alpha|=1$な複素数$\alpha$倍だけする$\Psi \to \Psi'=\alpha\Psi$を考えると,$\Psi,\Psi'$は同じ射線に属するからだ.つまり位相因子分の定数が一応許容される.)これは位相を選ぶことで恒等演算子に等しくできる.こうしておけば,(2.4.1)の微小ローレンツ変換で,$U(1+\omega,\epsilon)$は恒等演算子$1$に$\omega_{\rho\sigma}$と$\epsilon_\rho$について線形な項を加えたものとなる.これを以下のように書く\footnote{$\omega_{\rho\sigma}J^{\rho\sigma}$というのは,前の表記$\theta_a t_a$という形とは違い添え字が二つのように見えるが,実際は(例えば簡単のため添え字$\mu,\nu=0,1$だけとして)生成子が$t_1=J^{00},t_2=J^{01},t_3=J^{10},t_4=J^{11}$で与えられ
\begin{align*}
\theta_a t_a=\theta_1 J^{00}+\theta_2 J^{01}+\theta_3 J^{10}+\theta_4 J^{11}
\end{align*}
という形で生成子の線形結合で展開しており,それぞれの係数$\theta_1,\theta_2,\theta_3,\theta_4$を$\omega_{00},\omega_{01},\omega_{10},\omega_{11}$と改めて書いている.その結果$\theta_a t_a=\omega_{\mu\nu}J^{\mu\nu}$となっている.つまり添え字が二つの場合で展開しているというよりも,全ての生成子の線形結合をとって,その係数$\theta_a$をその生成子の添え字に合わせて$\omega_{\mu\nu}$と書いている.}.
\begin{align*}
U(1+\omega,\epsilon)=1+\frac{1}{2}i\omega_{\rho\sigma}J^{\rho\sigma}-i\epsilon_\rho P^\rho+\cdots 
\end{align*}
ここで$J^{\rho\sigma}$と$P^\rho$は$\omega$と$\epsilon$に依存しない演算子であり,$\cdots$は$\omega,\epsilon$についてより高次の項だ.この$U(1+\omega,\epsilon)$がユニタリーであるためには,(2.2.9)での議論より,演算子$J^{\rho\sigma}$と$P^\rho$はエルミートでなければならない.
\begin{align*}
(J^{\rho\sigma})^\dagger=J^{\rho\sigma},\quad (P^\rho)^\dagger=P^\rho
\end{align*}
$\omega_{\rho\sigma}$は反対称だから,その係数$J^{\rho\sigma}$も反対称としてよい.
\begin{align*}
J^{\rho\sigma}=-J^{\sigma\rho}
\end{align*}
後でみるように,$P^1,P^2,P^3$は運動量演算子の成分で,$J^{23},J^{31},J^{12}$は角運動量の成分,そして$P^0$はエネルギー演算子すなわちハミルトニアンだ.

\vskip\baselineskip

次に$J^{\rho\sigma}$と$P^\rho$のローレンツ変換性を調べる.積
\begin{align*}
U(\Lambda,a)U(1+\omega,\epsilon)U^{-1}(\Lambda,a)
\end{align*}
を考える.ここで$\tensor{\Lambda}{^\mu_\nu}$と$a^\mu$は新しい変換のパラメータであり,$\omega,\epsilon$には関係ない.(2.3.11)によれば,積$U(\Lambda^{-1},-\Lambda^{-1}a)U(\Lambda,a)=U(1,0)$なので,$U(\Lambda^{-1},-\Lambda^{-1}a)$は$U(\Lambda^{-1},-\Lambda^{-1}a)$は$U(\Lambda,a)$の逆だ.したがって
\begin{align*}
U(\Lambda,a)U(1+\omega,\epsilon)U^{-1}(\Lambda,a)=&U(\Lambda,a)U(1+\omega,\epsilon)U(\Lambda^{-1},-\Lambda^{-1}a) \\
=&U(\Lambda,a)U\left((1+\omega)\Lambda^{-1}, -(1+\omega)\Lambda^{-1}a+\epsilon \right) \\
=&U(\Lambda(1+\omega)\Lambda^{-1},\Lambda(-(1+\omega)\Lambda^{-1}a+\epsilon)+a) \\
=&U(\Lambda(1+\omega)\Lambda^{-1},\Lambda\epsilon -\Lambda \omega \Lambda^{-1} a) \\
=&U(1+\Lambda\omega\Lambda^{-1},\Lambda\epsilon -\Lambda \omega \Lambda^{-1} a)
\end{align*}
この右辺は再び$\omega'=\Lambda \omega \Lambda^{-1},\epsilon'=\Lambda \epsilon -\Lambda \omega \Lambda^{-1}a$で$U(1+\omega',\epsilon')$の形になっている.したがって(2.4.3)で展開することができる.したがって
\begin{align*}
(\mathrm{LHS})=&U(\Lambda,a)\left[1+\frac{1}{2}i\omega_{\rho\sigma}J^{\rho\sigma}-i\epsilon_\rho P^\rho +\cdots \right]U^{-1}(\Lambda,a)  \\
=&1+iU(\Lambda,a)\left[\frac{1}{2}\omega_{\rho\sigma}J^{\rho\sigma}-\epsilon_\rho P^\rho \right]U^{-1}(\Lambda,a) +\cdots \\
(\mathrm{RHS})=&1+\frac{1}{2}i(\Lambda\omega \Lambda^{-1})_{\mu\nu}J^{\mu\nu}-i\left(\Lambda \epsilon -\Lambda \omega \Lambda^{-1}a \right)P^\mu +\cdots  \\
\therefore \quad &U(\Lambda,a)\left[\frac{1}{2}\omega_{\rho\sigma}J^{\rho\sigma}-\epsilon_\rho P^\rho \right]U^{-1}(\Lambda,a)=\frac{1}{2}(\Lambda\omega \Lambda^{-1})_{\mu\nu}J^{\mu\nu}-\left(\Lambda \epsilon -\Lambda \omega \Lambda^{-1}a \right)P^\mu
\end{align*}
この式の両辺で$\omega_{\rho\sigma}$と$\epsilon_\rho$の係数を比べると,(2.3.10)も用いて
\begin{align*}
U(\Lambda,a)\frac{1}{2} \omega_{\rho\sigma}J^{\rho\sigma}U^{-1}(\Lambda,a)=&\frac{1}{2} \eta_{\mu\alpha}\tensor{\Lambda}{^\alpha_\beta}\tensor{\omega}{^\beta_\sigma}\tensor{(\Lambda^{-1})}{^\sigma_\nu}J^{\mu\nu}+\eta_{\mu\alpha}\tensor{\Lambda}{^\alpha_\beta}\tensor{\omega}{^\beta_\sigma}\tensor{(\Lambda^{-1})}{^\sigma_\nu}a^\nu P^\mu \\
=&\frac{1}{2}\eta_{\mu\alpha}\tensor{\Lambda}{^\alpha_\beta}\eta^{\beta\rho}\omega_{\rho\sigma}\tensor{\Lambda}{_\nu^\sigma}J^{\mu\nu} +\eta_{\mu\alpha}\tensor{\Lambda}{^\alpha_\beta}\eta^{\beta\rho}\tensor{\omega}{_\rho_\sigma}\tensor{\Lambda}{_\nu^\sigma}a^\nu P^\mu \\
=&\frac{1}{2}\tensor{\Lambda}{_\mu^\rho}\omega_{\rho\sigma}\tensor{\Lambda}{_\nu^\sigma}J^{\mu\nu} +\tensor{\Lambda}{_\mu^\rho}\tensor{\omega}{_\rho_\sigma}\tensor{\Lambda}{_\nu^\sigma}a^\nu P^\mu \\
=&\frac{1}{2}\tensor{\Lambda}{_\mu^\rho}\omega_{\rho\sigma}\tensor{\Lambda}{_\nu^\sigma}J^{\mu\nu} +\frac{1}{2}\tensor{\Lambda}{_\mu^\rho}\tensor{\omega}{_\rho_\sigma}\tensor{\Lambda}{_\nu^\sigma}a^\nu P^\mu-\frac{1}{2}\tensor{\Lambda}{_\mu^\rho}\tensor{\omega}{_\rho_\sigma}\tensor{\Lambda}{_\nu^\sigma}a^\mu P^\nu \\
=&\frac{1}{2}\tensor{\Lambda}{_\mu^\rho}\tensor{\Lambda}{_\nu^\sigma}\left(J^{\mu\nu}-a^\mu P^\nu +a^\nu P^\mu \right)\omega_{\rho\sigma} \\
U(\Lambda,a)[-\epsilon_\rho P^\rho]U^{-1}(\Lambda,a)=&-\eta_{\mu\alpha}\tensor{\Lambda}{^\alpha_\beta}\epsilon^\beta P^\mu \\
=&-\eta_{\mu\alpha}\tensor{\Lambda}{^\alpha_\beta}\eta^{\beta\rho}\epsilon_\rho P^\mu \\
=&-\tensor{\Lambda}{_\mu^\rho}\epsilon_\rho P^\mu
\end{align*}
よって
\begin{align*}
U(\Lambda,a)J^{\rho\sigma} U^{-1} (\Lambda,a)=&\tensor{\Lambda}{_\mu^\rho}\tensor{\Lambda}{_\nu^\sigma}\left(J^{\mu\nu}-a^\mu P^\nu +a^\nu P^\mu \right) \\
U(\Lambda,a)P^\rho U^{-1}(\Lambda,a)=&\tensor{\Lambda}{_\mu^\rho}P^\mu
\end{align*}
を得る.斉次ローレンツ変換($a^\mu=0$)では,これらの変換則は単に$J^{\mu\nu}$がテンソルであり$P^\mu$がベクトルであることを意味する.
\begin{align*}
U(\Lambda,0)J^{\rho\sigma}U^{-1}(\Lambda,0)=&\tensor{\Lambda}{_\mu^\rho}\tensor{\Lambda}{_\nu^\sigma} J^{\mu\nu} \\
U(\Lambda,0)P^\rho U^{-1}(\Lambda,0)=&\tensor{\Lambda}{_\mu^\rho}P^\mu
\end{align*}
純粋な並進$\tensor{\Lambda}{^\mu_\nu}=\delta^\mu_\nu$では,これらから$P^\mu$が並進不変であることを意味するが$J^{\mu\nu}$はそうではないことがわかる.
\begin{align*}
U(1,a)J^{\rho\sigma} U^{-1} (1,a)=&J^{\rho\sigma}-a^\rho P^\sigma +a^\sigma P^\rho \\
U(1,a)P^\rho U^{-1}(1,a)=&P^\rho
\end{align*}
特に,$J^{\rho\sigma}$の空間-空間成分の空間並進のもとでの変化は,角運動量がそれらを計算する原点の変化のもとでの変化になっている.(例えば,$z$軸周りの成分を考えると$J^{12}$について$a^1=\delta x,a^2=\delta y ,a^3=\delta z$として
\begin{align*}
U(1,a)J^{12} U^{-1} (1,a)=J^{12}-\delta x P^y +\delta y P^x =J^{12} -( \delta \bm{x} \times \mathbf{P})_z
\end{align*}
となっている.もっと一般的に書けば,$\mathbf{J}=\{ J^{23} ,J^{31} ,J^{12} \}$として
\begin{align*}
U(1,a) \mathbf{J} U^{-1} (1,a)=\mathbf{J}-\delta \bm{x} \times \mathbf{P}
\end{align*}
となる.これはまさに,角運動量ベクトル$\mathbf{r}\times \mathbf{p}$を平行移動して$(\mathbf{r}-\delta\mathbf{x})\times \mathbf{p}$となったときの変化に等しい.)

\vskip\baselineskip

次に,(2.4.8)(2.4.9)をそれ自身微小,つまり$\tensor{\Lambda}{^\mu_\nu}=\delta^\mu_\nu +\tensor{\omega}{^\mu_\nu},a^\mu=\epsilon^\mu$である場合に適用してみよう.ここでの微小量$\tensor{\omega}{^\mu_\nu},\epsilon^\mu$は前に用いた$\omega,\epsilon$とは無関係とする.(2.4.8)より
\begin{align*}
(\mathrm{LHS})=&\left[1+\frac{1}{2}i\omega_{\rho\sigma}J^{\rho\sigma}-i\epsilon_\rho P^\rho +\cdots \right]J^{\rho\sigma}\left[1+\frac{1}{2}i\omega_{\rho\sigma}J^{\rho\sigma}-i\epsilon_\rho P^\rho +\cdots \right] \\
=&J^{\rho\sigma}+\left[\frac{1}{2}i\omega_{\rho\sigma}J^{\rho\sigma}-i\epsilon_\rho P^\rho \right]J^{\rho\sigma}-J^{\rho\sigma} \left[\frac{1}{2}i\omega_{\rho\sigma}J^{\rho\sigma}-i\epsilon_\rho P^\rho  \right] +\cdots \\
=&J^{\rho\sigma}+i\left[\frac{1}{2}\omega_{\mu\nu}J^{\mu\nu}-\epsilon_\mu P^\mu ,J^{\rho\sigma}\right] \\
(\mathrm{RHS})=&(\delta^\rho_\mu +\tensor{\omega}{_\mu^\rho})(\delta^\sigma_\nu +\tensor{\omega}{_\nu^\sigma})(J^{\mu\nu}-\epsilon^\mu P^\nu +\epsilon^\nu P^\mu) \\
=&J^{\rho\sigma}+\tensor{\omega}{_\mu^\rho}J^{\mu\sigma}+\tensor{\omega}{_\nu^\sigma}J^{\rho\nu}-\epsilon^\rho P^\sigma +\epsilon^\sigma P^\rho \\
\therefore \quad &i\left[\frac{1}{2}\omega_{\mu\nu}J^{\mu\nu}-\epsilon_\mu P^\mu ,J^{\rho\sigma}\right] =\tensor{\omega}{_\mu^\rho}J^{\mu\sigma}+\tensor{\omega}{_\nu^\sigma}J^{\rho\nu}-\epsilon^\rho P^\sigma +\epsilon^\sigma P^\rho
\end{align*}
同様に(2.4.9)より
\begin{align*}
(\mathrm{LHS})=&\left[1+\frac{1}{2}i\omega_{\rho\sigma}J^{\rho\sigma}-i\epsilon_\rho P^\rho +\cdots \right]P^\rho\left[1+\frac{1}{2}i\omega_{\rho\sigma}J^{\rho\sigma}-i\epsilon_\rho P^\rho +\cdots \right] \\
=&P^\rho +i\left[\frac{1}{2}\omega_{\mu\nu}J^{\mu\nu}-\epsilon_\mu P^\mu ,P^\rho \right] \\
(\mathrm{RHS})=& (\delta^\rho_\mu+\tensor{\omega}{_\mu^\rho})P^\mu=P^\rho +\tensor{\omega}{_\mu^\rho}P^\mu \\
\therefore\quad & i\left[\frac{1}{2}\omega_{\mu\nu}J^{\mu\nu}-\epsilon_\mu P^\mu ,P^\rho \right]=\tensor{\omega}{_\mu^\rho}P^\mu
\end{align*}
これらの式の両辺の$\omega_{\mu\nu}$と$\epsilon_\mu$の係数を比べると(2.4.10)より
\begin{align*}
(\mathrm{LHS})=&i\frac{1}{2}\omega_{\mu\nu}[J^{\mu\nu},J^{\rho\sigma}]-i\epsilon_\mu[P^\mu ,J^{\rho\sigma}] \\
(\mathrm{RHS})=&\omega_{\mu\nu}\eta^{\nu\rho}J^{\mu\sigma}+\omega_{\nu\mu}\eta^{\mu\sigma}J^{\rho\nu}-\epsilon_\mu \eta^{\mu\rho} P^\sigma +\epsilon_\mu \eta^{\mu\sigma}P^\rho \\
=&\frac{1}{2}\omega_{\mu\nu}\eta^{\nu\rho}J^{\mu\sigma}+\frac{1}{2}\omega_{\mu\nu}\eta^{\nu\rho}J^{\mu\sigma}+\frac{1}{2}\omega_{\nu\mu}\eta^{\mu\sigma}J^{\rho\nu}+\frac{1}{2}\omega_{\nu\mu}\eta^{\mu\sigma}J^{\rho\nu} \\
&-\epsilon_\mu (\eta^{\mu\rho}P^\sigma-\eta^{\mu\sigma}P^\rho) \\
=&\frac{1}{2}\omega_{\mu\nu}\eta^{\nu\rho}J^{\mu\sigma}+\frac{1}{2}\omega_{\nu\mu}\eta^{\mu\rho}J^{\nu\sigma}+\frac{1}{2}\omega_{\nu\mu}\eta^{\mu\sigma}J^{\rho\nu}+\frac{1}{2}\omega_{\mu\nu}\eta^{\nu\sigma}J^{\rho\mu} \\
&-\epsilon_\mu (\eta^{\mu\rho}P^\sigma-\eta^{\mu\sigma}P^\rho) \\
=&\frac{1}{2}\omega_{\mu\nu}\eta^{\nu\rho}J^{\mu\sigma}-\frac{1}{2}\omega_{\mu\nu}\eta^{\mu\rho}J^{\nu\sigma}-\frac{1}{2}\omega_{\mu\nu}\eta^{\mu\sigma}J^{\rho\nu}+\frac{1}{2}\omega_{\mu\nu}\eta^{\nu\sigma}J^{\rho\mu} \\
&-\epsilon_\mu (\eta^{\mu\rho}P^\sigma-\eta^{\mu\sigma}P^\rho) \\
\therefore \quad & i[J^{\mu\nu},J^{\rho\sigma}]=\eta^{\nu\rho}J^{\mu\sigma}-\eta^{\mu\rho}J^{\nu\sigma}-\eta^{\mu\sigma}J^{\rho\nu}+\eta^{\nu\sigma}J^{\rho\mu} \\
& i[P^\mu,J^{\rho\sigma}]=\eta^{\mu\rho}P^\sigma -\eta^{\mu\sigma}P^\rho
\end{align*}
(2.4.11)では右辺に$\epsilon$に比例する項が存在しないので
\begin{align*}
[P^\mu,P^\rho]=0
\end{align*}
がわかる.これがポアンカレ群のリー代数だ.(交換子の左辺が代数の線形結合になっている.)

\vskip\baselineskip

量子力学では保存される演算子,つまりエネルギー演算子$H=P^0$と可換な演算子が特別な役割をする.(ハイゼンベルグ方程式より,時間に依らないこととハミルトニアンに対し可換なことは同値.)(2.4.13)(2.4.14)より
\begin{align*}
[P^i,P^0]=&0 \quad (当然 [P^0,P^0]=0 ) \\
i[P^0,J^{\mu\nu}]=&\eta^{0\rho} P^\rho -\eta^{0\sigma}P^\rho \quad(\mu,\nu がともに1,2,3ならば右辺はゼロ)
\end{align*}
なので三元運動量
\begin{align*}
\mathbf{P}=\{P^1 ,P^2, P^3 \}
\end{align*}
と三元角運動量
\begin{align*}
\mathbf{J}=&\{J_{23},J_{31},J_{12} \} \\
J_i=&\frac{1}{2}\epsilon_{ijk}J_{jk}=\frac{1}{2}\epsilon_{ijk}J^{jk},\quad J^{ij}=\epsilon_{ijk}J_k
\end{align*}
と,もちろんエネルギー$P^0$それ自身が保存することがわかる.残りの生成子はいわゆる三元「ブースト」演算子だ.\footnote{本文通りいくと$K_i=J^{i0}$だが,25.2節の脚注で述べられている通りこれは誤記らしい.なので後からこのノートを訂正している.ふざけんな.}
\begin{align*}
\mathbf{K}=&\{K_1,K_2,K_3\}=\{J_{10},J_{20},J_{30}\} \\
K_i=&J_{i0}=-J^{i0}
\end{align*}
これらは保存しない.このため$\mathbf{K}$はエルミート演算子だが,固有値は物理的状態を表すためには使わない.\par
上の三次元記法では交換関係(2.4.12)(2.4.13)(2.4.14)がどうなるかを見る.(2.4.23)は明らか
\begin{align*}
[J^i,H]=[P_i,H]=[H,H]=0
\end{align*}
(2.4.24)は
\begin{align*}
i[P^0,J^{i0}]=&\eta^{0i}P^0-\eta^{00}P^i =P^i \\
\therefore \quad [K_i ,H]=&-iP_i
\end{align*}
(2.4.22)は
\begin{align*}
i[P^i,J^{j0}]=&\eta^{ij}P^0 -\eta^{i 0}P^j = \delta^{ij}H \\
\therefore \quad [K_i ,P_j]=&-iH\delta_{ij}
\end{align*}
(2.4.21)は,レヴィチヴィタ記号をかけて
\begin{align*}
\frac{1}{2}\epsilon_{jkl}i[P^i,J^{kl}]=&\frac{1}{2}\epsilon_{jkl} \left(\eta^{ik}P^l -\eta^{il}P^k \right) \\
=&\frac{1}{2}\epsilon_{jkl} \left(\delta^{ik}P^l -\delta^{il}P^k \right) \\
=&\frac{1}{2} \epsilon_{jil}P_l -\frac{1}{2} \epsilon_{jki}P_k \\
=&-\epsilon_{ijl}P_l \\
\therefore \quad  [J_i,P_j]=&i\epsilon_{ijk}P_k
\end{align*}
(2.4.18)は\footnote{他の場の量子論の文献だと$U(\Lambda)$の生成子の符号や$J^{\mu\nu}$の交換関係などの符号がこの本と合わないことについてここで言及しておく.それは本質的にミンコフスキー計量の符号の違いからくる.もし他の文献のように$\eta=\mathrm{diag}(-1,-1,-1,+1)$をとると,ここまでの計算にその符号依存性は現れないが,ここの計算で$\eta^{ij}=\delta^{ij}$ではなく$-\delta^{ij}$が出るため,結果に余分なマイナスがつき,角運動量の交換関係を再現しない.そのために最初の$J^{\mu\nu}$の生成子にさらにマイナスをかけて,辻褄を合わせる必要があるのだ.運動量演算子$P^\mu$の符号については交換関係からの符号の制約は決まらない.だから$U(\Lambda)$の中の$\epsilon_\rho P^\rho$の符号は今のところただの約束事だ.これについては3.1節で述べるが,$P^0$が時間発展の生成子(ハミルトニアン)になるように符号を合わせる必要があるため,やはりここでも計量からの符号の差異が生まれる.}
\begin{align*}
i[J_i,J_j]=&\frac{1}{4} \epsilon_{ikl}\epsilon_{jmn} i[J^{kl},J^{mn}] \\
=&\frac{1}{4}\epsilon_{ikl}\epsilon_{jmn}\left(\eta^{lm}J^{kn}-\eta^{km}J^{ln}-\eta^{nk}J^{ml}+\eta^{nl}J^{mk}\right) \\
=&\frac{1}{4}\epsilon_{ikl}\epsilon_{jmn}\left(\delta^{lm}J^{kn}-\delta^{km}J^{ln}-\delta^{nk}J^{ml}+\delta^{nl}J^{mk}\right) \\
=&\frac{1}{4}\epsilon_{ikl}\epsilon_{jln}J^{kn}-\frac{1}{4}\epsilon_{ikl}\epsilon_{jkn}J^{ln}-\frac{1}{4}\epsilon_{ikl}\epsilon_{jmk}J^{ml}+\frac{1}{4}\epsilon_{ikl}\epsilon_{jml}J^{mk} \\
=&\frac{1}{4}\epsilon_{ikl}\epsilon_{jln}J^{kn}-\frac{1}{4}\epsilon_{ikl}\epsilon_{jkn}J^{ln}-\frac{1}{4}\epsilon_{ikl}\epsilon_{jkn}J^{ln}+\frac{1}{4}\epsilon_{ikl}\epsilon_{jln}J^{kn} \quad (添字取換)\\
=&\frac{1}{2}\epsilon_{ikl}\epsilon_{jln}J^{kn}-\frac{1}{2}\epsilon_{ilk}\epsilon_{jln}J^{kn} =\frac{1}{2}\epsilon_{ikl}\epsilon_{jln}J^{kn}+\frac{1}{2}\epsilon_{ilk}\epsilon_{jln}J^{nk} \\
=&\frac{1}{2}\epsilon_{ikl}\epsilon_{jln}J^{kn}+\frac{1}{2}\epsilon_{iln}\epsilon_{jlk}J^{kn} \quad (添字取換) \\
=&\frac{1}{2}(\epsilon_{jln}\epsilon_{kli}+\epsilon_{iln}\epsilon_{jlk})J^{kn} \\
=&-\frac{1}{2}\epsilon_{kln}\epsilon_{ilj}J^{kn} \quad (ヤコビ恒等式 ;\epsilon_{jln}\epsilon_{kli}+\epsilon_{iln}\epsilon_{jlk}+\epsilon_{kln}\epsilon_{ilj}=0) \\
=&-\epsilon_{ijl}\left(\frac{1}{2}\epsilon_{lkn}J^{kn}\right)=-\epsilon_{ijl}J_l \\
\therefore \quad & [J_i,J_j]=i\epsilon_{ijk}J_k
\end{align*}
(2.4.20)は
\begin{align*}
i[K_i,K_j]=&i[-J^{i0},-J^{j0}] \\
=&i[J^{i0},J^{j0}] \\
=&\eta^{0j}J^{i0}-\eta^{ij}J^{00}-\eta^{i0}J^{j0}+\eta^{00}J^{ji} \\
=&+J^{ij} \\
=&+\epsilon_{ijk}J_k \\
\therefore \quad [K_i,K_j]=&-i\epsilon_{ijk}J_k
\end{align*}
(2.4.19)は
\begin{align*}
i[J_i,K_j]=&\frac{1}{2}\epsilon_{ikl}i[J^{kl},-J^{j0}] \\
=&-\frac{1}{2}\epsilon_{ikl}i[J^{kl},J^{j0}] \\
=&-\frac{1}{2}\epsilon_{ikl}\left(\eta^{lj}J^{k0}-\eta^{kj}J^{l0}-\eta^{0k}J^{jl}+\eta^{0l}J^{jk}\right) \\
=&-\frac{1}{2}\epsilon_{ikl}(\delta^{lj}J^{k0}-\frac{1}{2}\delta^{kj}J^{l0}) \\
=&-\frac{1}{2}\epsilon_{ikj}J^{k0}+\frac{1}{2}\epsilon_{ijl}J^{l0}=\epsilon_{ijl}J^{l0}=-\epsilon_{ijl}K_l \\
\therefore \quad [J_i,K_j]=&i\epsilon_{ijk}K_k
\end{align*}
が得られる.ここで$i,j,k$は$1,2,3$の値をとる.交換関係(2.4.18)は角運動量演算子のものになっている.若干の誤植があるから,結果を一覧にしてまとめておこう.
\begin{align*}
[J_i,J_j]=&i\epsilon_{ijk}J_k \\
[J_i,K_j]=&i\epsilon_{ijk}K_k \\
[K_i,K_j]=&-i\epsilon_{ijk}J_k \\
[J_i,P_j]=&i\epsilon_{ijk}P_k \\
[K_i,P_j]=&-iH\delta_{ij} \\
[J_i,H]=&[P_i,H]=[H,H]=0 \\
[K_i,H]=&-iP_i
\end{align*}
この結果は後から何度も使う.\par
純粋な並進$T(1,a)$は非斉次ローレンツ群の部分群をなす.この群の積の法則は(2.3.7)によって
\begin{align*}
T(1,\bar{a})T(1,a)=T(1,\bar{a}+a)
\end{align*}
となっている.これは(2.2.24)と同じ意味で加法的であり,(2.4.3)を使って,(2.2.26)を導いたのと同様に
\begin{align*}
U(1,a)=&\left[U\left(1,\frac{1}{N}\right)\right]^N=\left[1-i\frac{a_\mu P^\mu }{N}\right]^N \\
=&\exp(-ia_\mu P^\mu)
\end{align*}
となる.また,パラメータ$\omega_{\mu\nu}$を
\begin{align*}
\tensor{\omega}{^\mu_\nu}=&\left(
\begin{matrix}
0         & \theta_3 & -\theta_2  & \beta_1 \\
-\theta_3 & 0        &  \theta_1  & \beta_2 \\
\theta_2  & -\theta_1& 0          & \beta_3 \\
\beta_1   & \beta_2  & \beta_3    & 0
\end{matrix}
\right) \\
\omega_{\mu\nu}=&\eta_{\mu\rho}\tensor{\omega}{^\rho_\nu}=\left(
\begin{matrix}
0         & \theta_3  &  -\theta_2 & \beta_1 \\
-\theta_3 & 0         &  \theta_1  & \beta_2 \\
\theta_2  & -\theta_1 & 0          & \beta_3 \\
-\beta_1  & -\beta_2  & -\beta_3   & 0
\end{matrix}
\right) \\
\theta_i=&(\tensor{\omega}{^2_3},\tensor{\omega}{^3_1},\tensor{\omega}{^1_2})=(\omega_{23},\omega_{31},\omega_{12})= \frac{1}{2}\epsilon_{ijk}\omega_{jk} \\
\beta_i =&(\tensor{\omega}{^1_0},\tensor{\omega}{^2_0},\tensor{\omega}{^3_0})=\tensor{\omega}{^i_0} =\omega_{i0}=-\omega_{0i} 
\end{align*}
とすると,
\begin{align*}
U(1+\omega,0)=&1+i\frac{1}{2}\omega_{\mu\nu}J^{\mu\nu} \\
=&1+i\frac{1}{2} \omega_{ij}J^{ij}+i\frac{1}{2}\omega_{i0}J^{i0}+i\frac{1}{2} \omega_{0i}J^{0i} \\
=&1+i\frac{1}{2} \omega_{ij}\epsilon_{ijk}J_k -i\omega_{i0} K_i \quad \because J_i=\epsilon_{ijk}J^{jk},K_i=J_{i0}\\
=&1+i(\theta_i J_i-\beta_i K_i) =1+i(\bm{\theta} \cdot \mathbf{J}-\bm{\beta}\cdot \mathbf{K})
\end{align*}
となるが,
\begin{align*}
\tensor{\Lambda}{^\nu_\mu}\tensor{{\Lambda'}}{^\mu_\rho}=(\delta_\mu^\nu +\tensor{\omega}{^\nu_\mu})(\delta^\mu_\rho+\tensor{{\omega'}}{^\mu_\rho})=\delta^\rho_\nu+\tensor{\omega}{^\nu_\rho}+\tensor{{\omega'}}{^\nu_\rho}
\end{align*}
であるから($\omega,\omega'$について二次の項は無視),微小ローレンツ変換$T(1+\omega,0)T(1+\omega')=T(1+\omega+\omega',0)$という加法的な群になる.
\begin{align*}
U(1+\omega,0)U(1+\omega',0)=&U(1+\omega+\omega',0) \\
=&1+i((\bm{\theta}+\bm{\theta}') \cdot \mathbf{J}-(\bm{\beta}+\bm{\beta}')\cdot \mathbf{K})
\end{align*}
よって有限な表現は
\begin{align*}
U(\Lambda,0)=&\left[U\left(1+\frac{\omega}{N},0\right)\right]^N \\
=&\left[1+\frac{i}{N}(\bm{\theta}\cdot \mathbf{J}-\bm{\beta}\cdot \mathbf{K}) \right]^N \\
=&\exp \Bigl(i(\bm{\theta}\cdot \mathbf{J}-\bm{\beta}\cdot \mathbf{K})\Bigr)
\end{align*}
となる.ローレンツ変換$\Lambda$が回転$R(\theta)$のとき$\beta=0$だから
\begin{align*}
U(R(\theta),0)=\exp(i\bm{\theta}\cdot \mathbf{J})
\end{align*}
だ.ブースト$\mathbf{x}\to \mathbf{x}+\mathbf{v}t$は$\beta_i=v_i,\theta_i=0$のローレンツ変換だから
\begin{align*}
U(\mathbf{v})=\exp(-i\mathbf{v}\cdot \mathbf{K})
\end{align*}
だ.よってパラメータ$\theta_i,\beta_i$は直感的にそれぞれ角度とラピディティ\footnote{$\beta$が速度であるように見えるが,実際は近似で,実際はラピディティ(無次元量)である.$x,t$に関するローレンツ変換が
\begin{align*}
\left(
\begin{matrix}
x' \\
t'
\end{matrix}
\right)=\left(
\begin{matrix}
\cosh \beta & \sinh \beta \\
\sinh \beta & \cosh \beta 
\end{matrix}
\right)\left(
\begin{matrix}
x \\
t
\end{matrix}
\right)\approx \left(
\begin{matrix}
1  & \beta \\
\beta & 1
\end{matrix}
\right)\left(
\begin{matrix}
x \\
t
\end{matrix}
\right)=\left(
\begin{matrix}
x+\beta t \\
t+\beta x
\end{matrix}
\right)
\end{align*}
と書けることを思い出す.光速$c=1$が$ct$として入っていることに注意.}に対応したものであることがわかる.(一般的なポアンカレ変換$x^\mu \to \tensor{\Lambda}{^\mu_\nu}x^\nu+a^\mu$の場合,(2.3.11)より$U(\Lambda,a)$は加法的な群にはならないので,
\begin{align*}
U(\Lambda,a)=\exp \left( -ia_\mu P^\mu +i(\bm{\theta}\cdot \mathbf{J}-\bm{\beta}\cdot \mathbf{K}) \right)
\end{align*}
とはならない.ポアンカレ変換はローレンツ変換$x^\mu\to \tensor{\Lambda}{^\mu_\nu}x^\nu$と並進$x^\mu \to x^\mu +a^\mu$の合成であるから
\begin{align*}
U(\Lambda,a)=\exp (-ia^\mu P^\mu)\exp \Bigl(i(\bm{\theta}\cdot \mathbf{J}-\bm{\beta}\cdot \mathbf{K})\Bigr)
\end{align*}
となる.ベーカー・キャンベル・ハウスドロフ(BCH)の公式より,この式は上の正しくない式とは等価にはならない.)



\vskip\baselineskip



ポアンカレ代数とニュートン力学の対称群,つまりガリレイ群のリー代数を比べる.(2.4.18)~(2.4.24)から低速度の極限として導く.これがイネヌ・ウィグナー縮約という.代表的な質量が$m$程度で,代表的な速度が$v$の粒子系では,運動量と角運動量,ブースト演算子は$\mathbf{J} \sim 1,\mathbf{P}\sim mv,\mathbf{K}\sim 1/v$の大きさだと考えられる.(運動量演算子$P^\mu$の大きさは自明.回転角$\theta$は無次元な量であり,$\exp$の肩$i\bm{\theta}\cdot \mathbf{J}$は無次元である必要があるから$\mathbf{J}$も無次元である必要がある.同じ理由で$\bm{\beta}\cdot \mathbf{K}$も無次元であることより$\mathbf{K}\sim 1/v$がわかる.)一方,$M$を全質量とし,質量以外のエネルギー(運動エネルギーとポテンシャルエネルギー)を$W$とすると,全エネルギー演算子$H=M+W$と書ける.(相対論的には$E=mc^2+mv^2/2+\cdots $となるからだ.)それぞれの大きさは$M\sim m,W\sim mv^2$となっている.(2.4.18)~(2.4.24)から,交換関係は$v\ll 1$のときには
\begin{align*}
&[J_i,J_j]=i\epsilon_{ijk} J_k ,\quad [J_i, K_j ]=i\epsilon_{ijk}K_k \\
&[K_i ,K_j]=0 \quad \because ( K に比べ J は無視できる大きさ) \\
&[J_i ,P_j]=i\epsilon_{ijk} P_k \\
&[K_i ,P_j]=-iM\delta_{ij} \quad \because (M に比べ W は無視できる大きさ) \\
&[J_i,M]=[P_i,M]=[K_i ,M]=[W,M]=0 \\
&[J_i ,W]=[P_i ,W]=0,\quad [K_i,W]=-iP_i
\end{align*}
最後の行は$[K_i,H]=[K_i,M]+[K_i,W]=[K_i,W]=-iP_i$から,また$[J_i,H],[P_i,H]$でも同様の手順を踏んで残りの式も得られる.\par
並進$\mathbf{x}\to \mathbf{x}+\mathbf{a}$とブースト$\mathbf{x}\to \mathbf{x}+\mathbf{v}t$の積は$\mathbf{x}\to \mathbf{x}+ \mathbf{v}t +\mathbf{a}$のはずだが,ヒルベルト空間においてはこれらの演算子はそうは働かない.BCH公式より
\begin{align*}
\exp(-i\mathbf{v}\cdot \mathbf{K})\exp(-i\mathbf{a}\cdot \mathbf{P})=&\exp\left[-i\mathbf{v}\cdot \mathbf{K} -i\mathbf{a}\cdot \mathbf{P}+\frac{1}{2}[-i\mathbf{v}\cdot \mathbf{K},-i\mathbf{a}\cdot \mathbf{P} ]+\cdots \right] \\
=&\exp\left[-i\left(\mathbf{v}\cdot \mathbf{K} +\mathbf{a}\cdot \mathbf{P}\right)+\frac{1}{2}i M\delta_{ij}v_i a_j\right] \\
=&\exp\left(\frac{i}{2}M\mathbf{a}\cdot \mathbf{v} \right)\exp\left[-i\left(\mathbf{v}\cdot \mathbf{K} +\mathbf{a}\cdot \mathbf{P}\right)\right]
\end{align*}
となる.二,三行目では,$M$が他の演算子と可換であるから,それ以上高次の項が現れないことを用いている.位相因子$\exp(\frac{i}{2}M\mathbf{a}\cdot \mathbf{v})$が現れることは,ここではこれらの表現が射影表現であり,超選択則によって質量の異なる重ね合わせ(例えば$m,m'$の質量をもつ1粒子状態の重ね合わせ$\Psi_m+\Psi_{m'}$が禁じられている.)2.2節でも若干説明したが,ガリレイ群のリー代数に,すべての生成子と可換でその固有値がいろいろな状態の質量になっている生成子$M$を付け加えて,ガリレイ群を形式的に拡張して射影表現を通常の非射影表現にできる.これを中心拡大と呼ぶ.(つまり,通常のガリレイ群は$[K_i,P_j]=0$なのだが,量子力学のユニタリ射影表現だとそれが満たせず,単位元に比例する中心電荷の項が現れてしまう.そこで,上の交換関係では$M$はまだ単なる数値だが,それを形式的に$M\Psi_m=m\Psi_m$となる演算子$M$で,かつ他の生成子と可換な演算子であると解釈し直し,ガリレイ群の生成子の一員として加えてやる.これで$[K_i,P_j]$の右辺は単位元ではなくなったから,中心電荷をリー代数から取り除くことができた.そして$\exp(iM\mathbf{a}\cdot \mathbf{v})$も位相因子ではなくなったから,非射影表現にできたことになる..)

\newpage

\subsection{1粒子状態}
1粒子状態を非斉次ローレンツ群のもとでの変換に従って分類する.\par
四元エネルギー運動量ベクトル$P^\mu$は互いに可換$[P^\mu,P^\nu]=0$だから,物理的状態ベクトルを,この四元ベクトルの固有値を使って表すことは自然だ.その他の自由度(粒子のスピンや種類など)をすべて添え字$\sigma$を使って表し,状態ベクトル$\Psi_{p,\sigma}$は
\begin{align*}
P^\mu \Psi_{p,\sigma}=p^\mu \Psi_{p,\sigma}
\end{align*}
を満たすとする.\par
一般的な状態は,例えばいくつかの束縛されていない粒子からなり,そのとき添え字$\sigma$は離散的のみならず連続的な値もとる.(初等的な量子力学において,ポテンシャルに束縛された波動関数は離散的なエネルギー・運動量をとることを思い出すといい.)\par
$\Rightarrow$「1粒子状態」の定義に「添え字$\sigma$が純粋に離散的」という条件を含めることにする.そしてここではその場合に話を限る\footnote{このノートでは添え字$\sigma$に連続パラメータ$\theta\in [0,2\pi)$をもつような場合も取り上げる.そのようなものは質量ゼロだがローレンツ群$SO(3,1)$(の小群$ISO(2)$の)無限次元表現をなす.}.\par
(水素原子の基底状態のような,2粒子以上の束縛状態は,1粒子状態として考える.これは「基本粒子」の1粒子状態ではないが,この違いはここでは重要ではないらしい.)

\vskip\baselineskip

(2.5.1)と(2.4.26)から,状態$\Psi_{p,\sigma}$が並進のもとで以下のように変換することがわかる.
\begin{align*}
U(1,a)\Psi_{p,\sigma}=&\exp(-ia_\mu P^\mu)\Psi_{p,\sigma} \\
=&\exp(-ia_\mu p^\mu)\Psi_{p,\sigma}
\end{align*}
これらの状態が,斉次ローレンツ変換のもとでどう変換するかを考える.\par
(2.4.9)を用いると,$\Psi_{p,\sigma}$に量子論的斉次ローレンツ変換$U(\Lambda,0)\equiv U(\Lambda)$を施すと,四元運動量の固有値が$\Lambda p$の状態ができることがわかる.
\begin{align*}
P^\mu \left[U(\Lambda)\Psi_{p,\sigma}\right]=&U(\Lambda)\left[U^{-1}(\Lambda)P^\mu U(\Lambda)\right]\Psi_{p,\sigma} \\
=&U(\Lambda)\left[\tensor{(\Lambda^{-1})}{_\rho^\mu}P^\rho\right]\Psi_{p,\sigma} \quad \because(2.4.9) \\
=&U(\Lambda)\tensor{\Lambda}{^\mu_\rho}p^\rho \Psi_{p,\sigma} \\
=&\tensor{\Lambda}{^\mu_\rho}p^\rho \left[U(\Lambda)\Psi_{p,\sigma}\right]
\end{align*}
したがって,$U(\Lambda)\Psi_{p,\sigma}$は状態ベクトル$\Psi_{\Lambda p,\sigma'}$の線形結合でなければならない.
\begin{align*}
U(\Lambda)\Psi_{p,\sigma}=\sum_{\sigma'} C_{\sigma'\sigma}(\Lambda,p)\Psi_{\Lambda p,\sigma'}
\end{align*}
(これを(2.5.2)に代入して,等式がなりたつことを確認できる.)一般に$\Psi_{p,\sigma}$の適切な線形結合を使って,$\sigma$の添え字を,行列$C_{\sigma'\sigma}(\Lambda,p)$がブロック対角になるように選べる.例えば両辺の線形結合をとって
\begin{align*}
U(\Lambda)\Psi'_{p,\alpha}=U(\Lambda)\sum_{\sigma}A_{\sigma\alpha}\Psi_{p,\sigma}=&\sum_{\alpha'\sigma\sigma'}A_{\sigma'\alpha'}(A^{-1})_{\alpha'\sigma''}C_{\sigma''\sigma}A_{\sigma\alpha}\Psi_{\Lambda p,\sigma'} \\
=&\sum_{\sigma\sigma'}(A^{-1})_{\alpha'\sigma''}C_{\sigma''\sigma}A_{\sigma\alpha}\left(\sum_{\alpha'}A_{\sigma'\alpha'}\Psi_{\Lambda p,\sigma'}\right) \\
=&\sum_{\alpha'}(A^{-1}C A)_{\alpha'\alpha}\Psi'_{\Lambda p,\alpha'}
\end{align*}
とする.この行列$A^{-1}C A=C'$がブロック対角行列にされ,$C'=D^1\oplus D^2 \oplus \cdots \oplus D^n$のように完全に分解され,これ以上分解することができないという意味で既約分解されたとき,この既約な成分$\Psi_{p,\alpha}'$をまとめて一種類の粒子と特定するのが自然だ.したがって,非斉次ローレンツ群の既約表現での係数$C_{\sigma'\sigma}(\Lambda,p)$の構造を調べることとする.

\vskip\baselineskip

全ての固有順時ローレンツ変換$\tensor{\Lambda}{^\mu_\nu}$のもとで不変な関数は,不変平方$p^2\equiv \eta_{\mu\nu}p^\mu p^\nu$と$p^2 \leq 0$(time-like)のときの$p^0$の符号であることに注意する.後者は,固有順時なら$\tensor{\Lambda}{^0_0}\geq 1$だから,
\begin{align*}
&(\tensor{\Lambda}{^0_0}p^0)^2-\left(\sqrt{(\tensor{\Lambda}{^0_0})^2-1}\sqrt{p^2+(p^0)^2}\right)^2 \\
=&-(\tensor{\Lambda}{^0_0})^2 p^2 +p^2+(p^0)^2 \\
=&-\left[(\tensor{\Lambda}{^0_0})^2-1\right]p^2 +(p^0)^2 \\
\geq& 0 \quad \because p^2 \leq 0,\tensor{\Lambda}{^0_0}\geq 1 \\
\therefore \quad &(\tensor{\Lambda}{^0_0}p^0)^2\geq \left(\sqrt{(\tensor{\Lambda}{^0_0})^2-1}\sqrt{p^2+(p^0)^2}\right)^2
\end{align*}
となって,$p^0\geq 0$のとき
\begin{align*}
\tensor{\Lambda}{^0_0}p^0 \geq \sqrt{(\tensor{\Lambda}{^0_0})^2-1}\sqrt{p^2+(p^0)^2}
\end{align*}
より
\begin{align*}
p'^0=&\tensor{\Lambda}{^0_\nu}p^\nu=\tensor{\Lambda}{^0_0}p^0+\tensor{\Lambda}{^0_i}p^i \\
\geq &\tensor{\Lambda}{^0_0}p^0-\left| \tensor{\Lambda}{^0_i}p^i \right| \\
\geq  &\tensor{\Lambda}{^0_0}p^0 -\sqrt{(\tensor{\Lambda}{^0_i})^2 } |\mathbf{p}| \quad \because シュワルツの不等式 \\
=& \tensor{\Lambda}{^0_0}p^0 -\sqrt{(\tensor{\Lambda}{^0_0})^2-1} \sqrt{p^2+(p^0)^2}  \geq 0
\end{align*}
となって,$p'^0\geq 0$であり,一方$p^0\leq 0$のとき
\begin{align*}
-\tensor{\Lambda}{^0_0}p^0 \geq \sqrt{(\tensor{\Lambda}{^0_0})^2-1}\sqrt{p^2+(p^0)^2}
\end{align*}
より
\begin{align*}
p'^0=&\tensor{\Lambda}{^0_\nu}p^\nu=\tensor{\Lambda}{^0_0}p^0+\tensor{\Lambda}{^0_i}p^i \\
\leq &\tensor{\Lambda}{^0_0}p^0+\left| \tensor{\Lambda}{^0_i}p^i \right| \\
\leq  &\tensor{\Lambda}{^0_0}p^0 +\sqrt{(\tensor{\Lambda}{^0_i})^2 } |\mathbf{p}| \\
=& \tensor{\Lambda}{^0_0}p^0 +\sqrt{(\tensor{\Lambda}{^0_0})^2-1} \sqrt{p^2+(p^0)^2}  \leq 0
\end{align*}
となって,$p'^0\leq 0$である.したがって$p^0$の符号は固有順時ローレンツ変換かつ$p^2\leq 0$のとき保存する.\par
したがって,$p^2$の値と$p^0$の符号それぞれに対して,「基準となる」四元運動量$k^\mu$を選んで,その類に属するどんな$p^\mu$も以下のように表すことにする.
\begin{align*}
p^\mu=\tensor{L}{^\mu_\nu}(p)k^\nu
\end{align*}
ここで$\tensor{L}{^\mu_\nu}$は基準となるローレンツ変換で,$p^\mu$に依存し,さらに$k^\mu$として何に選んだかにも依存する.(つまり別の$k'$をもってきて$p^\mu=\tensor{{L'}}{^\mu_\nu}(p)k'^\nu$とすると,$\tensor{L}{^\mu_\nu}(p)\neq \tensor{{L'}}{^\mu_\nu}(p)$である.まぁ当たり前か.)こうすると運動量$p$の状態$\Psi_{p,\sigma}$を
\begin{align*}
\Psi_{p,\sigma}\equiv N(p)U(L(p))\Psi_{k,\sigma}
\end{align*}
と定義できる.

\vskip\baselineskip

(2.5.5)に任意のローレンツ変換を施すと,以下を得る.
\begin{align*}
U(\Lambda)\Psi_{p,\sigma}=&N(p)U(\Lambda)U(L(p))\Psi_{k,\sigma} \\
=&N(p)U(\Lambda L(p))\Psi_{k,\sigma} \\
=&N(p)U\Bigl(L(\Lambda p) \cdot L^{-1}(\Lambda p) \Lambda L(p)\Bigr) \Psi_{k,\sigma} \\
=&N(p)U\Bigl(L(\Lambda p)\Bigr) U\Bigl(L^{-1}(\Lambda p) \Lambda L(p)\Bigr) \Psi_{k,\sigma}
\end{align*}
さて,ローレンツ変換$L^{-1}(\Lambda p)\Lambda L(p)$は,$L(p):k \mapsto p$,$\Lambda :p\mapsto \Lambda p$,$L^{-1}(\Lambda p):\Lambda p \mapsto k$とする合成変換だ.重要なのは,この変換がローレンツ部分群に属するということだ.この部分群は$k^\mu$を不変に保つローレンツ変換$\tensor{W}{^\mu_\nu}$からなっている.
\begin{align*}
\tensor{W}{^\mu_\nu}k^\nu =k^\mu
\end{align*}
この部分群は小群と呼ばれている.(2.5.7)を満たすどのような$\tensor{W}{^\mu_\nu}$についても,以下が成立する.
\begin{align*}
U(W)\Psi_{k,\sigma}=\sum_{\sigma'}D_{\sigma'\sigma}(W)\Psi_{k,\sigma'}
\end{align*}
(これは(2.5.3)と対応している.)係数$D(W)$は小群の表現を与える.つまり,どのような$W,\bar{W}$についても
\begin{align*}
&\sum_{\sigma'} D_{\sigma'\sigma}(\bar{W}W)\Psi_{k,\sigma'} \\
=&U(\bar{W}W)\Psi_{k,\sigma} =U(\bar{W})U(W)\Psi_{k,\sigma} \\
=&U(\bar{W})\sum_{\sigma''}D_{\sigma''\sigma}(W)\Psi_{k,\sigma''} =\sum_{\sigma',\sigma''}D_{\sigma''\sigma}(W)D_{\sigma'\sigma''}(\bar{W})\Psi_{k,\sigma'} \\
=&\sum_{\sigma'\sigma''}D_{\sigma'\sigma''}(\bar{W})D_{\sigma''\sigma}(W)\Psi_{k,\sigma'} 
\end{align*}
となり,したがって
\begin{align*}
D_{\sigma'\sigma}(\bar{W}W)=\sum_{\sigma''}D_{\sigma'\sigma''}(\bar{W})D_{\sigma''\sigma}(W)
\end{align*}
が成立する.特に(2.5.8)を小群の変換
\begin{align*}
W(\Lambda,p)\equiv L^{-1}(\Lambda p)\Lambda L(p)
\end{align*}
に適用して(2.5.6)を使うと
\begin{align*}
U(\Lambda)\Psi_{p,\sigma}=&N(p)U\Bigl(L(\Lambda p)\Bigr)U\Bigl(W(\Lambda,p)\Bigr)\Psi_{k,\sigma} \\
=&N(p)U\Bigl(L(\Lambda p)\Bigr)\sum_{\sigma'}D_{\sigma'\sigma}\Bigl(W(\Lambda,p)\Bigr)\Psi_{k,\sigma'} \\
=&N(p)\sum_{\sigma'}D_{\sigma'\sigma}\Bigl(W(\Lambda,p)\Bigr) U\Bigl(L(\Lambda p)\Bigr)\Psi_{k,\sigma'}
\end{align*}
となる.(2.5.5)の定義を思い出すとさらに
\begin{align*}
=&N(p)\sum_{\sigma'}D_{\sigma'\sigma}\Bigl(W(\Lambda,p)\Bigr) \frac{1}{N(\Lambda p)}\Psi_{\Lambda p,\sigma'} \\
=&\left(\frac{N(p)}{N(\Lambda p)}\right) \sum_{\sigma'}D_{\sigma'\sigma}\Bigl(W(\Lambda,p)\Bigr) \Psi_{\Lambda p,\sigma'} \left(= \sum_{\sigma'}C_{\sigma'\sigma}(\Lambda,p)\Psi_{\Lambda p,\sigma'}\right)
\end{align*}
となることがわかる.規格化の問題を除けば,変換則(2.5.3)の係数$C_{\sigma'\sigma}$を決めるという問題は,小群の表現$D$を求めるという問題に帰着する.\par
非斉次ローレンツ群のような群の表現を,小群の表現から求めるこの方法は誘導表現の方法といわれる.

\vskip\baselineskip

表2.1を見よう.(a)に関しては$(0,0,0)$ベクトルを$SO(3)$回転させても$(0,0,0)$ベクトルにしかならないため,$k^\mu=(0,0,0,M)$の三次元成分を$SO(3)$回転させても不変だ.よって基準となる$k^\mu$は$(0,0,0,M)$で,小群は$SO(3)$となる.(b)~(f)も同様に考える.($ISO(2)$の説明はこの後質量ゼロ状態について調べるときに述べる.)この中で(a)(c)(f)のみが物理的だ.なぜなら物理的にはエネルギー$E=p^0$は正である必要があり,そのために(b)と(d)は棄却される.さらに$p^2$は$-M^2$に等しくなることが四元運動量の定義よりわかる.
\begin{align*}
-d\tau^2 =&ds^2=\eta_{\mu\nu}dx^\mu dx^\nu \\
p^2=&\eta_{\mu\nu}p^\mu p^\nu=-m^2 \leq 0 ,\quad p^\mu=m\frac{dx^\mu}{d\tau}
\end{align*}
このとき$p^2=0$はゼロ質量の粒子の四元運動量であることがわかる.$p^2=-m^2$は質量$m$の粒子の四元運動量だ.よって(a)と(c)は物理的だ.また,$p^2\leq 0$より$(p^0)^2\geq |\mathbf{p}|^2$が得られる.したがって(e)は物理的でない.(f)は単に真空状態を表すので,物理的だ.\par
以下では(a)と(c)の場合のみを考える.これらはそれぞれ$M>0$と質量ゼロの場合だ.

\vskip\baselineskip

量子力学の通常の正規直交化により,基準運動量$k^\mu$の状態が以下の意味で正規直交になるように選ぶ.
\begin{align*}
\left(\Psi_{k',\sigma'},\Psi_{k,\sigma}\right)=\delta^3(\mathbf{k}-\mathbf{k'})\delta_{\sigma'\sigma}
\end{align*}
($\mathbf{k}$に関するデルタ関数は,$\Psi_{k,\sigma},\Psi_{k',\sigma'}$がそれぞれ固有値$\mathbf{k}$をもつエルミート演算子(ここでは$\mathbf{A}$とおく)であることより
\begin{align*}
&\int d^3\mathbf{k} \left( \Psi_{k',\sigma},\mathbf{A} \Psi_{k,\sigma}\right) = \int d^3\mathbf{k} \, \mathbf{k}\left( \Psi_{k',\sigma},\Psi_{k,\sigma}\right) \\
=&\int d^3\mathbf{k} \left( \mathbf{A} \Psi_{k',\sigma}, \Psi_{k,\sigma}\right)=\mathbf{k}'\int d^3\mathbf{k} \left(\Psi_{k',\sigma}, \Psi_{k,\sigma}\right)=\mathbf{k}'
\end{align*}
となって,デルタ関数に比例していなければならないことより現れる.最後の行では規格直交化されていることより積分の値が1になるようにした.)これより
\begin{align*}
\delta^3(\mathbf{k}-\mathbf{k'}) \delta_{\sigma'\sigma}=&\left(\Psi_{k',\sigma'},\Psi_{k,\sigma}\right) \\
=&\left(U(W)\Psi_{k',\sigma'},U(W)\Psi_{k,\sigma}\right) \\
=&\left( \sum_{\sigma'''}D_{\sigma'''\sigma'}(W)\Psi_{k',\sigma'''}\, ,\, \sum_{\sigma''} D_{\sigma''\sigma}(W)\Psi_{k,\sigma''}\right) \\
=&\sum_{\sigma''\sigma'''}D^*_{\sigma'''\sigma'}(W)D_{\sigma''\sigma}(W)\left(\Psi_{k',\sigma'''},\Psi_{k,\sigma''}\right) \\
=&\sum_{\sigma''\sigma'''}D^*_{\sigma'''\sigma'}(W)D_{\sigma''\sigma}(W)\left(\Psi_{k',\sigma'''},\Psi_{k,\sigma''}\right) \\
=&\sum_{\sigma''\sigma'''}D^*_{\sigma'''\sigma'}(W)D_{\sigma''\sigma}(W) \delta_{\sigma''\sigma'''}\delta^3(\mathbf{k}-\mathbf{k'}) \\
=&\sum_{\sigma''}D^*_{\sigma''\sigma'}(W)D_{\sigma''\sigma}(W) \delta^3(\mathbf{k}-\mathbf{k'}) \\
=&\sum_{\sigma''}D^\dagger_{\sigma'\sigma''}(W)D_{\sigma''\sigma}(W) \delta^3(\mathbf{k}-\mathbf{k'})
\end{align*}
よって
\begin{align*}
&\sum_{\sigma''}D^\dagger_{\sigma'\sigma''}(W)D_{\sigma''\sigma}(W)=\delta_{\sigma'\sigma'} \\
\therefore \quad &D^\dagger(W)=D^{-1}(W)
\end{align*}
となって,小群の表現がユニタリーであることがわかる.(有限次元であるとは限らない.実際,$p^2>0$と$p^\mu=0$に対応する小群$SO(2,1)$と$SO(3,1)$はコンパクト群でないから,自明でない有限次元ユニタリー表現を持たない.小群の下で非自明に変換する$p^2>0$か$p^\mu=0$の運動量の状態が一つでもあれば,必ず無限個存在し無限次元表現をなす.もう少し詳しく言えば,状態$\Psi_{k,\sigma}$にローレンツ変換が自明な表現として作用するとは
\begin{align*}
U(W)\Psi_{k,\sigma}=&\sum_{\sigma'}D_{\sigma'\sigma}(W)\Psi_{k,\sigma'} \\
=&\sum_{\sigma'}\delta_{\sigma'\sigma}(W)\Psi_{k,\sigma'}=\Psi_{k,\sigma}
\end{align*}
となって,$\Psi_{k,\sigma}$がローレンツ変換のもとで$D=1$という表現にしたがって変換することをいう.例えば(f)の場合を考えて,真空は自発的対称性の破れなどにより二種類以上存在するかもしれないが,それらは個別にローレンツ変換のもとで自明な変換をする必要がある.もし,真空が何種類かあって,そのいくつかは自明な変換をするとしても,ひとつでも空気を読まずに非自明な表現$D\neq 1$にしたがって変換するものが存在すれば,そのような非自明な変換性をもつ真空は必ず無限個存在していなければならない.)\par
さて,任意の運動量についてのスカラー積はどうなるだろうか.(2.5.5)(2.5.11)の演算子$U(\Lambda)$のユニタリー性から,スカラー積が次のようになることがわかる.
\begin{align*}
\left(\Psi_{p',\sigma'},\Psi_{p,\sigma}\right)=&N(p)\left(\Psi_{p',\sigma'},U(L(p))\Psi_{k,\sigma}\right) \quad \because (2.5.5)\\
=&N(p)\left( U^{-1}(L(p))\Psi_{p',\sigma'},\Psi_{k,\sigma} \right) \quad \because Uのユニタリー性\\
=&N(p)\left( U(L^{-1}(p))\Psi_{p',\sigma'},\Psi_{k,\sigma} \right) \quad \because U(\Lambda^{-1})=U^{-1}(\Lambda)\\
=&N(p)\left( N(p')\sum_{\sigma''}D_{\sigma''\sigma'}\Bigl(W(L^{-1}(p),p')\Bigr)\Psi_{k',\sigma''} \, ,\, \Psi_{k,\sigma} \right) \quad \because (2.5.11)上式\\
=&N(p)N(p')\sum_{\sigma''}D^*_{\sigma''\sigma'}\Bigl(W(L^{-1}(p),p')\Bigr)\left(\Psi_{k',\sigma''},\Psi_{k,\sigma}\right) \\
=&N(p)N(p')\sum_{\sigma''}D^*_{\sigma''\sigma'}\Bigl(W(L^{-1}(p),p')\Bigr)\delta_{\sigma''\sigma}\delta^3(\mathbf{k}-\mathbf{k'}) \\
=&N(p)N(p')D^*_{\sigma\sigma'}\Bigl(W(L^{-1}(p),p')\Bigr)\delta^3(\mathbf{k}-\mathbf{k'})
\end{align*}
ここで$k'\equiv L^{-1}(p)p'$と定義した.また$k=L^{-1}(p)p$だから$k-k'=L^{-1}(p)(p-p')$で,デルタ関数$\delta^3(\mathbf{k}-\mathbf{k}')$は$\delta^3(\mathbf{p}-\mathbf{p}')$に比例することがわかる.($\delta(ax)=\frac{1}{|a|}\delta(x)$などの公式を思い出せばよい.)このデルタ関数により,右辺では$p=p'$とできて,このとき,ここに現れる小群の変換は(2.5.10)より
\begin{align*}
W(L^{-1}(p),p)=L^{-1}(L^{-1}(p)p)L^{-1}(p)L(p)=L^{-1}(k)=1
\end{align*}
(最後の変形では$L(k)k=k$より$L(k)=1$を用いた)となり自明である.したがってスカラー積は.$D_{\sigma'\sigma}(1)=\delta_{\sigma'\sigma}$より
\begin{align*}
\left(\Psi_{p,\sigma'},\Psi_{p,\sigma}\right)=|N(p)|^2\delta_{\sigma'\sigma}\delta^3(\mathbf{k}-\mathbf{k'})
\end{align*}
となる.残りは$\delta^3(\mathbf{k}-\mathbf{k}')$と$\delta^3(\mathbf{p}-\mathbf{p}')$の間の比例係数を以下で求める.\par
任意のスカラー関数$f(p)$の四元運動量が$-p^2=M^2\geq 0$と$p^0>0$(つまり(a)か(c)の場合)の領域でのローレンツ不変な積分は以下のように書くことができる.
\begin{align*}
&\int d^4p \delta(p^2+M^2)\theta(p^0)f(p) \\
=&\int d^3\mathbf{p} dp^0 \delta((p^0)^2 -\mathbf{p}^2-M^2)\theta(p^0)f(\mathbf{p},p^0) \\
=&\int d^3\mathbf{p} dp^0 \frac{1}{2\sqrt{\mathbf{p}^2 +M^2}} \left[\delta (p^0+\sqrt{\mathbf{p}^2+M^2})+\delta (p^0-\sqrt{\mathbf{p}^2+M^2})\right]\theta(p^0)f(p) \\
&\qquad \qquad \qquad \qquad \because \delta(x^2-a^2)=\frac{1}{2a}\left[\delta(x+a)+\delta (x-a)\right] \\
=&\int d^3 \mathbf{p} \frac{1}{2\sqrt{\mathbf{p}^2+M^2}}\left[\theta(\sqrt{\mathbf{p}^2+M^2})f(\mathbf{p},\sqrt{\mathbf{p}^2+M^2})+\theta(-\sqrt{\mathbf{p}^2+M^2})f(\mathbf{p},-\sqrt{\mathbf{p}^2+M^2})\right] \\
=&\int d^3\mathbf{p} \frac{f(\mathbf{p},\sqrt{\mathbf{p}^2+M^2})}{2\sqrt{\mathbf{p}^2+M^2}} \quad \because \theta(x>0)=1,\theta(x<0)=0 
\end{align*}
最初の式のデルタ関数は$p^2+M^2=0$の領域に制限するため,階段関数は$p^0>0$の領域に制限するために付け加えている.最初の式がローレンツ不変な形式であるため,最後の式もローレンツ不変である.したがって質量殻$p^2+M^2=0$上で積分するとき(つまり$p^0=\sqrt{\mathbf{p}^2+M^2}>0$として関数$f(p)$の引数を$f(\mathbf{p},\sqrt{\mathbf{p}^2+M^2})$と制限して運動量積分するとき),ローレンツ不変な不変体積要素は
\begin{align*}
d^3\mathbf{p}/2\sqrt{\mathbf{p}^2+M^2}
\end{align*}
で与えられる.(これは(5.2.3)(5.2.4)などで,後からも出てくる.)デルタ関数の定義より
\begin{align*}
F(\mathbf{p})=&\int d^3\mathbf{p}' \delta^3(\mathbf{p}-\mathbf{p}')F(\mathbf{p}') \\
=&\int \frac{d^3\mathbf{p}' }{2\sqrt{\mathbf{p}'^2+M^2}}\left[ 2\sqrt{\mathbf{p}'^2+M^2}\delta^3(\mathbf{p}-\mathbf{p}') \right]F(\mathbf{p}')
\end{align*}
となるので,括弧の中はローレンツ不変でなければならない.したがって不変デルタ関数は
\begin{align*}
2\sqrt{\mathbf{p}'^2+M^2}\delta^3(\mathbf{p}-\mathbf{p}')=2p^0\delta^3(\mathbf{p}-\mathbf{p}')
\end{align*}
$p',p$はそれぞれローレンツ変換$L(p)$で$k',k$と関係していたので,この不変デルタ関数はどちらの変数を使っても等しくなる.
\begin{align*}
2p^0\delta^3(\mathbf{p}-\mathbf{p}')=2k^0\delta^3(\mathbf{k}-\mathbf{k}')
\end{align*}
したがって結局(2.5.14)は
\begin{align*}
\left(\Psi_{p',\sigma'},\Psi_{p,\sigma}\right)=|N(p)|^2\delta_{\sigma'\sigma}\frac{p^0}{k^0}\delta^3(\mathbf{p}-\mathbf{p}')
\end{align*}
となる.規格化定数$N(p)$はしばしば単に$N(p)=1$と選ばれる(Peskin,Srednickiなど)が,しかしそうするとスカラー積に因子$p^0/k^0$が登場する.このため,ここではよりよく用いられる定義
\begin{align*}
N(p)=\sqrt{k^0/p^0}
\end{align*}
をとり
\begin{align*}
\left(\Psi_{p',\sigma'},\Psi_{p,\sigma}\right)=\delta_{\sigma'\sigma}\delta^3(\mathbf{p}-\mathbf{p}')
\end{align*}
とする.\par
以下では,特に重要な(a)と(c)の場合について,つまり質量$M>0$と質量ゼロの粒子について考察する.

\vskip\baselineskip

(a)正定値質量\par
この場合,図2.2より小群は三次元回転$SO(3)$だ.(半整数スピンも射影表現ではなく表現として組み込むため,以下では回転群として$SU(2)$を考える.)これはコンパクト群であり,その場合のユニタリー表現は$j=0,\frac{1}{2},1,\cdots$に対し$2j+1$次元のユニタリー既約表現$D^{(j)}_{\sigma'\sigma}(R)$の直和に分解できる.一般にPeter-Weylの定理(小林・大島「リー群と表現論」参照)により,コンパクト群の既約ユニタリー表現は有限次元である.もちろん$SU(2)$表現の既約分解は既に量子力学で学んだ.(よりこの本と表記や考え方が近いWeinberg「Lectures on Quantum Mechanics(量子力学講義)」を参照のこと)\par
これは微小な行列$\Theta_{ik}=-\Theta_{ki}$による基準的な微小回転の行列$R_{ik}=\delta_{ik}+\Theta_{ik}$から作られる.
\begin{align*}
D^{(j)}_{\sigma'\sigma}(1+\Theta)=\delta_{\sigma'\sigma}+\frac{i}{2}\Theta_{ik}\left(J^{(j)}_{ik}\right)_{\sigma'\sigma}+\cdots
\end{align*}
ここで
\begin{align*}
\left(J^{(j)}_{21}\right)_{\sigma'\sigma}=&\left(J^{(j)}_3\right)_{\sigma'\sigma}=\sigma\delta_{\sigma'\sigma} \quad (\sigma=j,j-1,\cdots ,-j+1,-j) \\
(J^{(j)}_{\pm})_{\sigma'\sigma}=&\left(J^{(j)}_{23}\pm iJ^{(j)}_{31}\right)_{\sigma'\sigma}=\left(J^{(j)}_{1}\pm iJ^{(j)}_{2}\right)_{\sigma'\sigma}=\delta_{\sigma'\sigma+1}\sqrt{(j\mp \sigma)(j\pm \sigma+1)}
\end{align*}
である.ここで$\sigma$は$j,j-1,\cdots ,-j$の値をとる.(この表示は,量子力学での角運動量についての議論を思い出せばよい.磁気量子数$m$を,1粒子状態としての離散的添え字として採用するのである.のちに補足として議論しよう.)質量$M>0$でスピン$j$の粒子では,(2.5.11)は
\begin{align*}
U(\Lambda)\Psi_{p,\sigma}=&\sqrt{\frac{k^0}{p^0}}\sqrt{\frac{(\Lambda p)^0}{k^0}}\sum_{\sigma'}D^{(j)}_{\sigma'\sigma}\Bigl(W(\Lambda,p)\Bigr)\Psi_{\Lambda p,\sigma'} \\
=&\sqrt{\frac{(\Lambda p)^0}{p^0}}\sum_{\sigma'}D^{(j)}_{\sigma'\sigma}\Bigl(W(\Lambda,p)\Bigr)\Psi_{\Lambda p,\sigma'}
\end{align*}
となる.ここで小群$W(\Lambda,p)$(ウィグナー回転)は(2.5.10)で与えられる.
\begin{align*}
W(\Lambda,p)=L^{-1}(\Lambda p) \Lambda L(p)
\end{align*}
この回転を計算するためには,四元運動量を$k^\mu =(0,0,0,M)$から$p^\mu$に変化させる「基準ブースト」$L(p)$を使う必要がある.これは以下のように便宜的に選ばれる.
\begin{align*}
\tensor{L}{^i_k}(p)=&\delta_{ik}+(\gamma-1)\hat{p}_i \hat{p}_k \\
\tensor{L}{^i_0}(p)=&\tensor{L}{^0_i}=\hat{p}_i \sqrt{\gamma^2-1} \\
\tensor{L}{^0_0}(p)=&\gamma \\
L(p)=&\left(
\begin{matrix}
1+(\gamma-1)\hat{\mathbf{p}}\hat{\mathbf{p}}^T & \sqrt{\gamma^2-1}\hat{\mathbf{p}} \\
\sqrt{\gamma^2-1}\hat{\mathbf{p}}^T & \gamma
\end{matrix}
\right)
\end{align*}
ここで以下の定義を用いた.
\begin{align*}
\hat{p}_i \equiv p_i /|\mathbf{p}|, \quad \gamma\equiv \sqrt{\mathbf{p}^2+M^2}/M
\end{align*}
($\gamma$は相対論でよく使われる$1/\sqrt{1-\mathbf{v}^2}$と同じものである.実際$p^0=\sqrt{\mathbf{p}^2+M^2}$である一方,固有時の定義より
\begin{align*}
d\tau =\sqrt{dt^2-dx^2-dy^2-dz^2}=dt\sqrt{1-(d\mathbf{x}/dt)^2} =dt/\gamma
\end{align*}
で
\begin{align*}
p^0=M\frac{dx^0}{d\tau}=\gamma M \frac{dt}{dt}=\gamma M
\end{align*}
であることからすぐ$\gamma=p^0/M=\sqrt{\mathbf{p}^2+M^2}/M$がわかる.)この基準ブーストが,$\tensor{L}{^\mu_\nu}(p)k^\nu=p^\nu$を満たすことと,ローレンツ変換(2.3.5)を満たすことを確認する.$k^\mu=(0,0,0,M)$より
\begin{align*}
\tensor{L}{^0_\nu}k^\nu=&\tensor{L}{^0_0}(p)M=\frac{\sqrt{\mathbf{p}^2+M^2}}{M} M=\sqrt{\mathbf{p}^2+M^2}=p^0 \\
\tensor{L}{^i_\nu}k^\nu=&\tensor{L}{^i_0}(p)M=\hat{p}_i \sqrt{\gamma^2-1} M=\hat{p}_i \sqrt{\frac{\mathbf{p}^2+M^2}{M^2}-1}M =\hat{p}_i\sqrt{\mathbf{p}^2}=p_i
\end{align*}
となり前者が確認できる.次に(2.3.5)は
\begin{align*}
\eta_{\mu\nu}\tensor{L}{^\mu_0}(p)\tensor{L}{^\nu_0}(p)=&-(\tensor{L}{^0_0}(p))^2+(\tensor{L}{^i_0}(p))^2 \\
=&-\gamma^2 +\hat{p}_i^2 (\gamma^2-1)=-\gamma^2 +(\gamma^2-1) \\
=& -1 =\eta_{00} \\
\eta_{\mu\nu}\tensor{L}{^\mu_i}(p)\tensor{L}{^\nu_0}(p)=&-\tensor{L}{^0_i}(p)\tensor{L}{^0_0}(p)+\tensor{L}{^k_i}(p)\tensor{L}{^k_0}(p) \\
=&-\hat{p}_i \sqrt{\gamma^2-1}\gamma +(\delta_{ki}+(\gamma-1)\hat{p}_k\hat{p}_i)(\hat{p}_k \sqrt{\gamma^2-1}) \\
=&-\hat{p}_i \gamma \sqrt{\gamma^2-1}+\hat{p}_i \sqrt{\gamma^2-1}+\hat{p}_i(\gamma-1)\sqrt{\gamma^2-1} \\
=&0=\eta_{i0} \\
\eta_{\mu\nu}\tensor{L}{^\mu_i}(p)\tensor{L}{^\nu_j}(p) =&-\tensor{L}{^0_i}\tensor{L}{^0_j}+\tensor{L}{^k_i}(p)\tensor{L}{^k_j}(p) \\
=&-(\hat{p}_i\sqrt{\gamma^2-1}))(\hat{p}_j\sqrt{\gamma^2-1})+(\delta_{ki}+(\gamma-1)\hat{p}_k \hat{p}_i)(\delta_{kj}+(\gamma-1)\hat{p}_k \hat{p}_j) \\
=&-\hat{p}_i \hat{p}_j(\gamma^2-1)+\delta_{ij}+2\hat{p}_i \hat{p}_j (\gamma-1)+\hat{p}_i\hat{p}_j(\gamma-1)^2 \\
=&-\hat{p}_i \hat{p}_j(\gamma^2-1)+\delta_{ij}+2\hat{p}_i \hat{p}_j (\gamma-1)+\hat{p}_i\hat{p}_j(\gamma^2-2\gamma+1) \\
=&\delta_{ij}=\eta_{ij}
\end{align*}
となり,たしかに$\tensor{L}{^\mu_\nu}(p)$はローレンツ変換である.(あとから考えれば行列形式の方がわかりやすかった.この場合
\begin{align*}
&L(p)^T \eta L(p)\\
=&\left(
\begin{matrix}
1+(\gamma-1)\hat{\mathbf{p}}\hat{\mathbf{p}}^T & \sqrt{\gamma^2-1}\hat{\mathbf{p}} \\
\sqrt{\gamma^2-1}\hat{\mathbf{p}}^T & \gamma
\end{matrix}
\right) \left(
\begin{matrix}
1 & 0 \\
0 & -1
\end{matrix}
\right)\left(
\begin{matrix}
1+(\gamma-1)\hat{\mathbf{p}}\hat{\mathbf{p}}^T & \sqrt{\gamma^2-1}\hat{\mathbf{p}} \\
\sqrt{\gamma^2-1}\hat{\mathbf{p}}^T & \gamma
\end{matrix}
\right) \\
=&\left(
\begin{matrix}
1+(\gamma-1)\hat{\mathbf{p}}\hat{\mathbf{p}}^T & -\sqrt{\gamma^2-1}\hat{\mathbf{p}} \\
\sqrt{\gamma^2-1}\hat{\mathbf{p}}^T & -\gamma
\end{matrix}
\right)\left(
\begin{matrix}
1+(\gamma-1)\hat{\mathbf{p}}\hat{\mathbf{p}}^T & \sqrt{\gamma^2-1}\hat{\mathbf{p}} \\
\sqrt{\gamma^2-1}\hat{\mathbf{p}}^T & \gamma
\end{matrix}
\right) \\
=&\left(
\begin{matrix}
[1+(\gamma-1)\hat{\mathbf{p}}\hat{\mathbf{p}}^T][1+(\gamma-1)\hat{\mathbf{p}}\hat{\mathbf{p}}^T]-(\gamma^2-1)\hat{\mathbf{p}}\hat{\mathbf{p}}^T & [1+(\gamma-1)\hat{\mathbf{p}}\hat{\mathbf{p}}^T]\sqrt{\gamma^2-1}\hat{\mathbf{p}}-\gamma \sqrt{\gamma^2-1}\hat{\mathbf{p}} \\
\sqrt{\gamma^2-1}\hat{\mathbf{p}}^T[1+(\gamma-1)\hat{\mathbf{p}}\hat{\mathbf{p}}^T]-\gamma \sqrt{\gamma^2-1}\hat{\mathbf{p}} & (\gamma^2-1)\hat{\mathbf{p}}^T \hat{\mathbf{p}}-\gamma^2
\end{matrix}
\right) \\
=&\left(
\begin{matrix}
1 & 0 \\
0 & -1
\end{matrix}
\right)=\eta
\end{align*}
となる.最後の行は$\hat{\mathbf{p}}^T\hat{\mathbf{p}}=1$を使えば示せる.)さらに行列式が$+1$,つまり固有順時ローレンツ変換$SO(3,1)$の元であることも確認する.このためには,ブロック行列に関する行列式の公式を使うのが良い.すなわち,行列$M$が部分行列$A,B,C,D$によって
\begin{align*}
M=\left(
\begin{matrix}
A & B \\
C & D
\end{matrix}
\right)
\end{align*}
という形になっているならば
\begin{align*}
\det M=& \det A \det\left(D-CA^{-1}B\right) \quad (A が正則) \\
=&\det D \det\left(A-BD^{-1}C\right) \quad (D が正則)
\end{align*}
となる,というものである.これは簡単に証明できる.一つ目の式を示すには
\begin{align*}
M=\left(
\begin{matrix}
A & B \\
C & D
\end{matrix}
\right) =\left(
\begin{matrix}
I & O \\
CA^{-1} & I
\end{matrix}
\right)\left(
\begin{matrix}
A & O \\
O & D-CA^{-1}B
\end{matrix}
\right)\left(
\begin{matrix}
I & A^{-1}B \\
O & I
\end{matrix}
\right)
\end{align*}
とすればいい.(右辺から逆算すれば左辺が出てくる.)両辺の$\det$をとると,右辺の1つめと3つ目の行列式が$1$であることと,行列の積の$\det$は$\det$の積に分解できることを使えばすぐ示せる.二つ目の式も同様である.今回は二つ目の式を使い,$A=1+(\gamma-1)\hat{\mathbf{p}}\hat{\mathbf{p}}^T,B=\sqrt{\gamma^2-1}\hat{\mathbf{p}}^T ,C=\sqrt{\gamma^2-1}\hat{\mathbf{p}} ,D=\gamma$とすれば
\begin{align*}
A-BD^{-1}C=&[1+(\gamma-1)\hat{\mathbf{p}}\hat{\mathbf{p}}^T]-\frac{1}{\gamma}(\gamma^2-1)\hat{\mathbf{p}} \hat{\mathbf{p}}^T \\
=&1+\frac{1-\gamma}{\gamma}\hat{\mathbf{p}}\hat{\mathbf{p}}^T
\end{align*}
であり,
\begin{align*}
\det L(p)=\gamma \det\left(1+\frac{1-\gamma}{\gamma}\hat{\mathbf{p}}\hat{\mathbf{p}}^T\right)
\end{align*}
となる.さらに計算を進めるためには
\begin{align*}
\det(I+\mathbf{u}\mathbf{v}^T)=1+\mathbf{v}^T \mathbf{u}
\end{align*}
を使う.これは
\begin{align*}
\left(
\begin{matrix}
I & 0 \\
\mathbf{v}^T & 1
\end{matrix}
\right)\left(
\begin{matrix}
I+\mathbf{u}\mathbf{v}^T & \mathbf{u} \\
0 & 1
\end{matrix}
\right)\left(
\begin{matrix}
I & 0 \\
-\mathbf{v}^T & 1
\end{matrix}
\right)=\left(
\begin{matrix}
I & \mathbf{u}\\
0 & 1+\mathbf{v}^T\mathbf{u}
\end{matrix}
\right)
\end{align*}
であることから両辺detをとって示せる(これはmatrix determinant lemmaという名前で知られているらしい).これを使えば
\begin{align*}
\det L(p)=&\gamma \det\left(I+\frac{1-\gamma}{\gamma}\hat{\mathbf{p}}\hat{\mathbf{p}}^T\right) \\
=&\gamma\left(1+\frac{1-\gamma}{\gamma}\hat{\mathbf{p}}^T\hat{\mathbf{p}}\right) \\
=&1
\end{align*}
となる.$\tensor{L(p)}{^0_0}=\gamma>+1$は明らかだから,以上から$L(p)$が$SO(3,1)$の元であることがわかる.\par
$\tensor{\Lambda}{^\mu_\nu}$が任意の三次元回転$\mathcal{R}$であるとき,全ての$p$についてウィグナー回転$W(\Lambda,p)$が$\mathcal{R}$と同じであることを以下で示す.このために,ブースト(2.5.24)が以下のように表せることを使う.
\begin{align*}
L(p)=R(\mathbf{p})B(|\mathbf{p}|)R^{-1}(\mathbf{p})
\end{align*}
理由を以下で説明する.ここで$R(\mathbf{p})$は3軸(運動量ベクトルの$z$方向の成分という意)を$\mathbf{p}$の方向に向ける三次元回転で,$B(|\mathbf{p}|)$は
\begin{align*}
B(|\mathbf{p}|)=\left(
\begin{matrix}
1 & 0 & 0 & 0 \\
0 & 1 & 0 & 0 \\
0 & 0 & \gamma & \sqrt{\gamma^2-1} \\
0 & 0 & \sqrt{\gamma^2-1} & \gamma
\end{matrix}
\right)
\end{align*}
これは(2.5.24)で$\hat{p}_1=\hat{p}_2=0,\hat{p}_3=1$とおいたものだ.つまり$B$は$k^\mu=(0,0,0,M)$を「空間成分が3軸方向に向かっている$p^\mu=(0,0,\mathbf{p},p^0)$」にするローレンツ変換だ.よって
\begin{align*}
\tensor{\Bigl(R(\mathbf{p})B(|\mathbf{p}|)R^{-1}(\mathbf{p})\Bigr)}{^\mu_\nu}k^\nu=&R(\mathbf{p})B(|\mathbf{p}|)R^{-1}(\mathbf{p})\left(
\begin{matrix}
0 \\
0 \\
0 \\
M
\end{matrix}
\right) \\
=&R(\mathbf{p})B(|\mathbf{p}|)\left(
\begin{matrix}
0 \\
0 \\
0 \\
M
\end{matrix}
\right) \quad \because(空間成分がゼロベクトルなので回転不変) \\
=&R(\mathbf{p})\left(
\begin{matrix}
0 \\
0 \\
\sqrt{\gamma^2-1} \\
\gamma M
\end{matrix}
\right)=R(\mathbf{p})\left(
\begin{matrix}
0 \\
0 \\
|\mathbf{p}| \\
p^0
\end{matrix}
\right) \\
=&\left(
\begin{matrix}
p^1 \\
p^2 \\
p^3 \\
p^0
\end{matrix}
\right) \quad \because(R(\mathbf{p}) は3軸を \mathbf{p} に向ける回転) \\
=& p^\nu
\end{align*}
となる.たしかに$L(p)$となっていることがわかる.\par
こうすると,任意の回転$\mathcal{R}=\Lambda$に対して
\begin{align*}
W(\mathcal{R},p)=&L^{-1}(\Lambda p)\Lambda L(p)=L^{-1}(\mathcal{R}p)\mathcal{R} L(p) \\
=&\Bigl(R(\mathcal{R} p)B^{-1}(|\mathbf{p}|)R^{-1}(\mathcal{R} p) \Bigr) \mathcal{R} \Bigl( R(p)B(|\mathbf{p}|)R^{-1}( p) \Bigr) \\
=&R(\mathcal{R} p)B^{-1}(|\mathbf{p}|) \Bigl(R^{-1}(\mathcal{R} p) \mathcal{R}R(p)\Bigr) B(|\mathbf{p}|)R^{-1}( p)
\end{align*}
となる.回転$R^{-1}(\mathcal{R} p) \mathcal{R}R(p)$は,3軸を$\mathbf{p}$の方向に,次に$\mathcal{R} p$に回転し,そして次に元の3軸に回すので,3軸周りのある角度$\theta$の回転だとわかる.つまり
\begin{align*}
R^{-1}(\mathcal{R} p) \mathcal{R}R(p)=R(\theta)=\left(
\begin{matrix}
\cos\theta & \sin\theta & 0 & 0 \\
-\sin \theta & \cos\theta & 0 & 0 \\
0 & 0 & 1 & 0 \\
0 & 0 & 0 & 1
\end{matrix}
\right)
\end{align*}
と書ける.これは明らかに$B(|\mathbf{p}|)$と可換だ.わかりにくければ
\begin{align*}
B(|\mathbf{p}|)=\left(
\begin{matrix}
I & 0 \\
0 & A
\end{matrix}
\right) ,\quad R(\theta)=\left(
\begin{matrix}
B & 0 \\
0 & I 
\end{matrix}
\right)
\end{align*}
と書いてみればいい.$R(\theta)$と$B(|\mathbf{p}|)$は可換であることから
\begin{align*}
W(\mathcal{R},p)=&R(\mathcal{R} p)B^{-1}(|\mathbf{p}|) R(\theta) B(|\mathbf{p}|)R^{-1}( p) \\
=&R(\mathcal{R} p)B^{-1}(|\mathbf{p}|) B(|\mathbf{p}|)R(\theta ) R^{-1}( p) \\
=&R(\mathcal{R} p)R(\theta ) R^{-1}( p) \\
=&R(\mathcal{R} p)R^{-1}(\mathcal{R} p) \mathcal{R}R(p) R^{-1}( p) \\
=&\mathcal{R}
\end{align*}
となる.これが示したかったことだ.つまり,動いている質量がゼロでない粒子の状態(さらに拡張して,多粒子状態)は回転のもとで,非相対論の場合と同じように変換する.
\begin{align*}
U(\mathcal{R})\Psi_{p,\sigma}=\sum_{\sigma'}D^{(j)}_{\sigma'\sigma}(\mathcal{R})\Psi_{\mathcal{R} p,\sigma'}
\end{align*}
($\mathcal{R}$は三次元的回転で時間成分を不変にするので,$(\mathcal{R}p)^0=p^0$であり係数が1になる.)これは一見当たり前のように見えるが,$U(\mathcal{R})$はもともとローレンツ変換の対称性変換演算子であって,量子力学での回転の演算子とは別だ.ローレンツ変換を三次元回転に制限した結果,それらがたまたま全く同じ変換性を示すのだ.三次元的な回転に関する考察は量子力学でやったように,球面調和関数やクレブシュ・ゴルダン係数などの道具がすでに揃っていたのだった.すなわちこれは,非相対論的量子力学から相対論的量子力学にそれらの道具がそのまま持ってこれることを意味する.

\vskip\baselineskip

(c)ゼロ質量\par
この場合の小群はどのような構造を持つかを調べなければならない.$k^\mu$をこの場合の基準となる四元運動量$k^\mu=(0,0,1,1)$として($\kappa=1$),$\tensor{W}{^\mu_\nu}k^\nu=k^\mu$を満たす小群の元$\tensor{W}{^\mu_\nu}$を考える.そのようなローレンツ変換が時間的四元ベクトル$t^\mu=(0,0,0,1)$に働いてできた四元ベクトル$Wt$は,その長さおよび$Wk=k$との内積が$t$と同じだ.実際(2.3.5)より
\begin{align*}
&\eta_{\mu\nu}\tensor{W}{^\mu_\rho} t^\rho \tensor{W}{^\nu_\sigma}t^\sigma=\eta_{\rho\sigma}t^\rho t^\sigma \quad \therefore (Wt)_\mu (Wt)^\mu=t_\mu t^\mu =-1 \\
&\eta_{\mu\nu}\tensor{W}{^\mu_\rho} t^\rho \tensor{W}{^\nu_\sigma}k^\sigma=\eta_{\mu\nu}\tensor{W}{^\mu_\rho} t^\rho k^\nu=\eta_{\rho\sigma}t^\rho k^\sigma \quad \therefore (Wt)^\mu k_\mu=t^\mu k_\mu =-1
\end{align*}
となる.第二の条件を満たす任意の四元ベクトルは以下のように書ける.
\begin{align*}
(Wt)^\mu=(\alpha,\beta,\zeta,1+\zeta)
\end{align*}
代入してみればすぐわかる.そして第一の条件より
\begin{align*}
-1=& (Wt)_\mu (Wt)^\mu \\
=&\alpha^2+\beta^2+\zeta^2-(1+\zeta)^2 \\
=&\alpha^2+\beta^2-2\zeta-1 \\
\quad \therefore \zeta=&\frac{\alpha^2+\beta^2}{2}
\end{align*}
が得られる.したがって,$\tensor{W}{^\mu_\nu}$の$t^\nu$への影響は,ローレンツ変換
\begin{align*}
\tensor{S}{^\mu_\nu}(\alpha,\beta)=\left(
\begin{matrix}
1 & 0 & -\alpha & \alpha \\
0 & 1 & -\beta & \beta \\
\alpha & \beta & 1-\zeta & \zeta \\
\alpha & \beta &-\zeta & 1+\zeta
\end{matrix}
\right)
\end{align*}
と同じだ.実際$\tensor{S}{^\mu_\nu}t^\nu=(\alpha,\beta,\zeta,1+\zeta)=\tensor{W}{^\mu_\nu}t^\nu$だ.$\tensor{S}{^\mu_\nu}$の左1列以外の要素は,これがローレンツ変換であるための条件(2.3.5)を満たすようになっている.以下でこれがローレンツ変換であること,つまり
\begin{align*}
\eta_{\mu\nu}\tensor{S}{^\mu_\rho}\tensor{S}{^\nu_\sigma}=\eta_{\rho\sigma}
\end{align*}
を確かめる.これは行列表示で$S^T \eta S =\eta $を示すのが一番楽だと思う.
\begin{align*}
&S^T \eta S \\ 
=&\left(
\begin{matrix}
1 & 0 & \alpha & \alpha \\
0 & 1 & \beta & \beta \\
-\alpha & -\beta & 1-\zeta & -\zeta \\
\alpha & \beta & \zeta & 1+\zeta
\end{matrix}
\right) \left(
\begin{matrix}
1 & 0 & 0 & 0 \\
0 & 1 & 0 & 0 \\
0 & 0 & 1 & 0 \\
0 & 0 & 0 & -1
\end{matrix}
\right)\left(
\begin{matrix}
1 & 0 & -\alpha & \alpha \\
0 & 1 & -\beta & \beta \\
\alpha & \beta & 1-\zeta & \zeta \\
\alpha & \beta & -\zeta & 1+\zeta
\end{matrix}
\right) \\
=&\left(
\begin{matrix}
1 & 0 & \alpha & \alpha \\
0 & 1 & \beta & \beta \\
-\alpha & -\beta & 1-\zeta & -\zeta \\
\alpha & \beta & \zeta & 1+\zeta
\end{matrix}
\right) \left(
\begin{matrix}
1 & 0 & -\alpha & \alpha \\
0 & 1 & -\beta & \beta \\
\alpha & \beta & 1-\zeta & \zeta \\
-\alpha & -\beta & \zeta & -(1+\zeta)
\end{matrix}
\right) \\
=&\left(
\begin{matrix}
1+\alpha^2 -\alpha^2 & \alpha \beta -\alpha\beta & -\alpha +\alpha(1-\zeta)+\alpha\zeta &\alpha +\alpha\zeta -\alpha(1+\zeta) \\
\alpha\beta-\alpha\beta & \beta^2-\beta^2 & -\beta +\beta(1-\zeta)+\beta\zeta & \beta+\beta \zeta -\beta(1+\zeta) \\
-\alpha +\alpha(1-\zeta)+\alpha\zeta & -\beta +\beta(1-\zeta)+\beta\zeta & \alpha^2+\beta^2+(1-\zeta)^2-\zeta^2& -\alpha^2-\beta^2+2\zeta \\
\alpha+\alpha\zeta-\alpha(1+\zeta)& \beta+\beta\zeta-\beta(1+\zeta)& -\alpha^2-\beta^2+2\zeta& \alpha^2+\beta^2 +\zeta^2 -(1+\zeta)^2
\end{matrix}
\right) \\
=&\left(
\begin{matrix}
1 & 0 & 0 & 0 \\
0 & 1 & 0 & 0 \\
0 & 0 & 1 & 0 \\
0 & 0 & 0 & -1
\end{matrix}
\right) \quad \because \zeta=\frac{\alpha^2+\beta^2}{2} \\
=&\eta
\end{align*}
よって$S$はローレンツ変換だ.さて,$\tensor{S}{^\mu_\nu}t^\nu=(\alpha,\beta,\zeta,1+\zeta)=\tensor{W}{^\mu_\nu}t^\nu$は$W$が$S(\alpha,\beta)$に等しいことを意味するのではない.左から$S^{-1}$をかけて
\begin{align*}
\tensor{(S^{-1})}{^\mu_\nu} \tensor{W}{^\nu_\sigma}t^\sigma =t^\nu
\end{align*}
よって$S^{-1}(\alpha,\beta){W}$は$(0,0,0,1)$を不変にするローレンツ変換であり,空間成分の純粋な回転であることがわかる.一方
\begin{align*}
\tensor{W}{^\mu_\nu}k^\nu =\tensor{S}{^\mu_\nu}k^\nu=k^\mu
\end{align*}
も直接計算で確かめられ,よって$\tensor{S}{^\mu_\nu}$は$\tensor{W}{^\mu_\nu}$のように光円錐的四元ベクトル$(0,0,1,1)$を不変にする.したがって$S^{-1}(\alpha,\beta)W$は3軸を不変にする空間回転(つまり3軸周りの$\theta$回転)
\begin{align*}
S^{-1}(\alpha,\beta)W=R(\theta)
\end{align*}
である.ここで$R(\theta)$は3軸周りの$\theta$回転
\begin{align*}
R(\theta)=\left(
\begin{matrix}
\cos\theta & \sin\theta & 0 & 0 \\
-\sin\theta & \cos\theta & 0 & 0 \\
0 & 0 & 1 & 0 \\
0 & 0 & 0 & 1
\end{matrix}
\right)
\end{align*}
したがって,小群の最も一般的な弦は,次の形だ.
\begin{align*}
W(\theta,\alpha,\beta)=S(\alpha,\beta)R(\theta)
\end{align*}
これは何の群だろうか?$\theta=0$か$\alpha=\beta=0$のときの変換は部分群をなすことを見る.
\begin{align*}
S(\bar{\alpha},\bar{\beta})S(\alpha,\beta)=&\left(
\begin{matrix}
1 & 0 & -\bar{\alpha} & \bar{\alpha} \\
0 & 1 & -\bar{\beta} & \bar{\beta} \\
\bar{\alpha} & \bar{\beta} & 1-\bar{\zeta} & \bar{\zeta} \\
\bar{\alpha} & \bar{\beta} & -\bar{\zeta} & 1+\bar{\zeta}
\end{matrix}
\right)\left(
\begin{matrix}
1 & 0 & -\alpha & \alpha \\
0 & 1 & -\beta & \beta \\
\alpha & \beta & 1-\zeta & \zeta \\
\alpha & \beta & -\zeta & 1+\zeta
\end{matrix}
\right) \\
=&\left(
\begin{matrix}
I & \bar{B} \\
\bar{A} & \bar{C}
\end{matrix}
\right)\left(
\begin{matrix}
I & B \\
A & C
\end{matrix}
\right) \\
=&\left(
\begin{matrix}
I+\bar{B}A & B+\bar{B}C \\
\bar{A}+\bar{C}A & \bar{A}B+\bar{C}C
\end{matrix}
\right) \\
\bar{B}A=& \left(
\begin{matrix}
-\bar{\alpha} & \bar{\alpha} \\
-\bar{\beta} & \bar{\beta}
\end{matrix}
\right) \left(
\begin{matrix}
\alpha & \beta \\
\alpha & \beta
\end{matrix}
\right)=\left(
\begin{matrix}
-\bar{\alpha}\alpha+\bar{\alpha}\alpha & -\bar{\alpha}\beta+\bar{\alpha}\beta \\
-\bar{\beta}\alpha+\bar{\beta}\alpha & -\bar{\beta}\beta +\bar{\beta}\beta 
\end{matrix}
\right)=\left(
\begin{matrix}
0 & 0 \\
0 & 0
\end{matrix}
\right) \\
\bar{B}C=& \left(
\begin{matrix}
-\bar{\alpha} & \bar{\alpha} \\
-\bar{\beta} & \bar{\beta}
\end{matrix}
\right)\left(
\begin{matrix}
1-\zeta & \zeta \\
-\zeta & 1+\zeta
\end{matrix}
\right)=\left(
\begin{matrix}
-\bar{\alpha}(1-\zeta)-\bar{\alpha}\zeta & -\bar{\alpha} \zeta +\bar{\alpha}(1+\zeta) \\
-\bar{\beta}(1-\zeta )-\bar{\beta}\zeta & -\beta \zeta +\bar{\beta}(1+\zeta)
\end{matrix}
\right) \\
=& \left(
\begin{matrix}
-\bar{\alpha} & \bar{\alpha} \\
-\bar{\beta} & \bar{\beta}
\end{matrix}
\right) \\
\bar{C}A=&\left(
\begin{matrix}
1-\bar{\zeta} & \bar{\zeta} \\
-\bar{\zeta} & 1+\bar{\zeta}
\end{matrix}
\right)\left(
\begin{matrix}
\alpha & \beta \\
\alpha & \beta
\end{matrix}
\right)=\left(
\begin{matrix}
\alpha(1-\bar{\zeta})+\alpha\bar{\zeta} & \beta(1-\bar{\zeta})+\beta\bar{\zeta} \\
-\alpha\bar{\zeta}+\alpha(1+\bar{\zeta}) & -\beta\bar{\zeta}+\beta(1+\bar{\zeta})
\end{matrix}
\right) \\
=&\left(
\begin{matrix}
\alpha & \beta \\
\alpha & \beta
\end{matrix}
\right) \\
\bar{A}B=& \left(
\begin{matrix}
\bar{\alpha} & \bar{\beta} \\
\bar{\alpha} & \bar{\beta}
\end{matrix}
\right)\left(
\begin{matrix}
-\alpha & \alpha \\
-\beta & \beta
\end{matrix}
\right)=\left(
\begin{matrix}
-\bar{\alpha}\alpha-\bar{\beta}\beta & \bar{\alpha}\alpha +\bar{\beta}\beta \\
-\bar{\alpha}\alpha-\bar{\beta}\beta & \bar{\alpha}\alpha+\bar{\beta}\beta
\end{matrix}
\right) \\
\bar{C}C=&\left(
\begin{matrix}
1-\bar{\zeta} & \bar{\zeta} \\
-\bar{\zeta} & 1+\bar{\zeta}
\end{matrix}
\right) \left(
\begin{matrix}
1-\zeta & \zeta \\
-\zeta & 1+\zeta
\end{matrix}
\right) =\left(
\begin{matrix}
(1-\bar{\zeta})(1-\zeta)-\bar{\zeta}\zeta & (1-\bar{\zeta})\zeta +(1+\zeta)\bar{\zeta} \\
-\bar{\zeta}(1-\zeta)-\zeta(1+\bar{\zeta}) & -\zeta\bar{\zeta} +(1+\bar{\zeta})(1+\zeta)
\end{matrix}
\right) \\
=&\left(
\begin{matrix}
1-\zeta-\bar{\zeta} & \zeta+\bar{\zeta} \\
-\zeta-\bar{\zeta} & 1+\zeta+\bar{\zeta}
\end{matrix}
\right) \\
\bar{A}B+\bar{C}C=&\left(
\begin{matrix}
1-\frac{\alpha^2+2\bar{\alpha}\alpha+\bar{\alpha}^2}{2} - \frac{\beta^2+2\beta\bar{\beta}+\bar{\beta}^2}{2} & \frac{\alpha^2+2\bar{\alpha}\alpha+\bar{\alpha}^2}{2} +\frac{\beta^2+2\bar{\beta}\beta+\bar{\beta}^2}{2} \\
-\frac{\alpha^2+2\bar{\alpha}\alpha+\bar{\alpha}^2}{2} -\frac{\beta^2+2\bar{\beta}\beta+\bar{\beta}^2}{2} & 1+\frac{\alpha^2+2\bar{\alpha}\alpha +\bar{\alpha}^2}{2} +\frac{\beta^2+2\bar{\beta}\beta+\bar{\beta}^2}{2}
\end{matrix}
\right) \\
=&\left(
\begin{matrix}
1-\frac{(\bar{\alpha}+\alpha)^2+(\bar{\beta}+\beta)^2}{2} & \frac{(\bar{\alpha}+\alpha)^2+(\bar{\beta}+\beta)^2}{2} \\
-\frac{(\bar{\alpha}+\alpha)^2+(\bar{\beta}+\beta)^2}{2} & 1+\frac{(\bar{\alpha}+\alpha)^2+(\bar{\beta}+\beta)^2}{2}
\end{matrix}
\right) \\
=&\left(
\begin{matrix}
1-\zeta(\bar{\alpha}+\alpha,\bar{\beta}+\beta) & \zeta(\bar{\alpha}+\alpha,\bar{\beta}+\beta) \\
-\zeta(\bar{\alpha}+\alpha,\bar{\beta}+\beta) & 1+\zeta(\bar{\alpha}+\alpha,\bar{\beta}+\beta)
\end{matrix}
\right) \\
\therefore S(\bar{\alpha},\bar{\beta})S(\alpha,\beta)=&\left(
\begin{matrix}
1 & 0 & -(\alpha+\bar{\alpha}) & \alpha+\bar{\alpha} \\
0 & 1 & -(\beta+\bar{\beta}) & \beta+\bar{\beta} \\
\alpha +\bar{\alpha} & \beta+\bar{\beta} & 1-\zeta(\bar{\alpha}+\alpha,\bar{\beta}+\beta) & \zeta(\bar{\alpha}+\alpha,\bar{\beta}+\beta) \\
\alpha+\bar{\alpha} & \beta+\bar{\beta} & -\zeta(\bar{\alpha}+\alpha,\bar{\beta}+\beta) & 1+\zeta(\bar{\alpha}+\alpha,\bar{\beta}+\beta)
\end{matrix}
\right) \\
=& S(\alpha+\bar{\alpha},\beta+\bar{\beta})
\end{align*}
また$R$は回転行列であるから積は自明で,以上より
\begin{align*}
S(\bar{\alpha},\bar{\beta})S(\alpha,\beta)=S(\bar{\alpha}+\alpha,\bar{\beta}+\beta),\quad R(\bar{\theta})R(\theta)=R(\bar{\theta}+\theta )
\end{align*}
となる.これらは加法的なので互いに可換な,小群の部分群であることがわかる.さらに$\theta=0$の$S$部分群に対して,次の積を計算してみよう.
\begin{align*}
R(\theta)S(\alpha,\beta)R^{-1}(\theta)=&\left(
\begin{matrix}
\cos\theta & \sin\theta& 0 & 0 \\
-\sin\theta & \cos\theta & 0 & 0 \\
0 & 0 & 1 & 0 \\
0 & 0 & 0 & 1
\end{matrix}
\right)\left(
\begin{matrix}
1 & 0 & -\alpha & \alpha \\
0 & 1 & -\beta & \beta \\
\alpha & \beta & 1-\zeta & \zeta \\
\alpha & \beta & -\zeta & 1+\zeta
\end{matrix}
\right)\left(
\begin{matrix}
\cos\theta & -\sin\theta& 0 & 0 \\
\sin\theta & \cos\theta & 0 & 0 \\
0 & 0 & 1 & 0 \\
0 & 0 & 0 & 1
\end{matrix}
\right) \\
=& \left(
\begin{matrix}
R & 0 \\
0 & I
\end{matrix}
\right)\left(
\begin{matrix}
I & B \\
A & C
\end{matrix}
\right)\left(
\begin{matrix}
R^{-1} & 0 \\
0 & I
\end{matrix}
\right)=\left(
\begin{matrix}
R & RB \\
A & C
\end{matrix}
\right)\left(
\begin{matrix}
R^{-1} & 0 \\
0 & I
\end{matrix}
\right) \\
=&\left(
\begin{matrix}
I & RB \\
AR^{-1} & C
\end{matrix}
\right) \\
RB=&\left(
\begin{matrix}
\cos\theta &\sin\theta \\
-\sin\theta &\cos\theta
\end{matrix}
\right) \left(
\begin{matrix}
-\alpha & \alpha \\
-\beta & \beta
\end{matrix}
\right)=\left(
\begin{matrix}
-\alpha \cos\theta- \beta\sin\theta & \alpha \cos\theta +\beta\sin\theta \\
\alpha \sin\theta -\beta \cos\theta & -\alpha \sin\theta+\beta \cos\theta
\end{matrix}
\right) \\
AR^{-1}=&\left(
\begin{matrix}
\alpha & \alpha \\
\beta & \beta
\end{matrix}
\right) \left(
\begin{matrix}
\cos\theta & -\sin\theta \\
\sin\theta & \cos\theta 
\end{matrix}
\right) =\left(
\begin{matrix}
\alpha \cos\theta +\beta \sin\theta & -\alpha \sin\theta +\beta\cos\theta \\
\alpha\cos\theta +\beta \sin\theta & -\alpha \sin\theta +\beta \cos\theta
\end{matrix}
\right)
\end{align*}
ここで
\begin{align*}
\frac{(\alpha\cos\theta+\beta\sin\theta)^2+(-\alpha\sin\theta+\beta\cos\theta)^2}{2}=&\frac{\alpha^2+\beta^2}{2} \\
\therefore\quad  \zeta' \equiv \zeta(\alpha\cos\theta+\beta\sin\theta,-\alpha\sin\theta +\beta\cos\theta)=&\zeta(\alpha,\beta)
\end{align*}
より,$\alpha'\equiv =\alpha\cos\theta+\beta\sin\theta,\beta' \equiv-\alpha\sin\theta+\beta\cos\theta$とおくと
\begin{align*}
R(\theta)S(\alpha,\beta)R^{-1}(\theta)=&\left(
\begin{matrix}
1 & 0 & -\alpha' & \alpha' \\
0 & 1 & -\beta' & \beta' \\
\alpha' & \beta' & 1-\zeta' & \zeta' \\
\alpha' & \beta' & -\zeta' & 1+\zeta'
\end{matrix}
\right) \\
=&S(\alpha\cos\theta+\beta\sin\theta ,-\alpha\sin\theta+\beta\cos\theta)
\end{align*}
これから何が言えるかというと,小群の任意の元を$W(\theta,\alpha,\beta)$として(下での$\alpha',\beta'$は上の記号とは関係ない)
\begin{align*}
W(\theta,\alpha,\beta)S(\alpha',\beta')W^{-1}(\theta,\alpha,\beta)=&S(\alpha,\beta)R(\theta)S(\alpha',\beta')R^{-1}(\theta)S^{-1}(\alpha,\beta) \\
=&S(\alpha,\beta)S(\alpha' \cos\theta +\beta'\sin\theta,-\alpha'\sin\theta+\beta'\cos\theta)S^{-1}(\alpha,\beta) \\
=&S(\alpha' \cos\theta +\beta'\sin\theta,-\alpha'\sin\theta+\beta'\cos\theta)\quad \because(S は互いに可換)
\end{align*}
となる.これは$\theta=0$の$S$部分群に属する.よって$\theta=0$の部分群は,元の群の任意の元$W$に対して$WSW^{-1}$が再び$S$部分群に属するから,$S$部分群は正規部分群だ.平行移動が正規部分群となる群は,ユークリッド運動群$E(n)$と同じで,$S$は二次元平行移動,$R$は(3軸を不変とする)二次元回転,よって小群は二次元ユークリッド運動群$E(2)=ISO(2)\simeq SO(2)\ltimes \mathbb{R}^2$と同定できる.($\ltimes$は半直積.詳しい意味は小林・大島「リー群と表現論」p7など参照.)\par
不変な可換部分群を持たない群は,半単純と呼ばれる.これまでに見たように,小群$ISO(2)$は$S$と$R$の可換な部分群をもつので非斉次ローレンツ群と同じく,半単純ではない.このため事情は若干複雑になる.\par
$ISO(2)$のリー代数を調べる.$\theta,\alpha,\beta$が微小のとき,$\cos\theta\approx 1 ,\sin\theta\approx \theta,\alpha^2=\beta^2=0$であるから
\begin{align*}
W(\theta,\alpha,\beta)=&S(\alpha,\beta)R(\theta) \\
=&\left(
\begin{matrix}
1 & 0 & -\alpha & \alpha \\
0 & 1 & -\beta & \beta \\
\alpha & \beta & 1 & 0 \\
\alpha & \beta & 0 & 1
\end{matrix}
\right) \left(
\begin{matrix}
1 & \theta & 0 & 0 \\
-\theta & 1 & 0 & 0 \\
0 & 0 & 1 & 0 \\
0 & 0 & 0 & 1
\end{matrix}
\right) \\
=&\left(
\begin{matrix}
1 & \theta & -\alpha & \alpha \\
-\theta & 1 & -\beta & \beta \\
\alpha -\beta \theta & \alpha\theta+\beta & 1 & 0 \\
\alpha-\beta\theta & \alpha\theta +\beta & 0 & 1
\end{matrix}
\right) \\
=&\left(
\begin{matrix}
1 & \theta & -\alpha & \alpha \\
-\theta & 1 & -\beta & \beta \\
\alpha & \beta & 1 & 0 \\
\alpha & \beta & 0 & 1
\end{matrix}
\right) \quad \because(\theta,\alpha,\beta は微小量なので二次を無視)
\end{align*}
よって一般的な群の元は
\begin{align*}
\tensor{W}{^\mu_\nu}(\theta,\alpha,\beta)=&\delta^\mu_\nu +\tensor{\omega}{^\mu_\nu} \\
\omega_{\mu\nu}=&\eta_{\mu\rho}\tensor{\omega}{^\rho_\nu}=\left(
\begin{matrix}
1 & 0 & 0 & 0 \\
0 & 1 & 0 & 0 \\
0 & 0 & 1 & 0 \\
0 & 0 & 0 & -1
\end{matrix}
\right)\left(
\begin{matrix}
0 & \theta & -\alpha & \alpha \\
-\theta & 0 & -\beta & \beta \\
\alpha & \beta & 0 & 0 \\
\alpha & \beta & 0 & 0
\end{matrix}
\right) \\
=&\left(
\begin{matrix}
0 & \theta & -\alpha & \alpha \\
-\theta & 0 & -\beta & \beta \\
\alpha & \beta & 0 & 0 \\
-\alpha & -\beta & 0 & 0
\end{matrix}
\right) \\
\omega_{\mu\nu}=& -\omega_{\nu\mu}
\end{align*}
となる.(2.4.3)より,対応するヒルベルト空間の演算子は以下となる.
\begin{align*}
U(W(\theta,\alpha,\beta))=&1+\frac{i}{2}\omega_{\mu\nu}J^{\mu\nu} +\cdots \\
=&1+i\omega_{23}J^{23} +i\omega_{31} J^{31} +i\omega_{12} J^{12} \\
&+i\omega_{10}J^{10} +i\omega_{20}J^{20}+i\omega_{30}J^{30} \\
=&1-i\beta J_1+i\alpha J_2 +i\theta J_3 \\
&-i\alpha K_1 -i\beta K_2 \\
=&1+i\alpha(J_2 -K_1)+i\beta(-J_1-K_2)+i\theta J_3 \\
=&1+i\alpha A+i\beta B+i\theta J_3
\end{align*}
ここで$A,B$はエルミート演算子
\begin{align*}
A=&J^{31}+J^{10}=J_2-K_1 \\
B=&-J^{23}+J^{20}=-J_1-K_2
\end{align*}
であり\footnote{前に書いたように本当は$K_i=J_{i0}$であるから,正しいブースト演算子の符号が変わることに注意.本文の定義は誤植である.},以前と同じように$J_3=J_{12}$だ.(2.4.18)-(2.4.20)より
\begin{align*}
[J_3,A]=&[J_3,J_2]-[J_3,K_1]=-iJ_1-iK_2 \\
=&iB \\
[J_3,B]=&-[J_3,J_1]-[J_3,K_2]=-iJ_2+iK_1 \\
=&-iA \\
[A,B]=&-[J_2,J_1]-[J_2,K_2]+[K_1,J_1]+[K_1,K_2]  = iJ_3 +0+0-iJ_3 \\
=&0
\end{align*}
がわかる.\footnote{この交換関係が何を示しているかすぐにはわからないだろうが,実は$A,B$はそれぞれ$x,y$方向の運動量演算子$P_1,P_2$に対応している.実際(2.4.21)より$[J_3,P_1]=iP_2,[J_3,P_2]=-iP_1$であり,$[P_1,P_2]=0$であるから,$A \leftrightarrow P_1,B \leftrightarrow P_2$がわかる.これよりパラメータ$\alpha,\beta$は$x,y$方向への平行移動パラメータに対応しており,$J_3$は当然3軸周りの回転となり,$ISO(2)$は3軸周りの回転と$x,y$軸方向の平行移動を組み合わせた変換群であることが直感的に読み取れる.ただし本当の並進演算子である運動量演算子とは別物である.}$A,B$は可換なエルミート演算子だから,(非斉次ローレンツ群の運動量演算子のように)状態$\Psi_{k,a,b}$で同時対角化できる.
\begin{align*}
A\Psi_{k,a,b}=a\Psi_{k,a,b},\quad B \Psi_{k,a,b}=b\Psi_{k,a,b}
\end{align*}
(可換かつエルミートな演算子には同時固有状態が存在することは,$A\Psi_{k,a}=a\Psi_{k,a}$がなりたっているとして
\begin{align*}
A(B\Psi_{k,a})=B(A\Psi_{k,a})=a(B\Psi_{k,a})
\end{align*}
となり,固有値$a$で縮退していなければ$B\Psi_{k,a}\propto \Psi_{k,a}$がわかり,その比例定数を$b$とすれば固有値$b$の同時固有状態が見つかることからわかる.縮退していても,同じ固有値をもつ状態の線形結合をとってひとつの状態とすればよい.)(2.5.31)より$S$は$W$の$\theta=0$の場合であるから
\begin{align*}
U(R(\theta)S(\alpha,\beta)R^{-1}(\theta))=&U(R(\theta))\left[1+i\alpha A+i\beta B\right]U^{-1}(R(\theta)) \\
=&1+i\alpha U(R(\theta))A U^{-1}(R(\theta))+i\beta U(R(\theta))B U^{-1}(R(\theta)) \\
=U\Bigl(S(\alpha\cos\theta+\beta\sin\theta,-\alpha\sin\theta+\beta\cos\theta)\Bigr)=&1+i(\alpha\cos\theta+\beta\sin\theta)A+i(-\alpha\theta+\beta\cos\theta)B \\
=&1+i\alpha(A\cos\theta-B\sin\theta)+i\beta(A\sin\theta+B\cos\theta)
\end{align*}
$\alpha,\beta$の係数比較をすれば
\begin{align*}
U[R(\theta)]AU^{-1}[R(\theta)]=&A\cos\theta-B\sin\theta \\
U[R(\theta)]BU^{-1}[R(\theta)]=&A\sin\theta +B\cos\theta
\end{align*}
が導ける.したがって任意の$\theta$について
\begin{align*}
A\Psi^{\theta}_{k,a,b}=&AU^{-1}(R(\theta))\Psi_{k,a,b} \\
=&U^{-1}(R(\theta))\left[A\cos\theta-B\sin\theta\right]\Psi_{k,a,b} \\
=&\left[a\cos\theta-b\sin\theta\right]\Psi_{k,a,b}^{\theta}
\end{align*}
同様に
\begin{align*}
B\Psi^\theta_{k,a,b}=\left[a\sin\theta +b\cos\theta\right]\Psi^\theta_{k,a,b}
\end{align*}
が得られる.ここで
\begin{align*}
\Psi^\theta_{k,a,b}\equiv U^{-1}(R(\theta))\Psi_{k,a,b}
\end{align*}
である.連続パラメータ$\theta$を変化させると,$A,B$の固有値は連続的に変化する.しかし質量ゼロの粒子で$\theta$のような連続的な自由度を持っているものは観測されていない.したがって,上のような連続した状態を避けるために,物理的な状態(いまは$\Psi_{k,\sigma}$とよぶ)は$a=b=0$の固有値をもつ$A,B$の固有状態とする.
\begin{align*}
A\Psi_{k,\sigma}=B\Psi_{k,\sigma}=0
\end{align*}
これらの状態$\Psi_{k,\sigma},\Psi_{k,\sigma'}$は$A,B$により区別することはできないが,(2.5.35)(2.5.36)より
\begin{align*}
A(J_3\Psi_{k,\sigma})=0,\quad B(J_3\Psi_{k,\sigma})=0
\end{align*}
となり,$J_3\Psi_{k,\sigma}$は$A,B$の固有値$a=b=0$の状態$\Psi_{k,\sigma}$の線形結合
\begin{align*}
J_3\Psi_{k,\sigma}=\sum_{\sigma'} c_{\sigma\sigma'}\Psi_{k,\sigma'}
\end{align*}
で与えられることがわかる.$J_3$はエルミート演算子であるから,適当な定数行列を両辺に乗じて
\begin{align*}
J_3 \sum_{\sigma}P_{\alpha \sigma}\Psi_{k,\sigma}=\sum_{\sigma'}P_{\alpha\sigma}c_{\sigma\sigma''}P^{-1}_{\sigma'' \alpha'} P_{\alpha'\sigma'}\Psi_{k,\sigma'}
\end{align*}
この係数行列は対角化$(PcP^{-1})_{\alpha\alpha'}=k^\alpha \delta_{\alpha\alpha'}$できる.その固有値$k^\alpha$を,状態を区別する添え字$\sigma$として再定義し,状態も$(A\Psi)_{k,\alpha}\to \Psi_{k,\sigma}$と再定義して
\begin{align*}
J_3\Psi_{k,\sigma}=\sigma\Psi_{k,\sigma}
\end{align*}
とできる.($J_3$はエルミート演算子だから,固有値$\sigma$は実数.)これでそれぞれの状態の区別ができるようになった.運動量$\mathbf{k}=(0,0,1)$は3軸方向にあり,$J_3$は角運動量演算子の3軸成分であるから,$\sigma$は角運動量の運動方向成分,すなわちヘリシティを表す.\par
以上で,一般的な質量ゼロ粒子の状態のローレンツ変換性が計算できる.2.2節の一般的な議論より,(2.5.32)を有限な$\alpha,\beta$に一般化して
\begin{align*}
U(S(\alpha,\beta))=\exp(i\alpha A+i\beta B)
\end{align*}
を得る.これは(2.5.29)より$S$は加法的な群になることからわかる.また有限の$\theta$で
\begin{align*}
U(R(\theta))=\exp(i\theta J_3)
\end{align*}
となる.これも(2.5.30)から$R$は加法的な群であることからくる.よって小群の任意の元$W$は(2.5.28)の形に書けて
\begin{align*}
U(W)=\exp(i\alpha A+i\beta B)\exp(i\theta J_3)
\end{align*}
よって
\begin{align*}
U(W)\Psi_{k,\sigma}=\exp(i\alpha A +i\beta B)\exp(i\theta J_3)\Psi_{k,\sigma}=\exp(i\theta \sigma)\Psi_{k,\sigma}
\end{align*}
となる.したがって(2.5.8)から
\begin{align*}
D_{\sigma'\sigma}(W)=\exp(i\theta\sigma)\delta_{\sigma'\sigma}
\end{align*}
となる.ここで$\theta$は$W$を(2.5.28)のように表した時の角度だ.任意のヘリシティの質量ゼロ粒子のローレンツ変換性は(2.5.11)(2.5.18)から,以下のようになる.
\begin{align*}
U(\Lambda)\Psi_{p,\sigma}=\sqrt{\frac{(\Lambda p)^0}{p^0} }\exp(i\sigma\theta(\Lambda,p))\Psi_{\Lambda p,\sigma}
\end{align*}
ここで$\theta(\Lambda,p)$は
\begin{align*}
W(\Lambda,p)\equiv L^{-1}(\Lambda p)\Lambda L(p)\equiv S(\alpha(\Lambda,p),\beta(\Lambda,p))R(\theta(\Lambda,p))
\end{align*}
で定義される.\par
ここまでは,質量ゼロ粒子のヘリシティ$\sigma$が任意の実数であることを禁じる理由は見当たらない.2.7節でみるように,$\sigma$を整数か半整数に制限するのは質量がゼロでない粒子と同様,トポロジー的な理由だ.

\vskip\baselineskip

$\Lambda$と$p$が与えられたとき,小群の元(2.5.43)を計算するためには,$k^\mu=(0,0,\kappa,\kappa)$を$p^\mu$に変換するための基準となるローレンツ変換$L(p)$を決めておく必要がある.これには,次の形を使うのが便利だ.
\begin{align*}
L(p)=R(\hat{\mathbf{p}})B(|\mathbf{p}|/\kappa)
\end{align*}
ここで$B(u)$は3軸方向の純粋なブーストだ.
\begin{align*}
B(u)=\left(
\begin{matrix}
1 & 0 & 0 & 0 \\
0 & 1 & 0 & 0 \\
0 & 0 & (u^2+1)/2u & (u^2-1)/2u \\
0 & 0 & (u^2-1)/2u & (u^2+1)/2u
\end{matrix}
\right)
\end{align*}
また$R(\hat{\mathbf{p}})$は以前と同様に,3軸を単位ベクトル$\mathbf{p}$に回す純粋な三次元回転だ.この$L(p)$は実際に基準ブースト
\begin{align*}
\tensor{L}{^\mu_\nu}(p)k^\nu=&R(\hat{\mathbf{p}})\left(
\begin{matrix}
1 & 0 & 0 & 0 \\
0 & 1 & 0 & 0 \\
0 & 0 & (|\mathbf{p}|^2/\kappa^2+1)\kappa/2|\mathbf{p}| & (|\mathbf{p}|^2/\kappa^2-1)\kappa/2|\mathbf{p}| \\
0 & 0 & (|\mathbf{p}|^2/\kappa^2-1)\kappa/2|\mathbf{p}| &  (|\mathbf{p}|^2/\kappa^2+1)\kappa/2|\mathbf{p}|
\end{matrix}
\right)\left(
\begin{matrix}
0\\
0\\
\kappa \\
\kappa
\end{matrix}
\right) \\
=&R(\hat{\mathbf{p}})\left(
\begin{matrix}
0 \\
0 \\
|\mathbf{p}| \\
|\mathbf{p}|
\end{matrix}
\right)=\left(
\begin{matrix}
\mathbf{p} \\
p^0
\end{matrix}
\right)=p^\mu \quad \because 質量ゼロ粒子であるから|\mathbf{p}|=p^0
\end{align*}
となる.$\hat{\mathbf{p}}$が極座標$\theta,\phi$で
\begin{align*}
\hat{\mathbf{p}}=(\sin\theta\cos\phi,\sin\theta\sin\phi,\cos\theta)
\end{align*}
と書くと,$R(\hat{\mathbf{p}})$は$(0,0,1)$を$(\sin\theta,0,\cos\theta)$へと回す2軸周りの角度$\theta$の回転(以下は時間成分を省略して書く)
\begin{align*}
\left(
\begin{matrix}
\cos\theta & 0 & \sin\theta \\
0 & 1 & 0 \\
-\sin\theta & 0 & \cos\theta
\end{matrix}
\right)\left(
\begin{matrix}
0 \\
0 \\
1
\end{matrix}
\right)=\left(
\begin{matrix}
\sin\theta \\
0 \\
\cos\theta
\end{matrix}
\right)
\end{align*}
と,それに続く3軸周りの角度$\phi$の回転
\begin{align*}
\left(
\begin{matrix}
\cos\phi & -\sin\phi & 0 \\
\sin\phi & \cos\phi & 0 \\
0 & 0 & 1
\end{matrix}
\right)\left(
\begin{matrix}
\sin\theta \\
0 \\
\cos\theta
\end{matrix}
\right)=\left(
\begin{matrix}
\sin\theta \cos\phi \\
\sin\theta \sin\phi \\
\cos\theta
\end{matrix}
\right)
\end{align*}
で分けることができる.
\begin{align*}
R(\hat{\mathbf{p}})=\left(
\begin{matrix}
\cos\phi & -\sin\phi & 0 \\
\sin\phi & \cos\phi & 0 \\
0 & 0 & 1
\end{matrix}
\right)\left(
\begin{matrix}
\cos\theta & 0 & \sin\theta \\
0 & 1 & 0 \\
-\sin\theta & 0 & \cos\theta
\end{matrix}
\right)=R_3(\phi)R_2(\theta)
\end{align*}
無限小変換を考えると
\begin{align*}
R(\hat{\mathbf{p}})_{ij}=&\delta_{ij}+\omega_{ij} \\
=&\left(
\begin{matrix}
1 & -\phi & 0 \\
\phi & 1 & 0 \\
0 & 0 & 1
\end{matrix}
\right)\left(
\begin{matrix}
1 & 0 & \theta \\
0 & 1 & 0 \\
-\theta & 0 & 1
\end{matrix}
\right)=\left(
\begin{matrix}
1 & -\phi & \theta \\
\phi & 1 & 0 \\
-\theta & 0 & 1
\end{matrix}
\right) \\
\omega_{ij}=&\left(
\begin{matrix}
0 & -\phi & \theta \\
\phi & 0 & 0 \\
-\theta & 0 & 0
\end{matrix}
\right)
\end{align*}
したがってユニタリー演算子は(2.4.3)と同様
\begin{align*}
U(1+\omega)=&1+\frac{i}{2}\omega_{ij}J^{ij} \\
=&1-i\theta J_2 -i\phi J_3 \quad (J_2=J^{31},J_3=J^{12})
\end{align*}
となり,有限変換にすると
\begin{align*}
U(R(\hat{\mathbf{p}}))=U(R_3(\phi))U(R_2(\theta))=\exp(-i\phi J_3)\exp(-i\theta J_2)
\end{align*}
となる.ここで$0\leq \theta \leq \pi ,0\leq \phi \leq 2\pi$だ.($\theta$か$\phi$を$2\pi$ずらすと同じ$R(\hat{\mathbf{p}})$の回転$R_3(\phi+2\pi)R_2(\theta+2\pi)=R_3(\phi)R_2(\theta)$が得られるが,整数・半整数スピン状態のどちらに作用するかによって$U(R(\hat{\mathbf{p}}))$の符号が変わってしまう.たとえば,$\theta=0$としておいて$U(R(\hat{\mathbf{p}}))=\exp(-i\phi J_3)$としたとき,$\sigma$が整数か半整数かのとき
\begin{align*}
U(R(\hat{\mathbf{p}})) \Psi_{k,\sigma}=& \exp(-i \phi \sigma)\Psi_{k,\sigma} \\
U(R(\hat{\mathbf{p}})(\phi+2\pi)) \Psi_{k,\sigma}=&\exp(-i(\phi+2\pi)\sigma) \Psi_{k,\sigma} =(-1)^{2\sigma} \exp(-i\phi \sigma) \Psi_{k,\sigma}
\end{align*}
となり,整数のときは両者は等しいが半整数の場合は負符号がつくことになる.しかし$R(\hat{\mathbf{p}})=R(\hat{\mathbf{p}})(\phi+2\pi)$であるから,$R$とともに$\theta,\phi$の範囲を与えると矛盾が生じる.したがって$\theta,\phi$の範囲は$R$ではなく$U(R)$とともに設定する必要がある.(2.5.47)は回転によって3軸$(0,0,1)$を(2.5.46)の方向に回す変換で,たとえ他にそのような$R(\hat{\mathbf{p}})$があったとしても,せいぜい最初の3軸周りの回転の違いしかない(これは3軸(0,0,1)に対しては不変に保ち,$R'(\mathbf{p})=R_3(\phi)R_2(\phi)R_3(\phi')$という変換もまた3軸を$\mathbf{p}$にするからだ).これは単に1粒子状態の位相の再定義にしかすぎない.つまり
\begin{align*}
U(R'(\mathbf{p}))\Psi_{k,\sigma}=U(R(\mathbf{p}))e^{-i\phi'\sigma}\Psi_{k,\sigma}
\end{align*}
となるが,位相因子分しか違いはないので,位相を再定義すればどちらの回転が基準かは関係ない.\par
ヘリシティがローレンツ不変であることに注意しよう.実際(2.5.42)より,あるヘリシティ$\sigma$の質量ゼロの粒子はすべての慣性系で(運動量を除いて)不変に見えるのだ.実際,質量ゼロの粒子の異なるヘリシティ状態$\Psi_{\sigma},\Psi_{\sigma'}$は別の種類の粒子と考えてもよい.しかし,次の節でみるように,ヘリシティが反対の粒子は空間反転(パリティ)対称性で関係している.\par
$\Rightarrow$電磁的,重力的な相互作用はパリティ対称性をもつので,電磁的現象にともなうヘリシティ$\pm 1$の質量ゼロの粒子はともに光子と呼ばれる.また重力にともなうと考えられるヘリシティ$\pm 2$の質量ゼロの粒子はともに重力子と呼ばれる.一方,核子のベータ崩壊にともなって放出される質量ゼロ(厳密には違うが)ヘリシティ$\pm 1/2$の粒子は(重力以外に)空間反転について対称な相互作用を持たない.これらの粒子はそれぞれ異なる名前,つまりヘリシティが$-1/2$はニュートリノ,ヘリシティ$+1/2$は反ニュートリノを付けられている.

\vskip\baselineskip


質量ゼロの粒子のヘリシティはローレンツ不変だが,「状態そのもの」はそうではない.特に(2.5.42)のヘリシティに依存する位相因子$\exp(i\sigma \theta)$のせいで,反対のヘリシティをもつ1粒子状態の線形結合で作られる状態は,ローレンツ変換のもとで異なる線形結合に変換される.たとえば,一般の1光子状態は以下のように書ける.(光子はヘリシティ$\pm 1$であるから)
\begin{align*}
\Psi_{p;\alpha}=\alpha_{+} \Psi_{p,+1}+\alpha_{-} \Psi_{p,-1}
\end{align*}
ただし,それぞれの状態が規格化されるように
\begin{align*}
|\alpha_+|^2+|\alpha_-|^2=1
\end{align*}
一般的な場合は$|\alpha_{\pm }|$は両方ともゼロというわけではなく互いに異なり,楕円偏極と呼ばれる.線偏極は$\alpha_+$か$\alpha_-$がゼロの場合で,その反対の極端な場合$|\alpha_+|=|\alpha_-|(=1/\sqrt{2})$は円偏極と呼ばれる.(本書の用語はおそらく誤植で,反対だ.)\par
$\alpha_+$と$\alpha_-$に共通の位相は状態の再定義で完全に取り除けるので物理的になんら重要ではない.そのため円偏極ではこの位相を調整して
\begin{align*}
\alpha_+=&|\alpha_+|e^{i\delta_1} \\
\alpha_-=&|\alpha_-|e^{i\delta_2} \\
\Psi'_{p;\alpha'}=&e^{i\theta}\Psi_{p;\alpha}=e^{i\theta} \alpha_{+} \Psi_{p,+1}+e^{i\theta} \alpha_{-} \Psi_{p,-1} \\
=&\alpha'_+\Psi_{p,+1}+\alpha'_{-} \Psi_{p,-1} \\
\alpha'_-=&|\alpha_-|e^{i(\delta_2+\theta)}=|\alpha_+|e^{i(\delta_1+\delta_2+2\theta)}e^{-i(\delta_1+\theta)} \quad \because |\alpha_+|=|\alpha_-| \\
=&{\alpha'}_+^* e^{i(\delta_1+\delta_2+2\theta)} \\
=&{\alpha'}^*_+ \quad (\delta_1+\delta_2+2\theta=2\pi n となるように \theta を選ぶ)
\end{align*}
として$\alpha_-=\alpha^*_+$とできる.しかし,依然として相対的な位相は重要だ.実際$\alpha_-=\alpha_+^*$の線形結合
\begin{align*}
\Psi_{p;\alpha}=&\alpha_{+} \Psi_{p,+1}+\alpha^*_{+} \Psi_{p,-1} \\
=&\frac{1}{\sqrt{2}}\left[e^{i\delta}\Psi_{p,+1}+e^{-i\delta} \Psi_{p,-1} \right]
\end{align*}
では,$\alpha_+$の位相$\delta$は,$\mathbf{p}$に垂直なある基準方向と偏極面との角度を意味する.(ランダウ「場の古典論」p130,(48.9)など参照.)(2.5.42)から,ローレンツ変換$\tensor{\Lambda}{^\mu_\nu}$のもとで,この角度は
\begin{align*}
U(\Lambda)\Psi_{p;\alpha}=&U(\Lambda)\frac{1}{\sqrt{2}}\left[e^{i\delta}\Psi_{p,+1}+e^{-i\delta} \Psi_{p,-1} \right] \\
=&\sqrt{\frac{(\Lambda p)^0}{2p^0}}\left[e^{+i(\theta+\delta) }\Psi_{\Lambda p,+1}+e^{-i(\theta+\delta)}\Psi_{\Lambda p,-1}\right]
\end{align*}
となり$\theta(\Lambda,p)$だけ回転することがわかる.面偏極した重力子も同様に定義できて,このときは(2.5.42)から,ローレンツ変換$\Lambda$が偏極面を角度$2\theta$だけ回転させることがわかる.
\begin{align*}
\Psi_{p,\alpha}=&\frac{1}{\sqrt{2}}\left[e^{i\delta}\Psi_{p,+2}+e^{-i\delta} \Psi_{p,-2} \right] \\
U(\Lambda)\Psi_{p;\alpha}=&\sqrt{\frac{(\Lambda p)^0}{2p^0}}\left[e^{+i(2\theta+\delta) }\Psi_{\Lambda p,+2}+e^{-i(2\theta+\delta)}\Psi_{\Lambda p,-2}\right]
\end{align*}

\vskip\baselineskip

小群$ISO(2)=SO(2)\ltimes \mathbb{R}^2$はコンパクトではないから,そのユニタリー表現は必ず無限次元となる.しかし上で質量ゼロの有限次元ユニタリー表現が作ることができたのは,$\rho^2=0$を\uwave{手で}課すことによって$ISO(2)=SO(2)\ltimes \mathbb{R}^2$が部分群$SO(2)$へと縮約(この縮約は2.4節でのイネヌ・ウィグナー縮約(contraction)と同じ意味で,より小さい群になるということ)され,$SO(2)$はコンパクトであるから有限次元ユニタリー表現が作ることができたということである.もちろん$\rho=0$の場合,昇降演算子$T_\pm$はゼロ演算子となってしまうから,ヘリシティはひとつの状態しか許されない.これがローレンツ変換により慣性系が変化しても光子などの質量ゼロ粒子のヘリシティが不変である理由である.\par


\newpage

ちょっとした補足.ここで,Wignerによる完全な1粒子表現の分類について説明しておく.実際はここで考えたものよりもっと色んな種類の状態が考えられ,それをリストアップしたものが本文中の表2.1である.完全に全ての表現を定量的にリストアップするためには,ポアンカレ代数のカシミア演算子についての説明が不可避となる.それをここで与えよう.\par
最初に,運動量演算子の二乗$P_\mu P^\mu$という演算子を考える.これはポアンカレ代数の全ての生成子と可換である.(まぁ直感的にそう.)実際
\begin{align*}
[P^\rho,P_\mu P^\mu ]=&[P^\rho ,P_\mu]P^\mu +P_\mu [P^\rho,P^\mu]=0 \\
[J^{\rho\sigma},P_\mu P^\mu]=&[J^{\rho\sigma},P^\mu] P_\mu +P_\mu[J^{\rho\sigma},P^\mu] \\
=&(i\eta^{\mu\rho}P^\sigma -i\eta^{\mu\sigma}P^\rho)P_\mu +P_\mu(i\eta^{\mu\rho}P^\sigma -i\eta^{\mu\sigma}P^\rho) \\
=&i P^{\sigma}P^\rho -iP^\rho P^\sigma +iP^\rho P^\sigma -iP^\sigma P^\rho \\
=&0
\end{align*}
となる.この演算子$P^\mu P_\mu$は質量演算子と呼ばれる.\par
次に
\begin{align*}
W^\mu:=&\frac{1}{2}\epsilon^{\mu\nu\rho\sigma}J_{\nu\rho}P_\sigma =\frac{1}{2}\epsilon^{\mu\nu\rho\sigma}P_\nu J_{\rho\sigma}
\end{align*}
という演算子を考える($P$と$J$の交換で出てくるものは完全反対称性により消えるので二つ目の等号が出てくる).これはPauli-Lubanski(擬)ベクトルと呼ばれる.ここで$\epsilon^{\mu\nu\rho\sigma}$は$\epsilon^{0123}=-\epsilon_{0123}=1$となる完全反対称テンソルだ.$J^{\mu\nu},P^\mu$がエルミート演算子だから,$W^\mu$もエルミート演算子となる.
\begin{align*}
(W^\mu)^\dagger=\left(\frac{1}{2}\epsilon^{\mu\nu\rho\sigma}J_{\nu\rho}P_\sigma\right)^\dagger =\frac{1}{2}\epsilon^{\mu\nu\rho\sigma}P_\sigma J_{\nu\rho}=\frac{1}{2}\epsilon^{\mu\nu\rho\sigma}J_{\nu\rho}P_\sigma=W^\mu
\end{align*}
これの各成分を見てみると
\begin{align*}
W^0=&\frac{1}{2}\epsilon^{0ijk}J_{ij}P_k \\
=&\mathbf{J}\cdot \mathbf{P} \quad \because \epsilon^{0ijk}=\epsilon^{ijk},\frac{1}{2}\epsilon^{ijk}J_{jk}=J_i \\
W^i=&\frac{1}{2}\epsilon^{i\nu\rho\sigma} J_{\nu\rho}P_\sigma \\
=&\frac{1}{2}\epsilon^{i0jk}J_{0j}P_k +\frac{1}{2}\epsilon^{ij0k}J_{j0}P_k +\frac{1}{2}\epsilon^{ijk0} J_{jk}P_0 \\
=&+P^0 J_i +\epsilon^{ijk}K_j P_k \\
=&+J_i P^0 +(\mathbf{K}\times \mathbf{P})_i
\end{align*}
となる.これがどのような物理量と対応するかは後で見る.次にこのベクトルの性質を見てみる.完全反対称性により
\begin{align*}
P_\mu W^\mu=W^\mu P_\mu=0
\end{align*}
を満たす.これ自身は全ての成分がポアンカレ代数の生成子と可換なものとなっているわけではないが,それぞれの交換関係を見てみると,まず$P^\mu$との交換関係については
\begin{align*}
[P^\alpha,W^\mu]=&\frac{1}{2}\epsilon^{\mu\nu\rho\sigma}[P^\alpha ,J_{\nu\rho}P_\sigma] \\
=&\frac{1}{2}\epsilon^{\mu\nu\rho\sigma}[P^\alpha ,J_{\nu\rho}]P_\sigma \\
=&\frac{1}{2}\epsilon^{\mu\nu\rho\sigma}(-i\delta^{\alpha}_\nu P_\rho+i\delta^{\alpha}_{\rho}P_\nu)P_\sigma \\
=&-i\frac{1}{2}\epsilon^{\mu\alpha\rho\sigma}P_\rho P_\sigma +i\frac{1}{2}\epsilon^{\mu\nu\alpha\sigma}P_\nu P_\sigma\\
=&0 \quad \because \epsilon^{\mu\nu\rho\sigma} の完全反対称性とP^\mu 同士の可換性
\end{align*}
となり,また$J^{\mu\nu}$については
\begin{align*}
0=&[J^{\alpha\beta},P_\mu W^\mu] \\
=&[J^{\alpha\beta},P_\mu]W^\mu +P_\mu[J^{\alpha\beta},W^\mu] \\
=&(i\delta^\alpha_\mu P^\beta-i\delta^\beta_\mu P^\alpha)W^\mu+P_\mu[J^{\alpha\beta},W^\mu] \\
=&P_\mu(i\eta^{\beta\mu}W^\alpha -i\eta^{\alpha\mu}W^\beta +[J^{\alpha\beta},W^\mu]) \\
\therefore \quad [J^{\alpha\beta},W^\mu]=&i\eta^{\alpha\mu}W^\beta-i\eta^{\beta\mu}W^\alpha
\end{align*}
となる.(同値性が怪しいところだが,直接計算するとうまくいかない.というのも
\begin{align*}
[J^{\alpha\beta},W^\mu]=&\frac{1}{2}\epsilon^{\mu\nu\rho\sigma}[J^{\alpha\beta},J_{\nu\rho}P_\sigma] \\
=&\frac{1}{2}\epsilon^{\mu\nu\rho\sigma}[J^{\alpha\beta},J_{\nu\rho}]P_\sigma+\frac{1}{2}\epsilon^{\mu\nu\rho\sigma}J_{\nu\rho}[J^{\alpha\beta},P_\sigma] \\
=&\frac{1}{2}\epsilon^{\mu\nu\rho\sigma}(-i\delta^\beta_\nu \tensor{J}{^\alpha_\rho}+i\delta^\alpha_\nu \tensor{J}{^\beta_\rho}+i\delta^\alpha_\rho \tensor{J}{_\nu^\beta}-i\delta_\rho^\beta \tensor{J}{_\nu^\alpha})P_\sigma \\
&+\frac{1}{2}\epsilon^{\mu\nu\rho\sigma}J_{\nu\rho}(i\delta_\sigma^\alpha P^\beta -i\delta_\sigma^\beta P^\alpha) \\
=&-i\frac{1}{2}\epsilon^{\mu\beta\rho\sigma}\tensor{J}{^\alpha_\rho}P_\sigma +i\frac{1}{2}\epsilon^{\mu\alpha\rho\sigma}\tensor{J}{^\beta_\rho}P_\sigma+i\frac{1}{2}\epsilon^{\mu\nu\alpha\sigma}\tensor{J}{_\nu^\beta}P_\sigma-i\frac{1}{2}\epsilon^{\mu\nu\beta\sigma}\tensor{J}{_\nu^\alpha}P_\sigma \\
&+i\frac{1}{2}\epsilon^{\mu\nu\rho\alpha}\tensor{J}{_\nu_\rho}P^\beta-i\frac{1}{2}\epsilon^{\mu\nu\rho\beta}\tensor{J}{_\nu_\sigma}P^\alpha \\
=&-i\epsilon^{\mu\beta\rho\sigma}\tensor{J}{^\alpha_\rho}P_\sigma +i\epsilon^{\mu\alpha\rho\sigma}\tensor{J}{^\beta_\rho}P_\sigma \\
&+i\frac{1}{2}\epsilon^{\mu\nu\rho\alpha}\tensor{J}{_\nu_\rho}P^\beta-i\frac{1}{2}\epsilon^{\mu\nu\rho\beta}\tensor{J}{_\nu_\sigma}P^\alpha
\end{align*}
となるから,そのままでは$W$が出てこない.ただし$\alpha,\beta,\mu$に具体的な$0,1,2,3$の値を入れてみれば同じ結果になることがわかる.詳細な途中計算は省くが
\begin{align*}
[J^{0i},W^j]=&+i\epsilon^{j0kl}J_{ik}P_l +\frac{1}{2}i\epsilon^{jkl0}J_{kl}P^i \\
[J^{01},W^2]=&0 \\
=&i\eta^{01}W^2-i\eta^{12}W^0\\
[J^{01},W^1]=&-i(J_{23}P_1+J_{31}P_2+J_{12}P_2) \\
=&-iW^0 \\
=&i\eta^{01}W^1-i\eta^{11}W^0
\end{align*}
となる.全成分計算すれば実際に上の交換関係が正しいことがわかると思う.そこまではやりたくないけど.)この交換関係を見ると,$W^\mu$はポアンカレ代数の生成子$P^\mu,J^{\mu\nu}$とはまるで$P^\mu$と同じ交換関係を満たすことがわかる((2.4.13)参照).したがって$W^\mu W_\mu$は$P^\mu P_\mu$と同様,ポアンカレ代数の生成子とは可換であることが推測できる.実際
\begin{align*}
[P^\alpha ,W^\mu W_\mu]=&[P^\alpha,W^\mu]W_\mu +W_\mu[P^\alpha,W^\mu] \\
=&0 \\
[J^{\alpha\beta},W^\mu W_\mu]=&[J^{\alpha\beta},W^\mu]W_\mu +W_\mu [J^{\alpha\beta},W^\mu] \\
=&(i\eta^{\mu\alpha}W^\beta -i\eta^{\mu\beta}W^\alpha)W_\mu +W_\mu(i\eta^{\mu\alpha}W^\beta -i\eta^{\mu\beta}W^\alpha) \\
=&i W^{\beta}W^\alpha -iW^\alpha W^\beta +iW^\alpha W^\beta -iW^\beta W^\alpha \\
=&0
\end{align*}
となって可換である.あと$W^\mu$同士の交換関係は
\begin{align*}
[W^\alpha,W^\mu]=&\frac{1}{2}\epsilon^{\mu\nu\rho\sigma}[W^\alpha,J_{\nu\rho}P_\sigma] \\
=&\frac{1}{2}\epsilon^{\mu\nu\rho\sigma}[W^\alpha,J_{\nu\rho}]P_\sigma \\
=&\frac{1}{2}\epsilon^{\mu\nu\rho\sigma}(-i\delta^\alpha_\nu W_\rho+i\delta^\alpha_\rho W_\nu)P_\sigma \\
=&-\frac{1}{2}i\epsilon^{\mu\alpha\rho\sigma}W_\rho P_\sigma +\frac{1}{2}i\epsilon^{\mu\nu\alpha\sigma}W_\nu P_\sigma \\
=&i\epsilon^{\alpha\mu\rho\sigma}W_\rho P_\sigma
\end{align*}
となる.\par
以上から,$C_1:=P^\mu P_\mu ,C_2:=W^\mu W_\mu$はポアンカレ代数の生成子と可換なものであり,これらを2次のカシミア演算子,4次のカシミア演算子と呼ぶ.当然これら二つと可換なものとして,運動量演算子$P^\mu$がある.加えて,$W^\mu$もこれらと可換である.($[P^\alpha,W^\mu]=0$かつ$P^\mu P_\mu,W^\mu W_\mu$が全ての生成子と可換なことから自明だ.)以上より,
\begin{align*}
\{P^\mu P_\mu,W^\mu W_\mu,P^\mu ,W^3 \}
\end{align*}
は互いに可換な演算子の組となり,それらには同時固有状態が存在する.($W^\mu$同士の交換関係はゼロではないから,$W^\mu$のうち1つの成分のみを選ぶ必要がある.)それを$\Psi_{m,\rho;p,w}$と書き,それはこれらの演算子に対する固有値が$\{m^2,\rho^2,p^\mu ,w\}$となる固有状態
\begin{align*}
P^\mu P_\mu \Psi_{m,\rho;p,w}=&-m^2 \Psi_{m,\rho;p,w} \\
W^\mu W_\mu \Psi_{m,\rho;p,w}=&\rho^2 \Psi_{m,\rho;p,w} \\
P^\mu \Psi_{m,\rho;p,w}=& p^\mu \Psi_{m,\rho;p,w} \\
W^3 \Psi_{m,\rho;p,w}=& w \Psi_{m,\rho;p,w}
\end{align*}
となっている.$m^2$と$\rho^2$はどちらも質量次元$2$のパラメータである.これらの固有値および固有状態を分類することにより粒子の分類が完了する.$P^\mu P_\mu,W^\mu W_\mu$はどちらもポアンカレ変換のもとで不変な演算子
\begin{align*}
U(\Lambda,a)P^\mu P_\mu U(\Lambda,a)^{-1} =&P^\mu P_\mu \\
U(\Lambda,a)W^\mu W_\mu U(\Lambda,a)^{-1}=&W^\mu W_\mu
\end{align*}
となっている(ポアンカレ代数の生成子と可換だから明らか)から,その固有値$-m^2,\rho^2$は全ての慣性系で共通の固有値となっている.しかし$P^\mu ,W^3$は不変ではないから,その固有値は慣性系によって変化する\footnote{ローレンツ添え字の足がつぶれていればいいわけではない.ローレンツ代数だけでなく並進も含めたポアンカレ代数であるから,例えば$J^{\mu\nu}J_{\mu\nu}$や$\epsilon^{\mu\nu\rho\sigma}J_{\mu\nu}J_{\rho\sigma}$などはカシミア演算子にならない.}.\par
適当な慣性系をとることにより,有質量の場合は$k^\mu=(0,0,0,m)$とでき,質量ゼロの場合は$k^\mu=(0,0,\kappa,\kappa)$とできる.有質量の場合,$W^\mu$の値は
\begin{align*}
w^0=0,\quad w^i=mJ^i
\end{align*}
となる.よって$w^i/m$は角運動量ベクトルであり,$w=w^3$はスピンの3軸成分を表す.さらに$W^\mu W_\mu$の値は
\begin{align*}
\rho^2:=w^\mu w_\mu=m^2 \mathbf{J}^2=m^2j(j+1)
\end{align*}
となる.質量ゼロの場合,
\begin{align*}
w^0=&w^3=\kappa J_3 \\
w^1=&\kappa (J_1+K_2)=-\kappa B \\
w^2=&\kappa(J_2-K_1)=\kappa A
\end{align*}
となる.よって$w^0/\kappa=w^3/\kappa$はヘリシティを表す.$\epsilon_1:=(1,0,0,0)=\mathbf{e}_1,\epsilon_2:=(0,1,0,0)=\mathbf{e}_2$とを用いるとこれは
\begin{align*}
w^\mu=&k^\mu J_3-\epsilon_1^\mu \kappa B+\epsilon^\mu_2 \kappa A
\end{align*}
とも書ける.さらにカシミア演算子$W^\mu W_\mu$の値は
\begin{align*}
\rho^2:=w^\mu w_\mu =&\kappa^2 (A^2+B^2) \\
=&\kappa^2 (A+iB)(A-iB) =\kappa^2 T_+ T_-
\end{align*}
となる.すなわち質量ゼロ粒子についても,一般には質量的パラメータ$\rho^2$が存在し,それがゼロとなるか正の値となるかによって有限次元表現か無限次元表現かに分かれる.質量ゼロの場合,$[A,B]=0$であるから$w^3$の固有状態ととらずに$w^1,w^2$の固有状態をとることも可能になる.どちらをとるかによって表現の形が変化する.\par
これ以外にも,エネルギーが正か負か,質量が虚数のタキオン状態など,物理的でない状態も許せば様々な表現が存在する.それらも勘定すると,次のようにポアンカレ代数の表現が分類される.

\vskip\baselineskip

\textbf{1.Massive \& Positive-Energy Representation}\par
これは$P^\mu P_\mu$の固有値である質量$m$が非ゼロの値をとり,さらに$P^0$の固有値$E=p^0$が正の値をとる場合である.運動量$p^\mu$が基準運動量$k^\mu=(0,0,0,m)$となる慣性系に移動することにより,質量は正であることがわかる.小群は$SO(3)$となり,スピン$j=0,1/2,1,3/2,\cdots $とスピンの3軸成分$\sigma=\{-j,-j+1,\cdots ,j-1,j\}$でラベル付けされた$2j+1$個の基底がなす$2j+1$次元既約表現をなすことがわかるのだった.したがってそのような表現は,それぞれの固有値が
\begin{align*}
P^\mu P_\mu :& m^2>0 \\
W^\mu W_\mu :&\rho^2=m^2 j(j+1) \\
P^\mu :& p^2+m^2=0,\quad p^0>0\\
W^3:& w/m \in \{-j,-j+1,\cdots ,j-1,j\}
\end{align*}
となる.ここで$j(j+1)$は演算子$\mathbf{J}^2$の固有値であり,$\sigma$は演算子$J_3$の固有値となる.$W^\mu$はこの表現では$w^0=0,w^i=mJ_i$として作用する.一つの表現につき,$j$の値は$0,1/2,1,\cdots $のうち一つである.

\vskip\baselineskip

\textbf{2.Massive \& Negative-Energy Representation}\par
これは上の表現とほぼ同じだが,$p^0$が負の値をとる表現である.つまり質量が負の値$-m<0$となる.このときの基準運動量も$k^\mu=(0,0,0,-m)$であり,小群は$SO(3)$,固有値もほぼ変わらず
\begin{align*}
P^\mu P_\mu :& m^2>0 \\
W^\mu W_\mu :&\rho^2=m^2 j(j+1) \\
P^\mu :&p^2+m^2=0,\quad p^0 < 0\\
W^3:& w/m \in \{-j,-j+1,\cdots ,j-1,j\}
\end{align*}
となる.ひとつの表現につき,$j$の値は$0,1/2,1,\cdots$のうち一つである.

\vskip\baselineskip

\textbf{3.Massless \& Positive-Energy \& Finite-Spin Representation}\par
これは,$m^2=0$かつ$\rho^2=0$となることで前に示した通り,$A,B$の作用が自明となり有限次元表現となる場合である.かつ正エネルギーであるから,物理的な場合である.基準運動量が$k^\mu=(0,0,\kappa,\kappa)$となる慣性系に移れば,固有値は
\begin{align*}
P^\mu P_\mu :& m^2>0 \\
W^\mu W_\mu :&\rho^2=0 \\
P^\mu :&p^2=0,\quad p^0 > 0\\
W^3:&w/\kappa=h
\end{align*}
この場合,一つの表現につき,ヘリシティ$h$は$0,\pm 1/2,\pm 1,\cdots $のうち一つである.$\rho^2=0$により昇降演算子が存在しないため,この場合ヘリシティは各表現につき一つだけである.

\vskip\baselineskip

\textbf{4.Massless \& Negative-Energy \& Finite-Spin Representation}\par
上と同様だが,エネルギーが負であることだけが違う.基準運動量は$k^\mu=(0,0,\kappa,-\kappa)$となる.固有値は
\begin{align*}
P^\mu P_\mu :& m^2>0 \\
W^\mu W_\mu :&\rho^2=0 \\
P^\mu :&p^2=0,\quad p^0 < 0\\
W^3:&w/\kappa=h
\end{align*}
となる.こちらも一つの表現につき,ヘリシティ$h$は$0,\pm 1/2,\pm 1,\cdots $のうち一つである.

\vskip\baselineskip

\textbf{5.Massless \& Positive-Energy \& Infinite-Spin Representation}\par
$\rho^2>0$となり,全ての生成子が非自明な作用をする.この場合は実は無限次元の表現となることを見る.基準運動量は再び$k^\mu=(0,0,\kappa,\kappa)$であり,固有値は
\begin{align*}
P^\mu P_\mu :& m^2>0 \\
W^\mu W_\mu :&\rho^2>0 \\
P^\mu :&p^2=0,\quad p^0 > 0\\
W^3:&w/\kappa=\{0,\pm 1,\pm2 ,\cdots \} \mathrm{or} \{\pm 1/2 ,\pm 3/2,\cdots \}
\end{align*}
となる.$\rho^2>0$であるから$T_\pm :=A\pm iB$が昇降演算子として働き,一つの表現につきヘリシティは$0,\pm 1,\pm 2,\cdots $の\uwave{全て},もしくは$\pm 1/2,\pm 3/2 ,\cdots $の\uwave{全て}をとる.この場合をCSP表現などという.以下でこれを具体的に構成してみよう.\par
前より少し詳しく小群のもとでの変換性を見る.簡単のため無次元化させた$\rho':=\rho/\kappa$を,プライムを落として新しく$\rho$として$\rho^2=A^2+B^2$とする.前に述べたように$[A,B]=0$であるから$W^3$ではなく$A,B$の固有状態をとることもできる\footnote{$W^3$の固有状態ととればspin基底と呼ばれる整数または半整数によりラベル付けされた基底になり,$A,B$の同時固有状態をとるとangle基底と呼ばれる角度$(0,2\pi]$によってラベル付けされた基底が得られる.ここでは後者を調べ,後に両者の関係も調べる.}.したがってその固有値を$(\rho\cos\theta,-\rho \sin\theta)$とすれば対応する固有状態は$\Psi^\theta_{k}$となり(ちゃんと書くと$\Psi_{m=0,\rho;k}^\theta$だが,$m=0$と$\rho$の値はどの慣性系でも共通だから,固定されているとして添え字を省略する)
\begin{align*}
A\Psi^\theta_{k}=&\rho\cos\theta \Psi^\theta_{k} \\
B\Psi^\theta_{k}=&-\rho\sin\theta \Psi^\theta_{k}
\end{align*}
という作用をする.(負符号は$U[R(\theta)]$の作用が減法ではなく加法的に作用するように選んだ.)さらに
\begin{align*}
U[R(\theta)]=&\exp(i\theta J_3) \\
U[R(\theta)]AU^{-1}[R(\theta)]=&A\cos\theta-B\sin\theta \\
U[R(\theta)]BU^{-1}[R(\theta)]=&A\sin\theta +B\cos\theta
\end{align*}
により
\begin{align*}
AU[R(\theta)]\Psi^\phi_{k}=&(\rho\cos\phi\cos\theta-\rho\sin\phi \sin\theta)U^{-1}[R(\theta)]\Psi^\phi_{k} \\
=&\rho\cos(\theta+\phi)U[R(\theta)]\Psi_{k}^\theta \\
BU[R(\theta)]\Psi^\phi_{\rho;k}=&(\rho\cos\phi\sin\theta+\rho\sin\phi \cos\theta)U[R(\theta)]\Psi^\phi_{k} \\
=&-\rho\sin(\theta+\phi)U[R(\theta)]\Psi_{k}^\theta \\
\therefore \quad U[R(\theta)]\Psi^\phi_{k}=&\Psi^{\theta+\phi}_{k}
\end{align*}
となる.$J_3$の$\Psi_k^\theta$への作用は,$\phi$を微小量とすることにより
\begin{align*}
\Psi^{\phi+\theta}_k=&\Psi_{k}^\phi+\theta\frac{\partial}{\partial \phi}\Psi_{k}^\phi \\
=U[R(\theta)]\Psi^\phi_k=&(1+i\theta J_3)\Psi_{k}^\phi \\
J_3 \Psi_{k}^\theta=&-i\frac{\partial}{\partial \theta}\Psi_{k}^\theta
\end{align*}
と書けて,したがって微分演算子として作用することがわかる.以上の表現は実際に,
\begin{align*}
[J_3,A]\Psi^\theta_k=&J_3 A\Psi^\theta_k-AJ_3 \Psi^\theta_k \\
=&\rho \cos\theta J_3 \Psi^\theta_k+i\frac{\partial}{\partial \theta}(A\Psi^\theta_k) \\
=&-i\rho \cos\theta \frac{\partial}{\partial \theta}\Psi^\theta_k +i\frac{\partial}{\partial \theta}(\rho \cos\theta)\Psi_{k}^\theta \\
=&-i\rho \sin\theta \Psi^\theta_k \\
=&iB\Psi^\theta_k \\
[J_3,B]\Psi^\theta_k=&J_3 B\Psi^\theta_k-BJ_3 \Psi^\theta_k \\
=&-\rho \sin\theta J_3 \Psi^\theta_k+i\frac{\partial}{\partial \theta}(B\Psi^\theta_k) \\
=&+i\rho \sin\theta \frac{\partial}{\partial \theta}\Psi^\theta_k -i\frac{\partial}{\partial \theta}(\rho \sin\theta)\Psi_{k}^\theta \\
=&-i\rho \cos\theta \Psi^\theta_k \\
=&-iA\Psi^\theta_k \\
[A,B]\Psi_{k}^\theta=&AB\Psi_k^\theta-BA\Psi_k^\theta \\
=&\rho^2\sin\theta\cos\theta\Psi_{k}^\theta-\rho^2\sin\theta\cos\theta\Psi_{k}^\theta \\
=&0\cdot \Psi_{k}^\theta
\end{align*}
となって交換関係と矛盾しない.さらに小群のユニタリー演算子は
\begin{align*}
U(W)=\exp(i\alpha A+i\beta B)\exp(i\theta J_3)
\end{align*}
であったから,これは$\Psi_{k}^\phi$に対して
\begin{align*}
U[W(\theta,\alpha,\beta)]\Psi_{k}^\phi=&\exp(i\rho \alpha \cos(\theta+\phi)+i\rho \beta \sin(\theta+\phi))\Psi_{k}^{\theta+\phi} \\
=&\int^{2\pi}_0 \frac{d\phi'}{2\pi} D_{\phi'\phi}[\theta,\alpha,\beta] \Psi_{k}^{\phi'} \\
D_{\phi'\phi}[\theta,\alpha,\beta]:=&(2\pi)\delta(\phi'-\phi-\theta)\exp(i\rho \alpha \cos \phi'+i\rho \beta \sin\phi')
\end{align*}
と作用する.これは明らかに連続表現として変換している.さらに
\begin{align*}
\Bigl(D[\theta,\alpha,\beta]^\dagger D[\theta,\alpha,\beta]\Bigr)_{\phi'\phi}=&\int^{2\pi}_0 \frac{d\phi''}{2\pi} D_{\phi'' \phi}[\theta,\alpha,\beta]^* D_{\phi''\phi'}[\theta,\alpha,\beta] \\
=&\int_0^{2\pi} \frac{d\phi''}{2\pi}(2\pi)\delta(\phi''-\phi-\theta)\exp(-i\rho \alpha \cos \phi''-i\rho \beta \sin\phi'') \\
& \qquad \qquad \times (2\pi)\delta(\phi''-\phi'-\theta)\exp(i\rho \alpha \cos \phi''+i\rho \beta \sin\phi'') \\
=&(2\pi)\delta(\phi'-\phi)
\end{align*}
であるから,$\Psi_{k}^\phi$がユニタリー表現として変換されることもわかる.実際
\begin{align*}
&\left( \int^{2\pi}_0 \frac{d\phi'''}{2\pi} D_{\phi''' \phi'}[\theta,\alpha,\beta] \Psi_{k'}^{\phi'''},\int^{2\pi}_0 \frac{d\phi''}{2\pi} D_{\phi''\phi}[\theta,\alpha,\beta] \Psi_k^{\phi''}\right) \\
=&\int^{2\pi}_0 \frac{d\phi''}{2\pi} \int^{2\pi}_0 \frac{d\phi'''}{2\pi} D_{\phi''' \phi'}[\theta,\alpha,\beta]^* D_{\phi''\phi}[\theta,\alpha,\beta](\Psi_{k'}^{\phi'''},\Psi_k^{\phi''}) \\
=&\delta^3(k'-k)\int^{2\pi}_0 \frac{d\phi''}{2\pi} \int^{2\pi}_0 \frac{d\phi'''}{2\pi} D_{\phi''' \phi'}[\theta,\alpha,\beta]^* D_{\phi''\phi}[\theta,\alpha,\beta](2\pi)\delta(\phi'''-\phi'') \\
=&\delta^3(k'-k) \int^{2\pi}_0 \frac{d\phi''}{2\pi} D_{\phi' \phi''}[\theta,\alpha,\beta]^\dagger D_{\phi''\phi}[\theta,\alpha,\beta] \\
=&\delta^3(k'-k) \Bigl(D[\theta,\alpha,\beta]^\dagger D[\theta,\alpha,\beta]\Bigr)_{\phi'\phi} \\
=&\delta^3(k'-k)2\pi \delta(\phi'-\phi)=(\Psi_{k'}^{\phi'},\Psi_k^\phi)
\end{align*}
となる.ここで正規直交化条件
\begin{align*}
(\Psi_{k'}^{\phi'},\Psi^{\phi}_{k})=\delta^3(k'-k)2\pi \delta(\phi'-\phi)
\end{align*}
を用いた.(まぁユニタリー演算子$U(W)$から作られたのだからユニタリー性は当然っちゃ当然なのだが…)このようにしてangle基底$\{\Psi_k^\theta\}_{\theta \in(0,2\pi]}$が得られる.これは$ISO(2)$の無限次元ユニタリー表現を構成する.\par
$A,B$を対角化するのではなく,$J_3$を対角化する基底であるspin基底を得る.ここでトポロジー的な理由により,3軸周りでの回転は$4\pi$回転で必ず元の状態に戻るという(2.7節で示す)ローレンツ群の2重連結性が要請する.つまり,angle基底はローレンツ群の1価表現であり
\begin{align*}
\Psi^0_{k}=\Psi^{2\pi}_k
\end{align*}
であるか,あるいは
\begin{align*}
\Psi^0_k=-\Psi^{2\pi}_k
\end{align*}
と($4\pi$回転で元に戻るように)ローレンツ群の2価表現であるかのどちらかである.まず前者の,ローレンツ群の1価表現$\Psi^{2\pi}_k=+\Psi_k^{0}$について考える.これは$\theta$について周期$2\pi$の周期関数とみなすことができるから,整数$n$によってフーリエ級数展開
\begin{align*}
\Psi^\theta_k=\sum_{n=-\infty}^\infty \Psi_k^n e^{in\theta}
\end{align*}
ができる.係数$\Psi^n_k$は
\begin{align*}
\int^{2\pi}_{0}\frac{d\theta}{2\pi} e^{-in'\theta}\Psi_k^\theta =&\sum_{n=-\infty}^\infty \Psi_k^n \int^{2\pi}_{0}\frac{d\theta}{2\pi} e^{i(n-n')\theta} =\Psi_k^{n'}
\end{align*}
で得られる.ここで$n=n'$ならば
\begin{align*}
\int^{2\pi}_{0}\frac{d\theta}{2\pi} e^{i(n-n')\theta}=\int^{2\pi}_0 \frac{d\theta}{2\pi}=1
\end{align*}
であり,$n\neq n'$ならば$n-n'\in \mathbb{Z}-\{0\}$より
\begin{align*}
\int^{2\pi}_{0}\frac{d\theta}{2\pi} e^{i(n-n')\theta}=\left[\frac{e^{i(n-n')\theta}}{i(n-n')}\right]_0^{2\pi}=0
\end{align*}
となるから
\begin{align*}
\int^{2\pi}_{0}\frac{d\theta}{2\pi} e^{i(n-n')\theta}=\delta_{nn'}
\end{align*}
であることを用いた.後者のローレンツ群の2価表現$\Psi^{2\pi}_k=-\Psi^0_k$についても考える.これは周期$2\pi$の周期関数にはなっていないが,$e^{i\theta/2}\Psi^\theta_k$は周期関数になっていることがすぐにわかる.実際$\theta=2\pi$を代入すると
\begin{align*}
e^{i\pi}\Psi^{2\pi}_k=(-1)(-\Psi_k^0)=\Psi_k^0=e^{i0/2}\Psi^0_k
\end{align*}
となり周期性が確かめられる.したがってこちらは整数$n$でフーリエ級数展開
\begin{align*}
e^{i\theta/2}\Psi_k^\theta=&\sum_{n=-\infty}^\infty \Psi_k^{n-\frac{1}{2}} e^{in\theta} \\
\therefore \quad \Psi_k^\theta=&\sum_{n=-\infty}^\infty \Psi_k^{n-\frac{1}{2}} e^{i(n-\frac{1}{2})\theta}
\end{align*}
ができる.係数$\Psi^{n-\frac{1}{2}}_k$は再び
\begin{align*}
\Psi^{n-\frac{1}{2}}_k=&\int^{2\pi}_0 \frac{d\theta}{2\pi} e^{-in\theta} e^{i\theta/2}\Psi^\theta_k \\
=&\int^{2\pi}_0 \frac{d\theta}{2\pi} e^{-i(n-\frac{1}{2})\theta} \Psi^\theta_k
\end{align*}
となる.簡単のため,和の$n$を半整数$\pm1/2,\pm3/2,\cdots $をとるように約束すれば,こちらも同様の形
\begin{align*}
\Psi_k^\theta= \sum_{n=\mathbb{Z}+1/2} \Psi^n_k e^{in\theta}
\end{align*}
で書ける.したがってspin基底を,このフーリエ変換によって
\begin{align*}
\Psi_{k}^n:=\int^{2\pi}_{0} \frac{d\phi}{2\pi}e^{-in\phi} \Psi_{k}^\phi
\end{align*}
として得る.$n$は以上の議論より,ローレンツ群の1価表現に対しては整数をとり,2価表現に対しては半整数をとる.これに$J_3$を作用させると
\begin{align*}
J_3 \Psi_{k}^n=&\int^{2\pi}_0 \frac{d\phi}{2\pi}e^{-in\phi} J_3\Psi_{k}^\phi \\
=&\int^{2\pi}_0 \frac{d\phi}{2\pi}e^{-in\phi} \left(-i\frac{\partial}{\partial \phi}\right)\Psi_{k}^\phi \\
=&[e^{-in\phi}\Psi_{k}^\phi]_0^{2\pi}-\int^{2\pi}_0 \frac{d\phi}{2\pi}\left(-i\frac{\partial}{\partial \phi} e^{-in\phi} \right) \Psi_{k}^\phi \\
=&e^{-i2\pi n}\Phi_k^{2\pi}-\Psi_{k}^{0}+n\Psi_{k}^n
\end{align*}
となる.第一項目と第二項目は$\Psi^\theta_k$が1価と2価の表現のどちらの場合でも,それぞれ対応して$n$が整数と半整数をとることに留意すればいずれの場合も打ち消しあうことがわかる.この結果
\begin{align*}
J_3 \Psi_{k}^n=n\Psi^n_k
\end{align*}
となり,$\Psi_k^n$は$J_3$の固有状態となる.そしてその固有値である$n$はヘリシティである.この状態に$T_\pm =A\pm iB$を作用させることを考えると,これは昇降演算子
\begin{align*}
[J_3,T_\pm]=&[J_3,A]\pm i[J_3,B]=iB\pm A=\pm T_\pm \\
J_3 T_\pm \Psi_{k}^n=&(n \pm 1)T_\pm \Psi_{k}^n \\
\therefore T_{\pm}\Psi_{k}^n \propto & \Psi_{k}^{n\pm 1}
\end{align*}
である.よって$T_{\pm}$でヘリシティ$n$を1ずつ上げ下げすることができる.もちろんあらわに書けば
\begin{align*}
T_\pm \Psi^n_k=&\int^{2\pi}_{0} \frac{d\phi}{2\pi}e^{-in\phi} (A\pm iB)\Psi_{k}^\phi \\
=&\int^{2\pi}_{0} \frac{d\phi}{2\pi}e^{-in\phi} (\rho \cos i\phi\mp \rho i\sin\phi )\Psi_{k}^\phi \\
=&\rho\int^{2\pi}_{0} \frac{d\phi}{2\pi}e^{-i(n\pm1)\phi} \Psi_{k}^\phi \\
=&\rho \Psi^{n\pm 1}_k
\end{align*}
となる.量子力学で$SO(3)$の表現を調べたときのように,直感的には$n$に上限$j$と下限$-j$が存在すると予想するかもしれない.しかし,$n$の最大・最小値が有限であることが$SO(3)$で得られたのは本質的にはカシミア演算子$\mathbf{J}^2=J_1^2+J_2^2+J_3^2$の中に$J_3$が入っていたからであり,今回のカシミア演算子$W^\mu W_\mu =\kappa^2(A^2+B^2)$の中に$J_3$は入っていない.つまりこの$J_3$の固有値は$T_\pm$で無限に上げ下げすることができてしまい,上限や下限は存在しない.以上よりspin基底は無限個$\{\Psi_{k}^n\}_{n\in \mathbb{Z}\,\mathrm{or}\,\mathbb{Z}+1/2}$存在する.よって$ISO(2)$の無限次元ユニタリー表現をなし,ヘリシティ$n$は一つの表現につき$n=0,\pm 1,\pm 2\cdots$の\uwave{全て},あるいは$n=\pm 1/2,\pm3/2,\cdots $の\uwave{全て}をとる.この表現に対する$U(W)$の作用は
\begin{align*}
U(W[\theta,\alpha,\beta])\Psi_k^n=&\int^{2\pi}_0 \frac{d\phi}{2\pi}e^{-in\phi}U(W[\theta,\alpha,\beta])\Psi_{k}^\phi \\
=&\int^{2\pi}_{0} \frac{d\phi}{2\pi} e^{-in\phi} \int^{2\pi}_0 \frac{d\phi'}{2\pi} D_{\phi'\phi}[\theta,\alpha,\beta] \Psi_k^{\phi'} \\
=&\int^{2\pi}_{0} \frac{d\phi}{2\pi} e^{-in\phi} \int^{2\pi}_0 \frac{d\phi'}{2\pi} D_{\phi'\phi}[\theta,\alpha,\beta] \sum_{n'} e^{in'\phi'}\Psi_k^{n'} \\
=&\sum_{n'}\left(\int^{2\pi}_{0}\frac{d\phi d\phi'}{(2\pi)^2}e^{in'\phi'}D_{\phi'\phi}[\theta,\alpha,\beta]e^{-in\phi}\right)\Psi_k^{n'} \\
=&\sum_{n'}D_{n'n}[\theta,\alpha,\beta] \Psi_k^{n'}
\end{align*}
となる.ここで行列$D_{n'n}[\theta,\alpha,\beta]$を調べよう.これは元々の$D_{\phi'\phi}[\theta,\alpha,\beta]$を代入することで
\begin{align*}
D_{n'n}[\theta,\alpha,\beta]:=&\int^{2\pi}_{0}\frac{d\phi d\phi'}{(2\pi)^2}e^{in'\phi'}D_{\phi'\phi}[\theta,\alpha,\beta]e^{-in\phi} \\
=&\int^{2\pi}_{0}\frac{d\phi d\phi'}{(2\pi)^2}e^{in'\phi'} (2\pi)\delta(\phi'-\phi-\theta)\exp(i\rho\alpha \cos\phi'+i\rho\beta \sin\phi')e^{-in\phi} \\
=&\int^{2\pi}_0 \frac{d\phi'}{2\pi}e^{in'\phi'} \exp(i\rho\alpha \cos\phi'+i\rho\beta \sin\phi')e^{-in(\phi'-\theta)}
\end{align*}
と書けて,さらに$R=\sqrt{\alpha^2+\beta^2},\varphi=\tan^{-1} (\beta/\alpha)$とおけば指数部分で三角関数の合成ができて
\begin{align*}
i\rho\alpha \cos\phi'+i\rho\beta \sin\phi'=i\rho R \cos(\phi'-\varphi)
\end{align*}
と書ける.さらにここでベッセル関数の母関数の公式
\begin{align*}
e^{\frac{z}{2}(\omega -\frac{1}{\omega})}=&\sum_{m=-\infty}^\infty J_m(z)\omega^n \\
\therefore \quad e^{iz\cos\theta}=&\sum_{m=-\infty}^\infty i^m J_m (z) e^{im\theta}
\end{align*}
を使うと
\begin{align*}
e^{i\rho R\cos(\phi'-\varphi)}=&\sum_{m=-\infty}^\infty i^m J_m(\rho R) e^{im(\phi'-\varphi)} 
\end{align*}
と書くことができるから
\begin{align*}
D_{n'n}[\theta,\alpha,\beta]=&\sum_{m=-\infty}^\infty \int^{2\pi}_0 \frac{d\phi'}{2\pi}e^{in'\phi'} i^m J_m(\rho R) e^{im(\phi'-\varphi)} e^{-in(\phi'-\theta)} \\
=&e^{in\theta} \sum_{m=-\infty}^\infty i^m e^{-im\varphi} J_m(\rho R) \int^{2\pi}_0 \frac{d\phi'}{2\pi} e^{i(n'-n+m)\phi'} \\
=&e^{in\theta} \sum_{m=-\infty}^\infty i^m e^{-im\varphi} J_m(\rho R) \delta_{n-n',m} \\
=&e^{in\theta} (i e^{-i\varphi})^{n-n'} J_{n-n'}(\rho R)=e^{in\theta} (i e^{i\varphi})^{n'-n} J_{n'-n}(\rho R)
\end{align*}
最後の変形はベッセル関数の性質$J_{-k}(z)=(-1)^k J_k(z)$と$-i=i^{-1}$を用いた.(添え字の順番を合わせたかっただけ.)したがって,小群はspin基底に対してベッセル関数$J_{n'-n}(\rho R)$として作用する.さて,$\rho\to 0$の極限をとろう.ベッセル関数は
\begin{align*}
J_n(z)=\sum_{m=0}^\infty \frac{(-1)^m}{m! \Gamma(m+n+1)}\left(\frac{z}{2}\right)^{2m+n}
\end{align*}
と級数展開できることを思い出すと,$J_{n'-n}(\rho R)$は$n'-n=0$のとき$J_{n'-n}=1$を与え,そうでなければ$J_{n'-n}=0$を与える.したがって極限$\rho\to 0$は$J_{n'-n}\to \delta_{n'n}$と書ける.したがって,$ISO(2)$を$SO(2)$に縮約(contraction)する極限$\rho\to 0$で
\begin{align*}
D_{n'n}[\theta,\alpha,\beta]\to e^{in\theta}\delta_{n'n}
\end{align*}
となる.これは質量ゼロ有限次元表現における小群のユニタリー表現(本文p101参照)に他ならない!すなわち,spin基底にて$\rho\to 0$の極限をとることによってベッセル関数の性質からちゃんと通常の有限次元表現が回復する.\par
最後に,spin基底のユニタリー性および直交関係を考えて終わりとする.行列$D_{n'n}[\theta,\alpha,\beta]$は
\begin{align*}
(D^\dagger D)_{n'n}=&\sum_{m}D_{mn'}^* D_{mn} \\
=&\sum_m \int^{2\pi}_{0}\frac{d\phi d\phi'}{(2\pi)^2}e^{-im\phi'}D_{\phi'\phi}e^{+in'\phi} \int^{2\pi}_{0}\frac{d\theta d\theta'}{(2\pi)^2}e^{im\theta'}D^*_{\theta'\theta}e^{-in\theta} \\
=& \int^{2\pi}_{0}\frac{d\phi d\phi'}{(2\pi)^2}\int^{2\pi}_{0}\frac{d\theta d\theta'}{(2\pi)^2} D^*_{\phi'\phi} D_{\theta'\theta}e^{+in'\phi} e^{-in\theta}\sum_m e^{im(\theta'-\phi')} \\
=&\int^{2\pi}_{0}\frac{d\phi d\phi'}{(2\pi)^2}\int^{2\pi}_{0}\frac{d\theta d\theta'}{(2\pi)^2} D^*_{\phi'\phi} D_{\theta'\theta}e^{+in'\phi} e^{-in\theta}(2\pi)\delta(\theta'-\phi') \\
=&\int^{2\pi}_{0}\frac{d\phi d\phi'}{(2\pi)^2}\int^{2\pi}_{0}\frac{d\theta}{2\pi} D_{\phi\phi'}^\dagger D_{\phi'\theta}e^{+in'\phi} e^{-in\theta} \\
=&\int^{2\pi}_{0}\frac{d\phi}{2\pi}\int^{2\pi}_{0}\frac{d\theta}{2\pi} (D^\dagger D)_{\phi\theta}e^{+in'\phi} e^{-in\theta} \\
=&\int^{2\pi}_{0}\frac{d\phi}{2\pi}\int^{2\pi}_{0}\frac{d\theta}{2\pi} (2\pi)\delta(\phi-\theta) e^{+in'\phi} e^{-in\theta} \\
=&\int^{2\pi}_{0}\frac{d\phi}{2\pi}e^{+i(n'-n)\phi} \\
=&\delta_{n'n}
\end{align*}
であるからユニタリー行列である.したがってspin基底を
\begin{align*}
(\Psi_{k'}^{n'},\Psi_k^n)=\delta^3(k'-k)\delta_{n'n}
\end{align*}
と正規直交化すれば,これもまたユニタリー表現をなす.
\begin{align*}
\left(\sum_{m'}D_{m'n'} \Psi_{k'}^{m'},\sum_m D_{mn}\Psi_k^{m}\right) =&\sum_{m'm}D_{m'n'}^* D_{mn}(\Psi_{k'}^{m'},\Psi_k^{m}) \\
=&\sum_{m'm}D_{n'm'}^\dagger D_{mn}\delta^3(k'-k)\delta_{m'm} \\
=&\delta^3(k'-k)(D^\dagger D)_{n'n} \\
=&\delta^3(k'-k)\delta_{n'n}=(\Psi_{k'}^{n'},\Psi_k^n)
\end{align*}
このように構成した表現は$ISO(2)$の無限次元ユニタリー表現に従い,連続スピン表現(Continuous Spin Representation,CSR)などと呼ばれる.\par
流石にこれ以上はうまく説明できなかったのでもっと詳しくはPhilip Schusterらによる2013年の「On the Theory of Continuous-Spin Particles: Wavefunctions and Soft-Factor Scattering Amplitudes」を参照のこと.多分ワインバーグからこの論文へ,以上の説明で滑らかに接続できたと思う.

\vskip\baselineskip

\textbf{6.Massless \& Negative-Energy \& Infinite-Spin Representation}\par
上の場合と同様だが,エネルギーの符号だけが異なる.基準運動量は$k^\mu=(0,0,\kappa,-\kappa)$であり,それぞれの固有値は
\begin{align*}
P^\mu P_\mu :& m^2>0 \\
W^\mu W_\mu :&\rho^2>0 \\
P^\mu :&p^2=0,\quad p^0 < 0\\
W^3:&w/\kappa=\{0,\pm 1,\pm2 ,\cdots \} \mathrm{or} \{\pm 1/2 ,\pm 3/2,\cdots \}
\end{align*}
となる.これ以上は特に説明しない.

\vskip\baselineskip

\textbf{7.Zero-Momentum(Vacuum) Representation}\par
これは$p^\mu=0$となる,所謂真空状態に対応する表現である.基準運動量は当然$k^\mu=(0,0,0,0)$であり,この場合の小群は$SO(3,1)$そのものとなり,明らかにコンパクトではない.したがって真空に対するローレンツ群の作用が自明
\begin{align*}
\exp\left(\frac{i}{2}\omega_{\mu\nu}J^{\mu\nu}\right)\Psi_{0}=&\Psi_0 \\
\Leftrightarrow \quad J^{\mu\nu}\Psi_0=&0
\end{align*}
でない限り無限次元表現とならざるを得ない.真空は複数種類存在してもよいが,それぞれの真空はこの条件を満たしていると基本的に仮定する.それぞれの固有値は
\begin{align*}
P^\mu P_\mu :& m^2=0 \\
W^\mu W_\mu :&\rho^2=0 \\
P^\mu :&p^\mu=0\\
W^3:&w=0
\end{align*}
となる.無限次元表現となる場合を考えたらなんか深堀りできる気もするが,よくわからなかったのでここではこれ以上は考えない.

\vskip\baselineskip

\textbf{8.Tachyon Representation}\par
これは$m^2<0$となり,したがって$m$が虚数となる場合の表現である.基準運動量は$k^\mu=(0,0,m,0)$となり,小群は3軸固定とした時間と空間を混ぜる$SO(2,1)$となる.この表現はよくわからなかったので,ここではこれ以上考えない.

\vskip\baselineskip

以上でポアンカレ代数の分類が終わった.多分これ以上の表現が存在しないんじゃないかと思う.もちろん,物理的に存在するだろうと考えられているのは(1)のMassiveかつ正エネルギーの表現と(3)の質量ゼロかつ正エネルギーの有限次元表現の場合だけである.もしかしたら(5)の連続スピン表現に属する粒子もあるかもしれないが,全く見つかっていない.\par
表現が考えられるからといって,対応する粒子が自然界に存在するかどうかは全くの別問題である.物理的でないという理由で排除した表現以外にも,無質量正エネルギーでヘリシティ2以上の粒子なども見つかっていない.実際,理論的にも無質量でヘリシティが2より大きい粒子の理論で,ちゃんと相互作用のある理論を作ることはできないと言われている.\footnote{自由場の理論はゲージ理論として普通に作ることができる.しかし,ヘリシティが2より大きい質量ゼロ粒子(高階スピン粒子)の理論はゲージ対称性の制限がきつすぎて相互作用項が書けないと言われているらしい.実際この本の13.1節でやる低エネルギー定理からも高階スピンが長距離力の媒介をすることはできないことがわかる.ただし,これはミンコフスキー時空上の理論の場合で,AdS時空上だと高階スピンの理論が作れるらしい.Vasiliev theoryとかhigher-spin gauge theoryとかで調べると色々出てくる.まじで意味わからんかったけど.}

\vskip\baselineskip

少し微妙な点だけ述べて終わりにする.上で2次と4次のカシミア演算子を作り,その固有値を分類することでポアンカレ代数の表現の分類をしたのだったが,本当はこれ以上高次のカシミア演算子が存在しないことを示す必要がある.もっとちゃんと言えば,6次や8次やそれより高次のカシミア演算子が作れたとしても,それは2次と4次のカシミア演算子の自明な積から構築されたものであって,
Pauli-Lubanskiベクトルのような非自明な積から作られる高次のカシミア演算子はもう存在しない,ということを示す必要がある.しかしそれはかなり大変だ.半単純なリー代数の場合は,その代数のランクとカシミア演算子の数が一致するという便利な定理(Chevalleyの定理)が存在するらしいが,ポアンカレ代数$\mathfrak{iso}(3,1)$は運動量$P^\mu$が不変可換部分代数をなし,半単純ではないためこの定理は使えない.しかしカシミア演算子の数が2個であることを「理解」する方法はある.それはde-Sitter群$SO(4,1)$からcontractionすることでポアンカレ群$ISO(3,1)$を得る方法である.\par
概要を述べる.$SO(4,1)$の代数は,($\mu,\nu,\rho,\sigma$はそれぞれ$0,1,\cdots ,4$をとるとして)
\begin{align*}
i[M^{\mu\nu},M^{\rho\sigma}]=\eta^{\nu\rho}M^{\mu\sigma}-\eta^{\mu\rho}M^{\nu\sigma}-\eta^{\mu\sigma}M^{\rho\nu}+\eta^{\nu\sigma}M^{\rho\mu}
\end{align*}
となっている.この代数は半単純でありそのランクは2であることが実は示される.したがってカシミア演算子の数も上の定理により2個であることが言える.しかしここで$J^{\mu\nu}:=M^{\mu\nu},P^\mu:=\frac{1}{R} M^{\mu 4}(\mu,\nu=0,1,\cdots ,3)$とする,つまり4番目の次元方向との回転を「並進」とみなすと
\begin{align*}
i[M^{\mu 4},M^{\rho\sigma}]=&\eta^{4\rho}M^{\mu\sigma}-\eta^{\mu\rho}M^{4\sigma}-\eta^{\mu\sigma}M^{\rho4}+\eta^{4\sigma}M^{\rho\mu} \\
i[M^{\mu4},M^{\rho 4}]=&\eta^{4\rho}M^{\mu4}-\eta^{\mu\rho}M^{44}-\eta^{\mu4}M^{\rho4}+\eta^{44}M^{\rho\mu}
\end{align*}
より
\begin{align*}
i[J^{\mu\nu},J^{\rho\sigma}]=&\eta^{\nu\rho}J^{\mu\sigma}-\eta^{\mu\rho}J^{\nu\sigma}-\eta^{\mu\sigma}J^{\rho\nu}+\eta^{\nu\sigma}J^{\rho\mu} \\
i[P^\mu,J^{\rho\sigma}]=&\eta^{\mu\rho}P^\sigma -\eta^{\mu\sigma}P^\rho \\
i[P^\mu,P^\nu]=&-\frac{1}{R^2}J^{\mu\nu}
\end{align*}
となる.ここで$R$は縮約パラメータであり,$R\to \infty$の極限をとると\footnote{この極限は,de-Sitter時空の曲率半径を無限大に飛ばし平坦時空(ミンコフスキー空間)にする,という物理的描像と対応しているらしい.}これはポアンカレ代数$\mathfrak{iso}(3,1)$に一致する.これにより$ISO(3,1)$のカシミア演算子も2個であることが「上から降ってくる」形で理解できる.



\newpage

\subsection{空間反転と時間反転}
2.3節で,どんな斉次ローレンツ変換も固有順時(つまり$\det \Lambda=+1$かつ$\tensor{\Lambda}{^0_0}\geq +1$)か,または固有順時変換に$\mathcal{P}$か$\mathcal{T}$か$\mathcal{P T}$をかけたものだとわかったのだった.ここで$\mathcal{P},\mathcal{T}$はそれぞれ以下の空間反転と時間反転だ.
\begin{align*}
\tensor{\mathcal{P}}{^\mu_\nu}=\left(
\begin{matrix}
-1 & 0 & 0 & 0 \\
0 & -1 & 0 & 0 \\
0 & 0 & -1 & 0 \\
0 & 0& 0& 1
\end{matrix}
\right) ,\quad \tensor{\mathcal{T}}{^\mu_\nu}=\left(
\begin{matrix}
1 & 0 & 0 & 0 \\
0 & 1 & 0 & 0 \\
0 & 0 & 1 & 0 \\
0 & 0& 0& -1
\end{matrix}
\right)
\end{align*}
かつては,ポアンカレ群の基本的な積の法則
\begin{align*}
U(\bar{\Lambda},\bar{a})U(\Lambda,a)=U(\bar{\Lambda}\Lambda ,\bar{\Lambda}a+\bar{a})
\end{align*}
は,$\Lambda$と$\bar{\Lambda}$の両方かどちらかに$\mathcal{P}$か$\mathcal{T}$か$\mathcal{PT}$が含まれていても明らかに正しいと考えられていた.特に,$\mathcal{P},\mathcal{T}$に対応する(線形ユニタリー・反線形反ユニタリー)演算子そのものが存在,つまり
\begin{align*}
\mathsf{P}\equiv U(\mathcal{P},0),\quad \mathsf{T}\equiv U(\mathcal{T},0)
\end{align*}
が任意の固有順時ローレンツ変換$\tensor{\Lambda}{^\mu_\nu}$と並進$a^\mu$に対して(ポアンカレ群において$U(\bar{\Lambda},0)U(\Lambda,a)U^{-1}(\bar{\Lambda},0)=U(\bar{\Lambda}\Lambda\bar{\Lambda}^{-1},\bar{\Lambda}a)$であることに対応し)
\begin{align*}
\mathsf{P}U(\Lambda,a)\mathsf{P}^{-1}=U(\mathcal{P}\Lambda \mathcal{P}^{-1},\mathcal{P}a) \\
\mathsf{T}U(\Lambda,a)\mathsf{T}^{-1}=U(\mathcal{T}\Lambda \mathcal{T}^{-1},\mathcal{T}a)
\end{align*}
を満たすと信じられていた.これらの変換則は,$\mathsf{P}$か$\mathsf{T}$が保存する,ということをほぼ意味している.(後ほど示す(2.6.13)を参照)\par
1956-57年に,$\mathsf{P}$についての変換則は,原子核のベータ崩壊などのおこす弱い相互作用を無視したときにのみ近似的に正しいと理解されるようになった.時間反転はそれからしばらく生き延びたが,1964年に$\mathsf{T}$について,これはやはり近似的正しいという間接的証拠が見つかった.(3.3節参照)以下では,(2.6.1)(2.6.2)を満たす演算子$\mathsf{P},\mathsf{T}$が実際に存在するとして話を進めるが,以上の話よりこれは近似でしかない.

\vskip\baselineskip

(2.6.1)(2.6.2)を微小変換
\begin{align*}
\tensor{\Lambda}{^\mu_\nu}=\delta^\mu_\nu+\tensor{\omega}{^\mu_\nu},\quad a^\mu=\epsilon^\mu
\end{align*}
に適用する.ここで$\omega_{\mu\nu}=-\omega_{\nu\mu}$と$\epsilon_\mu$はともに微小量とする.(2.4.3)を使い,(2.6.1)(2.6.2)の$\omega_{\rho\sigma},\epsilon_\rho$の係数を等しいとして,ポアンカレ生成子の$\mathsf{P},\mathsf{T}$変換性を(2.4.8)(2.4.9)と同様にして以下を得る.
\begin{align*}
\mathsf{P} iJ^{\rho\sigma} \mathsf{P}^{-1} =&i\tensor{\mathcal{P}}{_\mu^\rho}\tensor{\mathcal{P}}{_\nu^\sigma}J^{\mu\nu} \\
\mathsf{P}iP^\rho \mathsf{P}^{-1} =&i\tensor{\mathcal{P}}{_\mu^\rho}P^\mu \\
\mathsf{T} iJ^{\rho\sigma} \mathsf{T}^{-1} =&i\tensor{\mathcal{T}}{_\mu^\rho}\tensor{\mathcal{T}}{_\nu^\sigma}J^{\mu\nu} \\
\mathsf{T}iP^\rho \mathsf{T}^{-1} =&i\tensor{\mathcal{T}}{_\mu^\rho}P^\mu
\end{align*}
これは(2.4.8)(2.4.9)とよく似ているが,いまはまだ$\mathsf{P},\mathsf{T}$が線形ユニタリーか反線形反ユニタリーかを決めていないので,これらの式で両辺の$i$を打ち消してはいけない.\par
線形・反線形性を決めるのは簡単だ.(2.6.4)で$\rho=0$とすれば$P^0=H$より
\begin{align*}
\mathsf{P}iH\mathsf{P}^{-1}=iH
\end{align*}
となる.ここで$H$はエネルギー演算子だ.もし,$\mathsf{P}$が反線形反ユニタリーなら,$i$と反可換であり
\begin{align*}
\mathsf{P}H \mathsf{P}^{-1}=-H
\end{align*}
となる.しかしこのときには,エネルギーが$E>0$のどんな状態$\Psi$にも,エネルギーが$-E<0$の別の状態$\mathsf{P}^{-1}$が付随することになる.
\begin{align*}
H\Psi=&E\Psi \\
\mathsf{P} H\mathsf{P}^{-1} \mathsf{P} \Psi=&E\mathsf{P} \Psi \\
H(\mathsf{P}\Psi)=&-E(\mathsf{P}\Psi)
\end{align*}
負のエネルギー(真空より低いエネルギー)の状態は存在しないので,別の選択が必要である.すなわち,$\mathsf{P}$は線形ユニタリーで,$H$と可換$\mathsf{P}H \mathsf{P}^{-1}=H$であり,反可換ではない.\par
一方,(2.6.6)で$\rho=0$とすると
\begin{align*}
\mathsf{T}iH\mathsf{T}^{-1}=-iH
\end{align*}
となるが,先ほどと同様に,もし$\mathsf{T}$が線形ユニタリーならば単に$i$を打ち消し$\mathsf{T}H\mathsf{T}^{-1}=-H$となるが,これも負のエネルギー$-E$の状態が生じる.
\begin{align*}
H\Psi=&E\Psi \\
\mathsf{T} H\mathsf{T}^{-1}\mathsf{T}\Psi=&E\mathsf{T}\Psi \\
H(\mathsf{T}\Psi)=&-E(\mathsf{T}\Psi)
\end{align*}
したがって$\mathsf{T}$は反線形反ユニタリーだ.

\vskip\baselineskip

$\mathsf{P}$は線形で$\mathsf{T}$は反線形だと決めたので,(2.6.3)-(2.6.6)を三次元記法での生成子(2.4.15)-(2.4.17)を使って書いておく.
\begin{align*}
\mathsf{P}iJ_k\mathsf{P}^{-1}=&\mathsf{P}\frac{i}{2}\epsilon_{ijk}J^{jk}\mathsf{P}^{-1}=\frac{i}{2}\epsilon_{ijk}\tensor{\mathcal{P}}{_\mu^j}\tensor{\mathcal{P}}{_\nu^k}J^{\mu\nu}=\frac{i}{2}\epsilon_{ijk}(-\delta^j_\mu)(- \delta^k_\nu) J^{\mu\nu}=\frac{i}{2}\epsilon_{ijk}J^{ij}=iJ_k \\
\therefore \quad \mathsf{P}\mathbf{J}\mathsf{P}^{-1}=&+\mathbf{J} \\
\mathsf{P} iK_i\mathsf{P}^{-1}=&\mathsf{P} i(-J^{i0})\mathsf{P}^{-1}=-i\tensor{\mathcal{P}}{_\mu^i}\tensor{\mathcal{P}}{_\nu^0}J^{\mu\nu}=-i(-\delta^i_\mu) \delta^0_\nu J^{\mu\nu}=iJ^{i0}=-iK_i \\
\therefore \quad \mathsf{P} \mathbf{K}\mathsf{P}^{-1}=&-\mathbf{K} \\
\mathsf{P} iP^i \mathsf{P}^{-1}=&i\tensor{\mathcal{P}}{_\mu^i}P^\mu=i(-\delta^{i}_\mu) P^\mu=-iP^i \\
\therefore \quad \mathsf{P} \mathbf{P} \mathsf{P}^{-1}=&-\mathbf{P} \\
\mathsf{T}iJ_k\mathsf{T}^{-1}=&\mathsf{T}\frac{i}{2}\epsilon_{ijk}J^{jk}\mathsf{T}^{-1}=\frac{i}{2}\epsilon_{ijk}\tensor{\mathcal{T}}{_\mu^j}\tensor{\mathcal{T}}{_\nu^k}J^{\mu\nu}=\frac{i}{2}\epsilon_{ijk}\delta^j_\mu \delta^k_\nu J^{\mu\nu}=\frac{i}{2}\epsilon_{ijk}J^{ij}=iJ_k \\
\therefore \quad \mathsf{T}\mathbf{J}\mathsf{T}^{-1}=&-\mathbf{J} \\
\mathsf{T} iK_i\mathsf{T}^{-1}=&\mathsf{T} i(-J^{i0})\mathsf{T}^{-1}=-i\tensor{\mathcal{T}}{_\mu^i}\tensor{\mathcal{T}}{_\nu^0}J^{\mu\nu}=-i\delta^i_\mu(- \delta^0_\nu) J^{\mu\nu}=iJ^{i0}=-iK_i \\
\therefore \quad \mathsf{T} \mathbf{K}\mathsf{T}^{-1}=&+\mathbf{K} \\
\mathsf{T} iP^i \mathsf{T}^{-1}=&i\tensor{\mathcal{T}}{_\mu^i}P^\mu=i \delta^{i}_\mu P^\mu=iP^i \\
\therefore \quad \mathsf{T} \mathbf{P} \mathsf{T}^{-1}=&-\mathbf{P}
\end{align*}
そして,そうであるように決めたように
\begin{align*}
\mathsf{P}H\mathsf{P}^{-1}=\mathsf{T}H\mathsf{T}^{-1}=H
\end{align*}
が成立する.\par
空間反転$\mathsf{P}$が$\mathbf{J}$の符号を保つのは物理的に自然だ.なぜなら軌道部分はベクトル積$\mathbf{r}\times \mathbf{p}$であり,$\mathbf{r}$も$\mathbf{p}$も空間反転で符号が同時に変わるからだ.\par
一方,$\mathsf{T}$は$\mathbf{J}$を反転する.これは時間反転の後では,観測者はすべての物体が反対方向に回っているのを見ることになり,これは運動量のみが符号を変わるからだ.(ちなみに$\mathbf{J}\times \mathbf{J}=i\mathbf{J}$は,同じベクトルの外積によってゼロになりそうだが,これは
\begin{align*}
(\mathbf{J}\times \mathbf{J})_i=&\epsilon_{ijk}J_j J_k \\
=&\frac{1}{2}\epsilon_{ijk}J_j J_k -\frac{1}{2}\epsilon_{ijk}J_k J_j =\frac{1}{2}\epsilon_{ijk}[J_j ,J_k] \\
=&\frac{i}{2}\epsilon_{ijk}\epsilon_{jkl}J_l \\
=&i\delta_{il}J_l =i J_i \quad \because \epsilon_{ijk}\epsilon_{jkl}=2\delta_{il}
\end{align*}
とすればわかる.演算子は通常のベクトルのようには交換できない.)\par
ところで(2.6.10)は角運動量の交換関係$\mathbf{J}\times \mathbf{J}=i\mathbf{J}$と矛盾しない.一見,右辺は負符号が二つ出現し,右辺は一つだけが出現するように見えるが,$\mathsf{T}$は$\mathbf{J}$のみならず,$i$の符号も変えるからだ.(2.6.7)-(2.6.13)が(2.4.18)-(2.4.24)と矛盾しないことは,すべて容易に確かめられる.

\vskip\baselineskip

つぎに$\mathsf{P}$と$\mathsf{T}$が1粒子状態にどのように作用するかを調べる.\par
$\mathsf{P}:M>0$\par
1粒子状態$\Psi_{k,\sigma}$は($k^\mu=(0,0,0,M)$なので,)$\mathbf{P},H,J_3$がそれぞれ固有値$0,M,\sigma$をもつ状態とする.
(2.6.7)(2.6.9)(2.6.13)より
\begin{align*}
\mathbf{P} \Psi_{k,\sigma}=0 \quad \Rightarrow&\quad  \mathbf{P}(\mathsf{P}\Psi_{k,\sigma})= -\mathsf{P} \mathbf{P} \Psi_{k,\sigma} =0\cdot(\mathsf{P}\Psi_{k,\sigma}) \\
H \Psi_{k,\sigma}=M\Psi_{k,\sigma} \quad \Rightarrow& \quad H(\mathsf{P}\Psi_{k,\sigma})= \mathsf{P} H \Psi_{k,\sigma} =M\cdot(\mathsf{P}\Psi_{k,\sigma}) \\
J_3 \Psi_{k,\sigma}=\sigma \Psi_{k,\sigma}\quad \Rightarrow&\quad  J_3(\mathsf{P}\Psi_{k,\sigma})= \mathsf{P} J_3 \Psi_{k,\sigma} =\sigma\cdot(\mathsf{P}\Psi_{k,\sigma})
\end{align*}
となり$\mathsf{P}\Psi_{k,\sigma}$も同じ固有値をもつ状態である.よってこれらの状態は(縮退がないとして)高々位相だけしか違わない.
\begin{align*}
\mathsf{P}\Psi_{k,\sigma} =\eta_\sigma \Psi_{k,\sigma}
\end{align*}
ここで位相因子$|\eta|=1$はスピン$\sigma$に依存するかどうかはまだわからない.$\eta_\sigma$が$\sigma$に依存しないことを示すには,(2.5.8)(2.5.20)(2.5.21)から,$j$を粒子のスピン角運動量の大きさとして
\begin{align*}
(J_1\pm iJ_2)\Psi_{k,\sigma}=\sqrt{(j\mp \sigma)(j\pm \sigma+1)}\Psi_{k,\sigma\pm 1}
\end{align*}
であることを使う.両辺に$\mathsf{P}$を施すと
\begin{align*}
(\mathrm{LHS})=&\mathsf{P}(J_1\pm iJ_2)\mathsf{P}^{-1}\mathsf{P}\Psi_{k,\sigma}=(J_1\pm iJ_2)\eta_{\sigma}\Psi_{k,\sigma}=\eta_\sigma \sqrt{(j\mp \sigma)(j\pm \sigma+1)}\Psi_{k,\sigma\pm 1} \\
(\mathrm{RHS})=&\sqrt{(j\mp \sigma)(j\pm \sigma+1)}(\mathsf{P}\Psi_{k,\sigma\pm 1})=\eta_{\sigma\pm 1}\sqrt{(j\mp \sigma)(j\pm \sigma+1)}\Psi_{k,\sigma\pm 1} \\
\therefore &\quad \eta_\sigma =\eta_{\sigma\pm 1}
\end{align*}
となり,よって$\eta$は$\sigma$に依存しないことがわかる.そのため,次のように書けることがわかる.
\begin{align*}
\mathsf{P}\Psi_{k,\sigma}=\eta \Psi_{k,\sigma}
\end{align*}
ここで$\eta$は固有パリティと呼ばれる位相で,$\mathsf{P}$が作用する\uwave{粒子の種類のみ}に依る.\par
ゼロでない運動量をもつ状態を得るためには,ブースト(2.5.24)に対応するユニタリー演算子$U(L(p))$を施す.(2.5.5)(2.5.18)と$k^0=M$から
\begin{align*}
\Psi_{k,\sigma}=\sqrt{\frac{M}{p^0}}U(L(p))\Psi_{k,\sigma}
\end{align*}
ここで
\begin{align*}
\tensor{\mathcal{P}}{^\mu_\nu}\tensor{L(p)}{^\nu_\rho}\tensor{(\mathcal{P}^{-1})}{^\rho_\sigma}k^\sigma =&\tensor{\mathcal{P}}{^\mu_\nu}\tensor{L(p)}{^\nu_\rho} k^\rho \\
=&\tensor{\mathcal{P}}{^\mu_\nu}p^\nu=(\mathcal{P}p)^\mu=\tensor{L(\mathcal{P}p)}{^\mu_\nu}k^\nu \\
\therefore \quad  &\mathcal{P} L(p){\mathcal{P}}^{-1} =L(\mathcal{P} p) \\
\mathcal{P}p =&(-\mathbf{p},\sqrt{\mathbf{p}^2+M^2})
\end{align*}
したがって(2.6.1)(2.6.15)より
\begin{align*}
\mathsf{P}\Psi_{p,\sigma}=&\sqrt{\frac{M}{p^0}}\mathsf{P} U(L(p))\mathsf{P}^{-1} \mathsf{P}\Psi_{k,\sigma} \\
=&\sqrt{\frac{M}{p^0}}U(\mathcal{P}L(p)\mathcal{P}^{-1}) \eta \Psi_{k,\sigma}=\eta \sqrt{\frac{M}{p^0}}U(L(\mathcal{P}p))\Psi_{k,\sigma} \\
=&\eta \Psi_{\mathcal{P}p,\sigma}
\end{align*}
となる.

\vskip\baselineskip

$\mathsf{T}:M>0$\par
(2.6.10)(2.6.12)(2.6.13)より,$\mathsf{P}$のときと同様に
\begin{align*}
\mathbf{P}(\mathsf{T}\Psi_{k,\sigma})=0 ,\quad H(\mathsf{T}\Psi_{k,\sigma})=M(\mathsf{T}\Psi_{k,\sigma}) ,\quad J_3(\mathsf{T}\Psi_{k,\sigma})=-\sigma(\mathsf{T}\Psi_{k,\sigma})
\end{align*}
したがって
\begin{align*}
\mathsf{T}\Psi_{k,\sigma}=\zeta_\sigma \Psi_{k,-\sigma}
\end{align*}
ここで$\zeta_\sigma$は位相因子だ.再び演算子$\mathsf{T}$を(2.6.14)の両辺に作用させ,これは$\mathbf{J}$および虚数単位$ i $と反可換だから
\begin{align*}
(\mathrm{LHS})=&\mathsf{T}(J_1 \pm i J_2)\mathsf{T}^{-1} \mathsf{T} \Psi_{k,\sigma}=(-J_1\pm i J_2)\zeta_{\sigma}\Psi_{k,-\sigma} \\
=&-\zeta_\sigma \sqrt{(j\pm(-\sigma))(j\mp (-\sigma) +1)}\Psi_{k,-\sigma \mp 1} \\
=&-\zeta_\sigma \sqrt{(j \mp \sigma)(j\pm \sigma +1)}\Psi_{k,-\sigma \mp 1} \\
(\mathrm{RHS})=&\sqrt{(j\pm \sigma)(j\mp \sigma+1)}T\Psi_{k,\sigma\pm 1} \\
=&\sqrt{(j\mp \sigma)(j\pm \sigma +1)}\zeta_{\sigma\pm 1}\Psi_{k,-\sigma\mp 1} \\
\therefore \quad &-\zeta_{\sigma}=\zeta_{\sigma\pm 1}
\end{align*}
となる.これは粒子の種類にのみ依る別の位相因子$\zeta$を用いて,解を$\zeta_\sigma=(-1)^{j-\sigma}\zeta$と書ける.(実際これが$-\zeta_{\sigma}=\zeta_{\sigma\pm 1}$を満たすことはすぐ確かめられる.)よって
\begin{align*}
\mathsf{T} \Psi_{k,\sigma}=\zeta(-1)^{j-\sigma}\Psi_{k,-\sigma}
\end{align*}
となる.しかし,先ほどの固有パリティ$\eta$とは異なり,時間反転の位相$\zeta$はなんら物理的に重要ではない.これは一粒子状態の位相を
\begin{align*}
\Psi_{k,\sigma}\to \Psi_{k,\sigma}'=\zeta^{1/2} \Psi_{k,\sigma}
\end{align*}
と再定義すれば,時間反転の変換則から位相$\zeta$を完全に消すことができるからだ.
\begin{align*}
\mathsf{T}\Psi_{k,\sigma}'=&(\zeta^*)^{1/2}\mathsf{T}\Psi_{k,\sigma} \because \mathsf{T}は反ユニタリー \\
=&(\zeta^*)^{1/2} \zeta(-1)^{j-\sigma} \Psi_{k,-\sigma}=|\zeta|^{1/2} \zeta^{1/2} (-1)^{j-\sigma} \Psi_{k,-\sigma} \\
=&(-1)^{j-\sigma} \Psi_{k,-\sigma}' \quad \because |\zeta|=1
\end{align*}
以下では,(2.6.17)に任意の位相$\zeta$を置いておく.これは単に1粒子状態の位相を選べるようにしておくためだ.この位相は重要ではない.\par
ゼロでない運動量を扱うために,ブーストを再び施す.
\begin{align*}
\tensor{\mathcal{T}}{^\mu_\nu}\tensor{L}{^\nu_\rho}(p) \tensor{(\mathcal{T}^{-1})}{^\rho_\sigma} k^\sigma=&\tensor{\mathcal{T}}{^\mu_\nu}\tensor{L}{^\nu_\rho}(p) (-k^\rho) \quad \because k=(0,0,0,M)で時間成分だけマイナスにする \\
=&-\tensor{\mathcal{T}}{^\mu_\nu}p^\nu \\
=&-(p^1,p^2,p^3-,p^0)=(-p^1,-p^2,-p^3,p^0)=(\mathcal{P}p)^\mu=\tensor{L(\mathcal{P}p)}{^\mu_\sigma}k^\sigma
\end{align*}
したがって
\begin{align*}
\mathcal{T} L(p)\mathcal{T}^{-1}=L(\mathcal{P}p)
\end{align*}
(つまり,時間成分の添え字を奇数個持つ$\tensor{L}{^\mu_\nu}$の成分の符号をすべて変えることは,空間成分の添え字を奇数個持つ成分の符号をすべて変えることに同等.)$\mathsf{P}$のときと同様に(2.6.2)(2.5.5)より
\begin{align*}
\Psi_{p,\sigma}=&\sqrt{\frac{M}{p^0}}U(L(p))\Psi_{k,\sigma} \\
\Rightarrow \mathsf{T}\Psi_{p,\sigma}=&\sqrt{\frac{M}{p^0}}U(\mathcal{T})U(L(p))U(\mathcal{T})^{-1} \mathsf{T}\Psi_{k,\sigma} \quad \because U(\mathcal{T})=\mathsf{T} \\
=&\sqrt{\frac{M}{p^0}}U(\mathcal{T}L(p)\mathcal{T}^{-1})\mathsf{T}\Psi_{k,\sigma} \\
=&\sqrt{\frac{M}{p^0}}U(L(\mathcal{P}p))\mathsf{T}\Psi_{k,\sigma} \\
=&\zeta(-1)^{j-\sigma}\Psi_{\mathcal{P}p,-\sigma}
\end{align*}
次にゼロ質量の場合の考察をする.

\vskip\baselineskip

$\mathsf{P}:M=0$\par
状態$\Psi_{k,\sigma}$は固有値$k^\mu=(0,0,\kappa,\kappa)$を持つ$P^\mu$の固有ベクトルであり,また固有値$\sigma$を持つ$J_3$の固有ベクトルだ.この状態に演算子$\mathsf{P}$が作用した$\mathsf{P}\Psi_{k,\sigma}$の性質を調べる.(2.6.9)(2.6.13)より
\begin{align*}
P^\mu \mathsf{P}\Psi_{k,\sigma}=\left\{
\begin{array}{ll}
-\mathsf{P}\mathbf{P} \Psi_{k,\sigma}=-(0,0,\kappa)\mathsf{P}\Psi_{k,\sigma} \quad &(\mu=1,2,3)\\
\mathsf{P} H \Psi_{k,\sigma}=\kappa \mathsf{P}\Psi_{k,\sigma} \quad &(\mu=0)
\end{array}
 \right.
\end{align*}
となって,4元運動量が$(\mathcal{P}k)^\mu=(0,0,-\kappa,\kappa)$で,$J_3$の固有値は(2.6.7)より
\begin{align*}
J_3 \mathsf{P}\Psi_{k,\sigma}=+\mathsf{P}J_3 \Psi_{k,\sigma}=+\sigma \mathsf{P} \Psi_{k,\sigma}
\end{align*}
となって,$+\sigma$の状態であることがわかる.このため\uwave{ヘリシティ}$\sigma$の状態は,ヘリシティ$-\sigma$の状態になる.(ヘリシティは$\mathbf{J}\cdot \mathbf{P}$に比例していたことを思い出そう.運動量方向への射影が入っているので,反転によりマイナスがつく.)p102~p103で述べたように,空間反転対称性がここから成り立つためには,ゼロでないヘリシティ$\sigma$を持つどんな種類の質量ゼロ粒子も,その反対のヘリシティ$-\sigma$をもつ質量ゼロ粒子が伴わなければならない.非ゼロな場合と違い,今回の場合$\mathsf{P}$は基準運動量$k^\mu$を不変に保たないので,$\mathsf{P}\Psi_{k,\sigma}$をブーストして一般の運動量の場合の反転を調べることができない.(ブースト演算子は,自分で選んで固定した基準運動量$k$からしか$p$にブーストできないから.) \\
$\Rightarrow$代わりに演算子$U(R_2^{-1})\mathsf{P}$を考える方が便利.ここで$R_2$は$k$を$\mathcal{P}k$に変換する回転で,便宜上$R^{-1}_2$を2軸周りの$-180^\circ$回転とする.
\begin{align*}
R_2^{-1}=&\left(
\begin{matrix}
\cos(-\pi) & 0 & \sin(-\pi) & 0 \\
0 & 1 & 0 & 0 \\
-\sin(-\pi) & 0 & \cos(-\pi) & 0 \\
0 & 0 & 0 & 1 \\
\end{matrix}
\right) =\left(
\begin{matrix}
-1 & 0 & 0 & 0 \\
0 & 1 & 0 & 0 \\
0 & 0 & -1 & 0 \\
0 & 0 & 0 & 1 \\
\end{matrix}
\right) \\
U(R_2^{-1})=&\exp(-i\pi J_2) \quad (回転\exp(\frac{i}{2}\omega_{\mu\nu}J^{\mu\nu})=\exp(i\theta_i J_i)の角度パラメータが \theta_i=(0,-\pi,0)
\end{align*}
(ここはおそらく誤植で,$U(R_2)$ではなく$U(R_2^{-1})$が$\exp(-i\pi J_2)$である.でなければ(2.6.21)が出てこない.)$J_3=J^{12}$だから,(2.4.8)より
\begin{align*}
U(R^{-1}_2)J_3 U(R_2)=\tensor{(R_2^{-1})}{_\mu^1}\tensor{(R_2^{-1})}{_\nu^2} J^{\mu\nu}=\tensor{(R_2^{-1})}{_1^1}\tensor{(R_2^{-1})}{_2^2} J^{12}=-J^{12}=-J_3
\end{align*}
で,$J_3$の符号を反転させる.また
\begin{align*}
U(R^{-1}_2)P^\mu U(R_2)=\tensor{(R^{-1}_2)}{_\nu^\mu}P^\nu=(-P^1,P^2,-P^3,H)
\end{align*}
よって,$\mathsf{P}\Psi_{k,\sigma}$の代わりに$U(R^{-1}_2)\mathsf{P} \Psi_{k,\sigma}$に対する$P^\mu,J_3$の固有値を調べると
\begin{align*}
P^\mu U(R^{-1}_2)\mathsf{P} \Psi_{k,\sigma}=&U(R^{-1}_2)(-P^1,P^2,-P^3,H)\mathsf{P}\Psi_{k,\sigma} \\
=&(0,0,\kappa,\kappa) U(R^{-1}_2)\mathsf{P} \Psi_{k,\sigma}=k^\mu U(R^{-1}_2)\mathsf{P} \Psi_{k,\sigma} \\
J_3U(R^{-1}_2)\mathsf{P} \Psi_{k,\sigma}=&-U(R^{-1}_2)J_3 \mathsf{P}\Psi_{k,\sigma}=-\sigma U(R^{-1}_2 )\mathsf{P}\Psi_{k,\sigma}
\end{align*}
となり,固有値が$k^\mu,-\sigma$であることが確認できる.したがって再び縮退がないと仮定して
\begin{align*}
U(R^{-1}_2)\mathsf{P} \Psi_{k,\sigma}=\eta_{\sigma}\Psi_{k,-\sigma}
\end{align*}
と書くことができる.ここで$\eta_\sigma$は位相因子だ.これならブースト演算子を使うことができる!\par
さて$R^{-1}_2\mathcal{P}$は
\begin{align*}
R^{-1}_2 \mathcal{P} =\left(
\begin{matrix}
-1 & 0 & 0 & 0 \\
0 & 1 & 0 & 0 \\
0 & 0 & -1 & 0 \\
0 & 0 & 0 & 1 \\
\end{matrix}
\right) \left(
\begin{matrix}
-1 & 0 & 0 & 0 \\
0 & -1 & 0 & 0 \\
0 & 0 & -1 & 0 \\
0 & 0 & 0 & 1 \\
\end{matrix}
\right)=\left(
\begin{matrix}
1 & 0 & 0 & 0 \\
0 & -1 & 0 & 0 \\
0 & 0 & 1 & 0 \\
0 & 0 & 0 & 1 \\
\end{matrix}
\right)
\end{align*}
だから,ローレンツブースト$B(|\mathbf{p}|/\kappa)$(2.5.45)と可換だ.そして$\mathcal{P}$は3軸を$\mathbf{p}$に回す回転$R(\hat{p})$(p96あるいは(2.5.47)参照)と可換.(実際,$R(\hat{p})\mathcal{P}$は空間成分を$(0,0,|\mathbf{p}|)\to (0,0,-|\mathbf{p}|)\to (-p^1,-p^2,-p^3)$とするが,$\mathcal{P}R(\hat{p})$も同様に$(0,0,|\mathbf{p}|)\to (p^1,p^2,p^3)\to(-p^1,-p^2,-p^3)$とするので,両者は等しい.よって可換である.)したがって,(2.5.5)に$\mathsf{P}$を施すと,一般的な4元運動量$p^\mu$に対して,(2.5.44)より
\begin{align*}
\mathsf{P}\Psi_{p,\sigma}=&\sqrt{\frac{\kappa}{p^0}}U(\mathcal{P})U(L(p))\Psi_{k,\sigma} \\
=&\sqrt{\frac{\kappa}{p^0}}U(\mathcal{P})U\left(R(\hat{p})B\left(\frac{|\mathbf{p}|}{\kappa}\right)\right)\Psi_{k,\sigma} \quad \because(2.5.44) \\
=&\sqrt{\frac{\kappa}{p^0}}U(\mathcal{P})U\left(R(\hat{p})B\left(\frac{|\mathbf{p}|}{\kappa}\right)\right)\left( U(R^{-1}_2)U(\mathcal{P}) \right)^{-1}\left( U(R^{-1}_2)U(\mathcal{P}) \right)\Psi_{k,\sigma} \\
=&\sqrt{\frac{\kappa}{p^0}}U\left(\mathcal{P}R(\hat{p})B\left(\frac{|\mathbf{p}|}{\kappa}\right)(R_2^{-1}\mathcal{P})^{-1}\right) U(R^{-1}_2)\mathsf{P}\Psi_{k,\sigma} \\
=&\sqrt{\frac{\kappa}{p^0}}U\left(R(\hat{p})\mathcal{P}(R_2^{-1}\mathcal{P})^{-1}B\left(\frac{|\mathbf{p}|}{\kappa}\right)\right) U(R^{-1}_2)\mathsf{P}\Psi_{k,\sigma} \quad \because R(\hat{p})と\mathcal{P},およびBとR_2^{-1}\mathcal{P}の可換性 \\
=&\sqrt{\frac{\kappa}{p^0}}U\left(R(\hat{p})R_2B\left(\frac{|\mathbf{p}|}{\kappa}\right)\right) U(R^{-1}_2)\mathsf{P}\Psi_{k,\sigma} \\
=&\sqrt{\frac{\kappa}{p^0}}\eta_\sigma U\left(R(\hat{p})R_2B\left(\frac{|\mathbf{p}|}{\kappa}\right)\right)\Psi_{k,-\sigma} \quad \because (2.6.20)
\end{align*}
$R(\hat{p})R_2$は(3軸をマイナスにしてから$\mathbf{p}$の方向に回すので)3軸を$-\mathbf{p}$へ回す回転だ.しかし$U(R(\hat{p})R_2)$は$U(R(-\hat{p}))$と完全に同じではない.どのような違いがあるか?\\
$\Rightarrow$学部の物理でやったように,位置ベクトル$\mathbf{r}$を(2.5.46)のように極座標変数で書いたとき,$-\mathbf{r}$に対応して極座標変数は$(\theta,\phi)\to (\pi-\theta,\phi\pm\pi)$となることを思い出そう.実際
\begin{align*}
\Bigl(\sin(\pi-\theta)\cos(\phi\pm \pi),\sin(\pi-\theta)\sin(\phi\pm \pi),\cos(\pi-\theta)\Bigr)=\left(-\sin\theta \cos\phi ,-\sin\theta\sin\phi,-\cos\theta \right)=-\mathbf{p}
\end{align*}
となる.(学部でやるような普通の反転では$\phi+\pi$で良いが,ここで$\phi+\pi$が$0$から$2\pi$の範囲内から出てしまうことを防ぐために,$0\leq \phi < \pi$の場合$\phi+\pi $を,$\pi\leq \phi <2\pi$の場合$\phi-\pi$とする.)したがって(2.5.47)より
\begin{align*}
U(R(-\hat{p}))=\exp(-i(\phi\pm \pi)J_3)\exp(-i(\pi-\theta)J_2)
\end{align*}
となる.これと(2.5.47)(2.6.19)を用いると((2.6.19)の誤植を訂正し,$U(R_2)=\exp(i\pi J_2)$とする)
\begin{align*}
&U(R(-\hat{p}))^{-1}U(R(\hat{p})R_2)=U(R(-\hat{p}))^{-1}U(R(\hat{p}))U(R_2) \\
=&\Bigl\{\exp(-i(\phi\pm \pi)J_3)\exp(-i(\pi-\theta)J_2) \Bigr\}^{-1}\exp(-i\theta J_3)\exp(-i\theta J_2)\exp(+i\pi J_2) \\
=&\exp(i(\pi-\theta)J_2)\exp(i(\phi\pm \pi)J_3)\exp(-i\phi J_3) \exp(i(\pi-\theta)J_2) \\
=&\exp(i(\pi-\theta)J_2) \exp(\pm i\pi J_3) \exp(-i(\pi-\theta)J_2)
\end{align*}
となる.さらに,(2.6.19)の主張と同様に,3軸周りの$\pm 180^\circ$は$J_2$の符号を変える.実際$R_3=\mathrm{diag}(-1,-1,1,1)$を3軸周りの$\pm180^\circ$回転として,再び(2.4.8)より
\begin{align*}
U(R_3)=&\exp(\pm i\pi J_3) \\
U(R_3^{-1})J_2 U(R_3)=&U(R_3^{-1})J^{31} U(R_3)=\tensor{(R_3)}{_\mu^3}\tensor{(R_3)}{_\nu^1}J^{\mu\nu} =-J^{31}=-J^2
\end{align*}
となる.よって
\begin{align*}
\exp(i(\pi-\theta)J_2)\exp(\pm i\pi J_3)=\exp(\pm i\pi J_3)\exp(-i(\pi-\theta)J_2)
\end{align*}
を得るので,結局
\begin{align*}
U(R(-\hat{p}))^{-1}U(R(\hat{p})R_2)=&\exp(\pm i\pi J_3) \\
\therefore \quad U(R(\hat{p})R_2)=&U(R(-\hat{p}))\exp(\pm i\pi J_3)
\end{align*}
となる.以上から
\begin{align*}
\mathsf{P}\Psi_{p,\sigma}=&\sqrt{\frac{\kappa}{p^0}}\eta_\sigma U\left(R(\hat{p})R_2B\left(\frac{|\mathbf{p}|}{\kappa}\right)\right)\Psi_{k,-\sigma} \\
=&\sqrt{\frac{\kappa}{p^0}}\eta_\sigma U(R(-\hat{p}))\exp(\pm i\pi J_3)U\left(B\left(\frac{|\mathbf{p}|}{\kappa}\right)\right)\Psi_{k,-\sigma}
\end{align*}
ここで$B$はブースト$(0,0,\kappa,\kappa)\to(0,0,\mathbf{p},|\mathbf{p}|)$であるから,3軸周りの$\pm 180^\circ$回転$U(R_3)=\exp(\pm i\pi J_3)$とは可換である.よってさらに変形することができて
\begin{align*}
=&\sqrt{\frac{\kappa}{p^0}}\eta_\sigma U(R(-\hat{p}))U\left(B\left(\frac{|\mathbf{p}|}{\kappa}\right)\right)\exp(\pm i\pi J_3)\Psi_{k,-\sigma} \\
=&\sqrt{\frac{\kappa}{p^0}}\eta_\sigma \exp(\mp i\pi \sigma)U\left(R(-\hat{p})B\left(\frac{|\mathbf{p}|}{\kappa}\right)\right)\Psi_{k,-\sigma}
\end{align*}
$R(-\hat{p})B$は(2.5.44)の通り,$(0,0,\kappa,\kappa)\to (0,0,|\mathbf{p}|,p^0)\to (-p^1,-p^2,-p^3,p^0)=(-\mathbf{p},p^0)=\mathcal{P}p$の方向への基準ブースト$L(\mathcal{P}p)$だ.これにより結局
\begin{align*}
=&\sqrt{\frac{\kappa}{p^0}}\eta_\sigma \exp(\mp i\pi \sigma)U(L(\mathcal{P}p))\Psi_{k,-\sigma} =\eta_\sigma \exp(\mp i\pi \sigma)\Psi_{\mathcal{P}p,-\sigma} \\
\therefore \quad \mathsf{P}\Psi_{p,\sigma}=&\eta_\sigma \exp(\mp i\pi \sigma)\Psi_{\mathcal{P}p,-\sigma}
\end{align*}
を得る.これが結果である.ここで位相$\exp(\mp i\pi\sigma)$は,元の方位角が$0\leq \phi <\pi$か$\pi \leq \phi <2\pi$か,すなわち(2.5.46)より$\mathbf{p}$の第二成分が正か負によってそれぞれ$-i\pi\sigma$か$+i\pi \sigma$になる.例えばスピン$\sigma=1/2$の粒子の場合,$\exp(\mp i\pi\sigma)=\mp i$となる.\par
半整数スピンの質量ゼロ粒子にパリティが作用するときの位相の奇妙な符号変化は,(2.5.47)で任意の運動量の質量ゼロ粒子状態を定義するのに用いた回転のために生じる.回転群$SO(3)$は単連結ではない(普遍被覆群がスピン群$\mathrm{Spin}(3)=SU(2)$)ので,この種の非連続性は回避するのは難しい.(ニュートリノの生じる弱い相互作用はパリティが破れているので,前提を満たさないのでこの議論には参加できない.だから今のところこの困難が生じる物理的な模型は素粒子物理学では特になさそうに思われる.)\par
この結果,$\eta_\sigma$はスピン$\sigma$に依存する場合があるので,単純に議論することは難しい.しかし次に議論する$\mathsf{T}$の場合は幾分簡単な作用を見せてくれる.


\vskip\baselineskip

$\mathsf{T}:M=0$\par
状態$\Psi_{k,\sigma}$では,$P^\mu$と$J_3$が値$k^\mu=(0,0,\kappa,\kappa)$と$\sigma$を持つ.この状態に時間反転の演算子$\mathsf{T}$が作用すると
\begin{align*}
P^\mu \mathsf{T}\Psi_{k,\sigma}=\left\{
\begin{array}{ll}
-\mathsf{T}\mathbf{P} \Psi_{k,\sigma}=-(0,0,\kappa)\mathsf{T}\Psi_{k,\sigma} \quad &(\mu=1,2,3)\\
\mathsf{P} H \Psi_{k,\sigma}=\kappa \mathsf{P}\Psi_{k,\sigma} \quad &(\mu=0)
\end{array}
 \right.
\end{align*}
となって,4元運動量が$(\mathcal{P}k)^\mu=(0,0,-\kappa,\kappa)$で,$J_3$の固有値は(2.6.7)より
\begin{align*}
J_3 \mathsf{T}\Psi_{k,\sigma}=+\mathsf{T}J_3 \Psi_{k,\sigma}=+\sigma \mathsf{T} \Psi_{k,\sigma}
\end{align*}
となって,$P^\mu$と$J_3$が$(\mc{P}k)^\mu=(0,0,-\kappa,\kappa)$と$-\sigma$の値をもつ状態が生じる.したがって,ヘリシティは$+\sigma$となり,$\mathsf{T}$はヘリシティ$\mathbf{J}\cdot \mathbf{k}$を変えず,それだけではあるヘリシティ$\sigma$の質量ゼロ粒子とヘリシティ$-\sigma$の粒子を何も関連付けない.ここで,$\mathsf{T}$は$\mathsf{P}$と同様に基準運動量$k$を不変に保たないから,前と同じように$U(R^{-1}_2)\mathsf{T}$を考えるのが便利になる.$U(R_2^{-1})$は前に示した通り$J_3$と反可換$U(R_2^{-1})J_3 =-J_3 U(R^{-1}_2)$であるから,$U(R^{-1}_2)\mathsf{T}$は$J_3$と可換
\begin{align*}
J_3 U(R^{-1}_2)\mathsf{T}\Psi_{k,\sigma}=-U(R^{-1}_2)J_3 \mathsf{T} =U(R^{-1}_2) \mathsf{T} J_3
\end{align*}
となる.したがって$U(R^{-1}_2)\mathsf{T} \Psi_{k,\sigma}$も$J_3$の固有値$\sigma$の固有状態となる.
\begin{align*}
J_3U(R^{-1}_2)\mathsf{T} \Psi_{k,\sigma}= \sigma U(R^{-1}_2)\mathsf{T} \Psi_{k,\sigma}
\end{align*}
再び縮退がないとすれば,
\begin{align*}
U(R^{-1}_2)\mathsf{T} \Psi_{k,\sigma}=\zeta_\sigma \Psi_{k,\sigma}
\end{align*}
となる.ここで,$\zeta_\sigma$は位相だ.$\mc{P}$と同様,$R^{-1}_2\mc{T}$はブースト(2.5.45)と可換となる.なぜなら
\begin{align*}
R^{-1}_2 \mc{T}=\left(
\begin{matrix}
-1 &  & &  \\
 & 1& & \\
  & & -1 & \\
   & & & -1
\end{matrix}
\right)\left(
\begin{matrix}
1 &  & &  \\
 &1& & \\
  & & 1 & \\
   & & & -1
\end{matrix}
\right)=\left(
\begin{matrix}
-1 &  & &  \\
 & 1& & \\
  & & -1 & \\
   & & & -1
\end{matrix}
\right)=-R^{-1}_2 \mc{P}
\end{align*}
で,$R^{-1}_2 \mc{P}$は既にローレンツブースト$B$と可換であることを示しているので,当然$R^{-1}_2 \mc{T}$も可換となる.さらに$\mc{T}$は回転$R(\hat{p})$と可換だ.(空間回転なので自明だが,あらわに書いておくと
\begin{align*}
\mc{T}R(\hat{p}):(0,0,|\mathbf{p}|,p^0) \to (p^1,p^2,p^3,p^0) \to (p^1,p^2,p^3,-p^0) \\
R(\hat{p})\mc{T}:(0,0,|\mathbf{p}|,p^0) \to (0,0,|\mathbf{p}|,-p^0)  \to (p^1,p^2,p^3,-p^0)
\end{align*}
となることから分かる.)よって$\mc{P}$のときと同様に,$\mathsf{T}$を状態(2.5.5)に施すと
\begin{align*}
\mathsf{T} \Psi_{p,\sigma}=\sqrt{\frac{\kappa}{p^0}}\zeta_\sigma U\left(R(\hat{p})R_2B\left(\frac{|\mathbf{p}|}{\kappa}\right)\right)\Psi_{k,-\sigma}
\end{align*}
が得られる.さらに,(2.6.21)より同様に
\begin{align*}
\mathsf{T} \Psi_{p,\sigma}=&\sqrt{\frac{\kappa}{p^0}}\zeta_\sigma U\left(R(-\hat{p})\right)\exp(\pm i\pi J_3)U\left(B\left(\frac{|\mathbf{p}|}{\kappa}\right)\right)\Psi_{k,\sigma} \\
=&\sqrt{\frac{\kappa}{p^0}}\zeta_\sigma U\left(R(-\hat{p})\right)U\left(B\left(\frac{|\mathbf{p}|}{\kappa}\right)\right)\exp(\pm i\pi\sigma)\Psi_{k,\sigma} \\
=&\zeta_\sigma \exp(\pm i\pi \sigma)\Psi_{\mc{P}p,\sigma}
\end{align*}
となる.以前と同様,上と下の符号は,運動量$\mathbf{p}$の2軸成分が正か負かに対応する.

\vskip\baselineskip


時間反転の演算子の2乗$\mathsf{T}^2$は,質量ゼロでない粒子状態と質量ゼロの粒子状態に非常に簡単に作用する.(2.6.18)と,$\mathsf{T}$が反ユニタリーであることを使うと,質量ゼロでない1粒子状態では
\begin{align*}
\mathsf{T}^2 \Psi_{p,\sigma}=&\mathsf{T} \zeta(-1)^{j-\sigma} \Psi_{\mc{P}p,-\sigma} \\
=&\zeta^* (-1)^{j-\sigma} \mathsf{T} \Psi_{\mc{P}p,-\sigma} \quad \because \mathsf{T}の反ユニタリー性\\
=&\zeta^* (-1)^{j-\sigma} \zeta(-1)^{j-(-\sigma)} \Psi_{p,\sigma} \quad \because \mc{P}(\mc{P}p)=p \\
=&(-1)^{2j} \Psi_{p,\sigma} \quad \because |\zeta|=1
\end{align*}
となる.質量ゼロの粒子状態については,もし$\mathbf{p}$の2軸成分が正ならば,空間反転された運動量$\mc{P}\mathbf{p}=-\mathbf{p}$の2軸成分は負になり,逆に$\mathbf{p}$2軸成分が負ならば逆に$\mc{P}\mathbf{p}=-\mathbf{p}$の2軸成分は正になる.したがって$\mathsf{T}\Psi_{\mc{P}p,\sigma}$に現れる(2.6.25)での$\exp(\pm i\pi \sigma)$は,一つ目の$\mathsf{T}$で現れる符号と反対になる.よって
\begin{align*}
\mathsf{T}^2 \Psi_{p,\sigma}=&\mathsf{T} \zeta_{\sigma} \exp(\pm i\pi\sigma)\Psi_{\mc{P}p,\sigma} \\
=&\zeta_{\sigma}^* \exp(\mp i\pi \sigma)\mathsf{T} \Psi_{\mc{P}p,\sigma} \quad \because \mathsf{T}の反ユニタリー性\\
=&\zeta_{\sigma}^*\exp(\mp i\pi\sigma) \zeta_{\sigma} \exp(\mp i\pi\sigma) \Psi_{p,\sigma} \\
=&\exp(\mp 2i\pi\sigma)\Psi_{p,\sigma}
\end{align*}
となる.$\sigma$が整数か半整数である限り,これは
\begin{align*}
\mathsf{T}^2 \Psi_{p,\sigma}=(-1)^{2|\sigma|}\Psi_{p,\sigma}
\end{align*}
と書き換えられる.質量ゼロの粒子における「スピン」とは,通常ヘリシティの絶対値のことを意味するので$|\sigma|=j$と書けて,これは質量ゼロでない粒子と同じ表式となる.\par
$\mathsf{T}^2$が\uwave{相互作用しない}質量ゼロか質量ゼロでない粒子の系の,任意の状態$\Psi$に作用するとき,1粒子につき,それぞれ因子$(-1)^{2j}$か$(-1)^{2|\sigma|}$が出る.
\begin{align*}
\mathsf{T}^2 \Psi_{p_1,\sigma_1;p_2,\sigma_2;,\cdots ,;p_n,\sigma_n}=&(-1)^{2j_1}(-1)^{2|\sigma_2|}\cdots (-1)^{2|\sigma_n|}\Psi_{p_1,\sigma_1;p_2,\sigma_2;,\cdots ,;p_n,\sigma_n} \\
=&(-1)^{2(j_1+|\sigma_2|+\cdots +|\sigma_n|)}\Psi_{p_1,\sigma_1;p_2,\sigma_2;,\cdots ,;p_n,\sigma_n}
\end{align*}
したがって,もし状態に半整数のスピンかヘリシティを持つ粒子が奇数個(1個でもよい)あれば,(整数のスピンかヘリシティを持つ粒子は何個あってもよい)全体の符号の変化は
\begin{align*}
\mathsf{T}^2 \Psi=-\Psi
\end{align*}
となる.もし,いろいろな相互作用の「スイッチを入れ」ても(つまりこの状態を相互作用のない系ではなく,3章で議論するin,out状態のような相互作用のある系だとしても),この結果は,それらの相互作用がたとえ回転不変でなくても時間反転のもとで不変である限り,変わらない.なぜなら自由状態の系を相互作用のある系にするための相互作用ハミルトニアンが時間反転$\mathsf{T}$と可換なら
\begin{align*}
\mathsf{T} \Psi_{\alpha}^\pm=&\lim_{\tau \to \mp\infty} \mathsf{T}\exp(+iH\tau )\exp(-iH_0 \tau)\Phi_\alpha \\
=&\lim_{\tau \to \mp\infty} \exp(-iH\tau )\exp(+iH_0 \tau)\mathsf{T} \Phi_\alpha \\
=&-\lim_{\tau \to \mp\infty} \exp(-iH\tau )\exp(+iH_0 \tau) \Phi_\alpha \\
=&-\Psi_{\alpha}^\mp
\end{align*}
自由状態に作用するのと変わらないからだ.たとえば,この議論は系が任意の静的な重力場や電場のもとにあっても,重力相互作用は重力外場にも相互作用にも時間反転不変性があるからだ.電磁相互作用自体は時間反転不変だが,外場となる磁場が時間反転で$\mathbf{B} \to -\mathbf{B}$と変化するから,時間反転で不変$\mathbf{E}\to \mathbf{E}$な電場に限っている.)さて,$\Psi$がハミルトニアンの固有状態とする.$\mathsf{T}$はハミルトニアンと可換だから,$\mathsf{T}\Psi$もまたハミルトニアンの固有状態だ.これは同じ状態(同じ射線に属している)だろうか?もしそうなら,$\mathsf{T}\Psi$は$\Psi$と位相だけしか違わないことになる.
\begin{align*}
\mathsf{T}\Psi=\zeta \Psi
\end{align*}
しかしそのとき
\begin{align*}
\mathsf{T}^2 \Psi=\mathsf{T} (\zeta \Psi) = \zeta^*\mathsf{T} \Psi =|\zeta|^2 \Psi=\Psi
\end{align*}
となって,(2.6.28)と矛盾する.こうして,(2.6.28)を満たすどのようなエネルギー固有状態も,同じエネルギーの他の射線に属する固有状態と縮退していることがわかる.これは「クラマーの縮退」と呼ばれる.もちろん,この結論はもし系が回転不変な環境にあれば自明だ.なぜなら,この場合の状態は回転変換に対し不変空間となっており,そのような状態は全角運動量$j$によってラベル付けされる.(2.6.28)により全角運動量$j$は(半整数スピンかヘリシティの粒子が奇数個だから)半整数であり,$2j+1=2,4,\cdots$個の縮退した状態が存在するからだ.驚くべき結果は,回転不変性が静電場などの外場の摂動(例えば$H=e\mathbf{E}_0 \cdot \mathbf{r}$のシュタルク効果)によって破られていても,その外場が最低限$\mathsf{T}$のもとで不変な限り,$\Psi$と$\mathsf{T}\Psi$の2重縮退が残るということだ.特に,もしその粒子が電気(または重力)双極子モーメントを持っている(つまり先程のハミルトニアンを摂動で持つ)と仮定すると,摂動によって系の回転不変性が失われ,結果その$2j+1$個のスピン状態の間の縮退は外部電場か重力場のもとで完全に解ける.しかし先程言ったように二重縮退は残っていなければならない!したがって仮定が偽となり,半整数のスピンかヘリシティを持つ粒子が奇数個集まって作られる電気(重力)双極子モーメントは,時間反転不変性によって禁止される!

\vskip\baselineskip


議論を完全にするために,$\mathsf{P}$と$\mathsf{T}$は同じ質量の粒子の多重項により複雑に作用しうるとする場合も考える.(補遺Cで考察する.)この場合で物理的に意味のある場合は知られていない.


\newpage

\subsection{射影表現}
ここで,2.2節で述べた,対称性の群が物理的状態の上で射影表現になっている,つまり対称性群の元$T,\bar{T}$などが物理的ヒルベルト空間上で,ユニタリー演算子$U(T),U(\bar{T})$等で表されていて,次の結合則を満たしている場合を考える.
\begin{align*}
U(T)U(\bar{T})=\exp\Bigl(i\phi(T,\bar{T})\Bigr)U(T\bar{T})
\end{align*}
ここで$\phi$は実の位相だ.(2.7.1)の位相$\phi$が満たさなければならない基本的な条件は,結合則
\begin{align*}
U(T_3)\Bigl(U(T_2)U(T_1)\Bigr)=\Bigl(U(T_3)U(T_2)\Bigr)U(T_1)
\end{align*}
からくるもので,
\begin{align*}
(\mathrm{LHS})=&U(T_3)\exp(i\phi(T_2,T_1))U(T_2 T_1) \\
=&\exp(i\phi(T_2,T_1)+i\phi(T_3,T_2T_1))U(T_3 T_2 T_1 ) \\
(\mathrm{RHS})=&\exp(i\phi(T_3,T_2))U(T_3T_2)U(T_1) \\
=&\exp(i\phi(T_3,T_2)+i\phi(T_3T_2,T_1))U(T_3 T_2 T_1) \\
\therefore \quad &\phi(T_2, T_1)+\phi(T_3,T_2T_1)=\phi(T_3,T_2)+\phi(T_3T_2,T_1)
\end{align*}
となる.もちろん,次の形をしたどんな位相もこの関係式を満たす.
\begin{align*}
\phi(T,\bar{T})=\alpha(T\bar{T})-\alpha(T)-\alpha(\bar{T})
\end{align*}
実際
\begin{align*}
\phi(T_2,T_1)+\phi(T_3,T_2T_1)=&\left[\alpha(T_2T_1)-\alpha(T_2)-\alpha(T_1)\right]+\left[\alpha(T_3T_2 T_1)-\alpha(T_3)-\alpha(T_2T_1)\right] \\
=&-\alpha(T_3)-\alpha(T_2) \\
&+\alpha(T_3T_2T_1)-\alpha(T_1) \\
=&\alpha(T_3T_2)-\alpha(T_3)-\alpha(T_2) \\
&+\alpha(T_3T_2T_1)-\alpha(T_3T_2)-\alpha(T_1) \\
=&\phi(T_3,T_2)+\phi(T_3T_2,T_1)
\end{align*}
となって関係式を満たしていることが分かる.そのような位相をもつ射影表現は,以下のように$U(T)$の置き換えをすれば通常の表現となる.
\begin{align*}
\tilde{U}(T):= U(T)\exp(i\alpha(T))
\end{align*}
実際
\begin{align*}
\tilde{U}(T)\tilde{U}(\bar{T})=&\exp(i\alpha(T)+i\alpha(\bar{T}))U(T)U(\bar{T}) \\
=&\exp(i[\alpha(T)+\alpha(\bar{T})])\exp(i\phi(T,\bar{T}))U(T\bar{T}) \\
=&\exp(i\alpha(T\bar{T}))U(T\bar{T}) \\
=&\tilde{U}(T\bar{T})
\end{align*}
(2.7.2)を満たし,(2.7.3)の形をした$\Delta\phi(T,\bar{T})$だけしか互いに異ならない関数の集合
\begin{align*}
[\phi(T,\bar{T})]=\{\phi'(T,\bar{T})|\phi'(T,\bar{T})=\phi(T,\bar{T})+\Delta\phi(T,\bar{T})\}
\end{align*}
は「2コサイクル」と呼ばれる.\par
ここでコサイクルについて補足しておく.群$G$の$n$個直積$G^n$から実数$\mathbb{R}$への関数の集合$C^n:G^n\to \mathbb{R}$を$n$コチェインと呼び,その元$f\in C^n$に対して$n$コバウンダリ作用素$\delta_n:C^n\to C^{n+1}$を
\begin{align*}
(\delta_n f)(g_1,g_2,\cdots ,g_{n+1})=f(g_2,\cdots ,g_{n+1})+\sum^n_{i=1}(-1)^i f(g_1 ,\cdots ,g_i g_{i+1},\cdots ,g_{n+1})+(-1)^{n+1}f(g_1,\cdots ,g_n)
\end{align*}
で定義する.これは$\delta_{n+1}\circ\delta_n=0$を満たす意味で境界作用素の双対である.一般の場合で示すのは面倒だから,わかりやすく本質的な$n=1$の場合で示すと
\begin{align*}
(\delta_2(\delta_1 f))(g_1,g_2,g_3)=&(\delta_1 f)(g_2,g_3)-(\delta_1f)(g_1g_2,g_3)+(\delta_1 f)(g_1,g_2g_3)-(\delta_1 f)(g_1,g_2) \\
=&f(g_3)-f(g_2 g_3)+f(g_2) \\
&-f(g_3)+f(g_1g_2g_3)-f(g_1g_2) \\
&+f(g_2 g_3)-f(g_1g_2 g_3)+f(g_1) \\
&-f(g_2)+f(g_1g_2)-f(g_1) \\
=&0
\end{align*}
となる.コバウンダリ作用素の像
\begin{align*}
\mathrm{Im}\delta_{n-1}=\{\delta_{n-1}f|f\in C^{n-1}\} \in C^{n}
\end{align*}
を$n$コサイクル$B^n$と呼ぶ.具体的に2コサイクル$B^2$の元は
\begin{align*}
(\delta_1 h)(g_1,g_2)=h(g_2)-h(g_1g_2)+h(g_1)
\end{align*}
と書ける.一方
\begin{align*}
\mathrm{Ker}\delta_n =\{f|\delta_n f=0\} \in C^n
\end{align*}
を$n$コバウンダリ$Z^n$と呼ぶ.2コバウンダリ$Z^2$の元は
\begin{align*}
0=(\delta_2 f)(g_1,g_2,g_3)=f(g_2,g_3)-f(g_1g_2,g_3)+f(g_1,g_2g_3)-f(g_1,g_2)
\end{align*}
を満たす関数$f$である.$n$コホモロジー$H^n$は,それらの剰余類をとり
\begin{align*}
H^n=Z^n/B^n
\end{align*}
としたもので定義される.すなわち,$\Delta f\in B^n$だけ異なる二つの$Z^n$の元$f'=f+\Delta f$を同一視することで得られる同値類の集合である.\par
(2.7.2)を書き換えると$(\delta_2 \phi)(T_3,T_2,T_1)=0$となり,(2.7.3)は$\phi(T,\bar{T})=(\delta_1 \alpha )(T,\bar{T})$と書くことができる.したがって射影表現の位相$\phi$は2コサイクル$Z^2$の元であり,2コバウンダリ$B^2$だけの違いは$U(T)$の再定義によって同じ射影表現とみなすことができる.自明なコサイクルは関数$\phi=0$を含み,したがって(2.7.3)の形の関数のみでできている.これらは$U(T)$の再定義で吸収できるのだった.ここでは,対称群が自明でない2コサイクルが持てるかどうかに興味ある.つまり,それが,物理的なヒルベルト空間上で,$\phi(T,\bar{T})$が上のように消すことが\uwave{できない}という意味で,\uwave{固有}に射影的であるのかどうか,という点だ.

\vskip\baselineskip


この問題に答えるために,まず最初に(2.7.1)の位相$\phi$が微小変換の生成子の交換関係に与える影響を調べる.$\bar{T}$か$T$が恒等変換ならば,位相$\phi$は明らかに消える.
\begin{align*}
U(T)U(1)=&\exp(i\phi(T,1))U(T1) \quad \Rightarrow \quad \phi(T,1)=0 \\
U(1)U(\bar{T})=&\exp(i\phi(1,\bar{T}))U(1\bar{T}) \quad \Rightarrow \quad \phi(1,\bar{T})=0 
\end{align*}
よって,もし$T$と$\bar{T}$が共に恒等変換の近傍ならば,位相は小さいはずだ.座標$\theta^a$を使って(2.2節と同様に)群の元を$T(0)=1$となるようにパラメータ化すると,(2.7.4)から$\phi(T(\theta),T(\bar{\theta}))$の$\theta=\bar{\theta}=0$の周りの展開が$\theta \bar{\theta}$の次数から始まることがわかる.($\theta,\bar{\theta},\theta^2,\bar{\theta}^2$から始まると,片方だけ$\theta=0$か$\bar{\theta}=0$のときに(2.7.4)を満たさない.)
\begin{align*}
\phi\Bigl(T(\theta),T(\bar{\theta})\Bigr)=f_{ab}\theta^a \bar{\theta}^b+\cdots
\end{align*}
ここで$\phi$が実位相だから,$f_{ab}$は実の数係数だ.この展開を(2.7.1)のベキ級数展開に代入して再び$\theta\bar{\theta}$の項を比べると
\begin{align*}
(\mathrm{LHS})=&\left[1+i\theta^a t_a +\frac{1}{2}\theta^b \theta^c t_{bc} +\cdots \right] \left[1+i\bar{\theta}^a t_a +\frac{1}{2}\bar{\theta}^b \bar{\theta}^c t_{bc} +\cdots \right] \\
=&\cdots -\theta^b \bar{\theta}^ct_b t_c +\cdots \\
=(\mathrm{RHS})=&(1+if_{ab}\theta^a \bar{\theta}^b+\cdots ) \\
&\qquad \times \left[1+i(\theta^a +\bar{\theta}^a+\tensor{f}{^a_b_c}\theta^b \bar{\theta}^c+\cdots )t_a +\frac{1}{2}(\theta^b +\bar{\theta}^b+\cdots )(\theta^c +\bar{\theta}^c+\cdots )t_{bc}+\cdots \right] \\
=&\cdots + if_{bc}\theta^b \bar{\theta}^c+i\tensor{f}{^a_b_c}\theta^b \bar{\theta}^c t_a +t_{bc}\theta^b \bar{\theta}^c+\cdots \\
\therefore \quad & t_{bc}= -t_b t_c -i\tensor{f}{^a_b_c}t_a-if_{bc}
\end{align*}
$t_{cb}=t_{bc}$より
\begin{align*}
-t_c t_b -i\tensor{f}{^a_b_c}t_a -if_{cb}=&-t_b t_c -i\tensor{f}{^a_b_c}t_a-if_{bc} \\
\therefore \quad \left[t_b,t_c \right]=&i(-\tensor{f}{^a_b_c}+\tensor{f}{^a_{cb}})t_a+i(-f_{bc}+f_{cb}) \\
=&i\tensor{C}{^a_b_c}t_a +iC_{bc}1
\end{align*}
となる.ここで$C_{bc}$は反対称係数
\begin{align*}
C_{bc}:= -f_{bc}+f_{cb}
\end{align*}
である.交換関係の右辺に単位元$1$に比例する項(通常,中心電荷と呼ばれる)が現れることは,群の射影表現に位相$\phi$があることのリー代数への反映だ.\par
係数$C_{bc}$と$\tensor{C}{^a_{bc}}$はヤコビ恒等式から導かれるある重要な拘束条件に従う.(2.7.6)と$t_d$との交換子をとると
\begin{align*}
[[t_b,t_c],t_d]=&i\tensor{C}{^a_{bc}}[t_a,t_d]+iC_{bc}[1,t_d] \\
=&i\tensor{C}{^a_{bc}}(i\tensor{C}{^e_{ad}}t_e +iC_{ad}1) \\
=&-\tensor{C}{^a_{bc}}\tensor{C}{^e_{ad}}t_e-\tensor{C}{^a_{bc}}C_{ad}1
\end{align*}
であるから,ヤコビ恒等式から
\begin{align*}
0=&[[t_b,t_c],t_d]+[[t_c,t_d],t_b]+[[t_d,t_b],t_c] \\
=&(-\tensor{C}{^a_{bc}}\tensor{C}{^e_{ad}}t_e-\tensor{C}{^a_{bc}}C_{ad}1)+(-\tensor{C}{^a_{cd}}\tensor{C}{^e_{ab}}t_e-\tensor{C}{^a_{cd}}C_{ab}1)+(-\tensor{C}{^a_{db}}\tensor{C}{^e_{ac}}t_e-\tensor{C}{^a_{db}}C_{ac}1) \\
=&-(\tensor{C}{^a_{bc}}\tensor{C}{^e_{ad}}+\tensor{C}{^a_{cd}}\tensor{C}{^e_{ab}}+\tensor{C}{^a_{db}}\tensor{C}{^e_{ac}})t_e-(\tensor{C}{^a_{bc}}C_{ad}+\tensor{C}{^a_{cd}}C_{ab}+\tensor{C}{^a_{db}}C_{ac})1
\end{align*}
となって,二つの条件
\begin{align*}
\tensor{C}{^a_{bc}}\tensor{C}{^e_{ad}}+\tensor{C}{^a_{cd}}\tensor{C}{^e_{ab}}+\tensor{C}{^a_{db}}\tensor{C}{^e_{ac}}=&0 \\
\tensor{C}{^a_{bc}}C_{ad}+\tensor{C}{^a_{cd}}C_{ab}+\tensor{C}{^a_{db}}C_{ac}=&0
\end{align*}
を得る.これは常に自明なゼロでない解
\begin{align*}
C_{ab}=\tensor{C}{^e_a_b}\phi_e
\end{align*}
を持っている.ここで$\phi_e$は任意の実定数だ.(実際に二つ目の条件式に代入し,一つ目の条件式を使えば解になっていることがすぐ確かめられる.)これが解となっている射影表現(つまり位相$\phi$に現れる$f_{ab}$が,(2.7.7)と(2.7.16)で関係付いているとき)では,(2.7.6)から中心電荷を以下の生成子の再定義で消すことができる.
\begin{align*}
t_a\to \tilde{t}_a := t_a+\phi_a
\end{align*}
こうすると,新しい生成子は中心電荷のない交換関係を満たすことができる.
\begin{align*}
[\tilde{t}_b ,\tilde{t}_c]=&[t_b,t_c]+[t_b,\phi_c]+[\phi_b,t_c]+[\phi_b,\phi_c] \\
=&[t_b,t_c]=i\tensor{C}{^a_{bc}}t_a +iC_{bc}1 \\
=&i\tensor{C}{^a_{bc}}(t_a+\phi_a) \\
=&i\tensor{C}{^a_{bc}}\tilde{t}_a
\end{align*}
もちろん,リー代数は(2.7.9)の解として(2.7.10)以外のものを持つ場合も,そうでない場合もある.(中心電荷を消しただけであって,これで固有射影表現が消せるかどうかはまだわからない.)

\vskip\baselineskip

ここで固有の射影表現があるかどうか,を決める鍵となる定理を述べる.\par
任意の群の,どんな表現$U(T)$の位相も,もし以下の2つの条件が満たされれば(2.7.1)で$\phi=0$と選ぶことができる.\par
1.この群の生成子を(2.7.11)でしたように再定義して,リー代数から全ての中心電荷を消去する.\par
2.群が単連結である.つまり,どんな二つの群の元も,群の内部を通る経路で結ばれていて,そのような経路は連続的に変形できる.\par
この定理はこの章の補遺Bで証明する.そこではまた,単連結ではない群についても述べる.この定理によれば,固有射影表現が可能になるには,ちょうど二つの(同時でもよい)やりかたがある.一つは代数的なもので,群が単位元の近傍でも射影的に表現される場合で,もう一つは,トポロジー的で,群が単連結ではなく,「1から$\bar{T}$に行き,$\bar{T}$から$T\bar{T}$に行く経路」が,「1から$T\bar{T}$に行く他の経路」に連続的に変形できないことがある場合だ.後者の場合は,$U$演算子を定義するのに使った(2.7.1)の位相$\phi$は,様々な群の元と原点を結ぶ基準経路の選び方による.

\vskip\baselineskip


これらの可能性を非斉次ローレンツ群の特殊な場合について順に調べる.\par
\textbf{(A)代数}\par
中心電荷があるとした場合の,非斉次ローレンツ群の生成子の交換関係は(2.4.12)~(2.4.14)の代わりに以下で与えられる.
\begin{align*}
i[J^{\mu\nu},J^{\rho\sigma}]=&\eta^{\nu\rho}J^{\mu\sigma}-\eta^{\mu\rho}J^{\nu\sigma}-\eta^{\mu\sigma}J^{\rho\nu}+\eta^{\nu\sigma}J^{\rho\mu} +C^{\rho\sigma,\mu\nu}\\
i[P^\mu,J^{\rho\sigma}]=&\eta^{\mu\rho}P^\sigma -\eta^{\mu\sigma}P^\rho +C^{\rho\sigma,\mu}\\
i[J^{\mu\nu},P^\rho]=&\eta^{\nu\rho}P^\mu -\eta^{\mu\rho}P^\nu+C^{\rho,\mu\nu} \\
i[P^\mu,P^\rho]=&C^{\rho,\mu}
\end{align*}
(これは$\phi$を具体的に決めて得られるというより,単位元に比例する項を単に追加したものだ.)左辺の交換子の反対称性により
\begin{align*}
C^{\rho\sigma,\mu\nu}=&-C^{\mu\nu,\rho\sigma} \\
C^{\rho\sigma,\mu}=&-C^{\mu,\rho\sigma} \\
C^{\rho,\mu}=&-C^{\mu,\rho}
\end{align*}
が満たされている.さらに$J^{\rho\sigma}=-J^{\sigma\rho}$であることに対応して
\begin{align*}
C^{\rho\sigma,\mu\nu}=&-C^{\sigma\rho,\mu\nu}=-C^{\rho\sigma,\nu\mu} \\
C^{\rho,\mu\nu}=&-C^{\rho,\nu\mu}
\end{align*}
も満たされている.これらの定数全ては単に,ある代数的性質を持ち,そのために$J^{\mu\nu}$と$P^\mu$を定数分だけ(2.7.11)のようにずらすことによって取り除けることを示す.これを示すためには,(2.7.8)(2.7.9)と同様,ヤコビ恒等式を使う.
\begin{align*}
0=&\left[J^{\mu\nu},\left[P^\rho,P^\sigma\right]\right]+\left[P^\sigma,\left[J^{\mu\nu},P^\rho\right]\right]+\left[P^\rho,\left[P^\sigma,J^{\mu\nu}\right]\right] \\
0=&\left[J^{\lambda\eta},\left[J^{\mu\nu},P^\rho \right]\right]+\left[ P^\rho,\left[J^{\lambda\eta},J^{\mu\nu}\right] \right]+\left[J^{\mu\nu},\left[P^\rho,J^{\lambda\eta}\right]\right] \\
0=&\left[J^{\lambda\eta},\left[J^{\mu\nu},J^{\rho\sigma} \right]\right]+\left[ J^{\rho\sigma},\left[J^{\lambda\eta},J^{\mu\nu}\right] \right]+\left[J^{\mu\nu},\left[J^{\rho\sigma},J^{\lambda\eta}\right]\right]
\end{align*}
(ちなみに,(2.7.16)より$[P^\mu,[P^\rho,P^\sigma]]=0$なので,3つの$P^\mu$を含むヤコビ恒等式は自動的に満たされており,新しい情報を何ももたらさない.)一つ目の恒等式は
\begin{align*}
\left[J^{\mu\nu},[P^\rho,P^\sigma]\right]=&\left[J^{\mu\nu},-iC^{\sigma,\rho}\right]=0 \\
\left[P^\sigma,\left[J^{\mu\nu},P^\rho\right]\right]=&-i\eta^{\nu\rho}\left[P^\sigma,P^\mu\right]+i\eta^{\mu\rho}\left[P^\sigma,P^\nu \right]-i\left[P^\sigma,C^{\rho,\mu\nu}\right] \\
=&-\eta^{\nu\rho}C^{\mu,\sigma}+\eta^{\mu\rho}C^{\nu,\sigma} \\
\left[P^\rho,\left[P^\sigma,J^{\mu\nu}\right]\right]=&-i\eta^{\sigma\mu}\left[P^\rho,P^\nu\right]+i\eta^{\sigma\nu}\left[P^\rho,P^\mu\right]-i\left[P^\rho,C^{\mu\nu,\sigma}\right] \\
=&-\eta^{\sigma\mu}C^{\nu,\rho}+\eta^{\sigma\nu}C^{\mu,\rho} \\
\therefore \quad 0=&+\eta^{\nu\rho}C^{\mu,\sigma}-\eta^{\mu\rho}C^{\nu,\sigma}-\eta^{\sigma\nu}C^{\mu,\rho}+\eta^{\sigma\mu}C^{\nu,\rho}
\end{align*}
を与える.二つ目の恒等式からは
\begin{align*}
\left[J^{\lambda\eta},\left[J^{\mu\nu},P^\rho \right]\right]=&-i\eta^{\nu\rho}\left[J^{\lambda\eta},P^\mu\right]+i\eta^{\mu\rho}\left[J^{\lambda\eta},P^\nu\right]-i\left[J^{\lambda\eta},C^{\rho,\mu\nu}\right] \\
=&-\eta^{\nu\rho}\left(\eta^{\eta\mu}P^\lambda-\eta^{\lambda\mu}P^\eta +C^{\mu,\lambda\eta}\right)+\eta^{\mu\rho}\left(\eta^{\eta\nu}P^\lambda-\eta^{\lambda\nu}P^\eta+C^{\nu,\lambda\eta}\right) \\
\left[P^\rho,\left[J^{\lambda\eta},J^{\mu\nu}\right]\right]=&-i\eta^{\eta\mu}\left[P^\rho,J^{\lambda\nu}\right]+i\eta^{\lambda\mu}\left[P^\rho,J^{\eta\nu}\right]+i\eta^{\nu\lambda}\left[P^\rho,J^{\mu\eta}\right]-i\eta^{\nu\eta}\left[P^\rho,J^{\mu\lambda}\right]-i[P^\rho,C^{\mu\nu,\lambda\eta}] \\
=&-\eta^{\eta\mu}(\eta^{\rho\lambda}P^\nu-\eta^{\rho\nu}P^\lambda+C^{\lambda\nu,\rho})+\eta^{\lambda\mu}(\eta^{\rho\eta} P^\nu-\eta^{\rho\nu}P^\eta+C^{\eta\nu,\rho}) \\
&+\eta^{\nu\lambda}(\eta^{\rho\mu}P^\eta-\eta^{\rho\eta}P^\mu+C^{\mu\eta,\rho})-\eta^{\nu\eta}(\eta^{\rho\mu}P^\lambda-\eta^{\rho\lambda}P^\mu+C^{\mu\lambda,\rho}) \\
\left[J^{\mu\nu},\left[P^\rho,J^{\lambda\eta}\right]\right]=&-\left[J^{\mu\nu},\left[J^{\lambda\eta},P^\rho\right]\right] \\
=&+\eta^{\eta\rho}\left(\eta^{\nu\lambda}P^\mu-\eta^{\mu\lambda}P^\nu +C^{\lambda,\mu\nu}\right)-\eta^{\lambda\rho}\left(\eta^{\nu\eta}P^\mu-\eta^{\mu\eta}P^\nu+C^{\eta,\mu\nu}\right)
\end{align*}
(三つ目は,一つ目の$(\mu,\lambda)\leftrightarrow (\lambda,\eta)$の置換をすると簡単に得られる.)これらを足し合わせると(ちゃんと$P^\mu$に比例する項は全て打ち消しあって)
\begin{align*}
0=&\eta^{\nu\rho}C^{\mu,\lambda\eta}-\eta^{\mu\rho}C^{\nu,\lambda\eta}-\eta^{\mu\eta}C^{\rho,\lambda\nu}+\eta^{\lambda\mu}C^{\rho,\eta\nu} \\
&+\eta^{\lambda\nu}C^{\rho,\mu\eta}-\eta^{\eta\nu}C^{\rho,\mu\lambda}+\eta^{\rho\lambda}C^{\eta,\mu\nu}-\eta^{\rho\eta}C^{\lambda,\mu\nu}
\end{align*}
が得られる.($C^{\mu\nu,\rho}=-C^{\rho,\mu\nu}$に気を付ける.)三つ目の恒等式からは
\begin{align*}
\left[J^{\lambda\eta},\left[J^{\mu\nu},J^{\rho\sigma}\right]\right]=&-i\eta^{\nu\rho}\left[J^{\lambda\eta},,J^{\mu\sigma}\right]+i\eta^{\mu\rho}\left[J^{\lambda\eta},J^{\nu\sigma}\right]+i\eta^{\sigma\mu}\left[J^{\lambda\eta},J^{\rho\nu}\right]-i\eta^{\sigma\nu}\left[J^{\lambda\eta},J^{\rho\mu}\right] \\
&-i\left[J^{\lambda\eta},C^{\rho\sigma,\mu\nu}\right] \\
=&-\eta^{\nu\rho}(\eta^{\eta\mu}J^{\lambda\mu}-\eta^{\lambda\mu}J^{\eta\sigma}-\eta^{\sigma\lambda}J^{\mu\eta}+\eta^{\sigma\eta}J^{\mu\lambda}+C^{\mu\sigma,\lambda\eta}) \\
&+\eta^{\mu\rho}(\eta^{\eta\nu}J^{\lambda\sigma}-\eta^{\lambda\nu}J^{\eta\sigma}-\eta^{\sigma\lambda}J^{\nu\eta}+\eta^{\sigma\eta}J^{\nu\lambda}+C^{\nu\sigma,\lambda\eta}) \\
&-\eta^{\sigma\mu}(\eta^{\eta\nu}J^{\lambda\rho}-\eta^{\lambda\nu}J^{\eta\rho}-\eta^{\rho\lambda}J^{\nu\eta}+\eta^{\rho\eta}J^{\nu\lambda}+C^{\nu\rho,\lambda\eta}) \\
&+\eta^{\sigma\nu}(\eta^{\eta\mu}J^{\lambda\rho}-\eta^{\lambda\mu}J^{\eta\rho}-\eta^{\rho\lambda}J^{\mu\eta}+\eta^{\rho\eta}J^{\mu\lambda}+C^{\mu\rho,\lambda\eta}) \\
\left[J^{\rho\sigma},\left[J^{\lambda\eta},J^{\mu\nu}\right]\right]=&-\eta^{\eta\mu}(\eta^{\sigma\lambda}J^{\rho\lambda}-\eta^{\rho\lambda}J^{\sigma\nu}-\eta^{\nu\rho}J^{\lambda\sigma}+\eta^{\nu\sigma}J^{\lambda\rho}+C^{\lambda\nu,\rho\sigma}) \\
&+\eta^{\lambda\mu}(\eta^{\sigma\eta}J^{\rho\eta}-\eta^{\rho\eta}J^{\sigma\nu}-\eta^{\nu\rho}J^{\eta\sigma}+\eta^{\nu\sigma}J^{\eta\rho}+C^{\eta\nu,\rho\sigma}) \\
&-\eta^{\nu\lambda}(\eta^{\sigma\eta}J^{\rho\mu}-\eta^{\rho\eta}J^{\sigma\mu}-\eta^{\mu\rho}J^{\eta\sigma}+\eta^{\mu\sigma}J^{\eta\rho}+C^{\eta\mu,\rho\sigma}) \\
&+\eta^{\nu\eta}(\eta^{\sigma\lambda}J^{\rho\mu}-\eta^{\rho\lambda}J^{\sigma\mu}-\eta^{\mu\rho}J^{\lambda\sigma}+\eta^{\mu\sigma}J^{\lambda\rho}+C^{\lambda\mu,\rho\sigma}) \\
\left[J^{\mu\nu},\left[J^{\rho\sigma},J^{\lambda\eta}\right]\right]=&-\eta^{\sigma\lambda}(\eta^{\nu\rho}J^{\mu\rho}-\eta^{\mu\rho}J^{\nu\eta}-\eta^{\eta\mu}J^{\rho\nu}+\eta^{\eta\nu}J^{\rho\mu}+C^{\rho\eta,\mu\nu}) \\
&+\eta^{\rho\lambda}(\eta^{\nu\sigma}J^{\mu\eta}-\eta^{\mu\sigma}J^{\nu\eta}-\eta^{\eta\mu}J^{\sigma\nu}+\eta^{\eta\nu}J^{\sigma\mu}+C^{\sigma\eta,\mu\nu}) \\
&-\eta^{\eta\rho}(\eta^{\nu\sigma}J^{\mu\lambda}-\eta^{\mu\sigma}J^{\nu\lambda}-\eta^{\lambda\mu}J^{\sigma\nu}+\eta^{\lambda\nu}J^{\sigma\mu}+C^{\sigma\lambda,\mu\nu}) \\
&+\eta^{\eta\sigma}(\eta^{\nu\rho}J^{\mu\lambda}-\eta^{\mu\rho}J^{\nu\lambda}-\eta^{\lambda\mu}J^{\rho\nu}+\eta^{\lambda\sigma}J^{\rho\mu}+C^{\rho\lambda,\mu\nu}) 
\end{align*}
(一つ目から$(\lambda,\eta,\mu,\nu,\rho,\sigma)\to (\rho,\sigma,\lambda,\eta,\mu,\nu)$と$(\lambda,\eta,\mu,\nu,\rho,\sigma)\to(\mu,\nu,\rho,\sigma,\lambda,\eta)$の置換で二つ目と三つ目が得られる.)これらを足し合わせると,(ちゃんと$J^{\mu\nu}$に比例する項は全て打ち消しあって)
\begin{align*}
0=&+\eta^{\nu\rho}C^{\mu\sigma,\lambda\eta}-\eta^{\mu\rho}C^{\nu\sigma,\lambda\eta} +\eta^{\sigma\mu}C^{\nu\rho,\lambda\eta}-\eta^{\sigma\nu}C^{\mu\rho,\lambda\eta} \\
&+\eta^{\eta\mu}C^{\lambda\nu,\rho\sigma}-\eta^{\lambda\mu}C^{\eta\nu,\rho\sigma}+\eta^{\nu\lambda}C^{\eta\mu,\rho\sigma}-\eta^{\nu\eta}C^{\lambda\mu,\rho\sigma} \\
&+\eta^{\sigma\lambda}C^{\rho\eta,\mu\nu}-\eta^{\rho\lambda}C^{\sigma\eta,\mu\nu}+\eta^{\eta\rho}C^{\sigma\lambda,\mu\nu}-\eta^{\eta\sigma}C^{\rho\lambda,\mu\nu}
\end{align*}
が得られる.(2.7.23)の両辺を$\eta_{\nu\rho}$と縮約させて
\begin{align*}
0=&\eta_{\nu\rho}\eta^{\nu\rho}C^{\mu,\sigma}-\eta_{\nu\rho}\eta^{\mu\rho}C^{\nu,\sigma}-\eta_{\nu\rho}\eta^{\sigma\nu}C^{\mu,\rho}+\eta_{\nu\rho}\eta^{\sigma\mu}C^{\nu,\rho}\\
=&4C^{\mu,\sigma}-C^{\mu,\sigma}-C^{\mu,\sigma}+0 \\
=&2C^{\mu,\sigma} \\
\therefore \quad C^{\mu,\sigma}=&0
\end{align*}
を得る.(ここで,(2.7.19)より
\begin{align*}
\eta_{\mu\nu}C^{\mu,\nu}=&\eta_{\nu\mu}C^{\nu,\mu} \quad (和の添え字の取り換え) \\
=&-\eta_{\mu\nu}C^{\mu,\nu} \\
\therefore \quad \eta_{\mu\nu}C^{\mu,\nu}=&0
\end{align*}
であることを使った.)(2.7.24)の両辺を$\eta_{\nu\rho}$と縮約すると
\begin{align*}
0=&4C^{\mu,\lambda\eta}-C^{\mu,\lambda\eta}-\eta^{\mu\eta}\eta_{\nu\rho}C^{\rho,\lambda\nu}+\eta^{\lambda\mu}\eta_{\nu\rho}C^{\rho,\eta\nu} \\
&+C^{\lambda,\mu\eta}-C^{\eta,\mu\lambda}+C^{\eta,\mu\lambda}-C^{\lambda,\mu\eta} \\
=&3C^{\mu,\lambda\eta}-\eta^{\mu\eta}\eta_{\rho\nu}C^{\rho,\lambda\nu}+\eta^{\lambda\mu}\eta_{\rho\nu}C^{\rho,\eta\nu} \\
\therefore \quad C^{\mu,\lambda\eta}=&\frac{1}{3}\eta^{\mu\eta}\eta_{\rho\nu}C^{\rho,\lambda\nu}-\frac{1}{3}\eta^{\lambda\mu}\eta_{\rho\nu}C^{\rho,\eta\nu} \\
=&\eta^{\mu\eta}C^\lambda-\eta^{\mu\lambda}C^\eta \\
C^\lambda:=&\frac{1}{3}\eta_{\rho\nu}C^{\rho,\lambda\nu}
\end{align*}
となる.(注意,$C^{\mu,\rho\sigma}$は$\rho,\sigma$に関して反対称だが,コンマを超えて入れ替えてはいけない.よって$\eta_{\rho\nu}C^{\rho,\lambda\nu}=0$ではない.以下で行う$C^{\mu\nu,\rho\sigma}$についても同様で,$\mu,\nu$と$\rho,\sigma$に関してはそれぞれ反対称だが,コンマを超えて入れ替えて反対称だとは考えてはいけない.$\eta_{\nu\rho}C^{\nu,\rho}=0$は2つの添え字しかなく$\nu,\rho$について反対称だという性質(2.7.19)からうまくゼロになるだけだ.もちろん$\eta_{\nu\rho}C^{\mu,\nu\rho}=0$と$\eta_{\nu\rho}C^{\nu\rho,\mu\sigma}=0$は成り立っている.)(2.7.25)と$\eta_{\nu\rho}$を縮約して
\begin{align*}
0=&4C^{\mu\sigma,\lambda\eta}-C^{\mu\sigma,\lambda\eta}+\eta^{\rho\mu}\eta_{\nu\rho}C^{\rho\nu,\lambda\eta}+C^{\sigma\mu,\lambda\eta} \\
&+\eta^{\eta\mu}\eta_{\nu\rho}C^{\lambda\nu,\rho\sigma}-\eta^{\lambda\mu}\eta_{\nu\rho}C^{\eta\nu,\rho\sigma} -C^{\mu\eta,\lambda\sigma}+C^{\mu\lambda,\eta\sigma} \\
&+\eta^{\sigma\lambda}\eta_{\nu\rho}C^{\rho\eta,\mu\nu}-C^{\sigma\eta,\mu\lambda}+C^{\sigma\lambda,\mu\eta}-\eta^{\eta\sigma}\eta_{\nu\rho}C^{\rho\lambda,\mu\nu} \\
=&2C^{\mu\sigma,\lambda\eta}+\eta^{\eta\mu}(\eta_{\nu\rho}C^{\lambda\nu,\rho\sigma})-\eta^{\lambda\mu}(\eta_{\nu\rho}C^{\eta\nu,\rho\sigma})+\eta^{\sigma\lambda}(\eta_{\nu\rho}C^{\rho\eta,\mu\nu})+\eta^{\eta\sigma}(\eta_{\nu\rho}C^{\lambda\rho,\mu\nu}) \\
=&2C^{\mu\sigma,\lambda\eta}-\eta^{\eta\mu}(\eta_{\nu\rho}C^{\lambda\nu,\sigma\rho})+\eta^{\lambda\mu}(\eta_{\nu\rho}C^{\eta\nu,\sigma\rho})-\eta^{\sigma\lambda}(\eta_{\rho\nu}C^{\eta\rho,\mu\nu})+\eta^{\eta\sigma}(\eta_{\rho\nu}C^{\lambda\rho,\mu\nu}) \\
\therefore \quad C^{\mu\sigma,\lambda\eta}=&\frac{1}{2}\eta^{\eta\mu}(\eta_{\nu\rho}C^{\lambda\nu,\sigma\rho})-\frac{1}{2}\eta^{\lambda\mu}(\eta_{\nu\rho}C^{\eta\nu,\sigma\rho})+\frac{1}{2}\eta^{\sigma\lambda}(\eta_{\rho\nu}C^{\eta\rho,\mu\nu})-\frac{1}{2}\eta^{\eta\sigma}(\eta_{\rho\nu}C^{\lambda\rho,\mu\nu}) \\
=&\eta^{\eta\mu}C^{\lambda\sigma}-\eta^{\lambda\mu}C^{\eta\sigma}+\eta^{\sigma\lambda}C^{\eta\mu}-\eta^{\eta\sigma}C^{\lambda\mu} \\
C^{\lambda\sigma}:=&\frac{1}{2}\eta_{\nu\rho}C^{\lambda\nu,\sigma\rho}
\end{align*}
を得る.(縮約することで所謂,情報を少なくしてこれらの式を出したが,逆にこれらを(2.7.24)(2.7.25)に代入すると自動的に満たしているから,ヤコビ恒等式からはこれ以上有益な情報は得られない.)これらの結果と交換関係(2.4.12)(2.4.13)(2.4.14)の右辺との類似性を見比べよう.$J^{\mu\nu}\to C^{\mu\nu},P^\rho \to C^\rho$と置き換えたものと同じだ!よって,もしそれぞれの$C$がゼロでない(中心電荷が存在する)なら,新しい生成子を以下のように定義してこれらの係数を消すことができる.
\begin{align*}
\tilde{P}^\mu:=&P^\mu +C^\mu \\
\tilde{J}^{\mu\sigma}:=& J^{\mu\sigma}+C^{\mu\sigma}
\end{align*}
実際この結果,交換関係は
\begin{align*}
i[\tilde{J}^{\mu\nu},\tilde{J}^{\rho\sigma}]=&i[J^{\mu\nu},J^{\rho\sigma}]+i[J^{\mu\nu},C^{\rho\sigma}]+i[C^{\mu\nu},J^{\rho\sigma}]+i[C^{\mu\nu},C^{\rho\sigma}] \\
=&i[J^{\mu\nu},J^{\rho\sigma}] \\
=&\eta^{\nu\rho}J^{\mu\sigma}-\eta^{\mu\rho}J^{\nu\sigma}-\eta^{\mu\sigma}J^{\rho\nu}+\eta^{\nu\sigma}J^{\rho\mu} +C^{\rho\sigma,\mu\nu}\\
=&\eta^{\nu\rho}J^{\mu\sigma}-\eta^{\mu\rho}J^{\nu\sigma}-\eta^{\mu\sigma}J^{\rho\nu}+\eta^{\nu\sigma}J^{\rho\mu} \\
&+\eta^{\nu\rho}C^{\mu\sigma}-\eta^{\mu\rho}C^{\nu\sigma}-\eta^{\mu\sigma}C^{\rho\nu}+\eta^{\nu\sigma}C^{\rho\mu} \\
=&\eta^{\nu\rho}(J^{\mu\sigma}+C^{\mu\sigma})-\eta^{\mu\rho}(J^{\nu\sigma}+C^{\nu\sigma})-\eta^{\mu\sigma}(J^{\rho\nu}+C^{\rho\nu})+\eta^{\nu\sigma}(J^{\rho\mu}+C^{\rho\mu}) \\
=&\eta^{\nu\rho}\tilde{J}^{\mu\sigma}-\eta^{\mu\rho}\tilde{J}^{\nu\sigma}-\eta^{\mu\sigma}\tilde{J}^{\rho\nu}+\eta^{\nu\sigma}\tilde{J}^{\rho\mu} \\
i[\tilde{J}^{\mu\nu},\tilde{P}^\rho]=&i[J^{\mu\nu},P^\rho] \\
=&\eta^{\nu\rho}P^\mu -\eta^{\mu\rho}P^\nu+C^{\rho,\mu\nu} \\
=&\eta^{\nu\rho}P^\mu -\eta^{\mu\rho}P^\nu +\eta^{\nu\rho}C^\mu -\eta^{\mu\rho}C^\nu \\
=&\eta^{\nu\rho}(P^\mu+C^\mu) -\eta^{\mu\rho}(P^\nu+C^\nu) \\
=&\eta^{\nu\rho}\tilde{P}^\mu -\eta^{\mu\rho}\tilde{P}^\nu \\
i[\tilde{P}^\mu,\tilde{P}^\rho]=&i[P^\mu,P^\rho]=C^{\mu,\rho}=0
\end{align*}
となって,通常のポアンカレ代数と同じ形になる.交換関係は今後常にこの形(中心電荷は存在しない)として,チルダを落とすことにする.\par
ところで,$J^{\mu\nu}$だけの代数(ローレンツ代数)に中心電荷が存在しないことは,この代数が「半単純」であることからすぐに結論付けることができる.(半単純リー代数とは,「不変可換」部分代数,つまりお互いに可換な生成子の集合で,かつそれと他の生成子との交換関係が必ずこの集合に属するもの,が存在しないリー代数をいう.たとえば$ISO(2)$のリー代数(2.5.35)(2.5.36)(2.5.37)は不変可換部分代数は$A,B$で張られ,実際に他の生成子$J_3$との交換関係が$A,B$で張られる集合に属しているから,これは半単純ではない.)一般的な定理によれば,半単純リー代数のどのような中心電荷も(2.7.32)のような生成子の再定義で取り除ける.一方,$J^{\mu\nu}$と$P^\mu$によって張られる全ポアンカレ代数は半単純ではない.実際$P^\mu$が不変可換部分代数をなす.したがって,その中心電荷も消せることを証明するために長い議論が必要だったのだ.実際,2.4節で論じた非半単純ガリレイ部分代数には中心電荷である質量演算子$M$がある.\par
非斉次ローレンツ群はこうして,中心電荷が存在しても生成子の再定義によってそれらを取り除けることがわかったので,固有射影表現を排除するための2つの条件のうち(a)の条件は満たされていることがわかった!次に調べるべきは,ポアンカレ群が単連結かどうかだ.

\vskip\baselineskip


\textbf{(B)トポロジー}\par
非斉次ローレンツ群のトポロジーを調べるには,非斉次ローレンツ変換を$2\times 2$複素行列で表すのが便利だ.任意の実4元ベクトル$V^\mu$から,エルミート$2\times 2$行列が
\begin{align*}
v:= V^\mu \sigma_\mu =\left(
\begin{matrix}
V^0 +V^3 & V^1 -iV^2 \\
V^1 +i V^2 & V^0-V^3 
\end{matrix}
\right)
\end{align*}
ここで$\sigma_\mu$は,
\begin{align*}
\sigma_0 :=1 =\left(
\begin{matrix}
1 & 0 \\
0 & 1
\end{matrix}
\right) ,\quad \sigma_1=\left(
\begin{matrix}
0 & 1 \\
1 & 0
\end{matrix}
\right),\quad \sigma_2 =\left(
\begin{matrix}
0 & -i \\
i & 0
\end{matrix}
\right) ,\quad \sigma_3=\left(
\begin{matrix}
1 & 0 \\
0 & -1
\end{matrix}
\right)
\end{align*}
からなる.逆に,任意の$2\times2$エルミート行列は上の形に書ける(後に示す)ので,実4元ベクトル$V^\mu$が定義できる.\par
エルミート性は,任意の複素$2\times 2$行列$\lambda$を使った以下の変換で保たれる.
\begin{align*}
v\to \lambda v \lambda^\dagger 
\end{align*}
(実際$(\lambda v \lambda^\dagger)^\dagger=\lambda v^\dagger \lambda^\dagger =\lambda v \lambda^\dagger$となってエルミートだ.)さらに4元ベクトルの不変二乗積$V_\mu V^\mu$は$v$を用いて
\begin{align*}
V_\mu V^\mu=&(V^1)^2+(V^2)^2 +(V^3)^2 -(V^0)^2 \\
=&-\left[(V^0+V^3)(V^0-V^3)-(V^1-iV^2)(V^1+iV^2)\right]=-\mathrm{Det}v
\end{align*}
と書ける.この行列式は変換(2.7.37)のもとで
\begin{align*}
-\mathrm{Det}v \to& -\mathrm{Det}(\lambda v \lambda^\dagger) \\
=& -(\mathrm{Det}\lambda)(\mathrm{Det}\lambda)^* \mathrm{Det} v \quad \mathrm{Det}(AB)=(\mathrm{Det}A)(\mathrm{Det}B)\\
=&-|\mathrm{Det}\lambda|^2 \mathrm{Det} v
\end{align*}
と変換される.したがって(複素数の絶対値は必ず正だから)
\begin{align*}
|\mathrm{Det}\lambda|=1
\end{align*}
ならば不変二乗積$V^\mu V_\mu$は不変だ.(2.7.39)を満たす複素$2\times 2$行列$\lambda$は(2.7.38)を不変にする実線形変換$\tensor{\Lambda}{^\mu_\nu}$,つまり斉次ローレンツ変換$\Lambda(\lambda)$を定義する.実際,$\lambda \sigma_\mu \lambda^\dagger$は再びエルミート行列であるから$\sigma_\nu$に実係数をかけて一意に展開することができて,
\begin{align*}
\lambda \sigma_\mu \lambda^\dagger=&\tensor{\Lambda(\lambda)}{^\nu_\mu}\sigma_\nu
\end{align*}
とおくと,
\begin{align*}
v=V^\mu \sigma_\mu \to& \lambda v\lambda^\dagger \\
=&V^\mu \lambda \sigma_\mu \lambda^\dagger \\
=&V^\mu \tensor{\Lambda(\lambda)}{^\nu_\mu}\sigma_\nu \\
=&(\tensor{\Lambda(\lambda)}{^\nu_\mu} V^\mu) \sigma_\mu
\end{align*}
となって,さらに$\lambda$が$|\mathrm{Det}\lambda|=1$を満たすようなものであれば二乗積は不変だったのだから
\begin{align*}
|\mathrm{Det}\lambda|=&1 \quad \Rightarrow \quad \eta_{\mu\nu}V^\mu V^\nu =\eta_{\mu\nu}\tensor{\Lambda(\lambda)}{^\mu_\rho}V^\rho \tensor{\Lambda(\lambda)}{^\nu_\sigma}V^\sigma \\
\therefore \quad \eta_{\mu\nu}=&\eta_{\mu\nu}\tensor{\Lambda(\lambda)}{^\mu_\rho} \tensor{\Lambda(\lambda)}{^\nu_\sigma}
\end{align*}
したがって$\Lambda(\lambda)$はローレンツ変換となる.構成から明らかに$\lambda$に対して$\Lambda(\lambda)$がただ一つ定まり,これは写像となる.さらにそのような二つの行列$\lambda,\bar{\lambda}$について
\begin{align*}
(\lambda \bar{\lambda})V^\mu \sigma_\mu (\lambda \bar{\lambda})^\dagger=&\Bigl(\tensor{\Lambda(\lambda\bar{\lambda})}{^\mu_\nu}V^\nu\Bigr)\sigma_\mu \\
=\lambda (\bar{\lambda} V^\mu \bar{\lambda}^\dagger ) \lambda^\dagger=&\lambda \tensor{\Lambda(\bar{\lambda})}{^\mu_\nu}V^\nu \sigma_\mu \lambda^\dagger \\
=&\tensor{\Lambda(\lambda)}{^\mu_\rho}\tensor{\Lambda(\bar{\lambda})}{^\rho_\nu}V^\nu \sigma_\mu
\end{align*}
となるから,以下が成立する.
\begin{align*}
\Lambda(\lambda\bar{\lambda})=\Lambda(\lambda)\Lambda(\bar{\lambda})
\end{align*}
したがってこの$\lambda$から$\Lambda(\lambda)$への写像は積を保存し,すなわち準同型写像である.\par
ところで,二つの$\lambda$が全体の位相だけ異なっている$(\lambda'=\lambda e^{i\theta}(\phi\in \mathbb{R}))$とき,(2.7.37)で$v$には同じ影響を及ぼす
\begin{align*}
\lambda' v \lambda'^\dagger=\lambda e^{i\theta} v e^{-i\theta} \lambda^\dagger =\lambda v \lambda^\dagger
\end{align*}
よってこの全体の位相だけ異なっている二つの$\lambda$は同一のローレンツ変換に対応するから,(2.7.39)により$\mathrm{Det}\lambda=e^{i\theta}(\theta\in \mathbb{R})$となっているなら$\lambda\to \lambda e^{-i\theta}$と再定義して左辺の位相を消すだけの自由度がある.よって$\lambda$の位相を
\begin{align*}
\mathrm{Det}\lambda=1
\end{align*}
となるように選んでおく.これは(2.7.39)を満たしているから(2.7.41)と矛盾しない.行列式が1の複素$2\times 2$行列は$SL(2,\mathbb{C})$と知られる群をなす.
\begin{align*}
SL(2,\mathbb{C})=\{ \lambda | \mathrm{Det}\lambda=1 ,\lambda \in GL(2,\mathbb{C})\}
\end{align*}
群の元は$4-1=3$個の複素パラメータ,つまり6個の実パラメータに依存する.これはローレンツ群と同じ数だ.($\tensor{\Lambda}{^\mu_\nu}$は$4\times 4$行列で$16$個の要素があるが,条件式$\eta_{\mu\nu}\tensor{\Lambda}{^\mu_\rho}\tensor{\Lambda}{^\nu_\sigma}=\eta_{\rho\sigma}$によって制限されている.$\rho,\sigma$について対称なのでこの条件式は$10$個あり,よって独立パラメータは6個だ.)ただし,$SL(2,\mathbb{C})$はローレンツ群と同一\uwave{ではない}.もし$\lambda$が$SL(2,\mathbb{C})$の行列ならば($n\times n$行列について$\mathrm{Det}(aA)=a^n\mathrm{Det}A$だから),$-\lambda$も
\begin{align*}
\mathrm{Det}(-\lambda)=(-1)^2\mathrm{Det}\lambda=\mathrm{Det}\lambda=1
\end{align*}
となり,$SL(2,\mathbb{C})$の行列であることがわかる.しかし(2.7.40)は$\lambda$について二次で入っているから,$\lambda$も$-\lambda$も同じローレンツ変換を与える.
\begin{align*}
(-\lambda)V^\mu \sigma_\mu (-\lambda)^\dagger =&(\tensor{\Lambda(-\lambda)}{^\mu_\nu}V^\nu)\sigma_\mu \\
=\lambda V^\mu \sigma_\mu \lambda=&(\tensor{\Lambda(\lambda)}{^\mu_\nu}V^\nu)\sigma_\mu \\
\Lambda(-\lambda)=\Lambda(\lambda)
\end{align*}
($\sigma_\mu$は$2\times 2$行列の基底を張り,一意的に展開されるから,それぞれの$\sigma_\mu$の係数は等しくなる.)実際
\begin{align*}
\lambda(\theta)=\left(
\begin{matrix}
e^{i\theta/2} & 0 \\
0 & e^{-i\theta/2}
\end{matrix}
\right)
\end{align*}
は$\mathrm{Det}\lambda=1$を満たすから$SL(2,\mathbb{C})$の元であり,(2.7.40)に用いると
\begin{align*}
\lambda v \lambda^\dagger =&\left(
\begin{matrix}
e^{i\theta/2} & 0 \\
0 & e^{-i\theta/2}
\end{matrix}
\right)\left(
\begin{matrix}
V^0+V^3 & V^1-iV^2 \\
V^1+iV^2 & V^0 -V^3
\end{matrix}
\right)\left(
\begin{matrix}
e^{-i\theta/2} & 0 \\
0 & e^{i\theta/2}
\end{matrix}
\right) \\
=&\left(
\begin{matrix}
e^{i\theta/2} & 0 \\
0 & e^{-i\theta/2}
\end{matrix}
\right)\left(
\begin{matrix}
(V^0+V^3)e^{-i\theta/2} & (V^1-iV^2)e^{i\theta/2} \\
(V^1+iV^2)e^{-i\theta/2} & (V^0 -V^3)e^{i\theta/2}
\end{matrix}
\right) \\
=&\left(
\begin{matrix}
V^0+V^3 & (V^1-iV^2)e^{i\theta} \\
(V^1+iV^2)e^{-i\theta} & V^0 -V^3
\end{matrix}
\right) \\
=V'^\mu \sigma_\mu=&\left(
\begin{matrix}
V'^0+V'^3 & V'^1-iV'^2 \\
V'^1+iV'^2 & V'^0 -V'^3
\end{matrix}
\right) \\
\therefore \quad (V^1-iV^2)e^{i\theta}=&V'^1-iV'^2 ,\quad (V^1+iV^2)e^{-i\theta}=V'^1+iV'^2 ,\quad V'^3=V^3 , V^0=V'^0 \\
V'^1=&\frac{e^{i\theta}+e^{-i\theta}}{2}V^1-i\frac{e^{i\theta}-e^{-i\theta}}{2}V^2=\cos\theta V^1+\sin\theta V^2 \\
V'^2=&\frac{e^{-i\theta}-e^{i\theta}}{2i}V^1+\frac{ie^{-i\theta}+ie^{i\theta}}{2i}V^2=-\sin\theta V^1 +\cos\theta V^2 \\
\therefore \quad \left(
\begin{matrix}
V'^1 \\
V'^2 \\
V'^3 \\
V'^0
\end{matrix}
\right)=&\left(
\begin{matrix}
\cos \theta & \sin\theta & 0 & 0 \\
-\sin\theta & \cos\theta & 0 & 0 \\
0 & 0 &1 & 0 \\
0 & 0 & 0 &1
\end{matrix}
\right)\left(
\begin{matrix}
V^1 \\
V^2 \\
V^3 \\
V^0
\end{matrix}
\right) \\
\Lambda(\lambda(\theta))=&\left(
\begin{matrix}
\cos \theta & \sin\theta & 0 & 0 \\
-\sin\theta & \cos\theta & 0 & 0 \\
0 & 0 &1 & 0 \\
0 & 0 & 0 &1
\end{matrix}
\right)
\end{align*}
となって,3軸周りの角度$\theta$の回転をおこすローレンツ変換$\Lambda(\lambda(\theta))$に対応することがわかる.$\theta=2\pi$とすると,これは角度$2\pi$の回転を引き起こし,それはこれは回転のない$\theta=0$の単位元$\Lambda(\lambda(2\pi))=1$に対応する.しかし$\lambda(2\pi)=-1$であるから,異なる$\lambda$に対して同じローレンツ変換が対応していることが簡単にわかる.したがって,ローレンツ群は$SL(2,\mathbb{C})/\mathbb{Z}_2$,すなわち行列式が$1$の複素$2\times 2$行列で,$\lambda$と$-\lambda$を同一視した群に同型だ.
\begin{align*}
SO(3,1) \simeq SL(2,\mathbb{C})/\mathbb{Z}_2
\end{align*}
つまり,$\Lambda:SL(2,\mathbb{C})\to SO(3,1)$は$\lambda=+1,-1$をともに単位元$\Lambda(\lambda)=1$に送る全射$\mathrm{Im}\Lambda=SO(3,1)$だから,$\mathrm{Ker}\Lambda=\mathbb{Z}_2=\{+1,-1\}$となり,したがって群の準同型定理により同型
\begin{align*}
SO(3,1)=\mathrm{Im}\Lambda \simeq SL(2,\mathbb{C})/\mathrm{Ker}\Lambda=SL(2,\mathbb{C})/\mathbb{Z}_2
\end{align*}
がなりたつ.ちなみに$\Lambda$の全射性は自明ではない.まず微小な$\omega$のときのローレンツ変換$\tensor{\Lambda}{^\mu_\nu}$に対する$\lambda$を示す必要があるが,これは(25.2.17)を導くときに繰り返すので,その計算を参照.さらに有限化する必要がある.微小ローレンツ行列
\begin{align*}
\tensor{\Lambda}{^\mu_\nu}=&\delta^\mu_\nu+\tensor{\omega}{^\mu_\nu} \\
=&\delta^\mu_\nu +\frac{i}{2}\omega_{\rho\sigma}(i\delta^\rho_\nu \eta^{\mu\sigma}-i\delta^\sigma_{\nu}\eta^{\mu\rho}) \\
=&\tensor{\left[1+\frac{i}{2}\omega_{\rho\sigma}\mc{J}^{\rho\sigma}\right]}{^\mu_\nu} \quad \Bigl(\tensor{\left(\mc{J}^{\rho\sigma}\right)}{^\mu_\nu}:=i\delta^\rho_\nu \eta^{\mu\sigma}-i\delta^\sigma_{\nu}\eta^{\mu\rho} \Bigr)
\end{align*})
から,有限で任意の$SO(3,1)$の元は
\begin{align*}
\tensor{\Lambda}{^\mu_\nu}=&\tensor{\left[\exp\left(\frac{i}{2}\omega_{\rho\sigma}\mc{J}^{\rho\sigma}\right)\right]}{^\mu_\nu}
\end{align*}
と書くことができる.これは(2.4.27)あたりの計算で示した方法と同じやり方だ.そしてそれに対応する(25.2.17)の$\lambda$は
\begin{align*}
\lambda=&1+\frac{1}{2}\left[\frac{1}{2}i\epsilon_{ijk}\omega_{ij}+\omega_{k0}\right] \\
=&1+\left[i\theta_k \frac{\sigma_k}{2}+\beta_k \frac{\sigma_k}{2}\right] \quad \because \theta_k=\frac{1}{2}\epsilon_{ijk}\omega_{ij} ,\quad \beta_i=\omega_{i0}
\end{align*}
より有限の大きさにして
\begin{align*}
\lambda=\exp\left(i\theta_k \frac{\sigma_k}{2}+\beta_k \frac{\sigma_k}{2}\right)
\end{align*}
と書くことができる.これは$\lambda \sigma_\mu \lambda^\dagger =\tensor{\Lambda}{^\nu_\mu}\sigma_\nu$を満たすことが示せる.(これも有限化すれば簡単に示せる.実際,二つの微小パラメータ$\omega,\bar{\omega}$を用いた$\lambda$をそれぞれ$\lambda(\omega),\bar{\lambda}(\bar{\omega})$とおけば
\begin{align*}
\bar{\lambda} \lambda=&1+\left[i(\theta_k+\bar{\theta}_k) \frac{\sigma_k}{2}+(\beta_k +\bar{\beta}_k)\frac{\sigma_k}{2}\right] \\
\bar{\lambda}\lambda \sigma_\mu (\bar{\lambda}\lambda)^\dagger=&\bar{\lambda}\lambda \sigma_\mu \lambda^\dagger \bar{\lambda}^\dagger \\
=&\bar{\lambda}(\sigma_\mu+\tensor{\omega}{^\nu_\mu}\sigma_\nu)\bar{\lambda} \\
=&\sigma_\mu+(\tensor{\omega}{^\nu_\mu}+\tensor{\bar{\omega}}{^\nu_\mu})\sigma_\nu \\
=&\tensor{\left[1+\frac{i}{2}(\omega+\bar{\omega})_{\rho\sigma}\mc{J}^{\rho\sigma}\right]}{^\nu_\mu} \sigma_\nu
\end{align*}
から両辺のパラメータが加法的になっていることに気付けば,両辺が自明に積分できて,有限な
\begin{align*}
\left[\exp\left(i\theta_k \frac{\sigma_k}{2}+\beta_k \frac{\sigma_k}{2}\right)\right] \sigma_\mu \left[\exp\left(i\theta_k \frac{\sigma_k}{2}+\beta_k \frac{\sigma_k}{2}\right)\right]^\dagger=\tensor{\left[\exp\left(\frac{i}{2}\omega_{\rho\sigma}\mc{J}^{\rho\sigma}\right)\right]}{^\nu_\mu} \sigma_\nu
\end{align*}
がなりたっていることがわかる.)$\mathrm{Det}\exp(A)=\exp(\mathrm{Tr}A)$の公式(後で示す)を使えば,$\mathrm{Tr}\sigma_k=0$と合わせてすぐ$\mathrm{Det}\lambda=1$が示せる.よってこれは$SL(2,\mathbb{C})$の元だ.($\lambda$と$-\lambda$のどちらかが存在することさえ示せればよい.二重性によりもう片方の存在は自動的に保証される.)これで任意の元$\Lambda\in SO(3,1)$に対して$\lambda \in SL(2,\mathbb{C})$が存在することが示せたので,写像は全射である.\par
(ちなみにだが,$SO(3,1)$の任意の元は上のように書けるが,$SL(2,\mathbb{C})$はそうではなく,上の表記で任意の元が書けるわけではない.例えば
\begin{align*}
\lambda=\left(
\begin{matrix}
-1 & 1 \\
0 & -1
\end{matrix}
\right)
\end{align*}
は明らかに$SL(2,\mathbb{C})$の元だが,上の表記のパラメータ$\theta_i ,\beta_i$をどのようにとってもこの元を表せない.あくまで,$SO(3,1)$に対応した$\lambda$か$-\lambda$の片方が上のように書ける,という表記であり,このような例外は所謂もう片方$-\lambda$に対応した元である.)
\vskip\baselineskip


ローレンツ群のトポロジーを本格的に調べる前に,行列の極分解定理について述べておく.これは今後も何度か使うものだからここでしっかりと示す必要がある.\par
任意の正方行列$\lambda \in M(n,\mathbb{C})$に対し,半正定値エルミート行列$H\in M_{\geq 0}(n,\mathbb{C})$とユニタリー行列$U\in U(n)$が存在し,
\begin{align*}
\lambda=UH
\end{align*}
と表すことができる.以下でこれを証明する.\par
まず,$\lambda^\dagger \lambda$は半正定値エルミート行列である.実際
\begin{align*}
(\lambda^\dagger \lambda)=\lambda^\dagger \lambda
\end{align*}
であるからエルミートであり,任意のゼロでない$\bm{x}\in \mathbb{C}^n$に対して
\begin{align*}
\bm{x}^\dagger (\lambda^\dagger \lambda)\bm{x} =(\lambda \bm{x})^\dagger \lambda \bm{x}=||\lambda \bm{x}||^2 \geq 0
\end{align*}
であるから半正定値である.エルミート行列には正規直交基底をなす固有ベクトルが存在するから,固有ベクトルで正規直交基底をなすものを
\begin{align*}
\{\bm{m}_1,\cdots ,\bm{m}_n\}
\end{align*}
とすると,これらは
\begin{align*}
\lambda^\dagger \lambda \bm{m}_i=&r_i \bm{m}_i \\
\bm{m}_i^\dagger \bm{m}_j=&\delta_{ij} \quad (i,j=1 ,\cdots , n)
\end{align*}
がなりたつ.ここで$r_i(i=1,\cdots,n)$は$\lambda^\dagger \lambda$の固有値であり,$\lambda^\dagger \lambda$はエルミートだから固有値は全て実数である.さらに
\begin{align*}
\bm{m}_i^\dagger \lambda^\dagger \lambda \bm{m}_i=&||\lambda \bm{m}_i ||^2 \geq 0 \\
=&r_i \bm{m}_i^\dagger \bm{m}_i \\
=&r_i
\end{align*}
より全ての固有値は非負の実数である(半正定値エルミート行列の性質).そこで固有値を正のものとゼロのものに分けて
\begin{align*}
r_i >&0 \quad (i=1,\cdots,r) \\
r_i=&0 \quad (i=r+1,\cdots ,n)
\end{align*}
とする.$r$は正の固有値の数である.$i=r+1,\cdots ,n$に関して
\begin{align*}
||\lambda \bm{m}_i ||^2= \bm{m}_i \lambda^\dagger \lambda \bm{m}_i =r_i =0
\end{align*}
であるから,内積の定義により$\lambda \bm{m}_i=0(i=r+1,\cdots ,n)$である.ところで,正規直交基底の性質から
\begin{align*}
I= \sum_{i=1}^n \bm{m}_i \bm{m}_i^\dagger
\end{align*}
と書くことができて,
\begin{align*}
\lambda=&\lambda I \\
=&\sum_{i=1}^n \lambda \bm{m}_i \bm{m}_i^\dagger \\
=&\sum_{i=1}^r \lambda \bm{m}_i \bm{m}_i^\dagger \quad \because \lambda \bm{m}_i=0(i=r+1,\cdots ,n)\\
\end{align*}
である.ここで
\begin{align*}
\bm{v}_i:=\frac{1}{\sqrt{r_i}}\lambda \bm{m}_i
\end{align*}
と定めると
\begin{align*}
\lambda =\sum_{i=1}^r \sqrt{r_i}\bm{v}_i \bm{m}_i^\dagger
\end{align*}
と書ける.$||\lambda\bm{m}_i||^2=r_i$より
\begin{align*}
\bm{v}_i^\dagger \bm{v}_j=\frac{r_i}{\sqrt{r_i}\sqrt{r_j}}\delta_{ij}=\delta_{ij}
\end{align*}
を満たし,$\bm{v}_i(i=1,\cdots ,r)$は正規直交基底をなす独立なベクトルである.したがって$\mathrm{Spec}\{\bm{v}_i,\cdots ,\bm{v}_r\}$は$n$次元ベクトル空間の中の$r$次元部分空間を構成する.この部分空間に対する$n-r$次元の補空間の正規直交基底を$\{\bm{v}_{r+1},\cdots ,\bm{v}_n\}$とすると,任意の$n$次元ベクトル$\mathbb{C}^n$の元は正規直交基底
\begin{align*}
&\{\bm{v}_1,\cdots ,\bm{v}_r,\bm{v}_{r+1},\cdots , \bm{v}_n\} \\
&\bm{v}_i^\dagger \bm{v}_j =\delta_{ij}
\end{align*}
で一意に展開できる.$i=r+1,\cdots ,n$に関して$r_i=0$であることから,上での$\lambda$の正規直交基底での展開を拡張して
\begin{align*}
\lambda=&\sum_{i=1}^r \sqrt{r_i}\bm{v}_i \bm{m}_i^\dagger +0 \\
=&\sum_{i=1}^n \sqrt{r_i}\bm{v}_i \bm{m}_i^\dagger
\end{align*}
とできる.さらに
\begin{align*}
\lambda=&\sum_{i,j=1}^n \sqrt{r_j}\bm{v}_i \delta_{ij} \bm{m}_j^\dagger \\
=&\sum_{i,j=1}^n \sqrt{r_j}\bm{v}_i \bm{m}^\dagger_i \bm{m}_j \bm{m}_j^\dagger \\
=&\sum_{i=1}^n \bm{v}_i \bm{m}^\dagger_i \sum_{j=1}^n\sqrt{r_j} \bm{m}_j \bm{m}_j^\dagger
\end{align*}
と書ける.ここで行列$U,H$を
\begin{align*}
U:=&\sum_{i=1}^n \bm{v}_i \bm{m}^\dagger_i \\
H:=&\sum_{j=1}^n\sqrt{r_j} \bm{m}_j \bm{m}_j^\dagger
\end{align*}
と定めれば,
\begin{align*}
\lambda=UH
\end{align*}
と書ける.これらの行列は
\begin{align*}
U^\dagger U=&\sum_{i,j=1}^n \bm{m}_i \bm{v}_i^\dagger\bm{v}_j \bm{m}_j^\dagger \\
=&\sum_{i,j=1}^n \bm{m}_i \delta_{ij} \bm{m}_j^\dagger \\
=&\sum_{i=1}^n \bm{m}_i \bm{m}_i^\dagger \\
=&I
\end{align*}
であるから$U$はユニタリー行列であり,
\begin{align*}
H^\dagger=&\left(\sum_{j=1}^n\sqrt{r_j} \bm{m}_j \bm{m}_j^\dagger\right)^\dagger \\
=&\sum_{j=1}^n\sqrt{r_j} \bm{m}_j \bm{m}_j^\dagger \\
=&H
\end{align*}
であるからエルミートである.また任意のゼロでないベクトル$\bm{x} \in \mathbb{C}^n$で
\begin{align*}
\bm{x}^\dagger H \bm{x}=&\sum_{j=1}^n\sqrt{r_j} \bm{x}^\dagger \bm{m}_j \bm{m}_j^\dagger \bm{x} \\
=&\sum_{j=1}^n\sqrt{r_j} || \bm{m}_j^\dagger \bm{x} ||^2 \geq 0
\end{align*}
より$H$は半正定値行列である.これで証明ができた.\par
これは右極分解と呼ばれる.同様に$\lambda=H'U$と別の半正定値エルミート行列$H'$で分解できることも証明でき,それを左極分解と呼ぶ($U$は右極分解と同じユニタリー行列).これを示すには,右極分解の証明での最後の分解を,代わりに
\begin{align*}
H':=\sum_{j=1}^n\sqrt{r_j} \bm{v}_j \bm{v}_j^\dagger
\end{align*}
として分解すればよい.\par
もし$\lambda$が特に正則行列だとすると,$\lambda^\dagger\lambda$の固有値は全て正$r_i>0$だから,$H$は正定値エルミート行列となる.さらにこのとき,極分解は一意的である.一意性を以下で証明する.\par
別の$V\in U(n),D\in M_{>0}(n,\mathbb{C})$を持ってきて,$\lambda=UH=VD$と表せたとする.まず$D\neq H$と仮定し矛盾を導く.
\begin{align*}
H^2=&H^\dagger H \quad \because Hのエルミート性\\
=&H^\dagger U^\dagger U H \quad \because Uのユニタリー性 \\
=&\lambda^\dagger \lambda
\end{align*}
同様に$D^2=\lambda^\dagger \lambda$だから,$H^2=D^2$となる.ここで
\begin{align*}
(H-D)\bm{v}=c\bm{v}
\end{align*}
を満たすゼロでない固有ベクトル$\bm{v}\in \mathbb{C}^n$とゼロでない固有値$c \in \mathbb{R}$が少なくとも一つ存在する.(なぜなら,$H,D$はともにエルミートだから,$(H-D)$の両側からユニタリー行列で挟むことで固有値が対角成分になるように対角化できて
\begin{align*}
\tilde{U}(H-D)\tilde{U}^{-1}=\left[\mathrm{diag}(c_1 ,\cdots ,c_n)\right]
\end{align*}
もし全ての固有値がゼロだと仮定すると,$H-D=\tilde{U}^{-1} 0 \tilde{U}=0$より$H=D$が導かれ仮定と反する.)このとき,任意の$\bm{u} \in \mathbb{C}^n$に対して$H,D$のエルミート性より$\bm{u}^\dagger HD \bm{u}=\bm{u}^\dagger DH \bm{u}$であるから
\begin{align*}
0=\bm{v}^\dagger(H^2-D^2)\bm{v}=&\bm{v}^\dagger(H+D)(H-D)\bm{v} \\
=&c \bm{v}^\dagger (H+D) \bm{v}=c \bm{v}^\dagger H \bm{v} +c\bm{v}^\dagger H\bm{v}
\end{align*}
となる.$H,D$の正定値性から$\bm{v}^\dagger H\bm{v},\bm{v}^\dagger D \bm{v}$はともに正で,よって右辺はゼロでない.よって矛盾が導けたので$H=D$である.$\lambda^\dagger \lambda=UH=VH$で$H$の正則性を使えば$U=V$も導かれる.これで証明が完了した.\par
さらに$H$が正定値エルミート行列であるから,その固有値$r_i(i=1,\cdots ,n)$は全て正であり,$r_i=e^{x_i}$を満たす実数$x_i$がただひとつ存在する.したがって
\begin{align*}
H=&P \mathrm{diag}(r_1 ,\cdots ,r_n)P^{-1} \\
=&P \mathrm{diag}(e^{x_1},\cdots e^{x_n})P^{-1} \\
=&P \exp[\mathrm{diag}(x_1,\cdots ,x_n)]P^{-1} \\
=&\exp[P\mathrm{diag}(x_1,\cdots, x_n)P^{-1}]
\end{align*}
ここで$X:=P\mathrm{diag}(x_1,\cdots, x_n)P^{-1}$とすると,これは
\begin{align*}
&X^\dagger =\left[P\mathrm{diag}(x_1,\cdots, x_n)P^{-1}\right]^\dagger =P\mathrm{diag}(x_1,\cdots, x_n)P^{-1}=X
\end{align*}
であるからエルミート行列となり,これを用いて$H=\exp X$と書ける.よって任意の正則行列はユニタリー行列$U$とエルミート行列$X$により
\begin{align*}
\lambda=U e^{X}
\end{align*}
の形で書ける.

\vskip\baselineskip

もう一つ線形代数に関する定理を示しておく.\par
任意の正方行列$A\in M(n,\mathbb{C})$に関して
\begin{align*}
\mathrm{Det}(e^A)=e^{\mathrm{Tr}A}
\end{align*}
である.以下でこれを証明する.\par
(エルミート行列に限らず)正方行列$A \in M(n,\mathbb{C})$にはジョルダン標準形$J$(対角成分が$A$の固有値で,上三角行列なもの)
\begin{align*}
J=\left(
\begin{matrix}
\lambda_1 &                  & \cdots       &              & \\
0             & \lambda_2 &                   &               & \\
0            &    0             & \lambda_3  &                &\\
\vdots   &                   &                  &  \ddots    &\\
0            &     0            &   \cdots     &     0        & \lambda_n
\end{matrix}
\right)
\end{align*}
が存在し,ユニタリー行列$P$と$A=PJP^{-1}$で関係している.(これは証明しないで認めることにする.)したがって$A$の指数行列$e^A$は
\begin{align*}
e^A=&e^{PJP^{-1}} \\
=&\sum_{n=0}^\infty \frac{1}{n!}(P J P^{-1})^n \\
=&P \left[\sum_{n=0}^\infty \frac{1}{n!}J ^n\right] P^{-1} \\
=&Pe^J P^{-1}
\end{align*}
と書ける.両辺の行列式をとると
\begin{align*}
\mathrm{Det}(e^A)=&\mathrm{Det}(P e^J P^{-1}) \\
=&\mathrm{Det}(e^J)
\end{align*}
ここで$J$は上三角行列で,対角成分は$A$の固有値$\lambda_i (i=1,\cdots,n)$だから,$e^J$も上三角行列で対角成分は$e^{\lambda_1}, \cdots,e^{\lambda_n} $である.($J$のベキを考えて,$e^A=\sum \frac{1}{n!}A^n$より明らか)
\begin{align*}
e^J=\left(
\begin{matrix}
e^{\lambda_1} &                  & \cdots       &              & \\
0             & e^{\lambda_2} &                   &               & \\
0            &    0             & e^{\lambda_3}  &                &\\
\vdots   &                   &                  &  \ddots    &\\
0            &     0            &   \cdots     &     0        & e^{\lambda_n}
\end{matrix}
\right)
\end{align*}
よって,行列式の定義を思い出すと,対角成分の積の項だけが生き残るから
\begin{align*}
\mathrm{Det}(e^A)=&\mathrm{Det}(e^J)e^J \\
=&e^{\lambda_1}e^{\lambda_2}\cdots e^{\lambda_n}  \\
=&e^{\lambda_1+\cdots +\lambda_n} \\
=&e^{\mathrm{Tr}A}
\end{align*}
がなりたつ.最後の変形は,行列のトレースとその行列の固有値の和が等しいことを用いた.(トレースの巡回性とジョルダン標準形から
\begin{align*}
\mathrm{Tr}A=\mathrm{Tr}(P JP^{-1})=\mathrm{Tr}J
\end{align*}
であることを用いてもよい.)これで証明が完了した.


\vskip\baselineskip


さて,上で示した通り,極分解定理により任意の特異でない(正則な)行列$\lambda$も以下の形で書ける.
\begin{align*}
\lambda=u e^h
\end{align*}
ここで$u$はユニタリーで$h$はエルミートだ.
\begin{align*}
u^\dagger u=1 ,\quad h^\dagger=h
\end{align*}
ユニタリー行列の行列式$\mathrm{Det}u$は位相因子だ.実際
\begin{align*}
\mathrm{Det}u(\mathrm{Det}u)^*=&\mathrm{Det}u \mathrm{Det}u^\dagger \\
=&\mathrm{Det}(u u^\dagger) =\mathrm{Det}1 =1
\end{align*}
となって,したがって$|\mathrm{Det}u|=1,\mathrm{Det}u=e^{i\theta}$である.また$\mathrm{Det} e^h=e^{\mathrm{Tr} h}$は実で正だ.
\begin{align*}
&\left[\mathrm{Det} e^h\right]^*=\mathrm{Det} (e^h)^\dagger=\mathrm{Det} e^{h^\dagger} =\mathrm{Det} e^h \\
& \mathrm{Tr}h \in \mathbb{R}, \quad e^{\mathrm{Tr}h} >0
\end{align*}
ここで$h$はエルミートだから全ての$h$の固有値は実数であることを用いた.よって(2.7.42)の条件は以下の二つを共に要請する.
\begin{align*}
\mathrm{Det}u=&1 \\
\mathrm{Tr}h=&0
\end{align*}
ちなみに,因子$u$は単にローレンツ群の回転部分群を構成する.実際,(2.7.36)より$V^0=\frac{1}{2}\mathrm{Tr}v$であるから,$h=0$のときの$\lambda=u$による変換は
\begin{align*}
V^0=\frac{1}{2}\mathrm{Tr}v \to \frac{1}{2}\mathrm{Tr}(u v u^\dagger) =\frac{1}{2}\mathrm{Tr}v=V^0
\end{align*}
となり,$\Lambda(u)$のもとで$V^0$は不変だ.したがって$u$による変換$\Lambda(u)$は純粋に空間成分の線形変換に対応しており,つまり回転だ.\par
さらにこの極分解は先に証明した通り一意的だから,$SL(2,\mathbb{C})$の元$\lambda$は$(u,h)$の組と1対1対応し,トポロジー的に全ての$u$のなす空間と全ての$h$の空間の直積(点の対の集合)との間に全単射が存在する.さて,$2\times 2$行列は独立な成分が4つあり,したがって4つの行列基底で展開できる.実際
\begin{align*}
(A,B)=\frac{1}{2}\mathrm{Tr}(A^\dagger B) ,\quad (A,B \in M(2,\mathbb{C}))
\end{align*}
で内積を定めれば(内積の公理を満たすことはすぐわかる.),エルミートな4つの行列の組$\sigma_\mu=(1,\bm{\sigma})$は正規直交する基底となる.
\begin{align*}
&(\sigma_0 ,\sigma_0)=\frac{1}{2}\mathrm{Tr}(I^2)=1 ,\quad (\sigma_i ,\sigma_i)=\frac{1}{2}\mathrm{Tr}(\sigma_i^2)=1 \\
&(\sigma_0 ,\sigma_i)=\frac{1}{2}\mathrm{Tr}(I\sigma_i)=0 ,\quad (\sigma_i,\sigma_j)= \frac{1}{2}\mathrm{Tr}(\sigma_i \sigma_j)=0 \quad (i\neq j)
\end{align*}
したがって任意の$2\times 2$行列$A$は$s^\mu \in \mathbb{C}(\mu=0,1,\cdots ,3)$を成分として$A=s^\mu \sigma_\mu$と一意に展開できる.このときエルミートやトレースレスに対応する各成分の制限は
\begin{align*}
\mathrm{Hermite}:& \quad A^\dagger= A \quad \Leftrightarrow s^\mu \in\mathbb{R} \\
\mathrm{Traceless} :& \quad \mathrm{Tr}A=0 \quad \Leftrightarrow s^0=0
\end{align*}
である.したがってトレースレスの任意の$2\times 2$エルミート行列$h$は
\begin{align*}
h=a\sigma_1+b \sigma_2+ c \sigma_3=\left(
\begin{matrix}
c & a- ib \\
a+ib & -c
\end{matrix}
\right)
\end{align*}
と表せる.ここで$a,b,c$は実数で,それ以外特に拘束されていない.$h$はこのように一意的に展開されるため,$h$と組$(a,b,c)$は1対1対応する.すなわち,全ての$h$の空間はトポロジー的に通常の3次元のフラットな空間$\mathbb{R}^3$に等しい.\par
一方,行列式が1の任意のユニタリー行列は以下のように書ける.
\begin{align*}
u=\left(
\begin{matrix}
d+ie & f+ig \\
-f+ig & d-ie
\end{matrix}
\right)=f\sigma_0 +(if)\sigma_1 +(ig)\sigma_2+(ie)\sigma_3
\end{align*}
これを示そう.まず$2\times 2$行列を
\begin{align*}
u= \left(
\begin{matrix}
\alpha & \beta \\
\gamma & \delta 
\end{matrix}
\right)\in M(2,\mathbb{C})
\end{align*}
と書き,$\alpha,\beta,\gamma,\delta\in\mathbb{C}$とする.このとき,まずユニタリー条件は
\begin{align*}
u^\dagger u=&\left(
\begin{matrix}
\alpha^* & \gamma^* \\
\beta^* & \delta^* 
\end{matrix}
\right)\left(
\begin{matrix}
\alpha & \beta \\
\gamma & \delta 
\end{matrix}
\right)=\left(
\begin{matrix}
|\alpha|^2+|\gamma|^2 & (\alpha\beta^*+\gamma\delta^*)^+ \\
\alpha\beta^*+\gamma\delta^* & |\beta|^2+|\delta|^2 
\end{matrix}
\right)=I \\
\Leftrightarrow\quad & |\alpha|^2+|\gamma|^2=|\beta|^2+|\delta|^2=1 ,\quad \alpha\beta^*+\gamma\delta^*=0 \\
\Leftrightarrow\quad &(\alpha , \gamma)^*\left(
\begin{matrix}
\alpha \\
\gamma
\end{matrix}
\right)=(\beta , \delta)^*\left(
\begin{matrix}
\beta \\
\delta
\end{matrix}
\right)=1 ,\quad (\beta , \delta)^*\left(
\begin{matrix}
\alpha \\
\gamma
\end{matrix}
\right)=0
\end{align*}
と書ける.すなわち$(\alpha,\gamma)^T$と$(\beta,\delta)^T$が2次元複素ベクトル空間$\mathbb{C}^2$の正規直交基底をなすことがわかる.しかし$(\alpha,\gamma)^T$は$(-\gamma^*,\alpha^*)^T$とも直交し,2次元ベクトル空間では一つのベクトルと正規直交する二つのベクトルは比例関係にある必要があるから
\begin{align*}
\left(
\begin{matrix}
\beta \\
\delta
\end{matrix}
\right)=&x\left(
\begin{matrix}
-\gamma^* \\
\alpha^*
\end{matrix}
\right) \quad x\in \mathbb{C} \\
\beta =&-x\gamma^* ,\quad \delta=x\alpha^*
\end{align*}
である.したがってユニタリーな$2\times2$行列$u$は
\begin{align*}
u= \left(
\begin{matrix}
\alpha & -x\gamma^* \\
\gamma & x\alpha^*
\end{matrix}
\right) ,\quad |\alpha|^2+|\gamma|^2=1
\end{align*}
という形をとる必要がある.さらに行列式が1であるという条件は
\begin{align*}
\mathrm{Det}u=x|\alpha|^2+x|\gamma|^2=x=1
\end{align*}
を与える.したがって行列式が1の,任意のユニタリーな$2\times 2$行列は,$\gamma=-\beta^*$と再定義して
\begin{align*}
u= \left(
\begin{matrix}
\alpha & \beta \\
-\beta^* & \alpha^*
\end{matrix}
\right) ,\quad |\alpha|^2+|\beta|^2=1
\end{align*}
の形をとらなければならないことが証明できた.$\alpha,\beta$を実部と虚部にわけて$\alpha=d+ie,\beta=f+ig$と実数$d,e,f,g$で書けば
\begin{align*}
u=\left(
\begin{matrix}
d+ie & f+ig \\
-f+ig & d-ie
\end{matrix}
\right) ,\quad d^2+e^2 +f^2+g^2=1 ,\quad d,e,f,g\in \mathbb{R}
\end{align*}
とできる.これが示したかったことだ.したがって全ての$u$が作る$SU(2)$空間は,$d^2+e^2+f^2+g^2=1$を満たす実数パラメータ空間と同型であり,この条件はトポロジー的にフラットな4次元空間$\mathbb{R}^4$内に埋め込まれた3次元球殻$S_3$だ.\par
以上より,$SL(2,\mathbb{C})$はトポロジー的に直積$\mathbb{R}^3 \times S_3$と同型である.この空間は単連結(2点を繋ぐ任意の曲線が連続的に変形できる.あるいは任意のループが1点可縮)だ.(なぜなら$\mathbb{R}^2$と$S_3$は単連結であり,単連結な空間の直積空間は再び単連結となる.)\par
しかし,ここで興味があるのは$SL(2,\mathbb{C})$自体ではなく,ローレンツ群$SO(3,1)\simeq SL(2,\mathbb{C})/\mathbb{Z}_2$だ.$\lambda$と$-\lambda$を同一視することは,ユニタリー因子$u$と$-u$を同一視することと等しい.($e^h$は正定値エルミートである必要があるから,こちらにこの条件を課すことはできない.)したがって,ローレンツ群は$\mathbb{R}^3 \times S_3/\mathbb{Z}_2$と同じトポロジーを持つ.ここで$S_3/\mathbb{Z}_2$は,3次元球殻$S_3$上の点とその反対側にある点を同一視したものだ.この空間は単連結ではない.実際$S_3/\mathbb{Z}_2$は2重連結だ.つまり南半球と北半球を分けて,赤道を通る経路を禁止すると単連結空間になっている.($n$重連結とは,単連結でない空間だが$n$本のカットを入れると空間が単連結になるような空間のこと.)任意の2点を結ぶ経路は,$u\to -u$の反転を含むかどうかで2種類に分けられる.つまり,反転を二度含むような経路は反転を一度も含まない経路と同じだ.(補遺Bで論じるが,これは数学的には$S_3/\mathbb{Z}_2$の基本群,あるいは第一ホモトピー群が$\mathbb{Z}_2$でるという.)同様に,非斉次ローレンツ群はさらに平行移動の$4$つの実数パラメータが存在し,これは$\mathbb{R}^4$に等しい.したがって$\mathbb{R}^4 \times \mathbb{R}^3 \times S_3/\mathbb{Z}_2$と同じトポロジーを持ち,2重連結だ.
\begin{figure}[H]
\centering
\begin{tikzpicture}[scale=0.5]

  % 円を描く
  \draw[thick] (0,0) circle (3);

  % 中央を横に点線で区切る
  \draw[dashed] (-3,0) -- (3,0);

  % 点Pを上半分、点Qを下半分に配置
  \filldraw[black] (60:3) circle (2pt) node[anchor=west] {$u$};
  \filldraw[black] (60:-3) circle (2pt) node[anchor=west] {$-u$};

\draw[thick,->,red] (60:3) .. controls (-1,-2) and (-2,3) .. 
node[pos=0.5 , left] {$\alpha$} (60:3);

  % 内部を通る曲線(ベジェ曲線)
 \draw[thick,->,blue] 
    (60:3) .. controls (1,0) and (-1,0) .. 
    node[midway, below] {$\beta$} (60:-3);

\end{tikzpicture}
\end{figure}

\noindent 次元を一つ落として,わかりやすい図で理解しよう.これは3次元空間に埋め込まれた2次元球殻$S_2$の反対側にある対蹠点を同一視した空間$S_2/\mathbb{Z}_2$を図示したものである.(三次元の図を書くのは手間だったので,矢印線は手前側の球殻面を通っているとでもイメージしてほしい.)$u$と$-u$は同一視された点だから経路$\alpha,\beta$はともに$u$を基点としたループとなる.しかしループ$\alpha$は1点可縮だが,ループ$\beta$はそうではない.したがってこの空間は単連結ではないことがわかる.では2重ループはどうなるだろうか?つまり,$u$から$-u(=u)$を横切り,$u$から再出発し$-u$へ向かうループ$\beta*\beta$を考える.(区別しやすいよう,一周目の経路を$\beta_1$,二週目の経路を$\beta_2$と書こう.)

\begin{figure}[H]
\centering
\begin{tikzpicture}[scale=0.5]

  % 円を描く
  \draw[thick] (0,0) circle (3);

  % 中央を横に点線で区切る
  \draw[dashed] (-3,0) -- (3,0);

  % 点Pを上半分、点Qを下半分に配置
  \filldraw[black] (60:3) circle (2pt) node[anchor=west] {$u$};
  \filldraw[black] (60:-3) circle (2pt) node[anchor=west] {$-u$};

\draw[thick,->,red] (60:3) .. controls (-1,0) and (-1,1) .. 
node[pos=0.5 , left] {$\beta_2$} (60:-3);

  % 内部を通る曲線(ベジェ曲線)
 \draw[thick,->,blue] 
    (60:3) .. controls (1,0) and (-1,0) .. 
    node[midway, below] {$\beta_1$} (60:-3);

\end{tikzpicture}
\end{figure}

\noindent 実はこれは1点可縮である.なぜならば点$P(u)$から出発し,$Q'(-u)$へ行き,$P'(u)$から再出発し$Q(-u)$へと到着するルートの点$P',Q'$は途中通過点に過ぎないから,$Q'(-u)\to Q'(-v)$と移動させるのと同時に$P'(u)\to P'(v)$と連続的にルートを変形することができる.最終的に$Q'(-v)$を$P(u)$に,$P'(v)$を$Q(-u)$へと一致させることで,1点にループを連続的に縮めることができる.

\begin{figure}[H]
\centering
\begin{minipage}{0.25\textwidth}
\begin{tikzpicture}[scale=0.5]

  % 円を描く
  \draw[thick] (0,0) circle (3);

  % 中央を横に点線で区切る
  \draw[dashed] (-3,0) -- (3,0);

  % 点Pを上半分、点Qを下半分に配置
  \filldraw[black] (60:3) circle (2pt) node[anchor=west] {$u$};
  \filldraw[black] (60:-3) circle (2pt) node[anchor=west] {$-u$};

  \filldraw[black] (-60:3) circle (2pt) node[anchor=west] {$-v$};
  \filldraw[black] (-60:-3) circle (2pt) node[anchor=south] {$v$};


\draw[thick,->,red] (-60:-3) .. controls (0,1) and (0,-1) .. 
node[pos=0.5 , left] {$\beta_2$} (60:-3);

  % 内部を通る曲線(ベジェ曲線)
 \draw[thick,->,blue] 
    (60:3) .. controls (0,1) and (0,-1) .. 
    node[midway, right] {$\beta_1$} (-60:3);

\end{tikzpicture}
\end{minipage}
$\to$
\hspace{0.05\textwidth}
\centering
\begin{minipage}{0.25\textwidth}
\begin{tikzpicture}[scale=0.5]

  % 円を描く
  \draw[thick] (0,0) circle (3);

  % 中央を横に点線で区切る
  \draw[dashed] (-3,0) -- (3,0);

  % 点Pを上半分、点Qを下半分に配置
  \filldraw[black] (60:3) circle (2pt) node[anchor=west] {$u$};
  \filldraw[black] (60:-3) circle (2pt) node[anchor=east] {$-u$};

\draw[thick,->,red] (60:-3) .. controls (1,2) and (2,-3) .. 
node[pos=0.5 , left] {$\beta_2$} (60:-3);

  % 内部を通る曲線(ベジェ曲線)
 \draw[thick,->,blue] 
    (60:3) .. controls (-2,+1) and (2,-2) .. 
    node[midway, below] {$\beta_1$} (60:3);

\end{tikzpicture}

\end{minipage}
$\to$
\hspace{0.05\textwidth}
\centering
\begin{minipage}{0.25\textwidth}
\begin{tikzpicture}[scale=0.5]

  % 円を描く
  \draw[thick] (0,0) circle (3);

  % 中央を横に点線で区切る
  \draw[dashed] (-3,0) -- (3,0);

  % 点Pを上半分、点Qを下半分に配置
  \filldraw[black] (60:3) circle (2pt) node[anchor=west] {$u$};
  \filldraw[black] (60:-3) circle (2pt) node[anchor=west] {$-u$};

\end{tikzpicture}

\end{minipage}

\end{figure}

\noindent (数学的にちゃんと示すには,ホモトピー群を用いて$S_3/\mathbb{Z}_2 \simeq \mathbb{R}P^2$(実射影空間)の三角形分割を与えて示す必要がある.その方法は中原幹夫「理論物理学のための幾何学とトポロジー」p145等参照)

\vskip\baselineskip


ローレンツ群(斉次でも非斉次でも)は単連結になっていないので,固有射影表現を持てる.しかし,1から$\bar{\Lambda}$へ,そして$\Lambda \bar{\Lambda}$へ行き,また$1$へ戻ることをもう一度繰り返してできる2重ループは1点に縮めることができるのだった!つまり
\begin{align*}
\left[U^{-1}(\Lambda \bar{\Lambda})U(\Lambda)U(\bar{\Lambda})\right]^2=1
\end{align*}
を得る.したがって
\begin{align*}
&U^{-1}(\Lambda \bar{\Lambda})U(\Lambda)U(\bar{\Lambda})=\pm 1 \\
\therefore \quad &U(\Lambda)U(\bar{\Lambda})=\pm U(\Lambda \bar{\Lambda})
\end{align*}
つまり位相$e^{i\phi(\Lambda,\bar{\Lambda})}$は単に符号$\pm 1$だ.同様に非斉次ローレンツ群についても
\begin{align*}
U(\Lambda,a)U(\bar{\Lambda},\bar{a})=\pm U(\Lambda \bar{\Lambda},\Lambda\bar{a}+a)
\end{align*}
となる.これらの「符号を除いた表現」はなじみのものだ.この符号はどこから来るだろうか?これは角運動量の3軸成分が$\sigma$の状態に3軸周りの$2\pi$回転が働いたときに位相$e^{2i\pi \sigma}$が出るために発生するものだ.つまり3軸周りの角度$\theta$回転に対応する$\lambda(\theta)$に対して,そのユニタリー変換が(2.4.27)から
\begin{align*}
U\Bigl(\Lambda[\lambda(\theta)]\Bigr)=\exp(i\theta J_3) \quad (0\leq \theta<2\pi)
\end{align*}
であることを用いると
\begin{align*}
U\Bigl(\Lambda[\bar{\lambda}]\Bigr)U\Bigl(\Lambda[\lambda(2\pi)]\Bigr)=&U\Bigl(\Lambda[\bar{\lambda}]\Bigr)\exp(-2\pi i J_3)
\end{align*}
であるが,一方
\begin{align*}
U\Bigl(\Lambda[\bar{\lambda}] \Lambda[\lambda(2\pi)]\Bigr)=&U\Bigl( \Lambda[\bar{\lambda} \lambda(2\pi)] \Bigr) \quad \because (2.7.41)\\
=&U\Bigl( \Lambda[-\bar{\lambda}] \Bigr) \\
=&U\Bigl( \Lambda[\bar{\lambda}] \Bigr) \quad \because (\Lambda は \lambda と-\lambda の同一視をする) \\
\end{align*}
であり,よってこの両辺が$\Psi_{p,\sigma}$に作用すると
\begin{align*}
U\Bigl(\Lambda[\bar{\lambda}]\Bigr)U\Bigl(\Lambda[\lambda(2\pi)]\Bigr)\Psi_{p,\sigma}=&e^{2\pi i \sigma}U\Bigl(\Lambda[\bar{\lambda}]\Bigr)\Psi_{p,\sigma} \\
=&\left\{
\begin{array}{ll}
+U \Bigl( \Lambda[ \bar{\lambda} ] \Bigr) \Psi_{p,\sigma} & (\sigma=0,1,2,\cdots )\\
-U \Bigl( \Lambda[ \bar{\lambda} ] \Bigr) \Psi_{p,\sigma} & (\sigma=\frac{1}{2},\frac{3}{2},\cdots)
\end{array}
\right. \\
=&\left\{
\begin{array}{ll}
+U\Bigl(\Lambda[\bar{\lambda}] \Lambda[\lambda(2\pi)]\Bigr) \Psi_{p,\sigma} & (\sigma=0,1,2,\cdots )\\
-U\Bigl(\Lambda[\bar{\lambda}] \Lambda[\lambda(2\pi)]\Bigr) \Psi_{p,\sigma} & (\sigma=\frac{1}{2},\frac{3}{2},\cdots)
\end{array}
\right.
\end{align*}
つまり,3軸周りの$2\pi$回転を含む場合,整数スピンの状態には何の影響もない.したがって整数スピン状態は厳密に$SO(3,1)$ローレンツ群の表現になっている.しかし一方半整数スピンでは符号を変えるため,半整数スピン状態は$SO(3,1)$の表現にはなっておらず射影表現となっている.したがって,(2.7.43)あるいは(2.7.44)は超選択則を与える.つまり,整数と半整数のスピン状態を重ね合わせることはできない.\par
質量がゼロでないときには,小群の生成子のよく知られた表現,つまり単に整数か半整数の$j$の軌道角運動量行列$\mathbf{J}^{(j)}$を使って,純粋に代数的にスピンが整数か半整数に限られることを以前に導いた.一方,質量がゼロの場合は,小群の物理的1粒子状態への作用は,単に運動量の周りの回転であり,この場合ヘリシティが整数か半整数に限られる\uwave{代数的}理由はなかった.しかし,\uwave{トポロジー}的な理由はある.つまり運動量の周りの$4\pi$回転は,上のトポロジー的な議論から,連続的に恒等変換に変形できる必要がある.したがって因子$e^{4\pi i \sigma}$は1であり,
\begin{align*}
4\pi i \sigma=& 2\pi i n \quad (n=0,1,2\cdots ) \\
\therefore \quad \sigma=& \frac{n}{2}=0,\frac{1}{2},1,\frac{3}{2},2, \cdots 
\end{align*}
したがって,$\sigma$は整数か半整数でなければならない.\par

\vskip\baselineskip


なぜ射影表現が大事なのだろうか?ここで改めて論じることにする.ローレンツ変換$\Lambda:SL(2,\mathbb{C})\to SO(3,1)$は$\lambda$と$-\lambda$を同一視し,いわば全射によって「つぶしている」.その影響により,表現になっているのは整数スピンの状態だけだ.半整数スピン表現は$SO(3,1)$表現でなく,$SL(2,\mathbb{C})$表現にしか存在しない.それを無理やり$SO(3,1)$表現の枠組みに組み込もうとするから射影表現にならざるを得ない.それは量子力学で回転$SO(3)$によるスピノルの変換を調べるときにも問題なのだった.($SU(2)$にしかスピノル表現は存在しない.)そこで,半整数スピンの表現を実際に調べる上では,射影表現ではなく被覆群である$SL(2,\mathbb{C})$の表現を調べる.すなわち,$U:SO(3,1)\to B(\mc{H}) ;\Lambda[\lambda]\mapsto U\Bigl(\Lambda[\lambda]\Bigr)$ではなく,$\Lambda$を経由せずに$U:SL(2,\mathbb{C})\to B(\mc{H}) ;\lambda \mapsto U(\lambda)$を考える.(ここで$B(\mc{H})$は$\mc{H}$への作用素のなす空間)こちらは,例えば3軸成分の回転に対応した$\lambda(\theta)$はその表式から明らかなように$2\pi$回転で戻ることが課されず,$4\pi$回転で元に戻るから,そのユニタリー作用素は
\begin{align*}
U(\lambda(\theta))=\exp(i\theta J_3) \quad (0\leq \theta <4\pi)
\end{align*}
となっている.この$SL(2,\mathbb{C})$のユニタリー作用素$U$は半整数スピン状態も含めて表現になっている.
\begin{align*}
U(\lambda )U(\bar{\lambda})=U(\lambda \bar{\lambda})
\end{align*}
上で$2\pi$回転を考えたときに食い違いが生じたのは途中で$\Lambda(\lambda)=\Lambda(-\lambda)$を同一視を用いたからだということに気付けば,この場合にはそのような食い違いが生じないことがわかるだろう.\par
しかし数学的ではなく物理的に考える上ではこの違いは実は超選択則を与える以外には\uwave{重要ではない}.なぜならば我々は状態を射線,すなわち絶対値が$1$だけ異なるベクトルを同値とみなしているからだ.つまり,物理的状態の空間は,元々のヒルベルト空間$\mc{H} (\ni \Psi)$自体ではなく射影空間$\mc{H}/\sim (\ni [\Psi])$であり,位相分だけ異なる違いを許容する.したがって,元のヒルベルト空間に対する作用が表現となっていなくても,射影空間上に写像した結果に限り表現となっている場合が存在する.それが射影表現だ.
\begin{align*}
&U(\Lambda)U(\bar{\Lambda})\Psi=e^{i\phi}U(\Lambda \bar{\Lambda})\Psi \\
&U(\Lambda)U(\bar{\Lambda})[\Psi]=e^{i\phi} U(\Lambda \bar{\Lambda})[\Psi] =U(\Lambda \bar{\Lambda})[\Psi]
\end{align*}
我々はベクトルそのものを観測しているわけではなく,射線$\mc{R}=[\Psi]$から得られる確率や物理的な観測可能量を観測しているだけであり,射線$\mc{R}=[\Psi]$のどの$\Psi$を代表元としてとっているか(つまり状態$\Psi$全体にかかっている位相が何か)を判別する方法は存在しない.それが射影表現を物理的には許容する理由だ.ただし何度も言うように,全体の状態が何らかの状態の重ね合わせ$\Psi=\Psi_A+\Psi_B$となっているときに,$\Psi_A$と$\Psi_B$の間の相対的な位相は依然として重要だ(それは量子力学的な干渉,例えば二重スリット実験の干渉縞やアハラノフ・ボーム効果のような結果を実際に及ぼし,実験的に観測可能なものだからだ).世界の真の対称性だと公理化したのに,変な重ね合わせをした結果,重ね合わせ全体が真の対称性変換に従わない変換をするならば,そのような重ね合わせはそもそも世界が禁止していると結論付ける以外にない.それが超選択則だ.しかし,我々が公理として定める真の対称性である「ローレンツ対称性」を与える群が,実際にこの世界において$SO(3,1)$なのか$SL(2,\mathbb{C})$なのかを本当に判別する方法はない.$SO(3,1)$対称性を公理として認めるならば,それは「整数スピンと半整数スピンの重ね合わせ禁止」の超選択則という形で状態の重ね合わせを制限する.しかし$SL(2,\mathbb{C})$を「($SO(3,1)$のさらに根底にある)ローレンツ対称性の群」として公理として定めても,それは超選択則を与えないだけで他に影響はない.もちろんそれは物理的な状態を,整数スピンと半整数スピンの状態の線形結合として\uwave{用意できることは意味しない}.ただ,自然界の観測された「ローレンツ不変性」からそのような重ね合わせが不可能である,という証明ができないだけだ.実験的に,そのような重ね合わせを実際に作り出すことができたならば,この世界の真の対称性は$SL(2,\mathbb{C})$(あるいはもっと大きな群)かもしれない,という程度しかいえない.\par
この議論から,我々がローレンツ群$SO(3,1)$の射影表現を扱うとしても,超選択則で禁止されるような重ね合わせをしなければ,射影表現に出てくる位相は射影空間で無視して通常の表現とみなして考えればよいことがわかる.つまり単に$\phi=0$としてよい.あるいは半整数スピンでも非射影表現を与える$SL(2,\mathbb{C})$に対称群を拡張してしまい,やはり$\phi=0$で計算してもよい.これで今まで2.5節まで頑張って計算してきた内容が保証されるというものだ.\par


\vskip\baselineskip

同様のことはどの対称群にも当てはまる.もし,そのリー代数が中心電荷を持つならば,全てと可換でその固有値が中心電荷の値になるような生成子をもつように代数を中心拡大すればよい.これは2.4節でガリレイ群のリー代数に質量演算子$M$を付け加えたのと同じ操作だ.このようにして拡張されたリー代数では,もちろん中心電荷をもたず,群の単位元に近い部分では通常の表現しか持たないから,超選択則を必要としない.同様に,リー群$G$が単連結でなければ,$G$の「普遍被覆群」として知られる単連結の群$C$と,$C$の不変部分群$H$を使って$G$は常に$C/H$と表せる.一般に,対称群を$G$の代わりに$C$としてよい.これは$G$からは超選択則\footnote{$SO(3,1)$以外では,例えば電磁理論により$U(1)_{\mathrm{em}}$ゲージ対称性は世界の真の対称性だと考えられている.しかし$U(1)\simeq S_1$は単連結でないから,電荷の異なる状態は重ね合わせられないという超選択則が導かれる.$U(1)$の普遍被覆群は$\mathbb{R}$である.}が導かれ,$C$ではそうではないということを除いて,その結果に差異がないからだ.\par
物理的状態は任意の重ね合わせとして用意できるものも,用意できないものもある.しかし,これは対称性原理から決めることはできない.なぜなら,自然界の対称群をどのようなものと考えるにせよ,常に超選択則を導かないこと以外は同じ結果を与える他の群があるからだ.

\vskip\baselineskip

ちなみにここまでは固有順時ローレンツ群$SO(3,1)$の被覆群について説明してきたが,全ての連結成分を含んだ一般ローレンツ群$O(3,1)$全体の被覆群については何も説明していない.一般に回転群$SO(3)$やローレンツ群$SO(3,1)$に対するスピノル表現も含んだ被覆群はそれぞれ$\mathrm{Spin}(3) \simeq SU(2)$や$\mathrm{Spin}(3,1) \simeq SL(2,\mathbb{C})$となっており,これらをスピン群と呼ぶ.$O(3,1)$は$\mathrm{Det}\Lambda=+1$の制限がないから,その被覆群も当然「Special」でない群である必要があり,$O(3)$や$O(3,1)$に対する被覆群をピン群と呼び$\mathrm{Pin}(3)$や$\mathrm{Pin}(3,1)$と書かれる.スピン(Spin)群から「Special」のSをとったのでピン(Pin)群…という数学者セールの冗談から命名された不遇な群.)$\mathrm{Spin}(n)$や$\mathrm{Pin}(n)$に関しては本間泰史「スピン幾何学」などを参照.どのみち,時間反転や空間反転の対称性は破れていることが実験的に示されているから,これを世界の真の対称性と公理化したとき超選択則がどのように与えられるか,その被覆群がどうか,などを議論することに特に意味はないだろう.


\newpage



\subsection*{補遺A:対称性の表現に関する定理}
この補遺では,どのような対称性変換もヒルベルト空間上で線形ユニタリー
\begin{align*}
(U\Phi,U\Psi)=(\Phi,\Psi) ,\quad U(\xi \Phi+\eta \Psi)=\zeta U\Phi +\eta U\Psi
\end{align*}
か反線形反ユニタリー
\begin{align*}
(U\Phi,U\Psi)=(\Phi,\Psi)^* ,\quad U(\xi \Phi+\eta \Psi)=\xi U\Phi +\eta^* U\Psi
\end{align*}
な演算子で表されるというウィグナーの基礎的な定理の証明を与える.このために主に使う対称性変換の性質は,それが遷移確率を変えない射線変換だということだ.つまり,もし$\Psi_1$と$\Psi_2$が射線$\mc{R}_1$と$\mc{R}_2$に属する状態ベクトルなら,変換を$T:\mc{R}_k \to T\mc{R}_k$とすると,変換された射線$T\mc{R}_1$と$T\mc{R}_2$に属する任意の状態ベクトル$\Psi'_1$と$\Psi_2'$は以下を満たす.
\begin{align*}
|(\Psi_1',\Psi_2')|^2=|(\Psi_1,\Psi_2)|^2
\end{align*}
また対称性変換は逆を持ち(群の逆元の存在から),逆変換も上と同様に遷移確率を変えないものとする.\par
ここで注意.(2.A.1)の$\Psi_k$と$\Psi_k'$はそれぞれ射線$\mc{R}_k$と$\mc{R}_k'$という集合の\uwave{任意の元}であり,したがって対称性変換で直接$\Psi_k \mapsto \Psi'_k$と対応していなくとも構わない.$\mc{R}_k'$の中の元は全て同値関係で結ばれているから,対称性変換$T$が引き起こす実際のヒルベルト空間の変換は$\Psi_k\mapsto e^{i\theta}\Psi_k'$という対応であってもかまわない.対称性変換は射線に対する写像であって,ヒルベルト空間の元たるベクトルの写像はそれとは別であることに注意.むしろ大事なのは,$\mc{R}$の元$\Psi$に対して$T\mc{R}$の\uwave{どの元$\Psi'$を$U\Psi$に対応させるか},ということである.(前提として定まっている対応は,対称性変換が$\mc{R}$に対して$T\mc{R}$を対応させるということだけだ.)結局この節でやることを大まかにいえば,この「対応$U$」を我々がただ一つに定めるのだが,その対応をうまく決めることで(反)ユニタリー(反)線形にできる,ということを示しているに過ぎない.ZFCの公理論的集合論で集合の対応から写像を構築する話を知らないと少し難しいかもしれないが,それを知っていればかなりイメージがしやすいと思う.\par


\vskip\baselineskip

まず,射線$\mc{R}_k(k=1,2,\cdots)$に属し,以下を満たす完全正規直交な状態ベクトルの族$\{\Psi_k\}_{k\in \mathbb{N}}$を考える.
\begin{align*}
(\Psi_k,\Psi_l)=\delta_{kl} \quad (\forall k\in \mathbb{N},\Psi_k \in \mc{R}_k)
\end{align*}
また$\Psi'_k$を,変換された$T\mc{R}_k$に属する任意の状態ベクトルとする.仮定(2.A.1)より
\begin{align*}
|(\Psi'_k,\Psi'_l)|^2=&|(\Psi_k,\Psi_l)|^2=\delta_{kl} \\
\therefore \quad |(\Psi'_k,\Psi'_k)|=&1 , \quad |(\Psi'_k,\Psi'_l)|=0 \quad (k\neq l)
\end{align*}
となる.よって$(\Psi'_k,\Psi'_k)=e^{i\theta}$の形になるが,内積の公理より自分自身との内積は自動的に実かつ正なので$(\Psi'_k,\Psi'_k)=+1$となる.よって
\begin{align*}
(\Psi'_k,\Psi'_l)=\delta_{kl}
\end{align*}
が得られる.\par


\vskip\baselineskip


次に,$\{\Psi_k'\}_{k\in \mathbb{N}}$は完全系をなす,すなわちヒルベルト空間内の任意の元$\Psi$は$\Psi_k'$で展開できることを示す.もし全ての$\Psi'_k$と直交するゼロでない状態ベクトル$\Psi'$が存在すると仮定する(つまり$\Psi_k'$の貼る部分空間に対する直交補空間が存在すると仮定する)ならば,$\Psi'$が属する射線の逆変換$T^{-1}$はゼロでない状態ベクトル$\Psi''$であり,(2,A,1)より全ての$k$について
\begin{align*}
|(\underset{\mc{R}_k}{\underline{\Psi_k}},\underset{\mc{R}}{\underline{\Psi''}})|^2= |(\underset{\mc{R}_k'}{\underline{\Psi_k'}},\underset{\mc{R}'}{\underline{\Psi'}})|^2=0
\end{align*}
となる.しかし仮定より$\{\Psi_k\}_{k\in\mathbb{N}}$は完全系をなすので,これは矛盾.よって直交補空間は存在せず,$\{\Psi_k'\}_{k\in \mathbb{N}}$は完全系をなす.

\vskip\baselineskip


ここまではまだ対称性変換によって$\Psi_k$から,$\mc{R}_k'$の中のどの$\Psi'_k$か,つまり位相分だけ異なる$e^{i\theta}\Psi_k'$のどれに移すのかの任意性があり,写像(対応)が正確に決まっていない(位相分だけ違っても$(e^{i\theta_k}\Psi_k',e^{i\theta_l}\Psi_l')=\delta_{kl}$だから上で示した規格条件と矛盾しないから).よって次に$\Psi'_k$の位相を決める.\par
まず,$\Psi_k$のうち一つを取り出し,それを$\Psi_1$とする.$k\neq 1$についての射線$\mc{S}_k$に属する以下のベクトルを考える.
\begin{align*}
\Upsilon_k := \frac{1}{\sqrt{2}}\left[\Psi_1+\Psi_k\right] \in \mc{S}_k
\end{align*}
(これは規格化されているから,この$\Upsilon_k$の絶対値1の複素数倍の集合$\mc{S}_k$は実際に射線となる.)$\mc{S}_k$から変換された射線$T\mc{S}_k$に属する任意のベクトル$\Upsilon_k'$は,完全性をなす状態ベクトル$\Psi_k'$で展開できるから
\begin{align*}
\Upsilon'_k = \sum_l c_{kl}\Psi_l'
\end{align*}
と書ける.仮定(2.A.1)より
\begin{align*}
|(\underset{T\mc{S}_k}{\Upsilon_k'},\underset{T\mc{R}_1}{\Psi'_1})|^2=&|(\underset{\mc{S}_k}{\Upsilon_k},\underset{\mc{R}_1}{\Psi_1})|=\frac{1}{2} \\
=\left|\sum_l c^*_{kl}(\Psi_l',\Psi'_1)\right|^2=&|c_{k1}|^2 \\
\therefore \quad |c_{k1}|=&\frac{1}{\sqrt{2}} \\
|(\underset{T\mc{S}_k}{\Upsilon_k'},\underset{T\mc{R}_k}{\Psi'_k})|^2=&|(\underset{\mc{S}_k}{\Upsilon_k},\underset{\mc{R}_k}{\Psi_k})|=\frac{1}{2} \\
=\left|\sum_l c^*_{kl}(\Psi_l',\Psi'_k)\right|^2=&|c_{kk}|^2 \\
\therefore \quad |c_{kk}|=&\frac{1}{\sqrt{2}}
\end{align*}
が分かる.$l(\neq 1, k)$に対しては
\begin{align*}
|(\underset{T\mc{S}_k}{\Upsilon_k'},\underset{T\mc{R}_l}{\Psi'_l})|^2=&|(\underset{\mc{S}_k}{\Upsilon_k},\underset{\mc{R}_l}{\Psi_l})|=0 \\
=\left|\sum_{m} c^*_{km}(\Psi_{m}',\Psi'_l)\right|^2=&|c_{kl}|^2 \\
\therefore \quad c_{kl}=&0
\end{align*}
となる.\par
$|c_{k1}|=\frac{1}{\sqrt{2}},|c_{kk}|=\frac{1}{\sqrt{2}}$より$c_{k1}=e^{i\theta}\frac{1}{\sqrt{2}},c_{kk}=e^{i\theta'}\frac{1}{\sqrt{2}}$となって
\begin{align*}
\Upsilon'_k =& \frac{1}{\sqrt{2}}\left[e^{i\theta}\Psi'_1+e^{i\theta'}\Psi'_k\right] \\
\therefore \quad e^{-i\theta}\Upsilon'_k=&\frac{1}{\sqrt{2}}\left[\Psi'_1+e^{i(\theta'-\theta)}\Psi'_k\right]
\end{align*}
よって$\Psi_1'$の位相を自分で一つ決めると,2つの$T\mc{S}_k$と$\mc{R}_k$から状態ベクトル$\Upsilon'_k$と$\Psi_k'$を選んで対応させるとき,どの位相の元を持ってくるかは
\begin{align*}
\Upsilon'_k=&\frac{1}{\sqrt{2}}\left[\Psi'_1+\Psi'_k\right]
\end{align*}
がなりたつように要請すれば自然にただ一つ選び方が決まる.(位相を二つ定めるために$\theta',\theta$の2自由度が対応している.$\Psi_1'$の位相だけは自分の手で決められることは,最終的に$U$と$e^{i\theta}U$が同じ(反)ユニタリー性(反)線形性をもたらすから,全体の位相だけは任意性があることと対応している.)このように選ぶことで対応$\Psi_1 \mapsto \Psi'_1 ,\Psi_k\mapsto \Psi_k',\Upsilon_k \mapsto \Upsilon_k' (k\neq 1)$がただ一つに決まる.全ての$k(\neq 1)$についてこの操作を行うことができて,このように選んだ状態ベクトル$\Upsilon_k'(k\neq 1)$と$\Psi_k'(k\in \mathbb{N})$をそれぞれ$U\Upsilon_k$と$U\Psi_k$と書く.これにより
\begin{align*}
U\frac{1}{\sqrt{2}}[\Psi_1+\Psi_k]=&U\Upsilon_k \\
=&\Upsilon_k'=\frac{1}{\sqrt{2}}[\Psi'_1+\Psi_k']=\frac{1}{\sqrt{2}}[U\Psi_1+U\Psi_k]
\end{align*}
となる.これで$\Psi_k$からの対応$U\Psi_k$が定義できた.これは線形性を表しているように見えるが,しかしまだ一般的な状態ベクトル$\Psi$に対する対応$U\Psi$を定義していないため,作用素$U$が定義できていない.

\vskip\baselineskip


ここで,任意の射線$\mc{R}$に属する任意の状態ベクトル$\Psi$を考える.$\Psi_k$は完全系をなすので,$\Psi_k$で$\Psi$を展開できる.
\begin{align*}
\Psi=\sum_k C_k \Psi_k
\end{align*}
$\Psi_k'$も完全性をなすから,変換された射線$T\mc{R}$に属する任意の状態$\Psi'$は$\Psi'_k=U\Psi_k$で展開できる.
\begin{align*}
\Psi'=\sum_k C'_k U\Psi_k
\end{align*}
ここで仮定(2.A.1)より
\begin{align*}
|(\underset{\mc{R}_k}{\Psi_k} ,\underset{\mc{R}}{\Psi})|^2=&|(\underset{T\mc{R}_k}{U\Psi_k},\underset{T\mc{R}}{\Psi'})|^2 \\
\left|\sum_m C_m(\Psi_k,\Psi_m)\right|^2=& \left|\sum_m C'_m (U\Psi_k,U\Psi_m)\right|^2 \\
\therefore \quad |C_k|^2=&|C'_k|^2 \quad \because (2.A.2)(2.A.3)
\end{align*}
が成立する.(ここでの$k$は$k=1$を含む全ての$k$である.(2.A.3)は今では$(U\Psi_k,U\Psi_l)=\delta_{kl}$と書かれることに注意.)また,全ての$k\neq 1$に対して再び仮定(2.A.1)より
\begin{align*}
|(\underset{\mc{S}_k}{\Upsilon_k},\underset{\mc{R}}{\Psi})|^2=&|(\underset{T\mc{S}_k}{U\Upsilon_k},\underset{T\mc{R}}{\Psi'})|^2 \\
(\mathrm{LHS})=\left|\left(\frac{1}{\sqrt{2}}[\Psi_1+\Psi_k],\sum_l C_l \Psi_l \right)\right|^2=&\left|\frac{1}{\sqrt{2}}\sum_lC_l (\Psi_1,\Psi_l)+\frac{1}{\sqrt{2}}\sum_l C_l (\Psi_k,\Psi_l)\right|^2 \\
=&\frac{1}{2}|C_1+C_k|^2 \\
(\mathrm{RHS})=\left|\left(\frac{1}{\sqrt{2}}[U\Psi_1+U\Psi_k],\sum_l C_l U\Psi_l \right) \right|^2=&\left|\frac{1}{\sqrt{2}}\sum_lC_l (U\Psi_1,U\Psi_l)+\frac{1}{\sqrt{2}}\sum_l C_l (U\Psi_k,U\Psi_l)\right|^2 \\
=&\frac{1}{2}|C_1'+C'_k|^2 \\
\therefore \quad |C_1+C_k|^2=&|C_1'+C'_k|^2 \quad (k\neq 1)
\end{align*}
である.(2.A.9)(2.A.8)の比をとると,全ての$k\neq 1$について
\begin{align*}
\left|1+\left(\frac{C_k}{C_1}\right)\right|^2=&\left|1+\left(\frac{C_k'}{C_1'}\right)\right|^2 \\
1+2\mathrm{Re}\left(\frac{C_k}{C_1}\right)+\left|\frac{C_k}{C_1}\right|^2=&1+2\mathrm{Re}\left(\frac{C_k'}{C'_1}\right)+\left|\frac{C'_k}{C'_1}\right|^2 \\
\therefore \quad \mathrm{Re}\left(\frac{C_k}{C_1}\right)=&\mathrm{Re}\left(\frac{C'_k}{C'_1}\right) \quad \because (2.A.8)
\end{align*}
ここで,任意の2つの複素数$\alpha,\beta \in \mathbb{C}$に対して
\begin{align*}
|\alpha+\beta|^2=&(\alpha+\beta)(\alpha^*+\beta^*) \\
=&|\alpha|^2+|\beta|^2+\alpha\beta^*+\alpha^* \beta \\
=&|\alpha|^2+|\beta|^2+2\mathrm{Re}(\alpha\beta^*)
\end{align*}
であることを用いた.よって(2.A.8)より
\begin{align*}
|C_k|^2=&|C_k'|^2 \\
\left|\frac{C_k}{C_1}\right|^2=&\left|\frac{C'_k}{C'_1}\right|^2 \\
\left\{\mathrm{Re}\left(\frac{C_k}{C_1}\right)\right\}^2+\left\{\mathrm{Im}\left(\frac{C_k}{C_1}\right)\right\}^2=&\left\{\mathrm{Re}\left(\frac{C'_k}{C'_1}\right)\right\}^2+\left\{\mathrm{Im}\left(\frac{C'_k}{C'_1}\right)\right\}^2 \\
\therefore \quad \left\{\mathrm{Im}\left(\frac{C_k}{C_1}\right)\right\}^2=&\left\{\mathrm{Im}\left(\frac{C'_k}{C'_1}\right)\right\}^2 \quad (k\neq 1) \quad \because (2.A.10) \\
\therefore \quad \mathrm{Im}\left(\frac{C_k}{C_1}\right)=&\pm\mathrm{Im}\left(\frac{C'_k}{C'_1}\right) \quad (k\neq 1)
\end{align*}
を得る.(2.A.10)(2.A.11)の両辺を足すと
\begin{align*}
\mathrm{Re}\left(\frac{C_k}{C_1}\right)+i\mathrm{Im}\left(\frac{C_k}{C_1}\right)=\mathrm{Re}\left(\frac{C'_k}{C'_1}\right) \pm i\mathrm{Im}\left(\frac{C'_k}{C'_1}\right) \quad (k\neq 1)
\end{align*}
したがって
\begin{align*}
\quad \frac{C_k}{C_1}=\frac{C'_k}{C'_1} ,\quad \mathrm{or} \quad \quad \frac{C_k}{C_1}=\left(\frac{C'_k}{C'_1}\right)^* \quad (k\neq 1)
\end{align*}
が導かれる.

\vskip\baselineskip

どの$k\neq 1$についても,このどちらか一方だけを選ばなければならないことを以下で示す.つまり,ある$k$では$C_k/C_1=C'_k/C'_1$であり,別の$l\neq k$では$C_l/C_1=(C'_l/C_1')^*$である,と仮定して矛盾を導き,この両立が不可能であることを示す.ここで,これらが本当に異なる場合になるように,両者の比が共に\uwave{複素数である}という前提があるとする.(もし全ての比が実だと,どのような組$(k,l)$をとってきても$C_k/C_1=C'_k/C_1=(C'_k/C_1')^*$が成り立ってしまい,両者のどちらの関係式も満たしてしまう.この例外は言い換えると,全ての$C_k$の位相が一致している($C_1=c_1e^{i\theta},C_k=c_ke^{i\theta} ,\Psi=e^{i\theta}\sum_k c_k \Psi_k$)ことに対応しており,したがって全体の位相を再定義することで全ての係数が実数となることに対応している.ここの議論は後最終的に演算子が線形か反線形かを分類できることを示しているから,この例外(実数係数)は演算子が線形か反線形かを分ける意味がそもそもないことを示している.よってこの可能性は考慮しなくてよい.どちらも比が複素数となる$C_k,C_l$が二つ存在するだけでいいから,いくつかの比が実数となることを禁止しているわけでもない.このあと定義する$\Phi$は早速この比が全て実数になるが,そうなる場合を別に禁止しているわけではない.)前提条件より$k\neq 1 ,l \neq 1 ,k\neq l$とする.この話を整理して,示すべき命題は次の通りだ.

\vskip\baselineskip

射線$C_k$を$\Psi=\sum_k C_k \Psi_k$と展開したときの係数,$C_k'$も$\Psi'=\sum_k C_k'U\Psi_k$と展開したときの係数とする.任意の$k\neq 1$について比$C_k/C_1$は実数でないとする.このとき,全ての$k \neq 1$について$C_k/C_1=C_k'/C'_1$を満たしているか,そうでなければ全ての$k \neq 1$について$C_k/C_1=(C_k'/C'_1)^*$を満たしている.

\vskip\baselineskip

証明しよう.前述の通り,ある$k$では$C_k/C_1=C'_k/C'_1$であり,別の$l\neq k$では$C_l/C_1=(C'_l/C_1')^*$であるような対$(k,l)$がとれる,と仮定する.まず,状態ベクトル
\begin{align*}
\Phi := \frac{1}{\sqrt{3}}\left[\Psi_1 +\Psi_k +\Psi_l \right]
\end{align*}
を定義する.これは規格化されているから,射線に属する.この状態ベクトルの係数の比は全て実数
\begin{align*}
C_1=&\frac{1}{\sqrt{3}},C_k=\frac{1}{\sqrt{3}},C_l=\frac{1}{\sqrt{3}} \\
\frac{C_k}{C_1}=&1 =\frac{C'_k}{C_1'}=\frac{C'_k}{C_1'} \quad \therefore C'_k =C_1' \\
\frac{C_l}{C_1}=&1 =\frac{C'_l}{C_1'}=\frac{C'_l}{C_1'} \quad \therefore C'_l =C_1' \\
\therefore \quad C_1'=&C_k'=C_l'
\end{align*}
($n\neq k$かつ$n\neq l$な$n$では$C_n=0$より$C'_n=0$)ここから
\begin{align*}
|C_1|^2=&\frac{1}{3}=|C_1'|^2 \\
\therefore \quad C_1' =&C_k'=C_l'=\frac{\alpha}{\sqrt{3}} \\
\Phi'=&\frac{\alpha}{\sqrt{3}} \left[U\Psi_1 +U\Psi_k+U\Psi_l\right]
\end{align*}
となる.ここで$\alpha$は$|\alpha|=1$となる任意の複素数だ.\par
遷移確率について仮定(2.A.1)より
\begin{align*}
|(\Phi,\Psi)|^2=&|(\Phi',\Psi')|^2 \\
(\mathrm{LHS})=&\left|\left(\frac{1}{\sqrt{3}}[\Psi_1+\Psi_k +\Psi_l] \, ,\,\sum_m C_m \Psi_m\right)\right|^2 \\
=&\frac{1}{3}|C_1+C_k+C_l|^2 \\
(\mathrm{RHS})=& \left| \left(\frac{\alpha}{\sqrt{3}}[U\Psi_1+U\Psi_k +U\Psi_l]\,,\,\sum_m C'_m U\Psi_m\right) \right|^2 \\
=&\frac{1}{3}|C_1'+C_k'+C_l'|^2
\end{align*}
$|C_1|^2=|C_1'|^2$で両辺で比をとって
\begin{align*}
\left|1+\frac{C_k}{C_1}+\frac{C_l}{C_1}\right|^2=\left|1+\frac{C_k'}{C_1'}+\frac{C_l'}{C_1'}\right|^2
\end{align*}
また仮定より$C_k/C_1=C_k'/C_1',C_l/C_1=(C_l'/C_1')^*$であるから
\begin{align*}
\left|1+\frac{C_k}{C_1}+\frac{C_l}{C_1}\right|^2=\left|1+\frac{C_k}{C_1}+\frac{C_l^*}{C_1^*}\right|^2
\end{align*}
となる.展開して
\begin{align*}
(\mathrm{LHS})=&1^2+\left|\frac{C_k}{C_1}\right|^2+\left|\frac{C_l}{C_1}\right|^2+2\mathrm{Re}\left(1\cdot \frac{C^*_k}{C^*_1}\right)+2\mathrm{Re}\left(\frac{C_k}{C_1}\cdot \frac{C_l^*}{C_1^*}\right)+2\mathrm{Re}\left(\frac{C_l}{C_1}\cdot 1\right) \\
(\mathrm{RHS})=&1^2+\left|\frac{C_k}{C_1}\right|^2+\left|\frac{C^*_l}{C^*_1}\right|^2+2\mathrm{Re}\left(1\cdot \frac{C^*_k}{C^*_1}\right)+2\mathrm{Re}\left(\frac{C_k}{C_1}\cdot \frac{C_l}{C_1}\right)+2\mathrm{Re}\left(\frac{C_l^*}{C_1^*}\cdot 1\right) \\
=&1^2+\left|\frac{C_k}{C_1}\right|^2+\left|\frac{C_l}{C_1}\right|^2+2\mathrm{Re}\left(1\cdot \frac{C^*_k}{C^*_1}\right)+2\mathrm{Re}\left(\frac{C_k}{C_1}\cdot \frac{C_l}{C_1}\right)+2\mathrm{Re}\left(\frac{C_l}{C_1}\cdot 1\right) \\
\therefore \quad &\mathrm{Re}\left(\frac{C_k}{C_1}\cdot \frac{C_l^*}{C_1^*}\right)=\mathrm{Re}\left(\frac{C_k}{C_1}\cdot \frac{C_l}{C_1}\right)
\end{align*}
ここで,複素数$\alpha=a+ib,\beta=c+id$について
\begin{align*}
\mathrm{Re}(\alpha \beta^*)=&\mathrm{Re}(\alpha \beta) \\
\mathrm{Re}((a+ib)(c-id))=&\mathrm{Re}((a+ib)(c+id)) \\
ac+bd=&ac-bd \\
bd=&0 \\
\mathrm{Im}(\alpha)\mathrm{Im}(\beta)=&0
\end{align*}
と同値変形できることから,$\alpha=C_k/C_1,\beta=C_l/C_1$とおけば上の関係式は
\begin{align*}
&\mathrm{Im}\left(\frac{C_k}{C_1}\right)\mathrm{Im}\left(\frac{C_l}{C_1}\right)=0 \\
\therefore \quad & \mathrm{Im}\left(\frac{C_k}{C_1}\right)=0 ,\quad \mathrm{or} \quad \mathrm{Im}\left(\frac{C_k}{C_1}\right)
\end{align*}
を与える.したがって$C_k/C_1$か$C_l/C_1$のどちらかは必ず実となり,これは仮定に反する.したがって仮定が偽であり,どのような対$(k,l)$を任意に取ろうとも$C_k/C_1=C_k'/C_1'$かつ$C_l/C_1=(C_l'/C_1')^*$であることは不可能,すなわち全ての$k\neq 1$について$C_k/C_1=C_k'/C_1'$か,そうでなければ$C_k/C_1=(C_k'/C_1')^*$を満たす.これが示したいことだった.\par
よって与えられた対称性変換$T$が,状態ベクトル$\sum_k C_k \Psi_k$に施されたとき,全ての$k$について(2.A.12)か(2.A.13)である.

\vskip\baselineskip


ここまでくれば,任意の射線$\mc{R}$の任意の$\Psi$に対して$T\mc{R}$のどの$\Psi'$を選ぶかを定め,対応$\Psi\to U\Psi$をただ一つに定めることができ,写像$U$を決めることができる.(2.A.7)より,まず(2.A.12)がなりたっている場合
\begin{align*}
\Psi'=&\sum_k C'_k U\Psi_k \\
=&\frac{C'_1}{C_1} \sum_k C_k U\Psi_k
\end{align*}
と書ける.この場合,$U\Psi$は$C_1=C'_1$となるように位相が選ばれた$\Psi'$だとする.すなわち$\Psi$に対して
\begin{align*}
U\Psi=\sum_k C_k U\Psi_k
\end{align*}
が対応する.一方,(2.A.13)がなりたっている場合
\begin{align*}
\Psi'=&\sum_k C'_k U\Psi_k \\
=&\frac{C'_1}{C^*_1} \sum_k C^*_k U\Psi_k
\end{align*}
と書ける.この場合,$U\Psi$は$C_1=C'^*_1$となるように位相が選ばれた$\Psi'$だとする.すなわち$\Psi$に対して
\begin{align*}
U\Psi=\sum_k C^*_k U\Psi_k
\end{align*}
が対応する.($|C_1|=|C_1'|$はすでに(2.A.8)で示せるが,全体の位相がどうなるように決めるかの任意性があったのだった.そこで決められた$\Psi$に対して$\Psi'$の位相を,$\Psi_1$の係数$C_1$と$\Psi'_1$の係数$C_1'$(あるいは$C'^*_1$)の位相が等しくなるように定める.これで$\Psi$に対して$T\mc{R}$のどの元を対応させればいいのかがただ一つに定まったことになる.くどい説明で申し訳ない.)このように定めると前者は
\begin{align*}
U\Psi=&U\left(\sum_k C_k \Psi_k\right) \\
=\Psi'=&\sum_k C_k U\Psi_k \\
\therefore \quad & U\left(\sum_k C_k \Psi_k\right)=\sum_k C_k U\Psi_k
\end{align*}
を与え,後者は
\begin{align*}
U\Psi=&U\left(\sum_k C_k \Psi_k\right) \\
=\Psi'=&\sum_k C_k^* U\Psi_k \\
\therefore \quad & U\left(\sum_k C_k \Psi_k\right)=\sum_k C_k^* U\Psi_k
\end{align*}
を与える.


\vskip\baselineskip


一つのベクトルを$\Psi_k$で展開したとき,対称性変換の前後でその係数には共通の関係があることがわかった.しかし,同じ対称性変換でも,二つの異なるベクトル$\Psi,\Psi'$を持ってきたら,$\Psi=\sum_k C_k \Psi_k$の方は(2.A.12)の方を満たすように変換されるが別の$\Psi'=\sum_k D_k \Psi_k$の方は(2.A.13)を満たされるように変換されるかもしれない.同じ対称性変換なのに変換性がベクトルによって違うなんてことが起きてしまうと困る.そのようなことがないことを証明しよう.

\vskip\baselineskip


ある変換$T$に対し,ある状態が(2.A.12)
\begin{align*}
U\left(\sum_k A_k \Psi_k \right)=\sum_k A_k U\Psi_k
\end{align*}
と変換されるが,別の状態では(2.A.13)
\begin{align*}
U\left(\sum_k B_k \Psi_k \right)=\sum_k B^*_k U\Psi_k
\end{align*}
と変換されると仮定する.仮定(2.A.1)より
\begin{align*}
\left|\left(\sum_k B_k \Psi_k,\sum_l A_l \Psi_l \right)\right|^2=&\left|\left( \sum_k B^*_kU\Psi_k, \sum_l A_k U\Psi_k \right)\right|^2 \\
\therefore \quad \left|\sum_k B^*_k A_k\right|^2 =&\left|\sum_k B_k A_k \right|^2
\end{align*}
となる.
\begin{align*}
(\mathrm{LHS})=&\left(\sum_k B^*_k A_k \right)\left(\sum_l B_l A^*_l \right) \\
=&\sum_{kl}B^*_k B_l A_k A^*_l \\
=&\frac{1}{2}\sum_{kl}\left(B^*_k B_l A_k A^*_l+B_k B^*_l A^*_k A_l\right) \\
=&\sum_{kl}\mathrm{Re}(B^*_k B_l A_k A^*_l) \\
(\mathrm{RHS})=&\left(\sum_k B_k A_k \right)\left(\sum_l B_l^* A^*_l \right) \\
=&\sum_{kl}\mathrm{Re}(B_k B^*_l A_k A^*_l) 
\end{align*}
$B_k^*B_l=(B_k B^*_l)^*$を用いてさらに変形して
\begin{align*}
(\mathrm{LHS})=&\sum_{kl} \mathrm{Re}\Bigl\{\left[\mathrm{Re}(B_k B^*_l)-i\mathrm{Im}(B_k B^*_l)\right]\left[\mathrm{Re}(A_k A^*_k)+i \mathrm{Im}(A_k A^*_l)\right]\Bigr\} \\
=&\sum_{kl} \Bigl[\mathrm{Re}(B_k B^*_l)\mathrm{Re}(A_k A^*_k)+\mathrm{Im}(B_k B^*_l)\mathrm{Im}(A_k A^*_l)\Bigr]\\
(\mathrm{RHS})=&\sum_{kl} \mathrm{Re}\Bigl\{\left[\mathrm{Re}(B_k B^*_l)+i\mathrm{Im}(B_k B^*_l)\right]\left[\mathrm{Re}(A_k A^*_k)+i \mathrm{Im}(A_k A^*_l)\right]\Bigr\} \\
=&\sum_{kl} \Bigl[\mathrm{Re}(B_k B^*_l)\mathrm{Re}(A_k A^*_k)-\mathrm{Im}(B_k B^*_l)\mathrm{Im}(A_k A^*_l)\Bigr]\\
\therefore \quad & \sum_{kl} \mathrm{Im}(B_k B^*_l)\mathrm{Im}(A_k A^*_l)=0
\end{align*}
となる.この等式が,異なる射線に属する二つの状態ベクトル$\sum_k A_k \Psi_k,\sum_kB_k \Psi_k$について満たされている可能性はある.しかし,$A_k$も$B_k$も\uwave{全て同じ位相ではない}状態ベクトルの任意の対について,常に第三の状態ベクトル$\sum_k C_k\Psi_k$があり,以下が成立するようにできる.
\begin{align*}
\sum_{kl}\mathrm{Im}(C_k^* C_l)\mathrm{Im}(A^*_k A_l)\neq 0
\end{align*}
かつ
\begin{align*}
\sum_{kl}\mathrm{Im}(C^*_k C_l)\mathrm{Im}(B^*_k B_l) \neq 0
\end{align*}
(どちらか(あるいはどちらも)のベクトルの全ての位相が同じな場合,例えば前者のベクトルについて$A_k=|A_k|e^{i\theta},A_l=|A_l|e^{i\theta}$となっているならば$\mathrm{Im}(A^*_k A_l)=\mathrm{Im}(|A_k||A_l|)=0$となり,全ての対$(k,l)$についてそれが成り立っているから上の式は自動的に満たされる.しかしこの場合は心配ない.先程の議論と同様,再び全体の位相を再定義して全ての係数が実となるようにできて,その時は前述の通り$U$が線形か反線形かを区別する必要性がそもそもなくなる.)

\vskip\baselineskip

これを示そう.\par
(1)ある対$(k,l)$で,$A^*_kA_l$と$B_k^*B_l$が共に複素数なら,
\begin{align*}
\mathrm{Im}(A^*_k A_l )\neq 0,\quad \mathrm{Im}(B_k^*B_l)\neq 0
\end{align*}
だから,$C_k,C_l$を除くすべての$C_i$を全てゼロにして,$\mathrm{Im}(C^*_k C_l)\neq 0$となるように,つまり$C_k$と$C_l$の位相が異なるように選べばよい.\par
(2)ある対$(k,l)$で,$A^*_k A_l$が複素数で$B_k^*B_l$が実数なら,
\begin{align*}
\mathrm{Im}(A^*_kA_l)\neq 0, \quad \mathrm{Im}(B_k^*B_l)=0
\end{align*}
なので,$B_k$と$B_l$は位相が同じだ.前提より$\{B_k\}_{k \in \mathbb{N}}$は必ずどれかの位相が違うので,別の位相を持つ対$B_m,B_n$をもう一度選べば$B^*_mB_n$が複素数となり$\mathrm{Im}(B^*_mB_n)\neq 0$となる.($B_k$と位相が違う1つを持ってくればいいので,$B_m,B_n$のどちらかは$B_k$か$B_l$としてよい.ただし当然,両方とも等しくはしない.)このように別の対を選んだとき,$\{A_k\}_{k\in \mathbb{N}}$側も添え字に対応して別の対$A_m,A_n$が選ばれるが,新しい対$A^*_m A_n$が複素数か実数かで$C_k$の選び方が次の二通りに分かれる.\par
(2)(i)このように別の対を持ってきたとき,もし$A^*_m A_n$も複素数なら,
\begin{align*}
\mathrm{Im}(A^*_m A_n )\neq 0, \quad \mathrm{Im}(B_m^*B_n)\neq 0
\end{align*}
となり,$C_m,C_n$以外の$C_i$をゼロとして$\mathrm{Im}(C_m^*C_n)\neq 0$となるように,つまり$C_m$と$C_n$の位相が異なるように選べばよい.\par
(2)(ii)$A_m^*A_n$が実数なら,
\begin{align*}
&\mathrm{Im}(A^*_kA_l)\neq 0, \quad \mathrm{Im}(B_k^*B_l)=0 \\
&\mathrm{Im}(A^*_m A_n )= 0, \quad \mathrm{Im}(B_m^*B_n)\neq 0
\end{align*}
なので,$C_k,C_l,C_m,C_n$を除いたすべての$C_i$がゼロとなるようにして,これらの係数が全て異なるように,つまり$\mathrm{Im}(C^*_k C_l)\neq 0$かつ$\mathrm{Im}(C^*_m C_n)\neq 0$になるように選ぶ.これで全体の和はちゃんと(2.A.17)(2.A.18)を満たすようにできる.\par
(3)ある対$(k,l)$で,今度は$A^*_kA_l$が実数で$B^*_kB_l$が複素数となる場合.この場合も(2)と同様の手順で$\{C_k\}_{k\in \mathbb{N}}$をうまく選べる.\par
これで証明ができた.


\vskip\baselineskip



(2.A.17)から,このようにして選んだ$\sum_k C_k \Psi_k$は対称性変換$T$のもとで$\sum_k A_k \Psi_k$と共通する(2.A.14)を満たさなければならない.そうでなければ$\sum_k C_k \Psi_k$は(2.A.15)を満たすはずだが,そのとき(2.A.17)の右辺は(2.A.16)を示したのと同様にしてゼロとなってしまい矛盾する.しかし同様に,(2.A.18)から,$\sum_k C_k \Psi_k$は対称性変換$T$のもとで$\sum_k B_k \Psi_k$と共通する(2.A.15)を満たさなければならない.これは矛盾だ.したがって仮定が偽となり,異なるベクトル$\sum_k A_k \Psi_k,\sum_k B_k \Psi_k$の変換則は(2.A.14)か(2.A.15)を共通して選ばれていなければならない.したがって,与えられた対称性$T$について全ての状態ベクトルは,(2.A.14)を満たすか,もしくは全ての状態ベクトルは(2.A.15)を満たす.

\vskip\baselineskip

ここまでくれば,定義に従って,量子力学的演算子$U$が線形ユニタリーか反線形反ユニタリーであることは簡単に示せる.(2.A.14)が全ての状態ベクトル$\sum_k C_k \Psi_k$について成立するとする.$\Psi_k$は完全系をなすから,どのような二つの状態ベクトル$\Psi$と$\Phi$も
\begin{align*}
\Psi=\sum_k A_k \Psi_k ,\quad \Phi=\sum_k B_k \Psi_k
\end{align*}
と一意的に展開でき,したがって(2.A.14)より
\begin{align*}
U(\alpha \Psi+\beta \Phi)=& U\left(\sum_k (\alpha A_k +\beta B_k )\Psi_k\right) \\
=&\sum_k (\alpha A_k+  \beta B_k)U \Psi_k \quad \because (2.A.14),C_k=\alpha A_k +\beta B_k とおく \\
=&\alpha \sum_k A_k U\Psi_k +\beta \sum_k B_k U\Psi_k \\
=&\alpha U \left(\sum_k A_k \Psi_k\right) +\beta \left( U \sum_k B_k \Psi\right) \quad \because (2.A.14) \\
=&\alpha U\Psi +\beta U\Phi
\end{align*}
となり,$U$は\uwave{線形}演算子となる.また(2.A.2)(2.A.3)より
\begin{align*}
(U\Psi,U\Phi)=&\sum_{kl}A^* B_l (U\Psi_k ,U\Psi_l) \quad \because (2.A.14)\\
=&\sum_{kl} A^*_k B_l \delta_{kl} \quad \because (2.A.3) \\
=&\sum_k A^*_k B_k
\end{align*}
一方
\begin{align*}
(\Psi,\Phi)=&\sum_{kl} A^*_k B_l(\Psi_k,\Psi_l) \\
=&\sum_{kl} A^*_k B_l \delta_{kl} \quad \because (2.A.2) \\
=&\sum_k A^*_k B_k
\end{align*}
より
\begin{align*}
(U\Psi ,U\Phi)=(\Psi,\Phi)
\end{align*}
であり,$U$は\uwave{ユニタリー}演算子だ.\par
一方,今度は(2.A.15)が全ての状態ベクトル$\sum_k C_k \Psi_k$について成立するとする.同様に
\begin{align*}
U(\alpha \Psi+\beta \Phi)=& U\left(\sum_k (\alpha A_k +\beta B_k )\Psi_k\right) \\
=&\sum_k (\alpha A_k+  \beta B_k)^*U \Psi_k \quad \because (2.A.15),C_k=\alpha A_k +\beta B_k とおく \\
=&\alpha^* \sum_k A_k^* U\Psi_k +\beta^* \sum_k B_k^* U\Psi_k \\
=&\alpha^* U \left(\sum_k A_k \Psi_k\right) +\beta^* \left( U \sum_k B_k \Psi\right) \quad \because (2.A.15) \\
=&\alpha^* U\Psi +\beta^* U\Phi
\end{align*}
よって$U$は\uwave{反線形}演算子となる.\par
また(2.A.2)(2.A.3)より同様に
\begin{align*}
(U\Psi,U\Phi)=&\sum_{kl}A_k B_l^* (U\Psi_k ,U\Psi_l) \quad \because (2.A.15)\\
=&\sum_{kl} A_k B^*_l \delta_{kl} \quad \because (2.A.3) \\
=&\sum_k A_k B^*_k \\
(\Psi,\Phi)^*=&\left(\sum_k A^*_k B_k\right)^*=\sum_k A_k B^*_k
\end{align*}
となって
\begin{align*}
(U\Psi,U\Psi)=(\Psi,\Phi)^*
\end{align*}
であり$U$は\uwave{反ユニタリー}演算子だ.


\newpage


\subsection*{補遺B:群の演算子とホモトピー類}
この補遺では,2.7節で述べた定理を証明する.有限対称性変換$T$の演算子$U(T)$は,\par
(a)リー代数に中心電荷が現れないように群の生成子の再定義が可能\par
(b)群が単連結\\
が満たされるとき,その位相を選ぶことにより群の射影表現ではなく通常の表現をなすようにできる.

\vskip\baselineskip

変換パラメータとして,実変数$\theta^a$の集合$M$を
\begin{align*}
T(\bar{\theta})T(\theta)=T\Bigl(f(\bar{\theta},\theta)\Bigr)
\end{align*}
が成立するように導入する.さて,ヒルベルト空間の演算子$U(T(\theta))=:U[\theta] \in B(\mc{H})$が,対応して
\begin{align*}
U[\bar{\theta}]U[\theta]=U\Bigl[f(\bar{\theta},\theta)\Bigr]
\end{align*}
を満たすように構成したい.(ここで角括弧$[\cdots]$は,群のパラメータの関数として構成した$U$と,群の変換そのものの関数として表したものを区別するために用いている.$U[\theta]$は後者だ.)これを行うために,群のパラメータ空間に,原点からそれぞれの点$\theta \in M$への任意の基準経路$\Theta^a_\theta(s)$を
\begin{align*}
\Theta^a_\theta :I=[0,1] \to M , \quad s \mapsto \Theta^a_\theta(s)
\end{align*}
で定義する.このとき境界条件は$\Theta^a_\theta(0)=0$と$\Theta^a_\theta(1)=\theta^a$とし,それぞれの経路に沿って関数
\begin{align*}
U_\theta:I\to B(\mc{H}),s\mapsto U_\theta(s)
\end{align*}
を以下の微分方程式で定義する.
\begin{align*}
\frac{d}{ds}U_\theta(s)=it_\alpha U_\theta(s) \tensor{h}{^a_b}(\Theta_\theta(s))\frac{d \Theta_\theta^b(s)}{ds}
\end{align*}
初期条件は
\begin{align*}
U_\theta(0)=1
\end{align*}
であり,ここで行列$h$を
\begin{align*}
\tensor{\left[h^{-1}\right]}{^a_b}(\theta):=\left[\frac{\partial f^a(\bar{\theta},\theta)}{\partial \bar{\theta}^b}\right]_{\bar{\theta}=0}=\frac{\partial f^a}{\partial \bar{\theta}^b}(0,\theta)
\end{align*}
と定めた.最終的には演算子$U[\theta]$を$U_\theta(1)$と同定する.

\vskip\baselineskip

とはいえ,急に微分方程式をお出しされても意味がわからないと思うから,簡易的なモチベーションを説明しておく.(2.B.1)から,$\bar{\theta}$を微小な$\delta \theta$とおき,$U[\theta]$の微小な変化を見てやる.
\begin{align*}
U[\delta \theta]U[\theta]=&U[f(\delta \theta,\theta)] 
\end{align*}
パラメータ$\delta \theta^a$が微小なとき,演算子$U[\theta]$は単位元周りで生成子$t_a$を用いて(2.2.17)と書けるから
\begin{align*}
(1+i\delta \theta^a t_a)U[\theta]=&U\left[f^a(0,\theta)+\left.\frac{\partial f^a(\bar{\theta},\theta)}{\partial \bar{\theta}^b}\right|_{\bar{\theta}=0} \delta \theta^b\right] \\
=&U\Bigl[\theta^a+\tensor{[h^{-1}]}{^a_b}(\theta) \delta \theta^b\Bigr] \quad \because (2.2.16)(2.B.4)
\end{align*}
とテイラー展開できるはずだ.$\theta$はパラメータ空間$M$の点であり,$\theta$から$\delta \theta$だけ移動させ$f^a(\delta \theta,\theta)$にずらしたとき,$U[\theta]$もそれに伴って$U[f(\delta\theta,\theta)]$へとずれる.そのずらす経路に沿って,経路の各点でこの関係式が成り立っていると解釈できる.つまり,これを拡張して経路$\Theta^a_\theta(s)$に沿った方程式にするには$\theta \to \Theta_\theta(s)$と単に置き換えて
\begin{align*}
(1+i\delta \theta^a t_a)U[\Theta_\theta(s)]=U\Bigl[\Theta^a_\theta(s)+\tensor{[h^{-1}]}{^a_b}(\Theta_\theta(s)) \delta \theta^b\Bigr]
\end{align*}
と書けばいいと気付く.さて,右辺の引数の$\Theta^a_\theta(s)+\tensor{[h^{-1}]}{^a_b}(\Theta_\theta(s)) \delta \theta^b$は$\Theta^a_\theta(s+\delta s)$と解釈できる.なぜなら,これは最初の式より$f(\delta\theta,\theta)$に対応し,$\Theta_\theta(s)$から($s$でパラメータ化された)経路に沿って微小に変化させた$M$の点を表しており,それは経路パラメータ$s$を微小に$\delta s$だけ変化させた$s+\delta s$における点$\Theta^a_\theta(s+\delta s)$に違いないからだ.これを少し変形してやると
\begin{align*}
\Theta^a_\theta(s+\delta s):=&\Theta^a_\theta(s)+\tensor{[h^{-1}]}{^a_b}(\Theta_\theta(s)) \delta \theta^b \\
\therefore \quad \delta \theta^a =&\tensor{h}{^a_b}(\Theta_\theta(s))\Bigl(\Theta^a_\theta(s+\delta s) -\Theta^a_\theta(s)\Bigr)\\
=&\tensor{h}{^a_b}(\Theta_\theta(s))\delta \Theta^a_\theta(s)
\end{align*}
したがって元の式は
\begin{align*}
(1+i\delta \theta^a t_a)U[\Theta_\theta(s)]=&U\Bigl[\Theta^a_\theta(s)+\tensor{[h^{-1}]}{^a_b}(\Theta_\theta(s)) \delta \theta^b\Bigr] \\
=&U[\Theta_\theta(s+\delta s)] \\
U[\Theta_\theta(s+\delta s)]-U[\Theta_\theta(s)]=&i\delta \theta^a t_a U[\Theta_\theta(s)] \\
=&it_a U_\theta[\Theta_\theta(s)] \tensor{h}{^a_b}(\Theta_\theta(s))\delta \Theta^b_\theta(s)
\end{align*}
両辺を$\delta s$で割ってやると,微分の定義から
\begin{align*}
\frac{d}{ds}U[\Theta_\theta(s)]=it_a U[\Theta_\theta(s)]\tensor{h}{^a_b}(\Theta_\theta(s))\frac{d \Theta^b_\theta(s)}{ds}
\end{align*}
と書ける.$U_\theta(s):=U[\Theta_\theta(s)]$とおけば,(2.B.2)の微分方程式
\begin{align*}
\frac{d}{ds}U_\theta(s)=it_\alpha U_\theta(s) \tensor{h}{^a_b}(\Theta_\theta(s))\frac{d \Theta_\theta^b(s)}{ds}
\end{align*}
が得られる.この説明により,最終的に$U_\theta(1)=U[\theta]$となぜ同定したいかの見当が若干つくだろう.$U_\theta(s)$の定義より$U_\theta(1)=U[\Theta_\theta(1)]=U[\theta]$だからだ.\par



\vskip\baselineskip


まず合成則を調べるために,2点$\theta_1,\theta_2$を考え,まず$0$から$\theta_1$へ行き,その後$f(\theta_2,\theta_1)$へ進む経路$\mc{P}$を定義する.
\begin{align*}
\Theta^a_\mc{P}(s) \equiv \left\{ 
\begin{array}{ll}
\Theta^a_{\theta_1}(2s)  & \left(0\leq s \leq \frac{1}{2}\right) \\
f^a(\Theta_{\theta_2}(2s-1),\theta_1) & \left(\frac{1}{2}\leq s \leq 1\right)
\end{array}
 \right.
\end{align*}
実際,これがそのような経路になっていることを確認する.始点$s=0$は
\begin{align*}
\Theta^a_\mc{P}(0) =\Theta^a_{\theta_1}(0)=0
\end{align*}
だ.中間の$s=1/2$では
\begin{align*}
\lim_{s\to \frac{1}{2}-0}\Theta^a_\mc{P}(s)=&\Theta^a_{\theta_1}(1)=\theta_1 \\
\lim_{s\to \frac{1}{2}+0}\Theta^a_\mc{P}(s)=&f^a(\Theta_{\theta_2}(0),\theta_1)=f^a(0,\theta_1)=\theta_1 \\
\therefore \quad \Theta^a_\mc{P}(1/2)=&\theta_1
\end{align*}
となって中間が$\theta_1$で連続だ.終点の$s=1$では
\begin{align*}
\Theta^a_\mc{P}(1)=f^a(\Theta_{\theta_2}(1),\theta_1)=f^a(\theta_2,\theta_1)
\end{align*}
となっている.正しく経路になっていることが確認できる.\par
$U_\mc{P}(s)$は(2.B.2)により微分方程式
\begin{align*}
\frac{d}{ds}U_{\mc{P}}(s)=it_a U_\mc{P}(s)\tensor{h}{^a_b}(\Theta_\mc{P}(s))\frac{d\Theta^b_\mc{P}(s)}{ds}
\end{align*}
で定義される.第一区分$0\leq s \leq \frac{1}{2}$では
\begin{align*}
\frac{d}{ds}U_{\mc{P}}(s)=it_a U_\mc{P}(s)\tensor{h}{^a_b}(\Theta_{\theta_1}(2s))\frac{d\Theta^b_{\theta_1}(2s)}{ds}
\end{align*}
である.一方,$U_{\theta_1}(2s)$の微分方程式を(2.B.2)に従って立てると
\begin{align*}
\frac{d}{ds}U_{\theta_1}(2s)=it_a U_{\theta_1}(2s)\tensor{h}{^a_b}(\Theta_{\theta_1}(2s))\frac{d\Theta^b_{\theta_1}(2s)}{ds}
\end{align*}
となり,両者は同じ微分方程式を満たすことがわかる.初期条件も,ともに$U_{\mc{P}}(0)=1=U_{\theta_1}(0)$となり一致するから,常微分方程式の解の一意性により$0\leq s \leq \frac{1}{2}$では$U_{\mc{P}}(s)=U_{\theta_1}(2s)$となる.よって最初の区分の終わり$s=\frac{1}{2}$では$U_{\mc{P}}(\frac{1}{2})=U_{\theta_1}(1)$となる.\par
$U_\mc{P}(s)$の第二区分$(\frac{1}{2}\leq s \leq 1)$でのふるまいを求めるためには,途中で$f^a(\Theta_{\theta_2}(2s-1),\theta_1)$の微分が必要となるから先に計算しておく.このために,基本結合則
\begin{align*}
T(\theta_3)T(\theta_2)T(\theta_1)=&T\Bigl(f(\theta_3,\theta_2)\Bigr)T(\theta_1)=T\Bigl(f(f(\theta_3,\theta_2),\theta_1)\Bigr) \\
=&T(\theta_3)T\Bigl(f(\theta_2,\theta_1)\Bigr)=T\Bigl(f(\theta_3,f(\theta_2,\theta_1))\Bigr) \\
\therefore \quad f^a(f(\theta_3,\theta_2),\theta_1)=&f^a(\theta_3,f(\theta_2,\theta_1))
\end{align*}
を使う.$\theta_3\to 0$の極限で
\begin{align*}
f^a(f(\theta_3,\theta_2),\theta_1)=& f^a\left(f^b(0,,\theta_2)+\left. \frac{\partial f^b(\bar{\theta},\theta_2)}{\partial \bar{\theta}^c}\right|_{\bar{\theta}=0}\theta^c_3 \, ,\, \theta_1\right) \\
=&f^a\Bigl(\theta^b_2+\tensor{[h^{-1}]}{^b_c}(\theta_2) \theta^c_3 \, ,\,\theta_1\Bigr) \\
=&f^a(\theta_2,\theta_1)+\frac{\partial f^a(\theta_2,\theta_1)}{\partial \theta_2^b}\tensor{[h^{-1}]}{^b_c}(\theta_2) \theta^c_3 \\
=f^a(\theta_3,f(\theta_2,\theta_1))=&f^a(0,f(\theta_2,\theta_1))+\left. \frac{\partial f^a(\bar{\theta},f(\theta_2,\theta_1))}{\partial \bar{\theta}^b}\right|_{\bar{\theta}=0}\theta_3^b \\
=&f^a(\theta_2,\theta_1)+\tensor{[h^{-1}]}{^a_b}(f(\theta_2,\theta_1))\theta_3^b
\end{align*}
$\theta_3^c$の係数を比較して,
\begin{align*}
\frac{\partial f^a(\theta_2,\theta_1)}{\partial \theta_2^b}\tensor{[h^{-1}]}{^b_c}(\theta_2)=&\tensor{[h^{-1}]}{^a_c}(f(\theta_2,\theta_1)) \\
\therefore \quad \tensor{h}{^c_a}(f(\theta_2,\theta_1)) \frac{\partial f^a(\theta_2,\theta_1)}{\partial \theta_2^b} =&\tensor{h}{^c_b}(\theta_2) \\
\tensor{h}{^c_a}(f(\theta_2,\theta_1)) \frac{\partial f^a}{\partial \theta_2^b}(\theta_2,\theta_1) =&\tensor{h}{^c_b}(\theta_2)
\end{align*}
(最後に別の表記をしたのは,こっちの方が次の計算の変形がわかりやすくなると思ったからだ.)さて,$\frac{1}{2}\leq s \leq 1$では$U_\mc{P}(s)$の微分方程式は(2.B.2)より
\begin{align*}
\frac{d}{ds}U_\mc{P}(s)=&it_a U_{\mc{P}}(s) \tensor{h}{^a_b}\Bigl(f(\Theta_{\theta_2}(2s-1),\theta_1)\Bigr)\frac{d f^b(\Theta_{\theta_2}(2s-1),\theta_1)}{ds} \\
=&it_a U_{\mc{P}}(s) \tensor{h}{^a_b}\Bigl(f(\Theta_{\theta_2}(2s-1),\theta_1)\Bigr)\left.\frac{\partial f^b(\theta_2,\theta_1)}{\partial \theta_2^c}\right|_{\theta_2=\Theta_{\theta_2}(2s-1)}\frac{d \Theta_{\theta_2}(2s-1)}{ds} \\
=&it_a U_{\mc{P}}(s) \uwave{\tensor{h}{^a_b}\Bigl(f(\Theta_{\theta_2}(2s-1),\theta_1)\Bigr) \frac{\partial f^b}{\partial \theta_2^c}(\Theta_{\theta_2}(2s-1),\theta_1)}\frac{d \Theta_{\theta_2}(2s-1)}{ds} \\
=&it_a U_{\mc{P}}(s) \tensor{h}{^a_c}(\Theta_{\theta_2}(2s-1))\frac{d \Theta_{\theta_2}(2s-1)}{ds}
\end{align*}
となる.一方,$U_{\theta_2}(2s-1)$の微分方程式は(2.B.2)より
\begin{align*}
\frac{d}{ds}U_{\theta_2}(2s-1)=it_a U_{\theta_2}(2s-1) \tensor{h}{^a_c}(\Theta_{\theta_2}(2s-1))\frac{d \Theta_{\theta_2}(2s-1)}{ds}
\end{align*}
となり,両者は同じ微分方程式を満たすことが分かる.しかし初期条件が違い,$U_{\theta_2}(2s-1)$は$s=\frac{1}{2}$で$U_{\theta_2}(0)=1$だが,$U_{\mc{P}}(s)$は先程示した通り$s=\frac{1}{2}$で$U_{\mc{P}}(\frac{1}{2})=U_{\theta_1}(1) \neq 1$となる.そこで$U_{\mc{P}}(s)$ではなく$U_{\mc{P}}(s)U^{-1}_{\theta_1}(1)$を代わりに考えると,$U^{-1}_{\theta_1}(1)$は$s$に対して定数だからこれも同じ微分方程式を満たし,しかも$s=\frac{1}{2}$で$U_{\mc{P}}(\frac{1}{2})U^{-1}_{\theta_1}(1)=1$となり初期条件も一致する.したがって常微分方程式の解の一意性より$\frac{1}{2}\leq s \leq 1$では
\begin{align*}
U_{\mc{P}}(s)U^{-1}_{\theta_1}(s)=U_{\theta_2}(2s-1)
\end{align*}
となる.特に
\begin{align*}
U_{\mc{P}}(1)=U_{\theta_2}(1) U_{\theta_1}(1)
\end{align*}
となる.\par
一見,$\theta_1$での$U_{\theta_1}(1)$と$\theta_2$での$U_{\theta_2}(1)$が合成された結果,$\theta_1$と$\theta_2$の合成$f(\theta_2,\theta_1)=\Theta_{\mc{P}}(1)$での$U_{\mc{P}}(1)$になったように見える.しかし,これは$U_{\theta}(1)$が複合則(2.B.1)を満たすことは意味しない.なぜなら,たしかに経路$\Theta_{\mc{P}}(1)$は$\theta^a=0$から$\theta^a=f^a(\theta_2,\theta_1)$へのものだが,前に選んだ基準経路$\Theta_{f(\theta_2,\theta_1)}(s)$とは一般には違うものだからだ.$U_\theta(1)$は経路の取り方で変化するかもしれない.\par
そこで,$U_\theta(1)$が$0$と$\theta$を結ぶ経路の取り方に依らないことを示す必要がある.

\vskip\baselineskip


0から$\theta$への基準経路$\Theta_{\theta}(s)$に微小変化$\delta \Theta(s)$を加え,そのもとでの$U_\theta(s)$への変化$\delta U$を調べよう.(2.B.2)の変分をとると,変分は微分と同じように振舞うから
\begin{align*}
\frac{d}{ds}\delta U=&it_a \delta U \tensor{h}{^a_b}(\Theta)\frac{d \Theta^b }{ds}+it_a U \delta(\tensor{h}{^a_b}(\Theta))\frac{d\Theta^b}{ds}+it_a U\tensor{h}{^a_b}(\Theta)\frac{d \delta \Theta^b}{ds} \\
=&it_a \delta U \tensor{h}{^a_b}(\Theta)\frac{d \Theta^b }{ds}+it_a U \frac{ \partial \tensor{h}{^a_b}(\Theta)}{\partial \Theta^c}\delta \Theta^c\frac{d\Theta^b}{ds}+it_a U\tensor{h}{^a_b}(\Theta)\frac{d \delta \Theta^b}{ds} \\
=&it_a \delta U \tensor{h}{^a_b}(\Theta)\frac{d \Theta^b }{ds}+it_a U \tensor{h}{^a_{b,c}}\delta \Theta^c\frac{d\Theta^b}{ds}+it_a U\tensor{h}{^a_b}(\Theta)\frac{d \delta \Theta^b}{ds} \\
\end{align*}
ここで
\begin{align*}
\tensor{h}{^a_{b,c}} :=\frac{ \partial \tensor{h}{^a_b}(\Theta)}{\partial \Theta^c}
\end{align*}
と定義した.
\begin{align*}
0=\frac{d}{ds}1=\frac{d}{ds}(U^{-1}U) =&\left(\frac{d}{ds}U^{-1}\right) U+U^{-1}\left(\frac{d}{ds}U\right) \\
\therefore \quad \left(\frac{d}{ds}U^{-1}\right)=&-U^{-1}\left(\frac{d}{ds}U\right)U^{-1}
\end{align*}
であることを使うと
\begin{align*}
\frac{d}{ds}(U^{-1}\delta U)=&\left(\frac{d}{ds}U^{-1}\right) \delta U+U^{-1} \left(\frac{d}{ds}\delta U\right) \\
=&-U^{-1}\left(\frac{d}{ds}U\right)U^{-1}\delta U+U^{-1} \left(\frac{d}{ds}\delta U\right) \\
=&-iU^{-1}t_a \delta U \tensor{h}{^a_b}(\Theta )\frac{d \Theta^b}{ds} \\
&+iU^{-1}t_a \delta U \tensor{h}{^a_b}(\Theta)\frac{d \Theta^b }{ds}+iU^{-1}t_a U \tensor{h}{^a_{b,c}}\delta \Theta^c\frac{d\Theta^b}{ds}+iU^{-1}t_a U\tensor{h}{^a_b}(\Theta)\frac{d \delta \Theta^b}{ds} \\
=&iU^{-1}t_a U \tensor{h}{^a_{b,c}}\delta \Theta^c\frac{d\Theta^b}{ds}+iU^{-1}t_a U\tensor{h}{^a_b}(\Theta)\frac{d \delta \Theta^b}{ds} \\
=&iU^{-1}t_a U \tensor{h}{^a_{b,c}}\delta \Theta^c\frac{d\Theta^b}{ds}+iU^{-1}t_a U\tensor{h}{^a_b}(\Theta)\frac{d \delta \Theta^b}{ds} \\
&-iU^{-1}t_a U \tensor{h}{^a_{c,b}}\delta \Theta^c\frac{d\Theta^b}{ds}+iU^{-1}t_a U \frac{\partial \tensor{h}{^a_c}(\Theta)}{\partial \Theta^b}\delta \Theta^c\frac{d\Theta^b}{ds} \quad (\leftarrow 合わせて\pm 0)\\
=&iU^{-1}t_a U \delta \Theta^c\frac{d\Theta^b}{ds}(\tensor{h}{^a_{b,c}}-\tensor{h}{^a_{c,b}}) \\
&+iU^{-1}t_a U\tensor{h}{^a_b}(\Theta)\frac{d \delta \Theta^b}{ds}+iU^{-1}t_a U \frac{\partial \tensor{h}{^a_c}(\Theta)}{\partial \Theta^b}\frac{d\Theta^b}{ds}\delta \Theta^c \\
=&iU^{-1}t_a U \delta \Theta^c\frac{d\Theta^b}{ds}(\tensor{h}{^a_{b,c}}-\tensor{h}{^a_{c,b}}) \\
&+iU^{-1}t_a U\tensor{h}{^a_b}(\Theta)\frac{d \delta \Theta^b}{ds}+iU^{-1}t_a U \frac{\partial \tensor{h}{^a_c}(\Theta)}{ds}\delta \Theta^c \\
=&iU^{-1}t_a U \delta \Theta^c\frac{d\Theta^b}{ds}(\tensor{h}{^a_{b,c}} -\tensor{h}{^a_{c,b}}) +iU^{-1}t_a U \frac{d}{ds}\left(\tensor{h}{^a_b}(\Theta) \delta \Theta^b \right) \\
=&iU^{-1}t_a U \delta \Theta^c\frac{d\Theta^b}{ds}(\tensor{h}{^a_{b,c}}-\tensor{h}{^a_{c,b}} )+\frac{d}{ds}\left( iU^{-1}t_a U \tensor{h}{^a_b}(\Theta) \delta \Theta^b \right) \\
&-i\frac{d}{ds}\left( iU^{-1}t_a U\right) \tensor{h}{^a_b}(\Theta) \delta \Theta^b 
\end{align*}
となり,さらに仮定(a)より(中心電荷なしの)リー交換関係(2.2.22)を使って
\begin{align*}
\frac{d}{ds}\left(U^{-1}t_a U\right)=& -U^{-1}\left(\frac{d}{ds}U\right)U^{-1} t_a U +U^{-1} t_a \left(\frac{d}{ds}U\right) \\
=&-iU^{-1}t_b t_a U \tensor{h}{^b_c}\frac{d \Theta^c}{ds}+iU^{-1}t_a t_b U \tensor{h}{^b_c}\frac{d \Theta^c}{ds} \quad \because (2.B.2) \\
=&iU^{-1}[t_a, t_b ]U \tensor{h}{^b_c} \frac{d\Theta^c}{ds} \\
=&-U^{-1} \tensor{C}{^d_{ab}}t_d U \tensor{h}{^b_c} \frac{d \Theta^c}{ds} \\
-i \frac{d}{ds}(U^{-1}t_a U)\tensor{h}{^a_e}\delta \Theta^e=&iU^{-1} t_d U \tensor{C}{^d_{ab}} \tensor{h}{^b_c} \tensor{h}{^a_e}\delta\Theta^e \frac{d \Theta^c}{ds}
\end{align*}
と書けるから,以上を合わせて
\begin{align*}
\frac{d}{ds}(U^{-1}\delta U)=&\frac{d}{ds}\left( iU^{-1}t_a U \tensor{h}{^a_b}(\Theta) \delta \Theta^b \right) \\
&+iU^{-1}t_a U \delta \Theta^c\frac{d\Theta^b}{ds}(\tensor{h}{^a_{b,c}} -\tensor{h}{^a_{c,b}}+\tensor{C}{^a_{ed}}\tensor{h}{^e_b}\tensor{h}{^d_c})
\end{align*}
となる.ここで,結合条件(2.B.6)で極限$\theta_3 ,\theta_2\to 0$をとる.まず$\theta_3\to 0$は前と同じようにできて((2.B.7)の計算を見返す)
\begin{align*}
\frac{\partial f^a(\theta_2,\theta_1)}{\partial \theta_2^b} \left.\frac{\partial f^b(\bar{\theta},\theta_2)}{\partial \bar{\theta}^c}\right|_{\bar{\theta}=0}=&\left. \frac{\partial }{\partial \bar{\theta}^c}f\Bigl(\bar{\theta},f(\theta_2,\theta_1)\Bigr)\right|_{\bar{\theta}=0}
\end{align*}
さらに加えて$\theta_2 \to 0$とする.一つずつ見ていくと
\begin{align*}
\frac{\partial f^a(\theta_2,\theta_1)}{\partial \theta_2^b}=&\left.\frac{\partial f^a(\theta_2,\theta_1)}{\partial \theta_2^b}\right|_{\theta_2=0}+ \left. \frac{\partial^2 f^a(\theta_2,\theta_1)}{\partial \theta_2^e \partial \theta_2^b} \right|_{\theta_2=0}\theta_2^e\\
=&\tensor{[h^{-1}]}{^a_b}(\theta_1) +\left. \frac{\partial^2 f^a(\theta_2,\theta_1)}{\partial \theta_2^e \partial \theta_2^b} \right|_{\theta_2=0} \theta_2^e  \\
\left.\frac{\partial f^b(\bar{\theta},\theta_2)}{\partial \bar{\theta}^c}\right|_{\bar{\theta}=0}=&\left. \frac{\partial}{\partial \bar{\theta}^c}\left( f^b(\bar{\theta},0)+\left. \frac{\partial f^b(\bar{\theta},\theta_2)}{\partial \theta^d_2} \right|_{\theta_2=0}\theta^d_2\right)\right|_{\bar{\theta}=0} \\
=&\left. \frac{\partial}{\partial \bar{\theta}^c}\left( \bar{\theta}^b+\left. \frac{\partial f^b(\bar{\theta},\theta_2)}{\partial \theta^d_2} \right|_{\theta_2=0}\theta^d_2\right)\right|_{\bar{\theta}=0} \\
=&\delta^b_c +\left.\frac{\partial^2 f^b(\bar{\theta},\theta_2) }{\partial \bar{\theta}^c \partial \theta^d_2}\right|_{\bar{\theta}=\theta_2=0}\theta^d_2 \\
=&\delta^b_c +\tensor{f}{^b_{cd}} \theta^d_2 \quad \because (2.2.19) \\
\left. \frac{\partial }{\partial \bar{\theta}^c}f\Bigl(\bar{\theta},f(\theta_2,\theta_1)\Bigr)\right|_{\bar{\theta}=0}=&\left. \frac{\partial}{\partial \bar{\theta}^c}f^a\left(\bar{\theta},f^d(0,\theta_1)+\left .\frac{\partial f^d(\theta_2,\theta_1)}{\partial \theta_2^e}\right|_{\theta_2=0} \theta^e_2\right)\right|_{\bar{\theta}=0} \\
=&\left. \frac{\partial}{\partial \bar{\theta}^c}f^a\Bigl(\bar{\theta},\theta^d_1+\tensor{[h^{-1}]}{^d_e}(\theta_1)\theta^e_2\Bigr)\right|_{\bar{\theta}=0} \\
=&\left. \frac{\partial}{\partial \bar{\theta}^c} \left[ f^a(\bar{\theta},\theta_1)+\frac{\partial f^a(\bar{\theta},\theta_1)}{\partial \theta^d_1}\tensor{[h^{-1}]}{^d_e}(\theta_1)\theta^e_2 \right]\right|_{\bar{\theta}=0} \\
=&\left. \frac{\partial f^a(\bar{\theta},\theta_1)}{\partial \bar{\theta}^c}\right|_{\bar{\theta}=0}+ \left. \frac{\partial f^a(\bar{\theta},\theta_1)}{\partial \theta^d_1 \partial \bar{\theta}^c } \right|_{\bar{\theta}=0} \tensor{[h^{-1}]}{^d_e}(\theta_1)\theta^e_2 \\
=&\tensor{[h^{-1}]}{^a_c}(\theta_1)+\left(\frac{\partial}{\partial \theta^d_1}\tensor{[h^{-1}]}{^a_c}(\theta_1)\right)\tensor{[h^{-1}]}{^d_e}(\theta_1)\theta^e_2
\end{align*}
よって元の式に代入して,$\theta_2$について1次の項の係数比較をすれば
\begin{align*}
(\mathrm{LHS})=&\frac{\partial f^a(\theta_2,\theta_1)}{\partial \theta_2^b} \left.\frac{\partial f^b(\bar{\theta},\theta_2)}{\partial \bar{\theta}^c}\right|_{\bar{\theta}=0} \\
=&\tensor{[h^{-1}]}{^a_c}(\theta_1) + \left. \frac{\partial^2 f^a(\theta_2,\theta_1)}{\partial \theta_2^e \partial \theta_2^c} \right|_{\theta_2=0} \theta_2^e +\tensor{[h^{-1}]}{^a_b}(\theta_1) \tensor{f}{^b_{cd}} \theta^d_2 \\
=&\tensor{[h^{-1}]}{^a_c}(\theta_1) + \left[\left. \frac{\partial^2 f^a(\theta_2,\theta_1)}{\partial \theta_2^d \partial \theta_2^c} \right|_{\theta_2=0} +\tensor{[h^{-1}]}{^a_b}(\theta_1) \tensor{f}{^b_{cd}} \right]\theta^d_2 \\
(\mathrm{RHS})=&\tensor{[h^{-1}]}{^a_c}(\theta_1)+\left(\frac{\partial}{\partial \theta^b_1}\tensor{[h^{-1}]}{^a_c}(\theta_1)\right)\tensor{[h^{-1}]}{^b_d}(\theta_1)\theta^d_2 \\
\therefore \quad &\left. \frac{\partial^2 f^a(\theta_2,\theta_1)}{\partial \theta_2^d \partial \theta_2^c} \right|_{\theta_2=0} +\tensor{[h^{-1}]}{^a_b}(\theta_1) \tensor{f}{^b_{cd}}=\left(\frac{\partial}{\partial \theta^b_1}\tensor{[h^{-1}]}{^a_c}(\theta_1)\right)\tensor{[h^{-1}]}{^b_d}(\theta_1)
\end{align*}
が得られる.さらに右辺は$\frac{\partial}{\partial \theta}h^{-1}=-h^{-1} \left(\frac{\partial}{\partial \theta}h \right) h^{-1}$を用いて
\begin{align*}
\left(\frac{\partial}{\partial \theta^b_1}\tensor{[h^{-1}]}{^a_c}(\theta_1)\right)\tensor{[h^{-1}]}{^b_d}(\theta_1) =& -\tensor{[h^{-1}]}{^a_m}(\theta_1)\left(\frac{\partial}{\partial \theta^b_1}\tensor{h}{^m_n}(\theta_1)\right) \tensor{[h^{-1}]}{^n_c}(\theta_1) \tensor{[h^{-1}]}{^b_d}(\theta_1) \\
=&-\tensor{[h^{-1}]}{^a_m}(\theta_1)\tensor{h}{^m_{n,b}}(\theta_1) \tensor{[h^{-1}]}{^n_c}(\theta_1) \tensor{[h^{-1}]}{^b_d}(\theta_1)
\end{align*}
と書ける.左から$h$を一つ,右から$h$を2つかけて行列計算することで
\begin{align*}
\tensor{h}{^a_{b,c}}(\theta)=-\tensor{f}{^a_{de}} \tensor{h}{^d_b}(\theta) \tensor{h}{^e_c}(\theta)+\tensor{h}{^a_d}(\theta)\left. \frac{\partial^2 f^a(\theta_2,\theta)}{\partial \theta_2^d \partial \theta_2^e} \right|_{\theta_2=0} \tensor{h}{^d_b}(\theta)\tensor{h}{^e_c}(\theta)
\end{align*}
となる.右辺第二項目は(微分が$d,e$について対称だから)$b,c$について対称になっている.したがって$b,c$について反対称化させると消えて
\begin{align*}
\tensor{h}{^a_{c,b}}(\theta)-\tensor{h}{^a_{b,c}}(\theta)=&-\tensor{f}{^a_{de}} \tensor{h}{^d_c}(\theta) \tensor{h}{^e_b}(\theta)+\tensor{f}{^a_{de}} \tensor{h}{^d_b}(\theta) \tensor{h}{^e_c}(\theta) \\
=&-\tensor{f}{^a_{ed}} \tensor{h}{^d_b}(\theta) \tensor{h}{^e_c}(\theta)+\tensor{f}{^a_{de}} \tensor{h}{^d_b}(\theta) \tensor{h}{^e_c}(\theta) \\
=&-(-\tensor{f}{^a_{de}}+\tensor{f}{^a_{ed}})\tensor{h}{^d_b}(\theta) \tensor{h}{^e_c}(\theta) \\
=&-\tensor{C}{^a_{de}}\tensor{h}{^d_b}(\theta) \tensor{h}{^e_c}(\theta) \quad\quad  \because (2.2.23) \\
\end{align*}
となるから
\begin{align*}
\tensor{h}{^a_{c,b}}(\theta)-\tensor{h}{^a_{b,c}}(\theta)+\tensor{C}{^a_{ed}}\tensor{h}{^e_b}(\theta) \tensor{h}{^d_c}(\theta)=0
\end{align*}
が成立することが分かる.これにより(2.B.9)の最後の項は消えることから
\begin{align*}
\frac{d}{ds}\left[U^{-1}\delta U - iU^{-1}t_a U \tensor{h}{^a_b}(\Theta) \delta \Theta^b \right]=0
\end{align*}
となり,したがって
\begin{align*}
A:=U^{-1}\delta U - iU^{-1}t_a U \tensor{h}{^a_b}(\Theta) \delta \Theta^b
\end{align*}
という量は経路のパラメータ$s$に対して定数であることがわかる.\par
$\delta \Theta$は始点$\Theta_\theta(0)=0$と終点$\Theta_\theta(1)=\theta$は変えない,つまり$\delta \Theta(0)=\delta \Theta(1)=0$である.さらに初期条件(2.B.3)$U_\theta(0)=1$より$\delta U(0)=0$もわかっているから,$s=0$とおいて
\begin{align*}
A=U(0)\underset{=0}{\uwave{\delta U(0)}} -i U^{-1}(0) t_a U(0) \tensor{h}{^a_b} \underset{=0}{\uwave{\delta \Theta^b(0)}}=0
\end{align*}
より$s$の値にかかわらず$A=0$がわかり,さらに$s=1$とおいて
\begin{align*}
0=A=&U^{-1}(1)\delta U(1)-iU^{-1}(1) t_a U(1) \tensor{h}{^a_b} \underset{=0}{\uwave{\delta \Theta^b(1)}} \\
=&U^{-1}(1)\delta U(1)
\end{align*}
よって$\delta U(1)=0$であり,$U_\theta(1)$は始点$\Theta_\theta(0)=0$と終点$\Theta_\theta(1)=\theta$を変えない経路の微小変分$\delta \Theta$のもとで停留することが分かる.($\delta U(s)$とは,$\delta \Theta$だけ変化させたときの$U_\theta(s)$の変化分だったことを思い出そう.)仮定(b)より,パラメータ空間$M$は単連結であり,どのような経路も互いに連続的に変形できるので,よって$U_\theta(1)$は$0$から$\theta$の間の途中経路に依らない.したがって$U_\theta(1)$を$\theta$のみの関数とできる!
\begin{align*}
U[\theta]:=U_\theta(1)
\end{align*}
特に,経路$\mc{P}$は$0$から$\theta=f(\theta_2,\theta_1)$を繋ぐので
\begin{align*}
U_{\mc{P}}(1)=U[f(\theta_2,\theta_1)]
\end{align*}
が成立する.したがって(2.B.8)より
\begin{align*}
U[f(\theta_2,\theta_1)]=&U_{\mc{P}}(1) \\
=&U_{\theta_2}(1)U_{\theta_1}(1) \\
=&U[\theta_2]U[\theta_1]
\end{align*}
が成立し,$U[\theta]$が群の積の法則(2.B.1)を満たすことが分かる.これで証明が完了した.

\vskip\baselineskip


これで非射影的表現$U[\theta]$が構成できた.しかしまだ,別の表現をとってきたら射影表現になっており,位相だけの再定義しても非射影表現にならないような表現があるかもしれない.よって最後に,同じ表現の生成子$t_a$を持つ同じ群のどのような射影表現$\tilde{U}[\theta]$も$U[\theta]$と位相だけしか違わない
\begin{align*}
\tilde{U}[\theta]=e^{i\alpha(\theta)}U[\theta]
\end{align*}
ことを証明する.これが成立すれば,射影表現$\tilde{U}[\theta]$の積の法則での位相$\phi$
\begin{align*}
\tilde{U}[\theta']\tilde{U}[\theta]=e^{i\phi(\theta',\theta)}\tilde{U}[f(\theta',\theta)]
\end{align*}
は単に$\tilde{U}[\theta]$の位相変化$\tilde{U}\to e^{-i\alpha}\tilde{U}$で取り除けることになる.これを見るには,演算子
\begin{align*}
U[\theta]^{-1} U[\theta']^{-1} \tilde{U}[\theta']\tilde{U}[\theta]=&\Bigl( U[\theta'] U[\theta]\Bigr)^{-1}\Bigl( \tilde{U}[\theta']\tilde{U}[\theta] \Bigr) \\
=&U[f(\theta',\theta)]^{-1} \tilde{U}[f(\theta',\theta)]e^{i\phi(\theta',\theta)}
\end{align*}
を考える.仮定より,$U[\theta]$と$\tilde{U}[\theta]$は同じ生成子$t_a$を持つ
\begin{align*}
\left. \frac{\partial}{\partial \theta^a} U[\theta]\right|_{\theta=0}=\left. \frac{\partial}{\partial \theta^a} \tilde{U}[\theta]\right|_{\theta=0}=it_a
\end{align*}
から,左辺の$\theta'^a$微分は$\theta'=0$でゼロになる.
\begin{align*}
\left. \frac{\partial}{\partial \theta^a} U[\theta]^{-1}\right|_{\theta=0}=&-U[0]^{-1}\left.\left( \frac{\partial}{\partial \theta^a} U[\theta] \right)\right|_{\theta=0} U[0]^{-1}=-it_a \\
\left. \frac{\partial}{\partial \theta'^a} \left\{U[\theta']^{-1}\tilde{U}[\theta']\right\}\right|_{\theta=0}=&-it_a +it_a=0 \\
\therefore \quad \frac{\partial}{\partial \theta'^a}\left\{U[\theta]^{-1} U[\theta']^{-1} \tilde{U}[\theta']\tilde{U}[\theta] \right\}_{\theta'=0}=&0
\end{align*}
したがって,両辺の$\theta'$微分は
\begin{align*}
&\frac{\partial}{\partial \theta'^a}\left\{U[\theta]^{-1} U[\theta']^{-1} \tilde{U}[\theta']\tilde{U}[\theta] \right\} \\ =&\frac{\partial}{\partial \theta'^a} \left\{U[f(\theta',\theta)]^{-1} \tilde{U}[f(\theta',\theta)]e^{i\phi(\theta',\theta)} \right\} \\
=&\frac{\partial}{\partial \theta'^a}\left\{U[f(\theta',\theta)]^{-1} \tilde{U}[f(\theta',\theta)]\right\}e^{i\phi(\theta',\theta)} \\
&+ U[f(\theta',\theta)]^{-1} \tilde{U}[f(\theta',\theta)]^{-1} e^{i\phi(\theta',\theta)}i\frac{\partial}{\partial \theta'^a}\phi(\theta',\theta) \\
=&e^{i\phi(\theta',\theta)}\left[\frac{\partial f^b(\theta',\theta)}{\partial \theta'^a}\frac{\partial}{\partial \theta^b}\left\{ U[\theta]^{-1} \tilde{U}[\theta]\right\}_{\theta=f(\theta',\theta)}+U[f(\theta',\theta)]^{-1} \tilde{U}[f(\theta',\theta)]^{-1}i\frac{\partial}{\partial \theta'^a}\phi(\theta',\theta)\right]
\end{align*}
となり,$\theta'=0$とおけば左辺はゼロになるから,
\begin{align*}
0=&\left. \frac{\partial f^b(\theta',\theta)}{\partial \theta'^a}\right|_{\theta'=0}\frac{\partial}{\partial \theta^b}\left\{ U[\theta]^{-1} \tilde{U}[\theta]\right\}_{\theta=f(0,\theta)} \\
&+ i\left[\frac{\partial}{\partial \theta'^a}\phi(\theta',\theta)\right]_{\theta'=0} U[f(0,\theta)]^{-1} \tilde{U}[f(0,\theta)] \\
=&\tensor{[h^{-1}]}{^b_a}(\theta)\frac{\partial}{\partial \theta^b}\left\{ U[\theta]^{-1} \tilde{U}[\theta]\right\} \\
&+ i\left[\frac{\partial}{\partial \theta'^a}\phi(\theta',\theta)\right]_{\theta'=0} U[\theta]^{-1} \tilde{U}[\theta] \\
\therefore \quad 0=&\frac{\partial}{\partial \theta^b}\left\{ U[\theta]^{-1} \tilde{U}[\theta]\right\} +i \left[\frac{\partial}{\partial \theta'^a}\phi(\theta',\theta)\right]_{\theta'=0}\tensor{h}{^a_b}(\theta) U[\theta]^{-1} \tilde{U}[\theta] \\
=&\frac{\partial}{\partial \theta^b}\left\{ U[\theta]^{-1} \tilde{U}[\theta]\right\} +i \left[\frac{\partial}{\partial \theta'^a}\phi(\theta',\theta)\right]_{\theta'=0}\tensor{h}{^a_b}(\theta) U[\theta]^{-1} \tilde{U}[\theta] \\
=&\frac{\partial}{\partial \theta^b}\left\{ U[\theta]^{-1} \tilde{U}[\theta]\right\} +i \phi_b(\theta) U[\theta]^{-1} \tilde{U}[\theta]
\end{align*}
となる.ここで
\begin{align*}
\phi_b(\theta):=\left[\frac{\partial}{\partial \theta'^a}\phi(\theta',\theta)\right]_{\theta'=0}\tensor{h}{^a_b}(\theta)
\end{align*}
だ.もう一度$\theta^c$で微分すると
\begin{align*}
0=&\frac{\partial^2}{\partial \theta^c \partial \theta^b}\left\{ U[\theta]^{-1} \tilde{U}[\theta]\right\} +i \frac{\partial \phi_b(\theta) }{\partial \theta^c} U[f(\theta',\theta)]^{-1} \tilde{U}[f(\theta',\theta)]^{-1} \\
&+i \phi_b(\theta) \frac{\partial}{\partial \theta^c}\left\{U[\theta]^{-1} \tilde{U}[\theta]\right\} \\
=&\frac{\partial^2}{\partial \theta^c \partial \theta^b}\left\{ U[\theta]^{-1} \tilde{U}[\theta]\right\} +i \frac{\partial \phi_b(\theta) }{\partial \theta^c} U[f(\theta',\theta)]^{-1} \tilde{U}[f(\theta',\theta)]^{-1} \\
&+ \phi_b(\theta) \phi_c(\theta)U[\theta]^{-1}\tilde{U}[\theta] \quad \because \frac{\partial}{\partial \theta^b}\left\{ U[\theta]^{-1} \tilde{U}[\theta]\right\} =-i \phi_b(\theta) U[\theta]^{-1} \tilde{U}[\theta]
\end{align*}
となる.第一項目と第三項目は$b,c$について対称だから,$b,c$について反対称化すると
\begin{align*}
0=\frac{\partial \phi_b(\theta) }{\partial \theta^c} -\frac{\partial \phi_c (\theta)}{\partial \theta^b} 
\end{align*}
を得る.パラメータ空間$M$は単連結だから,これはポアンカレの定理(あるいは可積分条件.単連結な空間上で,$d \omega=0$ならば$\omega =d\zeta$となる$\zeta$の存在($d$は外微分))より
\begin{align*}
d(\phi_a d\theta^a) =&\frac{\partial \phi^a}{\partial \theta^b} d\theta^b \wedge d\theta^a \\
=&\frac{\partial \phi^a}{\partial \theta^b} (d\theta^b \otimes d\theta^a-d\theta^a \otimes d\theta^b) \\
=&\left(\frac{\partial \phi^b}{\partial \theta^a}-\frac{\partial \phi^a}{\partial \theta^b}\right)d\theta^a d\theta^b \\
=&0 \\
\therefore \quad \exists \beta \quad & \mathrm{s.t}\quad \phi_b d\theta^a=d\beta=\frac{\partial \beta(\theta)}{\partial \theta^b}d\theta^b \\
\phi_b(\theta)=&\frac{\partial \beta(\theta)}{\partial \theta^b}
\end{align*}
と書ける.したがって,
\begin{align*}
\frac{\partial}{\partial \theta^b}\left[U[\theta]^{-1} \tilde{U}[\theta]e^{i\beta(\theta)}\right]=&\frac{\partial}{\partial \theta^b}\left[U[\theta]^{-1} \tilde{U}[\theta]\right]e^{i\beta(\theta)}+U[\theta]^{-1} \tilde{U}[\theta]i\frac{\partial \beta(\theta)}{\partial \theta^b}e^{i\beta(\theta)} \\
=&-i \phi_b(\theta) U[\theta]^{-1} \tilde{U}[\theta]+i \phi_b(\theta) U[\theta]^{-1} \tilde{U}[\theta] \\
=&0
\end{align*}
から,量
\begin{align*}
B:=U[\theta]^{-1} \tilde{U}[\theta]e^{i\beta(\theta)}
\end{align*}
は$\theta$について一定であることがわかる.その値を$\theta=0$での値ととれば
\begin{align*}
B:=U[0]^{-1} \tilde{U}[0]e^{i\beta(0)}=&e^{i\beta(0)} \quad \because U[0]=\tilde{U}[0]=1\\
U[\theta]^{-1} \tilde{U}[\theta]e^{i\beta(\theta)}=&B=e^{i\beta(0)} \\
\therefore \quad \tilde{U}[\theta]=&U[\theta]\exp(-i\beta(\theta)+i\beta(0))
\end{align*}
となる.これで証明が完了した.

\vskip\baselineskip


この解析により,「リー代数に中心電荷がない」が,「群は単連結でない」とき,群の積の法則に現れる位相因子の性質についてある程度のことがわかる.いま,0から$\theta$へ,そして$f(\bar{\theta},\theta)$へ進む経路$\mc{P}$が,0から$f(\bar{\theta},\theta)$へ行く基準経路に変形できないとする.言い換えると,0から$\theta$,$f(\bar{\theta},\theta)$へと進み,再び0へと戻るある経路が連続的に1点に縮めることができない,とする.このとき,このとき$U^{-1}[f(\theta_2,\theta_1)]U[\theta_2]U[\theta_1]$は位相因子$\exp(i\phi(\theta_2,\theta_1))\neq 1$となりうる.しかし$\phi$は互いに連続的に変形できる経路については\uwave{すべて同一}だ.(原点0からスタートして0に戻ってきたときの原点での$U[0\to 0]=e^{i\phi}$は1ではないが,その経路から若干変化させたときの$\delta U$は,$\Theta$が連続的に変形できる限り0になり,よって位相因子$e^{i\phi}$は変化しないからだ.)原点から出て原点に戻る全てのループのうち,あるループに連続的に変形できるものの集合は,そのループの\uwave{ホモトピー類}を作る.これまでに見たように,$\phi(\theta_2,\theta_1)$は0から$\theta$,$f(\bar{\theta},\theta)$へと進み0に戻る経路のホモトピー類にのみ依存する.ホモトピー類の集合は群をなす(ホモトピー群).$\mc{L}_1$と$\mc{L}_2$のホモトピー群の「積」は$\mc{L}_1$と$\mc{L}_2$を続いてまわるループのホモトピー類だ.ループ$\mc{L}$のホモトピー類の「逆元」は$\mc{L}$を逆方向に回るループのホモトピー類だ.「単位元」は原点に縮められるループのホモトピー類だ.この群は,この空間の第一ホモトピー群,または基本群という.位相因子がこの群の表現となっていることはすぐにわかる.実際,ループ$\mc{L}$を一周して位相因子$e^{i\phi}$が出て,ループ$\bar{\mc{L}}$を一周して位相因子$e^{i\bar{\phi}}$が出るなら,両方のループを回れば位相因子$e^{i\phi}e^{i\bar{\phi}}$が出る.したがって,もし与えられた(中心電荷を持たない)群$\mc{G}$のパラメータ空間$M$の第一ホモトピー群の1次元表現がわかれば,$\mc{G}$の射影表現として適切なものを全て書き出せることになる.例えば,$SO(3,1)$の第一ホモトピー群は$\mathbb{Z}_2$であり,したがって射影表現の位相因子は$\mathbb{Z}_2$の一次元表現$\{+1,-1\}$であることがわかる.\par
ホモトピー群は4巻で詳しく調べる.






\newpage






\subsection*{補遺C:反転と縮退した多重項}
通常,反転$\mathsf{T}$と$\mathsf{P}$は,1粒子状態を同じ種類の別の1粒子状態へ,しばしば粒子の種類に依る位相因子を伴って変換する,と仮定する.2.6節では,1粒子状態の縮退した多重項に対してはこれより複雑に作用するかもしれない,と簡潔に述べた.ここでは一般化された反転演算子を調べる.\par
時間反転から調べる.より一般的な可能性を探るために,質量がゼロでない1粒子には次のように作用するとする.
\begin{align*}
\mathsf{T}\Psi_{\mathbf{p},\sigma,n}=(-1)^{j-\sigma}\sum_m \mc{T}_{mn}\Psi_{-\mathbf{p},-\sigma,m}
\end{align*}
ここで$\mathbf{p},j,\sigma$はそれぞれ,この粒子の運動量,スピン,スピンの$z$成分であり,$n,m$はその粒子の縮退した多重項の成分を区別するための添え字だ.(2.6節で述べたように,1粒子でもスピン1/2の粒子が奇数個,つまり1個あれば時間反転でクラマーの縮退は起きる.)因子$(-1)^{j-\sigma}$は(2.6.17)と同様に導かれる.行列$\mc{T}_{mn}$は,$\mathsf{T}$が反ユニタリーだから
\begin{align*}
\delta^3(\mathbf{p}'-\mathbf{p})\delta_{\sigma'\sigma}\delta_{n'n}=&\left(\Psi_{\mathbf{p}',\sigma',n'},\Psi_{\mathbf{p},\sigma,n}\right)=(\mathsf{T}\Psi_{\mathbf{p},\sigma,n},\mathsf{T}\Psi_{\mathbf{p}',\sigma',n'}) \\
=&(-1)^{j-\sigma}(-1)^{j-\sigma'}\sum_{mm'}\mc{T}^*_{mn}\mc{T}_{m'n'}(\Psi_{-\mathbf{p},-\sigma,m},\Psi_{-\mathbf{p}',-\sigma',m'}) \\
=&(-1)^{j-\sigma}(-1)^{j-\sigma'}\sum_{mm'}\mc{T}^\dagger_{nm}\mc{T}_{m'n'}\delta^3(\mathbf{p}'-\mathbf{p})\delta_{mm'}\delta_{\sigma\sigma'} \\
=&(-1)^{2j-2\sigma}\left(\mc{T}^\dagger\mc{T}\right)_{nn'}\delta^3(\mathbf{p}'-\mathbf{p})\delta_{\sigma\sigma'}
\end{align*}
$(-1)^{2j-2\sigma}$は$j-\sigma$が必ず整数だから必ず$1$となる.したがってこれが成り立つためには$\mc{T}$はユニタリー行列でなければならない.ただしそれ以外は未知だ.\par
さて,1粒子状態の基底を適当に選びなおして,$\mc{T}$を対角化できるかなどで,この変換を簡単化できるか見てみる.新しい状態を
\begin{align*}
\Psi_{\mathbf{p},\sigma,n}'=\sum_{m}\mc{U}_{mn}\Psi_{\mathbf{p},m}
\end{align*}
で定義する.このとき,変換(2.C.1)は
\begin{align*}
\mathsf{T}\Psi'_{\mathbf{p},\sigma,n}=&\sum_{m}\mc{U}_{mn}^*\mathsf{T}\Psi_{\mathbf{p},m} \\
=&(-1)^{j-\sigma}\sum_{mm'} \mc{U}^*_{mn}\mc{T}_{mm'}\Psi_{-\mathbf{p},-\sigma,m'} \\
=&(-1)^{j-\sigma}\sum_{mm'kl} \mc{U}^{-1}_{nm}\mc{T}_{mm'}\mc{U}_{m'l}^* (\mc{U}^\dagger_{lk})^*\Psi_{-\mathbf{p},-\sigma,k} \quad \because (ユニタリー性 \mc{U}^\dagger =\mc{U}^{-1})\\
=&(-1)^{j-\sigma}\sum_{mm'kl} \mc{U}^{-1}_{nm}\mc{T}_{mm'}\mc{U}_{m'l}^* \mc{U}_{kl}\Psi_{-\mathbf{p},-\sigma,k} \\
=&(-1)^{j-\sigma}\sum_{m}(\mc{U}^{-1}\mc{T} \mc{U}^*)_{nm}\Psi'_{-\mathbf{p},-\sigma,m}
\end{align*}
となるから,これは行列$\mc{T}$を以下と入れ替えたもので与えられる.
\begin{align*}
\mc{T}' :=\mc{U}^{-1}\mc{T} \mc{U}^*
\end{align*}
$\mc{T}$はユニタリー行列だから,うまいユニタリー行列$\mc{U}$で$\mc{U}^{-1}\mc{T}\mc{U}$と挟めれば対角化できる.しかし今回は右の$\mc{U}$が$\mc{U}^*$となっており,この形では一般的には対角化できない.したがって一般的には,$\mathsf{T}$がユニタリーのときのように,このように1粒子状態の基底の選択によって$\mc{T}'$を\uwave{対角的}にすることはできない.しかし,その代わりに\uwave{ブロック対角}にはできる.そのブロックは,$1\times 1$の位相か,以下の形の$2\times 2$行列になる.
\begin{align*}
\left(
\begin{matrix}
0 & e^{i\phi/2} \\
e^{-i\phi/2} & 0
\end{matrix}
\right)
\end{align*}
ここで$\phi$は色々な実数の位相である.まずこの命題を証明しよう.



\vskip\baselineskip


まず(2.C.2)より
\begin{align*}
\mc{T}'\mc{T}'^*=&\mc{U}^{-1} \mc{T} \mc{U}^* (\mc{U}^{-1})^* \mc{T}^* \mc{U} \\
=&\mc{U}^{-1} \mc{T} \mc{T}^* \mc{U}
\end{align*}
が得られる.これはユニタリー行列$\mc{U}$による相似変換になっているから,ユニタリー行列$\mc{T}'\mc{T}'^*$が対角的になっているように$\mc{U}$を選ぶことができる.\par
エルミート行列が対角化可能なことは有名だが,ユニタリー行列もユニタリー行列による相似変換によって対角化できることはあまり有名ではないから,ここで証明しておく.まず,任意の$n\times n$ユニタリー行列$U$の固有値は絶対値1の複素数となることを示す.固有値$\lambda$とその固有ベクトル$\bm{x}$をとることで
\begin{align*}
U\bm{x} =&\lambda \bm{x} \\
(\lambda \bm{x})^\dagger (\lambda \bm{x})=&|\lambda|^2 ||\bm{x}||^2 \\
=(U\bm{x})^\dagger (U\bm{x})=& \bm{x}^\dagger U^\dagger U \bm{x} =||\bm{x}||^2
\end{align*}
となり,$|\lambda|=1$がわかる.次に,ユニタリー行列においてもエルミート行列と同様に異なる固有値に属する固有ベクトルは直交することを示す.(一般の行列の場合,固有ベクトル同士が正規直交することは普通成り立たない.)これは固有値が絶対値1の複素数であることから$e^{i\lambda} (\lambda \in \mathbb{R})$と書けて,異なる固有値をもつベクトルをそれぞれ$\bm{u},\bm{v}$とすれば
\begin{align*}
U \bm{u} =e^{i\lambda} \bm{u} ,&\quad U\bm{v}=e^{i\mu} \bm{v}  \quad \lambda,\mu \in \mathbb{R},e^{i\lambda}\neq e^{i\mu} \\
(U \bm{u})^\dagger (U\bm{v})=& e^{i(\mu-\lambda)}\bm{u}^\dagger \bm{v} \\
=\bm{u}^\dagger U^\dagger U\bm{v} =&\bm{u}^\dagger \bm{v} \\
\therefore \quad [e^{i(\mu-\lambda)}-1]\bm{u}^\dagger \bm{v}=&0
\end{align*}
となり,仮定より$e^{i(\mu-\lambda)}\neq 1$であるから$\bm{u}^\dagger \bm{v}\neq 0$が示せる.したがって,$n\times n$ユニタリー行列$U$に対して$n$個の正規直交基底がとれて,それらを$\bm{u}_1,\cdots ,\bm{u}_n$とすれば,それらが固有値$\lambda_1=e^{i\phi_1},\cdots ,\lambda_n=e^{i\phi_n}$を持つとすれば
\begin{align*}
P^{-1}UP=\mathrm{diag}(e^{i\phi_1},\cdots ,e^{i\phi_n}) ,\quad P=[\bm{u}_1,\cdots ,\bm{u}_n]
\end{align*}
としてユニタリー行列$P$を用いて対角化することが可能となる.\par
これが$\mc{U}$によって行われ,$D$が位相$e^{i\phi_n}$を対角成分とする対角行列$D_{nm}=e^{i\phi_n}\delta_{nm}$となっていて,$\mc{T}'\mc{T}'^*=D$となっているとする.プライムを省略して
\begin{align*}
\mc{T} \mc{T}^* =&\mc{T} (\mc{T}^T)^{-1}=D \\
\therefore \quad \mc{T} =&D \mc{T}^T
\end{align*}
となる.これからすぐに
\begin{align*}
\mc{T}_{nm}=&\sum_{l}D_{nl}\mc{T}_{lm}^T \\
=&e^{i\phi_n}\mc{T}_{mn} \\
\therefore \quad \mc{T}_{nn}=&e^{i\phi_n}\mc{T}_{nn}
\end{align*}
となり,$e^{i\phi_n}=1$でない限り対角成分$\mc{T}_{nn}$はゼロになることがわかる.さらに,
\begin{align*}
\mc{T}_{nm}=& e^{i\phi_n} \mc{T}_{mn} \\
=&e^{i\phi_n} e^{i\phi_m}\mc{T}_{nm}
\end{align*}
だから,$e^{i\phi_n}=1$でも$e^{i\phi_m}\neq 1$ならば$\mc{T}_{nm}=\mc{T}_{mn} =0$となることがわかる.(ただし$e^{i\phi_n}\neq 1$かつ$e^{i\phi_m}\neq 1$となる場合でも$e^{i\phi_m}=e^{-i\phi_n}$となるときに限り,対角成分$\mc{T}_{nn}$はゼロだが非対角成分$\mc{T}_{nm}$はゼロとは限らない.$e^{i\phi_m}=e^{-i\phi_n}$も満たさない$(n,m)$では$\mc{T}_{nm}=\mc{T}_{mn}=0$だ.)したがって,$e^{i\phi_n}=e^{i\phi_m}=1$となる組$(n,m)$の成分では$\mc{T}_{nn},\mc{T}_{nm},\mc{T}_{mn}$は非ゼロになることができ,特にそのような$(n,m)$成分だけから構成される部分行列$\mc{T}_{nm}$は$\mc{T}_{nm}=\mc{T}_{mn}$対称行列となる.\par
よって,$e^{i\phi_n}=1$となる全ての行と列を最初に並べると約束すれば,そのような$n$で最大のものを$M$とおいて,$N\times N$行列$\mc{T}$は
\begin{align*}
\mc{T}=\left(
\begin{matrix}
\mc{A}  & 0 \\
0 & \mc{B}
\end{matrix}
\right)
\end{align*}
という形になる.(さらにちゃんと説明すると,対角にした後に$ii$成分と$jj$成分を入れ替えるのは
\begin{align*}
\mc{P}_{ij}=&\left(
\begin{array}{ccccccccccc}
1          &      0      & \cdots  &              &              &            &            &            &             &            &        \\
0          & \ddots   &             &              &              &            &            &            &             &            &        \\
\vdots  &              & 1          &              &              &            &            &            &             &            &        \\
            &              &             &  0          & \cdots   &            &            & 1         &             &            &        \\
            &              &             &  \vdots  & 1           &            &            & \vdots  &             &            &        \\
            &              &             &              &              &\ddots  &            &            &             &            &        \\
            &              &             &              &              &            &  1        &            &             &            &        \\
            &              &             &  1          &              &            & \cdots & 0         &             &            &        \\
            &              &             &              &              &            &            &            &1           &            &        \\
            &              &             &              &              &            &            &            &             & \ddots &        \\
            &              &             &              &              &            &            &            &             &            &  1    
\end{array}
\right) \\
=&I-E_{ii}-E_{jj}+E_{ij}+E_{ji}
\end{align*}
で$\mc{P}_{ij}^{-1}D\mc{P}_{ij}$と挟めばいい.これを繰り返し$\mc{P}=\mc{P}_{i_1 j_1}\mc{P}_{i_2 j_2}\cdots \mc{P}_{i_nj_n}$で$\mc{P}^{-1}D\mc{P}$と作用させて,望みの順番になるようにしてやればいい.ここで$E_{ij}$は$ij$成分のみが1でそれ以外全ての成分がゼロの行列(行列基底)だ.この行列で挟む操作は,行列の$i$列目と$j$列目を入れ替えるのと$i$行目と$j$行目を入れ替える操作に対応する.これも実直交行列となっているから,$\mc{T}$から$\mc{T}'$への変換は$D$を対角化させる$\mc{U}$と一緒にして$\mc{U}\mc{P}$で変換してやればいい.)ここで$\mc{A}$は$M\times M$対称行列となっており,$\mc{B}$は対角成分が全てゼロの$(N-M)\times (N-M)$行列となっている.実際上での考察を用いれば,行列成分$(n,m)$の行成分$n$が$n \leq N$を満たすものであれば$e^{i\phi_n}=1$を満たし,さらに列成分$m$も$m \leq N$ならば$e^{i\phi_m}=1$も満たす.したがってこれは上の考察より$\mc{T}_{nn},\mc{T}_{nm},\mc{T}_{mn},$が非ゼロになるかもしれず,$\mc{T}_{nm}=\mc{T}_{mn}$の対称な形なっている.したがって$\mc{A}$は対称行列だ.もし行成分は$n\leq N$だが列成分が$m>N$ならば,$e^{i\phi_n}=1$だが$e^{i\phi_m}\neq 1$であるから,そのような$\mc{T}_{nm}= 0$となる.逆でも同様であり,したがって右上と左下の行列成分はゼロとなることがわかる.$n>N$かつ$m>N$($\mc{B}$の成分)では$e^{i\phi_n}\neq 1$かつ$e^{i\phi_m}\neq 1$を満たすから,対角成分はゼロ$\mc{T}_{nn}=0$となるが,$e^{i\phi_n}=e^{-i\phi_m}$を満たすかもしれないから,非対角成分が非ゼロにはなるかもしれない.\par
さらに
\begin{align*}
\mc{T}\mc{T}^*=&\left(
\begin{matrix}
\mc{A}  & 0 \\
0 & \mc{B}
\end{matrix}
\right)\left(
\begin{matrix}
\mc{A}^*  & 0 \\
0 & \mc{B}^*
\end{matrix}
\right) \\
=&\left(
\begin{matrix}
\mc{A}  & 0 \\
0 & \mc{B}
\end{matrix}
\right)\left(
\begin{matrix}
\mc{A}^\dagger  & 0 \\
0 & \mc{B}^*
\end{matrix}
\right) \quad \because \mc{A}^T =\mc{A} \\
=&\left(
\begin{matrix}
\mc{A}\mc{A}^\dagger  & 0 \\
0 & \mc{B}\mc{B}^*
\end{matrix}
\right) \\
=D=&\left(
\begin{matrix}
1  & 0  \\
0 & e^{i\phi}
\end{matrix}
\right)
\end{align*}
という形になるから(ここで$e^{i\phi}$は,対角成分が$e^{i\phi}$となる$(N-M)\times (N-M)$行列),$\mc{A}$は対称な上にユニタリーである.したがって$\mc{A}$は対称かつ反エルミートな行列の指数行列として表すことができる.実際$\mc{A}$はユニタリー行列であるから,前に述べたように,あるユニタリー行列$P$を用いて
\begin{align*}
\mc{A} =&P^{-1} \mathrm{diag}[e^{i\lambda_1},\cdots ,e^{i\lambda_N}]P \\
=&P^{-1}\exp( i \mathrm{diag}[\lambda_1,\cdots ,\lambda_N])P \\
=&\exp(i P^{-1} \mathrm{diag}[\lambda_1,\cdots ,\lambda_N]P) \\
=&e^{iX}
\end{align*}
と書くことができる.$P^{-1}\mathrm{diag}[\lambda_1,\cdots ,\lambda_N]P$は$\lambda_1,\cdots ,\lambda_N$が実数であることと$P$がユニタリーであることから自明に$X$エルミート行列であり,したがって虚数$i$と合わせた行列$X$は反エルミートとなる.さらに$\mc{A}$が対称であることから$X$も対称である必要がある.エルミートかつ対称な$X$は実対称行列である.実対称行列は実直交行列で対角化可能だったのだから,$X$を対角化させるような対称行列を$G$とし(ちなみに一般に固有値の重複などのせいで$P$とは異なる可能性あり)
\begin{align*}
G^T X G=\mathrm{diag}[x_1,\cdots, x_N]
\end{align*}
,変換(2.C.2)で,$\mc{U}$を$\mc{A}$だけに作用するように
\begin{align*}
\mc{U}=\left(
\begin{matrix}
G  & 0 \\
0 & 1
\end{matrix}
\right)
\end{align*}
と選べば,$G$が実直交行列であることからこれは実際にユニタリー$\mc{U}^\dagger =\mc{U}^{-1}$で
\begin{align*}
\mc{T}'=&\mc{U}^{-1}\mc{T} \mc{U}^* \\
=& \left(
\begin{matrix}
G^T  & 0 \\
0 & 1
\end{matrix}
\right)\left(
\begin{matrix}
\mc{A}  & 0 \\
0 & \mc{B}
\end{matrix}
\right)\left(
\begin{matrix}
G  & 0 \\
0 & 1
\end{matrix}
\right) \\
=&\left(
\begin{matrix}
G^T e^{iX}G  & 0 \\
0 & \mc{B}
\end{matrix}
\right) \\
=&\left(
\begin{matrix}
e^{iG^TXG}  & 0 \\
0 & \mc{B}
\end{matrix}
\right) \quad \because G^T =G^{-1}\\
=&\left(
\begin{matrix}
\mathrm{diag}[e^{ix_1},\cdots e^{ix_N}]  & 0 \\
0 & \mc{B}
\end{matrix}
\right)
\end{align*}
と部分行列$\mc{A}$は対角化できる.これがブロック対角化における$1\times 1$の位相因子の部分だ.この$\mc{A}$の部分は(2.C.1)では位相因子を与えるのみだから,ここからは$(N-M)\times (N-M)$正方行列$\mc{B}$のみを考えればよい.\par

\vskip\baselineskip

残りの位相因子について考える.対角行列$D$の$e^{i\phi}=1$となる部分については完全に終わったので,$e^{i\phi}\neq 1$となる位相因子の中で一つ持ってきて,それを$e^{i\phi_1}$とする.同じ位相因子$e^{i\phi}=e^{i\phi_1}$を与える$e^{i\phi}$全てを対角成分の(1の後の)最初に持ってくる.その位相因子の個数を$L^+_1$個とする.次に$e^{i\phi_1}$と反対の位相$e^{i\phi}=e^{-i\phi_1}$を持つ位相因子を全て持ってくる.この位相因子は$L^-_1$個とする.その次に$e^{i\phi_1},e^{-i\phi_1}$のどちらとも異なる残りの位相因子から一つ選び,それを$e^{i\phi_2}$とする.同じ位相因子となるもの($L_2^+$個)と反対の位相因子となるもの($L^-_2$個)を並べる…というのを繰り返す.すると
\begin{align*}
\mc{B}=\left(
\begin{matrix}
\mc{B}_1 & 0             &\cdots \\
0            &  \mc{B}_2 &\cdots \\
\vdots &  \vdots     &\ddots 
\end{matrix}
\right) ,\quad \mc{B}_i=\left(
\begin{matrix}
0 & e^{i\phi_i/2}\mc{C}_i \\
e^{-\phi_i/2} \mc{C}_i^T & 0
\end{matrix}
\right)
\end{align*}
という形になる.なぜなら,$e^{i\phi_i}e^{i\phi_j}=1$となる場合以外はゼロになるのだったから,$i$番目の$e^{i\phi_i}$と同じ位相因子となる$L^+_i$個の$\mc{T}_{ij}(j=1,\cdots ,L^+_i)$はゼロになるから,$\mc{T}_{ii}=\mc{T}_{ij}=\mc{T}_{ji}=0(j=1,\cdots ,L^+_i)$となり,$\mc{T}_{ij}(i=2,\cdots ,L^+_i)$についてそれは同じだから,したがって$\mc{B}_i$の左上の$L^+_i\times L^+_i$行列はゼロ行列となる.同様に右下の$L^-_i\times L^-_i$行列についてもゼロ行列となる.一方,右上の$L^+_i\times L^-_i$行列についてはゼロではない.なぜなら$\mc{T}_{ij}$の$i$について$e^{i\phi_i}\neq 1$であって,仮定より$L^-_i$個の$j$について$e^{i\phi_j}=e^{-i\phi_i}$であるから,$\mc{T}_{ij}\neq 0$である.$L^+_i$個の$i$についてこれが成り立つから,$\mc{B}_i$の右上の$L^+_i \times L^-_i$行列はある非ゼロな行列$e^{i\phi_i/2}\mc{C}_i$と書ける(すぐ後の便利のため位相因子を抜き出しておく).左下の部分行列についても同様のことが言えて,$L^-_i\times L^+_i$行列$\mc{D}_i$と書けて
\begin{align*}
\mc{B}_i=\left(
\begin{matrix}
0 & e^{i\phi_i/2}\mc{C}_i \\
\mc{D}_i & 0
\end{matrix}
\right)
\end{align*}
という形に書ける.しかし$e^{i\phi}e^{i\phi_j}=1$を満たす組$i,j$について
\begin{align*}
\mc{T}_{ji}=&e^{i\phi_j}\mc{T}_{ij}=e^{-i\phi_i}\mc{T}_{ij} \\
(\mc{D}_i)_{nm}=&e^{-i\phi_i}(e^{i\phi_i/2}\mc{C}_i)_{mn}=e^{-i\phi_i/2}(\mc{C}_i^T)_{nm} \\
\therefore \quad \mc{D}_i=&e^{i\phi_i/2}\mc{C}_i^T
\end{align*}
がわかる.これで示すことができた.\par
さらに$\mc{T}$のユニタリー性は$\mc{B}$のユニタリー性を導き,それにより$\mc{B}_i$のユニタリー性を導く.よって
\begin{align*}
1=&\mc{B}_i^\dagger \mc{B}_i \\
=&\left(
\begin{matrix}
0 & e^{i\phi_i/2}\mc{C}_i \\
e^{-i\phi_i/2}\mc{C}_i^T & 0
\end{matrix}
\right)\left(
\begin{matrix}
0 &  e^{i\phi_i/2}\mc{C}_i^*\\
e^{-i\phi_i/2}\mc{C}_i^\dagger & 0
\end{matrix}
\right) \\
=&\left(
\begin{matrix}
\mc{C}_i\mc{C}^\dagger_i  & 0 \\
0 & \mc{C}_i^T \mc{C}^*_i
\end{matrix}
\right) \\
\Leftrightarrow \quad 1=&\mc{C}_i\mc{C}^\dagger_i ,\quad 1=\mc{C}_i^\dagger \mc{C}_i
\end{align*}
左上の$1=\mc{C}_i\mc{C}^\dagger_i$は$L^+_i \times L^+_i$正方行列であり,右下の$1=\mc{C}^\dagger_i \mc{C}_i$は$L^-_i \times L^-_i$正方行列となっている.行列のrankに関して
\begin{align*}
\mathrm{rank}(AB) \leq \mathrm{rank}(A)
\end{align*}
がなりたつから,前者から
\begin{align*}
\mathrm{rank}(C_iC^\dagger_i)=L^+_i \leq \mathrm{rank}(C^\dagger_i) =L^-_i
\end{align*}
がわかり,後者から
\begin{align*}
\mathrm{rank}(C_i^\dagger C_i)=L^-_i \leq \mathrm{rank}(C_i) =L^+_i
\end{align*}
がわかる.したがって$L^+_i=L^-_i=L_i$が導かれる.したがって$\mc{C}_i$は$L_i\times L_i$正方行列かつユニタリー行列であることがわかる.(同時に,$e^{i\phi_i}\neq 1$を与える位相因子に対して$e^{i\phi_j}=e^{-i\phi_i}$を与える位相因子は同数だけ存在することがわかる.)$\mc{U}$が$\mc{T}$と同じ意味でブロック対角であり
\begin{align*}
\mc{U}=&\left(
\begin{matrix}
1 & 0 \\
0 & \mathrm{diag}[\mc{U}_i]
\end{matrix}
\right)
\end{align*}
となっていて,$i$番目ブロックが任意のユニタリー行列$V_i,W_i$を使って
\begin{align*}
\mc{U}_i=\left(
\begin{matrix}
V_i & 0 \\
0 & W_i
\end{matrix}
\right)
\end{align*}
となるようにすれば,この$\mc{U}$による変換(2.C.2)は
\begin{align*}
\mc{T}'=&\mc{U}^{-1} \mc{T} \mc{U} \\
=&\left(
\begin{matrix}
1 & 0 \\
0 & \mathrm{diag}[\mc{U}_i]^{-1}
\end{matrix}
\right)\left(
\begin{matrix}
\mathrm{diag}[e^{ix_1},\cdots e^{ix_N}]  & 0 \\
0 & \mathrm{diag}[\mc{B}_i]
\end{matrix}
\right)\left(
\begin{matrix}
1 & 0 \\
0 & \mathrm{diag}[\mc{U}_i]^*
\end{matrix}
\right) \\
=&\left(
\begin{matrix}
\mathrm{diag}[e^{ix_1},\cdots e^{ix_N}]  & 0 \\
0 & \mathrm{diag}[\mc{U}^{-1}_i\mc{B}_i \mc{U}_i^*]
\end{matrix}
\right) \\
\mc{U}^{-1}_i\mc{B}_i \mc{U}_i^*=&\left(
\begin{matrix}
V_i^{-1} & 0 \\
0 & W_i^{-1}
\end{matrix}
\right)\left(
\begin{matrix}
0 & e^{i\phi_i/2}\mc{C}_i \\
e^{-\phi_i/2} \mc{C}_i^T & 0
\end{matrix}
\right)\left(
\begin{matrix}
V_i^* & 0 \\
0 & W_i^*
\end{matrix}
\right) \\
=&\left(
\begin{matrix}
0 & e^{i\phi_i/2}V^{-1}_i \mc{C}_i W^*_i \\
e^{-\phi_i/2} W^{-1}\mc{C}_i^TV^*_i & 0
\end{matrix}
\right) \\
=&\left(
\begin{matrix}
0 & e^{i\phi_i/2}V^{-1}_i \mc{C}_i W^*_i \\
e^{-\phi_i/2} [V^{-1}\mc{C}_iW^*_i]^T & 0
\end{matrix}
\right)
\end{align*}
となり,部分行列の変換$\mc{C}_i \to V^{-1}_i\mc{C}_i W^*_i$が対応していることが分かる.$V_i=\mc{C}_i,W_i=1$と選べば$\mc{C}'_i=1$となるようにできる.さらに実直交行列$\mc{P}_{nm}$を繰り返しかけて作った$\mc{P}_i$を用いて
\begin{align*}
\mc{T} \to \mc{T}` =&\mc{P}^{-1} \mc{T}\mc{P}^* \\
\mc{P}=&\left(
\begin{matrix}
1 & 0 \\
0 & \mathrm{diag}[\mc{P}_i]
\end{matrix}
\right)
\end{align*}
と変換してやれば,この$\mc{T}'$は$1\times 1$の位相因子の対角部分と,$2\times 2$の
\begin{align*}
\left(
\begin{matrix}
0 & e^{i\phi} \\
e^{-i\phi} & 0
\end{matrix}
\right)
\end{align*}
のブロック対角の部分に分かれることが分かる.$\mc{P}_i$は,列の置換
\begin{align*}
(1,2,\cdots ,L_i,L_i+1,\cdots ,2L_i)\to(1,L_i+1,2,L_i+2,3,\cdots ,L_i,2L_i)
\end{align*}
を考えて作る.$L_i=3$あたりで実験して確かめよう.$L_i=3$では
\begin{align*}
&(1,2,3,4,5,6) \\
\to&(1,4,3,2,5,6) \quad 2,4個目の入れ替え \\
\to &(1,4,5,2,3,6) \quad 3,5個目の入れ替え\\
\to &(1,4,2,5,3,6) \quad 3,4個目の入れ替え
\end{align*}
を考えればいいので,$\mc{P}=\mc{P}_{2,4}\mc{P}_{3,5}\mc{P}_{3,4}$と構成すればいいことがわかる.実際
\begin{align*}
\mc{C}_i=&\left(
\begin{matrix}
0             & 0            & 0             & e^{i\phi}   & 0            &  0         \\
0             &           0  & 0              &  0           & e^{i\phi}  & 0           \\
0             &            0 & 0              & 0            &    0         & e^{i\phi} \\
e^{-i\phi} & 0             & 0             & 0            & 0            & 0           \\
0             & e^{-i\phi} & 0             & 0            & 0            & 0           \\
0             & 0             & e^{-i\phi} & 0            & 0            & 0
\end{matrix}
\right) \\
\longrightarrow \mc{P}_{2,4}^{-1} \mc{C}_i \mc{P}_{2,4}^*=&\left(
\begin{matrix}
0             & e^{i\phi} & 0               &  0              & 0            &  0         \\
e^{-i\phi} &           0  & 0              &  0           & 0              & 0           \\
0             &            0 & 0              & 0            &    0         & e^{i\phi} \\
0             & 0             & 0             & 0            & e^{i\phi}            & 0           \\
0             & 0              & 0             & e^{-i\phi} & 0            & 0           \\
0             & 0             & e^{-i\phi} & 0            & 0            & 0
\end{matrix}
\right) \\
\longrightarrow \mc{P}_{3,5}^{-1}\mc{P}_{2,4}^{-1} \mc{C}_i \mc{P}_{2,4}^*\mc{P}_{3,5}^*=&\left(
\begin{matrix}
0             & e^{i\phi} & 0               &  0              & 0            &  0         \\
e^{-i\phi} &           0  & 0              &  0           & 0              & 0           \\
0             &            0 & 0              & e^{-i\phi}   &    0         & 0              \\
0             & 0             & e^{i\phi}   & 0            & 0            & 0           \\
0             & 0              & 0             & 0              & 0           & e^{i\phi} \\
0             & 0             & 0              & 0            & e^{-i\phi}  & 0
\end{matrix}
\right) \\
\longrightarrow \mc{P}^{-1}_{3,4}\mc{P}_{3,5}^{-1}\mc{P}_{2,4}^{-1} \mc{C}_i \mc{P}_{2,4}^*\mc{P}_{3,5}^*\mc{P}_{3,4}^*=&\left(
\begin{matrix}
0             & e^{i\phi} & 0               &  0              & 0            &  0         \\
e^{-i\phi} &           0  & 0              &  0           & 0              & 0           \\
0             &            0 & 0              & e^{i\phi}   &    0         & 0              \\
0             & 0             & e^{-i\phi}   & 0            & 0            & 0           \\
0             & 0              & 0             & 0              & 0           & e^{i\phi} \\
0             & 0             & 0              & 0            & e^{-i\phi}  & 0
\end{matrix}
\right) 
\end{align*}
という変換をしてやればよい.これで証明が完了した.

\vskip\baselineskip


ここで,(2.C.3)のブロック対角行列で$e^{i\phi}\neq 1$となる場合,時間反転を対角化するように状態を選ぶことは不可能であることがわかる.この部分からは,状態の対$\Psi_{\mathbf{p},\sigma,\pm}$があり,それに$\mathsf{T}$が行列(2.C.3)で作用するときには
\begin{align*}
\mathsf{T}\Psi_{\mathbf{p},\sigma,\pm}=e^{\pm i\phi/2}(-1)^{j-\sigma}\Psi_{-\mathbf{p},-\sigma,\mp}
\end{align*}
となる.したがって,これらの状態の任意の一次結合に対して,時間反転は
\begin{align*}
\mathsf{T}(c_+ \Psi_{\mathbf{p},\sigma,+}+c_- \Psi_{\mathbf{p},\sigma,-}) \\
=&(-1)^{j-\sigma}\Bigl(e^{i\phi/2}c_+^* \Psi_{-\mathbf{p},-\sigma,-}+e^{-i\phi/2}c^*_- \Psi_{-\mathbf{p},-\sigma,+}\Bigr)
\end{align*}
と作用する.($c_\pm$の複素共役は$\mathsf{T}$の反線形性によるもの.)したがってこの一次結合が$\mathsf{T}$のもとで位相$\lambda$で変換される
\begin{align*}
\mathsf{T}(c_+ \Psi_{\mathbf{p},\sigma,+}+c_- \Psi_{\mathbf{p},\sigma,-})=\lambda(c_+ \Psi_{\mathbf{p},\sigma,+}+c_- \Psi_{\mathbf{p},\sigma,-})
\end{align*}
ためには,
\begin{align*}
e^{i\phi/2}c_+^*=\lambda c_-,\quad e^{-i\phi/2}c_-^*=\lambda c_+
\end{align*}
とならなければならない.しかし,これらの式を組み合わせると$e^{\pm i\phi/2}c_\pm^*=|\lambda|^2 c_\pm^* e^{\mp i\phi/2}$となる.これは自明な結合$c_+=c_-=0$か$e^{i\phi}=1$がなりたっていなければ不可能だ.したがって,$e^{i\phi}\neq1$のときには,時間反転不変性はこれらの状態に,スピンに関する縮退の他に,2重の縮退を必ず要請する.\par
もし余分な内部対称性演算子$\mathsf{S}$がこれらの状態を
\begin{align*}
\mathsf{S}\Psi_{\mathbf{p},\sigma,\pm}=e^{\pm i\phi/2}\Psi_{-\mathbf{p},\sigma,\mp}
\end{align*}
と変換するならば,時間反転演算子を新しく$\mathsf{T}' := \mathsf{S}^{-1} \mathsf{T}$と再定義し,この演算子が$\Psi_{\mathbf{p},\sigma,\pm}$を互いに混ぜないようにできる.したがって,このような内部対称性がないときにのみ,粒子状態の二重性は時間反転不変性で説明できる.

\vskip\baselineskip

$\mathsf{T}^2$の問題.変換(2.C.8)を繰り返すと
\begin{align*}
\mathsf{T}^2 \Psi_{\mathbf{p},\sigma,\pm}=(-1)^{2j}e^{\mp i\phi} \Psi_{\mathbf{p},\sigma,\pm}
\end{align*}
となる.位相$\phi$によって,ここに現れる因子は$1$になったり1でない位相因子になったりする.\par

\vskip\baselineskip


パリティ演算子$\mathsf{P}$の場合でも,同様に複雑な作用を考えることができる.
\begin{align*}
\mathsf{P}\Psi_{\mathbf{p},\sigma,n}=\sum_m \mc{P}_{nm}\Psi_{-\mathbf{p},\sigma,m}
\end{align*}
しかしこの場合$\mc{P}$はユニタリーだが,$\mathsf{P}$は今度は反線形性がないので通常通り$\Psi_{\mathbf{p},\sigma,n}$のユニタリーな線形結合で新しい基底を選べばいつでも$\mc{P}$を対角化できる.ただし,この基底の選び方が,上で$\mc{U}$による時間反転の簡単な作用をするように選んだ基底の選び方と異なっている場合がある.したがって原則的には$\mathsf{P},\mathsf{T}$の両方が簡単に作用するものはほぼなく,両方を同時に使うと余分に縮退が起きるらしい.

\vskip\baselineskip


どんな場の量子論も$\mathsf{CPT}$という対称性を満たすと考えられている.これは1粒子状態に以下のような作用をする.
\begin{align*}
\mathsf{CPT}\Psi_{\mathbf{p},\sigma,n}=(-1)^{j-\sigma}\Psi_{\mathbf{p},-\sigma,n^c}
\end{align*}
ここで$n^c$は粒子$n$の反粒子(荷電共役)を意味する.この変換ではどんな位相も行列も許されない.これにより
\begin{align*}
(\mathsf{CPT})^2\Psi_{\mathbf{p},\sigma,n}=(-1)^{2j}\Psi_{\mathbf{p},-\sigma,n}
\end{align*}
となる.(5章を書いてからここらへん更新予定.)







\newpage







\part{散乱理論}
\setcounter{section}{3}
\setcounter{subsection}{0}
\subsection{In状態とout状態}
1粒子状態$\Psi_{p,\sigma,n}$全体の張るヒルベルト空間$\mc{H}_1$
\begin{align*}
\mc{H}_1\ni \int d^3\mathbf{p} f_{\sigma,n}(p) \Psi_{p,\sigma,n}
\end{align*}
を用いて,$N$個の\uwave{相互作用していない}粒子からなる多粒子状態を考えよう.ここで1粒子状態を表示するのに,その4元運動量$p^\mu$,スピンの$z$成分(質量ゼロ粒子の場合ではヘリシティ)$\sigma$,2種類以上の粒子を取り扱う場合にはさらに質量・スピン・電荷等を含めた粒子のタイプを表す添え字$n$を用いる.$N$個のテンソル積の「(反)対称化」によって,$N$粒子状態を表す状態空間
\begin{align*}
\mc{H}_N:=\mathrm{Sym}_{\pm}\otimes^N \mc{H}_1 \ni \Psi= \int d^3\mathbf{p}_1 \cdots d^3\mathbf{p}_N f_{\sigma_1,n_1;\cdots ;\sigma_N,n_N}(p_1,\cdots ,p_N)\Psi_{p_1,\sigma_1,n_1;\cdots ;p_N,\sigma_N,n_N}
\end{align*}
が得られる.$N$粒子状態はその基底$\Psi_{p_1,\sigma_1,n_1;\cdots ;p_N,\sigma_N,n_N}$で定義される.すなわち相互作用していない粒子からなる多粒子状態は,非斉次ローレンツ群のもとで1粒子状態のテンソル積として変換するものとみなせる.全系の状態空間$\mc{H}$は
\begin{align*}
\mc{H}=\bigoplus_{N=0}^\infty \mc{H}_N
\end{align*}
で定義される.ポアンカレ群による変換則は1粒子状態への変換則から一意的に定義される.一般の変換則は,
\begin{align*}
U(\Lambda,a)=\exp(-ia_\mu P^\mu) U(\Lambda)
\end{align*}
と(2.5.23)を用いて
\begin{align*}
&U(\Lambda ,a)\Psi_{p_1,\sigma_1,n_1;p_2,\sigma_2,n_2\cdots } \\
=&e^{-ia_\mu P^\mu }U(\Lambda) \Psi_{p_1,\sigma_1,n_1;p_2,\sigma_2,n_2\cdots } \\
=&\prod_i \left[e^{-ia_\mu (\Lambda p_i)^\mu}\sqrt{\frac{(\Lambda p_i)^0}{p_i^0}} \sum_{\sigma'_i}D_{\sigma'_i\sigma_i}^{(j_i)}\Bigl(W(\Lambda,p_i)\Bigr)\right]\Psi_{\Lambda p_1,\sigma_1',n_1;\Lambda p_2,\sigma'_2,n_2\cdots } \\
=&\exp\left[-ia_\mu ((\Lambda p_1)^\mu+(\Lambda p_2)^\mu)+\cdots \right] \sqrt{\frac{(\Lambda p_1)^0(\Lambda p_2)^0\cdots }{p^0_1p_2^0\cdots}} \\
&\times \sum_{\sigma_1,\sigma_2\cdots }D^{(j_1)}_{\sigma'_1\sigma_1}\Bigl(W(\Lambda,p_1)\Bigr)D^{(j_2)}_{\sigma'_2\sigma_2}\Bigl(W(\Lambda,p_2)\Bigr)\cdots \Psi_{\Lambda p_1,\sigma_1',n_1;\Lambda p_2,\sigma'_2,n_2\cdots }
\end{align*}
ここで$W(\Lambda,p)$はウィグナーの回転(2.5.10),$D^{(j)}_{\sigma',\sigma}$は,質量があれば3次元回転群の表現である通常の$(2j+1)$次元ユニタリー行列,質量ゼロなら$D^{(j)}_{\sigma'\sigma}=\delta_{\sigma'\sigma}\exp(i\sigma\theta(\Lambda,p))$となる.ここで$\theta(\Lambda,p)$は(2.5.43)で定義される角度だ.状態は(2.5.19)のように規格化される.
\begin{align*}
&(\Psi_{p_1',\sigma'_1,n'_1;p'_2,\sigma'_2,n'_2\cdots },\Psi_{p_1,\sigma_1,n_1;p_2,\sigma_2,n_2\cdots }) \\
=&\delta^3(\mathbf{p}'_1-\mathbf{p}_1)\delta_{\sigma'_1\sigma_1}\delta_{n_1'n_1}\delta^3(\mathbf{p}'_2-\mathbf{p}_2)\delta_{\sigma'_2\sigma_2}\delta_{n'_2n_2}\cdots \\
&\pm[置換]
\end{align*}
ここで$\pm[置換]$の項は,粒子のタイプ$n_1',n_2',\cdots$の入れ替えが粒子のタイプ$n_1,n_2$と同じになる可能性を考慮して付けてある.4章でより完全に述べられるらしいが,その符号はこの入れ替えが半整数スピン粒子の奇数回の入れ替えで$-1$,それ以外では$+1$となる.(例えば$n_1,n_2$が共に電子であり,$n_1',n_2'$も共に電子であるような場合
\begin{align*}
(\Psi_{p'_1,\sigma_1';p'_2,\sigma_2'},\Psi_{p_1,\sigma_1;p_2,\sigma_2})=&\delta^3(\mathbf{p}_1'-\mathbf{p}_1)\delta_{\sigma'_1\sigma_1}\delta^3(\mathbf{p}'_2-\mathbf{p}_2)\delta_{\sigma'_2\sigma_2} \\
&-\delta^3(\mathbf{p}_2'-\mathbf{p}_1)\delta_{\sigma'_2\sigma_1}\delta^3(\mathbf{p}'_1-\mathbf{p}_2)\delta_{\sigma'_1\sigma_2}
\end{align*}
となる.$\Psi_{p_1,\sigma_1,n_1;p_2,\sigma_2,n_2;\cdots}$も組$(p_i,\sigma_i,n_i)$と$(p_j,\sigma_j,n_j)$の組の入れ替えがフェルミオン同士だったらマイナスが出るとする.)これはこの章の内容では重要ではない.

\vskip\baselineskip

一つのギリシャ文字,例えば$\alpha$が$p_1,\sigma_1,n_1;p_2,\sigma_2,n_2;\cdots$全体を表す省略記法をしばしば用いる.
\begin{align*}
\Psi_{\alpha}=\Psi_{p_1,\sigma_1,n_1;p_2,\sigma_2,n_2\cdots }
\end{align*}
この記法では特に規格化条件(3.1.2)は
\begin{align*}
(\Psi_{\alpha'},\Psi_\alpha)=\delta(\alpha'-\alpha)
\end{align*}
と書かれる.右辺のデルタ関数は,離散的な添え字(スピン添え字$\sigma$と粒子のタイプ$n$)についてはクロネッカーのデルタであり,連続的な運動量添え字についてはデルタ関数として扱う.また状態和をとるとき
\begin{align*}
\int d\alpha := \sum_{n_1\sigma_1 n_2 \sigma_2\cdots }\frac{1}{\prod_i N_i!} \int d^3\mathbf{p}_1 d^3\mathbf{p}_2\cdots 
\end{align*}
と書く.つまり離散的な添え字($\sigma,n$)については和をとり,連続的な運動量添え字については積分する.ただし,和の中で$n_i$と種類が同じ粒子が二種類以上入っているときはその数を$N_i$として,$N_i !$で割る必要がある(例えば電子2個と陽電子1個の3粒子状態の項については係数に$1/2!1!$が出てくる.重複が無いように注意).特に,(3.1.3)で規格化された状態の完全性は
\begin{align*}
\Psi=\int d\alpha \Psi_\alpha (\Psi_\alpha, \Psi)
\end{align*}
と表示できる.実際,$\Psi_\alpha$が完全性をなすとすれば
\begin{align*}
\Psi=& \int d\alpha \Psi_{\alpha} f_\alpha \\
=&\sum_{n_1\sigma_1 n_2 \sigma_2\cdots }\frac{1}{\prod_i N_i!}\int d^3\mathbf{p}_1 d^3\mathbf{p}_2\cdots \Psi_{p_1,\sigma_1,n_1;p_2,\sigma_2,n_2,\cdots } f_{p_1,\sigma_1,n_1;p_2,\sigma_2,n_2\cdots }
\end{align*}
として展開できるが,係数$f_\alpha$を得るためには,両辺を$\Psi_{\beta}$と内積することによって得られて
\begin{align*}
&(\Psi_{p'_1,\sigma'_1,n'_1;p'_2\sigma'_2,n'_2,\cdots} ,\Psi) \\
=&\sum_{n_1\sigma_1 n_2 \sigma_2\cdots }\frac{1}{\prod_i N_i !}\int d^3\mathbf{p}_1 d^3\mathbf{p}_2\cdots (\Psi_{p'_1,\sigma'_1,n'_1;p'_2\sigma'_2,n'_2,\cdots},\Psi_{p_1,\sigma_1,n_1;p_2,\sigma_2,n_2\cdots }) f_{p_1,\sigma_1,n_1;p_2,\sigma_2,n_2\cdots } \\
=&\sum_{n_1\sigma_1 n_2 \sigma_2\cdots }\frac{1}{\prod_i N_i !}\int d^3\mathbf{p}_1 d^3\mathbf{p}_2\cdots \Bigl[\delta^3(\mathbf{p}'_1-\mathbf{p}_1)\delta_{\sigma'_1\sigma_1}\delta_{n_1'n_1}\delta^3(\mathbf{p}'_2-\mathbf{p}_2)\delta_{\sigma'_2\sigma_2}\delta_{n'_2n_2}\cdots \pm[置換]\Bigr] \\
&\qquad \qquad \qquad \qquad \qquad \qquad \times  f_{p_1,\sigma_1,n_1;p_2,\sigma_2,n_2\cdots } \\
=&\frac{1}{\prod_i N_i!}\left[f_{p'_1,\sigma'_1,n'_1;p'_2\sigma'_2,n'_2,\cdots}\pm [置換]\right]
\end{align*}
となる.置換の総数は$\prod_i N_i!$個である.これを完全性の展開に用いると
\begin{align*}
\Psi=&\sum_{n_1\sigma_1 n_2 \sigma_2\cdots }\frac{1}{\prod_i N_i!}\int d^3\mathbf{p}_1 d^3\mathbf{p}_2\cdots \Psi_{p_1,\sigma_1,n_1;p_2,\sigma_2,n_2,\cdots } f_{p_1,\sigma_1,n_1;p_2,\sigma_2,n_2\cdots } \\
=&\sum_{n_1\sigma_1 n_2 \sigma_2\cdots }\frac{1}{\prod_i N_i!}\int d^3\mathbf{p}_1 d^3\mathbf{p}_2\cdots \Psi_{p_1,\sigma_1,n_1;p_2,\sigma_2,n_2,\cdots } \frac{1}{\prod_j N_j!}\left[f_{p_1,\sigma_1,n_1;p_2\sigma_2,n_2,\cdots}\times \prod_j N_j! \right] \\
=&\sum_{n_1\sigma_1 n_2 \sigma_2\cdots }\frac{1}{\prod_i N_i!}\int d^3\mathbf{p}_1 d^3\mathbf{p}_2\cdots \Psi_{p_1,\sigma_1,n_1;p_2,\sigma_2,n_2,\cdots } \frac{1}{\prod_j N_j!}\left[f_{p_1,\sigma_1,n_1;p_2\sigma_2,n_2,\cdots}\pm [置換]\right] \\
=&\sum_{n_1\sigma_1 n_2 \sigma_2\cdots }\frac{1}{\prod_i N_i!}\int d^3\mathbf{p}_1 d^3\mathbf{p}_2\cdots \Psi_{p_1,\sigma_1,n_1;p_2,\sigma_2,n_2,\cdots }(\Psi_{p_1,\sigma_1,n_1;p_2\sigma_2,n_2,\cdots},\Psi) \\
=&\int d\alpha \Psi_\alpha(\Psi_\alpha,\Psi)
\end{align*}
となり示すことができる.3個目の等号では,$(p_1,\sigma_1,n_1),(p_2,\sigma_2,n_2),\cdots$の添え字の組についての和と積分を入れ替え,その後$\Psi_{\psi_1,\sigma_1,n_1;\cdots}$の添え字を入れ替えるとフェルミオン同士の置換でマイナスが出てきて,それが$f_{p_1,\sigma_1,n_1,\cdots}$の添え字についての置換でのマイナスと一致することを用いた.電子が2個の2粒子状態についての完全性で具体的に確認するとわかりやすい.\par
(ちなみに,peskinなどで採用されるような規格化条件
\begin{align*}
(\Psi_{p',\sigma',n'},\Psi_{p,\sigma,n})=2p^0(2\pi)^3\delta^3(\mathbf{p}'-\mathbf{p})\delta_{\sigma'\sigma}\delta_{n'n}
\end{align*}
を採用すると,これで再び完全性が$\Psi=\int d\alpha \Psi_\alpha(\Psi_\alpha ,\Psi)$の形になるためには積分測度$d\alpha$の定義も変更する必要があり
\begin{align*}
\int d\alpha=\sum_{n_1\sigma_1 n_2 \sigma_2\cdots }\frac{1}{\prod_i N_i!}\int \frac{d^3\mathbf{p}_1}{(2\pi)^3 2p^0_1} \frac{d^3\mathbf{p}_2}{(2\pi)^32p^0_2}\cdots 
\end{align*}
となる.こっちの積分測度は(2.5.15)から明確にローレンツ不変である.)

\vskip\baselineskip

変換則(3.1.1)は粒子が相互作用していない場合にのみに可能となる.$\tensor{\Lambda}{^\mu_\nu}=\delta^\mu_\nu$および$a^\mu=(0,0,0,\tau)$とおくと
\begin{align*}
U(1,a)=&\exp(-ia_\mu P^\mu) \\
=&\exp(iH\tau) \quad \because (2.4.26)
\end{align*}
となり,(2.5.1)より
\begin{align*}
H\Psi_\alpha=&(p^0_1+p^0_2 +\cdots)\Psi_\alpha \\
=&E_\alpha \Psi_\alpha
\end{align*}
よって$\Psi_\alpha$はハミルトニアン$H$のエネルギー固有値
\begin{align*}
E_\alpha=p^0_1+p^0_2+\cdots
\end{align*}
のエネルギー固有状態となっている.このエネルギーは1粒子のエネルギーの和であり,相互作用項(すなわち同時に2粒子以上の相互作用によって生まれるエネルギーの項(クーロンエネルギー等))はない.\par

\vskip\baselineskip

他方,変換則(3.1.1)は散乱過程の時刻$t\to \pm \infty$で成り立つ.典型的な散乱実験は,時刻$t\to -\infty$で粒子が十分離れていてまだ相互作用していない状態から出発し,$t\to +\infty$で粒子が十分離れて相互作用しなくなる状態で終わる.だから(3.1.1)のように変換する状態は1つではなく\uwave{2つ}存在する.すなわち,「in」状態$\Psi^+_\alpha$と「out」状態である.それぞれの状態は,$\tensor{\Psi}{_\alpha^+}$は$t\to -\infty$で,$\tensor{\Psi}{_\alpha^-}$は$t\to +\infty$で観測することにより,添え字$\alpha=(p_1,\sigma_1,n_1)(p_2,\sigma_2,n_2)\cdots$で表される自由粒子を含む状態を見出す.\par
定義の枠組みに注意する.ローレンツ不変性を保つため,ここでは状態ベクトルは時間とともに変化しない.つまり状態ベクトルは$\Psi$粒子系の全時空にわたる履歴を記述する.これはハイゼンベルグ表示として知られていて,状態は変わらないが演算子は時間と共に変化するという描像である.演算子が時間的に一定で,状態ベクトルが時間と共に変化する場合はシュレーディンガー表示である.したがってハイゼンベルグ表示でのin(out)状態$\tensor{\Psi}{_\alpha^\pm}$はシュレーディンガー表示の状態ベクトル$\Psi(t)$の$t \to \pm \infty$の極限,とは解釈しない.(イメージ的に,シュレーディンガー表示での状態ベクトルは,時間によって刻々と変化する映画のフィルムのようなもの.ハイゼンベルグ表示は映画全編が入ったDVD1枚のようなもの.後者では見たいシーン(時間$t$の物理量)はプレイヤー(時間変化する演算子)を通して選ぶ.伝われ.)\par
しかし,状態を定義する際に,観測者が系を観測する慣性系\footnote{同じ「系」という言葉が使われるが,前者はsystemの意であり,後者の慣性系はframeの意.}が暗に洗濯されている.すなわち,異なる観測者$\mc{O},\mc{O}'$は等価な状態ベクトルを見るのであり,同じ状態ベクトルを見るわけではない.2.2節の最初に述べた通り,$\mc{O}$が射線$\mc{R}$の状態$\Psi$を観測したとすると,異なる観測者$\mc{O}'$は等価だが異なる射線$\mc{R}'$の状態$\Psi'$を観測する.そしてその二つは対称性変換$T:\mc{R}\to \mc{R}'$およびユニタリー演算子$U(T):\Psi \to \Psi'$で結びついているのだった.もし観測者$\mc{O},\mc{O}'$が別の慣性系に属しているならば両者はポアンカレ対称性変換で結びついており,その観測者らが同じ系を見たとすると,その両者が観測する状態ベクトルは等価だが演算子$U(\Lambda,a)$で結びついており同じ状態ではない.\par
特に,標準的な観測者$\mc{O}$は衝突過程の\uwave{ある瞬間}\footnote{衝突の瞬間でなくてよい.衝突の3秒後みたいな時刻点でもよい.}の時刻が$t=0$となるように時計を合わせ,その観測者に対して静止している別の観測者$\mc{O}'$は時刻$t'=0$を$t=\tau$になるように時計を合わせるとする.つまり$t'=t-\tau$で関係している.この両者は$\tensor{\Lambda}{^\mu_\nu}=\delta^\mu_\nu,a^\mu=(0,0,0,-\tau)$によるポアンカレ変換
\begin{align*}
T(\Lambda,a)=T(1,-\tau):x^\mu \mapsto x'^\mu=\tensor{\Lambda}{^\mu_\nu}x^\nu+a^\mu (t\mapsto t'=t-\tau)
\end{align*}
で結びついている.したがって系が状態$\Psi$であることを$\mc{O}$が観測したとき,観測者$\mc{O}'$はその系が状態
\begin{align*}
U(1,-\tau)\Psi=\exp(-iH\tau)\Psi
\end{align*}
にあることを観測するだろう.\par
観測者$\mc{O}'$の存在自体は本質ではない.このようにして,我々が観測するであろう状態ベクトルが,時刻$\tau$だけ前後で状態がどうなっているかを調べるためには,($\mc{O}$がどのような基底ベクトルをとっているかにかかわらず)時間推進演算子$U(1,-\tau)=\exp(-iH\tau)$をかけてやれば知ることができる,ということが重要である.我々が既に定義した状態はただ二つだけの$\tensor{\Psi}{_\alpha^\pm}$だから,ここから前後の状態を調べるには,衝突のずっと後の状態は$\exp(-iH\tau)\tensor{\Psi}{_\alpha^+}$で$\tau\to +\infty$をとることで,ずっと前の状態は$\exp(-iH\tau)\tensor{\Psi}{_\alpha^-}$で$\tau \to -\infty$をとることで得ることができる.しかし,状態が実際にエネルギー固有状態ならば,その状態は真に時間に依存せず,演算子$\exp(-iH\tau)$は重要でない位相因子$\exp(-iE_\alpha \tau)$を与えるだけだ.

\vskip\baselineskip

さて,$\tensor{\Psi}{_\alpha^\pm}$は散乱前後の極限$\tau \to \mp \infty$で,粒子は互いに十分離れて相互作用をしていないと仮定した.この過程は暗に,空間的に局在化した状態である波束状態を考えていることになる.なぜなら,粒子の波動関数が空間全体に広がって重なり合っていると,互いに相互作用していない状況が保証されないからである.しかし,今考えている$\tensor{\Psi}{_\alpha^\pm}$は,それぞれの運動量$\mathbf{p}_1,\mathbf{p}_2,\cdots $は確定した運動量であり,したがって量子力学における不確定性原理よりそれぞれの粒子は空間的に局在せず,空間全体に広がっていることになる.だとすると,粒子は互いに空間的に必ず重なり合っており,どれだけ過去へ遡っても粒子間の相互作用を無視することができない.また,これらの状態は規格化もできない(例えば1粒子状態は$(\Psi_{p},\Psi_p)=\infty$となる.)から,数学的にきちんと取り扱おうとすると問題が生じる.\par
この問題を解決するために,空間的に局在化した波束を考えたい.そこで,単純に特定の運動量$\mathbf{p}_0$を持った1粒子状態を扱うのではなく,任意の関数$g(\mathbf{p};\mathbf{p}_0,\sigma)$を用いて重みづけていろいろな運動量$\mathbf{p}$を重ね合わせて,波束状態を
\begin{align*}
\sum_{\sigma}\int d^3\mathbf{p} g(\mathbf{p};\mathbf{p}_0,\sigma)\Psi_{p,\sigma}
\end{align*}
と定義する\footnote{このように定義した状態も$\Psi_{p,\sigma}$と同じ$-P^\mu P_\mu$の固有状態となっている.例えば質量非ゼロ状態ならば.$m^2$が固有値となり,質量ゼロ状態ならば$0$が固有値として出てくる.1粒子状態である.$g$の$\sigma$依存性は,初期状態が偏極している状態ならば$\delta_{\sigma'\sigma}$に比例するが,固定していなければ成分ごとに重みづけをする.}.このような1粒子状態は運動量の広がりを許し,したがって空間的に広がりが有限の広さとなり局在した状態を扱うことができる.これが上手くいくことは,非相対論的な量子力学でポテンシャルによる1粒子の単純な散乱を扱って初めて理解できると思う.そのため一旦その議論をしてから一般の多粒子系での相対論的量子力学へ移行する.

\vskip\baselineskip

非相対論的量子力学で考える.\footnote{この話はWeinberg「Lectures on Quantum Mechanics」の7.1節にほぼ依る.参照してくれと言いたいけど,流石にこの話をせずに本文の話が理解できるとは思えないので書きます.}質量$\mu$の非相対論的粒子がポテンシャル$V(\mathbf{x})$の中にいるとする.まずハミルトニアンは
\begin{align*}
H=H_0+V(\mathbf{x})
\end{align*}
である.$ H_0 =\hat{\mathbf{P}}^2 / 2 \mu$は運動エネルギー演算子である.状態ベクトル$\Psi_{\mathbf{p}}^+$は時間に依存しないベクトルである.上付き添え字$+$はやはり,測定が散乱される十分前に行われたとした場合に,この状態が散乱の中心から離れたところで運動量$\mathbf{p}$として観測されることを示す.非常に早い時期には粒子はポテンシャルが無視できる場所にいるから,そのエネルギーは$\mathbf{p}^2/2\mu$であり,この状態ベクトルはハミルトニアンの固有状態となる.
\begin{align*}
H \Psi_{\mathbf{p}}^+=&E(\mathbf{p})\Psi_{\mathbf{p}}^+ \\
=&\frac{\mathbf{p}^2}{2\mu}\Psi_{\mathbf{p}}^+
\end{align*}
この$\Psi_{\mathbf{p}}^+$の定義を解釈するには,上で話したように,エネルギーの広がりを持ち時間依存性の異なる状態ベクトルの重ね合わせ
\begin{align*}
\Psi_{g}(t):=&\int d^3 \mathbf{p} \, g(\mathbf{p})\exp(-iE(\mathbf{p})t) \Psi_{\mathbf{p}}^+ \\
=&\int d^3 \mathbf{p} \, g(\mathbf{p})\exp\left(-i\frac{\mathbf{p}^2}{2\mu}t\right) \Psi_{\mathbf{p}}^+
\end{align*}
を考える必要がある.$g(\mathbf{p})$は滑らかな関数であり,ある運動量$\mathbf{p}_0$にピークをもつ.状態ベクトル$\Psi_{\mathbf{p}}^+$は固有値$E(\mathbf{p})=\mathbf{p}^2/2\mu$をもつ方程式$H\Psi=E(\mathbf{p})\Psi$の特殊解で,さらに任意の十分滑らかな関数$g(\mathbf{p})$について$t\to -\infty$で
\begin{align*}
\Psi_g(t) \to& \int d^3\mathbf{p} \, g(\mathbf{p})\exp(-iE(\mathbf{p})t)\Phi_{\mathbf{p}} \\
=&\int d^3\mathbf{p} \, g(\mathbf{p})\exp\left(-i\frac{\mathbf{p}^2}{2\mu}t\right)\Phi_{\mathbf{p}}
\end{align*}
を満足する解だと定義できる.ここで$\Phi_{\mathbf{p}}$は3元運動量演算子$\hat{\mathbf{P}}$の規格直交化された固有ベクトルであり,その固有値は$\mathbf{p}$であるようなもの
\begin{align*}
\hat{\mathbf{P}}\Phi_{\mathbf{p}}=\mathbf{p}\Phi_{\mathbf{p}} ,\quad (\Phi_{\mathbf{p}'},\Phi_{\mathbf{p}})=\delta^3 (\mathbf{p}'-\mathbf{p})
\end{align*}
したがって$\Phi_{\mathbf{p}}$は($H$ではなく)$H_0$の固有ベクトルである.固有値は$E(\mathbf{p})=\mathbf{p}^2/2\mu$である.そうすると,規格化の条件$(\Psi_g,\Psi_g)=1$は,$t \to -\infty$で
\begin{align*}
(\Psi_g(t),\Psi_g(t))\to& \int d^3\mathbf{p}' \, g^*(\mathbf{p}')\exp(+iE(\mathbf{p}')t) \int d^3\mathbf{p} \, g(\mathbf{p})\exp(-iE(\mathbf{p})t) (\Phi_{\mathbf{p}'},\Phi_{\mathbf{p}}) \\
=&\int d^3\mathbf{p}' \, g^*(\mathbf{p}')\exp(+iE(\mathbf{p}')t) \int d^3\mathbf{p} \, g(\mathbf{p})\exp(-iE(\mathbf{p})t) \delta^3(\mathbf{p}'-\mathbf{p}) \\
=&\int d^3 \mathbf{p} |g(\mathbf{p})|^2
\end{align*}
と書けることから
\begin{align*}
\int d^3 \mathbf{p} |g(\mathbf{p})|^2=1
\end{align*}
と等価である.\par
位置の確定した状態$\Phi_{\mathbf{x}}$とのスカラー積を考える.普通の平面波の波動関数は運動量が確定していて,
\begin{align*}
(\Phi_{\mathbf{x}},\Phi_{\mathbf{p}})=(2\pi)^{-3/2} e^{i\mathbf{p}\cdot \mathbf{x}}
\end{align*}
の形をしている.これより$t\to -\infty$について
\begin{align*}
(\Phi_{\mathbf{x}},\Psi_g(t))\to (2\pi)^{-3/2}\int d^3\mathbf{p}\, g(\mathbf{p}) \exp\left(i\mathbf{p}\cdot \mathbf{x}-i\frac{\mathbf{p}^2}{2\mu}t \right)
\end{align*}
関数$g(\mathbf{p})$について簡単な例を
\begin{align*}
g(\mathbf{p})\propto& \exp\left( -\frac{\Delta_0^2 }{2}(\mathbf{p}-\mathbf{p}_0)^2 -i\frac{\mathbf{p}\cdot \mathbf{p}_0 t_0}{\mu}+\frac{it_0 \mathbf{p}^2}{2\mu} \right)
\end{align*}
ととる.ここで$t_0$は始まりの時間で,大きな負の値である.$\mathbf{p}_0$は入射運動量であり,$g(\mathbf{p})$がピークをとる値である.$\Delta_0$は定数である.指数関数の中の$t_0$に比例する項は,次の計算で分かる通り$\Delta_0$が時間$t=t_0$での座標空間での波動関数の広がりであるように選んである.これを代入すると,$t\to -\infty$で
\begin{align*}
(\Phi_{\mathbf{x}},\Psi_g(t))\propto& \int d^3\mathbf{p}\, \exp\left(-\frac{\Delta_0^2 }{2} (\mathbf{p}-\mathbf{p}_0)^2 -i\frac{\mathbf{p}^2}{2\mu}(t-t_0)+i\mathbf{p}\cdot \left(\mathbf{x}-\frac{\mathbf{p}_0 t_0}{\mu}\right) \right) \\
=&e^{-\Delta_0^2 \mathbf{p}^2_0/2}\int d^3\mathbf{p}\, \exp\left(-\frac{1}{2}\left[\Delta_0^2 + i\frac{(t-t_0)}{\mu}\right]\mathbf{p}^2 +\mathbf{p}\cdot \left[\Delta_0^2\mathbf{p}_0 +i\left(\mathbf{x}-\frac{\mathbf{p}_0 t_0}{\mu}\right)\right] \right) \\
\propto &\int d^3\mathbf{p}\, \exp\left(-\frac{1}{2}A\mathbf{p}^2 +\mathbf{p}\cdot \mathbf{B}\right) \quad A=\Delta_0^2 + i\frac{(t-t_0)}{\mu} ,\mathbf{B}=\Delta_0^2\mathbf{p}_0 +i\left(\mathbf{x}-\frac{\mathbf{p}_0 t_0}{\mu}\right) \\
=&\left(\frac{2\pi}{A}\right)^{3/2}\exp\left(\frac{\mathbf{B}^2}{2A}\right)
\end{align*}
これの絶対値をとると,空間的確率分布が得られる.計算すると
\begin{align*}
\Bigl|\Bigl(\Phi_{\mathbf{x}},\Psi_g(t)\Bigr)\Bigr|^2\propto& \frac{1}{|A|^3}\exp\left( \frac{\mathbf{B}^{2}}{2A} +\frac{\mathbf{B}^{*2}}{2A^*} \right) \\
=&\frac{1}{|A|^3}\exp\left( \frac{1}{2|A|^2}(\mathbf{B}^{2}A^* +\mathbf{B}^{*2}A )\right) \\
|A|^2=&\Delta_0^4 + \frac{(t-t_0)^2}{\mu^2}=\Delta_0^2\left(\Delta_0^2 + \frac{(t-t_0)^2}{\mu^2\Delta_0^2}\right) \\
A^*\mathbf{B}^2=&\left(\Delta^2_0 -i \frac{(t-t_0)}{\mu}\right)\left( (\Delta_0^2\mathbf{p}_0)^2+2i\Delta_0^2\mathbf{p}_0\cdot \left(\mathbf{x}-\frac{\mathbf{p}_0t_0}{\mu}\right) -\left(\mathbf{x}-\frac{\mathbf{p}_0t_0}{\mu}\right)^2  \right) \\
A^*\mathbf{B}^2+A\mathbf{B}^{*2}=&2\Delta^6_0\mathbf{p}_0^2-2\Delta_0^2\left(\mathbf{x}-\frac{\mathbf{p}_0t_0}{\mu}\right)^2+4\Delta_0^2\frac{(t-t_0)}{\mu}\mathbf{p}_0 \cdot \left(\mathbf{x}-\frac{\mathbf{p}_0t_0}{\mu}\right) \\
=&2\Delta_0^2\left[\Delta_0^2\left(\Delta_0^2+\frac{(t-t_0)^2}{\mu^2\Delta_0^2}\right)\mathbf{p}_0^2-\left(\mathbf{x}-\frac{\mathbf{p}_0}{\mu}t\right)^2\right] \\
\therefore \quad \Bigl|\Bigl(\Phi_{\mathbf{x}},\Psi_g(t)\Bigr)\Bigr|^2\propto& \Delta(t)^{-3} \exp\left(-\frac{1}{\Delta(t)^{2}}\left[\mathbf{x}-\frac{\mathbf{p}_0}{\mu}t\right]^2\right) \\
\Delta(t):=&\left(\Delta_0^2+\frac{(t-t_0)^2}{\mu^2\Delta_0^2}\right)^{-1/2}
\end{align*}
となる.したがって確率分布は,速度が運動量の平均$\mathbf{p}_0$を質量$\mu$で割った値を中心としており,それは$t=0$で散乱の中心$\mathbf{x}=0$に近づく.\par
この分布の広がりは,始まり$t=t_0$で$\Delta_0$であるが,$t>t_0$では増え始める.$t=t_0$から$t=0$までの時間内に波束が目に見えて広がらないためには
\begin{align*}
\Delta^2_0 >&\frac{(0-t_0)^2}{\mu^2 \Delta_0^2} \\
\therefore \quad& \Delta_0^2>\frac{|t_0|}{\mu} 
\end{align*}
を要請する.しかし,一方
\begin{align*}
\Delta_0 \ll \frac{|\mathbf{p}_0|}{\mu} |t_0|
\end{align*}
でなければならない.なぜなら,$t_0$は十分早くて,$t=t_0$で波束は散乱中心まで広がっていないことを要請しなければならないからである.右辺は$t=t_0$の粒子の位置から散乱中心までの距離を表しており,初期の波束の広がり$\Delta_0$が中心に及んでいないためにはこれが要請される.この二つの条件は
\begin{align*}
\Delta_0^2<&\frac{|t_0|}{\mu} \ll \frac{|\mathbf{p}_0|^2}{\mu^2} |t_0|^2 \\
\therefore \quad 1 \ll& \frac{\mathbf{p}_0^2 |t_0|}{\mu}
\end{align*}
ならば両立する.\par
今回,$E(\mathbf{p})=\mathbf{p}^2/2\mu$であるから,運動量の幅$\Delta p=1/\Delta_0$に対応して
\begin{align*}
\Delta E=\left. \frac{dE}{dp}\right|_{\mathbf{p}=\mathbf{p}_0} \Delta p=\frac{|\mathbf{p}_0|}{\mu}\frac{1}{\Delta_0}
\end{align*}
となる.したがって上の条件式と合わせて
\begin{align*}
|t_0|\ll 1/\Delta E
\end{align*}
が導かれる.したがって,最初の条件式へ戻ると,$g(\mathbf{p})$が$\Delta E$の幅に渡ってゼロでなく滑らかに変化するようなものだとして,
\begin{align*}
\Psi_g(t)=\exp(-iHt )\int d^3\mathbf{p} \, g(\mathbf{p})\Psi_{\mathbf{p}}^+=\int d^3\mathbf{p} \, g(\mathbf{p})e^{-iE(\mathbf{p})t}\Psi_{\mathbf{p}}^+
\end{align*}
が$t \ll -1/\Delta E$のときに対応する自由粒子状態の重ね合わせ
\begin{align*}
\int d^3\mathbf{p} \, g(\mathbf{p})e^{-iE(\mathbf{p})t}\Phi_{\mathbf{p}}=\exp(-iH_0 t)\int d^3\mathbf{p} \, g(\mathbf{p})\Phi_{\mathbf{p}}
\end{align*}
になるように$\Psi_{\mathbf{p}}^+$を定義する.(時間は次元のある量であるから,スケール依存性があり,したがってこの条件が必要となる.)


\vskip\baselineskip


相対論的な議論へ戻ろう.以上の議論を同様に多粒子状態であるin状態とout状態においても用いる.つまり重ね合わせ
\begin{align*}
&\sum_{n_1\sigma_1 n_2 \sigma_2\cdots }\frac{1}{\prod_i N_i!}\int d^3\mathbf{p}_1 d^3\mathbf{p}_2\cdots g(p_1,\sigma_1,n_1,;p_2,\sigma_2,n_2;\cdots )\Psi_{p_1,\sigma_1,n_1;p_2,\sigma_2,n_2,\cdots } \\
&=\int d\alpha g(\alpha)\tensor{\Psi}{_{\alpha}^{\pm}}
\end{align*}
を考える.$g(\alpha)$はそれぞれの運動量$\mathbf{p}_1,\mathbf{p}_2,\cdots$の滑らかな関数となっており,エネルギーに関してピーク$E_0=\sqrt{\mathbf{p}^2_1+m_1^2}+\sqrt{\mathbf{p}^2_2+m_2^2}+\cdots$から幅$\Delta E$で滑らかに変化する分布だとみなせる.このような波束に対して時間発展演算子を作用させた重ね合わせ
\begin{align*}
\exp(-iH\tau)\int d\alpha g(\alpha)\tensor{\Psi}{_{\alpha}^{\pm}}=\int d\alpha e^{-iE_\alpha \tau}g(\alpha)\tensor{\Psi}{_{\alpha}^{\pm}}
\end{align*}
がそれぞれ$\tau \ll -1/\Delta E$あるいは$\tau \gg 1/\Delta E$で対応する自由粒子状態の重ね合わせ
\begin{align*}
\int d\alpha e^{-iE_\alpha \tau}g(\alpha)\Phi_\alpha
\end{align*}
になるようにin状態とout状態を定義する.\par
時間推進演算子$H$は自由粒子ハミルトニアン$H_0$と相互作用部分$V$の二つに分解
\begin{align*}
H=H_0+V
\end{align*}
でき,$H_0$はハミルトニアン全体の固有状態$\Psi_\alpha^+$および$\Psi_{\alpha}^-$と同じに見える固有状態
\begin{align*}
H_0\Phi_{\alpha}=E_\alpha \Phi_\alpha \\
(\Phi_{\alpha'} ,\Phi_{\alpha})=\delta(\alpha'-\alpha)
\end{align*}
ここで$H_0$は全体のハミルトニアン$H$と同じスペクトル(連続的に添え字づけられた固有値)をもつと仮定している.このため,$H_0$に現れる質量は\uwave{実際に測定される物理的な質量}であり,$H$に現れる「裸」の質量項と必ずしも同じではない.つまりその間に差があれば,それは$H_0$ではなく相互作用項$V$に含まれなければならない.例えばスカラー場のラグランジアン密度を
\begin{align*}
\mc{L}=-\frac{1}{2}\partial_\mu \Phi_B \partial^\mu \Phi_B -\frac{1}{2}m_B^2 \Phi^2_B -V_B(\Phi_B)
\end{align*}
として,くりこまれたスカラー場と質量
\begin{align*}
\Phi_R:=Z^{-1/2}\Phi_B ,\quad m^2_R :=m^2_B +\delta m^2
\end{align*}
を導入すると
\begin{align*}
\mc{L}=&-\frac{1}{2}\partial_\mu \Phi_R \partial^\mu \Phi_R -\frac{1}{2}m^2_R \Phi^2_R \\
&-\frac{1}{2}(Z-1)\left[\partial_\mu \Phi_R \partial^\mu \Phi_R +m^2_R \Phi^2_R\right] +\frac{1}{2}Z\delta m^2 \Phi^2_R -V(\Phi) \quad (V(\Phi):=V_B(\sqrt{Z}\Phi_R)) \\
=&\mc{L}_0 +\mc{L}_1 \\
\mc{L}_0=&-\frac{1}{2}\partial_\mu \Phi_R \partial^\mu \Phi_R -\frac{1}{2}m^2_R \Phi^2_R \\
\mc{L}_1=&-\frac{1}{2}(Z-1)\left[\partial_\mu \Phi_R \partial^\mu \Phi_R +m^2_R \Phi^2_R\right] +\frac{1}{2}Z\delta m^2 \Phi^2_R -V(\Phi) \quad (V(\Phi):=V_B(\sqrt{Z}\Phi_R))
\end{align*}
となる.共役運動量は
\begin{align*}
\Pi_R:=\frac{\partial \mc{L}}{\partial \dot{\Phi}_R}=Z\dot{\Phi}_R=Z^{1/2}\dot{\Phi}_B
\end{align*}
で定められ,ハミルトニアン密度は
\begin{align*}
\mc{H}=&\Pi_R \dot{\Phi}_R -\mc{L} \\
=&\mc{H}_0 +\mc{V} \\
\mc{H}_0=&\frac{1}{2} \Pi^2_R +\frac{1}{2}(\nabla \Phi_R)^2 +\frac{1}{2}m_R^2\Phi^2_R \\
\mc{V}=&-\frac{1}{2}(1-Z^{-1})\Pi_R^2+\frac{1}{2}(Z-1)\left[\frac{1}{2}(\nabla \Phi_R)^2+\frac{1}{2}m_R^2 \Phi_R^2 \right]-\frac{1}{2}Z\delta m^2 \Phi_R^2 +V(\Phi_R)
\end{align*}
$m_R$が物理的な質量$m_P$であるくりこみ(on-shell scheme)を考えると,物理的質量と裸の質量の差$\delta m^2$は相互作用項$V$に含まれる.また,$H$のスペクトル中の任意の束縛状態は$H_0$の中にあたかも素粒子であるかのように入れておかねばならない.(クォークの束縛状態を,例えばメソンならばスカラー場,バリオンならばディラック場として入れておく,ということだと思う.)

\vskip\baselineskip

非相対論のときと同じように,in状態とout状態は$H_0$ではなく$H$の固有状態として
\begin{align*}
H\tensor{\Psi}{_\alpha^\pm}=E_\alpha \tensor{\Psi}{_\alpha^\pm}
\end{align*}
を満たし,かつそれらは$\tau \to -\infty$または$\tau \to +\infty$で,それぞれ次の条件
\begin{align*}
\int d\alpha e^{-iE_\alpha \tau}g(\alpha)\tensor{\Psi}{_\alpha^\pm} \to \int d\alpha e^{-iE_\alpha \tau} g(\alpha)\Phi_{\alpha}
\end{align*}
を満たすものだとして定義できる.\par
この条件は$\tau \to -\infty$または$\tau \to +\infty$でそれぞれ
\begin{align*}
\exp(-iH\tau)\int d\alpha g(\alpha)\tensor{\Psi}{_\alpha^\pm} \to \exp(-iH_0 \tau)\int d\alpha g(\alpha)\Phi_\alpha
\end{align*}
と書き直せる.これはin状態とout状態の公式として
\begin{align*}
\tensor{\Psi}{_\alpha^\pm} =&\lim_{\tau \to \mp \infty}\exp(+iH\tau)\exp(-iH_0 \tau)\Phi_\alpha \\
=&\Omega(\mp \infty)\Phi_\alpha \\
\Omega(\tau):=&\exp(+iH\tau)\exp(-iH_0 \tau)
\end{align*}
と書き直せる\footnote{クーロン散乱のような長距離相互作用によるポテンシャル散乱などを考えるとこれは厳密には正しくなく,その場合Dollardによって変更された$\lim_{t\to \mp \infty}e^{iHt}C(t)e^{-iH_0t}$という演算子を使う必要があるらしい.量子力学ではDollardによる論文を参照.場の量子論ではこれは「荷電粒子はphoton cloudをまとっており1粒子状態がFock状態として存在しない」という問題に関わってくるらしく,Kulish・Faddeevなどによる論文などを参照してほしい.}.ただし,$\Omega(\mp \infty)$はエネルギー固有状態の滑らかな重ね合わせにかかったときのみ極限$\tau$で意味のある結果を与えており,それ単体では意味がないことに留意する.

\vskip\baselineskip

定義(3.1.12)からただちに結論されるのは,in状態とout状態が自由粒子状態と全く同様に規格直交化されているということだ.これを見るには,(3.1.12)の左辺はユニタリー演算子$\exp(-iH\tau)$を時間に依存しない状態にかけて得られるので,そのノルム
\begin{align*}
\left(\int d\beta g(\beta)\tensor{\Psi}{_\beta^\pm},\int d\alpha g(\alpha)\tensor{\Psi}{_\alpha^\pm} \right) =&\left(\exp(-iH\tau)\int d\beta g(\beta)\tensor{\Psi}{_\beta^\pm},\exp(-iH\tau )\int d\alpha g(\alpha)\tensor{\Psi}{_\alpha^\pm} \right) \\
=&\left(\int d\beta e^{-iE_\beta \tau}g(\beta)\tensor{\Psi}{_\beta^\pm},\int d\alpha e^{-iE_\alpha\tau}g(\alpha)\tensor{\Psi}{_\alpha^\pm} \right) \\
=&\int d\alpha d\beta \exp(-i(E_\alpha -E_\beta)\tau) g(\alpha)g^*(\beta )(\tensor{\Psi}{_\beta^\pm},\tensor{\Psi}{_\alpha^\pm})
\end{align*}
は時間に依存せず,よって$\tau \to \pm\infty$の極限のノルム,すなわち(3.1.12)の右辺のノルム
\begin{align*}
\left(\int d\beta e^{-iE_\beta \tau}g(\beta)\tensor{\Psi}{_\beta^\pm},\int d\alpha e^{-iE_\alpha\tau}g(\alpha)\tensor{\Psi}{_\alpha^\pm} \right) \to&\left(\int d\beta e^{-iE_\beta \tau}g(\beta)\Phi_\beta,\int d\alpha e^{-iE_\alpha\tau}g(\alpha)\Phi_\alpha \right) \\
=&\int d\alpha d\beta \exp(-i(E_\alpha -E_\beta)\tau) g(\alpha)g^*(\beta )(\Phi_\beta,\Phi_\alpha)
\end{align*}
に等しい.
\begin{align*}
\int d\alpha d\beta \exp(-i(E_\alpha -E_\beta)\tau) g(\alpha)g^*(\beta )(\tensor{\Psi}{_\beta^\pm},\tensor{\Psi}{_\alpha^\pm})=\int d\alpha d\beta \exp(-i(E_\alpha -E_\beta)\tau) g(\alpha)g^*(\beta )(\Phi_\beta,\Phi_\alpha)
\end{align*}
このことは全ての滑らかな関数$g(\alpha)$について正しいと考えられるので,スカラー積は
\begin{align*}
(\tensor{\Psi}{_\beta^\pm},\tensor{\Psi}{_\alpha^\pm})=(\Phi_\beta,\Phi_\alpha)=\delta(\beta-\alpha) \quad \because(3.1.10)
\end{align*}
に等しいはずだ\footnote{$\Omega\subset \mathbb{R}^n$上の$L^2(\Omega)$空間上での双線形形式$\braket{f}{g}=\int_\Omega f(x)g(x)dx$は非退化になり,したがって任意の$f\in L^2(\Omega)$について$ \int_\Omega f(x)g(x)dx=\int_\Omega f(x)h(x)dx$を満たすならば$g=h$が成り立つ.}.

\vskip\baselineskip


条件(3.1.12)を満たすエネルギー固有状態(3.1.11)の形式的かつ具体的な解を求めておく.このために(3.1.11)を
\begin{align*}
(E_\alpha -H_0)\tensor{\Psi}{_\alpha^\pm}=V\tensor{\Psi}{_\alpha^\pm} 
\end{align*}
と書く.$(E_\alpha -H_0)$は$\Phi_\alpha$だけを消すわけではなく,違う状態であってもエネルギーが同じ$E_\beta=E_\alpha$な状態$\Phi_\beta$をも消す.つまり
\begin{align*}
(E_\alpha-H_0)\Phi_\beta=(E_\alpha-E_\beta)\Phi_\beta =(E_\alpha-E_\alpha)\Phi_\beta=0
\end{align*}
であるから,$(E_\alpha-H_0)$を作用させて$V\tensor{\Psi}{_\alpha^\pm}$となる状態$\tensor{\Psi}{_\alpha^\pm}$を見つけたいのだが,それはいくらでも存在することになる.(例えば以下での解(3.1.16)にそのような$\Phi_\beta$を付け足すだけで別の解ができる.)そこで,in状態とout状態は相互作用項が存在しない$V\to 0$で$\Phi_\alpha$になるはずだと考え,とりあえず$V\tensor{\Psi}{_{\alpha}^\pm}$に比例する項を$\Phi_\alpha$に付け加えることで
\begin{align*}
\tensor{\Psi}{_\alpha^\pm}=\Phi_\alpha +(E_\alpha -H_0\pm i\epsilon)^{-1}V\tensor{\Psi}{_\alpha^\pm}
\end{align*}
とする.これはとりあえず方程式$(E_\alpha -H_0)\tensor{\Psi}{_\alpha^\pm}=V\tensor{\Psi}{_\alpha^\pm}$の解になっている.第二項目を完全性(3.1.5)で展開して
\begin{align*}
V\tensor{\Psi}{_\alpha^\pm}=&\int d\beta (\Phi_\beta ,V\tensor{\Psi}{_\alpha^\pm})\Phi_\beta \\
=&\int d\beta \tensor{T}{_{\beta\alpha}^\pm}\Phi_\beta \\
\tensor{\Psi}{_\alpha^\pm}=&\Phi_\alpha +(E_\alpha -H_0\pm i\epsilon)^{-1}V\tensor{\Psi}{_\alpha^\pm} \\
=&\Phi_\alpha +\int d\beta \tensor{T}{_{\beta\alpha}^\pm}(E_\alpha -H_0\pm i\epsilon)^{-1} \Phi_\beta \\
=&\Phi_\alpha +\int d\beta \frac{\tensor{T}{_{\beta\alpha}^\pm}\Phi_\beta}{E_\alpha -E_\beta\pm i\epsilon} \\
\tensor{T}{_{\beta\alpha}^\pm}:=&(\Phi_\beta,V\tensor{\Psi}{_\alpha^\pm}) 
\end{align*}
最後の行では
\begin{align*}
(E_\alpha -H_0 \pm i\epsilon)\Phi_\beta=(E_\alpha -E_\beta \pm i\epsilon)\Phi_\beta
\end{align*}
であり,固有値は$\pm i\epsilon$を挿入しているおかげで任意の$\Phi_\beta$について非ゼロになるから逆演算子が存在することを用いた.ここで$\epsilon$は正の微小量であり,$(E_\alpha-H_0)$の逆演算子に意味を持たせるためのものだ.右辺の$T$には$\tensor{\Psi}{_\alpha^\pm}$が含まれているから,これは形式的な解であることに注意.これをリップマン・シュウィンガー方程式という.\par
先程述べたようにこれらには$\Phi_\beta$を加えるだけの任意性が残っている.これを排除するためにin,out状態に関する条件(3.1.12)を用いる.すなわち,次の重ね合わせを考える.
\begin{align*}
\tensor{\Psi}{_g^\pm}(t):=&\int d\alpha e^{-iE_\alpha t} g(\alpha)\tensor{\Psi}{_\alpha^\pm} \\
\Phi_g(t):=&\int d\alpha e^{-iE_\alpha t}g(\alpha) \Phi_\alpha
\end{align*}
条件(3.1.12)は,この$\tensor{\Psi}{_g^+}(t)$と$\tensor{\Psi}{_\alpha^-}(t)$がそれぞれ$t \to -\infty$と$t \to +\infty$で$\Phi_g(t)$に近づくことを示すことができれば良い.(3.1.17)をこの式の中に代入して
\begin{align*}
\tensor{\Psi}{_g^\pm}(t) =& \int d\alpha e^{-iE_\alpha t} g(\alpha)\left[ \Phi_\alpha + \int d\beta \frac{\tensor{T}{_{\beta\alpha}^\pm}\Phi_\beta}{E_\alpha -E_\beta\pm i\epsilon} \right] \\
=& \Phi_g (t) +\int d\alpha \int d\beta \frac{e^{-iE_\alpha t} g(\alpha) \tensor{T}{_{\beta\alpha}^\pm}\Phi_\beta}{E_\alpha - E_\beta \pm i\epsilon} \\
=&\Phi_g (t) + \int d\beta \Phi_\beta \left[\int d\alpha \frac{e^{-iE_\alpha t} g(\alpha) \tensor{T}{_{\beta\alpha}^\pm}}{E_\alpha - E_\beta \pm i\epsilon}\right]
\end{align*}
となる.したがって積分
\begin{align*}
\tensor{\mc{I}}{_\beta^\pm} := \int d\alpha \frac{e^{-iE_\alpha t} g(\alpha) \tensor{T}{_{\beta\alpha}^\pm}}{E_\alpha - E_\beta \pm i\epsilon}
\end{align*}
を考える.エネルギー変数$E_\alpha$を複素平面に拡張することを考えると,指数関数部分は
\begin{align*}
e^{-iE_\alpha t}=\exp(-i\mathrm{Re}(E_\alpha) t+\mathrm{Im}(E_\alpha)t)
\end{align*}
であるから,複素数化した$E_\alpha$が$\mathrm{Im}E_\alpha >0$ならば$t\to -\infty$の極限でゼロに向かう.逆に$E_\alpha$が$\mathrm{Im}E_\alpha <0$ならば$t\to +\infty$の極限でゼロに向かう.エネルギー変数$E_\alpha$は(3.1.7)
\begin{align*}
E_\alpha =\sqrt{\mathbf{p}_1^2+m_1^2}+\sqrt{\mathbf{p}_2^2+m_2^2}+\cdots
\end{align*}
で運動量変数$\mathbf{p}_1,\mathbf{p}_2,\cdots $と結びついている.\par
簡単のために$\alpha$が1粒子の場合で考えてみる.
\begin{align*}
\tensor{\mc{I}}{_\beta^\pm} := \int d^3\mathbf{p} \frac{e^{-iE(\mathbf{p}) t} g(\mathbf{p}) \tensor{T}{_{\beta\alpha}^\pm}}{E(\mathbf{p}) - E_\beta \pm i\epsilon}
\end{align*}
を評価してやる.ここで$E(\mathbf{p})=\sqrt{\mathbf{p}^2+m^2}$である.極座標に移行して
\begin{align*}
\tensor{\mc{I}}{_\beta^\pm} = \int d\Omega \int_0^\infty dp p^2 \frac{e^{-i\sqrt{p^2+m^2} t} g(p,\theta,\phi) \tensor{T}{_{\beta\alpha}^\pm}}{\sqrt{p^2+m^2} - E_\beta \pm i\epsilon}
\end{align*}
$p=\pm im \sim \pm i\infty$の部分にbranch cutが存在する.まず$t\to -\infty$での場合を考えるために,$i\epsilon$の符号を正にとる.$p$積分の部分だけ考えよう.評価すべき積分は
\begin{align*}
\int_0^\infty dp p^2 \frac{e^{-i\sqrt{p^2+m^2} t} g(p,\theta,\phi) T_{\beta\alpha}^+(p,\theta,\phi)}{\sqrt{p^2+m^2} - E_\beta + i\epsilon}
\end{align*}
となる.ここで,それぞれの運動量変数の複素平面上の経路を
\begin{align*}
C_1 =& \{p =x |x \in [0,R]\} \\
C_2 =& \{p =R+iy | y \in [0,\delta] \} \\
C_3 =& \{p =x + i\delta |x \in [R,0]\} \\
C_4 =& \{p =iy | y \in [\delta,0] \}
\end{align*}
のように右上平面での長方形にとる.ここでbranch cutを横切らないように$\delta <m $とする.分母からくる極は
\begin{align*}
0=&\sqrt{p^2+m^2}-E_\beta +i\epsilon \\
\therefore \quad p=&\pm\left(\sqrt{E_\beta^2-m^2} -i\epsilon \right)=\pm (p_0 - i\epsilon)
\end{align*}
にある.(一般に$T_{\beta\alpha}^\pm$からも極が発生することに注意.)このときそれぞれの経路による複素積分
\begin{align*}
\oint dp p^2 \frac{e^{-i\sqrt{p^2+m^2} t} g(p,\theta,\phi) T_{\beta\alpha}^+(p,\theta,\phi)}{\sqrt{p^2+m^2} - E_\beta + i\epsilon}=I_1+I_2+I_3+I_4
\end{align*}
を考えると,まず$C_1$に関して
\begin{align*}
I_1=\int_0^R dp p^2 \frac{e^{-i\sqrt{p^2+m^2} t} g(p,\theta,\phi) T_{\beta\alpha}^+(p,\theta,\phi)}{\sqrt{p^2+m^2} - E_\beta + i\epsilon}
\end{align*}
である.$C_2$に関して
\begin{align*}
I_2=&\int_{0}^\delta idy (R+iy)^2 \frac{e^{-i\sqrt{(R+iy)^2+m^2} t} g(R+iy,\theta,\phi) T_{\beta\alpha}^+(R+iy,\theta,\phi)}{\sqrt{(R+iy)^2+m^2} - E_\beta + i\epsilon} \\
=&\int_{0}^\delta idy (R+iy)^2 \frac{e^{-iRt+yt/2} g(R+iy,\theta,\phi) T_{\beta\alpha}^+(R+iy,\theta,\phi)}{R+iy/2 - E_\beta + i\epsilon}
\end{align*}
となる.ここで$R \gg m >y$で
\begin{align*}
\sqrt{(R+iy)^2+m^2} \approx R+i\frac{y}{2}
\end{align*}
であることを使った.この積分は$g(R+iy,\theta,\phi)$が十分遠い$R\to \infty$で滑らかに消えることからゼロ$I_2 \to 0$になる.$C_4$に関しても$g(iy,\theta,\phi)$が滑らかに消えることから同様に$I_4\to 0$である.$C_3$に関しては
\begin{align*}
I_3=\int_R^0 dp (p+i\delta)^2 \frac{e^{-i\sqrt{(p+i\delta)^2+m^2} t} g(p+i\delta,\theta,\phi) T_{\beta\alpha}^+(p+i\delta,\theta,\phi)}{\sqrt{(p+i\delta)^2+m^2} - E_\beta + i\epsilon}
\end{align*}
ここで近似
\begin{align*}
\sqrt{(p+i\delta)^2+m^2}=\sqrt{p^2+m^2-\delta^2-2ip\delta} \approx \sqrt{p^2+m^2} +i\frac{p\delta}{\sqrt{p^2+m^2}}
\end{align*}
を使うと\footnote{$p$は$[0,R]$の範囲を動く変数だから,どの範囲においてもこの近似がなりたつかどうかを議論する必要はある.実際に確かめると,$m$が有限の大きさかつ$\delta$が十分小さく$m>\delta$をとっている今回のような場合は$p$がどの範囲であってもこの近似は有効である.},この積分は
\begin{align*}
-\int_0^R dp (p+i\delta)^2 \frac{e^{-i\sqrt{p^2+m^2} t+p\delta t/\sqrt{p^2+m^2}} g(p+i\delta,\theta,\phi) T_{\beta\alpha}^+(p+i\delta,\theta,\phi)}{\sqrt{(p+i\delta)^2+m^2} - E_\beta + i\epsilon}
\end{align*}
となる.$p$積分は$g(p)$が限られた範囲のみで値をとりそれ以外では滑らかにゼロになることから有限となり,さらに$t\to -\infty$のもとで指数部分が
\begin{align*}
\exp\left(\frac{p\delta}{\sqrt{p^2+m^2}} t\right)=\exp\left(t \delta \frac{1}{\sqrt{1+p^2/m^2}}\right) < \exp(t \delta)\to 0
\end{align*}
になることからゼロ$I_3 \to 0$になる.以上より周回積分は$t\to -\infty$の極限を考える限り,$R\to \infty$で
\begin{align*}
\int_0^\infty dp p^2 \frac{e^{-i\sqrt{p^2+m^2} t} g(p,\theta,\phi) T_{\beta\alpha}^+(p,\theta,\phi)}{\sqrt{p^2+m^2} - E_\beta + i\epsilon} =\oint dp p^2 \frac{e^{-i\sqrt{p^2+m^2} t} g(p,\theta,\phi) T_{\beta\alpha}^+(p,\theta,\phi)}{\sqrt{p^2+m^2} - E_\beta + i\epsilon}
\end{align*}
となることがわかる.左辺の周回積分は留数定理により,留数の和の$2\pi i$倍で与えられるが,分母の極は$p=\pm(p_0-i\epsilon)$で与えられるのだったから,これは今回考えている右上平面上の経路の内側には存在しない.$T_{\beta\alpha}^+$には複素平面上で一般に極がエネルギー変数で書いて$E=E_0 -i\Gamma/2$の形で存在する(束縛状態)が,これも下平面に存在するからこの場合回避可能である.以上により経路の内側に極は存在せず,$t\to -\infty$では$\mc{I}_{\beta}^+\to 0$となる.\par
一方$t\to +\infty$を考えると,$-i\epsilon$の符号を選ぶことになり,そのときの評価すべき積分は
\begin{align*}
\int_0^\infty dp p^2 \frac{e^{-i\sqrt{p^2+m^2} t} g(p,\theta,\phi) T_{\beta\alpha}^-(p,\theta,\phi)}{\sqrt{p^2+m^2} - E_\beta - i\epsilon}
\end{align*}
となる.この場合の経路は右下平面上で
\begin{align*}
C_1 =& \{p =x |x \in [0,R]\} \\
C_2 =& \{p =R+iy | y \in [0,-\delta] \} \\
C_3 =& \{p =x + i\delta |x \in [R,0]\} \\
C_4 =& \{p =iy | y \in [-\delta,0] \}
\end{align*}
と長方形にとる.この場合にも同様の議論ができて,評価すべき積分は周回積分,すなわち留数の和に$2\pi i$倍したものに等しくなる.分母の極は$p=\pm (p_0+i\epsilon)$に存在するから,再びこれは回避される.束縛状態による$T_{\beta\alpha}^-$の極$E=E_0 -i\Gamma/2$は存在するかもしれないが,この留数による寄与は(何位の極だとしても)指数関数部分から$e^{-iE_0t}e^{-\Gamma t/2}$に比例する寄与を生み出し,これは$t\to +\infty$でゼロになる.\par
以上の処置を1粒子だけでなく一般の多粒子状態に対しても適用できるとする.(多分できると思うが,それぞれの運動量変数$(\mathbf{p}_n)_i(i=1,2,3)$の経路を,$\sqrt{p_i^2+m_i^2}$によるbranch cutを避けるために$\delta_i <m_i$は十分小さくとって
\begin{align*}
C_1 =& \{(p_n)_i =x |x \in [0,R]\} \\
C_2 =& \{(p_n)_i =R+iy | y \in [0,\delta_i] \} \\
C_3 =& \{(p_n)_i =x + i\delta_i |x \in [R,0]\} \\
C_4 =& \{(p_n)_i =iy | y \in [\delta_i,0] \}
\end{align*}
ととるのだと思うと,複雑な関数の複素多変数積分になるので1粒子の場合と同様に解析するのが困難だと思う.しかし本質的に$g(\alpha)$の減衰性と$e^{-iE_\alpha t}$の指数関数的振る舞い,分母の極の位置の解析は同様のはずだから,本質的な困難ではないと思う.)これにより$\tensor{\Psi}{_g^\pm}(t)$は$t\to \mp \infty$で$\Phi_g(t)$になり,リップマン・シュウィンガー方程式の解(3.1.17)は漸近条件を満たす.

\vskip\baselineskip

(3.1.17)の因子$(E_\alpha -E_\beta\pm i\epsilon)^{-1}$の便利な表現を与える.一般に
\begin{align*}
(E\pm i\epsilon)^{-1}=&\frac{E\mp i\epsilon}{E^2+\epsilon^2} \\
=&\frac{E}{E^2+\epsilon^2 }\mp i\pi \frac{\epsilon}{\pi(E^2+\epsilon^2)} \\
=& \frac{\mc{P}_\epsilon}{E} \mp i\pi \delta_{\epsilon}(E)
\end{align*}
と書ける.ここで
\begin{align*}
\frac{\mc{P}_\epsilon}{E}:=&\frac{E}{E^2+\epsilon^2 } \\
\delta_\epsilon(E):=&\frac{\epsilon}{\pi(E^2+\epsilon^2)} 
\end{align*}
である.$\mc{P_\epsilon}/E$は$|E|\gg \epsilon$でちょうど$1/E$のように振る舞い,$E\to 0$ではゼロになる.したがってこれは$\epsilon \to 0$の極限では主値関数$\mc{P}/E$のように振舞う.これは,$1/E$の$E=0$近傍の微小間隔を取り除くことによって$1/E$と滑らかな関数が広義積分可能になっているものだ.実際,
\begin{align*}
\lim_{a\to 0}\int_{-\infty}^{\infty} \frac{x}{x^2+a^2}\varphi(x)dx =\lim_{a\to 0}\left[\int_{|x|>\epsilon} \frac{x}{x^2+a^2}\varphi(x)dx+\int_{|x|<\epsilon} \frac{x}{x^2+a^2}\varphi(x)dx\right]
\end{align*}
と任意の小さい$\epsilon>0$で積分を分解すると,第二項目はどの点でも滑らかだから$\epsilon \to 0$でゼロになり,第一項目も被積分関数はどの点でも滑らかだから極限を交換することができて\footnote{極限の交換に関する詳しい条件は知らん.ルベーグ積分を復習する余裕あったらいつかちゃんと書きます.}
\begin{align*}
&\lim_{a\to 0}\lim_{\epsilon \to +0}\left[\int_\epsilon^\infty \frac{x}{x^2+a^2}\varphi(x)dx+\int_{-\infty}^{-\epsilon} \frac{x}{x^2+a^2}\varphi(x)dx\right] \\
=&\lim_{\epsilon \to 0}\lim_{a \to +0}\left[\int_\epsilon^\infty \frac{x}{x^2+a^2}\varphi(x)dx+\int_{-\infty}^{-\epsilon} \frac{x}{x^2+a^2}\varphi(x)dx\right]\\
=&\lim_{\epsilon\to 0}\left[\int_\epsilon^\infty \frac{1}{x}\varphi(x)dx+\int_{-\infty}^{-\epsilon} \frac{1}{x}\varphi(x)dx\right] \\
=&\mc{P} \int^{\infty}_{-\infty} \frac{1}{x}\varphi(x)dx
\end{align*}
となって,主値積分になっていることが理解できる.関数$\delta_\epsilon(E)$は$|E|\gg \epsilon$で$\epsilon$の大きさ(ほぼゼロ)であり,$E$の全域で積分すると1になる.実際,極は$E=\pm i\epsilon$にあるから,上を通る半円の経路で(下でもいい)周回複素積分すると
\begin{align*}
&\oint \frac{\epsilon}{\pi(E^2+\epsilon^2)} dE =\int_0^{\pi} \frac{\epsilon}{\pi(R^2e^{2i\theta}+\epsilon^2)}iRe^{i\theta}d\theta +\int^{R}_{-R}\frac{\epsilon}{\pi(E^2+\epsilon)^2} dE \to \int^\infty_{-\infty} \frac{\epsilon}{\pi(E^2+\epsilon^2)} dE \\
=&2\pi i \left\{\lim_{E\to i\epsilon} (E-i\epsilon)\frac{\epsilon}{\pi (E^2+\epsilon^2)}\right\}=2\pi i \frac{\epsilon}{\pi 2i\epsilon}=1
\end{align*}
となる.したがって$\epsilon\to 0$の極限でデルタ関数$\delta(E)$のように振舞う.実際
\begin{align*}
&\oint \frac{\epsilon}{\pi(E^2+\epsilon)^2}\varphi(E) dE =\int_0^{\pi} \frac{\epsilon}{\pi(R^2e^{2i\theta}+\epsilon^2)}iRe^{i\theta}\varphi(R e^{i\theta})d\theta +\int^{R}_{-R}\frac{\epsilon}{\pi(E^2+\epsilon)^2}\varphi(E) dE \\
&\to  \int^\infty_{-\infty} \frac{\epsilon}{\pi(E^2+\epsilon)^2}\varphi(E) dE=\int^\infty_{-\infty} \delta_{\epsilon}(E)\varphi(E) dE \\
=&2\pi i \left\{\lim_{E\to i\epsilon} (E-i\epsilon)\frac{\epsilon}{\pi (E^2+\epsilon^2)}\varphi(E)\right\}=\varphi(i\epsilon) \\
&\epsilon\to 0 で \quad \int^\infty_{-\infty} \delta_{\epsilon}(E)\varphi(E) dE=\varphi(0)
\end{align*}
このように理解することにすれば,(3.1.22)の添え字$\epsilon$は落として,超関数(distribution)の意味で等式
\begin{align*}
(E \pm i\epsilon)^{-1} = \frac{\mc{P}}{E}\mp i\pi \delta(E)
\end{align*}
がなりたっているといえる.


\newpage



\subsection{$S$行列}
実験屋は一般に特定の粒子からなる状態を$t\to -\infty$で用意し,$t\to +\infty$でこの状態がどう見えるかを測定する.もし用意された状態が$t\to -\infty$で粒子の内容として$\alpha$を持つなら,それはin状態$\tensor{\Psi}{_\alpha^+}$だ.また,$t\to +\infty$で粒子の内容$\beta$を持つことが見いだされれば,それはout状態$\tensor{\Psi}{_\beta^-}$だ.したがって,遷移$\alpha \to \beta$の確率振幅はスカラー積
\begin{align*}
S_{\beta\alpha}=(\tensor{\Psi}{_\beta^-},\tensor{\Psi}{_\alpha^+})
\end{align*}
で与えられる.この複素振幅の配列は$S$行列として知られている.もし相互作用が中間になければ,in状態とout状態は同じとなり,$S_{\beta\alpha}$はちょうど$\delta(\alpha-\beta)$となるはずだ.したがって,反応$\alpha\to \beta$の\uwave{確率}は$|S_{\beta\alpha}-\delta(\alpha-\beta)|^2$に比例する.$S_{\beta\alpha}$が,測定される反応率や断面積とどう関係しているかは3.4節で詳しく見る.

\vskip\baselineskip


in状態とout状態は二つの異なる種類のヒルベルト空間に存在するわけではなく,両者の違いはラベルのつけ方のみだ.すなわち,$t\to -\infty$でどう見えるか,あるいは$t\to +\infty$でどう見えるかの違いだけだ.任意のin状態はout状態の和で展開でき
\begin{align*}
\tensor{\Psi}{_\alpha^+}=\int d\alpha S_{\beta\alpha}\tensor{\Psi}{_\beta^-}
\end{align*}
その展開係数は$S$行列で与えられる.\par
$S_{\beta\alpha}$は直交する状態の2組の完全系を結ぶ行列なので,ユニタリーでなければならない.このことをより詳細に見るために,完全性の関係をout状態に適用して
\begin{align*}
\int d\beta S_{\beta \gamma}^* S_{\beta\alpha}=\int d\beta (\tensor{\Psi}{_\gamma^+},\tensor{\Psi}{_\beta^-})(\tensor{\Psi}{_\beta^-},\tensor{\Psi}{_\alpha^+})=(\tensor{\Psi}{_\gamma^+},\tensor{\Psi}{_\alpha^+})
\end{align*}
(3.1.15)を用いると
\begin{align*}
\int d\beta S_{\beta \gamma}^* S_{\beta\alpha}=\delta(\gamma-\alpha)
\end{align*}
あるいは短く書いて
\begin{align*}
S^\dagger S=1
\end{align*}
を与える.同様にin状態の完全性から
\begin{align*}
\int d\beta S_{\gamma\beta}S^*_{\alpha\beta}=\int d\beta (\tensor{\Psi}{_\gamma^-},\tensor{\Psi}{_\beta^+})(\tensor{\Psi}{_\beta^+},\tensor{\Psi}{_\alpha^-})=(\tensor{\Psi}{_\gamma^+},\tensor{\Psi}{_\alpha^-})=\delta(\gamma-\alpha)
\end{align*}
すなわち
\begin{align*}
SS^\dagger =1
\end{align*}
である.以上より$S$行列はユニタリー性を満たしていることがわかった\footnote{この行列$S_{\beta\alpha}$は有限次元ではないので,$SS^\dagger=1$と$S^\dagger S=1$は同値ではなくどちらも示さなければユニタリー性の証明とはならない.なぜなら,有限次元では任意の行列$A$に対して$AB=1$を満たす行列$B$が存在すればそれは一意的で$BA=AB=1$を満たすが,無限次元ではこの性質が成立しない場合があるからだ.実際,任意の無限次元のベクトル$(a_1,a_2,\cdots )$に対して,左にシフトする演算子
\begin{align*}
A:(a_1,a_2,\cdots )\mapsto (0,a_1,a_2,\cdots )
\end{align*}
と,右にシフトする演算子
\begin{align*}
B:(a_1,a_2,\cdots )\mapsto (a_2,a_3,\cdots )
\end{align*}
を考えると$BA=1$を満たすが$AB \neq 1$である.}.


\vskip\baselineskip

$S$行列を扱う代わりに,自由粒子状態間の行列要素が$S$行列の対応する要素に等しくなるような演算子,すなわち
\begin{align*}
(\Phi_{\beta},S\Phi_{\alpha})=S_{\beta\alpha}
\end{align*}
となるように定義された演算子$S$を扱った方が便利なことがあるらしい.(3.1.13)より$S$演算子は
\begin{align*}
S_{\beta\alpha}=&(\tensor{\Psi}{_\beta^-},\tensor{\Psi}{_\alpha^+}) \\
=&(\Omega(+\infty)\Phi_\beta,\Omega(-\infty)\Phi_\alpha) \\
=&(\Phi_\beta,\Omega(+\infty)^\dagger \Omega(-\infty)\Phi_\alpha) \\
S=&\Omega (\infty)^\dagger \Omega(-\infty) =U(+\infty,-\infty) \\
U(\tau ,\tau_0):=&\Omega(\tau)^\dagger \Omega(\tau_0)=\exp(iH_0 \tau)\exp(-iH(\tau-\tau_0))\exp(-iH_0 \tau_0)
\end{align*}
となる.\par
前節の積分による方法は,$S$行列の別の公式を導くのにも使われる.in状態$\tensor{\Psi}{_\alpha^+}$の式(3.1,21)に戻って,今度は$t\to +\infty$ととる.
\begin{align*}
\tensor{\Psi}{_g^+}(t)=&\Phi_g (t) + \int d\beta \Phi_\beta \left[\int d\alpha \frac{e^{-iE_\alpha t} g(\alpha) \tensor{T}{_{\beta\alpha}^\pm}}{E_\alpha - E_\beta \pm i\epsilon}\right] \\
\tensor{\mc{I}}{_\beta^+}=&\int d\alpha \frac{e^{-iE_\alpha t} g(\alpha) \tensor{T}{_{\beta\alpha}^+}}{E_\alpha - E_\beta + i\epsilon}
\end{align*}
再び一粒子で考えよう.すると,$e^{-iE_\alpha t}$の因子のために今度は$p$積分は右下平面を通る経路
\begin{align*}
C_1 =& \{p =x |x \in [0,R]\} \\
C_2 =& \{p =R+iy | y \in [0,-\delta] \} \\
C_3 =& \{p =x + i\delta |x \in [R,0]\} \\
C_4 =& \{p =iy | y \in [-\delta,0] \}
\end{align*}
でとる必要がある.この経路内部にある特異点は$E_\alpha= E_\beta-i\epsilon$の点からくる$p=p_0-i\epsilon(p_0=\sqrt{E_\beta^2-m^2})$の寄与のみだ.(束縛状態による他の極もあるが,それは前と同じように$t\to +\infty$で消える.)これは時計回りの経路だから,留数の方法によって
\begin{align*}
\tensor{\mc{I}}{_\beta^+}(t)=&\int d\Omega \int_0^\infty dp p^2 \frac{e^{-i\sqrt{p^2+m^2} t} g(p,\theta,\phi) T_{\beta\alpha}^+(p,\theta,\phi)}{\sqrt{p^2+m^2} - E_\beta + i\epsilon} \\
=&\int d\Omega (-2\pi i)\left[\lim_{p\to p_0-i\epsilon} (p-p_0+i\epsilon)p^2 e^{-i\sqrt{p^2+m^2} t} g(p,\theta,\phi) T_{\beta\alpha}^+(p,\theta,\phi)\frac{\sqrt{p^2+m^2}+E_\beta}{p^2+m^2-E_\beta+i\epsilon}\right] \\
=&-2\pi i\int d\Omega \left[\lim_{p\to p_0-i\epsilon} p^2 e^{-i\sqrt{p^2+m^2} t} g(p,\theta,\phi) T_{\beta\alpha}^+(p,\theta,\phi)\frac{\sqrt{p^2+m^2}+E_\beta}{p+p_0-i\epsilon}\right] \\
=&-2\pi i\int d\Omega \left[p^2_0 e^{-iE_\beta t} g(p_0,\theta,\phi) T_{\beta\alpha}^+(p_0,\theta,\phi)\frac{E_\beta}{\sqrt{E_\beta^2-m^2}}\right] \quad (\epsilon \to 0)\\
=&-2\pi i \int d\Omega \int^\infty_0 \delta(p-p_0) \left[p^2 e^{-iE_\beta t} g(p,\theta,\phi) T_{\beta\alpha}^+(p,\theta,\phi)\frac{E_\beta}{\sqrt{E^2_\beta-m^2}}\right] \\
=&-2\pi i \int d\Omega \int^\infty_0 dp\Biggl[p^2 e^{-iE_\beta t} g(p,\theta,\phi) T_{\beta\alpha}^+(p,\theta,\phi) \\
&\qquad \qquad \times \frac{E_\beta}{\sqrt{E^2_\beta-m^2}}\left\{\delta\left(p-\sqrt{E_\beta^2-m^2}\right)+\delta\left(p+\sqrt{E_\beta^2-m^2}\right)\right\}\Biggr] \\
=&-2\pi i \int d\Omega \int^\infty_0 dp\, p^2\Biggl[ e^{-iE_\beta t} g(p,\theta,\phi) T_{\beta\alpha}^+(p,\theta,\phi)\delta(\sqrt{p^2+m^2}-E_\beta)\Biggr] \\
=&-2\pi i e^{-iE_\beta t} \int d^3\mathbf{p} \delta(E(\mathbf{p})-E_\beta) g(\mathbf{p}) T_{\beta\alpha}^+(\mathbf{p})
\end{align*}
が得られる.(極限をとる順番に注意.$t\to +\infty$は最後の行で暗黙に行っている.ここでは書いていないが全ての行で束縛状態による留数からくる$e^{-\Gamma t/2}$に比例する項が存在し,それらは$\Gamma$が有限の大きさであることから最後の$t\to +\infty$の極限で消えてしまう.)ここで2行目では留数定理を使い,また分母を有理化している.4行目では積分を実行してしまったから$\epsilon$をゼロにもっていった.6行目では$p>0$であることを用いて,全域でゼロになるデルタ関数を挿入した.7行目ではデルタ関数の公式を使って
\begin{align*}
\delta(\sqrt{p^2+m^2}-E_\beta)=\frac{E_\beta}{\sqrt{E^2_\beta-m^2}}\left\{\delta\left(p-\sqrt{E_\beta^2-m^2}\right)+\delta\left(p+\sqrt{E_\beta^2-m^2}\right)\right\}
\end{align*}
を用いた.以上の手続きが一般の$n$粒子の場合でも同様に行えるとして,$t\to +\infty$によって
\begin{align*}
\tensor{\mc{I}}{_\beta^+}(t)\to -2\pi i e^{-iE_\beta t} \int d\alpha \delta(E_\alpha-E_\beta)g(\alpha)T_{\beta\alpha}^+
\end{align*}
と漸近する.したがって$t\to +\infty$では
\begin{align*}
\tensor{\Psi}{_g^+}(t)=&\int d\alpha e^{-iE_\beta}g(\beta)\Phi_\beta + \int d\beta \tensor{\mc{I}}{_\beta^+}(t) \\
\to &\int d\beta e^{-iE_\beta t} \Phi_\beta\left[g(\beta)-2\pi i \int d\alpha \delta(E_\alpha-E_\beta)g(\alpha)T_{\beta\alpha}^+\right]
\end{align*}
という漸近的振る舞いをする.一方,(3.1.19)をout状態の完全系で展開すると
\begin{align*}
\tensor{\Psi}{_g^+}(t)=&\int d\alpha e^{-iE_\alpha t}g(\alpha)\int d\beta \tensor{\Psi}{_\beta^-}(\tensor{\Psi}{_\beta^-},\tensor{\Psi}{_\alpha^-}) \\
=&\int d\alpha e^{-iE_\alpha t}g(\alpha)\int d\beta \tensor{\Psi}{_\beta^-}S_{\beta\alpha} \\
=&\int d\beta \tensor{\Psi}{_\beta^-} \int d\alpha e^{-iE_\alpha t} g(\alpha) S_{\beta\alpha}
\end{align*}
$\tensor{\Psi}{_\alpha^+},\tensor{\Psi}{_\beta^-}$はそれぞれハミルトニアン$H$の固有状態であり,その内積はエネルギースペクトルが等しくなければ直交する.したがって$S_{\beta\alpha}$は$\delta(E_\alpha-E_\beta)$の因子を必ず含み
\begin{align*}
\tensor{\Psi}{_g^+}(t)=\int d\beta \tensor{\Psi}{_\beta^-} e^{-iE_\beta t}\int d\alpha g(\alpha) S_{\beta\alpha}
\end{align*}
と変形できる.$g(\alpha)$が$E_\alpha$から幅$\Delta E$でゼロになる滑らかさな波束関数であることを用いると,$\int d\alpha g(\alpha)S_{\beta\alpha}$も$E_\beta$から幅$\Delta E$で滑らかにゼロになる波束関数である.よって$t\to +\infty$で$\tensor{\Psi}{_\beta^-}$とそのような波束での重ね合わせは$\Phi_\beta$の重ね合わせに漸近するというout状態の定義を用いれば
\begin{align*}
\tensor{\Psi}{_g^+}(t) \to \int d\beta \Phi_\beta e^{-iE_\beta t}\int d\alpha g(\alpha) S_{\beta\alpha}
\end{align*}
となる.これで$\tensor{\Psi}{_g^+}$の二通りの漸近が得られたが,それぞれは等しいはずだから
\begin{align*}
\int d\alpha g(\alpha) S_{\beta\alpha}=&g(\beta)-2\pi i \int d\alpha \delta(E_\alpha-E_\beta)g(\alpha)T_{\beta\alpha}^+ \\
=&\int d\alpha g(\alpha ) \left[\delta(\alpha-\beta)-2\pi i \delta(E_\alpha-E_\beta)T_{\beta\alpha}^+\right]
\end{align*}
$g(\alpha)$は滑らかだから
\begin{align*}
S_{\beta\alpha}=&\delta(\alpha-\beta)-2\pi i \delta(E_\alpha-E_\beta) T_{\beta\alpha}^+ \\
T_{\beta\alpha}^+=&(\Phi_\beta, V \tensor{\Psi}{_\alpha^+})
\end{align*}
が得られる.これが欲しかった結果だ.\par
相互作用が弱く(3.1.18)のin状態と自由粒子状態との差が十分小さいとできる場合,$T_{\beta\alpha}^+\approx (\Phi_\beta,V\Phi_\alpha)$と近似でき
\begin{align*}
S_{\beta\alpha}\approx \delta(\alpha-\beta)-2\pi i \delta(E_\alpha-E_\beta) (\Phi_\beta, V \Phi_\alpha)
\end{align*}
が得られる.これはボルン近似として知られている.\par
これが実際に我々の知っているボルン近似と同じものであることを見るには,スピンに依らない1粒子の非相対論的ポテンシャル散乱を考えるといい.ポテンシャル$V(\mathbf{r})$による散乱が起きるとすると,非相対論的な場合を考えているので位置表示の完全性$\Phi_{\mathbf{p}}=(2\pi)^{3/2}\int d^3 \mathbf{x} \Phi_\mathbf{x}e^{i\mathbf{p}\cdot \mathbf{x}}$が使えて,さらに$(\Phi_{\mathbf{x}'},\Phi_{\mathbf{x}})=\delta^3(\mathbf{x}'-\mathbf{x})$を使うと
\begin{align*}
(\Phi_{\mathbf{p}'},V \Phi_{\mathbf{p}})=&\frac{1}{(2\pi)^3}\int d^3 \mathbf{r} V(\mathbf{r})e^{-i(\mathbf{p}'-\mathbf{p})\cdot \mathbf{r}}=\tilde{V}(\mathbf{q})
\end{align*}
が得られる.ここで$\tilde{V}(\mathbf{q})$はポテンシャル$V(\mathbf{r})$のフーリエ変換であり,$\mathbf{q}=\mathbf{p}'-\mathbf{p}$とおいた.よって$S$行列は
\begin{align*}
S_{\mathbf{p}'\mathbf{p}}=&\delta^3(\mathbf{p}'-\mathbf{p})-2\pi i \delta(E(\mathbf{p}')-E(\mathbf{p}))(\Phi_{\mathbf{p}'},V(\mathbf{r})\Phi_{\mathbf{p}}) \\
=&\delta^3(\mathbf{p}'-\mathbf{p})-2\pi i \delta(E(\mathbf{p}')-E(\mathbf{p}))\tilde{V}(\mathbf{q})
\end{align*}
という形になる.よって散乱振幅はポテンシャル$V(\mathbf{r})$のフーリエ変換によって得られ,これは確かにボルン近似の性質である.\par
さらに後の計算のために,スピンに依らないポテンシャル$V(\mathbf{r}_1-\mathbf{r}_2)$による2粒子の散乱$1,2\to 1',2'$も考えておく.特に11章で使う.ここで$1$と$1'$,$2$と$2'$の組が同じ種類の粒子とする.このとき平面波は
\begin{align*}
(\Phi_{\mathbf{r}_1,\mathbf{r}_2},\Phi_{\mathbf{p}_1,\mathbf{p}_2})=\frac{1}{(2\pi)^3} e^{i\mathbf{p}_1 \cdot \mathbf{r}_1} e^{i\mathbf{p}_2\cdot \mathbf{r}_2}
\end{align*}
で与えられて
\begin{align*}
(\Phi_{\mathbf{p}'_1,\mathbf{p}_2'} ,V\Phi_{\mathbf{p}_1,\mathbf{p}_2})=\frac{1}{(2\pi)^6}\int d^3\mathbf{r}_1 d^3 \mathbf{r}_2 e^{-i\mathbf{p}'_1 \cdot \mathbf{r}_1} e^{-i\mathbf{p}_2'\cdot \mathbf{r}_2} V(\mathbf{r}_2-\mathbf{r}_1) e^{i\mathbf{p}_1 \cdot \mathbf{r}_1} e^{i\mathbf{p}_2\cdot \mathbf{r}_2}
\end{align*}
ここで重心座標と相対座標
\begin{align*}
\mathbf{R}:=\frac{m_1\mathbf{r}_1+m_2\mathbf{r}_2}{m_1+m_2},\mathbf{r}:=\mathbf{r}_2-\mathbf{r}_1
\end{align*}
および全運動量と相対運動量
\begin{align*}
\mathbf{P}:=\mathbf{p}_1+\mathbf{p}_2,\mathbf{p}:=\frac{m_2\mathbf{p}_1-m_1 \mathbf{p}_2}{m_1+m_2}
\end{align*}
を導入すると,$\mathbf{p}_1\cdot \mathbf{r}_1+\mathbf{p}_2\cdot \mathbf{r}_2=\mathbf{P}\cdot \mathbf{R}+\mathbf{p}\cdot \mathbf{r}$とうまく分解できることを使って
\begin{align*}
(\Phi_{\mathbf{p}'_1,\mathbf{p}_2'} ,V\Phi_{\mathbf{p}_1,\mathbf{p}_2})=&\frac{1}{(2\pi)^6}\int d^3\mathbf{r} d^3 \mathbf{R} e^{-i\mathbf{P}' \cdot \mathbf{R}} e^{-i\mathbf{p}'\cdot \mathbf{r}} V(\mathbf{r}) e^{i\mathbf{P} \cdot \mathbf{R}} e^{i\mathbf{p}\cdot \mathbf{r}} \\
=&\frac{1}{(2\pi)^3}\delta^3(\mathbf{P}'-\mathbf{P})\int d^3\mathbf{r} V(\mathbf{r})e^{-i(\mathbf{p}'-\mathbf{p})\cdot \mathbf{r}} \\
=&\delta^3(\mathbf{p}_1'+\mathbf{p}_2-\mathbf{p}_1-\mathbf{p}_2)\tilde{V}(\mathbf{q})
\end{align*}
が得られる.ここで再び$\tilde{V}(\mathbf{q})$はポテンシャル$V(\mathbf{r})$のフーリエ変換だ.(変数変換したとき,$d^3\mathbf{r}_1d^3 \mathbf{r}_2$から生じるヤコビアンが1になることは確認しなければならない.ただ直接示そうとすると
\begin{align*}
\frac{\partial(\mathbf{R},\mathbf{r})}{\partial(\mathbf{r}_1,\mathbf{r}_2)}=\left(
\begin{matrix}
\frac{\partial \mathbf{R}}{\partial \mathbf{r}_1} & \frac{\partial \mathbf{R}}{\partial \mathbf{r}_2} \\
\frac{\partial \mathbf{r}}{\partial \mathbf{r}_1} & \frac{\partial \mathbf{r}}{\partial \mathbf{r}_2}
\end{matrix}
\right)=\left(
\begin{matrix}
\frac{m_1}{m_1+m_2} \mathbf{I} & \frac{m_2}{m_1+m_2} \mathbf{I} \\
\mathbf{I} & -\mathbf{I}
\end{matrix}
\right)
\end{align*}
となって,この行列式を計算するのは若干骨が折れる.ここで実は2.5節で示しているブロック行列に関する行列式の定理
\begin{align*}
\det \left(
\begin{matrix}
A & B \\
C & D
\end{matrix}
\right)=\det D \det(A-BD^{-1}C)
\end{align*}
を使うと楽に計算できる.その結果ちゃんとヤコビアンは1になる.)したがって$S$行列は
\begin{align*}
S_{\mathbf{p}'_1,\mathbf{p}_2';\mathbf{p}_1,\mathbf{p}_2}=\delta^3(\mathbf{p}_1'+\mathbf{p}_2-\mathbf{p}_1-\mathbf{p}_2)-2\pi i \delta^4(p_\beta -p_\alpha) \tilde{V}(\mathbf{q})
\end{align*}
と書ける.

\vskip\baselineskip


in状態とout状態に関するリップマン・シュウィンガー方程式(3.1.16)を使えば,(3.2.7)だけでなく,これらの状態の正規直交性と$S$行列のユニタリー性を$t\to \mp \infty$の極限を考えなくても証明できる.\par
まず,(3.1.16)を行列要素$(\tensor{\Psi}{_\beta^\pm},V\tensor{\Psi}{_\alpha^\pm})$の左側または右側に用いて,それらの結果を等しいとおく.つまり
\begin{align*}
(右側): (\tensor{\Psi}{_\beta^\pm},V\tensor{\Psi}{_\alpha^\pm}) =&(\tensor{\Psi}{_\beta^\pm},V\Phi_\alpha)+(\tensor{\Psi}{_\beta^\pm},V(E_\alpha-H_0 \pm i\epsilon)^{-1} V \tensor{\Psi}{_\alpha^\pm}) \\
(左側) :(\tensor{\Psi}{_\beta^\pm},V\tensor{\Psi}{_\alpha^\pm}) =&(\Phi_\beta,V\tensor{\Psi}{_\alpha^\pm}) +((E_\beta-H_0\pm i\epsilon)^{-1} V \Phi_\beta,V\tensor{\Psi}{_\alpha^\pm}) \\
=&(\Phi_\beta,V\tensor{\Psi}{_\alpha^\pm}) +( \Phi_\beta,V (E_\beta-H_0\mp i\epsilon)^{-1} V\tensor{\Psi}{_\alpha^\pm})
\end{align*}
であるから,これらが等しいとおいて
\begin{align*}
&(\tensor{\Psi}{_\beta^\pm},V\Phi_\alpha)+(\tensor{\Psi}{_\beta^\pm},V(E_\alpha-H_0 \pm i\epsilon)^{-1} V \tensor{\Psi}{_\alpha^\pm}) \\
&\qquad \qquad =(\Phi_\beta,V\tensor{\Psi}{_\alpha^\pm}) +( \Phi_\beta,V (E_\beta-H_0\mp i\epsilon)^{-1} V\tensor{\Psi}{_\alpha^\pm})
\end{align*}
とできる.ここで$T_{\beta\alpha}^\pm=(\Phi_\beta,V\tensor{\Psi}{_\alpha^\pm})$と
\begin{align*}
(\tensor{\Psi}{_\beta^\pm},V\Phi_\alpha)=(V\tensor{\Psi}{_\beta^\pm},\Phi_\alpha) =(\Phi_\alpha,V\tensor{\Psi}{_\beta^\pm})^*=T^{\pm*}_{\alpha\beta}
\end{align*}
であることを使うと
\begin{align*}
T^{\pm*}_{\alpha\beta}-T^\pm_{\beta\alpha}=-(\tensor{\Psi}{_\beta^\pm},V(E_\alpha-H_0 \pm i\epsilon)^{-1} V \tensor{\Psi}{_\alpha^\pm})+( \Phi_\beta,V (E_\beta-H_0\mp i\epsilon)^{-1} V\tensor{\Psi}{_\alpha^\pm})
\end{align*}
と書ける.中間状態$\Phi_\gamma$について完全性を使うと,
\begin{align*}
T^{\pm*}_{\alpha\beta}-T^\pm_{\beta\alpha}=&-\int d\gamma \left(\tensor{\Psi}{_\beta^\pm},V(E_\alpha-H_0 \pm i\epsilon)^{-1} \Phi_\gamma (\Phi_\gamma ,V \tensor{\Psi}{_\alpha^\pm})\right) \\
&+\int d\gamma \left( \Phi_\beta,V (E_\beta-H_0\mp i\epsilon)^{-1} \Phi_\gamma (\Phi_\gamma ,V\tensor{\Psi}{_\alpha^\pm})\right) \\
=&-\int d\gamma (\tensor{\Psi}{_\beta^\pm},V\Phi_\gamma)(\Phi_\gamma,V\tensor{\Psi}{_\alpha^\pm})\frac{1}{E_\alpha-E_\gamma\pm i\epsilon} \\
&+\int d\gamma (\tensor{\Psi}{_\beta^\pm},V\Phi_\gamma)(\Phi_\gamma,V\tensor{\Psi}{_\alpha^\pm})\frac{1}{E_\beta-E_\gamma \mp i\epsilon} \\
=& -\int d\gamma T^{\pm *}_{\gamma\beta}T^\pm_{\gamma\alpha}\left[(E_\alpha-E_\gamma\pm i\epsilon)^{-1}- (E_\beta- E_\gamma\mp i\epsilon)^{-1}\right]
\end{align*}
ここで
\begin{align*}
\frac{1}{E_\alpha-E_\gamma\pm i\epsilon}-\frac{1}{E_\beta-E_\gamma \mp i\epsilon} =-\frac{E_\alpha-E_\beta \pm 2i\epsilon}{(E_\alpha-E_\gamma\pm i\epsilon)(E_\beta-E_\gamma\mp i\epsilon)}
\end{align*}
と書けるから,両辺を$E_\alpha-E_\beta \pm 2i\epsilon$で割って
\begin{align*}
(\mathrm{LHS})=&\frac{T_{\alpha\beta}^{\pm *}}{E_\alpha-E_\beta \pm 2i\epsilon}-\frac{T_{\beta\alpha}^{\pm}}{E_\alpha-E_\beta \pm 2i\epsilon} \\
=&-\frac{T_{\alpha\beta}^{\pm *}}{E_\beta-E_\alpha \mp 2i\epsilon}-\frac{T_{\beta\alpha}^{\pm}}{E_\alpha-E_\beta \pm 2i\epsilon} \\
=&-\left(\frac{T^\pm_{\alpha\beta}}{E_\beta-E_\alpha \pm 2i\epsilon}\right)^* -\frac{T^\pm_{\beta\alpha}}{E_\alpha-E_\beta \pm 2i\epsilon} \\
(\mathrm{RHS})=&\int d\gamma \frac{T^{\pm*}_{\gamma\beta}}{E_\beta-E_\gamma \mp i\epsilon} \frac{T^\pm_{\gamma\alpha}}{E_\alpha-E_\gamma \pm i\epsilon} \\
=&\int d\gamma \left(\frac{T^\pm_{\gamma\beta}}{E_\beta-E_\gamma \pm i\epsilon}\right)^* \frac{T^\pm_{\gamma\alpha}}{E_\alpha-E_\gamma \pm i\epsilon} \\
\therefore \quad & \left(\frac{T_{\alpha\beta}^{\pm}}{E_\beta-E_\alpha \pm 2i\epsilon}\right)^*+\frac{T_{\beta\alpha}^{\pm}}{E_\alpha-E_\beta \pm 2i\epsilon}=-\int d\gamma \left(\frac{T^\pm_{\gamma\beta}}{E_\beta-E_\gamma \pm i\epsilon}\right)^* \frac{T^\pm_{\gamma\alpha}}{E_\alpha-E_\gamma \pm i\epsilon}
\end{align*}
が得られる.左辺の分母の$2\epsilon$は,これが正の微小量であることだけが重要であるから,$\epsilon$で置き換えて
\begin{align*}
\left(\frac{T_{\alpha\beta}^{\pm}}{E_\beta-E_\alpha \pm i\epsilon}\right)^*+\frac{T_{\beta\alpha}^{\pm}}{E_\alpha-E_\beta \pm i\epsilon}=-\int d\gamma \left(\frac{T^\pm_{\gamma\beta}}{E_\beta-E_\gamma \pm i\epsilon}\right)^* \frac{T^\pm_{\gamma\alpha}}{E_\alpha-E_\gamma \pm i\epsilon}
\end{align*}
となる.この関係式を用いると,$\delta(\beta-\alpha)+T^\pm_{\beta\alpha}/(E_\alpha-E_\beta\pm i\epsilon)$がユニタリー行列であることが示せる.実際これを$U_{\beta\alpha}$とおくと
\begin{align*}
U^\pm_{\beta\alpha}:=&\delta (\beta-\alpha)+\frac{T^\pm_{\beta\alpha}}{E_\alpha-E_\beta\pm i\epsilon} \\
[U^{\pm\dagger} U^\pm]_{\beta\alpha}=&\int d\gamma U^{\pm*}_{\gamma\beta} U^\pm_{\gamma\alpha} \\
=&\int d\gamma\left[\delta (\gamma-\beta)+\frac{T^\pm_{\gamma\beta}}{E_\beta -E_\gamma \pm i\epsilon}\right]^* \left[\delta (\gamma-\alpha)+\frac{T^\pm_{\gamma\alpha}}{E_\alpha-E_\gamma\pm i\epsilon}\right] \\
=&\delta(\beta-\alpha) +\left(\frac{T^\pm_{\alpha\beta}}{E_\beta -E_\alpha \pm i\epsilon}\right)^*+\frac{T^\pm_{\beta\alpha}}{E_\alpha -E_\beta \pm i\epsilon}+\int d\gamma \left(\frac{T^\pm_{\gamma\beta}}{E_\beta-E_\gamma \pm i\epsilon}\right)^* \frac{T^\pm_{\gamma\alpha}}{E_\alpha-E_\gamma \pm i\epsilon} \\
=&\delta(\alpha-\beta)
\end{align*}
が得られ,略記すればこれは$U^{\pm \dagger} U^\pm=1$となる.また条件式の両辺を複素共役すれば
\begin{align*}
\frac{T^\pm_{\alpha\beta}}{E_\beta -E_\alpha \pm i\epsilon}+\left(\frac{T^\pm_{\beta\alpha}}{E_\alpha -E_\beta \pm i\epsilon}\right)^*=-\int d\gamma \frac{T^\pm_{\gamma\beta}}{E_\beta-E_\gamma \pm i\epsilon} \left( \frac{T^\pm_{\gamma\alpha}}{E_\alpha-E_\gamma \pm i\epsilon} \right)^*
\end{align*}
が得られ,したがって
\begin{align*}
[U_{\pm}U_{\pm}^\dagger]_{\beta\alpha}=& \int d\gamma U^\pm_{\gamma\beta}U^{\pm *}_{\gamma\alpha} \\
=&\int d\gamma\left[\delta (\gamma-\beta)+\frac{T^\pm_{\gamma\beta}}{E_\beta -E_\gamma \pm i\epsilon}\right] \left[\delta (\gamma-\alpha)+\frac{T^\pm_{\gamma\alpha}}{E_\alpha-E_\gamma\pm i\epsilon}\right]^* \\
=&\delta(\beta-\alpha) +\frac{T^\pm_{\alpha\beta}}{E_\beta -E_\alpha \pm i\epsilon}+\left(\frac{T^\pm_{\beta\alpha}}{E_\alpha -E_\beta \pm i\epsilon}\right)^*+\int d\gamma \frac{T^\pm_{\gamma\beta}}{E_\beta-E_\gamma \pm i\epsilon} \left( \frac{T^\pm_{\gamma\alpha}}{E_\alpha-E_\gamma \pm i\epsilon} \right)^* \\
=&\delta(\beta-\alpha)
\end{align*}
を得る.これも略記すれば$U^\pm U^{\pm\dagger} =1$となる.したがってこれはユニタリーだ.したがって(3.1.17)より
\begin{align*}
\tensor{\Psi}{_\alpha^\pm}=&\int d\beta \left[\delta (\beta-\alpha)+\frac{T^\pm_{\beta\alpha}}{E_\alpha-E_\beta\pm i\epsilon}\right] \Phi_\beta \\
=&\int d\beta U^\pm_{\beta\alpha} \Phi_\beta
\end{align*}
と書けるから
\begin{align*}
(\tensor{\Psi}{_{\beta}^\pm},\tensor{\Psi}{_\alpha^\pm})=& \int d\alpha' d\beta' (U^\pm_{\beta'\beta}\Phi_{\beta'} , U^\pm_{\alpha'\alpha}\Phi_{\alpha'}) \\
=&\int d\alpha' d\beta' U^{\pm *}_{\beta'\beta} U^\pm_{\alpha'\alpha} \delta(\beta'-\alpha') \\
=&\int d\alpha' U^{\pm *}_{\alpha'\beta} U^\pm_{\alpha'\alpha} \\
=&\delta(\beta-\alpha)
\end{align*}
となって,規格直交であることが示すことができた.\par
$S$行列のユニタリー性を示すためには,(3.2.9)の両辺に$(E_\alpha -E_\beta\pm 2i\epsilon)^{-1}$ではなく$\delta(E_\beta-E_\alpha)$をかけることで,$f(x)\delta(x-a)=f(a)\delta(x-a)$の性質を使うと
\begin{align*}
T^{\pm *}_{\alpha\beta}\delta(E_\beta-E_\alpha)-T^\pm_{\beta\alpha}\delta(E_\beta-E_\alpha) =&\int d\gamma T^{\pm *}_{\gamma\beta}T^\pm_{\gamma\alpha} \left[\frac{E_\alpha-E_\beta \pm 2i\epsilon}{(E_\alpha-E_\gamma\pm i\epsilon)(E_\beta-E_\gamma\mp i\epsilon)}\right]\delta(E_\beta-E_\alpha) \\
=&\pm 2\pi i \int d\gamma T^{\pm *}_{\gamma\beta}T^\pm_{\gamma\alpha} \left[\frac{\epsilon}{\pi(E_\alpha-E_\gamma)^2+\epsilon^2}\right]\delta(E_\beta-E_\alpha) \\
=&\pm 2\pi i \int d\gamma T^{\pm *}_{\gamma\beta}T^\pm_{\gamma\alpha} \delta (E_\alpha-E_\gamma) \delta(E_\beta-E_\alpha) \quad \because (3.1.24)\\
=& \pm 2\pi i \int d\gamma T^{\pm *}_{\gamma\beta}T^\pm_{\gamma\alpha} \delta (E_\alpha-E_\gamma) \delta(E_\beta-E_\gamma) \\
\therefore \quad T^{+ *}_{\alpha\beta}\delta(E_\beta-E_\alpha)-T^+_{\beta\alpha}\delta(E_\beta-E_\alpha) =&2\pi i \int d\gamma T^{+ *}_{\gamma\beta}T^+_{\gamma\alpha} \delta (E_\alpha-E_\gamma) \delta(E_\beta-E_\gamma)
\end{align*}
が得られる.よって(3.2.7)の$S_{\beta\alpha}$は
\begin{align*}
[S^\dagger S]_{\beta\alpha}=&\int d\gamma S^*_{\gamma\beta}S_{\gamma\alpha} \\
=&\int d\gamma \left[\delta(\gamma-\beta)-2\pi i \delta(E_\gamma-E_\beta)T^+_{\gamma\beta}\right]^*\left[\delta(\gamma-\alpha)-2\pi i \delta(E_\gamma-E_\alpha)T^+_{\gamma\alpha}\right] \\
=&\delta(\beta-\alpha)+2\pi i \delta(E_\beta -E_\alpha)T^+_{\alpha\beta}-2\pi i \delta(E_\alpha-E_\beta)T^{+*}_{\beta\alpha} \\
&-4\pi^2\int d\gamma T^{\pm *}_{\gamma\beta}T^\pm_{\gamma\alpha} \delta (E_\alpha-E_\gamma) \delta(E_\beta-E_\gamma) \\
=&\delta(\beta-\alpha)
\end{align*}
を満たす.これは略記すれば$S^\dagger S=1$となる.条件式に両辺複素共役したものを用いれば同様に$SS^\dagger=1$も示せる.

\newpage


\subsection{$S$行列の対称性}
$S$行列の不変性が意味することや,どのような条件をハミルトニアンに課すとそのような不変性が保証されるかを考察する.

\vskip\baselineskip

\textbf{(A)ローレンツ不変性}\par
任意の非斉次ローレンツ$ISO(3,1)$変換$x\to \Lambda x+a$に対して,ユニタリー演算子$U(\Lambda,a)$を(3.1.1)のようにin状態またはout状態の\uwave{どちらかに}作用するとして定義することができる.理論がローレンツ不変であるとは,それは同じ演算子$U(\Lambda,a)$が(3.1.1)のようにin状態およびout状態の\uwave{両方}に作用することを意味する.そのとき,演算子$U(\Lambda,a)$はユニタリーであるから
\begin{align*}
S_{\beta\alpha}=\Bigl(\Psi_{\beta}^- , \Psi_\alpha^+\Bigr)=\Bigl(U(\Lambda,a)\Psi_{\beta}^- , U(\Lambda,a)\Psi_\alpha^+\Bigr)
\end{align*}
を満たす.よって(3.1.1)を用いると,$S$行列のローレンツ不変性(実際は共変性)が得られる.すなわち,任意のローレンツ変換$\tensor{\Lambda}{^\mu_\nu}$と並進$a^\mu$に対して(添え字$\alpha,\beta$等をあらわに書いて)
\begin{align*}
&S_{p_1',\sigma_1',n_1';p_2',\sigma_2',n_2';\cdots , \, p_1,\sigma_1,n_1;p_2,\sigma_2,n_2;\cdots} \\
=&\Bigl(U(\Lambda,a)\Psi_{p_1',\sigma_1',n_1';p_2',\sigma_2',n_2';\cdots}^- \, , \, U(\Lambda,a)\Psi_{p_1,\sigma_1,n_1;p_2,\sigma_2,n_2;\cdots}^+\Bigr) \\
=&\Biggl(\prod_i \left[e^{-ia_\mu (\Lambda p'_i)^\mu}\sqrt{\frac{(\Lambda p'_i)^0}{p'^0_i}} \sum_{\bar{\sigma}_i'}D_{\bar{\sigma}'_i\sigma'_i}^{(j'_i)}\Bigl(W(\Lambda,p'_i)\Bigr)\right] \Psi_{\Lambda p_1',\bar{\sigma}_1',n_1';\Lambda p_2',\bar{\sigma}_2',n_2';\cdots}^- \\
&\qquad \qquad \qquad , \prod_i \left[e^{-ia_\mu (\Lambda p_i)^\mu}\sqrt{\frac{(\Lambda p_i)^0}{p^0_i}} \sum_{\bar{\sigma}_i}D_{\bar{\sigma}_i\sigma_i}^{(j_i)}\Bigl(W(\Lambda,p_i)\Bigr)\right]\Psi_{\Lambda p_1,\bar{\sigma}_1,n_1;\Lambda p_2,\bar{\sigma}_2,n_2;\cdots}^+ \Biggr) \\
=&\exp\Bigl(-ia_\mu \tensor{\Lambda}{^\mu_\nu}(p_1^\nu+p_2^\nu+\cdots -p'^\nu_1-p'^\nu_2-\cdots) \Bigr) \\
&\qquad \times \sqrt{\frac{(\Lambda p_1)^0 (\Lambda p_2)^0\cdots (\Lambda p_1')^0(\Lambda p_2')^0}{p_1^0 p_2^0\cdots p'^0_1 p'^0_2\cdots }} \\
&\qquad \times \sum_{\bar{\sigma}_1 ,\bar{\sigma}_2,\cdots }D^{(j_1)}_{\bar{\sigma}_1\sigma_1}\Bigl( W(\Lambda,p_1) \Bigr)D^{(j_2)}_{\bar{\sigma}_2\sigma_2}\Bigl( W(\Lambda,p_2) \Bigr)\cdots \\
&\qquad \times \sum_{\bar{\sigma}_1' ,\bar{\sigma}_2',\cdots }D^{(j'_1)}_{\bar{\sigma}'_1\sigma'_1}\Bigl( W(\Lambda,p'_1) \Bigr)D^{(j'_2)}_{\bar{\sigma}'_2\sigma'_2}\Bigl( W(\Lambda,p'_2) \Bigr)\cdots \\
&\qquad \times S_{\Lambda p'_1,\bar{\sigma}'_1,n_1';\Lambda p'_2,\bar{\sigma}'_2,n_2';\cdots ,\Lambda p_1,\bar{\sigma}_1,n_1;\Lambda p_2,\bar{\sigma}_2,n_2;\cdots }
\end{align*}
となる.(これは定理ではなく定義であることに注意!)特に左辺は$a^\mu$に依らないので,右辺も同様に$a^\mu$に依ってはいけない.よって$S$行列は4元運動量が保存しなければゼロだ.したがって$S$行列の粒子間の相互作用を表す部分は(3.2.8)からの類推で
\begin{align*}
S_{\beta\alpha}-\delta(\beta-\alpha)=-2\pi i M_{\beta\alpha}\delta^4(p_\beta-p_\alpha)
\end{align*}
の形で書ける\footnote{係数因子は単に定義であり,Peskinなどの教科書では
\begin{align*}
S_{\beta\alpha}-\delta(\beta-\alpha)=+(2\pi)^4 i M_{\beta\alpha}\delta^4(p_\beta-p_\alpha)
\end{align*}
となっている.これは一つのデルタ関数とセットで$2\pi$が出てくることに対応した定義である.(規格化の定義$(\Psi_{p',\sigma'},\Psi_{p,\sigma})=2E_p(2\pi)^3\delta^3(\mathbf{p}'-\mathbf{p})\delta_{\sigma'\sigma}$の影響でもある.)ただしこの定義の違いで3.4節での散乱断面積の計算に影響が出るので気を付ける.また,$S$演算子の行列要素が$S_{\beta\alpha}$であることに対応して,行列要素が$M_{\beta\alpha}$であるような演算子$T$を導入し
\begin{align*}
S=1+iT
\end{align*}
と書く文脈もあることを覚えておく.このように書く場合,もっぱらpeskinのような定義が採用されている.}.(しかし,4.4節の(4.4.6)の後で書いてあるように,$M_{\beta\alpha}$自身に他のデルタ関数因子を含む項がある.)

\vskip\baselineskip

(3.3.1)は定理というより,$S$行列のローレンツ不変性の定義とみなせる.なぜなら,in状態とout状態の\uwave{両方}に(3.1.1)のように作用するユニタリー演算子が存在するのは,特別なハミルトニアンに対してのみだ.したがって,$S$行列のローレンツ不変性を保証するための,そのようなハミルトニアンに対する条件を定式化する必要がある.ここで一通りまとめておこう.\par
自由粒子状態に対して,in状態かout状態のどちらかに同じ作用を実現する演算子を定義することはできる.実際,線形空間$\mc{X}$に対して$A:\mc{X}\to \mc{X}$がある既知の線形演算子だとして,別の線形空間$\mc{Y}$との線形同型写像$\varphi:\mc{X}\to \mc{Y}$が存在するならば$B=\varphi\circ A \circ \varphi^{-1}:\mc{Y}\to \mc{Y}$は$\mc{Y}$に対して$A$と同じ作用をする線形演算子となる.($A,B$がそれぞれ$\mc{X},\mc{Y}$に表現として作用する演算子ならば,同値表現.)今回の場合,$\mc{X}$が自由粒子状態のなすヒルベルト空間であり,$\mc{Y}$はin状態あるいはout状態のどちらかの属するヒルベルト空間として対応付ける.今回の散乱理論の仮定では自由状態とin状態(およびout状態)は同型だとするから\footnote{前の脚注で書いた通り,QEDなどの長距離力を考える理論ではこの仮定はうまくいかないが,今回はそのような場合は考えない},自由粒子状態に作用するポアンカレ演算子$U_0(\Lambda,a)$に対してin状態あるいはout状態に作用するポアンカレ演算子$U(\Lambda,a)$が定義できる.このときの$\varphi$は(3.1.13)より$\Omega(\mp \infty)$である.しかし,例えばin状態に作用する$U_{\mathrm{in}}(\Lambda,a)=\Omega(-\infty) U_0(\Lambda,a)\Omega(-\infty)^{-1}$を定義できたからといって,それがout状態に対しても同じ演算子とは限らない.out状態に作用する演算子は$U_{\mathrm{out}}(\Lambda,a)=\Omega(+\infty)U_0(\Lambda,a)\Omega(+\infty)^{-1}$であり,一般に$\Omega(-\infty)\neq \Omega(+\infty)$だから$U_{\mathrm{out}}(\Lambda)\neq U_{\mathrm{in}}(\Lambda,a)$だ.しかし,$U_{\mathrm{out}}(\Lambda,a)=U_{\mathrm{in}}(\Lambda,a)=:U(\Lambda,a)$となる特別な理論が存在するとき,その理論をローレンツ不変な理論と呼ぶ.これは$\Omega(+\infty)=\Omega(-\infty)$を意味するわけではなく,実際この条件式を書き直すと,$U_0(\Lambda,a)^{-1}\Omega(+\infty)^{-1}\Omega(-\infty) U_0(\Lambda,a)=\Omega(+\infty)^{-1}\Omega(-\infty)$であり,(3.2.5)を用いるとこれは$S$演算子と$U_0(\Lambda,a)$の可換性$U_0 (\Lambda,a)^{-1}S U_0(\Lambda,a)=S$となる.\par
別の解釈からも同じ条件式を導く.つまり,(3.2.4)
\begin{align*}
S_{\beta\alpha}=(\Phi_\beta,S\Phi_\alpha)
\end{align*}
からスタートする.自由粒子状態$\Phi_\alpha$は1粒子状態の直積として変換し,したがって非斉次ローレンツ群の表現を与える.つまりこれらの状態に(3.1.1)の変換
\begin{align*}
U_0(\Lambda,a)\Phi_{p_1,\sigma_1,n_1;p_2,\sigma_2,n_2;\cdots}=&\exp\Bigl[-ia_\mu ((\Lambda p_1)^\mu+(\Lambda p_2)^\mu)+\cdots \Bigr] \sqrt{\frac{(\Lambda p_1)^0(\Lambda p_2)^0\cdots }{p^0_1p_2^0\cdots}} \\
&\times \sum_{\sigma_1,\sigma_2\cdots }D^{(j_1)}_{\sigma'_1\sigma_1}\Bigl(W(\Lambda,p_1)\Bigr)D^{(j_2)}_{\sigma'_2\sigma_2}\Bigl(W(\Lambda,p_2)\Bigr)\cdots \Phi_{\Lambda p_1,\sigma_1',n_1;\Lambda p_2,\sigma'_2,n_2\cdots }
\end{align*}
を引き起こすユニタリー演算子$U_0(\Lambda,a)$を常に定義できる.このユニタリー演算子が$S$演算子と
\begin{align*}
U_0(\Lambda,a)^{-1}S U_0(\Lambda,a)=S
\end{align*}
のように交換するなら(3.3.1)は成り立つ.実際(3.3.1)は
\begin{align*}
(\mathrm{LHS})=&(\Phi_\alpha , S \Phi_\beta) \\
(\mathrm{RHS})=&(U_0(\Lambda,a)\Phi_\beta,S U_0(\Lambda,a)\Phi_\alpha) \\
=&(\Phi_\beta,U_0(\Lambda,a)^{-1} S U_0(\Lambda,a)\Phi_\alpha)
\end{align*}
と書けるから,これが成り立つためには$S$と$U_0(\Lambda,a)$が交換すれば十分だ.\par
この条件は微小ローレンツ変換から生じる演算子で表現することもできる.2.4節と全く同様に,1組のエルミート演算子,すなわち運動量$\mathbf{P}_0$,角運動量$\mathbf{J}_0$,ブースト演算子$\mathbf{K}_0$が存在し,それらは$H_0$とともに自由粒子状態に作用したときに微小非斉次ローレンツ変換を生成する.上の条件は,$S$演算子はこれらの生成子と交換するということと同等だ.
\begin{align*}
[H_0,S]=[\mathbf{P}_0 ,S]=[\mathbf{J}_0,S]=[\mathbf{K}_0,S]=0
\end{align*}
演算子$H_0,\mathbf{P}_0,\mathbf{J}_0,\mathbf{K}_0$は$\Phi_\alpha$の微小非斉次ローレンツ変換を生成するから,それらは自動的に交換関係(2.4.18)~(2.4.24)を満足する.
\begin{align*}
[J_0^i ,J_0^j]=&i\epsilon_{ijk}J^k_0 \\
[J_0^i,K^j_0]=&i\epsilon_{ijk}K^k_0 \\
[K_0^i,K_0^j]=&-i\epsilon_{ijk}J_0^k \\
[J_0^i,P^j_0]=&i\epsilon_{ijk}P_0^k \\
[K_0^i,P_0^j]=&-iH_0 \delta_{ij} \\
[J_0^i,H_0]=&[P_0^i,H_0]=[P_0^i,P_0^j]=0 \\
[K_0^i,H_0]=&-iP_0^i
\end{align*}
同様にして,全ハミルトニアン$H$とともに,例えばin状態に変換(3.1.1)を生成する「正確な生成子」の組$\mathbf{P},\mathbf{J},\mathbf{K}$を定義できる.(前述の通り,これらの演算子が同じ変換をout状態に引き起こすことは自明ではない.)群の構造により,2.4節の計算を繰り返せばこれらの正確な生成子も同じ交換関係
\begin{align*}
[J^i ,J^j]=&i\epsilon_{ijk}J^k \\
[J^i,K^j]=&i\epsilon_{ijk}K^k \\
[K^i,K^j]=&-i\epsilon_{ijk}J^k \\
[J^i,P^j]=&i\epsilon_{ijk}P^k \\
[K^i,P^j]=&-iH \delta_{ij} \\
[J^i,H]=&[P^i,H]=[P^i,P^j]=0 \\
[K^i,H]=&-iP^i
\end{align*}
を満たすことが分かる.(この説の最初の説明と同様に,この真の生成子は,例えばin状態に作用する真の運動量演算子$\mathbf{P}$は自由粒子状態に作用する演算子$\mathbf{P}_0$と$\mathbf{P}=\Omega(- \infty) \mathbf{P}_0 \Omega(- \infty)^{-1}$で関係している.くどいようだが,さらに$\mathbf{P}=\Omega(+ \infty) \mathbf{P}_0 \Omega(+ \infty)^{-1}$を満たすかどうかは別問題だ.)\par
事実上,全ての既知の場の理論において,相互作用の効果は相互作用項$V$を自由ハミルトニアン$H_0$に付け加えることだけであり,運動量や角運動量を変えない\footnote{例外として,位相的にtwistされた場を含む理論がある.これについては夏学の菅野浩明「位相的弦理論と分配関数と数え上げ」p12を参照.twistとはSUSYでのR対称性をゲージ化して局所対称性へと格上げさせ,それに伴って生じるゲージ場をスピン接続と同一視する,という操作のことを指すらしい.実際にtwistによってフェルミオンのスピンが変化し,スピンゼロになるものとスピン1になるものが生じる.このような理論は位相場の理論であり,物理量は散乱振幅やエネルギースペクトルなどではなくコホモロジー類や位相的不変量などであり,そもそもWeinbergがこの本で定める「場の量子論」の定義に当てはまらない.もっぱら今回の範疇では考えないことにする.}.
\begin{align*}
H=H_0+V,\quad \mathbf{P}=\mathbf{P}_0,\quad \mathbf{J}=\mathbf{J}_0
\end{align*}
二組の交換関係(3.3.4)~(3.3.10),(3.3.11)~(3.3.17)とこれらの仮定から他の条件を導こう.(3.3.11)(3.3.14)は何の情報も与えない.(3.3.16)より,相互作用項は自由粒子の運動量および角運動量と
\begin{align*}
[V,\mathbf{P}_0]=[V,\mathbf{J}_0]=0
\end{align*}
のように交換することを要請する.この交換関係により,リップマン・シュウィンガー方程式(3.1.16)あるいはそれと同等な(3.1.13)から,in状態とout状態の\uwave{両方に}作用するとき,並進と回転を生成する演算子$\mathbf{P},\mathbf{J}$は単に$\mathbf{P}_0,\mathbf{J}_0$だということが容易にわかる.
\begin{align*}
\mathbf{P}=&\Omega(\mp \infty) \mathbf{P}_0 \Omega(\mp \infty)^{-1} \\
=& \lim_{t\to \mp \infty}\exp(i(H_0+V)t)\exp(-iH_0 t)\mathbf{P}_0 \exp(iH_0 t) \exp(-i(H_0+V)t) \\
=&\mathbf{P}_0 \\
\mathbf{J}=&\Omega(\mp \infty) \mathbf{J}_0 \Omega(\mp \infty)^{-1}=\mathbf{J}_0
\end{align*}
(この条件は(3.3.18)と同じことを言っているのではなく,例えば(3.3.11)~(3.3.17)がin状態に作用する正確な演算子だとしたら,もう一方のout状態の正確な演算子もそれと同じ演算子であることを示している.)同様にして$\mathbf{P}_0,\mathbf{J}_0$は(3.2.6)で定義される$U(t,t_0)=\exp(iH_0t)\exp(-iH(t-t_0))\exp(iH_0t_0)$と交換し,よって$S$演算子$U(\infty,-\infty)$とも交換することがわかる.
\begin{align*}
[\mathbf{P}_0,S]=[\mathbf{J}_0,S]=0
\end{align*}
さらに,(3.2.7)の二項ともにエネルギー保存のデルタ関数があるから
\begin{align*}
E_\beta S_{\beta\alpha}=E_\alpha S_{\beta\alpha}
\end{align*}
を満たしており,したがって
\begin{align*}
(\mathrm{LHS})=&E_\beta (\Phi_\beta ,S \Phi_\alpha) \\
=&(H_0\Phi_\beta,S \Phi_\alpha) \\
=&(\Phi_\beta,H_0 S \Phi_\alpha) \\
(\mathrm{RHS})=&E_\alpha (\Phi_\beta ,S \Phi_\alpha) \\
=&(\Phi_\beta,S H_0\Phi_\alpha) \\
\therefore \quad [H_0,S]=&0
\end{align*}
が得られ,$S$演算子と$H_0$が交換することもわかる.最後に示さなければならないのはブースト生成子$\mathbf{K}_0$だけだ!

\vskip\baselineskip

ブースト生成子も$\mathbf{K}=\mathbf{K}_0$と等しいとおきたいが,それは不可能である.なぜなら,もしそう仮定すると(3.3.15)と(3.3.8)から直ちに$H=H_0$が導かれ,これは相互作用のある理論では明らかに正しくない.このように,相互作用$V$を自由ハミルトニアン$H_0$に付け加えると,ブースト生成子は同時に補正$\mathbf{W}$が付け加わることとなる.
\begin{align*}
\mathbf{K}=\mathbf{K}_0+\mathbf{W}
\end{align*}
残った交換関係のうち,(3.3.17)を用いると\footnote{$[K_i,H]=-iP_i$が本文と誤植があるのにここからしばらく何度も使うので,逐一誤植に気を付けること.}
\begin{align*}
(\mathrm{LHS})=&[\mathbf{K},H]\\
=&[\mathbf{K}_0,H_0]+[\mathbf{K}_0,V]+[\mathbf{W},H] \\
=&-i\mathbf{P}_0+[\mathbf{K}_0,V]+[\mathbf{W},H] \\
(\mathrm{RHS})=&-i\mathbf{P}=-i\mathbf{P}_0 \\
\therefore \quad [\mathbf{K}_0,V]=&-[\mathbf{W},H]
\end{align*}
を得る.この条件自体には$[\mathbf{K}_0,S]=0$を示すための有益な情報は含まれていない.なぜなら,この交換関係より
\begin{align*}
(\Psi_\beta,[\mathbf{K}_0,V]\Psi_\alpha)=&-(\Psi_\beta,[\mathbf{W},H]\Psi_\alpha) \\
=&(\Psi_\beta ,H\mathbf{W}\Psi_\alpha)-(\Psi_\beta ,\mathbf{W}H\Psi_\alpha) \\
=&(E_\beta-E_\alpha)(\Psi_\beta ,\mathbf{W}\Psi_\alpha) \\
\therefore \quad (\Psi_\beta ,\mathbf{W}\Psi_\alpha)=&\frac{(\Psi_\beta,[\mathbf{K}_0,V]\Psi_\alpha)}{E_\beta-E_\alpha}
\end{align*}
であるから,逆に任意の相互作用$V$が与えられれば,それに対して$H$の固有状態$\Psi_\alpha,\Psi_\beta$との間の$\mathbf{W}$の行列要素を$(\Psi_\beta,[\mathbf{K}_0,V]\Psi_\alpha)/(E_\beta-E_\alpha)$と与えることにより常に演算子$\mathbf{W}$を定義することができるからだ.$[\mathbf{K}_0,S]=0$を証明するためには,$\mathbf{W}$の行列要素がエネルギーの滑らかな関数であり,特に上のような$(E_\beta-E_\alpha)^{-1}$の形の特異点を持たない,という要請を課すことで(3.3.21)は意味を持つようになり,$[\mathbf{K}_0,S]=0$がわかる.\par
これを証明するために,$\mathbf{K}_0$と(3.2.6)で定義される演算子$U(t,t_0)$の交換子を,有限な$t,t_0$について考える.(3.3.10)と$\mathbf{P}_0$が$H_0$と交換する(3.3.9)から,BCH公式より
\begin{align*}
[\mathbf{K}_0,\exp(iH_0t)] =& -\Bigl(\exp(iH_0 t)\mathbf{K}_0 \exp(-iH_0 t)-\mathbf{K}_0\Bigr) \exp(iH_0 t) \\
=&-\left(\mathbf{K}_0 +it[H_0 ,\mathbf{K}_0]+\frac{1}{2}(it)^2[H_0,[H_0,\mathbf{K}_0]]+\cdots -\mathbf{K}_0 \right)\exp(iH_0 t) \\
=&-it(+i\mathbf{P}_0)\exp(iH_0 t) \\
=&+t \mathbf{P}_0 \exp(iH_0 t)
\end{align*}
が得られる.ここで(3.3.9)(3.3.10)より$n\geq 2$の$n$重交換子
\begin{align*}
[H_0,\cdots ,[H_0,[H_0,\mathbf{K}_0]]\cdots]=[H_0,\cdots,[H_0 ,+i\mathbf{P}_0]\cdots] 
\end{align*}
がゼロになることを使った.一方,(3.3.21)($\Leftrightarrow$(3.3.17))(3.3.16)から,同様に
\begin{align*}
[\mathbf{K},\exp(iHt)]=&+t\mathbf{P} \exp(iHt) \\
=&+t \mathbf{P}_0\exp(iH t)
\end{align*}
が得られる.運動量演算子は$\mathbf{K}_0$と$U$の交換子において相殺して
\begin{align*}
\Bigl[\mathbf{K}_0,U(\tau,\tau_0)\Bigr]=&\Bigl[\mathbf{K}_0 ,e^{iH_0\tau}e^{-iH(\tau-\tau_0)}e^{-iH_0 \tau_0}\Bigr] \\
=&\Bigl[\mathbf{K}_0 ,e^{iH_0\tau}\Bigr]e^{-iH(\tau-\tau_0)}e^{-iH_0 \tau_0} \\
&+ e^{iH_0\tau}\Bigl[\mathbf{K}_0,e^{-iH(\tau-\tau_0)}\Bigr]e^{-iH_0 \tau_0} \\
&+e^{iH_0\tau}e^{-iH(\tau-\tau_0)}\Bigl[\mathbf{K}_0 ,e^{-iH_0 \tau_0}\Bigr] \\
=&\Bigl(+\tau \mathbf{P}_0e^{iH_0\tau}\Bigr)e^{-iH(\tau-\tau_0)}e^{-iH_0 \tau_0} \\
&+ e^{iH_0\tau}\Bigl[\mathbf{K}-\mathbf{W},e^{-iH(\tau-\tau_0)}\Bigr]e^{-iH_0 \tau_0} \\
&+e^{iH_0\tau}e^{-iH(\tau-\tau_0)}\Bigl(-\tau_0 \mathbf{P}_0e^{-iH_0\tau_0}\Bigr) \\
=&\Bigl(+\tau \mathbf{P}_0e^{iH_0\tau}\Bigr)e^{-iH(\tau-\tau_0)}e^{-iH_0 \tau_0} \\
&+ e^{iH_0\tau}\Bigl[\mathbf{K},e^{-iH(\tau-\tau_0)}\Bigr]e^{-iH_0 \tau_0} -e^{iH_0\tau}\Bigl[\mathbf{W},e^{-iH(\tau-\tau_0)}\Bigr]e^{-iH_0 \tau_0} \\
&+e^{iH_0\tau}e^{-iH(\tau-\tau_0)}\Bigl(-\tau_0 \mathbf{P}_0e^{-iH_0\tau_0}\Bigr) \\
=&\Bigl(+\tau \mathbf{P}_0e^{iH_0\tau}\Bigr)e^{-iH(\tau-\tau_0)}e^{-iH_0 \tau_0} \\
&+ e^{iH_0\tau}\Bigl(-(\tau-\tau_0)\mathbf{P}_0e^{-iH(\tau-\tau_0)}\Bigr)e^{-iH_0 \tau_0} \\
&-e^{iH_0\tau}\Bigl[\mathbf{W},e^{-iH(\tau-\tau_0)}\Bigr]e^{-iH_0 \tau_0} \\
&+e^{iH_0\tau}e^{-iH(\tau-\tau_0)}\Bigl(-\tau_0 \mathbf{P}_0e^{-iH_0\tau_0}\Bigr) \\
=&-e^{iH_0\tau}\Bigl[\mathbf{W},e^{-iH(\tau-\tau_0)}\Bigr]e^{-iH_0 \tau_0} \quad \because [\mathbf{P}_0,H_0]=[\mathbf{P}_0,H]=0 \\
=&-e^{iH_0\tau}\mathbf{W}e^{-iH(\tau-\tau_0)}e^{-iH_0 \tau_0} \\
&+e^{iH_0\tau}e^{-iH(\tau-\tau_0)}\mathbf{W}e^{-iH_0 \tau_0} \\
=&-\Bigl(e^{iH_0\tau}\mathbf{W}e^{-iH_0\tau}\Bigr)e^{iH_0\tau}e^{-iH(\tau-\tau_0)}e^{-iH_0 \tau_0} \\
&+e^{iH_0\tau}e^{-iH(\tau-\tau_0)}e^{-iH_0\tau_0}\Bigl(e^{iH_0\tau_0}\mathbf{W}e^{-iH_0 \tau_0}\Bigr) \\
=&-\mathbf{W}(\tau)U(\tau,\tau_0)+U(\tau,\tau_0)\mathbf{W}(\tau_0)
\end{align*}
ここで
\begin{align*}
\mathbf{W}(\tau):=\exp(iH_0\tau)\mathbf{W}\exp(-iH_0\tau)
\end{align*}
とした.さて,in状態とout状態の定義より$\tau \to + \infty,\tau_0\to -\infty$の極限で
\begin{align*}
&\lim_{\tau\to +\infty }\lim_{\tau_0\to -\infty} \left(\int d\beta g(\beta) \Phi_\beta ,\mathbf{W}(\tau) U(\tau,\tau_0)\int d\alpha g(\alpha) \Phi_\alpha \right) \\
=&\lim_{\tau\to +\infty }\lim_{\tau_0\to -\infty} \left(\int d\beta g(\beta) \Phi_\beta ,\mathbf{W}(\tau) \Omega(\tau)^\dagger \int d\alpha g(\alpha) \Omega(\tau_0) \Phi_\alpha \right) \\
=&\lim_{\tau\to +\infty } \left(\int d\beta g(\beta) \Phi_\beta ,\mathbf{W}(\tau) e^{iH_0\tau}e^{-iH\tau}\int d\alpha g(\alpha) \Psi_\alpha^+ \right) \\
=&\lim_{\tau\to +\infty } \left(\int d\beta g(\beta) \Phi_\beta ,\mathbf{W}(\tau) e^{iH_0\tau}\int d\alpha g(\alpha) e^{-iE_\alpha \tau}\Psi_\alpha^+ \right) \\
=&\lim_{\tau\to +\infty } \left(\int d\beta g(\beta) e^{-iE_\beta \tau} \Phi_\beta ,\mathbf{W} \int d\alpha g(\alpha) e^{-iE_\alpha \tau}\Psi_\alpha^+ \right) \\
=&\lim_{\tau\to +\infty } \left(\int d\beta g(\beta) e^{-iE_\beta \tau} \Psi_\beta^- ,\mathbf{W} \int d\alpha g(\alpha) e^{-iE_\alpha \tau}\Psi_\alpha^+ \right) \\
=&\lim_{\tau\to +\infty } \int d\alpha d\beta g(\alpha) g^*(\beta) e^{-i(E_\alpha-E_\beta) \tau}\left(\Psi_\beta^- ,\mathbf{W} \Psi_\alpha^+ \right)
\end{align*}
となるから,もし$H$の任意のエネルギー固有状態間の$\mathbf{W}$の行列要素$(\Psi_\beta,\mathbf{W}\Psi_\alpha)$がエネルギーの滑らかな関数だと仮定\footnote{本文中では$H_0$のエネルギー固有状態間について,としているがそれではうまくいかないと思う.その場合$\mathbf{W}(\tau)$の行列要素が$\tau \to \pm \infty$でゼロになることはリーマン・ルベーグの補題からやはり簡単に示せるが,今回示したいのは$\mathbf{W}(\tau)U(\tau,\tau_0)$などの形だからだ.}すれば,これは$\tau + \infty$の極限でリーマン・ルベーグの補題によりゼロになる.同様に
\begin{align*}
&\lim_{\tau\to +\infty }\lim_{\tau_0\to -\infty} \left(\int d\beta g(\beta) \Phi_\beta ,U(\tau,\tau_0)\mathbf{W}(\tau_0) \int d\alpha g(\alpha) \Phi_\alpha \right) \\
=&\lim_{\tau\to +\infty }\lim_{\tau_0\to -\infty}\left(\int d\beta g(\beta) \Omega(\tau)\Phi_\beta ,\Omega(\tau_0)\mathbf{W}(\tau_0) \int d\alpha g(\alpha)\Phi_\alpha \right) \\
=&\lim_{\tau_0\to -\infty } \left(\int d\beta g(\beta) \Psi^-_\beta ,e^{iH\tau_0} e^{-iH_0\tau_0} \mathbf{W}(\tau_0) \int d\alpha g(\alpha) \Phi_\alpha \right) \\
=&\lim_{\tau_0\to -\infty } \left(\int d\beta g(\beta)e^{-iE_\beta \tau_0} \Psi^-_\beta ,e^{-iH_0\tau_0} \mathbf{W}(\tau_0) \int d\alpha g(\alpha) \Phi_\alpha \right) \\
=&\lim_{\tau_0\to -\infty } \left(\int d\beta g(\beta) e^{-iE_\beta \tau_0} \Psi^-_\beta ,\mathbf{W} \int d\alpha g(\alpha) e^{-iE_\alpha \tau}\Phi_\alpha \right) \\
=&\lim_{\tau_0\to -\infty } \left(\int d\beta g(\beta) e^{-iE_\beta \tau_0} \Psi_\beta^- ,\mathbf{W} \int d\alpha g(\alpha) e^{-iE_\alpha \tau}\Psi_\alpha^+ \right) \\
=&\lim_{\tau_0\to -\infty } \int d\alpha d\beta g(\alpha) g^*(\beta) e^{-i(E_\alpha-E_\beta) \tau_0}\left(\Psi_\beta^- ,\mathbf{W} \Psi_\alpha^+ \right)
\end{align*}
これもゼロになる.したがって(3.3.22)の左辺は任意の自由粒子状態の滑らかな重ね合わせに対して$\tau \to +\infty ,\tau_0 \to -\infty$でゼロになる\footnote{$E_\alpha=E_\beta$部分から有限の寄与が出てくるかもしれないが,その部分は指数が消えて$\tau,\tau_0$依存性がなくなるため,第一項目と第二項目でキャンセルする.}.したがって右辺はその極限のもとでゼロ演算子とできて
\begin{align*}
0=\lim_{\substack{\tau \to +\infty \\ \tau_0 \to -\infty }}\Bigl[\mathbf{K}_0,U(\tau,\tau_0)\Bigr]=[\mathbf{K}_0,S]
\end{align*}
を与える.これが示したいことだった.すなわち,(3.3.21)と,$\mathbf{W}$の行列要素のエネルギーについての滑らかさの条件は,$S$行列のローレンツ不変性(3.3.3)の十分条件を構成する.この滑らかさの条件は自然なものだ.なぜなら,まさに$S$行列の考え方を数学的に正当化するためには$V(t)$が$t\to \pm \infty$でゼロになることが要求され,そのために必要な$V$の行列要素の条件に,この条件は含まれているらしい.

\vskip\baselineskip

以上で(3.3.21)が示すことができた.最初に述べた通り,これらの「真の生成子」がこの条件を満たすことと,in状態とout状態に同じ演算子で作用することは同値である.すなわち
\begin{align*}
\mathbf{P}=&\Omega(\mp \infty) \mathbf{P}_0 \Omega(\mp \infty)^{-1} \\
\mathbf{J}=&\Omega(\mp \infty) \mathbf{J}_0 \Omega(\mp \infty)^{-1} \\
\mathbf{K}=&\Omega(\mp \infty) \mathbf{K}_0 \Omega(\mp \infty)^{-1} \\
H=&\Omega(\mp \infty) H_0 \Omega(\mp \infty)^{-1}
\end{align*}
である.

\vskip\baselineskip

\textbf{(B)内部対称性}\par
原子物理学における中性子と陽子の入れ替えに関する対称性や,粒子と反粒子の間の荷電共役対称性などのように,ローレンツ不変性とは直接関係がなく,全ての慣性系で同じに見える様々な対称性が存在する.そのような対称性変換$T : \mc{R} \to T\mc{R}$は物理的状態のヒルベルト空間にユニタリー演算子$U(T)$として作用し,粒子の種類を記述する添え字$n$について
\begin{align*}
U(T)\Psi_{p_1 \sigma_1 n_1;p_2 \sigma_2 n_2;\cdots}=&\prod_i \left[\sum_{\bar{n}_i}\mc{D}_{\bar{n}_i n_i}(T) \right]\Psi_{p_1 \sigma_1 \bar{n}_1;p_2 \sigma_2 \bar{n}_2;\cdots} \\
=&\sum_{\bar{n}_1\bar{n}_2\cdots} \mc{D}_{\bar{n}_1 n_1}(T)\mc{D}_{\bar{n}_2 n_2}(T)\cdots \Psi_{p_1 \sigma_1 \bar{n}_1;p_2 \sigma_2 \bar{n}_2;\cdots}
\end{align*}
のように線形変換を引き起こす.2.2節での一般的な議論に従って,$U(T)$は群の乗法則
\begin{align*}
U(\bar{T})U(T)=U(\bar{T}T)
\end{align*}
を満たさねばならない\footnote{もちろん射影表現の場合も存在するが,その場合は超選択則により禁止される重ね合わせを除けば物理的状態空間上(射影空間上)でこの表式が許される}.ここで$\bar{T}T$は最初に$T$を実行し,それからある別の変換$\bar{T}$を実行して得られる変換である.(3.3.29)に$U(\bar{T})$を作用させると,
\begin{align*}
U(\bar{T})U(T)\Psi_{p_1 \sigma_1 n_1;p_2 \sigma_2 n_2;\cdots}=&\sum_{\bar{n}_1\bar{n}_2\cdots} \mc{D}_{\bar{n}_1 n_1}(T)\mc{D}_{\bar{n}_2 n_2}(T)\cdots U(\bar{T})\Psi_{p_1 \sigma_1 \bar{n}_1;p_2 \sigma_2 \bar{n}_2;\cdots} \\
=&\sum_{\substack{\bar{n}_1\bar{n}_2\cdots \\ \bar{n}'_1\bar{n}'_2\cdots}} \mc{D}_{\bar{n}_1 n_1}(T)\mc{D}_{\bar{n}_2 n_2}(T)\cdots \mc{D}_{\bar{n}'_1 \bar{n}_1}(\bar{T})\mc{D}_{\bar{n}'_2 \bar{n}_2}(\bar{T})\cdots\Psi_{p_1 \sigma_1 \bar{n}'_1;p_2 \sigma_2 \bar{n}'_2;\cdots} \\
=&\sum_{\bar{n}_1\bar{n}_2\cdots} \Bigl(\mc{D}(\bar{T})\mc{D}(T)\Bigr)_{\bar{n}_1 n_1}\Bigl(\mc{D}(\bar{T})\mc{D}(T)\Bigr)_{\bar{n}_2 n_2}\cdots \Psi_{p_1 \sigma_1 \bar{n}_1;p_2 \sigma_2 \bar{n}_2;\cdots} \\
=U(\bar{T}T) \Psi_{p_1 \sigma_1 n_1;p_2 \sigma_2 n_2;\cdots} =&\sum_{\bar{n}_1\bar{n}_2\cdots} \mc{D}_{\bar{n}_1 n_1}(\bar{T}T)\mc{D}_{\bar{n}_2 n_2}(\bar{T}T)\cdots \Psi_{p_1 \sigma_1 \bar{n}_1;p_2 \sigma_2 \bar{n}_2;\cdots}
\end{align*}
となる.したがって表現行列$\mc{D}$は群と同じ乗法則
\begin{align*}
\mc{D}(\bar{T})\mc{D}(T)=\mc{D}(\bar{T}T)
\end{align*}
を満たすことが分かる.また,$U(T)$を二つの異なるin状態(あるいはout状態)に作用させて得られる状態のスカラー積をとれば,規格化条件(3.1.2)より
\begin{align*}
&\Bigl( \Psi_{p_1' \sigma_1' n_1';p'_2 \sigma'_2 n'_2;\cdots} , \Psi_{p_1 \sigma_1 n_1;p_2 \sigma_2 n_2;\cdots} \Bigr)=\delta^3(\mathbf{p}'_1-\mathbf{p}_1)\delta_{\sigma'_1\sigma_1}\delta_{n_1'n_1}\delta^3(\mathbf{p}'_2-\mathbf{p}_2)\delta_{\sigma'_2\sigma_2}\delta_{n'_2n_2}\cdots \pm[置換] \\
=&\Bigl(U(T)\Psi_{p_1 \sigma_1 n_1;p_2 \sigma_2 n_2;\cdots} , U(T)\Psi_{p_1 \sigma_1 n_1;p_2 \sigma_2 n_2;\cdots}\Bigr) \\
=&\sum_{\bar{n}'_1\bar{n}'_2\cdots} \mc{D}^*_{\bar{n}'_1 n'_1}(\bar{T}T)\mc{D}^*_{\bar{n}'_2 n'_2}(\bar{T}T)\cdots \sum_{\bar{n}_1\bar{n}_2\cdots} \mc{D}_{\bar{n}_1 n_1}(\bar{T}T)\mc{D}_{\bar{n}_2 n_2}(\bar{T}T)\cdots \\
&\times \Bigl( \Psi_{p_1' \sigma_1' \bar{n}_1';p'_2 \sigma'_2 \bar{n}'_2;\cdots} , \Psi_{p_1 \sigma_1 \bar{n}_1;p_2 \sigma_2 \bar{n}_2;\cdots} \Bigr) \\
=&\sum_{\bar{n}'_1\bar{n}'_2\cdots} \mc{D}^\dagger_{n'_1\bar{n}'_1}(\bar{T}T)\mc{D}^\dagger_{n'_2\bar{n}'_2}(\bar{T}T)\cdots \sum_{\bar{n}_1\bar{n}_2\cdots} \mc{D}_{\bar{n}_1 n_1}(\bar{T}T)\mc{D}_{\bar{n}_2 n_2}(\bar{T}T)\cdots \\
&\times \delta^3(\mathbf{p}'_1-\mathbf{p}_1)\delta_{\sigma'_1\sigma_1}\delta_{\bar{n}_1'\bar{n}_1}\delta^3(\mathbf{p}'_2-\mathbf{p}_2)\delta_{\sigma'_2\sigma_2}\delta_{\bar{n}'_2\bar{n}_2}\cdots \pm[置換] \\
=&\delta^3(\mathbf{p}'_1-\mathbf{p}_1)\delta_{\sigma'_1\sigma_1}(\mc{D}^\dagger (T)\mc{D}(T))_{n_1'n_1}\delta^3(\mathbf{p}'_2-\mathbf{p}_2)\delta_{\sigma'_2\sigma_2}(\mc{D}^\dagger(T)\mc{D}(T))_{n'_2n_2}\cdots \pm[置換]
\end{align*}
を得る.したがって$\mc{D}(T)$はユニタリーでなければならないことがわかる.
\begin{align*}
\mc{D}^\dagger(T)=\mc{D}^{-1}(T)
\end{align*}
以上より$\mc{D}$は対称性変換群のユニタリー表現をなす.最後に,$U(T)$をout状態とin状態に作用させて得られる状態のスカラー積をとると,$\mc{D}$は
\begin{align*}
&S_{p_1'\sigma_1'n_1';p_2'\sigma_2'n_2';\cdots , \, p_1\sigma_1n_1;p_2\sigma_2n_2;\cdots} \\
=&\Bigl(U(T)\Psi_{p_1',\sigma_1',n_1';p_2',\sigma_2',n_2';\cdots}^- \, , \, U(T)\Psi_{p_1,\sigma_1,n_1;p_2,\sigma_2,n_2;\cdots}^+\Bigr) \\
=&\Biggl(\prod_i \left[ \sum_{\bar{n}_i'}\mc{D}_{\bar{n}'_i n'_i}(T)\right] \Psi_{p'_1 \sigma'_1 \bar{n}_1';p_2',\sigma_2',\bar{n}_2';\cdots}^- , \prod_i \left[ \sum_{\bar{n}_i'}\mc{D}_{\bar{n}'_i n'_i}(T)\right] \Psi_{p_1 \sigma_1 \bar{n}_1;p_2,\sigma_2,\bar{n}_2;\cdots}^+ \Biggr) \\
=&\sum_{\bar{N}'_1\bar{N}'_2\cdots} \sum_{\bar{N}_1\bar{N}_2\cdots}\mc{D}^*_{\bar{N}'_1n'_1}(\bar{T}T)\mc{D}^*_{\bar{N}'_2 n'_2}(\bar{T}T)\cdots \mc{D}_{\bar{N}_1 n_1}(\bar{T}T)\mc{D}_{\bar{N}_2 n_2}(\bar{T}T)\cdots S_{p_1'\sigma_1'\bar{N}_1';p_2'\sigma_2'\bar{N}_2';\cdots , \, p_1\sigma_1\bar{N}_1;p_2\sigma_2\bar{N}_2;\cdots}
\end{align*}
の意味で,$S$行列と交換する.前と同じく,これは「内部対称性変換$T$に関して理論が不変である」ことの\uwave{定義}だ.なぜなら,この式を導くためには\uwave{同じ}ユニタリー演算子$U(T)$が変換(3.3.39)をin状態とout状態の両方に引き起こすことを示す必要があるからだ.これは,まず自由状態にこれらの変換
\begin{align*}
U_0(T)\Phi_{p_1 \sigma_1 n_1;p_2 \sigma_2 n_2;\cdots}=&\prod_i \left[\sum_{\bar{N}_i}\mc{D}_{\bar{N}_i n_i}(T) \right]\Phi_{p_1 \sigma_1 \bar{N}_1;p_2 \sigma_2 \bar{N}_2;\cdots} \\
=&\sum_{\bar{N}_1\bar{N}_2\cdots} \mc{D}_{\bar{N}_1 n_1}(T)\mc{D}_{\bar{N}_2 n_2}(T)\cdots \Phi_{p_1 \sigma_1 \bar{N}_1;p_2 \sigma_2 \bar{N}_2;\cdots}
\end{align*}
のように引き起こし,かつハミルトニアンの自由粒子部分と相互作用部分の両方と
\begin{align*}
U^{-1}_0(T)H_0 U_0(T)=H_0 \\
U_0^{-1}(T)V U_0(T)=V
\end{align*}
のように交換する,所謂「摂動を受けない」変換演算子$U_0(T)$が存在すれば,これは正しい.実際,これを満たす$U_0(T)$を定義できればin状態(あるいはout状態)に同じ作用をする
\begin{align*}
U_{\mathrm{in}}(T):=\Omega(-\infty)^{-1}U_0(T)\Omega(-\infty) \\
U_{\mathrm{out}}(T):=\Omega(+\infty)^{-1}U_0(T)\Omega(+\infty)
\end{align*}
を定義することができる.しかしこれは条件(3.3.35)(3.3.36)により$[U_0,H]=[U_0,H_0]=0$を満たすから$U_0(T)$は(3.1.14)の$\Omega(\tau)=e^{iH\tau}e^{-iH_0\tau}$と可換となり
\begin{align*}
U_{\mathrm{in}}(T)=U_{\mathrm{out}}=U_0(T)=:U(T)
\end{align*}
が得られる.したがって演算子$U_0(T)$は変換(3.3.29)を自由粒子状態だけでなくin状態とout状態の\uwave{両方}に同じ作用を引き起こす.$U(T)$を$U_0(T)$で定義して(3.3.33)を導ける.\par
内部対称性は粒子の種類$n$を入れ替えるような変換\uwave{だけ}から構成され,ローレンツ対称性は運動量$p$とスピン(あるいはヘリシティ)$\sigma$を変えるような変換\uwave{だけ}から構成される.ではこの二つを混ぜ込むような対称性は存在しないのか?と考えるのは当然である.しかしこれは24.B節で示すコールマン・マンデューラの定理により(いくつかの物理的に妥当な仮定を課せば),物理的状態の$p,\sigma,n$を混ぜ込むような対称性群は必ずローレンツ対称性と内部対称性の直積で表され,非自明にこれらを混ぜ込むようなものは(超対称性を除き)存在しないことが示される.


\vskip\baselineskip

物理的に非常に重要な特別な場合は,$T$が単一のパラメータ$\theta$の関数で
\begin{align*}
T(\bar{\theta})T(\theta)=T(\bar{\theta}+\theta)
\end{align*}
を満たす1パラメータのリー群となっている場合だ.2.2節で示したように,この場合のヒルベルト空間への演算子は,$Q$をエルミート演算子として
\begin{align*}
U\Bigl(T(\theta)\Bigr)=\exp(iQ\theta)
\end{align*}
の形をとらねばならない.同様にユニタリー行列$\mc{D}(T)$は,$q_n$を粒子の種類$n$に依存する実数の組として
\begin{align*}
\mc{D}_{n'n}\Bigl(T(\theta)\Bigr)=\delta_{n'n}\exp(iq_n \theta)
\end{align*}
の形をとらねばならない\footnote{というより,そうなるように基底を選ぶことができる.$\mc{D}(T(\theta))$はユニタリー行列となるから,補遺Cで示したように$P^{-1} \mc{D}(T) P$が対角行列で,その対角成分が$e^{iq_n}$となるようにユニタリー行列$P$を選ぶことができる.すなわち基底を$\Psi'_{p\sigma n}:=\sum_{n'}P_{n'n}\Psi_{p\sigma n'}$と選べばよい.
\begin{align*}
U(T(\theta))\Psi'_{p\sigma n}=\sum_{N}(P^{-1}\mc{D}(T(\theta))P)_{Nn}\Psi'_{p\sigma N}=\sum_{N}\delta_{Nn}\exp(iq_n \theta)\Psi'_{p\sigma N}=\exp(iq_n \theta)\Psi'_{p\sigma n}
\end{align*}
この線形結合の添え字は粒子の種類$n$のみだから,この基底の変更はローレンツ対称性に関する議論を何ら変更しない.ただし,これはもちろん対称性変換$T$が単一のパラメータ$\theta$にのみ依存する場合であり,一般に(2.2.26)のように複数パラメータに依存する場合は対角化できない.(当たり前だが,カルタン部分代数のところは同時対角化できる.例えば$SU(2)$なら$t_3$,$SU(3)$なら$\lambda_3$と$\lambda_8$だけ対角化するように基底を選べる.)}.ここで(3.3.33)から
\begin{align*}
&S_{p_1'\sigma_1'n_1';p_2'\sigma_2'n_2';\cdots , \, p_1\sigma_1n_1;p_2\sigma_2n_2;\cdots} \\
=&\sum_{\bar{N}'_1\bar{N}'_2\cdots} \sum_{\bar{N}_1\bar{N}_2\cdots}\mc{D}^*_{\bar{N}'_1n'_1}(\bar{T}T)\mc{D}^*_{\bar{N}'_2 n'_2}(\bar{T}T)\cdots \mc{D}_{\bar{N}_1 n_1}(\bar{T}T)\mc{D}_{\bar{N}_2 n_2}(\bar{T}T)\cdots S_{p_1'\sigma_1'\bar{N}_1';p_2'\sigma_2'\bar{N}_2';\cdots , \, p_1\sigma_1\bar{N}_1;p_2\sigma_2\bar{N}_2;\cdots} \\
=&\exp\left[i\theta(-q_{n'_1}-q_{n'_2}-\cdots +q_{n_1}+q_{n_2}+\cdots )\right]S_{p_1'\sigma_1'n_1';p_2'\sigma_2'n_2';\cdots , \, p_1\sigma_1n_1;p_2\sigma_2n_2;\cdots}
\end{align*}
を与える.左辺に$\theta$依存性がないから,右辺もそうでなくてはならず,したがって
\begin{align*}
q_{n'_1}+q_{n'_2}+\cdots =q_{n_1}+q_{n_2}+\cdots 
\end{align*}
でなければ$S_{\beta\alpha}$はゼロになる.これは単に全$q$量の保存則を表している.そのような保存則の典型的な例は$U(1)_{\mathrm{em}}$電荷の保存則である.また,知られている全ての過程はバリオン数$B$(陽子,中性子,ハイペロン等のバリオンの数から,それらの反粒子の数を引いたもの)およびレプトン数$L$(電子,ミューオン,$\tau$粒子,またそれらのニュートリノなどのレプトンの数からそれらの反粒子の数を引いたもの)を保存する.ただし4巻で見るように,これらの保存則は非常に良い近似に過ぎず,実際は$B-L$が保存量になっていると考えられている.\par
また,この種の保存則の中で,明らかに近似的にしか成立しないものもある.この例としては,ロチェスターとバトラーのストレンジネス保存則がある.$K^+,K^0$中間子にはストレンジネス$+1$が\footnote{ストレンジネスは,現在はストレンジクォーク$s$の\uwave{反粒子$\bar{s}$の数}(から$s$の数を引いたもの)として定義される.歴史的に$K^+$にストレンジネス+1を割り当ててしまったがために,このような奇妙な割り当てがされてしまうようになった.電流の向き問題の再来.},ハイペロン$\Lambda^0,\Sigma^+,\Sigma^0,\Sigma^-$にはストレンジネス$-1$が割り当てられる.陽子,中性子,$\pi$中間子(パイオン)にはストレンジネスがゼロである.強い相互作用におけるストレンジネス保存則は,なぜストレンジ粒子が$\pi^+n\to K^+ +\Lambda^0$のような反応で常に対で生成されるかを説明する.一方,$\Lambda^0 \to p+\pi^-$および$K^+\to \pi^+ +\pi^0$ではストレンジ粒子が非ストレンジネス粒子に比較的ゆっくり崩壊する反応は,ストレンジネスを保存しない相互作用が非常に\uwave{弱い}ことを示している.(実際,強い相互作用と電磁相互作用はストレンジネスを保存するが,弱い相互作用は保存しない.保存しない反応の例で挙げたものは弱い相互作用が媒介しているものだ.前者は$\Lambda^0$はクォークレベルで見ると$uds$対であるから,$s$が弱い相互作用により$u$と仮想$W^-$になり,仮想$W^-$がさらに$d\bar{u}$対になることで$(uud)(d\bar{u})$対になり,$p+\pi^0$になる.後者は$K^+=u\bar{s}$から,$\bar{s}$が弱い相互作用により$\bar{u}$と仮想$W^+$の対になり,仮想$W^+$が$u,\bar{d}$対になることで$(u\bar{d})(u\bar{u})$対になり,$p+\pi^0$になる.)

\begin{figure}
  \begin{center}
    \begin{tikzpicture}
      \begin{feynhand}
        \vertex (u_i) at (-3,1) {$u$};
        \vertex (d_i) at (-3,0) {$d$};
        \vertex (s_i) at (-3,-1) {$s$};
        \vertex (i_1) at (-1,-1);
        \vertex (i_2) at (1,-2);
        \vertex (u_f1) at (3,1) {$u$};
        \vertex (d_f1) at (3,0) {$d$};
        \vertex (u_f2) at (3,-1) {$u$};
        \vertex (bar_u_f) at (3,-2) {$\overline{u}$};
        \vertex (d_f2) at (3,-3) {$d$};
        \propag[fermion] (u_i) to (u_f1);
        \propag[fermion] (d_i) to (d_f1);
        \propag[fermion] (s_i) to (i_1);
        \propag[fermion] (i_1) to (u_f2);
        \propag[photon] (i_1) to (i_2);
        \draw(0,-2)node{$W^-$};
        \propag[fermion] (bar_u_f) to (i_2);
        \propag[fermion] (i_2) to (d_f2);
        \draw (-3.5,1.25) to [out=260, in=100] (-3.5,-1.25);
        \draw (3.5,1.25) to [out=280, in=80] (3.5,-1.25);
        \draw (3.5,-1.75) to [out=280, in=80] (3.5,-3.25);
        \draw (-3.75,0)node[left]{$\Lambda$};
        \draw (3.75,0)node[right]{$p$};
        \draw (3.75,-2.5)node[right]{$\pi^-$};
      \end{feynhand}
    \end{tikzpicture}
  \end{center}
  \caption{$\Lambda \to p \pi$のダイアグラム}
\end{figure}

\vskip\baselineskip


生成子が互いに可換でない「非可換」な対称性についても考える.これの典型的な例は,アイソスピンの対称性だ.これは,陽子-中性子間の強い力と同様に,陽子-陽子間の強い力が存在することを示す実験を基礎に1937年に提唱されたものだ.数学的に,その対称性群は$SU(2)$群であり,その生成子は$t_i(i=1,2,3)$で(2.4.18)
\begin{align*}
[t_i,t_j]=i\epsilon_{ijk}t_k
\end{align*}
を満たすものだ.アイソスピン対称性は,粒子が整数または半整数$T$とその$t_3$の値で区別される$2T+1$個の成分を添え字に持つ縮退した多重項を持つことを(対称性が保存する程度に)要求する.これはちょうど,回転不変性により縮退したスピン多重項が要求されるのと同じだ.これらのアイソスピン多重項は,$T=\frac{1}{2}$表現で$t_3=\frac{1}{2},-\frac{1}{2}$の核子$p,n$($t_i$の形は$T$の値に応じて表現(2.5.21)(2.5.22)を使い与える)
\begin{align*}
&t_1=\frac{1}{2}\left(
\begin{matrix}
0 & 1 \\
1 & 0
\end{matrix}
\right), \quad t_2=\frac{1}{2}\left(
\begin{matrix}
0 & -i \\
i & 0
\end{matrix}
\right),\quad \frac{1}{2}\left(
\begin{matrix}
1 & 0 \\
0 & -1
\end{matrix}
\right) \\
&t_3\left(
\begin{matrix}
p \\
n
\end{matrix}
\right)=\left(
\begin{matrix}
+(1/2)p \\
-(1/2)n
\end{matrix}
\right)
\end{align*}
$T=1$表現で$t_3=+1,0,-1$のパイオン
\begin{align*}
&t_1=\frac{1}{\sqrt{2}}\left(
\begin{matrix}
0 & 1 & 0 \\
1 & 0 & 1 \\
0 & 1 & 0 
\end{matrix}
\right), \quad  t_2 =\frac{1}{\sqrt{2}}\left(
\begin{matrix}
0 & -i & 0 \\
i & 0 & -i \\
0 & i & 0 
\end{matrix}
\right) ,\quad t_3=\left(
\begin{matrix}
1 & 0 & 0 \\
0 & 0 & 0 \\
0 & 0 & -1 
\end{matrix}
\right) \\
&t_3\left(
\begin{matrix}
\pi^+ \\
\pi^0 \\
\pi^-
\end{matrix}
\right)=\left(
\begin{matrix}
+\pi^+ \\
0\cdot \pi^0 \\
-\pi^-
\end{matrix}
\right)
\end{align*}
および$T=0,t_3=0$の$\Lambda^0$ハイペロンなどを含む.\par
以上に挙げた三種類の保存則の例は,電荷$Q$,バリオン数$B$,ストレンジネス$S$,アイソスピンの第三成分$t_3$の間の関係
\begin{align*}
Q=t_3+(B+S)/2
\end{align*}
の実例となっている.例えば,$\pi^+$は$Q=+1,t_3=+1,B=0,S=0$であり,$p$は$Q=+1,t_3=+1/2,B=+1,S=0$であり,$\Lambda^0$は$Q=0,t_3=0,B=1,S=-1$を持つ.これらは確かに上の関係式を満たしている.この関係は元々,観測された結果から経験則的に得られたものだが,1960年にゲルマンとンーマンによって,アイソスピン$\mathbf{t}$とハイパーチャージ$Y:=B+S$の両方を,非可換群$SU(3)$に基づくもっと大きいがもっとひどく破れている非可換内部対称性のリー代数$\mathfrak{su}(3)$に埋め込んだ結果として解釈された.19.4節19.7節で扱うように,今日ではアイソスピン$SU(2)$対称性と$SU(3)$対称性はそれぞれ,強い相互作用の現代的な理論である量子色力学QCDにおいて,最も軽い2個または3個のクォークの質量が小さく近似的に無視できることの付随的結果として理解されている.\par
強い相互作用をする粒子間の反応におけるアイソスピン$SU(2)$対称性の帰結は,回転$SU(2)$不変性の意味を導くために考え出されたよく知られた方法と同じ方法で導くことができる.すなわち,クレブシュゴルダン係数を使った既約分解ができる.特に,2体反応$A+B\to C+D$の場合
\begin{align*}
&S_{t_{C3}t_{D3};t_{A3}t_{B3}} \\
=&\Bigl(\Psi_{t_{C3}t_{D3}},S\Psi_{t_{A3}t_{D3}}\Bigr) \\
=&\sum_{T,t_3,T',t_3'}\Bigl(\Psi_{t_{C3}t_{D3}},\Psi_{Tt_3}^{T_{C}T_{D}}\Bigr)\Bigl(\Psi_{Tt_3}^{T_CT_C},S\Psi_{T't_3'}^{T_AT_B}\Bigr)\Bigl(\Psi^{T_AT_B}_{T't_3'},\Psi_{t_{A3}t_{D3}}\Bigr) \\
=&\sum_{T,t_3}C_{T_C T_D}(Tt_3;t_{C3},t_{D3})C_{T_A T_B}(T't'_3;t_{A3},t_{B3})S_{T}(t_3) \\
=&\sum_{T,t_3}C_{T_C T_D}(Tt_3;t_{C3},t_{D3})C_{T_A T_B}(T't'_3;t_{A3},t_{B3})S_{T}
\end{align*}
の形に書けることを要求する.3個目の行では,アイソスピン対称性により$S$演算子は$\mathbf{t}^2,t_3$の両方と可換だから,遷移の前後で$T,t_3$の両方が保存することを用いて$T'=T,t_3'=t_3$であることを使った.最後の行では,再びアイソスピン対称性により$t_3$を上下する$t_\pm=t_1\pm it_2$が$S$演算子と可換であることから,$t_3$の依存性がないことを用いた(ウィグナー・エッカルトの定理).ここで$S_T$は$T$および,ここでは書いていない全ての運動量とスピン変数に依存するが,アイソスピンの第3成分$t_{A3},t_{B3},t_{C3},t_{D3}$には依存しない$S$行列の既約成分だ.またクレブシュゴルダン係数は
\begin{align*}
\Psi_{j\sigma}^{j_1j_2}=\sum_{\sigma_1\sigma_2}C_{j_1j_2}(j\sigma;\sigma_1\sigma_2)\Psi_{j_1\sigma_1;j_2\sigma_2},\quad C_{j_1j_2}(j\sigma;\sigma_1\sigma_2)=\Bigl(\Psi_{j_1\sigma_1;j_2\sigma_2},\Psi_{j\sigma}^{j_1j_2}\Bigr)
\end{align*}
で与えられる.もちろんこれは,全てのアイソスピン対称性と同様に,近似的にしかなりたっていない.なぜならこの対称性は,例えば$p$と$n$のような同じアイソスピン多重項の異なるメンバーが異なる電荷と若干異なる質量をもつという事実に示されているように,電磁相互作用(およびその他の相互作用)では保存しないからだ.

\vskip\baselineskip

\textbf{(C)パリティ}\par
空間変換$\mathbf{x}\to -\mathbf{x}$に対する対称性が実際に正しい範囲で(つまり2.6節の話がなりたつ範囲内で),in状態とout状態の\uwave{両方}が(2.6.16)の1粒子状態の直積のように変換するユニタリー演算子$\mathsf{P}$が存在することを要請する.
\begin{align*}
\mathsf{P} \Psi_{p_1 \sigma_1 n_1;p_2 \sigma_2 n_2;\cdots}^\pm =\eta_{n_1}\eta_{n_2}\cdots \Psi_{\mc{P}p_1 \sigma_1 n_1;\mc{P}p_2 \sigma_2 n_2;\cdots}^\pm
\end{align*}
ここで$\eta_{n}$は種類$n$に依存する粒子の固有パリティであり,また$\mc{P}$は$p^\mu$の空間成分を反転させる.(これは全ての粒子が質量をもっている場合.質量ゼロ粒子の場合は(2.6.22)に変えればよい.)$S$行列がパリティを保存する条件は
\begin{align*}
&S_{p_1'\sigma_1'n_1';p_2'\sigma_2'n_2';\cdots , \, p_1\sigma_1n_1;p_2\sigma_2n_2;\cdots} =\Bigl(\mathsf{P}\Psi_{p_1 \sigma_1 n_1;p_2 \sigma_2 n_2;\cdots}^- , \mathsf{P}\Psi_{p_1 \sigma_1 n_1;p_2 \sigma_2 n_2;\cdots}^+  \Bigr) \\
=&\eta^*_{n'_1}\eta^*_{n'_2}\cdots \eta_{n_1}\eta_{n_2}\cdots S_{\mc{P}p_1'\sigma_1'n_1';\mc{P}p_2'\sigma_2'n_2';\cdots , \, \mc{P}p_1\sigma_1n_1;\mc{P}p_2\sigma_2n_2;\cdots}
\end{align*}
である.内部対称性の場合と同じように,これはパリティ対称性(パリティが保存)の定義である.(3.3.41)を満たす$\mathsf{P}$は,自由粒子状態にこのように作用するような演算子$\mathsf{P}_0$が存在し,それが$H_0$だけでなく$V$とも交換すれば,実際に存在する.\par
位相$\eta_n$は力学的な模型または実験のどちらかから推察できるが,それらによって$\eta$が一意的に決まるわけではない.これは,常に$\mathsf{P}$を,任意の保存する内部対称性の演算子と組み合わせて再定義する自由度があるからである.例えば,もし$\mathsf{P}$が保存するならば
\begin{align*}
\mathsf{P}':=\mathsf{P} \exp(i\alpha B+ i\beta L +i\gamma Q)
\end{align*}
も保存するはずだ.ここで$B,L,Q$はそれぞれバリオン数,レプトン数,電荷であり,$\alpha,\beta,\gamma$は任意の実位相である.実際(ほぼ自明だが)
\begin{align*}
&\Bigl(\Phi_{p_1 \sigma_1 n_1;p_2 \sigma_2 n_2;\cdots} , \mathsf{P}'^{-1} S \mathsf{P}'\Phi_{p_1 \sigma_1 n_1;p_2 \sigma_2 n_2;\cdots}  \Bigr) \\
=&\eta^*_{n'_1}\eta^*_{n'_2}\cdots \eta_{n_1}\eta_{n_2}\cdots \\
&\times \exp\Bigl[i\alpha(b_{n_1}+b_{n_2}+\cdots -b_{n_1'}-b_{n_2'}-\cdots) \\
&\qquad +i\beta(l_{n_1}+l_{n_2}+\cdots -l_{n_1'}-l_{n_2'}-\cdots) \\
&\qquad +i\gamma(q_{n_1}+q_{n_2}+\cdots -q_{n_1'}-q_{n_2'}-\cdots)\Bigr] S_{\mc{P}p_1'\sigma_1'n_1';\mc{P}p_2'\sigma_2'n_2';\cdots , \, \mc{P}p_1\sigma_1n_1;\mc{P}p_2\sigma_2n_2;\cdots} \\
=&\eta^*_{n'_1}\eta^*_{n'_2}\cdots \eta_{n_1}\eta_{n_2}\cdots S_{\mc{P}p_1'\sigma_1'n_1';\mc{P}p_2'\sigma_2'n_2';\cdots , \, \mc{P}p_1\sigma_1n_1;\mc{P}p_2\sigma_2n_2;\cdots} \\
=&S_{p_1'\sigma_1'n_1';p_2'\sigma_2'n_2';\cdots , \, p_1\sigma_1n_1;p_2\sigma_2n_2;\cdots}=\Bigl(\Phi_{p_1 \sigma_1 n_1;p_2 \sigma_2 n_2;\cdots} , S \Phi_{p_1 \sigma_1 n_1;p_2 \sigma_2 n_2;\cdots}  \Bigr) \\
\therefore \quad & \mathsf{P}'^{-1} S \mathsf{P}'=S
\end{align*}
となり,$\mathsf{P}$パリティと$B,L,Q$が過程の前後で保存しているならば$\mathsf{P}'$パリティも保存している\footnote{バリオン数は破れてるだろうと前述しておいてそれらが保存しているとするのはおかしい気もするが,どのみちパリティが保存していると仮定する範囲でしかこの議論はできないのだから,その範囲ではバリオン数もレプトン数も保存しているとみなしてよさそう.}.よって$\mathsf{P}'$と$\mathsf{P}$のどちらをパリティ演算子と呼んでもよい.中性子,陽子,電子は異なる$B,L,Q$数を持っている(中性子は$(1,0,0)$,陽子は$(1,0,1)$,電子は$(0,1,-1)$となっている)ので,実位相$\alpha,\beta,\gamma$をうまく選べばそれら三つの固有パリティを$+1$とできる!(具体的には
\begin{align*}
\mathsf{P}'\Psi_{p,n}=&\eta_{n}\exp(i\alpha)\Psi_{\mc{P}p,n}=e^{i\theta_n+i\alpha}\Psi_{\mc{P}p,n} \\
\mathsf{P}'\Psi_{p,p^+}=&\eta_{p}\exp(i\alpha+i\gamma)\Psi_{\mc{P}p,p^+}=e^{i\theta_p +i\alpha+i\gamma}\Psi_{\mc{P}p,p^+} \\
\mathsf{P}'\Psi_{p,e}=&\eta_{e}\exp(i\beta-i\gamma)\Psi_{\mc{P}p,e}=e^{i\theta_e+i\beta-i\gamma}\Psi_{\mc{P}p,e}
\end{align*}
となるから,$\alpha$の位相を$\alpha=-\theta_n$に合わせて,$\gamma=-\theta_p-\alpha=-\theta_p+\theta_n$,$\beta=-\theta_e+\gamma=-\theta_e-\theta_p+\theta_n$と選べばよい.)これで$p,n,e^-$の場合は全て固有パリティが$+1$となるようにできたが,既に位相を固定してしまったから,この三つ以外の粒子(例えば荷電パイオン$\pi^+$)の固有パリティはもはや任意ではない.例えば$\pi^+$の$B,L,Q$数は$(0,0,1)$だから
\begin{align*}
\mathsf{P}'\Psi_{\pi^+}=\eta_{\pi^+}\exp(i\gamma)\Psi_{\pi^+}=\eta_{\pi^+}e^{i(-\theta_n+\theta_p)}\Psi_{\pi^+}
\end{align*}
となり,もう位相$\gamma$を選んで調節することはできない.さらに,中性パイオン$\pi^0$のような$B,L,Q$数が全てゼロであるような任意の粒子はこのようなパリティ演算子の変更で固有パリティが変化しないから,どのような粒子(例えば今回は陽子・中性子・電子)を固有パリティ$+1$の基準としても,その基準からの相対位相がゼロでなく常に意味が生じる.


\vskip\baselineskip


以上の注意は,一般に固有パリティは常に値を$\pm 1$しかとらねばならないのかという疑問を解く助けになる.空間反転$\mathsf{P}$は群の乗法則$\mathsf{P}^2=1$を満たすように定義される,というのは容易だが,実際に保存すべきパリティ演算子$\mathsf{P}$が一般に$\mathsf{P}^2=1$を満たしているかは自明ではなく,位相分だけ違うかもしれない\footnote{なぜなら,$\mathsf{P}^2$は確かに射線を変えないが,2章でやったようにそれはヒルベルト空間上では恒等演算子と位相だけ異なっていてもいいからだ.後で議論する,必ずしも$\mathsf{P}^2=1$ととれない理論は$U(\mc{P})U(\mc{P})\neq U(\mc{P}^2)=U(1)$を表し,したがってまさに$O(3,1)$のパリティ$\mc{P}$に関して射影表現になっている.そもそも$SO(3,1)$が射影表現しかもたないんだけどね.}.$\mathsf{P}^2=1$か否かに関わらず,(3.3.41)より演算子$\mathsf{P}^2$は
\begin{align*}
\mathsf{P}^2 \Psi_{p_1 \sigma_1 n_1;p_2 \sigma_2 n_2;\cdots}^\pm=\eta_{n_1}^2\eta_{n_2}^2\cdots \Psi_{p_1 \sigma_1 n_1;p_2 \sigma_2 n_2;\cdots}^\pm
\end{align*}
となり,これはまさに$\mc{D}_{n'n}=\delta_{n'n}\eta_{n}^2=\delta_{n'n}e^{2i\theta_n}$としたときの内部対称性の変換則(3.3.29)(3.3.39)とみなせる.したがってもし$\mathsf{P}$が保存しているならば$\mathsf{P}^2$も保存しており,$\mathsf{P}^2$はなんらかの内部対称性に属しているとみなせる.もしこの内部対称性$\mathsf{P}^2$が$\alpha,\beta,\gamma$を任意の値とする位相変換$\exp(i\alpha B+i\beta L+i\gamma Q)$のなす群のような連続対称性群に属するならば,ならば,その逆の平方根もまたその群に属するはずだ.それを例えば$I_P$と書くと,定義より
\begin{align*}
\mathsf{P}^2 I_P^2=1,\quad [\mathsf{P},I_P]=0
\end{align*}
を満たす\footnote{後者の条件は若干非自明だが,$I_P$は可換対称群に属する変換(3.3.39)だから,粒子の種類を入れ替えず,よって$\mathsf{P}$パリティの変換に何の影響も与えないことからくる.}.例えば,$\mathsf{P}^2$は中性子,陽子,電子に対して
\begin{align*}
\mathsf{P}^2\Psi_{n}=&e^{2i\theta_n} \Psi_n=e^{2i\theta_n B}\Psi_n \\
\mathsf{P}^2\Psi_{p}=&e^{2i\theta_p} \Psi_p=e^{2i\theta_n B+2i(\theta_p-\theta_n) Q}\Psi_p \\
\mathsf{P}^2\Psi_{e}=&e^{2i\theta_e} \Psi_e=e^{2i(\theta_e+\theta_p-\theta_n)L+2i(\theta_p -\theta_n)Q}\Psi_n
\end{align*}
となるから,これは内部対称性演算子$\exp(i\alpha B+i\beta L+i\gamma Q)$で位相が$\alpha=2\theta_n,\beta=2(\theta_e+\theta_p-\theta_n),\gamma=2(\theta_p-\theta_n)$である場合とみなせる.
\begin{align*}
\mathsf{P}^2=\exp\Bigl(2i\theta_n B+2i(\theta_e+\theta_p-\theta_n) L+2i(\theta_p-\theta_n) Q\Bigr)
\end{align*}
その内部対称性には,$\mathsf{P}^2$の逆元の平方根
\begin{align*}
I_P=\exp\Bigl(-i\theta_n B-i(\theta_p+\theta_p-\theta_n) L-i(\theta_p-\theta_n) Q\Bigr)
\end{align*}
も存在する.したがって$\mathsf{P}':=\mathsf{P}I_P$とおけば,これは$\mathsf{P}'^2=1$を満たす.この演算子は$\mathsf{P}$と同程度\footnote{同程度に,とはパリティが保存する理論に限れば$B,L$などの真には保存しないであろう量も保存するとみなせて,それらを用いれば$\mathsf{P}'$も保存するということ.$\mathsf{P}$が多少破れていれば$\mathsf{P}'$も同程度に破れているだろう.}に保存するはずであるから,これをパリティ演算子と呼んでいけない理由はない.その場合,$\mathsf{P}'^2=1$よりその固有パリティは$\pm 1$しかとれない.今の例は中性子・陽子・電子のみを考えている場合だから,これ以外の粒子を含む場合はどうすればよいのかがわからないが,一般に$\mathsf{P}^2$は少なくともなんらかの内部対称性に属しているはずだから,(排中律により内部対称性は連続対称性か離散対称性のどちらかには属しているはずだから)それが連続内部対称性に属しているとみなせる限りその内部対称性を使って同様の議論で$\mathsf{P}^2= 1$となるようにできる\footnote{位相を固定してしまうから,このとき矛盾なしに陽子・中性子・電子がパリティ$+1$の基準にできるかは怪しい.}.問題は離散的対称性を含まなければ内部対称性と解釈できない場合だ.\par
全ての固有パリティが値$\pm 1$ととれるように定義することが必ずしも可能ではない唯一の種類の理論は,どの位相変換の連続対称性群にも属さない離散的な内部対称性がある理論だ.例えば,角運動量保存の結果,全ての半整数スピン粒子の総数$F$は偶数だけ変化できる\footnote{この$F$はFermion数の意味であり,フレーバー数とかの意味ではない.}.
\begin{align*}
&\Bigl((-1)^{F}\Phi_{p_1 \sigma_1 n_1;p_2 \sigma_2 n_2;\cdots} , S (-1)^F\Phi_{p_1 \sigma_1 n_1;p_2 \sigma_2 n_2;\cdots}  \Bigr) \\
=&(-1)^{f_{n1}+f_{n2}+\cdots +f_{n'1}+f_{n'2}+\cdots}\Bigl(\Phi_{p_1 \sigma_1 n_1;p_2 \sigma_2 n_2;\cdots} , S \Phi_{p_1 \sigma_1 n_1;p_2 \sigma_2 n_2;\cdots} \Bigr) \\
=&\Bigl(\Phi_{p_1 \sigma_1 n_1;p_2 \sigma_2 n_2;\cdots} , S \Phi_{p_1 \sigma_1 n_1;p_2 \sigma_2 n_2;\cdots}  \Bigr) \quad \because f_{n1}+f_{n2}+\cdots +f_{n'1}+f_{n'2}+\cdots \in 2\mathbb{Z}\\
\therefore \quad &(-1)^{-F}S (-1)^F=S
\end{align*}
よって内部対称性演算子$(-1)^{F}$は角運動量保存に伴って保存する.さらに,既知の半整数スピン粒子は全てバリオン数とレプトン数の和$B+L$は奇数である.したがって,$F$の偶奇と$B+L$の偶奇は一致し,$(-1)^F=(-1)^{B+L}$がなりたつ.もしこれがなりたつならば,$(-1)^F$は$\alpha$を任意の実数とする演算子$\exp(i\alpha(B+L))$からなる連続対称群の一部($\alpha=\pi$)であり,平方根の逆$I_P=\exp\left(-i\frac{\pi}{2}(B+L)\right)$をもつ.この場合,もし$\mathsf{P}^2=(-1)^F$なら,全ての固有パリティが$\pm 1$となるように$\mathsf{P}$を$\mathsf{P}':=\mathsf{P}I_P$によって再定義できる.しかし,もし($j=\frac{1}{2}$と$B+L=0$をもつ,いわゆるマヨラナニュートリノのように)半整数スピンかつ偶数の$B+L$をもつ粒子が発見されれば,$\mathsf{P}^2=(-1)^F$が連続対称性の一部とみなすことができず,したがってパリティ演算子自身が固有値$\pm 1$を持つように再定義することができない. ただし,この場合はもちろん$\mathsf{P}^4=1$となることができ,よって全ての粒子は$\pm 1$かまたは(マヨラナニュートリノのように)$\pm i$のいずれかの固有パリティを持つ\footnote{実際,(5.5.41)より,自分自身が反粒子となるマヨラナ粒子に対しては$\eta\eta^c=\eta^2=-1$がなりたち,よって$\eta=\pm i$が出てくる場合がある.}.


\vskip\baselineskip


(3.3.42)から,終状態の固有パリティの積が始状態の固有パリティの積に等しいか,またはこの積の逆符号に等しいなら,$S$行列はそれぞれ3元運動量について全体として偶,または奇でなければならない.例えば,$\pi^- d$原子($d$は重水素で,陽子1つと中性子1つからなる)の$\ell=0$の1s状態から,$\pi^-$は急激に原子核へ吸収され$\pi^-+d \to n+n$反応(クォークレベルで見ると$(d\bar{u})+(uud)(udd)\to (udd)+(udd)$)で2中性子の対になることが1951年に観測された\footnote{W. Chinowsky and J. Steinberger. $Phys.\, Rev.\, \mathbf{95}$, 1561 (1954).}.\footnote{軌道角運動量の量子数$\ell$は相対論的物理においても,非相対論的波動力学と同様に用いることができるらしい.3.7節でやるらしいが.}パイオンと重水素はそれぞれスピンゼロと1をもつ\footnote{重水素は陽子と中性子で構成されているから,原理的にはスピンゼロとスピン1の両方の状態がありえそうである.しかし核力の理論から,スピンが逆方向の陽子-中性子間のテンソル力がほぼ消えるという奇妙な性質があり,スピンが一致する場合のみのスピン1の重水素しか存在できないらしい.}から,1s状態の始状態は$j=1$のみがありえる.したがって全角運動量保存則より終状態も$j=1$である必要がある.中性子$n$はスピン$1/2$であるから,$n+n$の全スピンは$s=0,1$である.角運動量の合成よりこの2粒子系の軌道角運動量を$\ell$とすると,$|\ell-s|\leq j \leq \ell +s$より
\begin{align*}
(\ell,s)=(1,0),(0,1),(1,1),(2,1)
\end{align*}
が候補となる.しかし同種フェルミオンの2中性子の系はスピンと座標の両方の交換について反対称でなければならない.全スピンが$s=1$ならスピンについて対称的であり,全スピンが$s=0$ならスピンについて反対称的である.一方座標の交換では$\ell=0$では対称的,$\ell=1$では反対称的である.これを見るには,終状態は重心座標が静止している慣性系で
\begin{align*}
\Phi_{JM}^{\ell s}=&\int d^3 \mathbf{p} u(|\mathbf{p}|)\sum_{m,\sigma}C_{\ell s}(JM;m,\sigma)Y_{\ell m}(\hat{\mathbf{p}}) \sum_{\sigma_1,\sigma_2} C_{\frac{1}{2}\frac{1}{2}}(s\sigma;\sigma_1,\sigma_2) \Phi_{\mathbf{p},\sigma_1;-\mathbf{p},\sigma_2}
\end{align*}
と書けるから(2粒子に対してスピン角運動量の合成によって合成スピン角運動量$s=0,1$の状態にし,球面調和関数で全体の角運動量$\ell$で束縛させて,$\ell$と$s$についてさらに合成して角運動量$J$状態を作っている.$u(|\mathbf{p}|)$は動径波束であり,規格化条件$\int p^2|u(p)|^2dp=1$を満たすようにしている(例えばラゲールの多項式).束縛された2粒子が同種粒子ならば重複のためさらに$1/2$が必要.これにより実際に,規格化
\begin{align*}
(\Psi_{J'M'}^{\ell' s'},\Psi_{JM}^{\ell s})=&\int d^3 \mathbf{p} \int d^3\mathbf{q} u(|\mathbf{p}|)u^*(|\mathbf{q}|) \\
&\quad \times \sum_{m,\sigma,m',\sigma'}C_{\ell s}(JM;m,\sigma)C^*_{\ell' s'}(J'M';m,\sigma) Y_{\ell m}(\hat{\mathbf{p}})Y^*_{\ell' m'}(\hat{\mathbf{q}}) \\
&\quad \times \sum_{\sigma_1,\sigma_2,\sigma'_1,\sigma'_2} C_{\frac{1}{2}\frac{1}{2}}(s\sigma;\sigma_1,\sigma_2)C_{\frac{1}{2}\frac{1}{2}}(s'\sigma';\sigma_1',\sigma_2') (\Phi_{\mathbf{p},\sigma'_1;-\mathbf{p},\sigma'_2},\Phi_{\mathbf{q},\sigma_1;-\mathbf{q},\sigma_2}) \\
=&\int d^3 \mathbf{p} \int d^3\mathbf{q} u(|\mathbf{p}|)u^*(|\mathbf{q}|) \\
&\quad \times \sum_{m,\sigma,m',\sigma'}C_{\ell s}(JM;m,\sigma)C^*_{\ell' s'}(J'M';m',\sigma') Y_{\ell m}(\hat{\mathbf{p}})Y^*_{\ell' m'}(\hat{\mathbf{q}}) \\
&\quad \times \sum_{\sigma_1,\sigma_2,\sigma'_1,\sigma'_2} C_{\frac{1}{2}\frac{1}{2}}(s\sigma;\sigma_1,\sigma_2)C^*_{\frac{1}{2}\frac{1}{2}}(s'\sigma';\sigma_1',\sigma_2') \delta^3(\mathbf{p}-\mathbf{q})\delta_{\sigma'_1\sigma_1}\delta_{\sigma'_2\sigma_2} \\
=&\int dp p^2 |u(p)|^2 \\
&\quad \times \int d\Omega\sum_{m,\sigma,m',\sigma'}C_{\ell s}(JM;m,\sigma)C^*_{\ell' s'}(J'M';m',\sigma') Y_{\ell m}(\hat{\mathbf{p}})Y^*_{\ell' m'}(\hat{\mathbf{p}}) \\
&\quad \times \sum_{\sigma_1,\sigma_2,\sigma'_1,\sigma'_2} C_{\frac{1}{2}\frac{1}{2}}(s\sigma;\sigma_1,\sigma_2)C^*_{\frac{1}{2}\frac{1}{2}}(s'\sigma';\sigma_1,\sigma_2) \\
=&\sum_{m,\sigma,m',\sigma'}C_{\ell s}(JM;m,\sigma)C^*_{\ell' s'}(J'M';m',\sigma') \delta_{\ell'\ell}\delta_{m'm} \\
&\quad \times \sum_{\sigma_1,\sigma_2} C_{\frac{1}{2}\frac{1}{2}}(s\sigma;\sigma_1,\sigma_2)C_{\frac{1}{2}\frac{1}{2}}(s'\sigma';\sigma_1,\sigma_2) \\
=&\sum_{m,\sigma,m',\sigma'}C_{\ell s}(JM;m,\sigma)C^*_{\ell' s'}(J'M';m',\sigma') \delta_{\ell'\ell}\delta_{m'm} \delta_{ss'}\delta_{\sigma\sigma'} \\
=&\sum_{m,\sigma}C_{\ell s}(JM;m,\sigma)C^*_{\ell s}(J'M';m,\sigma) \delta_{\ell'\ell} \delta_{s's} \\
=&\delta_{J'J}\delta_{M'M}\delta_{\ell'\ell} \delta_{s's}
\end{align*}
が満たされている\footnote{正しい式変形ではない.最初の式変形で$(\Phi_{\mathbf{p},\sigma'_1;-\mathbf{p},\sigma'_2},\Phi_{\mathbf{q},\sigma_1;-\mathbf{q},\sigma_2})=(\delta^3(\mathbf{p}-\mathbf{q}))^2=\delta^3(\mathbf{p}-\mathbf{q})\delta^3(0)$なので,本当は無限大の因子が余分に出てくる.これはどちらの状態も重心系としたことが原因で,イメージ的には$\braket{x|x}=\delta(x-x)=\delta(0)$のようなもの.ちゃんと確かめるためには重心系ではなく一般の運動量で考えなければならない.面倒なのでここでは無視した.ちなみにPeskinの5.3節Bound Statesの話で同じような状況の計算が出てくる.俺たちは雰囲気で物理をやっている.}.ここで球面調和関数とクレブシュゴルダン係数の規格直交性
\begin{align*}
\int d\Omega Y_{\ell m}(\hat{\mathbf{p}})Y^*_{\ell' m'}(\hat{\mathbf{p}})=\delta_{\ell'\ell}\delta_{m'm},\quad \sum_{m,\sigma}C_{\ell s}(JM;m,\sigma)C^*_{\ell s}(J'M';m,\sigma)=\delta_{J'J}\delta_{M'M}
\end{align*}
を用いた.)この状態と$\Phi_{\mathbf{x}_1,\sigma'_1,\mathbf{x}_2,\sigma'_2}$の内積をとって
\begin{align*}
(\Phi_{\mathbf{x}_1,\sigma'_1,\mathbf{x}_2,\sigma'_2},\Phi_{JM}^{\ell s})=&\int d^3 \mathbf{p} u(|\mathbf{p}|) \sum_{m,\sigma}C_{\ell s}(JM;m,\sigma)Y_{\ell m}(\hat{\mathbf{p}}) \sum_{\sigma_1,\sigma_2} C_{\frac{1}{2}\frac{1}{2}}(s\sigma;\sigma_1,\sigma_2) (\Phi_{\mathbf{x}_1,\sigma'_1;\mathbf{x}_2,\sigma_2'} ,\Phi_{\mathbf{p},\sigma_1;-\mathbf{p},\sigma_2}) \\
=&\int d^3 \mathbf{p} u(|\mathbf{p}|) \sum_{m,\sigma}C_{\ell s}(JM;m,\sigma)Y_{\ell m}(\hat{\mathbf{p}}) \sum_{\sigma_1,\sigma_2} C_{\frac{1}{2}\frac{1}{2}}(s\sigma;\sigma_1,\sigma_2) e^{i\mathbf{p}\cdot (\mathbf{x}_1-\mathbf{x}_2)} \delta_{\sigma'_1 \sigma_1}\delta_{\sigma'_2\sigma_2} \\
=&\int d^3 \mathbf{p} u(|\mathbf{p}|) \sum_{m,\sigma}C_{\ell s}(JM;m,\sigma)Y_{\ell m}(\hat{\mathbf{p}}) e^{i\mathbf{p}\cdot (\mathbf{x}_1-\mathbf{x}_2)} C_{\frac{1}{2}\frac{1}{2}}(s\sigma;\sigma_1',\sigma_2')
\end{align*}
これは$(\mathbf{x}_1,\sigma_1') \leftrightarrow (\mathbf{x}_2,\sigma_2')$の入れ替えで
\begin{align*}
(\Phi_{\mathbf{x}_2,\sigma_2';\mathbf{x}_1,\sigma_1'},\Phi_{JM}^{\ell s})=&\int d^3 \mathbf{p} u(|\mathbf{p}|) \sum_{m,\sigma}C_{\ell s}(JM;m,\sigma)Y_{\ell m}(\hat{\mathbf{p}}) e^{i\mathbf{p}\cdot (\mathbf{x}_2-\mathbf{x}_1)} C_{\frac{1}{2}\frac{1}{2}}(s\sigma;\sigma_2',\sigma_1') \\
=&\int d^3 \mathbf{p} u(|\mathbf{p}|) \sum_{m,\sigma}C_{\ell s}(JM;m,\sigma)Y_{\ell m}(-\hat{\mathbf{p}}) e^{i\mathbf{p}\cdot (\mathbf{x}_1-\mathbf{x}_2)} C_{\frac{1}{2}\frac{1}{2}}(s\sigma;\sigma_1',\sigma_2') \\
=&(-1)^\ell \int d^3 \mathbf{p} u(|\mathbf{p}|) \sum_{m,\sigma}C_{\ell s}(JM;m,\sigma)Y_{\ell m}(\hat{\mathbf{p}}) e^{i\mathbf{p}\cdot (\mathbf{x}_1-\mathbf{x}_2)} (-1)^{s-\frac{1}{2}-\frac{1}{2}}C_{\frac{1}{2}\frac{1}{2}}(s\sigma;\sigma_1',\sigma_2') \\
=&(-1)^{\ell}(-1)^{s-1} (\Phi_{\mathbf{x}_1,\sigma_1';\mathbf{x}_2,\sigma_2'},\Phi_{JM}^{\ell s})
\end{align*}
ここでクレブシュ・ゴルダン係数についての性質
\begin{align*}
C_{j'j''}(jm,m_1m_2)=(-1)^{j-j'-j''}C_{j''j'}(jm,m_2m_1)
\end{align*}
と,球面調和関数に関する性質$Y_\ell^m (-\hat{\mathbf{p}})=(-1)^\ell Y_\ell^m (\hat{\mathbf{p}})$を用いた.これが奇であるためには,$\ell +s$が偶数でなければならない.したがって$(\ell,s)=(1,1)$の場合のみが許される.さて,これにパリティ演算子を作用させることで
\begin{align*}
\mathsf{P}\Phi_{JM}^{\ell s}=&\int d^3 \mathbf{p} \sum_{m,\sigma}C_{\ell s}(JM;m,\sigma)Y_{\ell m}(\hat{\mathbf{p}}) \sum_{\sigma_1,\sigma_2} C_{s_1,s_2}(s\sigma;\sigma_1,\sigma_2) \mathsf{P} \Phi_{\mathbf{p},\sigma_1;-\mathbf{p},\sigma_2} \\
=&\eta_{n}^2 \int d^3 \mathbf{p} \sum_{m,\sigma}C_{\ell s}(JM;m,\sigma)Y_{\ell m}(\hat{\mathbf{p}}) \sum_{\sigma_1,\sigma_2} C_{s_1,s_2}(s\sigma;\sigma_1,\sigma_2) \Phi_{-\mathbf{p},\sigma_1;+\mathbf{p},\sigma_2} \\
=&\eta_{n}^2 \int d^3 \mathbf{p} \sum_{m,\sigma}C_{\ell s}(JM;m,\sigma)Y_{\ell m}(-\hat{\mathbf{p}}) \sum_{\sigma_1,\sigma_2} C_{s_1,s_2}(s\sigma;\sigma_1,\sigma_2) \Phi_{+\mathbf{p},\sigma_1;-\mathbf{p},\sigma_2} \\
=&(-1)^\ell \eta_{n}^2 \int d^3 \mathbf{p} \sum_{m,\sigma}C_{\ell s}(JM;m,\sigma)Y_{\ell m}(\hat{\mathbf{p}}) \sum_{\sigma_1,\sigma_2} C_{s_1,s_2}(s\sigma;\sigma_1,\sigma_2) \Phi_{+\mathbf{p},\sigma_1;-\mathbf{p},\sigma_2} \\
=&(-1)^\ell \eta_{n}^2 \Phi_{JM}^{\ell s}
\end{align*}
が得られ,したがって$\ell=1$ではこの終状態の固有パリティは$-\eta_{n}^2$である.始状態の固有パリティは1s状態であることから同様に$\eta_d \eta_{\pi^-}$であることがわかる.パリティ対称性により,始状態の固有パリティと終状態の固有パリティは一致していなければならない.すなわち
\begin{align*}
\eta_{d}\eta_{\pi^-}=-\eta_{n}^2
\end{align*}
が要請される.重水素は偶数の軌道角運動量(主に$\ell=0$らしい)を持った陽子と中性子の束縛状態であることが知られており,中性子と陽子の固有パリティを同じにとるのだったから,
\begin{align*}
\eta_d=\eta_{p}\eta_n=\eta_{n}^2
\end{align*}
となる.以上より$\eta_{\pi^-}=-1$,すなわち負の荷電パイオンは擬スカラー粒子であると結論できる!アイソスピン不変性からこの性質は$\pi^0,\pi^+$に対しても同様であるはずであり,これらも擬スカラー粒子である.\par
パイオンの負パリティから考察を進めよう.3個のパイオン($3\pi^0,\pi^0\pi^+\pi^-$など)に崩壊するスピンゼロ粒子は固有パリティ$\eta^3_\pi=-1$を持たなければならない.同様に2個のパイオン($2\pi^0,\pi^+\pi^-$など)に崩壊するスピンゼロ粒子は固有パリティ$\eta_\pi^2=+1$を持たなければならない.それらはパリティが保存すると仮定する限り異なる粒子であるはずだ.では現実ではどうだろう.\par
宇宙線において二つの似たような質量の荷電粒子が発見された.一つは$\theta^+$で,$\pi^+\pi^0$に崩壊した.もう一つは$\tau^+$で,$\pi^+\pi^+\pi^-$に崩壊した($\pi^+\pi^0\pi^0$にも崩壊した).$\tau$崩壊の終状態の$\pi$の角運動量を調べたところ,これらの$\pi$は軌道角運動量をもっていなかった.したがって$\pi$のパリティが奇でスピンゼロであるから,$\tau^+$もまたパリティが奇でスピンがゼロでなければならない.一方,終状態が2つのパイオンであるから,もし$\theta^+$が$\tau^+$のようにスピンゼロだったら,パリティは偶となるだろう.したがって$\theta^+$と$\tau^+$は同じ粒子ではありえない.しかし測定が進歩するにつれて$\theta^+$と$\tau^+$の質量と平均寿命のいずれもが区別がないことがわかってきた.何らかの対称性があってこの二つの質量が同じになるということが考えられるが,さまざまな崩壊の仕方があるのに,どうして平均寿命まで同じになるのだろうか.1956年に,李(リー)と楊(ヤン)は$\theta^+,\tau^+$は実は同じ粒子(今日では$K^\pm$で知られている粒子)であり,パリティ対称性は電磁及び強い相互作用では守られるが,粒子の崩壊を引き起こすはるかに弱い相互作用では守られないのだという説を提唱した.(この相互作用の弱さは$K^+$粒子の寿命が長いことで示される.寿命は$1.235\times 10^{-8}$ sであり,特徴的な時間スケールの$\hbar/m_K c^2=1.3\times 10^{-24}$ sと比べてはるかに長い.)リーとヤンはさらに,空間反転の下での不変性は素粒子の全ての弱い相互作用(原子核のベータ崩壊含む)でひどく破れていると示唆し,それを確かめる実験を提唱した.間もなく彼らの正しいことが確かめられた.

\newpage


\subsection{反応率と断面積}
物理的な粒子の系全体が巨視的な体積$V$をもった大きな箱に閉じこまれているとする.例えば,この箱を立方体$V=L^3$ととることができ,反対側の点を同一視する周期的境界条件をとると,空間の波動関数の一価性の要請から
\begin{align*}
(\Psi_{x,y,z},\Psi_\mathbf{p})=&(\Psi_{x+L,y,z},\Psi_\mathbf{p}) \\
\frac{1}{(2\pi)^{3/2}} \exp\Bigl(i(p_x x+p_y y +p_z z)\Bigr)=&\frac{1}{(2\pi)^{3/2}} \exp\Bigl(i(p_x (x+L)+p_y y +p_z z)\Bigr) \\
\exp(ip_x L)=&1 \\
\therefore \quad p_x=&\frac{2\pi}{L}n_x \quad (n_x \in \mathbf{Z})
\end{align*}
同様に$y,z$方向についても行い,運動量が
\begin{align*}
\mathbf{p}=\frac{2\pi}{L}\mathbf{n}=\frac{2\pi}{L}(n_1,n_2,n_3) \quad (n_1, n_2,n_3 \in \mathbf{Z})
\end{align*}
と量子化される.これにより,全ての三次元デルタ関数のフーリエ積分表示の積分領域が$V$に制限され
\begin{align*}
\delta^3_V(\mathbf{p}'-\mathbf{p}):=&\frac{1}{(2\pi)^3}\int_V d^3x e^{i(\mathbf{p}'-\mathbf{p})\cdot x} \\
(\mathbf{p}'\neq \mathbf{p})=&\frac{1}{(2\pi)^3}\int_0^L e^{i(p_x'-p_x) x} dx \int_0^L e^{i(p_y'-p_y)y}dy \int_0^L e^{i(p_z'-p_z)z}dz \\
=&\frac{1}{(2\pi)^3}\frac{1}{i((p'_x-p_x))}\frac{1}{i((p'_y-p_y))}\frac{1}{i((p'_z-p_z))}\Bigl[e^{i(p'_x-p_x)x}\Bigr]_0^L\Bigl[e^{i(p'_y-p_y)y}\Bigr]_0^L\Bigl[e^{i(p'_z-p_z)z}\Bigr]_0^L \\
=&\frac{L^3}{-i(n_x'-n_x)(n_y'-n_y)(n_z'-n_z)}\Bigl[e^{i2\pi(n_x'-n_x)}-1\Bigr]\Bigl[e^{i2\pi(n_y'-n_y)}-1\Bigr]\Bigl[e^{i2\pi(n_z'-n_z)}-1\Bigr] \\
=&0 \\
(\mathbf{p}'=\mathbf{p})=&\frac{1}{(2\pi)^3}\int_V d^3x =\frac{V}{(2\pi)^3}
\end{align*}
したがって
\begin{align*}
\delta^3(\mathbf{p}'-\mathbf{p}):=&\frac{1}{(2\pi)^3}\int_V d^3x e^{i(\mathbf{p}'-\mathbf{p})\cdot x}= \frac{V}{(2\pi)^3}\delta_{\mathbf{p}',\mathbf{p}}
\end{align*}
となる.ここで$\delta_{\mathbf{p}',\mathbf{p}}$はクロネッカーのデルタ記号で,運動量が一致していれば1をとる.\par
規格化条件(3.1.2)は,この変更により$[V/(2\pi)^3]^N$だけ含まれることになる.
\begin{align*}
&(\Psi_{p_1',\sigma'_1,n'_1;p'_2,\sigma'_2,n'_2\cdots },\Psi_{p_1,\sigma_1,n_1;p_2,\sigma_2,n_2\cdots }) \\
=&\delta^3(\mathbf{p}'_1-\mathbf{p}_1)\delta_{\sigma'_1\sigma_1}\delta_{n_1'n_1}\delta^3(\mathbf{p}'_2-\mathbf{p}_2)\delta_{\sigma'_2\sigma_2}\delta_{n'_2n_2}\cdots \pm[置換] \\
=&\left[\frac{V}{(2\pi)^3}\right]^N \left(\delta^3_{\mathbf{p}'_1\mathbf{p}_1}\delta_{\sigma'_1\sigma_1}\delta_{n_1'n_1}\delta^3_{\mathbf{p}'_2\mathbf{p}_2}\delta_{\sigma'_2\sigma_2}\delta_{n'_2n_2}\cdots \pm[置換]\right)
\end{align*}
ここで$N$は状態に含まれる粒子の数.遷移確立を計算するためには,内積が1の規格化された状態を使わなければならないから,今考えている箱について近似的に規格化された状態
\begin{align*}
\Psi^{\mathrm{Box}}_{\alpha}:=\left[\frac{(2\pi)^3}{V}\right]^{N_\alpha/2}\Psi_\alpha
\end{align*}
を導入し,その内積を
\begin{align*}
\Bigl(\Psi^{\mathrm{Box}}_{\beta},\Psi^{\mathrm{Box}}_{\alpha}\Bigr)=\delta^3_{\mathbf{p}'_1\mathbf{p}_1}\delta_{\sigma'_1\sigma_1}\delta_{n_1'n_1}\delta^3_{\mathbf{p}'_2\mathbf{p}_2}\delta_{\sigma'_2\sigma_2}\delta_{n'_2n_2}\cdots \pm[置換]=:\delta_{\beta\alpha}
\end{align*}
とする.これに対応して$S$行列は
\begin{align*}
S_{\beta\alpha}=&(\Phi_\beta , S \Phi_\alpha) \\
&=\left[\frac{V}{(2\pi)^3}\right]^{(N_\beta+N_\alpha)/2}\Bigl(\Phi_\beta^{\mathrm{Box}} , S \Phi_\alpha^{\mathrm{Box}}\Bigr)=:\left[\frac{V}{(2\pi)^3}\right]^{(N_\beta+N_\alpha)/2}S^{\mathrm{Box}}_{\beta\alpha}
\end{align*}
と書かれる.$S^{\mathbf{Box}}$は状態(3,4,3)を用いて記述される.\par

\vskip\baselineskip

粒子を箱の中に永久に放置すれば,あらゆる可能な遷移が何度も起こるはず.意味のある遷移確率を計算するためには,系を「時間の箱」にも入れなければならない.すなわち,相互作用は有限時間$T$の間だけ働いていると考える.これにより,エネルギー保存のデルタ関数が
\begin{align*}
\delta_T(E_\alpha -E_\beta):=\frac{1}{2\pi}\int_{-T/2}^{T/2}\exp(i(E_\alpha-E_\beta)t)dt
\end{align*}
で置き換わる.相互作用がはたらく以前に状態$\alpha$にある多粒子状態が,相互作用が切れた後にちょうど状態$\beta$になっている確率$P(\alpha\to \beta)$は,振幅の絶対値の二乗で与えられるから
\begin{align*}
P(\alpha\to \beta)=\left|S_{\beta\alpha}^{\mathrm{Box}}\right|^2=\left[\frac{(2\pi)^3}{V}\right]^{(N_\alpha+N_\beta)}\left|S_{\beta\alpha}\right|^2
\end{align*}
となる.これは箱の中のある特定の状態$\beta=(\mathbf{p}'_1,\sigma_1',n_1';\mathbf{p}'_2,\sigma_2',n_2';\cdots )$についての確率であるから,運動量についての幅をとらなければ観測装置で測定することはできない.したがって$i$でラベル付けされた各粒子の運動量$\mathbf{p}_i$について$\mathbf{p}_i \sim \mathbf{p}_i +d\mathbf{p}_i$だけの範囲をとる.このとき箱の中の1粒子が運動量空間の体積$d^3\mathbf{p}$内にもつ状態数は(3.4.1)より
\begin{align*}
d^3\mathbf{n}=\left(\frac{L}{2\pi}\right)^3 d^3\mathbf{p}=\frac{V}{(2\pi)^3}d^3\mathbf{p}
\end{align*}
である.終状態の幅$d\beta$を終状態の各粒子の$d\beta=d^3\mathbf{p}_1' d^3\mathbf{p}_2'\cdots$の積(ラベル$i$で区別できない同種粒子に関しては重複しないように$1/\prod N_i!$もかける)と定義すると,この範囲内の全状態数は
\begin{align*}
d\mc{N}_\beta=&\left(\prod_i\frac{1}{N_i!}\right)d^3\mathbf{n}_1'd^3\mathbf{n}_2\cdots \\
=&\left[\frac{V}{(2\pi)^3}\right]^{N_\beta} \left(\prod_i\frac{1}{N_i!}\right)d^3\mathbf{p}_1'd^3\mathbf{p}_2\cdots \\
=&\left[\frac{V}{(2\pi)^3}\right]^{N_\beta}d\beta
\end{align*}
となる.よって系が微小範囲$d\beta$の範囲の終状態に達する全確率は,その範囲内の全状態数について和をとることで
\begin{align*}
dP(\alpha\to \beta)=P(\alpha\to \beta)d\mc{N}_\beta=\left[\frac{(2\pi)^3}{V}\right]^{N_\alpha} \left|S_{\beta\alpha}\right|^2 d\beta
\end{align*}
となる\footnote{観測で区別できないものについては和をとる,という大前提を思い出す.微小な幅$d\mathbf{p}_i$の中の運動量同士は区別できない.}.この節を通して,終状態$\beta$は始状態$\alpha$と(少しでも)異なるだけでなく,状態$\beta$の粒子のどの真部分集合(状態全体以外の部分集合)をとっても,状態$\alpha$の粒子の対応する部分集合と正確に同じ4元運動量を持つ\footnote{例えば,始状態$\alpha$が3粒子状態でそれぞれ$p^\mu_1,p^\mu_2,p^\mu_3$を持つとする.終状態も3粒子状態として運動量$p'^\mu_1,p'^\mu_2,p'^\mu_3$をもつとして,どれか一つでも始状態の運動量と違えば前者の条件$\beta\neq \alpha$は満たされるが,$p'^\mu_1\neq p^\mu_1,p'^\mu_2\neq p^\mu_2,p'^\mu_3=p^\mu_3$となっている場合は$\beta\neq \alpha$だが後者の条件を満たさない.}ことがないという,より厳しい条件を満たすものに限る.(次の章で導入する言葉を使えば,これは$S$行列の連結部分のみを考えることを意味する.)そのような状態については,デルタ関数を含まない行列要素$M_{\beta\alpha}$を以下のように定義できる.
\begin{align*}
S_{\beta\alpha}=:-2\pi i\delta^3_V(\mathbf{p}_\beta-\mathbf{p}_\alpha)\delta_T(E_\beta-E_\alpha)M_{\beta\alpha}
\end{align*}
箱を導入したことで,$\beta\neq \alpha$の場合の$|S_{\beta\alpha}|^2$の中のデルタ関数の二乗は有限で解釈できて
\begin{align*}
\left[\delta^3_V(\mathbf{p}_\beta-\mathbf{p}_\alpha)\right]^2=&\delta^3_V(\mathbf{p}_\beta-\mathbf{p}_\alpha)\delta^3_V(0)=\delta^3_V(\mathbf{p}_\beta-\mathbf{p}_\alpha)\frac{V}{(2\pi)^3} \\
\left[\delta_T(E_\beta-E_\alpha)\right]^2=&\delta_T(E_\beta-E_\alpha)\delta_T(0)=\delta_T(E_\beta-E_\alpha) \frac{T}{2\pi}
\end{align*}
と可能になる\footnote{$\delta_T$に関してはクロネッカーのデルタを含まないから,厳密に二つ目のデルタ関数の引数がゼロとはみなせない.十分$T$が大きくデルタ関数と近似できると仮定してこの計算は行っている.}.よって(3.4.9)は微分遷移確率
\begin{align*}
dP(\alpha\to \beta)=&\left[\frac{(2\pi)^3}{V}\right]^{N_\alpha} \left|S_{\beta\alpha}\right|^2 d\beta \\
=&\left[\frac{(2\pi)^3}{V}\right]^{N_\alpha} (2\pi)^2 \left[\delta^3_V(\mathbf{p}_\beta-\mathbf{p}_\alpha)\right]^2 \left[\delta_T(E_\beta-E_\alpha)\right]^2 \left|M_{\beta\alpha}\right|^2 d\beta \\
=&(2\pi)^2 \left[\frac{(2\pi)^3}{V}\right]^{N_\alpha-1}\frac{T}{2\pi} \left|M_{\beta\alpha}\right|^2 \delta^3_V(\mathbf{p}_\beta-\mathbf{p}_\alpha) \delta_T(E_\beta-E_\alpha) d\beta
\end{align*}
を与える.$V,T$を非常に大きくする極限で,このデルタ関数の積$\delta^3_V(\mathbf{p}_\beta-\mathbf{p}_\alpha) \delta_T(E_\beta-E_\alpha)$は通常の4次元デルタ関数$\delta^4(p_\beta-p_\alpha)$と解釈できる.この極限で,遷移確率$dP(\alpha\to \beta)$は単に相互作用がはたらいている時間$T$に比例し,その係数が微分遷移率
\begin{align*}
d\Gamma(\alpha\to \beta):=&dP(\alpha\to \beta)/T \\
=&(2\pi)^{3N_\alpha-2} V^{1-N_\alpha} \left|M_{\beta\alpha}\right|^2 \delta^4(p_\beta-p_\alpha) d\beta
\end{align*}
と解釈できる(なぜそう解釈できるかは,このあとすぐ見る不安定粒子の崩壊率に関する議論で理解できると思う).ここで$M_{\beta\alpha}$は,この極限のもとで
\begin{align*}
S_{\beta\alpha}=:-2\pi i \delta^4(p_\beta-p_\alpha)M_{\beta\alpha}
\end{align*}
で計算できる.これが,$S$行列の計算を解釈して,実際の実験に対する予言をするときに用いる基本公式となる!\par
特に重要なのは以下の二つの場合となる.

\vskip\baselineskip

$N_\alpha=1$:\par
この場合,(3.4.11)で体積$V$は相殺し,1粒子状態$\alpha$が一般の多粒子状態$\beta$に\uwave{崩壊}する遷移率
\begin{align*}
d\Gamma(\alpha\to \beta)=2\pi|M_{\beta\alpha}|^2\delta^4(p_\beta-p_\alpha)d\beta
\end{align*}
が得られる.もちろん,これは相互作用が働いている時間$T$が単一粒子$\alpha$の寿命$\tau_\alpha$よりずっと短い場合のみ意味がある.なぜなら,不安定粒子の崩壊確率$P(\alpha\to \beta)$は厳密には全崩壊率$\Gamma$と
\begin{align*}
P(T)=1-e^{-\Gamma T}
\end{align*}
で関係しているが,$T\ll \tau_\alpha =1/\Gamma$の近似で
\begin{align*}
P(T)\approx& \Gamma T \\
\therefore \quad P(T)/T&\approx \Gamma
\end{align*}
と書けるからだ.この近似がなりたつ$T$でなければ,$d\Gamma(\alpha\to \beta)=dP(\alpha\to \beta)/T$が遷移率として解釈することはできない.これにより,$\tau_\alpha$が有限の大きさである限り,$\delta_T(E_\alpha-E_\beta)$において$T\to \infty$の極限に行くことはできない.ではどのような場合に$\delta_T(E_\alpha-E_\beta)$は十分デルタ関数とみなせるだろうか.
\begin{align*}
\delta_T(E_\alpha-E_\beta)=&\frac{1}{2\pi}\int_{-T/2}^{T/2}\exp(i(E_\alpha-E_\beta)t)dt \\
=&\frac{1}{2\pi}\left[\frac{e^{i(E_\alpha-E_\beta)t}}{i(E_\alpha-E_\beta)}\right]_{-T/2}^{T/2} \\
=&\frac{1}{\pi}\frac{\sin\left(\frac{(E_\alpha-E_\beta)T}{2}\right)}{E_\alpha-E_\beta}=\frac{T}{2\pi} \mathrm{sinc}((\Delta E)T) \qquad (\Delta E=E_\alpha-E_\beta)
\end{align*}
であるから,このデルタ関数にはピーク$E_\alpha=E_\beta$から除去できない幅$\Delta E\simeq 1/T\gtrsim 1/\tau_\alpha$が存在する($\mathrm{sinc}\, x=\sin x/x$のグラフを書けばすぐわかる).よってこの幅が十分小さく$\delta_T(E_\alpha-E_\beta)$がデルタ関数とみなせて(3.4.13)が有効であるためには,幅の下界$1/\tau_\alpha=\Gamma$(全崩壊率)が過程の特徴的なエネルギースケールのどれよりもはるかに小さい場合でなければならない.

\vskip\baselineskip

$N_\alpha=2$:\par
この場合,遷移率(3.4.11)は
\begin{align*}
d\Gamma(\alpha\to \beta)=(2\pi)^4 \frac{1}{V}|M_{\beta\alpha}|^2\delta^4(p_\beta-p_\alpha)d\beta
\end{align*}
となり$1/V$,すなわち一方の粒子に対するもう一方の粒子の密度に比例する.実験屋が報告するのは密度あたりの遷移率ではなく,粒子束あたりの反応率,すなわち断面積である.どちらかの粒子の,相手の粒子の位置での粒子束$\Phi_\alpha$は,密度$1/V$と相対速度$u_\alpha$の積
\begin{align*}
\Phi_\alpha:=u_\alpha/V
\end{align*}
で定義される.($u_\alpha$の定義は後で.)すると微分断面積は
\begin{align*}
d\sigma(\alpha\to \beta):=&d\Gamma(\alpha\to \beta)/\Phi_\alpha \\
=&(2\pi)^4 u_\alpha^{-1}|M_{\beta\alpha}|^2\delta^4(p_\beta-p_\alpha)d\beta
\end{align*}
となる.

\vskip\baselineskip

次に,遷移率と断面積のローレンツ変換性の問題を考える.これは(3.4.15)の相対速度$u_\alpha$のより一般的な定義を与えるのに役立つらしい.\par
$S$行列のローレンツ変換性(3.3.1)は,各粒子のスピンに付随した運動量に依存する行列のために複雑な構造になっている.この複雑さを避けるために,(ローレンツ不変なデルタ関数$\delta^4(p_\beta-p_\alpha)$を(3.4.12)から取り除いた後に\footnote{一般に$\delta^{(n)}(Ax)=\frac{1}{|\mathrm{det}A|}\delta^{(n)}(x)$がなりたつから,ローレンツ変換の性質$|\mathrm{det}\Lambda|=1$を用いて不変性がわかる.})(3.3.1)の絶対値の二乗を考え,あらゆるスピンについて和をとる.(3.3.1)と(3.4.12)より
\begin{align*}
&M_{p'_1,\sigma'_1,n_1';p'_2,\sigma'_2,n_2';\cdots ,p_1,\sigma_1,n_1; p_2,\sigma_2,n_2;\cdots } \\
=&\sqrt{\frac{(\Lambda p_1)^0 (\Lambda p_2)^0\cdots (\Lambda p_1')^0(\Lambda p_2')^0}{p_1^0 p_2^0\cdots p'^0_1 p'^0_2\cdots }} \\
&\sum_{\bar{\sigma}_1 ,\bar{\sigma}_2,\cdots }D^{(j_1)}_{\bar{\sigma}_1\sigma_1}\Bigl( W(\Lambda,p_1) \Bigr)D^{(j_2)}_{\bar{\sigma}_2\sigma_2}\Bigl( W(\Lambda,p_2) \Bigr)\cdots \\
&\qquad \times \sum_{\bar{\sigma}_1' ,\bar{\sigma}_2',\cdots }D^{(j'_1)}_{\bar{\sigma}'_1\sigma'_1}\Bigl( W(\Lambda,p'_1) \Bigr)D^{(j'_2)}_{\bar{\sigma}'_2\sigma'_2}\Bigl( W(\Lambda,p'_2) \Bigr)\cdots \\
&\qquad \times M_{\Lambda p'_1,\bar{\sigma}'_1,n_1';\Lambda p'_2,\bar{\sigma}'_2,n_2';\cdots ,\Lambda p_1,\bar{\sigma}_1,n_1;\Lambda p_2,\bar{\sigma}_2,n_2;\cdots }
\end{align*}
であるから,行列$D^{(j)}_{\bar{\sigma}\sigma}(W)$(あるいは質量ゼロの場合はそれに対応する行列)のユニタリー性により
\begin{align*}
&\sum_{\sigma_1,\sigma_2,\cdots \sigma'_1,\sigma_2',\cdots }\left|M_{p'_1,\sigma'_1,n_1';p'_2,\sigma'_2,n_2';\cdots ,p_1,\sigma_1,n_1; p_2,\sigma_2,n_2;\cdots }\right|^2 \\
=&\frac{(\Lambda p_1)^0 (\Lambda p_2)^0\cdots (\Lambda p_1')^0(\Lambda p_2')^0}{p_1^0 p_2^0\cdots p'^0_1 p'^0_2\cdots } \\
&\times \sum_{\sigma_1,\sigma_2,\cdots \sigma'_1,\sigma_2',\cdots }\Biggl|\sum_{\bar{\sigma}_1 ,\bar{\sigma}_2,\cdots }D^{(j_1)}_{\bar{\sigma}_1\sigma_1}\Bigl( W(\Lambda,p_1) \Bigr)D^{(j_2)}_{\bar{\sigma}_2\sigma_2}\Bigl( W(\Lambda,p_2) \Bigr)\cdots \\
&\qquad \qquad \qquad \times \sum_{\bar{\sigma}_1' ,\bar{\sigma}_2',\cdots }D^{(j'_1)}_{\bar{\sigma}'_1\sigma'_1}\Bigl( W(\Lambda,p'_1) \Bigr)D^{(j'_2)}_{\bar{\sigma}'_2\sigma'_2}\Bigl( W(\Lambda,p'_2) \Bigr)\cdots \\
&\qquad \qquad \qquad \times M_{\Lambda p'_1,\bar{\sigma}'_1,n_1';\Lambda p'_2,\bar{\sigma}'_2,n_2';\cdots ,\Lambda p_1,\bar{\sigma}_1,n_1;\Lambda p_2,\bar{\sigma}_2,n_2;\cdots } \Biggr|^2 \\
=&\frac{(\Lambda p_1)^0 (\Lambda p_2)^0\cdots (\Lambda p_1')^0(\Lambda p_2')^0}{p_1^0 p_2^0\cdots p'^0_1 p'^0_2\cdots }\sum_{\bar{\sigma}_1,\bar{\sigma}_2,\cdots \bar{\sigma}'_1,\bar{\sigma}_2',\cdots } \left|M_{\Lambda p'_1,\bar{\sigma}'_1,n_1';\Lambda p'_2,\bar{\sigma}'_2,n_2';\cdots ,\Lambda p_1,\bar{\sigma}_1,n_1;\Lambda p_2,\bar{\sigma}_2,n_2;\cdots }\right|^2 \\
\therefore \quad & \sum_{\sigma_1,\sigma_2,\cdots \sigma'_1,\sigma_2',\cdots }\left|M_{p'_1,\sigma'_1,n_1';p'_2,\sigma'_2,n_2';\cdots ,p_1,\sigma_1,n_1; p_2,\sigma_2,n_2;\cdots }\right|^2 p_1^0 p_2^0\cdots p'^0_1 p'^0_2\cdots \\
&= \sum_{\sigma_1,\sigma_2,\cdots \sigma'_1,\sigma_2',\cdots } \left|M_{\Lambda p'_1,\sigma'_1,n_1';\Lambda p'_2,\sigma'_2,n_2';\cdots ,\Lambda p_1,\sigma_1,n_1;\Lambda p_2,\sigma_2,n_2;\cdots }\right|^2 (\Lambda p_1)^0 (\Lambda p_2)^0\cdots (\Lambda p_1')^0(\Lambda p_2')^0
\end{align*}
となり,(3.3.1)のエネルギー因子を除いて,和はローレンツ不変である.すなわち
\begin{align*}
\sum_{\mathrm{spin}}|M_{\beta\alpha}|^2 \prod_\beta E \prod_\alpha E=:R_{\beta\alpha}
\end{align*}
という量は状態$\alpha$と$\beta$の粒子の4元運動量のスカラー関数である.\par
スピンの和をとった1粒子の崩壊率(3.4.13)は
\begin{align*}
\sum_{\mathrm{spin}}d\Gamma(\alpha\to \beta)=2\pi E_{\alpha}^{-1} R_{\beta\alpha} \delta^4(p_\beta-p_\alpha)\frac{d\beta}{\prod_\beta E}
\end{align*}
と書ける.因子$d\beta/\prod_\beta E$は,あらわに書けば
\begin{align*}
\frac{d\beta}{\prod_\beta E}=&\left(\prod_i \frac{1}{N_i!}\right) \frac{d^3 \mathbf{p}'_1}{p'^0_1}\frac{d^3 \mathbf{p}'_2}{p'^0_2}\cdots \\
=&\left(\prod_i \frac{1}{N_i!}\right) \frac{d^3 \mathbf{p}'_1}{\sqrt{\mathbf{p}'^2_1+m_1^2}}\frac{d^3 \mathbf{p}'_2}{\sqrt{\mathbf{p}'^2_2+m_2^2}}\cdots
\end{align*}
であり,これは運動量空間のローレンツ不変な体積要素(2.5.15)の積になっているから,ローレンツ不変である.$R_{\beta\alpha}$と$\delta^4(p_\beta-p_\alpha)$もローレンツ不変であり,始状態の1粒子のエネルギー$E_\alpha$に関する因子$1/E_\alpha$だけがローレンツ不変でない!以上より,崩壊率が$1/E_\alpha$と同じローレンツ変換性をもつことがわかる.これはもちろん,まさに通常の特殊相対論の時間の遅れであり,粒子が速く運動すればするほど,ゆっくり崩壊することに対応する.\par
同様に,スピン和をとった断面積(3.4.15)は,始状態$\alpha$の2粒子のエネルギーを$E_1,E_2$として
\begin{align*}
\sum_{\mathrm{spin}}d\sigma(\alpha\to \beta)=(2\pi)^4 u_\alpha^{-1} E_1^{-1}E_2^{-1} R_{\beta\alpha} \delta^4(p_\alpha-p_\beta)\frac{d\beta}{\prod_\beta E}
\end{align*}
と書ける.断面積は(スピン和をとれば)4元運動量のローレンツ不変な関数となるように定義するのが慣例らしい.因子$R_{\beta\alpha},\delta^4(p_\beta-p_\alpha)$と$d\beta/\prod_\beta E$はすでにローレンツ不変だとわかったから,これより任意の慣性系における相対速度$u_\alpha$を,$u_\alpha E_1 E_2$がローレンツスカラーとなるように定義しなければならない.また,相対速度の定義より,一方の粒子(例えば粒子1)が静止しているローレンツ系では$u_\alpha$は他方の粒子の速度そのものとならなければならない.このことから,$u_\alpha$の値は一般のローレンツ系において
\begin{align*}
u_\alpha=\frac{\sqrt{(p_1\cdot p_2)^2-m_1^2m_2^2}}{E_1E_2}
\end{align*}
と一意的に\footnote{$u_\alpha E_1E_2=\sqrt{(p_1\cdot p_2)^2-m_1^2m_2^2}$がローレンツスカラーであることは明らか.また粒子1が静止しているとき,$\mathbf{p}_1=0,E_1=m_1$であるから,$(p_1\cdot p_2)=-m_1 E_2$となり,代入すると
\begin{align*}
u_\alpha=\frac{\sqrt{m_1^2 E_2^2-m_1^2m_2^2}}{m_1 E_2}=\frac{\sqrt{E_2^2-m_2^2}}{E_2}=\frac{|\mathbf{p}_2|}{E_2}=|\mathbf{v}_2|
\end{align*}
となり,これはちょうど粒子2の速度である.(相対論的に$\mathbf{p}=\gamma m \mathbf{v},E=\gamma m$であることを思い出す.)\par
一意性を示そう.スカラー$S:=u_\alpha E_1E_2$とおく.もし別の定義の相対速度$\tilde{u}_\alpha$が存在すると,$\tilde{S}:=\tilde{u}_\alpha E_1 E_2$もローレンツスカラーであるはずだが,そのようなスカラーが\uwave{全ての$p_1,p_2$について}ある系で一致するなら,全ての系で$\tilde{S}=S$である.一方の粒子が静止する系では一致することが相対速度の定義より要請されるから,よってこの定義での相対速度は一意的に定まる.}定まる.ここで$p_1,p_2$と$m_1,m_2$は始状態$\alpha$の2粒子の4元運動量と質量である.

\vskip\baselineskip

全3元運動量がゼロになる重心系では
\begin{align*}
p_1=(\mathbf{p},E_1),\quad p_2=(-\mathbf{p}_2,E_2)
\end{align*}
であり(実際に$p_1+p_2=(0,E_1+E_2)$で全3元運動量がゼロになっている),(3.4.17)を計算すると相対速度に期待される通り
\begin{align*}
p_1\cdot p_2=&-|\mathbf{p}|^2-E_1E_2 \\
u_\alpha=&\frac{\sqrt{(E_1E_2+|\mathbf{p}|^2)^2-m_1^2m_2^2}}{E_1E_2} \\
=&\frac{\sqrt{|\mathbf{p}|^4 +2E_1E_2|\mathbf{p}|^2+E_1^2E_2^2-m_1^2m_2^2}}{E_1E_2} \\
=&\frac{\sqrt{|\mathbf{p}|^4 +2E_1E_2|\mathbf{p}|^2+E_1^2E_2^2-(E_1^2-|\mathbf{p}|^2)(E_2^2-|\mathbf{p}|^2)}}{E_1E_2} \\
=&\frac{\sqrt{2E_1E_2|\mathbf{p}|^2+E_1^2|\mathbf{p}|^2+E_2^2|\mathbf{p}|^2}}{E_1E_2} \\
=&\frac{|\mathbf{p}|(E_1+E_2)}{E_1E_2}=\left|\frac{\mathbf{p}}{E_1}+\frac{\mathbf{p}}{E_2}\right| \\
=&\left|\frac{\mathbf{p}_1}{E_1}-\frac{\mathbf{p}_2}{E_2}\right| \quad \because \mathbf{p}_1=\mathbf{p},\mathbf{p}_2=-\mathbf{p} \\
=&|\mathbf{v}_1-\mathbf{v}_2|
\end{align*}
が得られる.しかし,この系では$u_\alpha$は実際には物理的な速度ではなく,特に(3.4.18)は極度に相対論的粒子の場合($|\mathbf{v}_1|=|\mathbf{v}_2|\approx 1=c$)には$2=2c$に近い値をとることを示している.これは,重心系での観測者にとって,両方から速度$c$で向かってくる粒子はあたかも相対速度$2c$であるとみなせることから来ている.光速を超えているから当然物理的な速度ではない.

\vskip\baselineskip

次に遷移率の一般式(3.4.11)や崩壊率(3.4.13)と断面積(3.4.15)に現れるいわゆる位相空間の因子$\delta^4(p_\beta-p_\alpha)d\beta$の議論に移る.ここでは始状態の全3元運動量が
\begin{align*}
\mathbf{p}_\alpha=\mathbf{p}_1+\mathbf{p}_2+\cdots =0
\end{align*}
となる重心系の場合に話を限る.($N_\alpha=1$の崩壊現象については,これはちょうど崩壊粒子が静止している場合である.)終状態が運動量$\mathbf{p}_1',\mathbf{p}_2',\cdots$の粒子からなるとすると,$E:=E_\alpha=E_1+E_2+\cdots$を始状態の全エネルギーとして
\begin{align*}
\delta^4(p_\beta-p_\alpha)d\beta=\delta^3(\mathbf{p}_1'+\mathbf{p}_2'+\cdots)\delta(E_1'+E_2'+\cdots -E) d^3\mathbf{p}_1'd^3\mathbf{p}_2'\cdots
\end{align*}
となる.(面倒なので重複の因子は無視した.)$\mathbf{p}_k'$の任意の一つ,例えば$\mathbf{p}_1'$についての積分は単に,運動量デルタ関数を落とし,測度が
\begin{align*}
\delta^4(p_\beta-p_\alpha)d\beta \to \delta(E_1'+E_2'+\cdots -E) d^3\mathbf{p}_2'\cdots
\end{align*}
となり,($E_1'=\sqrt{\mathbf{p}'^2_1+m'^2_1}$のように)$\mathbf{p}_1'$が現れるところには常に
\begin{align*}
\mathbf{p}_1'=-\mathbf{p}_2'-\mathbf{p}_3'-\cdots
\end{align*}
の置き換えをするだけでよい.同様に,残ったデルタ関数$\delta(E_1'+E_2'+\cdots -E)$を用いて任意の\uwave{1個の}積分を取り除ける.それを以下で実行しよう.\par
最も簡単なのは終状態が2粒子の場合である.この場合,(3.4.21)は
\begin{align*}
\delta^4(p_\beta-p_\alpha)d\beta \to \delta(E_1'+E_2' -E) d^3\mathbf{p}_2'
\end{align*}
となる.$\mathbf{p}_1',\mathbf{p}_2'$依存性をあらわに書けば,これは
\begin{align*}
\delta^4(p_\beta-p_\alpha)d\beta \to& \delta\left(\sqrt{|\mathbf{p}_1'|^2+m'^2_1}+\sqrt{|\mathbf{p}_2'|^2+m'^2_2} -E\right) d^3\mathbf{p}_2' \\
=&\delta\left(\sqrt{|\mathbf{p}_1'|^2+m'^2_1}+\sqrt{|\mathbf{p}_1'|^2+m'^2_2} -E\right) |\mathbf{p}_1'|^2d|\mathbf{p}_1'| d\Omega
\end{align*}
である.ここで(3.4.22)
\begin{align*}
\mathbf{p}_2'=-\mathbf{p}_1'
\end{align*}
と,極座標を用いて$d^3\mathbf{p}_1'=|\mathbf{p}_1'|^2d|\mathbf{p}_1'| d\Omega$と書いた\footnote{奇数個の変数がマイナスだけされたので$d^3\mathbf{p}_1'=-d^3\mathbf{p}_2'$となる$\cdots$と勘違いしないように.積分範囲を逆にするところからさらにマイナスが生じるのでマイナスがキャンセルして生じなくなる.よくあるミスなのでこれから何度も注意することになると思う.}.$d\Omega:=\sin\theta d\theta d\phi$は$\mathbf{p}_1'$の立体角微分である.デルタ関数の中身は$|\mathbf{p}_1'|$に関して単調増加関数であるから,$x=x_0$にだけ1位のゼロ点($\Leftrightarrow f'(x_0)\neq 0$)をもつ任意の実関数について成り立つ標準的な式
\begin{align*}
\delta(f(x))=\delta(x-x_0)/|f'(x_0)|
\end{align*}
を用いると簡単になる.(この式について簡単に証明しておく.任意の有界区間内に高々有限個の1位のゼロ点$(f(x_1)=\cdots=f(x_n)=0)$をもつ任意の関数$f(x)$を引数にもつデルタ関数$\delta(f(x))$を考える.各ゼロ点に対して,微小な$\epsilon$で開近傍をとることでゼロ点同士を分離することができて\footnote{厳密には$\mathbb{R}^n$が通常のノルムで距離空間になり,距離空間がハウスドロフ(2点分離可能)であることを使っている.$\sin(1/x)$のような$x=0$の近傍に無限個のゼロ点をもち,ゼロ点同士がどれだけ小さい近傍をとっても分離することができないので,このような関数はこの議論に使えない.$\sin x$のように$\mathbb{R}$上で無限個のゼロ点を持っていても,任意の\uwave{有界}区間内にゼロ点が有限個となっていればよい.}テスト関数$\varphi$を用いて
\begin{align*}
\int_{-\infty}^\infty \varphi(x)\delta(f(x))dx=&\sum_{i=1}^n\int_{x_i-\epsilon}^{x_i+\epsilon}\varphi(x)\delta(f(x))dx \\
=&\sum_{i=1}^n\int_{f^{-1}(x_i-\epsilon)}^{f^{-1}(x_i+\epsilon)}\varphi(f^{-1}(y))\delta(y)\frac{dy}{|f'(f^{-1}(y))|} \quad (y=f(x),dx=\frac{dy}{|f'(f^{-1}(y))|}) \\
=&\sum_{i=1}^n\varphi(f^{-1}(0)) \frac{1}{|f'(f^{-1}(0))|}  \\
=&\sum_{i=1}^n\left[\varphi(x_i) \frac{1}{|f'(x_i)|}\right]_{f(x_i)=0} \\
=&\int_{-\infty}^\infty \varphi(x)\left[\sum_{i=1}^n\frac{1}{|f'(x_i)|} \delta(x-x_i)\right]dx \\
\therefore \quad \delta(f(x))=&\sum_{i=1}^n\frac{1}{|f'(x_i)|} \delta(x-x_i)
\end{align*}
が得られる\footnote{途中で$f^{-1}(y)$をとるために,逆関数定理を使っており,微分係数$f'(x_i)$がゼロでないから$x_i$の近傍$(x_i-\epsilon,x_i+\epsilon)$で$f$が可逆になることを使っている.}.ゼロ点が1個しかなければ,今回用いる式になる.)今の場合,(3.4.23)のデルタ関数の引数$E_1'+E_2'-E$がゼロとなるゼロ点$|\mathbf{p}_1'|$の値$k'$は
\begin{align*}
k'=&\frac{\sqrt{(E^2-m'^2_1-m'^2_2)^2-4m'^2_1m'^2_2}}{2E}
\end{align*}
である.実際にこれは
\begin{align*}
E_1'=&\sqrt{k'^2+m'^2_1}=\sqrt{\frac{(E^2-m'^2_1-m'^2_2)^2-4m'^2_1m'^2_2}{4E^2}+m'^2_1} \\
=&\sqrt{\frac{E^4+m'^4_1+m'^4_2-2E^2m'^2_1-2E^2m'^2_2+2m'^2_1m'^2_2-4m'^2_1m'^2_2+4E^2m'^2_1}{4E^2}} \\
=&\sqrt{\frac{E^4+m'^4_1+m'^4_2+2E^2m'^2_1-2E^2m'^2_2-2m'^2_1m'^2_2}{4E^2}}=\sqrt{\frac{(E^2+m'^2_1-m'^2_2)^2}{4E^2}} \\
=&\frac{E^2+m'^2_1-m'^2_2}{2E} \\
E_2'=&\sqrt{k'^2+m'^2_2}=\sqrt{\frac{(E^2-m'^2_1-m'^2_2)^2-4m'^2_1m'^2_2}{4E^2}+m'^2_2} \\
=&\sqrt{\frac{E^4+m'^4_1+m'^4_2-2E^2m'^2_1+2E^2m'^2_2-2m'^2_1m'^2_2}{4E^2}}=\sqrt{\frac{(E^2-m'^2_1+m'^2_2)^2}{4E^2}} \\
=&\frac{E^2-m'^2_1+m'^2_2}{2E} \\
\therefore \quad &E_1'+E'_2-E=0
\end{align*}
となっているからゼロ点である.またデルタ関数の引数の微分は
\begin{align*}
&\left[\frac{d}{d|\mathbf{p}_1'|}\left(\sqrt{|\mathbf{p}_1'|^2+m'^2_1}+\sqrt{|\mathbf{p}_1'|^2+m'^2_2} -E\right)\right]_{|\mathbf{p}_1'|=k'} \\
=&\left[\frac{|\mathbf{p}_1'|}{\sqrt{|\mathbf{p}_1'|^2+m'^2_1}}+\frac{|\mathbf{p}_1'|}{\sqrt{|\mathbf{p}_1'|^2+m'^2_2}}\right]_{|\mathbf{p}_1'|=k'} \\
=&\frac{k'}{E_1'}+\frac{k'}{E_2'} \\
=&\frac{k'(E_1'+E_2')}{E_1'E_2'}=\frac{k' E}{E_1'E_2'}
\end{align*}
となっている.よって(3.4.23)は
\begin{align*}
\delta^4(p_\beta-p_\alpha)d\beta \to& \delta\left(\sqrt{|\mathbf{p}_1'|^2+m'^2_1}+\sqrt{|\mathbf{p}_2'|^2+m'^2_2} -E\right) d^3\mathbf{p}_2' \\
=&\delta\left(\sqrt{|\mathbf{p}_1'|^2+m'^2_1}+\sqrt{|\mathbf{p}_1'|^2+m'^2_2} -E\right) |\mathbf{p}_1'|^2d|\mathbf{p}_1'| d\Omega \\
=&\frac{E_1'E_2'}{k'E} \delta(|\mathbf{p}_1'|-k') |\mathbf{p}_1'|^2d|\mathbf{p}_1'| d\Omega \\
\to &\frac{k'E_1'E_2'}{E} d\Omega
\end{align*}
このようにして,$k',E_1',E_2'$が全て(3.4.24)~(2.4.26)で与えられると理解することで$d|\mathbf{p}_1'|$積分が実行できて,デルタ関数を落とすことができる.特に,運動量ゼロ,エネルギー$E$の1粒子状態の2粒子への微分崩壊率(3.4.13)は
\begin{align*}
d\Gamma(\alpha\to \beta)=&2\pi |M_{\beta\alpha}|^2 \frac{k'E_1'E_2'}{E} d\Omega \\
\therefore \quad \frac{d\Gamma(\alpha\to \beta)}{d\Omega}=&\frac{2\pi k'E_1'E_2'}{E}|M_{\beta\alpha}|^2 
\end{align*}
と書ける.2体散乱過程$12\to 1'2'$の微分断面積は(3.4.15)より,始重心系なので$k:=|\mathbf{p}_1|=|\mathbf{p}_2|$として
\begin{align*}
d\sigma(\alpha\to \beta)=&(2\pi)^4u_\alpha^{-1}|M_{\beta\alpha}|^2 \frac{k'E_1'E_2'}{E} d\Omega \\
\therefore \quad \frac{d\sigma(\alpha\to \beta)}{d\Omega}=&\frac{(2\pi)^4 k'E_1'E_2'}{Eu_\alpha}|M_{\beta\alpha}|^2 \\
=& \frac{(2\pi)^4 k'E_1'E_2'E_1E_2}{E^2 k}|M_{\beta\alpha}|^2 \quad \because (3.4.18),u_\alpha=\frac{k E}{E_1E_2}
\end{align*}
で与えられる.重心系での散乱に関するこれらの公式は至る所で便利!(定性的には,重心系であることから終状態の2粒子は互いに同じ大きさで反対の向きの運動量に向かなければならず,運動量の大きさも始状態のエネルギー$E$によって$k'$に定まってしまう.残された自由度はどの向きに互いに放出されるかという角度パラメータのみであり,これは力学的(kinematics)なものではなくハミルトニアン$H$(相互作用項$V$)の構造(dynamics)によって角度のどのような関数になるかは変化する.)

\vskip\baselineskip


最後に,Peskinなどでよく使われている微分遷移率との違いについて言及しておく.Peskinの(4.79)式で示されている2体散乱の微分断面積の式は(多少今回の記法に合わせて)
\begin{align*}
d\sigma=\frac{1}{2E_A2E_B|\mathbf{v}_A-\mathbf{v}_B|}\left(\prod_f \frac{d^3\mathbf{p}_f}{(2\pi)^3}\frac{1}{2E_f}\right)|M_{\beta\alpha}|^2(2\pi)^4\delta^4(p_\beta-p_\alpha)
\end{align*}
これは(3.4.15)とは異なる表式になっている.これはPeskinとWeinbergでは1粒子状態の定義が異なっており,規格化も異なっていることに起因する.Peskin(2.35)(3.106)などを見れば1粒子状態の定義に次の違いがあることがわかる.
\begin{align*}
\Psi_{p,\sigma}^{\mathrm{Peskin}}=\sqrt{(2\pi)^32E_p}\Psi_{p,\sigma}^{\mathrm{Weinberg}}
\end{align*}
(規格化の$(2\pi)^3$も補正しなければならないことに注意.)これにより,もしPeskin流の1粒子状態を採用することにすると,$M_{\beta\alpha}$は始・終状態の各粒子について$\sqrt{(2\pi)^32E_p}$で割らないといけない.さらにPeskinでは(4.73)より
\begin{align*}
S_{\beta\alpha}=+i(2\pi)^4 \delta^4(p_\beta-p_\alpha) M_{\beta\alpha}
\end{align*}
を採用しているから,(3.4.11)(3.4.11)での$|M_{\beta\alpha}|^2$では
\begin{align*}
|M^{\mathrm{W}}_{\beta\alpha}|^2=(2\pi)^6|M_{\beta\alpha}^{\mathrm{P}}|\prod_\alpha \frac{1}{(2\pi)^32E} \prod_\beta\frac{1}{(2\pi)^3 2E}
\end{align*}
を使わなければならない(添え字のW,PはWeinberg流,Peskin流の意味).これを用いると(3.4.11)は
\begin{align*}
d\Gamma(\alpha\to \beta)=&(2\pi)^{3N_{\alpha}+4} V^{1-N_\alpha} |M_{\beta\alpha}^{\mathrm{P}}|^2 \delta^4(p_\beta-p_\alpha) \left(\prod_\alpha \frac{1}{(2\pi)^32E} \right) \left(\prod_\beta\frac{1}{(2\pi)^3 2E}\right) d^3\mathbf{p}_1'd^3\mathbf{p}_2\cdots \\
=&(2\pi)^4 V^{1-N_\alpha}|M_{\beta\alpha}^{\mathrm{P}}|^2\delta^4(p_\beta-p_\alpha) \left(\prod_\alpha\frac{1}{2E}\right) \left(\prod_\beta \frac{d^3\mathbf{p}_\beta}{(2\pi)^3 2E_\beta}\right)
\end{align*}
この表式で$N_\alpha=2$とおいて,再び(3.4.14)を用いて微分断面積にすれば(重複の因子は無視して)
\begin{align*}
d\sigma(\alpha\to \beta)=\frac{1}{2E_12E_2u_\alpha}(2\pi)^4 |M_{\beta\alpha}^{\mathrm{P}}|^2\delta^4(p_\beta-p_\alpha) \left(\prod_\beta \frac{d^3\mathbf{p}_\beta}{(2\pi)^3 2E_\beta}\right)
\end{align*}
となって,無事Peskin流の断面積が復活する.(実は相対速度の定義も違うが,面倒だしそこまで本質的じゃない.)もちろん$N_\alpha=1$の崩壊率も同様に同じになることがわかる.\par
どちらの流儀で計算しても最終的に与えられる断面積などの結果は同じである.しかし5章で与えられる場の定義なども全て1粒子状態の定義に合わせてWeinbergとPeskinなどで異なるため,一度どちらかの流儀で計算すると決めたなら最後まで一方で計算しなければならない.


\newpage

\subsection{摂動論}
$S$行列が計算できれば,それを用いて測定量と比較できる断面積や崩壊率などの量を得ることができることが分かった.次に考えるべきは,そもそも$S$行列はどのようにして計算できるのかという点である.$S$行列を計算する上で歴史的に見て最も有用な技法は摂動論,すなわちハミルトニアン$H=H_0+V$の相互作用項$V$のベキでの展開である.(3.2.7)と(3.1.18)から$S$行列は
\begin{align*}
S_{\beta\alpha}=&\delta(\beta-\alpha)-2\pi i \delta(E_\beta-E_\alpha)T_{\beta\alpha}^+ \\
T_{\beta\alpha}^+=&(\Phi_{\beta},V\Psi_\alpha^-)
\end{align*}
と表せる.ここで$\Psi_{\alpha}^+$はリップマン・シュウィンガー方程式(3.1.17)を満たす.
\begin{align*}
\Psi_\alpha^+=\Phi_\alpha+\int d\gamma \frac{T^+_{\beta\alpha}\Phi_\gamma}{E_\alpha-E_\gamma+i\epsilon}
\end{align*}
この方程式に$V$を作用させ$\Phi_\beta$とのスカラー積をとると,
\begin{align*}
(\Phi_\beta,V\Psi_\alpha^+)=&(\Phi_\beta ,V\Phi_\alpha)+\int d\gamma \frac{(\Phi_\beta,V \Phi_\gamma)T^+_{\gamma\alpha}}{E_\alpha-E_\gamma+i\epsilon} \\
\therefore \quad T_{\beta\alpha}^+=& V_{\beta\alpha}+\int d\gamma \frac{V_{\beta\gamma}T^+_{\gamma\alpha}}{E_\alpha-E_\gamma+i\epsilon}
\end{align*}
という積分方程式が得られる.左辺を右辺に用いると
\begin{align*}
T_{\beta\alpha}^+=&V_{\beta\alpha}+\int d\gamma \frac{V_{\beta\gamma}V_{\gamma\alpha}}{E_\alpha-E_\gamma+i\epsilon}+\int d\gamma \frac{V_{\beta\gamma}}{E_\alpha-E_\gamma+i\epsilon }\int d\gamma' \frac{V_{\gamma\gamma'}T^+_{\gamma'\alpha}}{E_\alpha-E_{\gamma'}+i\epsilon} \\
=&V_{\beta\alpha}+\int d\gamma \frac{V_{\beta\gamma}V_{\gamma\alpha}}{E_\alpha-E_\gamma+i\epsilon}+\int d\gamma d\gamma' \frac{V_{\beta\gamma}V_{\gamma\gamma'}T^+_{\gamma'\alpha}}{(E_\alpha-E_\gamma+i\epsilon)(E_\alpha-E_{\gamma'}+i\epsilon)}
\end{align*}
これを繰り返すと
\begin{align*}
T_{\beta\alpha}^+=&V_{\beta\alpha}+\int d\gamma \frac{V_{\beta\gamma}V_{\gamma\alpha}}{E_\alpha-E_\gamma+i\epsilon}+\int d\gamma d\gamma' \frac{V_{\beta\gamma}V_{\gamma\gamma'}V_{\gamma'\alpha}}{(E_\alpha-E_\gamma+i\epsilon)(E_\alpha-E_{\gamma'}+i\epsilon)}+\cdots
\end{align*}
という形になる.この式に基づく$S$行列の計算は\uwave{旧式の}摂動論である.これの代わりに,以下で導出する\uwave{時間に依存する}摂動論を使う.

\vskip\baselineskip

(3.2.5)を用いて$S$演算子を
\begin{align*}
S=&U(+\infty,-\infty) =\lim_{\tau\to \infty} \lim_{\tau_0 \to -\infty} U(\tau,\tau_0) \\
U(\tau,\tau_0):=&e^{iH_0\tau}e^{-iH(\tau-\tau_0)}e^{-iH_0\tau_0}
\end{align*}
と表す.この$U(\tau,\tau_0)$を$\tau$で微分すると
\begin{align*}
\frac{d}{d\tau}U(\tau,\tau_0)=&\left[e^{iH_0\tau}iH_0\right] e^{-iH(\tau-\tau_0)}e^{-iH_0\tau_0}+e^{iH_0\tau}\left[-iH e^{-iH(\tau-\tau_0)}\right]e^{-iH_0\tau_0} \\
=&-i e^{iH_0\tau}Ve^{-iH(\tau-\tau_0)}e^{-iH_0\tau_0} \\
=&-i \left[e^{iH_0\tau}V e^{-iH_0\tau}\right] e^{iH_0\tau}e^{-iH(\tau-\tau_0)}e^{-iH_0\tau_0} \\
=&-iV(\tau)U(\tau,\tau_0) \\
i\frac{d}{d\tau}U(\tau,\tau_0)=&V_I(\tau)U(\tau,\tau_0)
\end{align*}
という微分方程式が得られる.ここで
\begin{align*}
V_I(t):=e^{iH_0t} V e^{-iH_0t}
\end{align*}
である.(この時間依存性はハイゼンベルグ表示で要求される
\begin{align*}
O_H(t)=e^{iHt} O e^{iHt}
\end{align*}
の時間依存性とは異なるので,区別するために相互作用表示で定義されているという.ここでは添え字に$I$をつけて区別しやすくするが,明らかに違いがわかる文脈では省略する.)この微分方程式の形式的な解は,積分方程式
\begin{align*}
U(\tau,\tau_0)=1-i\int^\tau_{\tau_0} dt V_I(t)U(t,\tau_0)
\end{align*}
であり,実際両辺$\tau$微分すれば微分方程式を満たしていることがわかり,さらに初期条件$U(\tau_0,\tau_0)=1$も満たしている.積分方程式の左辺を右辺に代入すれば
\begin{align*}
U(\tau,\tau_0)=&1-i\int^\tau_{\tau_0} dt V(t)\left[1-i\int^t_{\tau_0} dt' V_I(t')U(t',\tau_0)\right] \\
=&1-i\int^\tau_{\tau_0} dt V_I(t)+(-i)^2\int^\tau_{\tau_0} dt \int^{t}_{\tau_0}dt' V_I(t)V_I(t')U(t',\tau_0)
\end{align*}
これを繰り返せば,次のような$V_I$に関するベキ展開
\begin{align*}
U(\tau,\tau_0)=&1-i\int^\tau_{\tau_0} dt_1 V_I(t_1)+(-i)^2\int^\tau_{\tau_0} dt_1 \int^{t_1}_{\tau_0} dt_2 V_I(t_1)V_I(t_2) \\
&+(-i)^3\int^\tau_{\tau_0} dt_1 \int^{t_1}_{\tau_0} dt_2 \int^{t_2}_{\tau_0} dt_3 V_I(t_1)V_I(t_2)V_I(t_3)+\cdots \\
&+(-i)^n\int^\tau_{\tau_0} dt_1 \int^{t_1}_{\tau_0} dt_2 \cdots \int^{t_{n-1}}_{\tau_0} dt_n V_I(t_1)V_I(t_2)\cdots V_I(t_n)+\cdots
\end{align*}
を得る.$\tau=\infty$および$\tau_0=-\infty$をとれば,$S$演算子の摂動展開
\begin{align*}
S=&1-i\int^\infty_{-\infty} dt_1 V_I(t_1)+(-i)^2\int^\infty_{-\infty} dt_1 \int^{t_1}_{-\infty} dt_2 V_I(t_1)V_I(t_2) \\
&+(-i)^3\int^\infty_{-\infty} dt_1 \int^{t_1}_{-\infty} dt_2 \int^{t_2}_{-\infty} dt_3 V_I(t_1)V_I(t_2)V_I(t_3)+\cdots \\
&+(-i)^n\int^\infty_{-\infty} dt_1 \int^{t_1}_{-\infty} dt_2 \cdots \int^{t_{n-1}}_{-\infty} dt_n V_I(t_1)V_I(t_2)\cdots V_I(t_n)+\cdots
\end{align*}
を得る.これは実は旧式の摂動論の方からでも,エネルギー因子のフーリエ表示\footnote{右辺は収束しないので,被積分関数の中に収束因子$e^{-\epsilon\tau}$を入れて積分し,後から$\epsilon\to 0$をとって評価する.実際に右辺を積分してみれば
\begin{align*}
-i\int^{\infty}_0 d\tau \exp(i(E_\alpha-E_\gamma+i\epsilon)\tau) =\left[\frac{1}{E_\alpha-E_\gamma+i\epsilon} e^{i(E_\alpha-E_\gamma)}e^{-\epsilon\tau}\right]^\infty_0 =\frac{1}{E_\alpha-E_\gamma+i\epsilon}
\end{align*}
となる.}
\begin{align*}
\frac{1}{E_\alpha-E_\gamma+i\epsilon}=-i\int^{\infty}_0 d\tau \exp(i(E_\alpha-E_\gamma)\tau)
\end{align*}
を使えば導くことができる.実際
\begin{align*}
S_{\beta\alpha}=&\delta(\beta-\alpha)-2\pi i \delta(E_\beta-E_\alpha)T_{\beta\alpha}^+ \\
=&\delta(\beta-\alpha)-2\pi i \delta(E_\beta-E_\alpha) V_{\beta\alpha} -2\pi i \delta(E_\beta-E_\alpha)\int d\gamma \frac{V_{\beta\gamma}V_{\gamma\alpha}}{E_\alpha-E_\gamma+i\epsilon} \\
&-2\pi i \delta(E_\beta-E_\alpha)\int d\gamma d\gamma' \frac{V_{\beta\gamma}V_{\gamma\gamma'}V_{\gamma'\alpha}}{(E_\alpha-E_\gamma+i\epsilon)(E_\alpha-E_{\gamma'}+i\epsilon)}+\cdots
\end{align*}
となり,各項を調べてみると,まず第一項目と第二項目は
\begin{align*}
\delta(\beta-\alpha)=&(\Phi_\beta,1 \Phi_\alpha) \\
-2\pi i \delta(E_\beta-E_\alpha) V_{\beta\alpha}=&-i\int^\infty_{-\infty} dt_1 e^{i(E_\beta-E_\alpha)t_1}(\Phi_\beta,V\Phi_\alpha) \\
=&-i\int^\infty_{-\infty} dt_1 (e^{-iE_\beta t_1}\Phi_\beta,Ve^{-iE_\alpha t_1}\Phi_\alpha) \quad ((a \Phi,\Psi)=a^* (\Phi,\Psi) に注意)\\
=&-i\int^\infty_{-\infty} dt_1 (e^{-iH_0 t_1}\Phi_\beta,Ve^{-iH_0t_1}\Phi_\alpha)=-i\int^\infty_{-\infty} dt_1 (\Phi_\beta,e^{iH_0t_1}Ve^{-iH_0t_1}\Phi_\alpha) \\
=&\left(\Phi_\beta ,\left[-i \int^\infty_{-\infty} dt_1 V_I(t_1)\right]\Phi_\alpha\right)
\end{align*}
となり,第三項目は
\begin{align*}
&-2\pi i \delta(E_\beta-E_\alpha) \int d\gamma \frac{V_{\beta\gamma}V_{\gamma\alpha}}{E_\alpha-E_\gamma+i\epsilon} \\
=&(-i)^2\int^\infty_{-\infty} dt_1 e^{i(E_\beta-E_\alpha)t_1} \int d\gamma (\Phi_{\beta},V\Phi_{\gamma})(\Phi_\gamma,V\Phi_\alpha)\int^\infty_0 dt_2 e^{-i(E_\alpha-E_\gamma)t_2} \\
=&(-i)^2\int^\infty_{-\infty} dt_1 \int^\infty_0 dt_2 \int d\gamma (e^{-iE_\beta t_1} \Phi_{\beta},V e^{-iE_\gamma t_2}\Phi_{\gamma})( \Phi_\gamma ,V e^{-iE_\alpha (t_1-t_2)}\Phi_\alpha) \\
=&(-i)^2\int^\infty_{-\infty} dt_1 \int^\infty_0 dt_2 \int d\gamma (\Phi_{\beta},e^{iH_0t_1}V e^{-iH_0 t_2}\Phi_{\gamma})( \Phi_\gamma ,V e^{-iH_0 (t_1-t_2)}\Phi_\alpha) \\
=&(-i)^2\int^\infty_{-\infty} dt_1 \int^\infty_0 dt_2 (\Phi_{\beta},e^{iH_0t_1}V e^{-iH_0 t_2} V e^{-iH_0 (t_1-t_2)}\Phi_\alpha) \\
=&(-i)^2\int^\infty_{-\infty} dt_1 \int^{t_1}_{-\infty} dt_2 (\Phi_{\beta},e^{iH_0t_1}V e^{-iH_0t_1} e^{iH_0t_2}V e^{-iH_0 t_2}\Phi_\alpha) \quad (t_2 \to t'_2=t_1-t_2) \\
=&\left(\Phi_\beta \left[(-i)^2\int^\infty_{-\infty} dt_1 \int^{t_1}_{-\infty} dt_2 V_I(t_1) V_I(t_2)\right] \Phi_\alpha\right)
\end{align*}
これをさらに高次の項についても同様に繰り返してやれば,時間に依存する摂動が導ける.\par
$S$演算子の,各項の時間積分の積分範囲に注目する.
\begin{align*}
(-i)^n\int^\infty_{-\infty} dt_1 \int^{t_1}_{-\infty} dt_2 \cdots \int^{t_{n-1}}_{-\infty} dt_n V_I(t_1)V_I(t_2)\cdots V_I(t_n)
\end{align*}
積分範囲は$-\infty< t_n < t_{n-1}<\cdots < t_2 <t_1 < +\infty$となっていることに気付く.さらに,時間が速い$V(t)$が順に右から左に並べられていることにも気付く.したがって,時間に依存する任意の演算子の時間順序積を,時間が最も遅い演算子が最も左に来て,次に遅い演算子がその右に来て,…というように各演算子を並び替えた積で定義する.例えば
\begin{align*}
T\{V_I(t)\}:=&V_I(t) \\
T\{V_I(t_1)V_I(t_2)\}:=&V_I(t_1) \theta(t_1-t_2)V_I(t_2)+V_I(t_2) \theta(t_2-t_1)V_I(t_1)
\end{align*}
といった具合である\footnote{一般に$n$個の演算子$\mc{O}_1,\cdots \mc{O}_n$の時間順序積は
\begin{align*}
&T\left\{\prod_{j=1}^n \mcO_j(x_j) \right\}=T\left\{ \mcO_{1}(x_1)\cdots \mcO_{n}(x_n) \right\} \\
&:= \mcO_{1}(x_1)\theta(x^0_1-x^0_2)\mcO_{2}(x_2)\theta(x^0_2-x^0_3)\times\cdots \\
&\quad \times\mcO_{m-1}(x_{m-1})\theta(x^0_{m-1}-x^0_m)\mcO_{m}(x_m)\theta(x^0_m-x^0_{m+1})\mcO_{m+1}(x_{m+1})\cdots \theta(x^0_{n-1}-x^0_n)\mcO_n(x_n) \\
&\quad +((\mc{O}_i,t_i)の組の置換) \\
=&\sum_{\sigma_n\in S_n}\left[\left( \prod_{j=1}^n \mcO_{\sigma_n(j)} \right)\left(\prod_{k=1}^{n-1}\theta(x^0_{\sigma_n(k)}-x^0_{\sigma_n(k+1)})\right)\right]
\end{align*}
と定義される.この定義を直接使う場面は4巻の20章まで無いのだが,個人的な経験でいえば,階段関数は演算子と演算子の間に入れて,演算子と階段関数の引数がうまく隣り合う$\mc{O}_i(x_i)\theta(x^0_i-x^0_j)\mc{O}_j(x_j)$という感覚をもっておくのは大事だと思う.}.ここで$\theta(\tau)$は,$\tau$の符号に対応して$0,1$の値をとる
\begin{align*}
\theta(\tau)=\left\{
\begin{array}{ll}
+1 \quad & (\tau>0 )\\
0        & (\tau< 0)
\end{array}
\right.
\end{align*}
という階段関数である\footnote{$\tau=0$での値は$1,0,1/2$あたりから選ばれる.ほぼ使わないので,必要になったら適当に与えることにする.}.これを用いると,積分変数$t_1,\cdots ,t_n$についての$n!$個の置換に分けて
\begin{align*}
&\int^\infty_{-\infty} dt_1 \int^{t_1}_{-\infty} dt_2 \cdots \int^{t_{n-1}}_{-\infty} dt_n V_I(t_1)V_I(t_2)\cdots V_I(t_n) \\
=&\frac{1}{n!}\int^\infty_{-\infty} dt_1 \int^{t_1}_{-\infty} dt_2 \cdots \int^{t_{n-1}}_{-\infty} dt_n V_I(t_1)V_I(t_2)\cdots V_I(t_n) \\
&+\frac{1}{n!}\int^\infty_{-\infty} dt_2 \int^{t_2}_{-\infty} dt_1 \cdots \int^{t_{n-1}}_{-\infty} dt_n V_I(t_2)V_I(t_1)\cdots V_I(t_n) \\
&+\cdots \\
&+\frac{1}{n!}\int^\infty_{-\infty} dt_n \int^{t_n}_{-\infty} dt_{n-1} \cdots \int^{t_2}_{-\infty} dt_1 V_I(t_n)V_I(t_{n-1})\cdots V_I(t_1) \\
=&\frac{1}{n!}\int^\infty_{-\infty} dt_1 \int^{t_1}_{-\infty} dt_2 \cdots \int^{t_{n-1}}_{-\infty} dt_n T\{V_I(t_1)V_I(t_2)\cdots V_I(t_n)\} \\
&+\frac{1}{n!}\int^\infty_{-\infty} dt_2 \int^{t_2}_{-\infty} dt_1 \cdots \int^{t_{n-1}}_{-\infty} dt_n T\{V_I(t_1)V_I(t_2)\cdots V_I(t_n)\} \\
&+\cdots \\
&+\frac{1}{n!}\int^\infty_{-\infty} dt_n \int^{t_n}_{-\infty} dt_{n-1} \cdots \int^{t_2}_{-\infty} dt_1 T\{V_I(t_1)V_I(t_2)\cdots V_I(t_n)\} \\
=&\frac{1}{n!}\left[\int^\infty_{-\infty} dt_1 \int^{t_1}_{-\infty} dt_2 \cdots \int^{t_{n-1}}_{-\infty} dt_n+(t_1,\cdots ,t_n の置換)\right]T\{V_I(t_1)V_I(t_2)\cdots V_I(t_n)\} \\
=&\frac{1}{n!}\int^\infty_{-\infty} dt_1 \int^{\infty}_{-\infty} dt_2 \cdots \int^{\infty}_{-\infty} dt_n T\{V_I(t_1)V_I(t_2)\cdots V_I(t_n)\}
\end{align*}
となる.よって(3.5.8)は
\begin{align*}
S=1+\sum_{n=1}^\infty \frac{(-i)^n}{n!}\int_{-\infty}^\infty dt_1 dt_2 \cdots dt_n T\Bigl\{V_I(t_1)V_I(t_2)\cdots V_I(t_n)\Bigr\}
\end{align*}
と書きなおすことができる.これはダイソンの級数と知られている.形式的にこれはさらに
\begin{align*}
S=&T\left[1+\sum_{n=1}^\infty \frac{1}{n!}\left(-i \int_{-\infty}^\infty dt V_I(t)\right)^n \right] \\
=&T\exp\left(-i \int_{-\infty}^\infty dt V_I(t)\right)
\end{align*}
と書かれる.ここでの$T$は,指数関数の級数展開の各項を,時間の順序に並べて評価するという意味である.

\vskip\baselineskip

この形式までくると$S$行列が明白にローレンツ不変な,一つの大きな部類の理論を容易に見つけることができる!$S$行列の要素は,$S$演算子の$\Phi_\alpha,\Phi_\beta$など自由粒子状態間の行列要素$S_{\beta\alpha}=(\Phi_\beta,S\Phi_\alpha)$なので,3.3節より,$S$行列のローレンツ不変性のために示したいことは,$S$演算子がこれらの1粒子状態にローレンツ変換を生成する演算子$U_0(\Lambda,a)$と交換しなければならない.すなわち(3.3.3)の通り,$S$演算子は$U_0(\Lambda,a)$の生成子$H_0,\mathbf{P}_0,\mathbf{J}_0,\mathbf{K}_0$と交換しなければならないということである.\par
この要求を満たすために,$V_I(t)$が3次元空間積分
\begin{align*}
V_I(t)=\int d^3\mathbf{x} \mc{H}_I(\mathbf{x},t)
\end{align*}
で書けて,$\mc{H}_I(x)$が
\begin{align*}
U_0(\Lambda,a)\mc{H}_I(x)U^{-1}_0(\Lambda,a)=\mc{H}(\Lambda x+a)
\end{align*}
の意味でスカラーだと仮定してみる.(この仮定は
\begin{align*}
e^{iH_0 t}V e^{-iH_0t} =& e^{iH_0 t}V_I(0)e^{-iH_0t} \\
=& U_0(1,t) \left[\int d^3\mathbf{x}\mc{H}_I(\mathbf{x},0)\right] U_0^{-1}(1,t) \\
=&\int d^3\mathbf{x}\mc{H}_I(\mathbf{x},t) \\
=&V_I(t)
\end{align*}
となるから,(3.5.5)と無矛盾である.)すると,$S$演算子は4次元積分の和
\begin{align*}
S=1+\sum_{n=1}^\infty \frac{(-i)^n}{n!}\int d^4x_1 \cdots \int d^4x_n T\Bigl\{\mc{H}_I(x_1)\cdots \mc{H}_I(x_n)\Bigr\}
\end{align*}
と書ける.こで演算子の時間順序積に含まれる時間の依存性を除いて,全てが明白にローレンツ不変である!実際,ローレンツ変換によって\uwave{各$x_i$同士の過去と未来の順序が入れ替わることがない}と仮定すれば
\begin{align*}
&U_0(\Lambda,a)S U_0^{-1}(\Lambda,a) \\
=&U_0(\Lambda,a)U^{-1}_0(\Lambda,a)+\sum_{n=1}^\infty \frac{(-i)^n}{n!}\int d^4x_1 \cdots \int d^4x_n T\Bigl\{U_0(\Lambda,a) \mc{H}_I(x_1)\cdots \mc{H}_I(x_n) U_0^{-1}(\Lambda,a)\Bigr\} \\
=&1+\sum_{n=1}^\infty \frac{(-i)^n}{n!}\int d^4x_1 \cdots \int d^4x_n T\Bigl\{\left[U(\Lambda,a)\mc{H}_I(x_1)U^{-1}_0(\Lambda,a)\right]\cdots \left[U(\Lambda,a)\mc{H}_I(x_n)U^{-1}_0(\Lambda,a)\right]\Bigr\} \\
=&1+\sum_{n=1}^\infty \frac{(-i)^n}{n!}\int d^4x_1 \cdots \int d^4x_n T\Bigl\{\mc{H}_I(\Lambda x_1+a)\cdots \mc{H}_I(\Lambda x_n+a)\Bigr\}
\end{align*}
となり,$x^\mu \to x'^\mu=\tensor{\Lambda}{^\mu_\nu} x^\nu+a^\mu$の変数変換によるヤコビアンは$|\det \Lambda|d^4x=d^4x'$であるが,ローレンツ変換は$|\det \Lambda |=1$を満たすから,$d^4x=d^4x'$である.したがって
\begin{align*}
=&1+\sum_{n=1}^\infty \frac{(-i)^n}{n!}\int d^4x'_1 \cdots \int d^4x'_n T\Bigl\{\mc{H}_I(x'_1)\cdots \mc{H}_I(x'_n)\Bigr\} \\
=&S \\
\therefore \quad & U_0(\Lambda,a)S U_0^{-1}(\Lambda,a)=S
\end{align*}
となり,ローレンツ不変な理論であることが明白になる!\par
しかし,時間順序積に現れる$\theta(\tau)$があらわに時間に依存するから,ローレンツ変換による過去・未来の順序が入れ替わる場合にはこれは成り立たない.実際,簡単に$\mc{H}_I$について二次の項を見てみると
\begin{align*}
&\int d^4x_1 d^4x_2 U_0(\Lambda,a)T\Bigl\{ \mc{H}_I(x_1)\mc{H}_I(x_2)\Bigr\} U^{-1}_0(\Lambda,a)\\
=&\int d^4x_1 d^4x_2 U_0(\Lambda,a) \Bigl[\mc{H}_I(x_1)\theta(x^0_1-x^0_2)\mc{H}_I(x_2)+\mc{H}_I(x_2)\theta(x^0_2-x^0_1)\mc{H}_I(x_1)\Bigr] U_0^{-1}(\Lambda,a) \\
=&\int d^4x_1 d^4x_2 \Bigl[\mc{H}_I(\Lambda x_1+a) \theta(x^0_1-x^0_2)\mc{H}_I(\Lambda x_2+a)+\mc{H}_I(\Lambda x_2+a)\theta(x^0_2-x^0_1)\mc{H}_I(\Lambda x_1+a)\Bigr] \\
=&\int d^4x_1 d^4x_2 \left[\mc{H}_I(x_1) \theta\Bigl([\Lambda^{-1}(x_1-x_2)]^0\Bigr)\mc{H}_I(x_2)+\mc{H}_I(x_2) \theta\Bigl([\Lambda^{-1}(x_2-x_1)]^0\Bigr)\mc{H}_I(x_1)\right]
\end{align*}
となる.2.5節の最初に運動量$p^\mu$の場合で計算したのと同じ手順により,$\Lambda$が固有順時ローレンツ変換であるならば時間的(time-like)に離れている$(x_1-x_2)^2 \leq 0$である限り$(x_1-x_2)^0$の符号がローレンツ変換の下で変化しないことがわかる.すなわち引数の符号によって定まる階段関数は不変$\theta\Bigl([\Lambda^{-1}(x_1-x_2)]^0\Bigr)=\theta(x_1-x_2)$である.しかし,空間的(space-like)に離れている$(x_1-x_2)^2>0$の場合,$(x_1-x_2)^0$の符号が反転するような(つまり過去と未来の因果が反転するような)ローレンツ変換$\Lambda$が存在する\footnote{イメージ的には,$x-t$平面を描いてもらって,space-likeな2点(例えば$x_1$を座標原点に,$x_2$を第一象限だが斜め$45^\circ$線より下側に)を書けばわかりやすい.そのままだと$x_1$が過去で$x_2$は未来に位置するが,ローレンツ変換によって同時刻線が斜めになることを使えば,$x_2$が未来で$x_1$が過去になるように慣性系を選ぶことができることがわかるはず.time-likeな2点(例えば今度は$x_2$を斜め$45^\circ$線より上側に)を考えれば,ローレンツ変換のもとで同時刻線は最大でも$45^\circ$線までしか斜めにできないので,過去と未来が入れ替わるようにはできない.}.その場合$\theta\Bigl([\Lambda^{-1}(x_1-x_2)]^0\Bigr)=\theta(x_2^0-x_1^0)$であるから
\begin{align*}
=&\int d^4x_1 d^4x_2 \left[\mc{H}_I(x_1)\theta(x^0_2-x^0_1)\mc{H}_I(x_2)+\mc{H}_I(x_2)\theta(x^0_1-x^0_2)\mc{H}_I(x_1)\right] \\
\neq &\int d^4x_1 d^4x_2 \left[\mc{H}_I(x_1)\theta(x^0_1-x^0_2)\mc{H}_I(x_2)+\mc{H}_I(x_2)\theta(x_2-x_1)\mc{H}_I(x_1)\right]
= \int d^4x_1 d^4x_2 T\Bigl\{ \mc{H}_I(x_1)\mc{H}_I(x_2)\Bigr\}
\end{align*}
となり,一般に$U_0(\Lambda,a)SU^{-1}_0(\Lambda,a)\neq S$になってしまう.この式をよく見ると,これが再び$S$演算子の2次の項に戻るためには,$x_1,x_2$がspace-likeに離れている$(x_1-x_2)^2>0$場合には演算子$\mc{H}_I(x)$同士が交換
\begin{align*}
\mc{H}_I(x_1)\mc{H}_I(x_2)=&\mc{H}_I(x_2)\mc{H}_I(x_1) \\
\therefore \quad [\mc{H}_I(x_1),\mc{H}_I(x_2)]=&0
\end{align*}
すればよいことがわかる.より高次の項で考えても,$x_1,x_2,\cdots , x_n$のいくつか(例えば$x_3$と$x_5$など)が空間的に離れている場合,時間順序積に現れる階段関数がローレンツ変換のもとで不変でない組み合わせが存在する(例えば$\mc{H}_I(x_3)\theta(x_3-x_5)\mc{H}_I(x_5)$の現れる項).その組み合わせはローレンツ変換のもとで過去と未来が入れ替わる項が現れる($\mc{H}_I(x_3)\theta(x_3-x_5)\mc{H}_I(x_5)\to \mc{H}_I(x_3)\theta(x_5-x_3)\mc{H}_I(x_5)$)が,空間的に離れた相互作用が交換($[\mc{H}_I(x_3),\mc{H}_I(x_5)]=0$)すれば高次の項も時間順序積を保ち,$S$行列はローレンツ変換のもとで不変となる.以上より,$\mc{H}_I(x)$同士がspace-likeに離れているときにすべて交換
\begin{align*}
[\mc{H}_I(x),\mc{H}_I(x')]=0 \quad (x-x')^2 \geq 0
\end{align*}
すれば\footnote{上の議論から$(x-x')^2 > 0$のときに交換すればいいのだが,6章で出てくる$x=x'$での特異性によって少し条件を強めて$(x-x')^2 \geq 0$としているらしい.},(3.5.13)の時間順序での並べ替えによって特別なローレンツ系が導入されることはない.\par
慣性系によって過去と未来が入れ替わるような2点は相対論的に因果がないとするのであった.したがってこの条件はまさに因果律を制定するものである.イメージ的には,因果のない相互作用同士は互いに干渉しないことを意味している.

\vskip\baselineskip


3.3節の結果を用いて,(3.5.12)と(3.5.14)を満たす相互作用(3.5.11)が正しいローレンツ変換性をもつ$S$行列を導くことが,形式的だが非摂動論的に示せる.\par
微小なブースト$\mathbf{x}\to \mathbf{x}'=\mathbf{x}+\bm{\beta} t,t\to t'=t+\bm{\beta} \cdot \mathbf{x}$に対して,(3.5.12)は2.4節の結果より
\begin{align*}
U_0(\Lambda,0)\mc{H}_I(\mathbf{x},t)U_0(\Lambda,0)=&(1-i\mathbf{K}_0\cdot \bm{\beta}) \mc{H}_I(\mathbf{x},t)(1+i\mathbf{K}_0\cdot \bm{\beta}) \\
=&\mc{H}_I(\mathbf{x},t)-i\bm{\beta}\cdot [\mathbf{K}_0,\mc{H}_I(\mathbf{x},t)] \\
=\mc{H}_I(\mathbf{x}+\bm{\beta}t,t+\bm{\beta}\cdot \mathbf{x})=&\mc{H}_I(\mathbf{x},t)+t\bm{\beta}\cdot \nabla \mc{H}_I(\mathbf{x},t)+\bm{\beta}\cdot \mathbf{x} \frac{\partial}{\partial t}\mc{H}_I(\mathbf{x},t) \\
\therefore \quad -i[\mathbf{K}_0 ,\mc{H}_I(\mathbf{x},t)]=&t\bm{\nabla} \mc{H}_I(\mathbf{x},t)+\mathbf{x} \frac{\partial}{\partial t}\mc{H}_I(\mathbf{x},t)
\end{align*}
を与える.$\mathbf{x}$で積分して
\begin{align*}
(\mathrm{LHS})=&-i\int d^3\mathbf{x}[\mathbf{K}_0 ,\mc{H}_I(\mathbf{x},t)]=-i\left[\mathbf{K}_0 ,\int d^3\mathbf{x} \mc{H}_I(\mathbf{x},t)\right] \\
=&-i[\mathbf{K}_0 ,V_I(t)] \\
(\mathrm{RHS})=&t\int d^3\mathbf{x} \bm{\nabla}\mc{H}_I(\mathbf{x},t)+\frac{\partial}{\partial t}\int d^3\mathbf{x}\, \mathbf{x} \mc{H}_I(\mathbf{x},t)
\end{align*}
さらに$t=0$とおくと
\begin{align*}
-i[\mathbf{K}_0 ,V]=\left[\frac{\partial}{\partial t}\int d^3\mathbf{x}\, \mathbf{x} \mc{H}_I(\mathbf{x},t)\right]_{t=0}
\end{align*}
一方,$a^\mu=(0,0,0,\epsilon)$とおいて再び(3.5.12)より
\begin{align*}
U_0(1,a)\left[\int d^3\mathbf{x}\, \mathbf{x} \mc{H}_I(\mathbf{x},t)\right]U^{-1}_0(1,a)=&\int d^3\mathbf{x}\, \mathbf{x} U_0(1,a)\mc{H}_I(\mathbf{x},t)U(1,a) \\
=&\int d^3\mathbf{x}\, \mathbf{x} \mc{H}_I(\mathbf{x},t+\epsilon) \\
=&\int d^3\mathbf{x}\, \mathbf{x} \mc{H}_I(\mathbf{x},t)+\epsilon \frac{\partial}{\partial t}\int d^3\mathbf{x}\, \mathbf{x} \mc{H}_I(\mathbf{x},t) \\
=(1+iH_0 \epsilon)\left[\int d^3\mathbf{x}\, \mathbf{x} \mc{H}_I(\mathbf{x},t)\right](1-iH_0\epsilon)=&\int d^3\mathbf{x}\, \mathbf{x} \mc{H}_I(\mathbf{x},t)+i\epsilon \left[H_0,\int d^3\mathbf{x}\, \mathbf{x} \mc{H}_I(\mathbf{x},t)\right]
\end{align*}
で$t=0$とおくと
\begin{align*}
\left[\frac{\partial}{\partial t}\int d^3\mathbf{x}\, \mathbf{x} \mc{H}_I(\mathbf{x},t)\right]_{t=0}=i\left[H_0,\int d^3\mathbf{x}\, \mathbf{x} \mc{H}_I(\mathbf{x},0)\right]
\end{align*}
を与える.以上より
\begin{align*}
[\mathbf{K}_0,V]=&-[\mathbf{W},H_0] \\
\mathbf{W}:=&-\int d^3\mathbf{x}\, \mathbf{x} \mc{H}_I(\mathbf{x},0)
\end{align*}
となる.この$\mathbf{W}$がブースト生成子$\mathbf{K}_0$の補正であり,(3.3.21)を満たすためには
\begin{align*}
[\mathbf{K}_0,V]=&-[\mathbf{W},H_0]=-[\mathbf{W},H]+[\mathbf{W},V] \\
\therefore \quad [\mathbf{W},V]=&0
\end{align*}
でなければならない.これは書き換えれば
\begin{align*}
0=[\mathbf{W},V]=\int d^3\mathbf{x} \int d^3\mathbf{y}\,  \mathbf{x} [\mc{H}_I(\mathbf{x},0),\mc{H}_I(\mathbf{y},0)]
\end{align*}
であるから,任意の$\mathbf{x},\mathbf{y}$について
\begin{align*}
[\mc{H}_I(\mathbf{x},0),\mc{H}_I(\mathbf{y},0)]=0
\end{align*}
であることが十分条件である.これはまさに因果律(3.5.14)により導かれる!(時間$t=0$が一致している同時交換関係なので,任意の$\mathbf{x},\mathbf{y}$が$(\mathbf{x}-\mathbf{y})^2\geq 0$であることより因果律の関係性から導かれる.)したがって,3.3節のローレンツ不変性の条件は相互作用(3.5.11)と因果律(3.5.14)を仮定すれば満たされることがわかる.\par
ローレンツ不変性を満たすことを確認するだけならば前半の$S$行列の不変性だけで十分だが,3.3節はローレンツ不変性の一般的な条件を与えてり,今回の仮定が実際にその一般的な条件を満たすことを確認した.今回与えた仮定はローレンツ不変性の必要十分条件ではなく,3.3節で与えた条件に比べて幾分強い十分条件だが,4章で与えるクラスター分解原理も満たさなければならないことを考えると(3.5.11)(3.5.14)でほとんど一般的な条件となっている.

\vskip\baselineskip


この節でこれまでに述べた方法は,相互作用演算子$V$が十分小さいときに(つまり結合定数が十分小さいときに)有効である.この近似の修正版で歪曲波ボルン近似として知られているものもある.それは,相互作用が
\begin{align*}
V=V_{\mathrm{s}}+V_{\mathrm{w}}
\end{align*}
のように,弱い項$V_{\mathrm{w}}$と強い項$V_{\mathrm{s}}$の二つの項を含む場合に有用である.相互作用$V$の中に通常の摂動論が使えない程度に強い相互作用$V_{\mathrm{s}}$を含んでいる場合でも,もし$V_{\mathrm{w}}$のベキだけで散乱振幅が書けたらそれは摂動論が適用できるはず.$\Psi_{\mathrm{s}\alpha}^\pm$を,$V_{\mathrm{s}}$を全相互作用をみなしたときのin,out状態(3.1.16)
\begin{align*}
\Psi_{\mathrm{s}\alpha}^\pm=\Phi_\alpha+(E_\alpha-H_0\pm i\epsilon)^{-1}V_{\mathrm{s}}\Psi_{\mathrm{s}\alpha}^\pm
\end{align*}









\end{document}